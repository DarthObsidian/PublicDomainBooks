\chapter{Conduct Towards The Christians, From Nero To Constantine.}
\section{Part \thesection.}

\textit{The Conduct Of The Roman Government Towards The Christians, From
The Reign Of Nero To That Of Constantine.} \textsuperscript{1111}
\vspace{\onelineskip}

\pagenote[1111]{The sixteenth chapter I cannot help considering
as a very ingenious and specious, but very disgraceful
extenuation of the cruelties perpetrated by the Roman magistrates
against the Christians. It is written in the most contemptibly
factious spirit of prejudice against the sufferers; it is
unworthy of a philosopher and of humanity. Let the narrative of
Cyprian’s death be examined. He had to relate the murder of an
innocent man of advanced age, and in a station deemed venerable
by a considerable body of the provincials of Africa, put to death
because he refused to sacrifice to Jupiter. Instead of pointing
the indignation of posterity against such an atrocious act of
tyranny, he dwells, with visible art, on the small circumstances
of decorum and politeness which attended this murder, and which
he relates with as much parade as if they were the most important
particulars of the event. Dr. Robertson has been the subject of
much blame for his real or supposed lenity towards the Spanish
murderers and tyrants in America. That the sixteenth chapter of
Mr. G. did not excite the same or greater disapprobation, is a
proof of the unphilosophical and indeed fanatical animosity
against Christianity, which was so prevalent during the latter
part of the eighteenth century.—\textit{Mackintosh:} see Life, i. p.
244, 245.}

If we seriously consider the purity of the Christian religion,
the sanctity of its moral precepts, and the innocent as well as
austere lives of the greater number of those who during the first
ages embraced the faith of the gospel, we should naturally
suppose, that so benevolent a doctrine would have been received
with due reverence, even by the unbelieving world; that the
learned and the polite, however they may deride the miracles,
would have esteemed the virtues, of the new sect; and that the
magistrates, instead of persecuting, would have protected an
order of men who yielded the most passive obedience to the laws,
though they declined the active cares of war and government. If,
on the other hand, we recollect the universal toleration of
Polytheism, as it was invariably maintained by the faith of the
people, the incredulity of philosophers, and the policy of the
Roman senate and emperors, we are at a loss to discover what new
offence the Christians had committed, what new provocation could
exasperate the mild indifference of antiquity, and what new
motives could urge the Roman princes, who beheld without concern
a thousand forms of religion subsisting in peace under their
gentle sway, to inflict a severe punishment on any part of their
subjects, who had chosen for themselves a singular but an
inoffensive mode of faith and worship.

The religious policy of the ancient world seems to have assumed a
more stern and intolerant character, to oppose the progress of
Christianity. About fourscore years after the death of Christ,
his innocent disciples were punished with death by the sentence
of a proconsul of the most amiable and philosophic character, and
according to the laws of an emperor distinguished by the wisdom
and justice of his general administration. The apologies which
were repeatedly addressed to the successors of Trajan are filled
with the most pathetic complaints, that the Christians, who
obeyed the dictates, and solicited the liberty, of conscience,
were alone, among all the subjects of the Roman empire, excluded
from the common benefits of their auspicious government. The
deaths of a few eminent martyrs have been recorded with care; and
from the time that Christianity was invested with the supreme
power, the governors of the church have been no less diligently
employed in displaying the cruelty, than in imitating the
conduct, of their Pagan adversaries. To separate (if it be
possible) a few authentic as well as interesting facts from an
undigested mass of fiction and error, and to relate, in a clear
and rational manner, the causes, the extent, the duration, and
the most important circumstances of the persecutions to which the
first Christians were exposed, is the design of the present
chapter.\textsuperscript{1222}

\pagenote[1222]{The history of the first age of Christianity is
only found in the Acts of the Apostles, and in order to speak of
the first persecutions experienced by the Christians, that book
should naturally have been consulted; those persecutions, then
limited to individuals and to a narrow sphere, interested only
the persecuted, and have been related by them alone. Gibbon
making the persecutions ascend no higher than Nero, has entirely
omitted those which preceded this epoch, and of which St. Luke
has preserved the memory. The only way to justify this omission
was, to attack the authenticity of the Acts of the Apostles; for,
if authentic, they must necessarily be consulted and quoted. Now,
antiquity has left very few works of which the authenticity is so
well established as that of the Acts of the Apostles. (See
Lardner’s Cred. of Gospel Hist. part iii.) It is therefore,
without sufficient reason, that Gibbon has maintained silence
concerning the narrative of St. Luke, and this omission is not
without importance.—G.}

The sectaries of a persecuted religion, depressed by fear
animated with resentment, and perhaps heated by enthusiasm, are
seldom in a proper temper of mind calmly to investigate, or
candidly to appreciate, the motives of their enemies, which often
escape the impartial and discerning view even of those who are
placed at a secure distance from the flames of persecution. A
reason has been assigned for the conduct of the emperors towards
the primitive Christians, which may appear the more specious and
probable as it is drawn from the acknowledged genius of
Polytheism. It has already been observed, that the religious
concord of the world was principally supported by the implicit
assent and reverence which the nations of antiquity expressed for
their respective traditions and ceremonies. It might therefore be
expected, that they would unite with indignation against any sect
or people which should separate itself from the communion of
mankind, and claiming the exclusive possession of divine
knowledge, should disdain every form of worship, except its own,
as impious and idolatrous. The rights of toleration were held by
mutual indulgence: they were justly forfeited by a refusal of the
accustomed tribute. As the payment of this tribute was inflexibly
refused by the Jews, and by them alone, the consideration of the
treatment which they experienced from the Roman magistrates, will
serve to explain how far these speculations are justified by
facts, and will lead us to discover the true causes of the
persecution of Christianity.

Without repeating what has already been mentioned of the
reverence of the Roman princes and governors for the temple of
Jerusalem, we shall only observe, that the destruction of the
temple and city was accompanied and followed by every
circumstance that could exasperate the minds of the conquerors,
and authorize religious persecution by the most specious
arguments of political justice and the public safety. From the
reign of Nero to that of Antoninus Pius, the Jews discovered a
fierce impatience of the dominion of Rome, which repeatedly broke
out in the most furious massacres and insurrections. Humanity is
shocked at the recital of the horrid cruelties which they
committed in the cities of Egypt, of Cyprus, and of Cyrene, where
they dwelt in treacherous friendship with the unsuspecting
natives;\textsuperscript{1} and we are tempted to applaud the severe retaliation
which was exercised by the arms of the legions against a race of
fanatics, whose dire and credulous superstition seemed to render
them the implacable enemies not only of the Roman government, but
of human kind.\textsuperscript{2} The enthusiasm of the Jews was supported by the
opinion, that it was unlawful for them to pay taxes to an
idolatrous master; and by the flattering promise which they
derived from their ancient oracles, that a conquering Messiah
would soon arise, destined to break their fetters, and to invest
the favorites of heaven with the empire of the earth. It was by
announcing himself as their long-expected deliverer, and by
calling on all the descendants of Abraham to assert the hope of
Israel, that the famous Barchochebas collected a formidable army,
with which he resisted during two years the power of the emperor
Hadrian.\textsuperscript{3}

\pagenote[1]{In Cyrene, they massacred 220,000 Greeks; in Cyprus,
240,000; in Egypt, a very great multitude. Many of these unhappy
victims were sawn asunder, according to a precedent to which
David had given the sanction of his example. The victorious Jews
devoured the flesh, licked up the blood, and twisted the entrails
like a girdle round their bodies. See Dion Cassius, l. lxviii. p.
1145. * Note: Some commentators, among them Reimar, in his notes
on Dion Cassius think that the hatred of the Romans against the
Jews has led the historian to exaggerate the cruelties committed
by the latter. Don. Cass. lxviii. p. 1146.—G.}

\pagenote[2]{Without repeating the well-known narratives of
Josephus, we may learn from Dion, (l. lxix. p. 1162,) that in
Hadrian’s war 580,000 Jews were cut off by the sword, besides an
infinite number which perished by famine, by disease, and by
fire.}

\pagenote[3]{For the sect of the Zealots, see Basnage, Histoire
des Juifs, l. i. c. 17; for the characters of the Messiah,
according to the Rabbis, l. v. c. 11, 12, 13; for the actions of
Barchochebas, l. vii. c. 12. (Hist. of Jews iii. 115, \&c.)—M.}

Notwithstanding these repeated provocations, the resentment of
the Roman princes expired after the victory; nor were their
apprehensions continued beyond the period of war and danger. By
the general indulgence of polytheism, and by the mild temper of
Antoninus Pius, the Jews were restored to their ancient
privileges, and once more obtained the permission of circumcising
their children, with the easy restraint, that they should never
confer on any foreign proselyte that distinguishing mark of the
Hebrew race.\textsuperscript{4} The numerous remains of that people, though they
were still excluded from the precincts of Jerusalem, were
permitted to form and to maintain considerable establishments
both in Italy and in the provinces, to acquire the freedom of
Rome, to enjoy municipal honors, and to obtain at the same time
an exemption from the burdensome and expensive offices of
society. The moderation or the contempt of the Romans gave a
legal sanction to the form of ecclesiastical police which was
instituted by the vanquished sect. The patriarch, who had fixed
his residence at Tiberias, was empowered to appoint his
subordinate ministers and apostles, to exercise a domestic
jurisdiction, and to receive from his dispersed brethren an
annual contribution.\textsuperscript{5} New synagogues were frequently erected in
the principal cities of the empire; and the sabbaths, the fasts,
and the festivals, which were either commanded by the Mosaic law,
or enjoined by the traditions of the Rabbis, were celebrated in
the most solemn and public manner.\textsuperscript{6} Such gentle treatment
insensibly assuaged the stern temper of the Jews. Awakened from
their dream of prophecy and conquest, they assumed the behavior
of peaceable and industrious subjects. Their irreconcilable
hatred of mankind, instead of flaming out in acts of blood and
violence, evaporated in less dangerous gratifications. They
embraced every opportunity of overreaching the idolaters in
trade; and they pronounced secret and ambiguous imprecations
against the haughty kingdom of Edom.\textsuperscript{7}

\pagenote[4]{It is to Modestinus, a Roman lawyer (l. vi.
regular.) that we are indebted for a distinct knowledge of the
Edict of Antoninus. See Casaubon ad Hist. August. p. 27.}

\pagenote[5]{See Basnage, Histoire des Juifs, l. iii. c. 2, 3.
The office of Patriarch was suppressed by Theodosius the
younger.}

\pagenote[6]{We need only mention the Purim, or deliverance of
the Jews from he rage of Haman, which, till the reign of
Theodosius, was celebrated with insolent triumph and riotous
intemperance. Basnage, Hist. des Juifs, l. vi. c. 17, l. viii. c.
6.}

\pagenote[7]{According to the false Josephus, Tsepho, the
grandson of Esau, conducted into Italy the army of Eneas, king of
Carthage. Another colony of Idumæans, flying from the sword of
David, took refuge in the dominions of Romulus. For these, or for
other reasons of equal weight, the name of Edom was applied by
the Jews to the Roman empire. * Note: The false Josephus is a
romancer of very modern date, though some of these legends are
probably more ancient. It may be worth considering whether many
of the stories in the Talmud are not history in a figurative
disguise, adopted from prudence. The Jews might dare to say many
things of Rome, under the significant appellation of Edom, which
they feared to utter publicly. Later and more ignorant ages took
literally, and perhaps embellished, what was intelligible among
the generation to which it was addressed. Hist. of Jews, iii.
131. ——The false Josephus has the inauguration of the emperor,
with the seven electors and apparently the pope assisting at the
coronation! Pref. page xxvi.—M.}

Since the Jews, who rejected with abhorrence the deities adored
by their sovereign and by their fellow-subjects, enjoyed,
however, the free exercise of their unsocial religion, there must
have existed some other cause, which exposed the disciples of
Christ to those severities from which the posterity of Abraham
was exempt. The difference between them is simple and obvious;
but, according to the sentiments of antiquity, it was of the
highest importance. The Jews were a \textit{nation;} the Christians were
a \textit{sect:} and if it was natural for every community to respect
the sacred institutions of their neighbors, it was incumbent on
them to persevere in those of their ancestors. The voice of
oracles, the precepts of philosophers, and the authority of the
laws, unanimously enforced this national obligation. By their
lofty claim of superior sanctity the Jews might provoke the
Polytheists to consider them as an odious and impure race. By
disdaining the intercourse of other nations, they might deserve
their contempt. The laws of Moses might be for the most part
frivolous or absurd; yet, since they had been received during
many ages by a large society, his followers were justified by the
example of mankind; and it was universally acknowledged, that
they had a right to practise what it would have been criminal in
them to neglect. But this principle, which protected the Jewish
synagogue, afforded not any favor or security to the primitive
church. By embracing the faith of the gospel, the Christians
incurred the supposed guilt of an unnatural and unpardonable
offence. They dissolved the sacred ties of custom and education,
violated the religious institutions of their country, and
presumptuously despised whatever their fathers had believed as
true, or had reverenced as sacred. Nor was this apostasy (if we
may use the expression) merely of a partial or local kind; since
the pious deserter who withdrew himself from the temples of Egypt
or Syria, would equally disdain to seek an asylum in those of
Athens or Carthage. Every Christian rejected with contempt the
superstitions of his family, his city, and his province. The
whole body of Christians unanimously refused to hold any
communion with the gods of Rome, of the empire, and of mankind.
It was in vain that the oppressed believer asserted the
inalienable rights of conscience and private judgment. Though his
situation might excite the pity, his arguments could never reach
the understanding, either of the philosophic or of the believing
part of the Pagan world. To their apprehensions, it was no less a
matter of surprise, that any individuals should entertain
scruples against complying with the established mode of worship,
than if they had conceived a sudden abhorrence to the manners,
the dress,\textsuperscript{8111} or the language of their native country.\textsuperscript{8}

\pagenote[8]{From the arguments of Celsus, as they are
represented and refuted by Origen, (l. v. p. 247—259,) we may
clearly discover the distinction that was made between the Jewish
\textit{people} and the Christian \textit{sect}. See, in the Dialogue of
Minucius Felix, (c. 5, 6,) a fair and not inelegant description
of the popular sentiments, with regard to the desertion of the
established worship.}

\pagenote[8111]{In all this there is doubtless much truth; yet
does not the more important difference lie on the surface? The
Christians made many converts the Jews but few. Had the Jewish
been equally a proselyting religion would it not have encountered
as violent persecution?—M.}

The surprise of the Pagans was soon succeeded by resentment; and
the most pious of men were exposed to the unjust but dangerous
imputation of impiety. Malice and prejudice concurred in
representing the Christians as a society of atheists, who, by the
most daring attack on the religious constitution of the empire,
had merited the severest animadversion of the civil magistrate.
They had separated themselves (they gloried in the confession)
from every mode of superstition which was received in any part of
the globe by the various temper of polytheism: but it was not
altogether so evident what deity, or what form of worship, they
had substituted to the gods and temples of antiquity. The pure
and sublime idea which they entertained of the Supreme Being
escaped the gross conception of the Pagan multitude, who were at
a loss to discover a spiritual and solitary God, that was neither
represented under any corporeal figure or visible symbol, nor was
adored with the accustomed pomp of libations and festivals, of
altars and sacrifices.\textsuperscript{9} The sages of Greece and Rome, who had
elevated their minds to the contemplation of the existence and
attributes of the First Cause, were induced by reason or by
vanity to reserve for themselves and their chosen disciples the
privilege of this philosophical devotion.\textsuperscript{10} They were far from
admitting the prejudices of mankind as the standard of truth, but
they considered them as flowing from the original disposition of
human nature; and they supposed that any popular mode of faith
and worship which presumed to disclaim the assistance of the
senses, would, in proportion as it receded from superstition,
find itself incapable of restraining the wanderings of the fancy,
and the visions of fanaticism. The careless glance which men of
wit and learning condescended to cast on the Christian
revelation, served only to confirm their hasty opinion, and to
persuade them that the principle, which they might have revered,
of the Divine Unity, was defaced by the wild enthusiasm, and
annihilated by the airy speculations, of the new sectaries. The
author of a celebrated dialogue, which has been attributed to
Lucian, whilst he affects to treat the mysterious subject of the
Trinity in a style of ridicule and contempt, betrays his own
ignorance of the weakness of human reason, and of the inscrutable
nature of the divine perfections.\textsuperscript{11}

\pagenote[9]{Cur nullas aras habent? templa nulla? nulla nota
simulacra!—Unde autem, vel quis ille, aut ubi, Deus unicus,
solitarius, desti tutus? Minucius Felix, c. 10. The Pagan
interlocutor goes on to make a distinction in favor of the Jews,
who had once a temple, altars, victims, \&c.}

\pagenote[10]{It is difficult (says Plato) to attain, and
dangerous to publish, the knowledge of the true God. See the
Theologie des Philosophes, in the Abbé d’Olivet’s French
translation of Tully de Naturâ Deorum, tom. i. p. 275.}

\pagenote[11]{The author of the Philopatris perpetually treats
the Christians as a company of dreaming enthusiasts, \&c.; and in
one place he manifestly alludes to the vision in which St. Paul
was transported to the third heaven. In another place, Triephon,
who personates a Christian, after deriding the gods of Paganism,
proposes a mysterious oath.}

It might appear less surprising, that the founder of Christianity
should not only be revered by his disciples as a sage and a
prophet, but that he should be adored as a God. The Polytheists
were disposed to adopt every article of faith, which seemed to
offer any resemblance, however distant or imperfect, with the
popular mythology; and the legends of Bacchus, of Hercules, and
of Æsculapius, had, in some measure, prepared their imagination
for the appearance of the Son of God under a human form.\textsuperscript{12} But
they were astonished that the Christians should abandon the
temples of those ancient heroes, who, in the infancy of the
world, had invented arts, instituted laws, and vanquished the
tyrants or monsters who infested the earth, in order to choose
for the exclusive object of their religious worship an obscure
teacher, who, in a recent age, and among a barbarous people, had
fallen a sacrifice either to the malice of his own countrymen, or
to the jealousy of the Roman government. The Pagan multitude,
reserving their gratitude for temporal benefits alone, rejected
the inestimable present of life and immortality, which was
offered to mankind by Jesus of Nazareth. His mild constancy in
the midst of cruel and voluntary sufferings, his universal
benevolence, and the sublime simplicity of his actions and
character, were insufficient, in the opinion of those carnal men,
to compensate for the want of fame, of empire, and of success;
and whilst they refused to acknowledge his stupendous triumph
over the powers of darkness and of the grave, they
misrepresented, or they insulted, the equivocal birth, wandering
life, and ignominious death, of the divine Author of
Christianity.\textsuperscript{13}

\pagenote[12]{According to Justin Martyr, (Apolog. Major, c.
70-85,) the dæmon who had gained some imperfect knowledge of the
prophecies, purposely contrived this resemblance, which might
deter, though by different means, both the people and the
philosophers from embracing the faith of Christ.}

\pagenote[13]{In the first and second books of Origen, Celsus
treats the birth and character of our Savior with the most
impious contempt. The orator Libanius praises Porphyry and Julian
for confuting the folly of a sect., which styles a dead man of
Palestine, God, and the Son of God. Socrates, Hist. Ecclesiast.
iii. 23.}

The personal guilt which every Christian had contracted, in thus
preferring his private sentiment to the national religion, was
aggravated in a very high degree by the number and union of the
criminals. It is well known, and has been already observed, that
Roman policy viewed with the utmost jealousy and distrust any
association among its subjects; and that the privileges of
private corporations, though formed for the most harmless or
beneficial purposes, were bestowed with a very sparing hand.\textsuperscript{14}
The religious assemblies of the Christians who had separated
themselves from the public worship, appeared of a much less
innocent nature; they were illegal in their principle, and in
their consequences might become dangerous; nor were the emperors
conscious that they violated the laws of justice, when, for the
peace of society, they prohibited those secret and sometimes
nocturnal meetings.\textsuperscript{15} The pious disobedience of the Christians
made their conduct, or perhaps their designs, appear in a much
more serious and criminal light; and the Roman princes, who might
perhaps have suffered themselves to be disarmed by a ready
submission, deeming their honor concerned in the execution of
their commands, sometimes attempted, by rigorous punishments, to
subdue this independent spirit, which boldly acknowledged an
authority superior to that of the magistrate. The extent and
duration of this spiritual conspiracy seemed to render it
everyday more deserving of his animadversion. We have already
seen that the active and successful zeal of the Christians had
insensibly diffused them through every province and almost every
city of the empire. The new converts seemed to renounce their
family and country, that they might connect themselves in an
indissoluble band of union with a peculiar society, which every
where assumed a different character from the rest of mankind.
Their gloomy and austere aspect, their abhorrence of the common
business and pleasures of life, and their frequent predictions of
impending calamities,\textsuperscript{16} inspired the Pagans with the
apprehension of some danger, which would arise from the new sect,
the more alarming as it was the more obscure. “Whatever,” says
Pliny, “may be the principle of their conduct, their inflexible
obstinacy appeared deserving of punishment.”\textsuperscript{17}

\pagenote[14]{The emperor Trajan refused to incorporate a company
of 150 firemen, for the use of the city of Nicomedia. He disliked
all associations. See Plin. Epist. x. 42, 43.}

\pagenote[15]{The proconsul Pliny had published a general edict
against unlawful meetings. The prudence of the Christians
suspended their Agapæ; but it was impossible for them to omit the
exercise of public worship.}

\pagenote[16]{As the prophecies of the Antichrist, approaching
conflagration, \&c., provoked those Pagans whom they did not
convert, they were mentioned with caution and reserve; and the
Montanists were censured for disclosing too freely the dangerous
secret. See Mosheim, 413.}

\pagenote[17]{Neque enim dubitabam, quodcunque esset quod
faterentur, (such are the words of Pliny,) pervicacian certe et
inflexibilem obstinationem lebere puniri.}

The precautions with which the disciples of Christ performed the
offices of religion were at first dictated by fear and necessity;
but they were continued from choice. By imitating the awful
secrecy which reigned in the Eleusinian mysteries, the Christians
had flattered themselves that they should render their sacred
institutions more respectable in the eyes of the Pagan world.\textsuperscript{18}
But the event, as it often happens to the operations of subtile
policy, deceived their wishes and their expectations. It was
concluded, that they only concealed what they would have blushed
to disclose. Their mistaken prudence afforded an opportunity for
malice to invent, and for suspicious credulity to believe, the
horrid tales which described the Christians as the most wicked of
human kind, who practised in their dark recesses every
abomination that a depraved fancy could suggest, and who
solicited the favor of their unknown God by the sacrifice of
every moral virtue. There were many who pretended to confess or
to relate the ceremonies of this abhorred society. It was
asserted, “that a new-born infant, entirely covered over with
flour, was presented, like some mystic symbol of initiation, to
the knife of the proselyte, who unknowingly inflicted many a
secret and mortal wound on the innocent victim of his error; that
as soon as the cruel deed was perpetrated, the sectaries drank up
the blood, greedily tore asunder the quivering members, and
pledged themselves to eternal secrecy, by a mutual consciousness
of guilt. It was as confidently affirmed, that this inhuman
sacrifice was succeeded by a suitable entertainment, in which
intemperance served as a provocative to brutal lust; till, at the
appointed moment, the lights were suddenly extinguished, shame
was banished, nature was forgotten; and, as accident might
direct, the darkness of the night was polluted by the incestuous
commerce of sisters and brothers, of sons and of mothers.”\textsuperscript{19}

\pagenote[18]{See Mosheim’s Ecclesiastical History, vol. i. p.
101, and Spanheim, Remarques sur les Cæsars de Julien, p. 468,
\&c.}

\pagenote[19]{See Justin Martyr, Apolog. i. 35, ii. 14.
Athenagoras, in Legation, c. 27. Tertullian, Apolog. c. 7, 8, 9.
Minucius Felix, c. 9, 10, 80, 31. The last of these writers
relates the accusation in the most elegant and circumstantial
manner. The answer of Tertullian is the boldest and most
vigorous.}

But the perusal of the ancient apologies was sufficient to remove
even the slightest suspicion from the mind of a candid adversary.
The Christians, with the intrepid security of innocence, appeal
from the voice of rumor to the equity of the magistrates. They
acknowledge, that if any proof can be produced of the crimes
which calumny has imputed to them, they are worthy of the most
severe punishment. They provoke the punishment, and they
challenge the proof. At the same time they urge, with equal truth
and propriety, that the charge is not less devoid of probability,
than it is destitute of evidence; they ask, whether any one can
seriously believe that the pure and holy precepts of the gospel,
which so frequently restrain the use of the most lawful
enjoyments, should inculcate the practice of the most abominable
crimes; that a large society should resolve to dishonor itself in
the eyes of its own members; and that a great number of persons
of either sex, and every age and character, insensible to the
fear of death or infamy, should consent to violate those
principles which nature and education had imprinted most deeply
in their minds.\textsuperscript{20} Nothing, it should seem, could weaken the
force or destroy the effect of so unanswerable a justification,
unless it were the injudicious conduct of the apologists
themselves, who betrayed the common cause of religion, to gratify
their devout hatred to the domestic enemies of the church. It was
sometimes faintly insinuated, and sometimes boldly asserted, that
the same bloody sacrifices, and the same incestuous festivals,
which were so falsely ascribed to the orthodox believers, were in
reality celebrated by the Marcionites, by the Carpocratians, and
by several other sects of the Gnostics, who, notwithstanding they
might deviate into the paths of heresy, were still actuated by
the sentiments of men, and still governed by the precepts of
Christianity.\textsuperscript{21} Accusations of a similar kind were retorted upon
the church by the schismatics who had departed from its
communion,\textsuperscript{22} and it was confessed on all sides, that the most
scandalous licentiousness of manners prevailed among great
numbers of those who affected the name of Christians. A Pagan
magistrate, who possessed neither leisure nor abilities to
discern the almost imperceptible line which divides the orthodox
faith from heretical pravity, might easily have imagined that
their mutual animosity had extorted the discovery of their common
guilt. It was fortunate for the repose, or at least for the
reputation, of the first Christians, that the magistrates
sometimes proceeded with more temper and moderation than is
usually consistent with religious zeal, and that they reported,
as the impartial result of their judicial inquiry, that the
sectaries, who had deserted the established worship, appeared to
them sincere in their professions, and blameless in their
manners; however they might incur, by their absurd and excessive
superstition, the censure of the laws.\textsuperscript{23}

\pagenote[20]{In the persecution of Lyons, some Gentile slaves
were compelled, by the fear of tortures, to accuse their
Christian master. The church of Lyons, writing to their brethren
of Asia, treat the horrid charge with proper indignation and
contempt. Euseb. Hist. Eccles. v. i.}

\pagenote[21]{See Justin Martyr, Apolog. i. 35. Irenæus adv.
Hæres. i. 24. Clemens. Alexandrin. Stromat. l. iii. p. 438.
Euseb. iv. 8. It would be tedious and disgusting to relate all
that the succeeding writers have imagined, all that Epiphanius
has received, and all that Tillemont has copied. M. de Beausobre
(Hist. du Manicheisme, l. ix. c. 8, 9) has exposed, with great
spirit, the disingenuous arts of Augustin and Pope Leo I.}

\pagenote[22]{When Tertullian became a Montanist, he aspersed the
morals of the church which he had so resolutely defended. “Sed
majoris est Agape, quia per hanc adolescentes tui cum sororibus
dormiunt, appendices scilicet gulæ lascivia et luxuria.” De
Jejuniis c. 17. The 85th canon of the council of Illiberis
provides against the scandals which too often polluted the vigils
of the church, and disgraced the Christian name in the eyes of
unbelievers.}

\pagenote[23]{Tertullian (Apolog. c. 2) expatiates on the fair
and honorable testimony of Pliny, with much reason and some
declamation.}

\section{Part \thesection.}

History, which undertakes to record the transactions of the past,
for the instruction of future ages, would ill deserve that
honorable office, if she condescended to plead the cause of
tyrants, or to justify the maxims of persecution. It must,
however, be acknowledged, that the conduct of the emperors who
appeared the least favorable to the primitive church, is by no
means so criminal as that of modern sovereigns, who have employed
the arm of violence and terror against the religious opinions of
any part of their subjects. From their reflections, or even from
their own feelings, a Charles V. or a Lewis XIV. might have
acquired a just knowledge of the rights of conscience, of the
obligation of faith, and of the innocence of error. But the
princes and magistrates of ancient Rome were strangers to those
principles which inspired and authorized the inflexible obstinacy
of the Christians in the cause of truth, nor could they
themselves discover in their own breasts any motive which would
have prompted them to refuse a legal, and as it were a natural,
submission to the sacred institutions of their country. The same
reason which contributes to alleviate the guilt, must have tended
to abate the vigor, of their persecutions. As they were actuated,
not by the furious zeal of bigots, but by the temperate policy of
legislators, contempt must often have relaxed, and humanity must
frequently have suspended, the execution of those laws which they
enacted against the humble and obscure followers of Christ. From
the general view of their character and motives we might
naturally conclude: I. That a considerable time elapsed before
they considered the new sectaries as an object deserving of the
attention of government. II. That in the conviction of any of
their subjects who were accused of so very singular a crime, they
proceeded with caution and reluctance. III. That they were
moderate in the use of punishments; and, IV. That the afflicted
church enjoyed many intervals of peace and tranquility.
Notwithstanding the careless indifference which the most copious
and the most minute of the Pagan writers have shown to the
affairs of the Christians,\textsuperscript{24} it may still be in our power to
confirm each of these probable suppositions, by the evidence of
authentic facts.

\pagenote[24]{In the various compilation of the Augustan History,
(a part of which was composed under the reign of Constantine,)
there are not six lines which relate to the Christians; nor has
the diligence of Xiphilin discovered their name in the large
history of Dion Cassius. * Note: The greater part of the Augustan
History is dedicated to Diocletian. This may account for the
silence of its authors concerning Christianity. The notices that
occur are almost all in the lives composed under the reign of
Constantine. It may fairly be concluded, from the language which
he had into the mouth of Mæcenas, that Dion was an enemy to all
innovations in religion. (See Gibbon, \textit{infra}, note 105.) In
fact, when the silence of Pagan historians is noticed, it should
be remembered how meagre and mutilated are all the extant
histories of the period—M.}

1. By the wise dispensation of Providence, a mysterious veil was
cast over the infancy of the church, which, till the faith of the
Christians was matured, and their numbers were multiplied, served
to protect them not only from the malice but even from the
knowledge of the Pagan world. The slow and gradual abolition of
the Mosaic ceremonies afforded a safe and innocent disguise to
the more early proselytes of the gospel. As they were, for the
greater part, of the race of Abraham, they were distinguished by
the peculiar mark of circumcision, offered up their devotions in
the Temple of Jerusalem till its final destruction, and received
both the Law and the Prophets as the genuine inspirations of the
Deity. The Gentile converts, who by a spiritual adoption had been
associated to the hope of Israel, were likewise confounded under
the garb and appearance of Jews,\textsuperscript{25} and as the Polytheists paid
less regard to articles of faith than to the external worship,
the new sect, which carefully concealed, or faintly announced,
its future greatness and ambition, was permitted to shelter
itself under the general toleration which was granted to an
ancient and celebrated people in the Roman empire. It was not
long, perhaps, before the Jews themselves, animated with a
fiercer zeal and a more jealous faith, perceived the gradual
separation of their Nazarene brethren from the doctrine of the
synagogue; and they would gladly have extinguished the dangerous
heresy in the blood of its adherents. But the decrees of Heaven
had already disarmed their malice; and though they might
sometimes exert the licentious privilege of sedition, they no
longer possessed the administration of criminal justice; nor did
they find it easy to infuse into the calm breast of a Roman
magistrate the rancor of their own zeal and prejudice. The
provincial governors declared themselves ready to listen to any
accusation that might affect the public safety; but as soon as
they were informed that it was a question not of facts but of
words, a dispute relating only to the interpretation of the
Jewish laws and prophecies, they deemed it unworthy of the
majesty of Rome seriously to discuss the obscure differences
which might arise among a barbarous and superstitious people. The
innocence of the first Christians was protected by ignorance and
contempt; and the tribunal of the Pagan magistrate often proved
their most assured refuge against the fury of the synagogue.\textsuperscript{26}
If indeed we were disposed to adopt the traditions of a too
credulous antiquity, we might relate the distant peregrinations,
the wonderful achievements, and the various deaths of the twelve
apostles: but a more accurate inquiry will induce us to doubt,
whether any of those persons who had been witnesses to the
miracles of Christ were permitted, beyond the limits of
Palestine, to seal with their blood the truth of their testimony.\textsuperscript{27}
From the ordinary term of human life, it may very naturally be
presumed that most of them were deceased before the discontent of
the Jews broke out into that furious war, which was terminated
only by the ruin of Jerusalem. During a long period, from the
death of Christ to that memorable rebellion, we cannot discover
any traces of Roman intolerance, unless they are to be found in
the sudden, the transient, but the cruel persecution, which was
exercised by Nero against the Christians of the capital,
thirty-five years after the former, and only two years before the
latter, of those great events. The character of the philosophic
historian, to whom we are principally indebted for the knowledge
of this singular transaction, would alone be sufficient to
recommend it to our most attentive consideration.

\pagenote[25]{An obscure passage of Suetonius (in Claud. c. 25)
may seem to offer a proof how strangely the Jews and Christians
of Rome were confounded with each other.}

\pagenote[26]{See, in the xviiith and xxvth chapters of the Acts
of the Apostles, the behavior of Gallio, proconsul of Achaia, and
of Festus, procurator of Judea.}

\pagenote[27]{In the time of Tertullian and Clemens of
Alexandria, the glory of martyrdom was confined to St. Peter, St.
Paul, and St. James. It was gradually bestowed on the rest of the
apostles, by the more recent Greeks, who prudently selected for
the theatre of their preaching and sufferings some remote country
beyond the limits of the Roman empire. See Mosheim, p. 81; and
Tillemont, Mémoires Ecclésiastiques, tom. i. part iii.}

In the tenth year of the reign of Nero, the capital of the empire
was afflicted by a fire which raged beyond the memory or example
of former ages.\textsuperscript{28} The monuments of Grecian art and of Roman
virtue, the trophies of the Punic and Gallic wars, the most holy
temples, and the most splendid palaces, were involved in one
common destruction. Of the fourteen regions or quarters into
which Rome was divided, four only subsisted entire, three were
levelled with the ground, and the remaining seven, which had
experienced the fury of the flames, displayed a melancholy
prospect of ruin and desolation. The vigilance of government
appears not to have neglected any of the precautions which might
alleviate the sense of so dreadful a calamity. The Imperial
gardens were thrown open to the distressed multitude, temporary
buildings were erected for their accommodation, and a plentiful
supply of corn and provisions was distributed at a very moderate
price.\textsuperscript{29} The most generous policy seemed to have dictated the
edicts which regulated the disposition of the streets and the
construction of private houses; and as it usually happens, in an
age of prosperity, the conflagration of Rome, in the course of a
few years, produced a new city, more regular and more beautiful
than the former. But all the prudence and humanity affected by
Nero on this occasion were insufficient to preserve him from the
popular suspicion. Every crime might be imputed to the assassin
of his wife and mother; nor could the prince who prostituted his
person and dignity on the theatre be deemed incapable of the most
extravagant folly. The voice of rumor accused the emperor as the
incendiary of his own capital; and as the most incredible stories
are the best adapted to the genius of an enraged people, it was
gravely reported, and firmly believed, that Nero, enjoying the
calamity which he had occasioned, amused himself with singing to
his lyre the destruction of ancient Troy.\textsuperscript{30} To divert a
suspicion, which the power of despotism was unable to suppress,
the emperor resolved to substitute in his own place some
fictitious criminals. “With this view,” continues Tacitus, “he
inflicted the most exquisite tortures on those men, who, under
the vulgar appellation of Christians, were already branded with
deserved infamy. They derived their name and origin from Christ,
who in the reign of Tiberius had suffered death by the sentence
of the procurator Pontius Pilate.\textsuperscript{31} For a while this dire
superstition was checked; but it again burst forth;\textsuperscript{3111} and not
only spread itself over Judæa, the first seat of this mischievous
sect, but was even introduced into Rome, the common asylum which
receives and protects whatever is impure, whatever is atrocious.
The confessions of those who were seized discovered a great
multitude of their accomplices, and they were all convicted, not
so much for the crime of setting fire to the city, as for their
hatred of human kind.\textsuperscript{32} They died in torments, and their
torments were imbittered by insult and derision. Some were nailed
on crosses; others sewn up in the skins of wild beasts, and
exposed to the fury of dogs; others again, smeared over with
combustible materials, were used as torches to illuminate the
darkness of the night. The gardens of Nero were destined for the
melancholy spectacle, which was accompanied with a horse-race and
honored with the presence of the emperor, who mingled with the
populace in the dress and attitude of a charioteer. The guilt of
the Christians deserved indeed the most exemplary punishment, but
the public abhorrence was changed into commiseration, from the
opinion that those unhappy wretches were sacrificed, not so much
to the public welfare, as to the cruelty of a jealous tyrant.”\textsuperscript{33}
Those who survey with a curious eye the revolutions of mankind,
may observe, that the gardens and circus of Nero on the Vatican,
which were polluted with the blood of the first Christians, have
been rendered still more famous by the triumph and by the abuse
of the persecuted religion. On the same spot, \textsuperscript{34} a temple, which
far surpasses the ancient glories of the Capitol, has been since
erected by the Christian Pontiffs, who, deriving their claim of
universal dominion from an humble fisherman of Galilee, have
succeeded to the throne of the Cæsars, given laws to the
barbarian conquerors of Rome, and extended their spiritual
jurisdiction from the coast of the Baltic to the shores of the
Pacific Ocean.

\pagenote[28]{Tacit. Annal. xv. 38—44. Sueton in Neron. c. 38.
Dion Cassius, l. lxii. p. 1014. Orosius, vii. 7.}

\pagenote[29]{The price of wheat (probably of the \textit{modius},) was
reduced as low as \textit{terni Nummi;} which would be equivalent to
about fifteen shillings the English quarter.}

\pagenote[30]{We may observe, that the rumor is mentioned by
Tacitus with a very becoming distrust and hesitation, whilst it
is greedily transcribed by Suetonius, and solemnly confirmed by
Dion.}

\pagenote[31]{This testimony is alone sufficient to expose the
anachronism of the Jews, who place the birth of Christ near a
century sooner. (Basnage, Histoire des Juifs, l. v. c. 14, 15.)
We may learn from Josephus, (Antiquitat. xviii. 3,) that the
procuratorship of Pilate corresponded with the last ten years of
Tiberius, A. D. 27—37. As to the particular time of the death of
Christ, a very early tradition fixed it to the 25th of March, A.
D. 29, under the consulship of the two Gemini. (Tertullian adv.
Judæos, c. 8.) This date, which is adopted by Pagi, Cardinal
Norris, and Le Clerc, seems at least as probable as the vulgar
æra, which is placed (I know not from what conjectures) four
years later.}

\pagenote[3111]{This single phrase, Repressa in præsens
exitiabilis superstitio rursus erumpebat, proves that the
Christians had already attracted the attention of the government;
and that Nero was not the first to persecute them. I am surprised
that more stress has not been laid on the confirmation which the
Acts of the Apostles derive from these words of Tacitus, Repressa
in præsens, and rursus erumpebat.—G. ——I have been unwilling to
suppress this note, but surely the expression of Tacitus refers
to the expected extirpation of the religion by the death of its
founder, Christ.—M.}

\pagenote[32]{\textit{Odio humani generis convicti}. These words may
either signify the hatred of mankind towards the Christians, or
the hatred of the Christians towards mankind. I have preferred
the latter sense, as the most agreeable to the style of Tacitus,
and to the popular error, of which a precept of the gospel (see
Luke xiv. 26) had been, perhaps, the innocent occasion. My
interpretation is justified by the authority of Lipsius; of the
Italian, the French, and the English translators of Tacitus; of
Mosheim, (p. 102,) of Le Clerc, (Historia Ecclesiast. p. 427,) of
Dr. Lardner, (Testimonies, vol. i. p. 345,) and of the Bishop of
Gloucester, (Divine Legation, vol. iii. p. 38.) But as the word
\textit{convicti} does not unite very happily with the rest of the
sentence, James Gronovius has preferred the reading of
\textit{conjuncti}, which is authorized by the valuable MS. of
Florence.}

\pagenote[33]{Tacit. Annal xv. 44.}

\pagenote[34]{Nardini Roma Antica, p. 487. Donatus de Roma
Antiqua, l. iii. p. 449.}

But it would be improper to dismiss this account of Nero’s
persecution, till we have made some observations that may serve
to remove the difficulties with which it is perplexed, and to
throw some light on the subsequent history of the church.

1. The most sceptical criticism is obliged to respect the truth
of this extraordinary fact, and the integrity of this celebrated
passage of Tacitus. The former is confirmed by the diligent and
accurate Suetonius, who mentions the punishment which Nero
inflicted on the Christians, a sect of men who had embraced a new
and criminal superstition.\textsuperscript{35} The latter may be proved by the
consent of the most ancient manuscripts; by the inimitable
character of the style of Tacitus by his reputation, which
guarded his text from the interpolations of pious fraud; and by
the purport of his narration, which accused the first Christians
of the most atrocious crimes, without insinuating that they
possessed any miraculous or even magical powers above the rest of
mankind.\textsuperscript{36} 2. Notwithstanding it is probable that Tacitus was
born some years before the fire of Rome,\textsuperscript{37} he could derive only
from reading and conversation the knowledge of an event which
happened during his infancy. Before he gave himself to the
public, he calmly waited till his genius had attained its full
maturity, and he was more than forty years of age, when a
grateful regard for the memory of the virtuous Agricola extorted
from him the most early of those historical compositions which
will delight and instruct the most distant posterity. After
making a trial of his strength in the life of Agricola and the
description of Germany, he conceived, and at length executed, a
more arduous work; the history of Rome, in thirty books, from the
fall of Nero to the accession of Nerva. The administration of
Nerva introduced an age of justice and propriety, which Tacitus
had destined for the occupation of his old age;\textsuperscript{38} but when he
took a nearer view of his subject, judging, perhaps, that it was
a more honorable or a less invidious office to record the vices
of past tyrants, than to celebrate the virtues of a reigning
monarch, he chose rather to relate, under the form of annals, the
actions of the four immediate successors of Augustus. To collect,
to dispose, and to adorn a series of fourscore years, in an
immortal work, every sentence of which is pregnant with the
deepest observations and the most lively images, was an
undertaking sufficient to exercise the genius of Tacitus himself
during the greatest part of his life. In the last years of the
reign of Trajan, whilst the victorious monarch extended the power
of Rome beyond its ancient limits, the historian was describing,
in the second and fourth books of his annals, the tyranny of
Tiberius;\textsuperscript{39} and the emperor Hadrian must have succeeded to the
throne, before Tacitus, in the regular prosecution of his work,
could relate the fire of the capital, and the cruelty of Nero
towards the unfortunate Christians. At the distance of sixty
years, it was the duty of the annalist to adopt the narratives of
contemporaries; but it was natural for the philosopher to indulge
himself in the description of the origin, the progress, and the
character of the new sect, not so much according to the knowledge
or prejudices of the age of Nero, as according to those of the
time of Hadrian. 3 Tacitus very frequently trusts to the
curiosity or reflection of his readers to supply those
intermediate circumstances and ideas, which, in his extreme
conciseness, he has thought proper to suppress. We may therefore
presume to imagine some probable cause which could direct the
cruelty of Nero against the Christians of Rome, whose obscurity,
as well as innocence, should have shielded them from his
indignation, and even from his notice. The Jews, who were
numerous in the capital, and oppressed in their own country, were
a much fitter object for the suspicions of the emperor and of the
people: nor did it seem unlikely that a vanquished nation, who
already discovered their abhorrence of the Roman yoke, might have
recourse to the most atrocious means of gratifying their
implacable revenge. But the Jews possessed very powerful
advocates in the palace, and even in the heart of the tyrant; his
wife and mistress, the beautiful Poppæa, and a favorite player of
the race of Abraham, who had already employed their intercession
in behalf of the obnoxious people.\textsuperscript{40} In their room it was
necessary to offer some other victims, and it might easily be
suggested that, although the genuine followers of Moses were
innocent of the fire of Rome, there had arisen among them a new
and pernicious sect of Galilæans, which was capable of the most
horrid crimes. Under the appellation of Galilæans, two
distinctions of men were confounded, the most opposite to each
other in their manners and principles; the disciples who had
embraced the faith of Jesus of Nazareth,\textsuperscript{41} and the zealots who
had followed the standard of Judas the Gaulonite.\textsuperscript{42} The former
were the friends, the latter were the enemies, of human kind; and
the only resemblance between them consisted in the same
inflexible constancy, which, in the defence of their cause,
rendered them insensible of death and tortures. The followers of
Judas, who impelled their countrymen into rebellion, were soon
buried under the ruins of Jerusalem; whilst those of Jesus, known
by the more celebrated name of Christians, diffused themselves
over the Roman empire. How natural was it for Tacitus, in the
time of Hadrian, to appropriate to the Christians the guilt and
the sufferings,\textsuperscript{4211} which he might, with far greater truth and
justice, have attributed to a sect whose odious memory was almost
extinguished! 4. Whatever opinion may be entertained of this
conjecture, (for it is no more than a conjecture,) it is evident
that the effect, as well as the cause, of Nero’s persecution, was
confined to the walls of Rome,\textsuperscript{43} that the religious tenets of
the Galilæans or Christians,\textsuperscript{431} were never made a subject of
punishment, or even of inquiry; and that, as the idea of their
sufferings was for a long time connected with the idea of cruelty
and injustice, the moderation of succeeding princes inclined them
to spare a sect, oppressed by a tyrant, whose rage had been
usually directed against virtue and innocence.

\pagenote[35]{Sueton. in Nerone, c. 16. The epithet of
\textit{malefica}, which some sagacious commentators have translated
magical, is considered by the more rational Mosheim as only
synonymous to the \textit{exitiabilis} of Tacitus.}

\pagenote[36]{The passage concerning Jesus Christ, which was
inserted into the text of Josephus, between the time of Origen
and that of Eusebius, may furnish an example of no vulgar
forgery. The accomplishment of the prophecies, the virtues,
miracles, and resurrection of Jesus, are distinctly related.
Josephus acknowledges that he was the Messiah, and hesitates
whether he should call him a man. If any doubt can still remain
concerning this celebrated passage, the reader may examine the
pointed objections of Le Fevre, (Havercamp. Joseph. tom. ii. p.
267-273), the labored answers of Daubuz, (p. 187-232, and the
masterly reply (Bibliothèque Ancienne et Moderne, tom. vii. p.
237-288) of an anonymous critic, whom I believe to have been the
learned Abbé de Longuerue. * Note: The modern editor of Eusebius,
Heinichen, has adopted, and ably supported, a notion, which had
before suggested itself to the editor, that this passage is not
altogether a forgery, but interpolated with many additional
clauses. Heinichen has endeavored to disengage the original text
from the foreign and more recent matter.—M.}

\pagenote[37]{See the lives of Tacitus by Lipsius and the Abbé de
la Bleterie, Dictionnaire de Bayle a l’article Particle Tacite,
and Fabricius, Biblioth. Latin tem. Latin. tom. ii. p. 386, edit.
Ernest. Ernst.}

\pagenote[38]{Principatum Divi Nervæ, et imperium Trajani,
uberiorem, securioremque materiam senectuti seposui. Tacit. Hist.
i.}

\pagenote[39]{See Tacit. Annal. ii. 61, iv. 4. * Note: The
perusal of this passage of Tacitus alone is sufficient, as I have
already said, to show that the Christian sect was not so obscure
as not already to have been repressed, (repressa,) and that it
did not pass for innocent in the eyes of the Romans.—G.}

\pagenote[40]{The player’s name was Aliturus. Through the same
channel, Josephus, (de vitâ suâ, c. 2,) about two years before,
had obtained the pardon and release of some Jewish priests, who
were prisoners at Rome.}

\pagenote[41]{The learned Dr. Lardner (Jewish and Heathen
Testimonies, vol ii. p. 102, 103) has proved that the name of
Galilæans was a very ancient, and perhaps the primitive
appellation of the Christians.}

\pagenote[42]{Joseph. Antiquitat. xviii. 1, 2. Tillemont, Ruine
des Juifs, p. 742 The sons of Judas were crucified in the time of
Claudius. His grandson Eleazar, after Jerusalem was taken,
defended a strong fortress with 960 of his most desperate
followers. When the battering ram had made a breach, they turned
their swords against their wives their children, and at length
against their own breasts. They dies to the last man.}

\pagenote[4211]{This conjecture is entirely devoid, not merely of
verisimilitude, but even of possibility. Tacitus could not be
deceived in appropriating to the Christians of Rome the guilt and
the sufferings which he might have attributed with far greater
truth to the followers of Judas the Gaulonite, for the latter
never went to Rome. Their revolt, their attempts, their opinions,
their wars, their punishment, had no other theatre but Judæa
(Basn. Hist. des. Juifs, t. i. p. 491.) Moreover the name of
Christians had long been given in Rome to the disciples of Jesus;
and Tacitus affirms too positively, refers too distinctly to its
etymology, to allow us to suspect any mistake on his part.—G.
——M. Guizot’s expressions are not in the least too strong against
this strange imagination of Gibbon; it may be doubted whether the
followers of Judas were known as a sect under the name of
Galilæans.—M.}

\pagenote[43]{See Dodwell. Paucitat. Mart. l. xiii. The Spanish
Inscription in Gruter. p. 238, No. 9, is a manifest and
acknowledged forgery contrived by that noted imposter. Cyriacus
of Ancona, to flatter the pride and prejudices of the Spaniards.
See Ferreras, Histoire D’Espagne, tom. i. p. 192.}

\pagenote[431]{M. Guizot, on the authority of Sulpicius Severus,
ii. 37, and of Orosius, viii. 5, inclines to the opinion of those
who extend the persecution to the provinces. Mosheim rather leans
to that side on this much disputed question, (c. xxxv.) Neander
takes the view of Gibbon, which is in general that of the most
learned writers. There is indeed no evidence, which I can
discover, of its reaching the provinces; and the apparent
security, at least as regards his life, with which St. Paul
pursued his travels during this period, affords at least a strong
inference against a rigid and general inquisition against the
Christians in other parts of the empire.—M.}

It is somewhat remarkable that the flames of war consumed, almost
at the same time, the temple of Jerusalem and the Capitol of
Rome;\textsuperscript{44} and it appears no less singular, that the tribute which
devotion had destined to the former, should have been converted
by the power of an assaulting victor to restore and adorn the
splendor of the latter.\textsuperscript{45} The emperors levied a general
capitation tax on the Jewish people; and although the sum
assessed on the head of each individual was inconsiderable, the
use for which it was designed, and the severity with which it was
exacted, were considered as an intolerable grievance.\textsuperscript{46} Since
the officers of the revenue extended their unjust claim to many
persons who were strangers to the blood or religion of the Jews,
it was impossible that the Christians, who had so often sheltered
themselves under the shade of the synagogue, should now escape
this rapacious persecution. Anxious as they were to avoid the
slightest infection of idolatry, their conscience forbade them to
contribute to the honor of that dæmon who had assumed the
character of the Capitoline Jupiter. As a very numerous though
declining party among the Christians still adhered to the law of
Moses, their efforts to dissemble their Jewish origin were
detected by the decisive test of circumcision;\textsuperscript{47} nor were the
Roman magistrates at leisure to inquire into the difference of
their religious tenets. Among the Christians who were brought
before the tribunal of the emperor, or, as it seems more
probable, before that of the procurator of Judæa, two persons are
said to have appeared, distinguished by their extraction, which
was more truly noble than that of the greatest monarchs. These
were the grandsons of St. Jude the apostle, who himself was the
brother of Jesus Christ.\textsuperscript{48} Their natural pretensions to the
throne of David might perhaps attract the respect of the people,
and excite the jealousy of the governor; but the meanness of
their garb, and the simplicity of their answers, soon convinced
him that they were neither desirous nor capable of disturbing the
peace of the Roman empire. They frankly confessed their royal
origin, and their near relation to the Messiah; but they
disclaimed any temporal views, and professed that his kingdom,
which they devoutly expected, was purely of a spiritual and
angelic nature. When they were examined concerning their fortune
and occupation, they showed their hands, hardened with daily
labor, and declared that they derived their whole subsistence
from the cultivation of a farm near the village of Cocaba, of the
extent of about twenty-four English acres,\textsuperscript{49} and of the value of
nine thousand drachms, or three hundred pounds sterling. The
grandsons of St. Jude were dismissed with compassion and
contempt.\textsuperscript{50}

\pagenote[44]{The Capitol was burnt during the civil war between
Vitellius and Vespasian, the 19th of December, A. D. 69. On the
10th of August, A. D. 70, the temple of Jerusalem was destroyed
by the hands of the Jews themselves, rather than by those of the
Romans.}

\pagenote[45]{The new Capitol was dedicated by Domitian. Sueton.
in Domitian. c. 5. Plutarch in Poplicola, tom. i. p. 230, edit.
Bryant. The gilding alone cost 12,000 talents (above two millions
and a half.) It was the opinion of Martial, (l. ix. Epigram 3,)
that if the emperor had called in his debts, Jupiter himself,
even though he had made a general auction of Olympus, would have
been unable to pay two shillings in the pound.}

\pagenote[46]{With regard to the tribute, see Dion Cassius, l.
lxvi. p. 1082, with Reimarus’s notes. Spanheim, de Usu
Numismatum, tom. ii. p. 571; and Basnage, Histoire des Juifs, l.
vii. c. 2.}

\pagenote[47]{Suetonius (in Domitian. c. 12) had seen an old man
of ninety publicly examined before the procurator’s tribunal.
This is what Martial calls, Mentula tributis damnata.}

\pagenote[48]{This appellation was at first understood in the
most obvious sense, and it was supposed, that the brothers of
Jesus were the lawful issue of Joseph and Mary. A devout respect
for the virginity of the mother of God suggested to the Gnostics,
and afterwards to the orthodox Greeks, the expedient of bestowing
a second wife on Joseph. The Latins (from the time of Jerome)
improved on that hint, asserted the perpetual celibacy of Joseph,
and justified by many similar examples the new interpretation
that Jude, as well as Simon and James, who were styled the
brothers of Jesus Christ, were only his first cousins. See
Tillemont, Mém. Ecclesiast. tom. i. part iii.: and Beausobre,
Hist. Critique du Manicheisme, l. ii. c. 2.}

\pagenote[49]{Thirty-nine, squares of a hundred feet each, which,
if strictly computed, would scarcely amount to nine acres.}

\pagenote[50]{Eusebius, iii. 20. The story is taken from
Hegesippus.}

But although the obscurity of the house of David might protect
them from the suspicions of a tyrant, the present greatness of
his own family alarmed the pusillanimous temper of Domitian,
which could only be appeased by the blood of those Romans whom he
either feared, or hated, or esteemed. Of the two sons of his
uncle Flavius Sabinus,\textsuperscript{51} the elder was soon convicted of
treasonable intentions, and the younger, who bore the name of
Flavius Clemens, was indebted for his safety to his want of
courage and ability.\textsuperscript{52} The emperor for a long time,
distinguished so harmless a kinsman by his favor and protection,
bestowed on him his own niece Domitilla, adopted the children of
that marriage to the hope of the succession, and invested their
father with the honors of the consulship.

\pagenote[51]{See the death and character of Sabinus in Tacitus,
(Hist. iii. 74 ) Sabinus was the elder brother, and, till the
accession of Vespasian, had been considered as the principal
support of the Flavium family}

\pagenote[52]{Flavium Clementem patruelem suum \textit{contemptissimæ
inertiæ}.. ex tenuissimâ suspicione interemit. Sueton. in
Domitian. c. 15.}

But he had scarcely finished the term of his annual magistracy,
when, on a slight pretence, he was condemned and executed;
Domitilla was banished to a desolate island on the coast of
Campania;\textsuperscript{53} and sentences either of death or of confiscation
were pronounced against a great number of who were involved in
the same accusation. The guilt imputed to their charge was that
of \textit{Atheism} and \textit{Jewish manners;}\textsuperscript{54} a singular association of
ideas, which cannot with any propriety be applied except to the
Christians, as they were obscurely and imperfectly viewed by the
magistrates and by the writers of that period. On the strength of
so probable an interpretation, and too eagerly admitting the
suspicions of a tyrant as an evidence of their honorable crime,
the church has placed both Clemens and Domitilla among its first
martyrs, and has branded the cruelty of Domitian with the name of
the second persecution. But this persecution (if it deserves that
epithet) was of no long duration. A few months after the death of
Clemens, and the banishment of Domitilla, Stephen, a freedman
belonging to the latter, who had enjoyed the favor, but who had
not surely embraced the faith, of his mistress,\textsuperscript{5411} assassinated
the emperor in his palace.\textsuperscript{55} The memory of Domitian was
condemned by the senate; his acts were rescinded; his exiles
recalled; and under the gentle administration of Nerva, while the
innocent were restored to their rank and fortunes, even the most
guilty either obtained pardon or escaped punishment.\textsuperscript{56}

\pagenote[53]{The Isle of Pandataria, according to Dion. Bruttius
Præsens (apud Euseb. iii. 18) banishes her to that of Pontia,
which was not far distant from the other. That difference, and a
mistake, either of Eusebius or of his transcribers, have given
occasion to suppose two Domitillas, the wife and the niece of
Clemens. See Tillemont, Mémoires Ecclésiastiques, tom. ii. p.
224.}

\pagenote[54]{Dion. l. lxvii. p. 1112. If the Bruttius Præsens,
from whom it is probable that he collected this account, was the
correspondent of Pliny, (Epistol. vii. 3,) we may consider him as
a contemporary writer.}

\pagenote[5411]{This is an uncandid sarcasm. There is nothing to
connect Stephen with the religion of Domitilla. He was a knave
detected in the malversation of money—interceptarum pecuniaram
reus.—M.}

\pagenote[55]{Suet. in Domit. c. 17. Philostratus in Vit.
Apollon. l. viii.}

\pagenote[56]{Dion. l. lxviii. p. 1118. Plin. Epistol. iv. 22.}

II. About ten years afterwards, under the reign of Trajan, the
younger Pliny was intrusted by his friend and master with the
government of Bithynia and Pontus. He soon found himself at a
loss to determine by what rule of justice or of law he should
direct his conduct in the execution of an office the most
repugnant to his humanity. Pliny had never assisted at any
judicial proceedings against the Christians, with whose name
alone he seems to be acquainted; and he was totally uninformed
with regard to the nature of their guilt, the method of their
conviction, and the degree of their punishment. In this
perplexity he had recourse to his usual expedient, of submitting
to the wisdom of Trajan an impartial, and, in some respects, a
favorable account of the new superstition, requesting the
emperor, that he would condescend to resolve his doubts, and to
instruct his ignorance.\textsuperscript{57} The life of Pliny had been employed in
the acquisition of learning, and in the business of the world.

Since the age of nineteen he had pleaded with distinction in the
tribunals of Rome,\textsuperscript{58} filled a place in the senate, had been
invested with the honors of the consulship, and had formed very
numerous connections with every order of men, both in Italy and
in the provinces. From \textit{his} ignorance therefore we may derive
some useful information. We may assure ourselves, that when he
accepted the government of Bithynia, there were no general laws
or decrees of the senate in force against the Christians; that
neither Trajan nor any of his virtuous predecessors, whose edicts
were received into the civil and criminal jurisprudence, had
publicly declared their intentions concerning the new sect; and
that whatever proceedings had been carried on against the
Christians, there were none of sufficient weight and authority to
establish a precedent for the conduct of a Roman magistrate.

\pagenote[57]{Plin. Epistol. x. 97. The learned Mosheim expresses
himself (p. 147, 232) with the highest approbation of Pliny’s
moderate and candid temper. Notwithstanding Dr. Lardner’s
suspicions (see Jewish and Heathen Testimonies, vol. ii. p. 46,)
I am unable to discover any bigotry in his language or
proceedings. * Note: Yet the humane Pliny put two female
attendants, probably deaconesses to the torture, in order to
ascertain the real nature of these suspicious meetings:
necessarium credidi, ex duabus ancillis, quæ ministræ dicebantor
quid asset veri et \textit{per tormenta} quærere.—M.}

\pagenote[58]{Plin. Epist. v. 8. He pleaded his first cause A. D.
81; the year after the famous eruptions of Mount Vesuvius, in
which his uncle lost his life.}

\section{Part \thesection.}

The answer of Trajan, to which the Christians of the succeeding
age have frequently appealed, discovers as much regard for
justice and humanity as could be reconciled with his mistaken
notions of religious policy.\textsuperscript{59} Instead of displaying the
implacable zeal of an inquisitor, anxious to discover the most
minute particles of heresy, and exulting in the number of his
victims, the emperor expresses much more solicitude to protect
the security of the innocent, than to prevent the escape of the
guilty. He acknowledged the difficulty of fixing any general
plan; but he lays down two salutary rules, which often afforded
relief and support to the distressed Christians. Though he
directs the magistrates to punish such persons as are legally
convicted, he prohibits them, with a very humane inconsistency,
from making any inquiries concerning the supposed criminals. Nor
was the magistrate allowed to proceed on every kind of
information. Anonymous charges the emperor rejects, as too
repugnant to the equity of his government; and he strictly
requires, for the conviction of those to whom the guilt of
Christianity is imputed, the positive evidence of a fair and open
accuser. It is likewise probable, that the persons who assumed so
invidiuous an office, were obliged to declare the grounds of
their suspicions, to specify (both in respect to time and place)
the secret assemblies, which their Christian adversary had
frequented, and to disclose a great number of circumstances,
which were concealed with the most vigilant jealousy from the eye
of the profane. If they succeeded in their prosecution, they were
exposed to the resentment of a considerable and active party, to
the censure of the more liberal portion of mankind, and to the
ignominy which, in every age and country, has attended the
character of an informer. If, on the contrary, they failed in
their proofs, they incurred the severe and perhaps capital
penalty, which, according to a law published by the emperor
Hadrian, was inflicted on those who falsely attributed to their
fellow-citizens the crime of Christianity. The violence of
personal or superstitious animosity might sometimes prevail over
the most natural apprehensions of disgrace and danger but it
cannot surely be imagined,\textsuperscript{60} that accusations of so unpromising
an appearance were either lightly or frequently undertaken by the
Pagan subjects of the Roman empire.\textsuperscript{6011}

\pagenote[59]{Plin. Epist. x. 98. Tertullian (Apolog. c. 5)
considers this rescript as a relaxation of the ancient penal
laws, “quas Trajanus exparte frustratus est:” and yet Tertullian,
in another part of his Apology, exposes the inconsistency of
prohibiting inquiries, and enjoining punishments.}

\pagenote[60]{Eusebius (Hist. Ecclesiast. l. iv. c. 9) has
preserved the edict of Hadrian. He has likewise (c. 13) given us
one still more favorable, under the name of Antoninus; the
authenticity of which is not so universally allowed. The second
Apology of Justin contains some curious particulars relative to
the accusations of Christians. * Note: Professor Hegelmayer has
proved the authenticity of the edict of Antoninus, in his Comm.
Hist. Theol. in Edict. Imp. Antonini. Tubing. 1777, in 4to.—G.
——Neander doubts its authenticity, (vol. i. p. 152.) In my
opinion, the internal evidence is decisive against it.—M}

\pagenote[6011]{The enactment of this law affords strong
presumption, that accusations of the “crime of Christianity,”
were by no means so uncommon, nor received with so much mistrust
and caution by the ruling authorities, as Gibbon would insinuate.
—M.}

The expedient which was employed to elude the prudence of the
laws, affords a sufficient proof how effectually they
disappointed the mischievous designs of private malice or
superstitious zeal. In a large and tumultuous assembly, the
restraints of fear and shame, so forcible on the minds of
individuals, are deprived of the greatest part of their
influence. The pious Christian, as he was desirous to obtain, or
to escape, the glory of martyrdom, expected, either with
impatience or with terror, the stated returns of the public games
and festivals. On those occasions the inhabitants of the great
cities of the empire were collected in the circus or the theatre,
where every circumstance of the place, as well as of the
ceremony, contributed to kindle their devotion, and to extinguish
their humanity. Whilst the numerous spectators, crowned with
garlands, perfumed with incense, purified with the blood of
victims, and surrounded with the altars and statues of their
tutelar deities, resigned themselves to the enjoyment of
pleasures, which they considered as an essential part of their
religious worship, they recollected, that the Christians alone
abhorred the gods of mankind, and by their absence and melancholy
on these solemn festivals, seemed to insult or to lament the
public felicity. If the empire had been afflicted by any recent
calamity, by a plague, a famine, or an unsuccessful war; if the
Tyber had, or if the Nile had not, risen beyond its banks; if the
earth had shaken, or if the temperate order of the seasons had
been interrupted, the superstitious Pagans were convinced that
the crimes and the impiety of the Christians, who were spared by
the excessive lenity of the government, had at length provoked
the divine justice. It was not among a licentious and exasperated
populace, that the forms of legal proceedings could be observed;
it was not in an amphitheatre, stained with the blood of wild
beasts and gladiators, that the voice of compassion could be
heard. The impatient clamors of the multitude denounced the
Christians as the enemies of gods and men, doomed them to the
severest tortures, and venturing to accuse by name some of the
most distinguished of the new sectaries, required with
irresistible vehemence that they should be instantly apprehended
and cast to the lions.\textsuperscript{61} The provincial governors and
magistrates who presided in the public spectacles were usually
inclined to gratify the inclinations, and to appease the rage, of
the people, by the sacrifice of a few obnoxious victims. But the
wisdom of the emperors protected the church from the danger of
these tumultuous clamors and irregular accusations, which they
justly censured as repugnant both to the firmness and to the
equity of their administration. The edicts of Hadrian and of
Antoninus Pius expressly declared, that the voice of the
multitude should never be admitted as legal evidence to convict
or to punish those unfortunate persons who had embraced the
enthusiasm of the Christians.\textsuperscript{62}

\pagenote[61]{See Tertullian, (Apolog. c. 40.) The acts of the
martyrdom of Polycarp exhibit a lively picture of these tumults,
which were usually fomented by the malice of the Jews.}

\pagenote[62]{These regulations are inserted in the above
mentioned document of Hadrian and Pius. See the apology of
Melito, (apud Euseb. l iv 26)}

III. Punishment was not the inevitable consequence of conviction,
and the Christians, whose guilt was the most clearly proved by
the testimony of witnesses, or even by their voluntary
confession, still retained in their own power the alternative of
life or death. It was not so much the past offence, as the actual
resistance, which excited the indignation of the magistrate. He
was persuaded that he offered them an easy pardon, since, if they
consented to cast a few grains of incense upon the altar, they
were dismissed from the tribunal in safety and with applause. It
was esteemed the duty of a humane judge to endeavor to reclaim,
rather than to punish, those deluded enthusiasts. Varying his
tone according to the age, the sex, or the situation of the
prisoners, he frequently condescended to set before their eyes
every circumstance which could render life more pleasing, or
death more terrible; and to solicit, nay, to entreat, them, that
they would show some compassion to themselves, to their families,
and to their friends.\textsuperscript{63} If threats and persuasions proved
ineffectual, he had often recourse to violence; the scourge and
the rack were called in to supply the deficiency of argument, and
every art of cruelty was employed to subdue such inflexible, and,
as it appeared to the Pagans, such criminal, obstinacy. The
ancient apologists of Christianity have censured, with equal
truth and severity, the irregular conduct of their persecutors
who, contrary to every principle of judicial proceeding, admitted
the use of torture, in order to obtain, not a confession, but a
denial, of the crime which was the object of their inquiry.\textsuperscript{64}
The monks of succeeding ages, who, in their peaceful solitudes,
entertained themselves with diversifying the deaths and
sufferings of the primitive martyrs, have frequently invented
torments of a much more refined and ingenious nature. In
particular, it has pleased them to suppose, that the zeal of the
Roman magistrates, disdaining every consideration of moral virtue
or public decency, endeavored to seduce those whom they were
unable to vanquish, and that by their orders the most brutal
violence was offered to those whom they found it impossible to
seduce. It is related, that females, who were prepared to despise
death, were sometimes condemned to a more severe trial,\textsuperscript{6411} and
called upon to determine whether they set a higher value on their
religion or on their chastity. The youths to whose licentious
embraces they were abandoned, received a solemn exhortation from
the judge, to exert their most strenuous efforts to maintain the
honor of Venus against the impious virgin who refused to burn
incense on her altars. Their violence, however, was commonly
disappointed, and the seasonable interposition of some miraculous
power preserved the chaste spouses of Christ from the dishonor
even of an involuntary defeat. We should not indeed neglect to
remark, that the more ancient as well as authentic memorials of
the church are seldom polluted with these extravagant and
indecent fictions.\textsuperscript{65}

\pagenote[63]{See the rescript of Trajan, and the conduct of
Pliny. The most authentic acts of the martyrs abound in these
exhortations. Note: Pliny’s test was the worship of the gods,
offerings to the statue of the emperor, and blaspheming
Christ—præterea maledicerent Christo.—M.}

\pagenote[64]{In particular, see Tertullian, (Apolog. c. 2, 3,)
and Lactantius, (Institut. Divin. v. 9.) Their reasonings are
almost the same; but we may discover, that one of these
apologists had been a lawyer, and the other a rhetorician.}

\pagenote[6411]{The more ancient as well as authentic memorials
of the church, relate many examples of the fact, (of these
\textit{severe trials},) which there is nothing to contradict.
Tertullian, among others, says, Nam proxime ad lenonem damnando
Christianam, potius quam ad leonem, confessi estis labem
pudicitiæ apud nos atrociorem omni pœna et omni morte reputari,
Apol. cap. ult. Eusebius likewise says, “Other virgins, dragged
to brothels, have lost their life rather than defile their
virtue.” Euseb. Hist. Ecc. viii. 14.—G. The miraculous
interpositions were the offspring of the coarse imaginations of
the monks.—M.}

\pagenote[65]{See two instances of this kind of torture in the
Acta Sincere Martyrum, published by Ruinart, p. 160, 399. Jerome,
in his Legend of Paul the Hermit, tells a strange story of a
young man, who was chained naked on a bed of flowers, and
assaulted by a beautiful and wanton courtesan. He quelled the
rising temptation by biting off his tongue.}

The total disregard of truth and probability in the
representation of these primitive martyrdoms was occasioned by a
very natural mistake. The ecclesiastical writers of the fourth or
fifth centuries ascribed to the magistrates of Rome the same
degree of implacable and unrelenting zeal which filled their own
breasts against the heretics or the idolaters of their own times.

It is not improbable that some of those persons who were raised
to the dignities of the empire, might have imbibed the prejudices
of the populace, and that the cruel disposition of others might
occasionally be stimulated by motives of avarice or of personal
resentment.\textsuperscript{66} But it is certain, and we may appeal to the
grateful confessions of the first Christians, that the greatest
part of those magistrates who exercised in the provinces the
authority of the emperor, or of the senate, and to whose hands
alone the jurisdiction of life and death was intrusted, behaved
like men of polished manners and liberal education, who respected
the rules of justice, and who were conversant with the precepts
of philosophy. They frequently declined the odious task of
persecution, dismissed the charge with contempt, or suggested to
the accused Christian some legal evasion, by which he might elude
the severity of the laws.\textsuperscript{67} Whenever they were invested with a
discretionary power,\textsuperscript{68} they used it much less for the
oppression, than for the relief and benefit of the afflicted
church. They were far from condemning all the Christians who were
accused before their tribunal, and very far from punishing with
death all those who were convicted of an obstinate adherence to
the new superstition. Contenting themselves, for the most part,
with the milder chastisements of imprisonment, exile, or slavery
in the mines,\textsuperscript{69} they left the unhappy victims of their justice
some reason to hope, that a prosperous event, the accession, the
marriage, or the triumph of an emperor, might speedily restore
them, by a general pardon, to their former state. The martyrs,
devoted to immediate execution by the Roman magistrates, appear
to have been selected from the most opposite extremes. They were
either bishops and presbyters, the persons the most distinguished
among the Christians by their rank and influence, and whose
example might strike terror into the whole sect;\textsuperscript{70} or else they
were the meanest and most abject among them, particularly those
of the servile condition, whose lives were esteemed of little
value, and whose sufferings were viewed by the ancients with too
careless an indifference.\textsuperscript{71} The learned Origen, who, from his
experience as well as reading, was intimately acquainted with the
history of the Christians, declares, in the most express terms,
that the number of martyrs was very inconsiderable.\textsuperscript{72} His
authority would alone be sufficient to annihilate that formidable
army of martyrs, whose relics, drawn for the most part from the
catacombs of Rome, have replenished so many churches,\textsuperscript{73} and
whose marvellous achievements have been the subject of so many
volumes of Holy Romance.\textsuperscript{74} But the general assertion of Origen
may be explained and confirmed by the particular testimony of his
friend Dionysius, who, in the immense city of Alexandria, and
under the rigorous persecution of Decius, reckons only ten men
and seven women who suffered for the profession of the Christian
name.\textsuperscript{75}

\pagenote[66]{The conversion of his wife provoked Claudius
Herminianus, governor of Cappadocia, to treat the Christians with
uncommon severity. Tertullian ad Scapulam, c. 3.}

\pagenote[67]{Tertullian, in his epistle to the governor of
Africa, mentions several remarkable instances of lenity and
forbearance, which had happened within his knowledge.}

\pagenote[68]{Neque enim in universum aliquid quod quasi certam
formam habeat, constitui potest; an expression of Trajan, which
gave a very great latitude to the governors of provinces. * Note:
Gibbon altogether forgets that Trajan fully approved of the
course pursued by Pliny. That course was, to order all who
persevered in their faith to be led to execution: perseverantes
duci jussi.—M.}

\pagenote[69]{In Metalla damnamur, in insulas relegamur.
Tertullian, Apolog. c. 12. The mines of Numidia contained nine
bishops, with a proportionable number of their clergy and people,
to whom Cyprian addressed a pious epistle of praise and comfort.
See Cyprian. Epistol. 76, 77.}

\pagenote[70]{Though we cannot receive with entire confidence
either the epistles, or the acts, of Ignatius, (they may be found
in the 2d volume of the Apostolic Fathers,) yet we may quote that
bishop of Antioch as one of these \textit{exemplary} martyrs. He was
sent in chains to Rome as a public spectacle, and when he arrived
at Troas, he received the pleasing intelligence, that the
persecution of Antioch was already at an end. * Note: The acts of
Ignatius are generally received as authentic, as are seven of his
letters. Eusebius and St. Jerome mention them: there are two
editions; in one, the letters are longer, and many passages
appear to have been interpolated; the other edition is that which
contains the real letters of St. Ignatius; such at least is the
opinion of the wisest and most enlightened critics. (See Lardner.
Cred. of Gospel Hist.) Less, uber dis Religion, v. i. p. 529.
Usser. Diss. de Ign. Epist. Pearson, Vindic, Ignatianæ. It should
be remarked, that it was under the reign of Trajan that the
bishop Ignatius was carried from Antioch to Rome, to be exposed
to the lions in the amphitheatre, the year of J. C. 107,
according to some; of 116, according to others.—G.}

\pagenote[71]{Among the martyrs of Lyons, (Euseb. l. v. c. 1,)
the slave Blandina was distinguished by more exquisite tortures.
Of the five martyrs so much celebrated in the acts of Felicitas
and Perpetua, two were of a servile, and two others of a very
mean, condition.}

\pagenote[72]{Origen. advers. Celsum, l. iii. p. 116. His words
deserve to be transcribed. * Note: The words that follow should
be quoted. “God not permitting that all his class of men should
be exterminated:” which appears to indicate that Origen thought
the number put to death inconsiderable only when compared to the
numbers who had survived. Besides this, he is speaking of the
state of the religion under Caracalla, Elagabalus, Alexander
Severus, and Philip, who had not persecuted the Christians. It
was during the reign of the latter that Origen wrote his books
against Celsus.—G.}

\pagenote[73]{If we recollect that all the Plebeians of Rome were
not Christians, and that all the Christians were not saints and
martyrs, we may judge with how much safety religious honors can
be ascribed to bones or urns, indiscriminately taken from the
public burial-place. After ten centuries of a very free and open
trade, some suspicions have arisen among the more learned
Catholics. They now require as a proof of sanctity and martyrdom,
the letters B.M., a vial full of red liquor supposed to be blood,
or the figure of a palm-tree. But the two former signs are of
little weight, and with regard to the last, it is observed by the
critics, 1. That the figure, as it is called, of a palm, is
perhaps a cypress, and perhaps only a stop, the flourish of a
comma used in the monumental inscriptions. 2. That the palm was
the symbol of victory among the Pagans. 3. That among the
Christians it served as the emblem, not only of martyrdom, but in
general of a joyful resurrection. See the epistle of P. Mabillon,
on the worship of unknown saints, and Muratori sopra le Antichita
Italiane, Dissertat. lviii.}

\pagenote[74]{As a specimen of these legends, we may be satisfied
with 10,000 Christian soldiers crucified in one day, either by
Trajan or Hadrian on Mount Ararat. See Baronius ad Martyrologium
Romanum; Tille mont, Mém. Ecclesiast. tom. ii. part ii. p. 438;
and Geddes’s Miscellanies, vol. ii. p. 203. The abbreviation of
Mil., which may signify either \textit{soldiers} or \textit{thousands}, is said
to have occasioned some extraordinary mistakes.}

\pagenote[75]{Dionysius ap. Euseb l. vi. c. 41 One of the
seventeen was likewise accused of robbery. * Note: Gibbon ought
to have said, was falsely accused of robbery, for so it is in the
Greek text. This Christian, named Nemesion, falsely accused of
robbery before the centurion, was acquitted of a crime altogether
foreign to his character, but he was led before the governor as
guilty of being a Christian, and the governor inflicted upon him
a double torture. (Euseb. loc. cit.) It must be added, that Saint
Dionysius only makes particular mention of the principal martyrs,
[this is very doubtful.—M.] and that he says, in general, that
the fury of the Pagans against the Christians gave to Alexandria
the appearance of a city taken by storm. [This refers to plunder
and ill usage, not to actual slaughter.—M.] Finally it should be
observed that Origen wrote before the persecution of the emperor
Decius.—G.}

During the same period of persecution, the zealous, the eloquent,
the ambitious Cyprian governed the church, not only of Carthage,
but even of Africa. He possessed every quality which could engage
the reverence of the faithful, or provoke the suspicions and
resentment of the Pagan magistrates. His character as well as his
station seemed to mark out that holy prelate as the most
distinguished object of envy and danger.\textsuperscript{76} The experience,
however, of the life of Cyprian, is sufficient to prove that our
fancy has exaggerated the perilous situation of a Christian
bishop; and the dangers to which he was exposed were less
imminent than those which temporal ambition is always prepared to
encounter in the pursuit of honors. Four Roman emperors, with
their families, their favorites, and their adherents, perished by
the sword in the space of ten years, during which the bishop of
Carthage guided by his authority and eloquence the councils of
the African church. It was only in the third year of his
administration, that he had reason, during a few months, to
apprehend the severe edicts of Decius, the vigilance of the
magistrate and the clamors of the multitude, who loudly demanded,
that Cyprian, the leader of the Christians, should be thrown to
the lions. Prudence suggested the necessity of a temporary
retreat, and the voice of prudence was obeyed. He withdrew
himself into an obscure solitude, from whence he could maintain a
constant correspondence with the clergy and people of Carthage;
and, concealing himself till the tempest was past, he preserved
his life, without relinquishing either his power or his
reputation. His extreme caution did not, however, escape the
censure of the more rigid Christians, who lamented, or the
reproaches of his personal enemies, who insulted, a conduct which
they considered as a pusillanimous and criminal desertion of the
most sacred duty.\textsuperscript{77} The propriety of reserving himself for the
future exigencies of the church, the example of several holy
bishops,\textsuperscript{78} and the divine admonitions, which, as he declares
himself, he frequently received in visions and ecstacies, were
the reasons alleged in his justification.\textsuperscript{79} But his best apology
may be found in the cheerful resolution, with which, about eight
years afterwards, he suffered death in the cause of religion. The
authentic history of his martyrdom has been recorded with unusual
candor and impartiality. A short abstract, therefore, of its most
important circumstances, will convey the clearest information of
the spirit, and of the forms, of the Roman persecutions.\textsuperscript{80}

\pagenote[76]{The letters of Cyprian exhibit a very curious and
original picture both of the \textit{man} and of the \textit{times}. See
likewise the two lives of Cyprian, composed with equal accuracy,
though with very different views; the one by Le Clerc
(Bibliothèque Universelle, tom. xii. p. 208-378,) the other by
Tillemont, Mémoires Ecclésiastiques, tom. iv part i. p. 76-459.}

\pagenote[77]{See the polite but severe epistle of the clergy of
Rome to the bishop of Carthage. (Cyprian. Epist. 8, 9.) Pontius
labors with the greatest care and diligence to justify his master
against the general censure.}

\pagenote[78]{In particular those of Dionysius of Alexandria, and
Gregory Thaumaturgus, of Neo-Cæsarea. See Euseb. Hist.
Ecclesiast. l. vi. c. 40; and Mémoires de Tillemont, tom. iv.
part ii. p. 685.}

\pagenote[79]{See Cyprian. Epist. 16, and his life by Pontius.}

\pagenote[80]{We have an original life of Cyprian by the deacon
Pontius, the companion of his exile, and the spectator of his
death; and we likewise possess the ancient proconsular acts of
his martyrdom. These two relations are consistent with each
other, and with probability; and what is somewhat remarkable,
they are both unsullied by any miraculous circumstances.}

\section{Part \thesection.}

When Valerian was consul for the third, and Gallienus for the
fourth time, Paternus, proconsul of Africa, summoned Cyprian to
appear in his private council-chamber. He there acquainted him
with the Imperial mandate which he had just received,\textsuperscript{81} that
those who had abandoned the Roman religion should immediately
return to the practice of the ceremonies of their ancestors.
Cyprian replied without hesitation, that he was a Christian and a
bishop, devoted to the worship of the true and only Deity, to
whom he offered up his daily supplications for the safety and
prosperity of the two emperors, his lawful sovereigns.

With modest confidence he pleaded the privilege of a citizen, in
refusing to give any answer to some invidious and indeed illegal
questions which the proconsul had proposed. A sentence of
banishment was pronounced as the penalty of Cyprian’s
disobedience; and he was conducted without delay to Curubis, a
free and maritime city of Zeugitania, in a pleasant situation, a
fertile territory, and at the distance of about forty miles from
Carthage.\textsuperscript{82} The exiled bishop enjoyed the conveniences of life
and the consciousness of virtue. His reputation was diffused over
Africa and Italy; an account of his behavior was published for
the edification of the Christian world;\textsuperscript{83} and his solitude was
frequently interrupted by the letters, the visits, and the
congratulations of the faithful. On the arrival of a new
proconsul in the province the fortune of Cyprian appeared for
some time to wear a still more favorable aspect. He was recalled
from banishment; and though not yet permitted to return to
Carthage, his own gardens in the neighborhood of the capital were
assigned for the place of his residence.\textsuperscript{84}

\pagenote[81]{It should seem that these were circular orders,
sent at the same time to all the governors. Dionysius (ap. Euseb.
l. vii. c. 11) relates the history of his own banishment from
Alexandria almost in the same manner. But as he escaped and
survived the persecution, we must account him either more or less
fortunate than Cyprian.}

\pagenote[82]{See Plin. Hist. Natur. v. 3. Cellarius, Geograph.
Antiq. part iii. p. 96. Shaw’s Travels, p. 90; and for the
adjacent country, (which is terminated by Cape Bona, or the
promontory of Mercury,) l’Afrique de Marmol. tom. ii. p. 494.
There are the remains of an aqueduct near Curubis, or Curbis, at
present altered into Gurbes; and Dr. Shaw read an inscription,
which styles that city \textit{Colonia Fulvia}. The deacon Pontius (in
Vit. Cyprian. c. 12) calls it “Apricum et competentem locum,
hospitium pro voluntate secretum, et quicquid apponi eis ante
promissum est, qui regnum et justitiam Dei quærunt.”}

\pagenote[83]{See Cyprian. Epistol. 77, edit. Fell.}

\pagenote[84]{Upon his conversion, he had sold those gardens for
the benefit of the poor. The indulgence of God (most probably the
liberality of some Christian friend) restored them to Cyprian.
See Pontius, c. 15.}

At length, exactly one year\textsuperscript{85} after Cyprian was first
apprehended, Galerius Maximus, proconsul of Africa, received the
Imperial warrant for the execution of the Christian teachers. The
bishop of Carthage was sensible that he should be singled out for
one of the first victims; and the frailty of nature tempted him
to withdraw himself, by a secret flight, from the danger and the
honor of martyrdom;\textsuperscript{8511} but soon recovering that fortitude which
his character required, he returned to his gardens, and patiently
expected the ministers of death. Two officers of rank, who were
intrusted with that commission, placed Cyprian between them in a
chariot, and as the proconsul was not then at leisure, they
conducted him, not to a prison, but to a private house in
Carthage, which belonged to one of them. An elegant supper was
provided for the entertainment of the bishop, and his Christian
friends were permitted for the last time to enjoy his society,
whilst the streets were filled with a multitude of the faithful,
anxious and alarmed at the approaching fate of their spiritual
father.\textsuperscript{86} In the morning he appeared before the tribunal of the
proconsul, who, after informing himself of the name and situation
of Cyprian, commanded him to offer sacrifice, and pressed him to
reflect on the consequences of his disobedience. The refusal of
Cyprian was firm and decisive; and the magistrate, when he had
taken the opinion of his council, pronounced with some reluctance
the sentence of death. It was conceived in the following terms:
“That Thascius Cyprianus should be immediately beheaded, as the
enemy of the gods of Rome, and as the chief and ringleader of a
criminal association, which he had seduced into an impious
resistance against the laws of the most holy emperors, Valerian
and Gallienus.”\textsuperscript{87} The manner of his execution was the mildest
and least painful that could be inflicted on a person convicted
of any capital offence; nor was the use of torture admitted to
obtain from the bishop of Carthage either the recantation of his
principles or the discovery of his accomplices.

\pagenote[85]{When Cyprian; a twelvemonth before, was sent into
exile, he dreamt that he should be put to death the next day. The
event made it necessary to explain that word, as signifying a
year. Pontius, c. 12.}

\pagenote[8511]{This was not, as it appears, the motive which
induced St. Cyprian to conceal himself for a short time; he was
threatened to be carried to Utica; he preferred remaining at
Carthage, in order to suffer martyrdom in the midst of his flock,
and in order that his death might conduce to the edification of
those whom he had guided during life. Such, at least, is his own
explanation of his conduct in one of his letters: Cum perlatum ad
nos fuisset, fratres carissimi, frumentarios esse missos qui me
Uticam per ducerent, consilioque carissimorum persuasum est, ut
de hortis interim recederemus, justa interveniente causâ,
consensi; eo quod congruat episcopum in eâ civitate, in quâ
Ecclesiæ dominicæ præest, illie. Dominum confiteri et plebem
universam præpositi præsentis confessione clarificari Ep. 83.—G}

\pagenote[86]{Pontius (c. 15) acknowledges that Cyprian, with
whom he supped, passed the night custodia delicata. The bishop
exercised a last and very proper act of jurisdiction, by
directing that the younger females, who watched in the streets,
should be removed from the dangers and temptations of a nocturnal
crowd. Act. Preconsularia, c. 2.}

\pagenote[87]{See the original sentence in the Acts, c. 4; and in
Pontius, c. 17 The latter expresses it in a more rhetorical
manner.}

As soon as the sentence was proclaimed, a general cry of “We will
die with him,” arose at once among the listening multitude of
Christians who waited before the palace gates. The generous
effusions of their zeal and their affection were neither
serviceable to Cyprian nor dangerous to themselves. He was led
away under a guard of tribunes and centurions, without resistance
and without insult, to the place of his execution, a spacious and
level plain near the city, which was already filled with great
numbers of spectators. His faithful presbyters and deacons were
permitted to accompany their holy bishop.\textsuperscript{8711} They assisted him
in laying aside his upper garment, spread linen on the ground to
catch the precious relics of his blood, and received his orders
to bestow five-and-twenty pieces of gold on the executioner. The
martyr then covered his face with his hands, and at one blow his
head was separated from his body. His corpse remained during some
hours exposed to the curiosity of the Gentiles: but in the night
it was removed, and transported in a triumphal procession, and
with a splendid illumination, to the burial-place of the
Christians. The funeral of Cyprian was publicly celebrated
without receiving any interruption from the Roman magistrates;
and those among the faithful, who had performed the last offices
to his person and his memory, were secure from the danger of
inquiry or of punishment. It is remarkable, that of so great a
multitude of bishops in the province of Africa, Cyprian was the
first who was esteemed worthy to obtain the crown of martyrdom.\textsuperscript{88}

\pagenote[8711]{There is nothing in the life of St. Cyprian, by
Pontius, nor in the ancient manuscripts, which can make us
suppose that the presbyters and deacons in their clerical
character, and known to be such, had the permission to attend
their holy bishop. Setting aside all religious considerations, it
is impossible not to be surprised at the kind of complaisance
with which the historian here insists, in favor of the
persecutors, on some mitigating circumstances allowed at the
death of a man whose only crime was maintaining his own opinions
with frankness and courage.—G.}

\pagenote[88]{Pontius, c. 19. M. de Tillemont (Mémoires, tom. iv.
part i. p. 450, note 50) is not pleased with so positive an
exclusion of any former martyr of the episcopal rank. * Note: M.
de. Tillemont, as an honest writer, explains the difficulties
which he felt about the text of Pontius, and concludes by
distinctly stating, that without doubt there is some mistake, and
that Pontius must have meant only Africa Minor or Carthage; for
St. Cyprian, in his 58th (69th) letter addressed to Pupianus,
speaks expressly of many bishops his colleagues, qui proscripti
sunt, vel apprehensi in carcere et catenis fuerunt; aut qui in
exilium relegati, illustri itinere ed Dominum profecti sunt; aut
qui quibusdam locis animadversi, cœlestes coronas de Domini
clarificatione sumpserunt.—G.}

It was in the choice of Cyprian, either to die a martyr, or to
live an apostate; but on the choice depended the alternative of
honor or infamy. Could we suppose that the bishop of Carthage had
employed the profession of the Christian faith only as the
instrument of his avarice or ambition, it was still incumbent on
him to support the character he had assumed;\textsuperscript{89} and if he
possessed the smallest degree of manly fortitude, rather to
expose himself to the most cruel tortures, than by a single act
to exchange the reputation of a whole life, for the abhorrence of
his Christian brethren, and the contempt of the Gentile world.
But if the zeal of Cyprian was supported by the sincere
conviction of the truth of those doctrines which he preached, the
crown of martyrdom must have appeared to him as an object of
desire rather than of terror. It is not easy to extract any
distinct ideas from the vague though eloquent declamations of the
Fathers, or to ascertain the degree of immortal glory and
happiness which they confidently promised to those who were so
fortunate as to shed their blood in the cause of religion.\textsuperscript{90}
They inculcated with becoming diligence, that the fire of
martyrdom supplied every defect and expiated every sin; that
while the souls of ordinary Christians were obliged to pass
through a slow and painful purification, the triumphant sufferers
entered into the immediate fruition of eternal bliss, where, in
the society of the patriarchs, the apostles, and the prophets,
they reigned with Christ, and acted as his assessors in the
universal judgment of mankind. The assurance of a lasting
reputation upon earth, a motive so congenial to the vanity of
human nature, often served to animate the courage of the martyrs.

The honors which Rome or Athens bestowed on those citizens who
had fallen in the cause of their country, were cold and unmeaning
demonstrations of respect, when compared with the ardent
gratitude and devotion which the primitive church expressed
towards the victorious champions of the faith. The annual
commemoration of their virtues and sufferings was observed as a
sacred ceremony, and at length terminated in religious worship.
Among the Christians who had publicly confessed their religious
principles, those who (as it very frequently happened) had been
dismissed from the tribunal or the prisons of the Pagan
magistrates, obtained such honors as were justly due to their
imperfect martyrdom and their generous resolution. The most pious
females courted the permission of imprinting kisses on the
fetters which they had worn, and on the wounds which they had
received. Their persons were esteemed holy, their decisions were
admitted with deference, and they too often abused, by their
spiritual pride and licentious manners, the preëminence which
their zeal and intrepidity had acquired.\textsuperscript{91} Distinctions like
these, whilst they display the exalted merit, betray the
inconsiderable number of those who suffered, and of those who
died, for the profession of Christianity.

\pagenote[89]{Whatever opinion we may entertain of the character
or principles of Thomas Becket, we must acknowledge that he
suffered death with a constancy not unworthy of the primitive
martyrs. See Lord Lyttleton’s History of Henry II. vol. ii. p.
592, \&c.}

\pagenote[90]{See in particular the treatise of Cyprian de
Lapsis, p. 87-98, edit. Fell. The learning of Dodwell (Dissertat.
Cyprianic. xii. xiii.,) and the ingenuity of Middleton, (Free
Inquiry, p. 162, \&c.,) have left scarcely any thing to add
concerning the merit, the honors, and the motives of the
martyrs.}

\pagenote[91]{Cyprian. Epistol. 5, 6, 7, 22, 24; and de Unitat.
Ecclesiæ. The number of pretended martyrs has been very much
multiplied, by the custom which was introduced of bestowing that
honorable name on confessors. Note: M. Guizot denies that the
letters of Cyprian, to which he refers, bear out the statement in
the text. I cannot scruple to admit the accuracy of Gibbon’s
quotation. To take only the fifth letter, we find this passage:
Doleo enim quando audio quosdam improbe et insolenter discurrere,
et ad ineptian vel ad discordias vacare, Christi membra et jam
Christum confessa per concubitûs illicitos inquinari, nec a
diaconis aut presbyteris regi posse, sed id agere ut per paucorum
pravos et malos mores, multorum et bonorum confessorum gloria
honesta maculetur. Gibbon’s misrepresentation lies in the
ambiguous expression “too often.” Were the epistles arranged in a
different manner in the edition consulted by M. Guizot?—M.}

The sober discretion of the present age will more readily censure
than admire, but can more easily admire than imitate, the fervor
of the first Christians, who, according to the lively expressions
of Sulpicius Severus, desired martyrdom with more eagerness than
his own contemporaries solicited a bishopric.\textsuperscript{92} The epistles
which Ignatius composed as he was carried in chains through the
cities of Asia, breathe sentiments the most repugnant to the
ordinary feelings of human nature. He earnestly beseeches the
Romans, that when he should be exposed in the amphitheatre, they
would not, by their kind but unseasonable intercession, deprive
him of the crown of glory; and he declares his resolution to
provoke and irritate the wild beasts which might be employed as
the instruments of his death.\textsuperscript{93} Some stories are related of the
courage of martyrs, who actually performed what Ignatius had
intended; who exasperated the fury of the lions, pressed the
executioner to hasten his office, cheerfully leaped into the
fires which were kindled to consume them, and discovered a
sensation of joy and pleasure in the midst of the most exquisite
tortures. Several examples have been preserved of a zeal
impatient of those restraints which the emperors had provided for
the security of the church. The Christians sometimes supplied by
their voluntary declaration the want of an accuser, rudely
disturbed the public service of paganism,\textsuperscript{94} and rushing in
crowds round the tribunal of the magistrates, called upon them to
pronounce and to inflict the sentence of the law. The behavior of
the Christians was too remarkable to escape the notice of the
ancient philosophers; but they seem to have considered it with
much less admiration than astonishment. Incapable of conceiving
the motives which sometimes transported the fortitude of
believers beyond the bounds of prudence or reason, they treated
such an eagerness to die as the strange result of obstinate
despair, of stupid insensibility, or of superstitious frenzy.\textsuperscript{95}
“Unhappy men!” exclaimed the proconsul Antoninus to the
Christians of Asia; “unhappy men! if you are thus weary of your
lives, is it so difficult for you to find ropes and precipices?”\textsuperscript{96}
He was extremely cautious (as it is observed by a learned and
picus historian) of punishing men who had found no accusers but
themselves, the Imperial laws not having made any provision for
so unexpected a case: condemning therefore a few as a warning to
their brethren, he dismissed the multitude with indignation and
contempt.\textsuperscript{97} Notwithstanding this real or affected disdain, the
intrepid constancy of the faithful was productive of more
salutary effects on those minds which nature or grace had
disposed for the easy reception of religious truth. On these
melancholy occasions, there were many among the Gentiles who
pitied, who admired, and who were converted. The generous
enthusiasm was communicated from the sufferer to the spectators;
and the blood of martyrs, according to a well-known observation,
became the seed of the church.

\pagenote[92]{Certatim gloriosa in certamina ruebatur; multique
avidius tum martyria gloriosis mortibus quærebantur, quam nunc
Episcopatus pravis ambitionibus appetuntur. Sulpicius Severus, l.
ii. He might have omitted the word \textit{nunc}.}

\pagenote[93]{See Epist. ad Roman. c. 4, 5, ap. Patres Apostol.
tom. ii. p. 27. It suited the purpose of Bishop Pearson (see
Vindiciæ Ignatianæ, part ii. c. 9) to justify, by a profusion of
examples and authorities, the sentiments of Ignatius.}

\pagenote[94]{The story of Polyeuctes, on which Corneille has
founded a very beautiful tragedy, is one of the most celebrated,
though not perhaps the most authentic, instances of this
excessive zeal. We should observe, that the 60th canon of the
council of Illiberis refuses the title of martyrs to those who
exposed themselves to death, by publicly destroying the idols.}

\pagenote[95]{See Epictetus, l. iv. c. 7, (though there is some
doubt whether he alludes to the Christians.) Marcus Antoninus de
Rebus suis, l. xi. c. 3 Lucian in Peregrin.}

\pagenote[96]{Tertullian ad Scapul. c. 5. The learned are divided
between three persons of the same name, who were all proconsuls
of Asia. I am inclined to ascribe this story to Antoninus Pius,
who was afterwards emperor; and who may have governed Asia under
the reign of Trajan.}

\pagenote[97]{Mosheim, de Rebus Christ, ante Constantin. p. 235.}

But although devotion had raised, and eloquence continued to
inflame, this fever of the mind, it insensibly gave way to the
more natural hopes and fears of the human heart, to the love of
life, the apprehension of pain, and the horror of dissolution.
The more prudent rulers of the church found themselves obliged to
restrain the indiscreet ardor of their followers, and to distrust
a constancy which too often abandoned them in the hour of trial.\textsuperscript{98}
As the lives of the faithful became less mortified and
austere, they were every day less ambitious of the honors of
martyrdom; and the soldiers of Christ, instead of distinguishing
themselves by voluntary deeds of heroism, frequently deserted
their post, and fled in confusion before the enemy whom it was
their duty to resist. There were three methods, however, of
escaping the flames of persecution, which were not attended with
an equal degree of guilt: first, indeed, was generally allowed to
be innocent; the second was of a doubtful, or at least of a
venial, nature; but the third implied a direct and criminal
apostasy from the Christian faith.

\pagenote[98]{See the Epistle of the Church of Smyrna, ap. Euseb.
Hist. Eccles. Liv. c. 15 * Note: The 15th chapter of the 10th
book of the Eccles. History of Eusebius treats principally of the
martyrdom of St. Polycarp, and mentions some other martyrs. A
single example of weakness is related; it is that of a Phrygian
named Quintus, who, appalled at the sight of the wild beasts and
the tortures, renounced his faith. This example proves little
against the mass of Christians, and this chapter of Eusebius
furnished much stronger evidence of their courage than of their
timidity.—G——This Quintus had, however, rashly and of his own
accord appeared before the tribunal; and the church of Smyrna
condemn “\textit{his indiscreet ardor},” coupled as it was with weakness
in the hour of trial.—M.}

I. A modern inquisitor would hear with surprise, that whenever an
information was given to a Roman magistrate of any person within
his jurisdiction who had embraced the sect of the Christians, the
charge was communicated to the party accused, and that a
convenient time was allowed him to settle his domestic concerns,
and to prepare an answer to the crime which was imputed to him.\textsuperscript{99}
If he entertained any doubt of his own constancy, such a delay
afforded him the opportunity of preserving his life and honor by
flight, of withdrawing himself into some obscure retirement or
some distant province, and of patiently expecting the return of
peace and security. A measure so consonant to reason was soon
authorized by the advice and example of the most holy prelates;
and seems to have been censured by few except by the Montanists,
who deviated into heresy by their strict and obstinate adherence
to the rigor of ancient discipline.\textsuperscript{100}

II.The provincial governors, whose zeal was less prevalent than
their avarice, had countenanced the practice of selling
certificates, (or libels, as they were called,) which attested,
that the persons therein mentioned had complied with the laws,
and sacrificed to the Roman deities. By producing these false
declarations, the opulent and timid Christians were enabled to
silence the malice of an informer, and to reconcile in some
measure their safety with their religion\textsuperscript{101} A slight penance
atoned for this profane dissimulation.\textsuperscript{1011}

III. In every persecution there were great numbers of unworthy
Christians who publicly disowned or renounced the faith which
they had professed; and who confirmed the sincerity of their
abjuration, by the legal acts of burning incense or of offering
sacrifices. Some of these apostates had yielded on the first
menace or exhortation of the magistrate; whilst the patience of
others had been subdued by the length and repetition of tortures.
The affrighted countenances of some betrayed their inward
remorse, while others advanced with confidence and alacrity to
the altars of the gods.\textsuperscript{102} But the disguise which fear had
imposed, subsisted no longer than the present danger. As soon as
the severity of the persecution was abated, the doors of the
churches were assailed by the returning multitude of penitents
who detested their idolatrous submission, and who solicited with
equal ardor, but with various success, their readmission into the
society of Christians.\textsuperscript{103} \textsuperscript{1031}

\pagenote[99]{In the second apology of Justin, there is a
particular and very curious instance of this legal delay. The
same indulgence was granted to accused Christians, in the
persecution of Decius: and Cyprian (de Lapsis) expressly mentions
the “Dies negantibus præstitutus.” * Note: The examples drawn by
the historian from Justin Martyr and Cyprian relate altogether to
particular cases, and prove nothing as to the general practice
adopted towards the accused; it is evident, on the contrary, from
the same apology of St. Justin, that they hardly ever obtained
delay. “A man named Lucius, himself a Christian, present at an
unjust sentence passed against a Christian by the judge Urbicus,
asked him why he thus punished a man who was neither adulterer
nor robber, nor guilty of any other crime but that of avowing
himself a Christian.” Urbicus answered only in these words: “Thou
also hast the appearance of being a Christian.” “Yes, without
doubt,” replied Lucius. The judge ordered that he should be put
to death on the instant. A third, who came up, was condemned to
be beaten with rods. Here, then, are three examples where no
delay was granted.——[Surely these acts of a single passionate and
irritated judge prove the general practice as little as those
quoted by Gibbon.—M.] There exist a multitude of others, such as
those of Ptolemy, Marcellus, \&c. Justin expressly charges the
judges with ordering the accused to be executed without hearing
the cause. The words of St. Cyprian are as particular, and simply
say, that he had appointed a day by which the Christians must
have renounced their faith; those who had not done it by that
time were condemned.—G. This confirms the statement in the
text.—M.}

\pagenote[100]{Tertullian considers flight from persecution as an
imperfect, but very criminal, apostasy, as an impious attempt to
elude the will of God, \&c., \&c. He has written a treatise on this
subject, (see p. 536—544, edit. Rigalt.,) which is filled with
the wildest fanaticism and the most incoherent declamation. It
is, however, somewhat remarkable, that Tertullian did not suffer
martyrdom himself.}

\pagenote[101]{The \textit{libellatici}, who are chiefly known by the
writings of Cyprian, are described with the utmost precision, in
the copious commentary of Mosheim, p. 483—489.}

\pagenote[1011]{The penance was not so slight, for it was exactly
the same with that of apostates who had sacrificed to idols; it
lasted several years. See Fleun Hist. Ecc. v. ii. p. 171.—G.}

\pagenote[102]{Plin. Epist. x. 97. Dionysius Alexandrin. ap.
Euseb. l. vi. c. 41. Ad prima statim verba minantis inimici
maximus fratrum numerus fidem suam prodidit: nec prostratus est
persecutionis impetu, sed voluntario lapsu seipsum prostravit.
Cyprian. Opera, p. 89. Among these deserters were many priests,
and even bishops.}

\pagenote[103]{It was on this occasion that Cyprian wrote his
treatise De Lapsis, and many of his epistles. The controversy
concerning the treatment of penitent apostates, does not occur
among the Christians of the preceding century. Shall we ascribe
this to the superiority of their faith and courage, or to our
less intimate knowledge of their history!}

\pagenote[1031]{Pliny says, that the greater part of the
Christians persisted in avowing themselves to be so; the reason
for his consulting Trajan was the periclitantium numerus.
Eusebius (l. vi. c. 41) does not permit us to doubt that the
number of those who renounced their faith was infinitely below
the number of those who boldly confessed it. The prefect, he says
and his assessors present at the council, were alarmed at seeing
the crowd of Christians; the judges themselves trembled. Lastly,
St. Cyprian informs us, that the greater part of those who had
appeared weak brethren in the persecution of Decius, signalized
their courage in that of Gallius. Steterunt fortes, et ipso
dolore pœnitentiæ facti ad prælium fortiores Epist. lx. p.
142.—G.}

IV. Notwithstanding the general rules established for the
conviction and punishment of the Christians, the fate of those
sectaries, in an extensive and arbitrary government, must still
in a great measure, have depended on their own behavior, the
circumstances of the times, and the temper of their supreme as
well as subordinate rulers. Zeal might sometimes provoke, and
prudence might sometimes avert or assuage, the superstitious fury
of the Pagans. A variety of motives might dispose the provincial
governors either to enforce or to relax the execution of the
laws; and of these motives the most forcible was their regard not
only for the public edicts, but for the secret intentions of the
emperor, a glance from whose eye was sufficient to kindle or to
extinguish the flames of persecution. As often as any occasional
severities were exercised in the different parts of the empire,
the primitive Christians lamented and perhaps magnified their own
sufferings; but the celebrated number of \textit{ten} persecutions has
been determined by the ecclesiastical writers of the fifth
century, who possessed a more distinct view of the prosperous or
adverse fortunes of the church, from the age of Nero to that of
Diocletian. The ingenious parallels of the \textit{ten} plagues of
Egypt, and of the \textit{ten} horns of the Apocalypse, first suggested
this calculation to their minds; and in their application of the
faith of prophecy to the truth of history, they were careful to
select those reigns which were indeed the most hostile to the
Christian cause.\textsuperscript{104} But these transient persecutions served only
to revive the zeal and to restore the discipline of the faithful;
and the moments of extraordinary rigor were compensated by much
longer intervals of peace and security. The indifference of some
princes, and the indulgence of others, permitted the Christians
to enjoy, though not perhaps a legal, yet an actual and public,
toleration of their religion.

\pagenote[104]{See Mosheim, p. 97. Sulpicius Severus was the
first author of this computation; though he seemed desirous of
reserving the tenth and greatest persecution for the coming of
the Antichrist.}

\section{Part \thesection.}

The apology of Tertullian contains two very ancient, very
singular, but at the same time very suspicious, instances of
Imperial clemency; the edicts published by Tiberius, and by
Marcus Antoninus, and designed not only to protect the innocence
of the Christians, but even to proclaim those stupendous miracles
which had attested the truth of their doctrine. The first of
these examples is attended with some difficulties which might
perplex a sceptical mind.\textsuperscript{105} We are required to believe, \textit{that}
Pontius Pilate informed the emperor of the unjust sentence of
death which he had pronounced against an innocent, and, as it
appeared, a divine, person; and that, without acquiring the
merit, he exposed himself to the danger of martyrdom; \textit{that}
Tiberius, who avowed his contempt for all religion, immediately
conceived the design of placing the Jewish Messiah among the gods
of Rome; \textit{that} his servile senate ventured to disobey the
commands of their master; \textit{that} Tiberius, instead of resenting
their refusal, contented himself with protecting the Christians
from the severity of the laws, many years before such laws were
enacted, or before the church had assumed any distinct name or
existence; and lastly, \textit{that} the memory of this extraordinary
transaction was preserved in the most public and authentic
records, which escaped the knowledge of the historians of Greece
and Rome, and were only visible to the eyes of an African
Christian, who composed his apology one hundred and sixty years
after the death of Tiberius. The edict of Marcus Antoninus is
supposed to have been the effect of his devotion and gratitude
for the miraculous deliverance which he had obtained in the
Marcomannic war. The distress of the legions, the seasonable
tempest of rain and hail, of thunder and of lightning, and the
dismay and defeat of the barbarians, have been celebrated by the
eloquence of several Pagan writers. If there were any Christians
in that army, it was natural that they should ascribe some merit
to the fervent prayers, which, in the moment of danger, they had
offered up for their own and the public safety. But we are still
assured by monuments of brass and marble, by the Imperial medals,
and by the Antonine column, that neither the prince nor the
people entertained any sense of this signal obligation, since
they unanimously attribute their deliverance to the providence of
Jupiter, and to the interposition of Mercury.\textsuperscript{106} During the
whole course of his reign, Marcus despised the Christians as a
philosopher, and punished them as a sovereign. \textsuperscript{1061}

\pagenote[105]{The testimony given by Pontius Pilate is first
mentioned by Justin. The successive improvements which the story
acquired (as if has passed through the hands of Tertullian,
Eusebius, Epiphanius, Chrysostom, Orosius, Gregory of Tours, and
the authors of the several editions of the acts of Pilate) are
very fairly stated by Dom Calmet Dissertat. sur l’Ecriture, tom.
iii. p. 651, \&c.}

\pagenote[106]{On this miracle, as it is commonly called, of the
thundering legion, see the admirable criticism of Mr. Moyle, in
his Works, vol. ii. p. 81—390.}

\pagenote[1061]{Gibbon, with this phrase, and that below, which
admits the injustice of Marcus, has dexterously glossed over one
of the most remarkable facts in the early Christian history, that
the reign of the wisest and most humane of the heathen emperors
was the most fatal to the Christians. Most writers have ascribed
the persecutions under Marcus to the latent bigotry of his
character; Mosheim, to the influence of the philosophic party;
but the fact is admitted by all. A late writer (Mr. Waddington,
Hist. of the Church, p. 47) has not scrupled to assert, that
“this prince polluted every year of a long reign with innocent
blood;” but the causes as well as the date of the persecutions
authorized or permitted by Marcus are equally uncertain. Of the
Asiatic edict recorded by Melito. the date is unknown, nor is it
quite clear that it was an Imperial edict. If it was the act
under which Polycarp suffered, his martyrdom is placed by Ruinart
in the sixth, by Mosheim in the ninth, year of the reign of
Marcus. The martyrs of Vienne and Lyons are assigned by Dodwell
to the seventh, by most writers to the seventeenth. In fact, the
commencement of the persecutions of the Christians appears to
synchronize exactly with the period of the breaking out of the
Marcomannic war, which seems to have alarmed the whole empire,
and the emperor himself, into a paroxysm of returning piety to
their gods, of which the Christians were the victims. See Jul,
Capit. Script. Hist August. p. 181, edit. 1661. It is remarkable
that Tertullian (Apologet. c. v.) distinctly asserts that Verus
(M. Aurelius) issued no edicts against the Christians, and almost
positively exempts him from the charge of persecution.—M. This
remarkable synchronism, which explains the persecutions under M
Aurelius, is shown at length in Milman’s History of Christianity,
book ii. v.—M. 1845.}

By a singular fatality, the hardships which they had endured
under the government of a virtuous prince, immediately ceased on
the accession of a tyrant; and as none except themselves had
experienced the injustice of Marcus, so they alone were protected
by the lenity of Commodus. The celebrated Marcia, the most
favored of his concubines, and who at length contrived the murder
of her Imperial lover, entertained a singular affection for the
oppressed church; and though it was impossible that she could
reconcile the practice of vice with the precepts of the gospel,
she might hope to atone for the frailties of her sex and
profession by declaring herself the patroness of the Christians. \textsuperscript{107}
Under the gracious protection of Marcia, they passed in
safety the thirteen years of a cruel tyranny; and when the empire
was established in the house of Severus, they formed a domestic
but more honorable connection with the new court. The emperor was
persuaded, that in a dangerous sickness, he had derived some
benefit, either spiritual or physical, from the holy oil, with
which one of his slaves had anointed him. He always treated with
peculiar distinction several persons of both sexes who had
embraced the new religion. The nurse as well as the preceptor of
Caracalla were Christians;\textsuperscript{1071} and if that young prince ever
betrayed a sentiment of humanity, it was occasioned by an
incident, which, however trifling, bore some relation to the
cause of Christianity.\textsuperscript{108} Under the reign of Severus, the fury
of the populace was checked; the rigor of ancient laws was for
some time suspended; and the provincial governors were satisfied
with receiving an annual present from the churches within their
jurisdiction, as the price, or as the reward, of their
moderation.\textsuperscript{109} The controversy concerning the precise time of
the celebration of Easter, armed the bishops of Asia and Italy
against each other, and was considered as the most important
business of this period of leisure and tranquillity.\textsuperscript{110} Nor was
the peace of the church interrupted, till the increasing numbers
of proselytes seem at length to have attracted the attention, and
to have alienated the mind of Severus. With the design of
restraining the progress of Christianity, he published an edict,
which, though it was designed to affect only the new converts,
could not be carried into strict execution, without exposing to
danger and punishment the most zealous of their teachers and
missionaries. In this mitigated persecution we may still discover
the indulgent spirit of Rome and of Polytheism, which so readily
admitted every excuse in favor of those who practised the
religious ceremonies of their fathers.\textsuperscript{111}

\pagenote[107]{Dion Cassius, or rather his abbreviator Xiphilin,
l. lxxii. p. 1206. Mr. Moyle (p. 266) has explained the condition
of the church under the reign of Commodus.}

\pagenote[1071]{The Jews and Christians contest the honor of
having furnished a nurse is the fratricide son of Severus
Caracalla. Hist. of Jews, iii. 158.—M.}

\pagenote[108]{Compare the life of Caracalla in the Augustan
History, with the epistle of Tertullian to Scapula. Dr. Jortin
(Remarks on Ecclesiastical History, vol. ii. p. 5, \&c.) considers
the cure of Severus by the means of holy oil, with a strong
desire to convert it into a miracle.}

\pagenote[109]{Tertullian de Fuga, c. 13. The present was made
during the feast of the Saturnalia; and it is a matter of serious
concern to Tertullian, that the faithful should be confounded
with the most infamous professions which purchased the connivance
of the government.}

\pagenote[110]{Euseb. l. v. c. 23, 24. Mosheim, p. 435—447.}

\pagenote[111]{Judæos fieri sub gravi pœna vetuit. Idem etiam de
Christianis sanxit. Hist. August. p. 70.}

But the laws which Severus had enacted soon expired with the
authority of that emperor; and the Christians, after this
accidental tempest, enjoyed a calm of thirty-eight years.\textsuperscript{112}
Till this period they had usually held their assemblies in
private houses and sequestered places. They were now permitted to
erect and consecrate convenient edifices for the purpose of
religious worship;\textsuperscript{113} to purchase lands, even at Rome itself,
for the use of the community; and to conduct the elections of
their ecclesiastical ministers in so public, but at the same time
in so exemplary a manner, as to deserve the respectful attention
of the Gentiles.\textsuperscript{114} This long repose of the church was
accompanied with dignity. The reigns of those princes who derived
their extraction from the Asiatic provinces, proved the most
favorable to the Christians; the eminent persons of the sect,
instead of being reduced to implore the protection of a slave or
concubine, were admitted into the palace in the honorable
characters of priests and philosophers; and their mysterious
doctrines, which were already diffused among the people,
insensibly attracted the curiosity of their sovereign. When the
empress Mammæa passed through Antioch, she expressed a desire of
conversing with the celebrated Origen, the fame of whose piety
and learning was spread over the East. Origen obeyed so
flattering an invitation, and though he could not expect to
succeed in the conversion of an artful and ambitious woman, she
listened with pleasure to his eloquent exhortations, and
honorably dismissed him to his retirement in Palestine.\textsuperscript{115} The
sentiments of Mammæa were adopted by her son Alexander, and the
philosophic devotion of that emperor was marked by a singular but
injudicious regard for the Christian religion. In his domestic
chapel he placed the statues of Abraham, of Orpheus, of
Apollonius, and of Christ, as an honor justly due to those
respectable sages who had instructed mankind in the various modes
of addressing their homage to the supreme and universal Deity. \textsuperscript{116}
A purer faith, as well as worship, was openly professed and
practised among his household. Bishops, perhaps for the first
time, were seen at court; and, after the death of Alexander, when
the inhuman Maximin discharged his fury on the favorites and
servants of his unfortunate benefactor, a great number of
Christians of every rank and of both sexes, were involved in the
promiscuous massacre, which, on their account, has improperly
received the name of Persecution.\textsuperscript{117} \textsuperscript{1171}

\pagenote[112]{Sulpicius Severus, l. ii. p. 384. This computation
(allowing for a single exception) is confirmed by the history of
Eusebius, and by the writings of Cyprian.}

\pagenote[113]{The antiquity of Christian churches is discussed
by Tillemont, (Mémoires Ecclésiastiques, tom. iii. part ii. p.
68-72,) and by Mr. Moyle, (vol. i. p. 378-398.) The former refers
the first construction of them to the peace of Alexander Severus;
the latter, to the peace of Gallienus.}

\pagenote[114]{See the Augustan History, p. 130. The emperor
Alexander adopted their method of publicly proposing the names of
those persons who were candidates for ordination. It is true that
the honor of this practice is likewise attributed to the Jews.}

\pagenote[115]{Euseb. Hist. Ecclesiast. l. vi. c. 21. Hieronym.
de Script. Eccles. c. 54. Mammæa was styled a holy and pious
woman, both by the Christians and the Pagans. From the former,
therefore, it was impossible that she should deserve that
honorable epithet.}

\pagenote[116]{See the Augustan History, p. 123. Mosheim (p. 465)
seems to refine too much on the domestic religion of Alexander.
His design of building a public temple to Christ, (Hist. August.
p. 129,) and the objection which was suggested either to him, or
in similar circumstances to Hadrian, appear to have no other
foundation than an improbable report, invented by the Christians,
and credulously adopted by an historian of the age of
Constantine.}

\pagenote[117]{Euseb. l. vi. c. 28. It may be presumed that the
success of the Christians had exasperated the increasing bigotry
of the Pagans. Dion Cassius, who composed his history under the
former reign, had most probably intended for the use of his
master those counsels of persecution, which he ascribes to a
better age, and to and to the favorite of Augustus. Concerning
this oration of Mæcenas, or rather of Dion, I may refer to my own
unbiased opinion, (vol. i. c. 1, note 25,) and to the Abbé de la
Bleterie (Mémoires de l’Académie, tom. xxiv. p. 303 tom xxv. p.
432.) * Note: If this be the case, Dion Cassius must have known
the Christians they must have been the subject of his particular
attention, since the author supposes that he wished his master to
profit by these “counsels of persecution.” How are we to
reconcile this necessary consequence with what Gibbon has said of
the ignorance of Dion Cassius even of the name of the Christians?
(c. xvi. n. 24.) (Gibbon speaks of Dion’s \textit{silence}, not of his
\textit{ignorance}.—M) The supposition in this note is supported by no
proof; it is probable that Dion Cassius has often designated the
Christians by the name of Jews. See Dion Cassius, l. lxvii. c 14,
lxviii. l—G. On this point I should adopt the view of Gibbon
rather than that of M Guizot.—M}

\pagenote[1171]{It is with good reason that this massacre has
been called a persecution, for it lasted during the whole reign
of Maximin, as may be seen in Eusebius. (l. vi. c. 28.) Rufinus
expressly confirms it: Tribus annis a Maximino persecutione
commota, in quibus finem et persecutionis fecit et vitas Hist. l.
vi. c. 19.—G.}

Notwithstanding the cruel disposition of Maximin, the effects of
his resentment against the Christians were of a very local and
temporary nature, and the pious Origen, who had been proscribed
as a devoted victim, was still reserved to convey the truths of
the gospel to the ear of monarchs.\textsuperscript{118} He addressed several
edifying letters to the emperor Philip, to his wife, and to his
mother; and as soon as that prince, who was born in the
neighborhood of Palestine, had usurped the Imperial sceptre, the
Christians acquired a friend and a protector. The public and even
partial favor of Philip towards the sectaries of the new
religion, and his constant reverence for the ministers of the
church, gave some color to the suspicion, which prevailed in his
own times, that the emperor himself was become a convert to the
faith;\textsuperscript{119} and afforded some grounds for a fable which was
afterwards invented, that he had been purified by confession and
penance from the guilt contracted by the murder of his innocent
predecessor.\textsuperscript{120} The fall of Philip introduced, with the change
of masters, a new system of government, so oppressive to the
Christians, that their former condition, ever since the time of
Domitian, was represented as a state of perfect freedom and
security, if compared with the rigorous treatment which they
experienced under the short reign of Decius.\textsuperscript{121} The virtues of
that prince will scarcely allow us to suspect that he was
actuated by a mean resentment against the favorites of his
predecessor; and it is more reasonable to believe, that in the
prosecution of his general design to restore the purity of Roman
manners, he was desirous of delivering the empire from what he
condemned as a recent and criminal superstition. The bishops of
the most considerable cities were removed by exile or death: the
vigilance of the magistrates prevented the clergy of Rome during
sixteen months from proceeding to a new election; and it was the
opinion of the Christians, that the emperor would more patiently
endure a competitor for the purple, than a bishop in the capital. \textsuperscript{122}
Were it possible to suppose that the penetration of Decius
had discovered pride under the disguise of humility, or that he
could foresee the temporal dominion which might insensibly arise
from the claims of spiritual authority, we might be less
surprised, that he should consider the successors of St. Peter,
as the most formidable rivals to those of Augustus.

\pagenote[118]{Orosius, l. vii. c. 19, mentions Origen as the
object of Maximin’s resentment; and Firmilianus, a Cappadocian
bishop of that age, gives a just and confined idea of this
persecution, (apud Cyprian Epist. 75.)}

\pagenote[119]{The mention of those princes who were publicly
supposed to be Christians, as we find it in an epistle of
Dionysius of Alexandria, (ap. Euseb. l. vii. c. 10,) evidently
alludes to Philip and his family, and forms a contemporary
evidence, that such a report had prevailed; but the Egyptian
bishop, who lived at an humble distance from the court of Rome,
expresses himself with a becoming diffidence concerning the truth
of the fact. The epistles of Origen (which were extant in the
time of Eusebius, see l. vi. c. 36) would most probably decide
this curious rather than important question.}

\pagenote[120]{Euseb. l. vi. c. 34. The story, as is usual, has
been embellished by succeeding writers, and is confuted, with
much superfluous learning, by Frederick Spanheim, (Opera Varia,
tom. ii. p. 400, \&c.)}

\pagenote[121]{Lactantius, de Mortibus Persecutorum, c. 3, 4.
After celebrating the felicity and increase of the church, under
a long succession of good princes, he adds, “Extitit post annos
plurimos, execrabile animal, Decius, qui vexaret Ecclesiam.”}

\pagenote[122]{Euseb. l. vi. c. 39. Cyprian. Epistol. 55. The see
of Rome remained vacant from the martyrdom of Fabianus, the 20th
of January, A. D. 259, till the election of Cornelius, the 4th of
June, A. D. 251 Decius had probably left Rome, since he was
killed before the end of that year.}

The administration of Valerian was distinguished by a levity and
inconstancy ill suited to the gravity of the \textit{Roman Censor}. In
the first part of his reign, he surpassed in clemency those
princes who had been suspected of an attachment to the Christian
faith. In the last three years and a half, listening to the
insinuations of a minister addicted to the superstitions of
Egypt, he adopted the maxims, and imitated the severity, of his
predecessor Decius.\textsuperscript{123} The accession of Gallienus, which
increased the calamities of the empire, restored peace to the
church; and the Christians obtained the free exercise of their
religion by an edict addressed to the bishops, and conceived in
such terms as seemed to acknowledge their office and public
character.\textsuperscript{124} The ancient laws, without being formally repealed,
were suffered to sink into oblivion; and (excepting only some
hostile intentions which are attributed to the emperor Aurelian\textsuperscript{125}
the disciples of Christ passed above forty years in a state
of prosperity, far more dangerous to their virtue than the
severest trials of persecution.

\pagenote[123]{Euseb. l. vii. c. 10. Mosheim (p. 548) has very
clearly shown that the præfect Macrianus, and the Egyptian
\textit{Magus}, are one and the same person.}

\pagenote[124]{Eusebius (l. vii. c. 13) gives us a Greek version
of this Latin edict, which seems to have been very concise. By
another edict, he directed that the \textit{Cæmeteria} should be
restored to the Christians.}

\pagenote[125]{Euseb. l. vii. c. 30. Lactantius de M. P. c. 6.
Hieronym. in Chron. p. 177. Orosius, l. vii. c. 23. Their
language is in general so ambiguous and incorrect, that we are at
a loss to determine how far Aurelian had carried his intentions
before he was assassinated. Most of the moderns (except Dodwell,
Dissertat. Cyprian. vi. 64) have seized the occasion of gaining a
few extraordinary martyrs. * Note: Dr. Lardner has detailed, with
his usual impartiality, all that has come down to us relating to
the persecution of Aurelian, and concludes by saying, “Upon more
carefully examining the words of Eusebius, and observing the
accounts of other authors, learned men have generally, and, as I
think, very judiciously, determined, that Aurelian not only
intended, but did actually persecute: but his persecution was
short, he having died soon after the publication of his edicts.”
Heathen Test. c. xxxvi.—Basmage positively pronounces the same
opinion: Non intentatum modo, sed executum quoque brevissimo
tempore mandatum, nobis infixum est in aniasis. Basn. Ann. 275,
No. 2 and compare Pagi Ann. 272, Nos. 4, 12, 27—G.}

The story of Paul of Samosata, who filled the metropolitan see of
Antioch, while the East was in the hands of Odenathus and
Zenobia, may serve to illustrate the condition and character of
the times. The wealth of that prelate was a sufficient evidence
of his guilt, since it was neither derived from the inheritance
of his fathers, nor acquired by the arts of honest industry. But
Paul considered the service of the church as a very lucrative
profession.\textsuperscript{126} His ecclesiastical jurisdiction was venal and
rapacious; he extorted frequent contributions from the most
opulent of the faithful, and converted to his own use a
considerable part of the public revenue. By his pride and luxury,
the Christian religion was rendered odious in the eyes of the
Gentiles. His council chamber and his throne, the splendor with
which he appeared in public, the suppliant crowd who solicited
his attention, the multitude of letters and petitions to which he
dictated his answers, and the perpetual hurry of business in
which he was involved, were circumstances much better suited to
the state of a civil magistrate,\textsuperscript{127} than to the humility of a
primitive bishop. When he harangued his people from the pulpit,
Paul affected the figurative style and the theatrical gestures of
an Asiatic sophist, while the cathedral resounded with the
loudest and most extravagant acclamations in the praise of his
divine eloquence. Against those who resisted his power, or
refused to flatter his vanity, the prelate of Antioch was
arrogant, rigid, and inexorable; but he relaxed the discipline,
and lavished the treasures of the church on his dependent clergy,
who were permitted to imitate their master in the gratification
of every sensual appetite. For Paul indulged himself very freely
in the pleasures of the table, and he had received into the
episcopal palace two young and beautiful women as the constant
companions of his leisure moments.\textsuperscript{128}

\pagenote[126]{Paul was better pleased with the title of
\textit{Ducenarius}, than with that of bishop. The \textit{Ducenarius} was an
Imperial procurator, so called from his salary of two hundred
\textit{Sestertia}, or 1600\textit{l}. a year. (See Salmatius ad Hist. August.
p. 124.) Some critics suppose that the bishop of Antioch had
actually obtained such an office from Zenobia, while others
consider it only as a figurative expression of his pomp and
insolence.}

\pagenote[127]{Simony was not unknown in those times; and the
clergy some times bought what they intended to sell. It appears
that the bishopric of Carthage was purchased by a wealthy matron,
named Lucilla, for her servant Majorinus. The price was 400
\textit{Folles}. (Monument. Antiq. ad calcem Optati, p. 263.) Every
\textit{Follis} contained 125 pieces of silver, and the whole sum may be
computed at about 2400\textit{l}.}

\pagenote[128]{If we are desirous of extenuating the vices of
Paul, we must suspect the assembled bishops of the East of
publishing the most malicious calumnies in circular epistles
addressed to all the churches of the empire, (ap. Euseb. l. vii.
c. 30.)}

Notwithstanding these scandalous vices, if Paul of Samosata had
preserved the purity of the orthodox faith, his reign over the
capital of Syria would have ended only with his life; and had a
seasonable persecution intervened, an effort of courage might
perhaps have placed him in the rank of saints and martyrs.\textsuperscript{1281}

Some nice and subtle errors, which he imprudently adopted and
obstinately maintained, concerning the doctrine of the Trinity,
excited the zeal and indignation of the Eastern churches.\textsuperscript{129}

From Egypt to the Euxine Sea, the bishops were in arms and in
motion. Several councils were held, confutations were published,
excommunications were pronounced, ambiguous explanations were by
turns accepted and refused, treaties were concluded and violated,
and at length Paul of Samosata was degraded from his episcopal
character, by the sentence of seventy or eighty bishops, who
assembled for that purpose at Antioch, and who, without
consulting the rights of the clergy or people, appointed a
successor by their own authority. The manifest irregularity of
this proceeding increased the numbers of the discontented
faction; and as Paul, who was no stranger to the arts of courts,
had insinuated himself into the favor of Zenobia, he maintained
above four years the possession of the episcopal house and
office.\textsuperscript{1291} The victory of Aurelian changed the face of the
East, and the two contending parties, who applied to each other
the epithets of schism and heresy, were either commanded or
permitted to plead their cause before the tribunal of the
conqueror. This public and very singular trial affords a
convincing proof that the existence, the property, the
privileges, and the internal policy of the Christians, were
acknowledged, if not by the laws, at least by the magistrates, of
the empire. As a Pagan and as a soldier, it could scarcely be
expected that Aurelian should enter into the discussion, whether
the sentiments of Paul or those of his adversaries were most
agreeable to the true standard of the orthodox faith. His
determination, however, was founded on the general principles of
equity and reason. He considered the bishops of Italy as the most
impartial and respectable judges among the Christians, and as
soon as he was informed that they had unanimously approved the
sentence of the council, he acquiesced in their opinion, and
immediately gave orders that Paul should be compelled to
relinquish the temporal possessions belonging to an office, of
which, in the judgment of his brethren, he had been regularly
deprived. But while we applaud the justice, we should not
overlook the policy, of Aurelian, who was desirous of restoring
and cementing the dependence of the provinces on the capital, by
every means which could bind the interest or prejudices of any
part of his subjects.\textsuperscript{130}

\pagenote[1281]{It appears, nevertheless, that the vices and
immoralities of Paul of Samosata had much weight in the sentence
pronounced against him by the bishops. The object of the letter,
addressed by the synod to the bishops of Rome and Alexandria, was
to inform them of the change in the faith of Paul, the
altercations and discussions to which it had given rise, as well
as of his morals and the whole of his conduct. Euseb. Hist. Eccl.
l. vii c. xxx—G.}

\pagenote[129]{His heresy (like those of Noetus and Sabellius, in
the same century) tended to confound the mysterious distinction
of the divine persons. See Mosheim, p. 702, \&c.}

\pagenote[1291]{“Her favorite, (Zenobia’s,) Paul of Samosata,
seems to have entertained some views of attempting a union
between Judaism and Christianity; both parties rejected the
unnatural alliance.” Hist. of Jews, iii. 175, and Jost.
Geschichte der Israeliter, iv. 167. The protection of the severe
Zenobia is the only circumstance which may raise a doubt of the
notorious immorality of Paul.—M.}

\pagenote[130]{Euseb. Hist. Ecclesiast. l. vii. c. 30. We are
entirely indebted to him for the curious story of Paul of
Samosata.}

Amidst the frequent revolutions of the empire, the Christians
still flourished in peace and prosperity; and notwithstanding a
celebrated æra of martyrs has been deduced from the accession of
Diocletian,\textsuperscript{131} the new system of policy, introduced and
maintained by the wisdom of that prince, continued, during more
than eighteen years, to breathe the mildest and most liberal
spirit of religious toleration. The mind of Diocletian himself
was less adapted indeed to speculative inquiries, than to the
active labors of war and government. His prudence rendered him
averse to any great innovation, and though his temper was not
very susceptible of zeal or enthusiasm, he always maintained an
habitual regard for the ancient deities of the empire. But the
leisure of the two empresses, of his wife Prisca, and of Valeria,
his daughter, permitted them to listen with more attention and
respect to the truths of Christianity, which in every age has
acknowledged its important obligations to female devotion.\textsuperscript{132}
The principal eunuchs, Lucian\textsuperscript{133} and Dorotheus, Gorgonius and
Andrew, who attended the person, possessed the favor, and
governed the household of Diocletian, protected by their powerful
influence the faith which they had embraced. Their example was
imitated by many of the most considerable officers of the palace,
who, in their respective stations, had the care of the Imperial
ornaments, of the robes, of the furniture, of the jewels, and
even of the private treasury; and, though it might sometimes be
incumbent on them to accompany the emperor when he sacrificed in
the temple,\textsuperscript{134} they enjoyed, with their wives, their children,
and their slaves, the free exercise of the Christian religion.
Diocletian and his colleagues frequently conferred the most
important offices on those persons who avowed their abhorrence
for the worship of the gods, but who had displayed abilities
proper for the service of the state. The bishops held an
honorable rank in their respective provinces, and were treated
with distinction and respect, not only by the people, but by the
magistrates themselves. Almost in every city, the ancient
churches were found insufficient to contain the increasing
multitude of proselytes; and in their place more stately and
capacious edifices were erected for the public worship of the
faithful. The corruption of manners and principles, so forcibly
lamented by Eusebius,\textsuperscript{135} may be considered, not only as a
consequence, but as a proof, of the liberty which the Christians
enjoyed and abused under the reign of Diocletian. Prosperity had
relaxed the nerves of discipline. Fraud, envy, and malice
prevailed in every congregation. The presbyters aspired to the
episcopal office, which every day became an object more worthy of
their ambition. The bishops, who contended with each other for
ecclesiastical preëminence, appeared by their conduct to claim a
secular and tyrannical power in the church; and the lively faith
which still distinguished the Christians from the Gentiles, was
shown much less in their lives, than in their controversial
writings.

\pagenote[131]{The Æra of Martyrs, which is still in use among
the Copts and the Abyssinians, must be reckoned from the 29th of
August, A. D. 284; as the beginning of the Egyptian year was
nineteen days earlier than the real accession of Diocletian. See
Dissertation Preliminaire a l’Art de verifier les Dates. * Note:
On the æra of martyrs see the very curious dissertations of Mons
Letronne on some recently discovered inscriptions in Egypt and
Nubis, p. 102, \&c.—M.}

\pagenote[132]{The expression of Lactantius, (de M. P. c. 15,)
“sacrificio pollui coegit,” implies their antecedent conversion
to the faith, but does not seem to justify the assertion of
Mosheim, (p. 912,) that they had been privately baptized.}

\pagenote[133]{M. de Tillemont (Mémoires Ecclésiastiques, tom. v.
part i. p. 11, 12) has quoted from the Spicilegium of Dom Luc
d’Archeri a very curious instruction which Bishop Theonas
composed for the use of Lucian.}

\pagenote[134]{Lactantius, de M. P. c. 10.}

\pagenote[135]{Eusebius, Hist. Ecclesiast. l. viii. c. 1. The
reader who consults the original will not accuse me of
heightening the picture. Eusebius was about sixteen years of age
at the accession of the emperor Diocletian.}

Notwithstanding this seeming security, an attentive observer
might discern some symptoms that threatened the church with a
more violent persecution than any which she had yet endured. The
zeal and rapid progress of the Christians awakened the
Polytheists from their supine indifference in the cause of those
deities, whom custom and education had taught them to revere. The
mutual provocations of a religious war, which had already
continued above two hundred years, exasperated the animosity of
the contending parties. The Pagans were incensed at the rashness
of a recent and obscure sect, which presumed to accuse their
countrymen of error, and to devote their ancestors to eternal
misery. The habits of justifying the popular mythology against
the invectives of an implacable enemy, produced in their minds
some sentiments of faith and reverence for a system which they
had been accustomed to consider with the most careless levity.
The supernatural powers assumed by the church inspired at the
same time terror and emulation. The followers of the established
religion intrenched themselves behind a similar fortification of
prodigies; invented new modes of sacrifice, of expiation, and of
initiation;\textsuperscript{136} attempted to revive the credit of their expiring
oracles;\textsuperscript{137} and listened with eager credulity to every impostor,
who flattered their prejudices by a tale of wonders.\textsuperscript{138} Both
parties seemed to acknowledge the truth of those miracles which
were claimed by their adversaries; and while they were contented
with ascribing them to the arts of magic, and to the power of
dæmons, they mutually concurred in restoring and establishing the
reign of superstition.\textsuperscript{139} Philosophy, her most dangerous enemy,
was now converted into her most useful ally. The groves of the
academy, the gardens of Epicurus, and even the portico of the
Stoics, were almost deserted, as so many different schools of
scepticism or impiety;\textsuperscript{140} and many among the Romans were
desirous that the writings of Cicero should be condemned and
suppressed by the authority of the senate.\textsuperscript{141} The prevailing
sect of the new Platonicians judged it prudent to connect
themselves with the priests, whom perhaps they despised, against
the Christians, whom they had reason to fear. These fashionable
Philosophers prosecuted the design of extracting allegorical
wisdom from the fictions of the Greek poets; instituted
mysterious rites of devotion for the use of their chosen
disciples; recommended the worship of the ancient gods as the
emblems or ministers of the Supreme Deity, and composed against
the faith of the gospel many elaborate treatises,\textsuperscript{142} which have
since been committed to the flames by the prudence of orthodox
emperors.\textsuperscript{143}

\pagenote[136]{We might quote, among a great number of instances,
the mysterious worship of Mythras, and the Taurobolia; the latter
of which became fashionable in the time of the Antonines, (see a
Dissertation of M. de Boze, in the Mémoires de l’Académie des
Inscriptions, tom. ii. p. 443.) The romance of Apuleius is as
full of devotion as of satire. * Note: On the extraordinary
progress of the Mahriac rites, in the West, see De Guigniaud’s
translation of Creuzer, vol. i. p. 365, and Note 9, tom. i. part
2, p. 738, \&c.—M.}

\pagenote[137]{The impostor Alexander very strongly recommended
the oracle of Trophonius at Mallos, and those of Apollo at Claros
and Miletus, (Lucian, tom. ii. p. 236, edit. Reitz.) The last of
these, whose singular history would furnish a very curious
episode, was consulted by Diocletian before he published his
edicts of persecution, (Lactantius, de M. P. c. 11.)}

\pagenote[138]{Besides the ancient stories of Pythagoras and
Aristeas, the cures performed at the shrine of Æsculapius, and
the fables related of Apollonius of Tyana, were frequently
opposed to the miracles of Christ; though I agree with Dr.
Lardner, (see Testimonies, vol. iii. p. 253, 352,) that when
Philostratus composed the life of Apollonius, he had no such
intention.}

\pagenote[139]{It is seriously to be lamented, that the Christian
fathers, by acknowledging the supernatural, or, as they deem it,
the infernal part of Paganism, destroy with their own hands the
great advantage which we might otherwise derive from the liberal
concessions of our adversaries.}

\pagenote[140]{Julian (p. 301, edit. Spanheim) expresses a pious
joy, that the providence of the gods had extinguished the impious
sects, and for the most part destroyed the books of the
Pyrrhonians and Epicuræans, which had been very numerous, since
Epicurus himself composed no less than 300 volumes. See Diogenes
Laertius, l. x. c. 26.}

\pagenote[141]{Cumque alios audiam mussitare indignanter, et
dicere opportere statui per Senatum, aboleantur ut hæc scripta,
quibus Christiana Religio comprobetur, et vetustatis opprimatur
auctoritas. Arnobius adversus Gentes, l. iii. p. 103, 104. He
adds very properly, Erroris convincite Ciceronem... nam
intercipere scripta, et publicatam velle submergere lectionem,
non est Deum defendere sed veritatis testificationem timere.}

\pagenote[142]{Lactantius (Divin. Institut. l. v. c. 2, 3) gives
a very clear and spirited account of two of these philosophic
adversaries of the faith. The large treatise of Porphyry against
the Christians consisted of thirty books, and was composed in
Sicily about the year 270.}

\pagenote[143]{See Socrates, Hist. Ecclesiast. l. i. c. 9, and
Codex Justinian. l. i. i. l. s.}

\section{Part \thesection.}

Although the policy of Diocletian and the humanity of Constantius
inclined them to preserve inviolate the maxims of toleration, it
was soon discovered that their two associates, Maximian and
Galerius, entertained the most implacable aversion for the name
and religion of the Christians. The minds of those princes had
never been enlightened by science; education had never softened
their temper. They owed their greatness to their swords, and in
their most elevated fortune they still retained their
superstitious prejudices of soldiers and peasants. In the general
administration of the provinces they obeyed the laws which their
benefactor had established; but they frequently found occasions
of exercising within their camp and palaces a secret persecution,\textsuperscript{144}
for which the imprudent zeal of the Christians sometimes
offered the most specious pretences. A sentence of death was
executed upon Maximilianus, an African youth, who had been
produced by his own father\textsuperscript{1441} before the magistrate as a
sufficient and legal recruit, but who obstinately persisted in
declaring, that his conscience would not permit him to embrace
the profession of a soldier.\textsuperscript{145} It could scarcely be expected
that any government should suffer the action of Marcellus the
Centurion to pass with impunity. On the day of a public festival,
that officer threw away his belt, his arms, and the ensigns of
his office, and exclaimed with a loud voice, that he would obey
none but Jesus Christ the eternal King, and that he renounced
forever the use of carnal weapons, and the service of an
idolatrous master. The soldiers, as soon as they recovered from
their astonishment, secured the person of Marcellus. He was
examined in the city of Tingi by the president of that part of
Mauritania; and as he was convicted by his own confession, he was
condemned and beheaded for the crime of desertion.\textsuperscript{146} Examples
of such a nature savor much less of religious persecution than of
martial or even civil law; but they served to alienate the mind
of the emperors, to justify the severity of Galerius, who
dismissed a great number of Christian officers from their
employments; and to authorize the opinion, that a sect of
enthusiastics, which avowed principles so repugnant to the public
safety, must either remain useless, or would soon become
dangerous, subjects of the empire.

\pagenote[144]{Eusebius, l. viii. c. 4, c. 17. He limits the
number of military martyrs, by a remarkable expression, of which
neither his Latin nor French translator have rendered the energy.
Notwithstanding the authority of Eusebius, and the silence of
Lactantius, Ambrose, Sulpicius, Orosius, \&c., it has been long
believed, that the Thebæan legion, consisting of 6000 Christians,
suffered martyrdom by the order of Maximian, in the valley of the
Pennine Alps. The story was first published about the middle of
the 5th century, by Eucherius, bishop of Lyons, who received it
from certain persons, who received it from Isaac, bishop of
Geneva, who is said to have received it from Theodore, bishop of
Octodurum. The abbey of St. Maurice still subsists, a rich
monument of the credulity of Sigismund, king of Burgundy. See an
excellent Dissertation in xxxvith volume of the Bibliothèque
Raisonnée, p. 427-454.}

\pagenote[1441]{M. Guizot criticizes Gibbon’s account of this
incident. He supposes that Maximilian was not “produced by his
father as a recruit,” but was obliged to appear by the law, which
compelled the sons of soldiers to serve at 21 years old. Was not
this a law of Constantine? Neither does this circumstance appear
in the acts. His father had clearly expected him to serve, as he
had bought him a new dress for the occasion; yet he refused to
force the conscience of his son. and when Maximilian was
condemned to death, the father returned home in joy, blessing God
for having bestowed upon him such a son.—M.}

\pagenote[145]{See the Acta Sincera, p. 299. The accounts of his
martyrdom and that of Marcellus, bear every mark of truth and
authenticity.}

\pagenote[146]{Acta Sincera, p. 302. * Note: M. Guizot here
justly observes, that it was the necessity of sacrificing to the
gods, which induced Marcellus to act in this manner.—M.}

After the success of the Persian war had raised the hopes and the
reputation of Galerius, he passed a winter with Diocletian in the
palace of Nicomedia; and the fate of Christianity became the
object of their secret consultations.\textsuperscript{147} The experienced emperor
was still inclined to pursue measures of lenity; and though he
readily consented to exclude the Christians from holding any
employments in the household or the army, he urged in the
strongest terms the danger as well as cruelty of shedding the
blood of those deluded fanatics. Galerius at length extorted\textsuperscript{1471}
from him the permission of summoning a council, composed of a few
persons the most distinguished in the civil and military
departments of the state.

The important question was agitated in their presence, and those
ambitious courtiers easily discerned, that it was incumbent on
them to second, by their eloquence, the importunate violence of
the Cæsar. It may be presumed, that they insisted on every topic
which might interest the pride, the piety, or the fears, of their
sovereign in the destruction of Christianity. Perhaps they
represented, that the glorious work of the deliverance of the
empire was left imperfect, as long as an independent people was
permitted to subsist and multiply in the heart of the provinces.
The Christians, (it might specially be alleged,) renouncing the
gods and the institutions of Rome, had constituted a distinct
republic, which might yet be suppressed before it had acquired
any military force; but which was already governed by its own
laws and magistrates, was possessed of a public treasure, and was
intimately connected in all its parts by the frequent assemblies
of the bishops, to whose decrees their numerous and opulent
congregations yielded an implicit obedience. Arguments like these
may seem to have determined the reluctant mind of Diocletian to
embrace a new system of persecution; but though we may suspect,
it is not in our power to relate, the secret intrigues of the
palace, the private views and resentments, the jealousy of women
or eunuchs, and all those trifling but decisive causes which so
often influence the fate of empires, and the councils of the
wisest monarchs.\textsuperscript{148}

\pagenote[147]{De M. P. c. 11. Lactantius (or whoever was the
author of this little treatise) was, at that time, an inhabitant
of Nicomedia; but it seems difficult to conceive how he could
acquire so accurate a knowledge of what passed in the Imperial
cabinet. Note: * Lactantius, who was subsequently chosen by
Constantine to educate Crispus, might easily have learned these
details from Constantine himself, already of sufficient age to
interest himself in the affairs of the government, and in a
position to obtain the best information.—G. This assumes the
doubtful point of the authorship of the Treatise.—M.}

\pagenote[1471]{This permission was not extorted from Diocletian;
he took the step of his own accord. Lactantius says, in truth,
Nec tamen deflectere potuit (Diocletianus) præcipitis hominis
insaniam; placuit ergo amicorum sententiam experiri. (De Mort.
Pers. c. 11.) But this measure was in accordance with the
artificial character of Diocletian, who wished to have the
appearance of doing good by his own impulse and evil by the
impulse of others. Nam erat hujus malitiæ, cum bonum quid facere
decrevisse sine consilio faciebat, ut ipse laudaretur. Cum autem
malum. quoniam id reprehendendum sciebat, in consilium multos
advocabat, ut alioram culpæ adscriberetur quicquid ipse
deliquerat. Lact. ib. Eutropius says likewise, Miratus callide
fuit, sagax præterea et admodum subtilis ingenio, et qui
severitatem suam aliena invidia vellet explere. Eutrop. ix. c.
26.—G.——The manner in which the coarse and unfriendly pencil of
the author of the Treatise de Mort. Pers. has drawn the character
of Diocletian, seems inconsistent with this profound subtilty.
Many readers will perhaps agree with Gibbon.—M.}

\pagenote[148]{The only circumstance which we can discover, is
the devotion and jealousy of the mother of Galerius. She is
described by Lactantius, as Deorum montium cultrix; mulier
admodum superstitiosa. She had a great influence over her son,
and was offended by the disregard of some of her Christian
servants. * Note: This disregard consisted in the Christians
fasting and praying instead of participating in the banquets and
sacrifices which she celebrated with the Pagans. Dapibus
sacrificabat pœne quotidie ac vicariis suis epulis exhibebat.
Christiani abstinebant, et illa cum gentibus epulante, jejuniis
hi et oratiomibus insisteban; hine concepit odium Lact de Hist.
Pers. c. 11.—G.}

The pleasure of the emperors was at length signified to the
Christians, who, during the course of this melancholy winter, had
expected, with anxiety, the result of so many secret
consultations. The twenty-third of February, which coincided with
the Roman festival of the Terminalia,\textsuperscript{149} was appointed (whether
from accident or design) to set bounds to the progress of
Christianity. At the earliest dawn of day, the Prætorian præfect,\textsuperscript{150}
accompanied by several generals, tribunes, and officers of
the revenue, repaired to the principal church of Nicomedia, which
was situated on an eminence in the most populous and beautiful
part of the city. The doors were instantly broke open; they
rushed into the sanctuary; and as they searched in vain for some
visible object of worship, they were obliged to content
themselves with committing to the flames the volumes of the holy
Scripture. The ministers of Diocletian were followed by a
numerous body of guards and pioneers, who marched in order of
battle, and were provided with all the instruments used in the
destruction of fortified cities. By their incessant labor, a
sacred edifice, which towered above the Imperial palace, and had
long excited the indignation and envy of the Gentiles, was in a
few hours levelled with the ground.\textsuperscript{151}

\pagenote[149]{The worship and festival of the god Terminus are
elegantly illustrated by M. de Boze, Mém. de l’Académie des
Inscriptions, tom. i. p. 50.}

\pagenote[150]{In our only MS. of Lactantius, we read
\textit{profectus;} but reason, and the authority of all the critics,
allow us, instead of that word, which destroys the sense of the
passage, to substitute \textit{prœfectus}.}

\pagenote[151]{Lactantius, de M. P. c. 12, gives a very lively
picture of the destruction of the church.}

The next day the general edict of persecution was published;\textsuperscript{152}
and though Diocletian, still averse to the effusion of blood, had
moderated the fury of Galerius, who proposed, that every one
refusing to offer sacrifice should immediately be burnt alive,
the penalties inflicted on the obstinacy of the Christians might
be deemed sufficiently rigorous and effectual. It was enacted,
that their churches, in all the provinces of the empire, should
be demolished to their foundations; and the punishment of death
was denounced against all who should presume to hold any secret
assemblies for the purpose of religious worship. The
philosophers, who now assumed the unworthy office of directing
the blind zeal of persecution, had diligently studied the nature
and genius of the Christian religion; and as they were not
ignorant that the speculative doctrines of the faith were
supposed to be contained in the writings of the prophets, of the
evangelists, and of the apostles, they most probably suggested
the order, that the bishops and presbyters should deliver all
their sacred books into the hands of the magistrates; who were
commanded, under the severest penalties, to burn them in a public
and solemn manner. By the same edict, the property of the church
was at once confiscated; and the several parts of which it might
consist were either sold to the highest bidder, united to the
Imperial domain, bestowed on the cities and corporations, or
granted to the solicitations of rapacious courtiers. After taking
such effectual measures to abolish the worship, and to dissolve
the government of the Christians, it was thought necessary to
subject to the most intolerable hardships the condition of those
perverse individuals who should still reject the religion of
nature, of Rome, and of their ancestors. Persons of a liberal
birth were declared incapable of holding any honors or
employments; slaves were forever deprived of the hopes of
freedom, and the whole body of the people were put out of the
protection of the law. The judges were authorized to hear and to
determine every action that was brought against a Christian. But
the Christians were not permitted to complain of any injury which
they themselves had suffered; and thus those unfortunate
sectaries were exposed to the severity, while they were excluded
from the benefits, of public justice. This new species of
martyrdom, so painful and lingering, so obscure and ignominious,
was, perhaps, the most proper to weary the constancy of the
faithful: nor can it be doubted that the passions and interest of
mankind were disposed on this occasion to second the designs of
the emperors. But the policy of a well-ordered government must
sometimes have interposed in behalf of the oppressed Christians;\textsuperscript{1521}
nor was it possible for the Roman princes entirely to remove
the apprehension of punishment, or to connive at every act of
fraud and violence, without exposing their own authority and the
rest of their subjects to the most alarming dangers.\textsuperscript{153}

\pagenote[152]{Mosheim, (p. 922—926,) from man scattered passages
of Lactantius and Eusebius, has collected a very just and
accurate notion of this edict though he sometimes deviates into
conjecture and refinement.}

\pagenote[1521]{This wants proof. The edict of Diocletian was
executed in all its right during the rest of his reign. Euseb.
Hist. Eccl. l viii. c. 13.—G.}

\pagenote[153]{Many ages afterwards, Edward J. practised, with
great success, the same mode of persecution against the clergy of
England. See Hume’s History of England, vol. ii. p. 300, last 4to
edition.}

This edict was scarcely exhibited to the public view, in the most
conspicuous place of Nicomedia, before it was torn down by the
hands of a Christian, who expressed at the same time, by the
bitterest invectives, his contempt as well as abhorrence for such
impious and tyrannical governors. His offence, according to the
mildest laws, amounted to treason, and deserved death. And if it
be true that he was a person of rank and education, those
circumstances could serve only to aggravate his guilt. He was
burnt, or rather roasted, by a slow fire; and his executioners,
zealous to revenge the personal insult which had been offered to
the emperors, exhausted every refinement of cruelty, without
being able to subdue his patience, or to alter the steady and
insulting smile which in his dying agonies he still preserved in
his countenance. The Christians, though they confessed that his
conduct had not been strictly conformable to the laws of
prudence, admired the divine fervor of his zeal; and the
excessive commendations which they lavished on the memory of
their hero and martyr, contributed to fix a deep impression of
terror and hatred in the mind of Diocletian.\textsuperscript{154}

\pagenote[154]{Lactantius only calls him quidam, et si non recte,
magno tamer animo, \&c., c. 12. Eusebius (l. viii. c. 5) adorns
him with secular honora Neither have condescended to mention his
name; but the Greeks celebrate his memory under that of John. See
Tillemont, Memones Ecclésiastiques, tom. v. part ii. p. 320.}

His fears were soon alarmed by the view of a danger from which he
very narrowly escaped. Within fifteen days the palace of
Nicomedia, and even the bed-chamber of Diocletian, were twice in
flames; and though both times they were extinguished without any
material damage, the singular repetition of the fire was justly
considered as an evident proof that it had not been the effect of
chance or negligence. The suspicion naturally fell on the
Christians; and it was suggested, with some degree of
probability, that those desperate fanatics, provoked by their
present sufferings, and apprehensive of impending calamities, had
entered into a conspiracy with their faithful brethren, the
eunuchs of the palace, against the lives of two emperors, whom
they detested as the irreconcilable enemies of the church of God.

Jealousy and resentment prevailed in every breast, but especially
in that of Diocletian. A great number of persons, distinguished
either by the offices which they had filled, or by the favor
which they had enjoyed, were thrown into prison. Every mode of
torture was put in practice, and the court, as well as city, was
polluted with many bloody executions.\textsuperscript{155} But as it was found
impossible to extort any discovery of this mysterious
transaction, it seems incumbent on us either to presume the
innocence, or to admire the resolution, of the sufferers. A few
days afterwards Galerius hastily withdrew himself from Nicomedia,
declaring, that if he delayed his departure from that devoted
palace, he should fall a sacrifice to the rage of the Christians.

The ecclesiastical historians, from whom alone we derive a
partial and imperfect knowledge of this persecution, are at a
loss how to account for the fears and dangers of the emperors.
Two of these writers, a prince and a rhetorician, were
eye-witnesses of the fire of Nicomedia. The one ascribes it to
lightning, and the divine wrath; the other affirms, that it was
kindled by the malice of Galerius himself.\textsuperscript{156}

\pagenote[155]{Lactantius de M. P. c. 13, 14. Potentissimi
quondam Eunuchi necati, per quos Palatium et ipse constabat.
Eusebius (l. viii. c. 6) mentions the cruel executions of the
eunuchs, Gorgonius and Dorotheus, and of Anthimius, bishop of
Nicomedia; and both those writers describe, in a vague but
tragical manner, the horrid scenes which were acted even in the
Imperial presence.}

\pagenote[156]{See Lactantius, Eusebius, and Constantine, ad
Cœtum Sanctorum, c. xxv. Eusebius confesses his ignorance of the
cause of this fire. Note: As the history of these times affords
us no example of any attempts made by the Christians against
their persecutors, we have no reason, not the slightest
probability, to attribute to them the fire in the palace; and the
authority of Constantine and Lactantius remains to explain it. M.
de Tillemont has shown how they can be reconciled. Hist. des
Empereurs, Vie de Diocletian, xix.—G. Had it been done by a
Christian, it would probably have been a fanatic, who would have
avowed and gloried in it. Tillemont’s supposition that the fire
was first caused by lightning, and fed and increased by the
malice of Galerius, seems singularly improbable.—M.}

As the edict against the Christians was designed for a general
law of the whole empire, and as Diocletian and Galerius, though
they might not wait for the consent, were assured of the
concurrence, of the Western princes, it would appear more
consonant to our ideas of policy, that the governors of all the
provinces should have received secret instructions to publish, on
one and the same day, this declaration of war within their
respective departments. It was at least to be expected, that the
convenience of the public highways and established posts would
have enabled the emperors to transmit their orders with the
utmost despatch from the palace of Nicomedia to the extremities
of the Roman world; and that they would not have suffered fifty
days to elapse, before the edict was published in Syria, and near
four months before it was signified to the cities of Africa.\textsuperscript{157}

This delay may perhaps be imputed to the cautious temper of
Diocletian, who had yielded a reluctant consent to the measures
of persecution, and who was desirous of trying the experiment
under his more immediate eye, before he gave way to the disorders
and discontent which it must inevitably occasion in the distant
provinces. At first, indeed, the magistrates were restrained from
the effusion of blood; but the use of every other severity was
permitted, and even recommended to their zeal; nor could the
Christians, though they cheerfully resigned the ornaments of
their churches, resolve to interrupt their religious assemblies,
or to deliver their sacred books to the flames. The pious
obstinacy of Felix, an African bishop, appears to have
embarrassed the subordinate ministers of the government. The
curator of his city sent him in chains to the proconsul. The
proconsul transmitted him to the Prætorian præfect of Italy; and
Felix, who disdained even to give an evasive answer, was at
length beheaded at Venusia, in Lucania, a place on which the
birth of Horace has conferred fame.\textsuperscript{158} This precedent, and
perhaps some Imperial rescript, which was issued in consequence
of it, appeared to authorize the governors of provinces, in
punishing with death the refusal of the Christians to deliver up
their sacred books. There were undoubtedly many persons who
embraced this opportunity of obtaining the crown of martyrdom;
but there were likewise too many who purchased an ignominious
life, by discovering and betraying the holy Scripture into the
hands of infidels. A great number even of bishops and presbyters
acquired, by this criminal compliance, the opprobrious epithet of
\textit{Traditors;} and their offence was productive of much present
scandal and of much future discord in the African church.\textsuperscript{159}

\pagenote[157]{Tillemont, Mémoires Ecclesiast. tom. v. part i. p.
43.}

\pagenote[158]{See the Acta Sincera of Ruinart, p. 353; those of
Felix of Thibara, or Tibiur, appear much less corrupted than in
the other editions, which afford a lively specimen of legendary
license.}

\pagenote[159]{See the first book of Optatus of Milevis against
the Donatiste, Paris, 1700, edit. Dupin. He lived under the reign
of Valens.}

The copies as well as the versions of Scripture, were already so
multiplied in the empire, that the most severe inquisition could
no longer be attended with any fatal consequences; and even the
sacrifice of those volumes, which, in every congregation, were
preserved for public use, required the consent of some
treacherous and unworthy Christians. But the ruin of the churches
was easily effected by the authority of the government, and by
the labor of the Pagans. In some provinces, however, the
magistrates contented themselves with shutting up the places of
religious worship. In others, they more literally complied with
the terms of the edict; and after taking away the doors, the
benches, and the pulpit, which they burnt as it were in a funeral
pile, they completely demolished the remainder of the edifice.\textsuperscript{160}
It is perhaps to this melancholy occasion that we should
apply a very remarkable story, which is related with so many
circumstances of variety and improbability, that it serves rather
to excite than to satisfy our curiosity. In a small town in
Phrygia, of whose name as well as situation we are left ignorant,
it should seem that the magistrates and the body of the people
had embraced the Christian faith; and as some resistance might be
apprehended to the execution of the edict, the governor of the
province was supported by a numerous detachment of legionaries.
On their approach the citizens threw themselves into the church,
with the resolution either of defending by arms that sacred
edifice, or of perishing in its ruins. They indignantly rejected
the notice and permission which was given them to retire, till
the soldiers, provoked by their obstinate refusal, set fire to
the building on all sides, and consumed, by this extraordinary
kind of martyrdom, a great number of Phrygians, with their wives
and children.\textsuperscript{161}

\pagenote[160]{The ancient monuments, published at the end of
Optatus, p. 261, \&c. describe, in a very circumstantial manner,
the proceedings of the governors in the destruction of churches.
They made a minute inventory of the plate, \&c., which they found
in them. That of the church of Cirta, in Numidia, is still
extant. It consisted of two chalices of gold, and six of silver;
six urns, one kettle, seven lamps, all likewise of silver;
besides a large quantity of brass utensils, and wearing apparel.}

\pagenote[161]{Lactantius (Institut. Divin. v. 11) confines the
calamity to the \textit{conventiculum}, with its congregation. Eusebius
(viii. 11) extends it to a whole city, and introduces something
very like a regular siege. His ancient Latin translator, Rufinus,
adds the important circumstance of the permission given to the
inhabitants of retiring from thence. As Phrygia reached to the
confines of Isauria, it is possible that the restless temper of
those independent barbarians may have contributed to this
misfortune. Note: Universum populum. Lact. Inst. Div. v. 11.—G.}

Some slight disturbances, though they were suppressed almost as
soon as excited, in Syria and the frontiers of Armenia, afforded
the enemies of the church a very plausible occasion to insinuate,
that those troubles had been secretly fomented by the intrigues
of the bishops, who had already forgotten their ostentatious
professions of passive and unlimited obedience.\textsuperscript{162}

The resentment, or the fears, of Diocletian, at length
transported him beyond the bounds of moderation, which he had
hitherto preserved, and he declared, in a series of cruel edicts,\textsuperscript{1621}
his intention of abolishing the Christian name. By the first
of these edicts, the governors of the provinces were directed to
apprehend all persons of the ecclesiastical order; and the
prisons, destined for the vilest criminals, were soon filled with
a multitude of bishops, presbyters, deacons, readers, and
exorcists. By a second edict, the magistrates were commanded to
employ every method of severity, which might reclaim them from
their odious superstition, and oblige them to return to the
established worship of the gods. This rigorous order was
extended, by a subsequent edict, to the whole body of Christians,
who were exposed to a violent and general persecution.\textsuperscript{163}

Instead of those salutary restraints, which had required the
direct and solemn testimony of an accuser, it became the duty as
well as the interest of the Imperial officers to discover, to
pursue, and to torment the most obnoxious among the faithful.
Heavy penalties were denounced against all who should presume to
save a prescribed sectary from the just indignation of the gods,
and of the emperors. Yet, notwithstanding the severity of this
law, the virtuous courage of many of the Pagans, in concealing
their friends or relations, affords an honorable proof, that the
rage of superstition had not extinguished in their minds the
sentiments of nature and humanity.\textsuperscript{164}

\pagenote[162]{Eusebius, l. viii. c. 6. M. de Valois (with some
probability) thinks that he has discovered the Syrian rebellion
in an oration of Libanius; and that it was a rash attempt of the
tribune Eugenius, who with only five hundred men seized Antioch,
and might perhaps allure the Christians by the promise of
religious toleration. From Eusebius, (l. ix. c. 8,) as well as
from Moses of Chorene, (Hist. Armen. l. ii. 77, \&c.,) it may be
inferred, that Christianity was already introduced into Armenia.}

\pagenote[1621]{He had already passed them in his first edict. It
does not appear that resentment or fear had any share in the new
persecutions: perhaps they originated in superstition, and a
specious apparent respect for its ministers. The oracle of
Apollo, consulted by Diocletian, gave no answer; and said that
just men hindered it from speaking. Constantine, who assisted at
the ceremony, affirms, with an oath, that when questioned about
these men, the high priest named the Christians. “The Emperor
eagerly seized on this answer; and drew against the innocent a
sword, destined only to punish the guilty: he instantly issued
edicts, written, if I may use the expression, with a poniard; and
ordered the judges to employ all their skill to invent new modes
of punishment. Euseb. Vit Constant. l. ii c 54.”—G.}

\pagenote[163]{See Mosheim, p. 938: the text of Eusebius very
plainly shows that the governors, whose powers were enlarged, not
restrained, by the new laws, could punish with death the most
obstinate Christians as an example to their brethren.}

\pagenote[164]{Athanasius, p. 833, ap. Tillemont, Mém.
Ecclesiast. tom v part i. 90.}

\section{Part \thesection.}

Diocletian had no sooner published his edicts against the
Christians, than, as if he had been desirous of committing to
other hands the work of persecution, he divested himself of the
Imperial purple. The character and situation of his colleagues
and successors sometimes urged them to enforce and sometimes
inclined them to suspend, the execution of these rigorous laws;
nor can we acquire a just and distinct idea of this important
period of ecclesiastical history, unless we separately consider
the state of Christianity, in the different parts of the empire,
during the space of ten years, which elapsed between the first
edicts of Diocletian and the final peace of the church.

The mild and humane temper of Constantius was averse to the
oppression of any part of his subjects. The principal offices of
his palace were exercised by Christians. He loved their persons,
esteemed their fidelity, and entertained not any dislike to their
religious principles. But as long as Constantius remained in the
subordinate station of Cæsar, it was not in his power openly to
reject the edicts of Diocletian, or to disobey the commands of
Maximian. His authority contributed, however, to alleviate the
sufferings which he pitied and abhorred. He consented with
reluctance to the ruin of the churches; but he ventured to
protect the Christians themselves from the fury of the populace,
and from the rigor of the laws. The provinces of Gaul (under
which we may probably include those of Britain) were indebted for
the singular tranquillity which they enjoyed, to the gentle
interposition of their sovereign.\textsuperscript{165} But Datianus, the president
or governor of Spain, actuated either by zeal or policy, chose
rather to execute the public edicts of the emperors, than to
understand the secret intentions of Constantius; and it can
scarcely be doubted, that his provincial administration was
stained with the blood of a few martyrs.\textsuperscript{166}

The elevation of Constantius to the supreme and independent
dignity of Augustus, gave a free scope to the exercise of his
virtues, and the shortness of his reign did not prevent him from
establishing a system of toleration, of which he left the precept
and the example to his son Constantine. His fortunate son, from
the first moment of his accession, declaring himself the
protector of the church, at length deserved the appellation of
the first emperor who publicly professed and established the
Christian religion. The motives of his conversion, as they may
variously be deduced from benevolence, from policy, from
conviction, or from remorse, and the progress of the revolution,
which, under his powerful influence and that of his sons,
rendered Christianity the reigning religion of the Roman empire,
will form a very interesting and important chapter in the present
volume of this history. At present it may be sufficient to
observe, that every victory of Constantine was productive of some
relief or benefit to the church.

\pagenote[165]{Eusebius, l. viii. c. 13. Lactantius de M. P. c.
15. Dodwell (Dissertat. Cyprian. xi. 75) represents them as
inconsistent with each other. But the former evidently speaks of
Constantius in the station of Cæsar, and the latter of the same
prince in the rank of Augustus.}

\pagenote[166]{Datianus is mentioned, in Gruter’s Inscriptions,
as having determined the limits between the territories of Pax
Julia, and those of Ebora, both cities in the southern part of
Lusitania. If we recollect the neighborhood of those places to
Cape St. Vincent, we may suspect that the celebrated deacon and
martyr of that name had been inaccurately assigned by Prudentius,
\&c., to Saragossa, or Valentia. See the pompous history of his
sufferings, in the Mémoires de Tillemont, tom. v. part ii. p.
58-85. Some critics are of opinion, that the department of
Constantius, as Cæsar, did not include Spain, which still
continued under the immediate jurisdiction of Maximian.}

The provinces of Italy and Africa experienced a short but violent
persecution. The rigorous edicts of Diocletian were strictly and
cheerfully executed by his associate Maximian, who had long hated
the Christians, and who delighted in acts of blood and violence.
In the autumn of the first year of the persecution, the two
emperors met at Rome to celebrate their triumph; several
oppressive laws appear to have issued from their secret
consultations, and the diligence of the magistrates was animated
by the presence of their sovereigns. After Diocletian had
divested himself of the purple, Italy and Africa were
administered under the name of Severus, and were exposed, without
defence, to the implacable resentment of his master Galerius.
Among the martyrs of Rome, Adauctus deserves the notice of
posterity. He was of a noble family in Italy, and had raised
himself, through the successive honors of the palace, to the
important office of treasurer of the private Jemesnes. Adauctus
is the more remarkable for being the only person of rank and
distinction who appears to have suffered death, during the whole
course of this general persecution.\textsuperscript{167}

\pagenote[167]{Eusebius, l. viii. c. 11. Gruter, Inscrip. p.
1171, No. 18. Rufinus has mistaken the office of Adauctus, as
well as the place of his martyrdom. * Note: M. Guizot suggests
the powerful cunuchs of the palace. Dorotheus, Gorgonius, and
Andrew, admitted by Gibbon himself to have been put to death, p.
66.}

The revolt of Maxentius immediately restored peace to the
churches of Italy and Africa; and the same tyrant who oppressed
every other class of his subjects, showed himself just, humane,
and even partial, towards the afflicted Christians. He depended
on their gratitude and affection, and very naturally presumed,
that the injuries which they had suffered, and the dangers which
they still apprehended from his most inveterate enemy, would
secure the fidelity of a party already considerable by their
numbers and opulence.\textsuperscript{168} Even the conduct of Maxentius towards
the bishops of Rome and Carthage may be considered as the proof
of his toleration, since it is probable that the most orthodox
princes would adopt the same measures with regard to their
established clergy. Marcellus, the former of these prelates, had
thrown the capital into confusion, by the severe penance which he
imposed on a great number of Christians, who, during the late
persecution, had renounced or dissembled their religion. The rage
of faction broke out in frequent and violent seditions; the blood
of the faithful was shed by each other’s hands, and the exile of
Marcellus, whose prudence seems to have been less eminent than
his zeal, was found to be the only measure capable of restoring
peace to the distracted church of Rome.\textsuperscript{169} The behavior of
Mensurius, bishop of Carthage, appears to have been still more
reprehensible. A deacon of that city had published a libel
against the emperor. The offender took refuge in the episcopal
palace; and though it was somewhat early to advance any claims of
ecclesiastical immunities, the bishop refused to deliver him up
to the officers of justice. For this treasonable resistance,
Mensurius was summoned to court, and instead of receiving a legal
sentence of death or banishment, he was permitted, after a short
examination, to return to his diocese.\textsuperscript{170} Such was the happy
condition of the Christian subjects of Maxentius, that whenever
they were desirous of procuring for their own use any bodies of
martyrs, they were obliged to purchase them from the most distant
provinces of the East. A story is related of Aglae, a Roman lady,
descended from a consular family, and possessed of so ample an
estate, that it required the management of seventy-three
stewards. Among these Boniface was the favorite of his mistress;
and as Aglae mixed love with devotion, it is reported that he was
admitted to share her bed. Her fortune enabled her to gratify the
pious desire of obtaining some sacred relics from the East. She
intrusted Boniface with a considerable sum of gold, and a large
quantity of aromatics; and her lover, attended by twelve horsemen
and three covered chariots, undertook a remote pilgrimage, as far
as Tarsus in Cilicia.\textsuperscript{171}

\pagenote[168]{Eusebius, l. viii. c. 14. But as Maxentius was
vanquished by Constantine, it suited the purpose of Lactantius to
place his death among those of the persecutors. * Note: M. Guizot
directly contradicts this statement of Gibbon, and appeals to
Eusebius. Maxentius, who assumed the power in Italy, pretended at
first to be a Christian, to gain the favor of the Roman people;
he ordered his ministers to cease to persecute the Christians,
affecting a hypocritical piety, in order to appear more mild than
his predecessors; but his actions soon proved that he was very
different from what they had at first hoped. The actions of
Maxentius were those of a cruel tyrant, but not those of a
persecutor: the Christians, like the rest of his subjects,
suffered from his vices, but they were not oppressed as a sect.
Christian females were exposed to his lusts, as well as to the
brutal violence of his colleague Maximian, but they were not
selected as Christians.—M.}

\pagenote[169]{The epitaph of Marcellus is to be found in Gruter,
Inscrip. p 1172, No. 3, and it contains all that we know of his
history. Marcellinus and Marcellus, whose names follow in the
list of popes, are supposed by many critics to be different
persons; but the learned Abbé de Longuerue was convinced that
they were one and the same.
\begin{verse}
Veridicus rector lapsis quia crimina flere\\
Prædixit miseris, fuit omnibus hostis amarus.\\
Hinc furor, hinc odium; sequitur discordia, lites,\\
Seditio, cædes; solvuntur fœdera pacis.\\
Crimen ob alterius, Christum qui in pace negavit\\
Finibus expulsus patriæ est feritate Tyranni.\\
Hæc breviter Damasus voluit comperta referre:\\
Marcelli populus meritum cognoscere posset.
\end{verse}
We may observe that Damasus was made Bishop of Rome, A. D. 366.}

\pagenote[170]{Optatus contr. Donatist. l. i. c. 17, 18. * Note:
The words of Optatus are, Profectus (Roman) causam dixit; jussus
con reverti Carthaginem; perhaps, in pleading his cause, he
exculpated himself, since he received an order to return to
Carthage.—G.}

\pagenote[171]{The Acts of the Passion of St. Boniface, which
abound in miracles and declamation, are published by Ruinart, (p.
283—291,) both in Greek and Latin, from the authority of very
ancient manuscripts. Note: We are ignorant whether Aglae and
Boniface were Christians at the time of their unlawful
connection. See Tillemont. Mem, Eccles. Note on the Persecution
of Domitian, tom. v. note 82. M. de Tillemont proves also that
the history is doubtful.—G. ——Sir D. Dalrymple (Lord Hailes)
calls the story of Aglae and Boniface as of equal authority with
our \textit{popular} histories of Whittington and Hickathrift. Christian
Antiquities, ii. 64.—M.}

The sanguinary temper of Galerius, the first and principal author
of the persecution, was formidable to those Christians whom their
misfortunes had placed within the limits of his dominions; and it
may fairly be presumed that many persons of a middle rank, who
were not confined by the chains either of wealth or of poverty,
very frequently deserted their native country, and sought a
refuge in the milder climate of the West.\textsuperscript{1711} As long as he
commanded only the armies and provinces of Illyricum, he could
with difficulty either find or make a considerable number of
martyrs, in a warlike country, which had entertained the
missionaries of the gospel with more coldness and reluctance than
any other part of the empire.\textsuperscript{172} But when Galerius had obtained
the supreme power, and the government of the East, he indulged in
their fullest extent his zeal and cruelty, not only in the
provinces of Thrace and Asia, which acknowledged his immediate
jurisdiction, but in those of Syria, Palestine, and Egypt, where
Maximin gratified his own inclination, by yielding a rigorous
obedience to the stern commands of his benefactor.\textsuperscript{173} The
frequent disappointments of his ambitious views, the experience
of six years of persecution, and the salutary reflections which a
lingering and painful distemper suggested to the mind of
Galerius, at length convinced him that the most violent efforts
of despotism are insufficient to extirpate a whole people, or to
subdue their religious prejudices. Desirous of repairing the
mischief that he had occasioned, he published in his own name,
and in those of Licinius and Constantine, a general edict, which,
after a pompous recital of the Imperial titles, proceeded in the
following manner:—

\pagenote[1711]{A little after this, Christianity was propagated
to the north of the Roman provinces, among the tribes of Germany:
a multitude of Christians, forced by the persecutions of the
Emperors to take refuge among the Barbarians, were received with
kindness. Euseb. de Vit. Constant. ii. 53. Semler Select. cap. H.
E. p. 115. The Goths owed their first knowledge of Christianity
to a young girl, a prisoner of war; she continued in the midst of
them her exercises of piety; she fasted, prayed, and praised God
day and night. When she was asked what good would come of so much
painful trouble she answered, “It is thus that Christ, the Son of
God, is to be honored.” Sozomen, ii. c. 6.—G.}

\pagenote[172]{During the four first centuries, there exist few
traces of either bishops or bishoprics in the western Illyricum.
It has been thought probable that the primate of Milan extended
his jurisdiction over Sirmium, the capital of that great
province. See the Geographia Sacra of Charles de St. Paul, p.
68-76, with the observations of Lucas Holstenius.}

\pagenote[173]{[The viiith book of Eusebius, as well as the
supplement concerning the martyrs of Palestine, principally
relate to the persecution of Galerius and Maximin. The general
lamentations with which Lactantius opens the vth book of his
Divine Institutions allude to their cruelty.] “Among the
important cares which have occupied our mind for the utility and
preservation of the empire, it was our intention to correct and
reestablish all things according to the ancient laws and public
discipline of the Romans. We were particularly desirous of
reclaiming into the way of reason and nature, the deluded
Christians who had renounced the religion and ceremonies
instituted by their fathers; and presumptuously despising the
practice of antiquity, had invented extravagant laws and
opinions, according to the dictates of their fancy, and had
collected a various society from the different provinces of our
empire. The edicts, which we have published to enforce the
worship of the gods, having exposed many of the Christians to
danger and distress, many having suffered death, and many more,
who still persist in their impious folly, being left destitute of
\textit{any} public exercise of religion, we are disposed to extend to
those unhappy men the effects of our wonted clemency. We permit
them therefore freely to profess their private opinions, and to
assemble in their conventicles without fear or molestation,
provided always that they preserve a due respect to the
established laws and government. By another rescript we shall
signify our intentions to the judges and magistrates; and we hope
that our indulgence will engage the Christians to offer up their
prayers to the Deity whom they adore, for our safety and
prosperity for their own, and for that of the republic.”\textsuperscript{174} It
is not usually in the language of edicts and manifestos that we
should search for the real character or the secret motives of
princes; but as these were the words of a dying emperor, his
situation, perhaps, may be admitted as a pledge of his sincerity.}

\pagenote[174]{Eusebius (l. viii. c. 17) has given us a Greek
version, and Lactantius (de M. P. c. 34) the Latin original, of
this memorable edict. Neither of these writers seems to recollect
how directly it contradicts whatever they have just affirmed of
the remorse and repentance of Galerius. Note: But Gibbon has
answered this by his just observation, that it is not in the
language of edicts and manifestos that we should search * * for
the secre motives of princes.—M.}

When Galerius subscribed this edict of toleration, he was well
assured that Licinius would readily comply with the inclinations
of his friend and benefactor, and that any measures in favor of
the Christians would obtain the approbation of Constantine. But
the emperor would not venture to insert in the preamble the name
of Maximin, whose consent was of the greatest importance, and who
succeeded a few days afterwards to the provinces of Asia. In the
first six months, however, of his new reign, Maximin affected to
adopt the prudent counsels of his predecessor; and though he
never condescended to secure the tranquillity of the church by a
public edict, Sabinus, his Prætorian præfect, addressed a
circular letter to all the governors and magistrates of the
provinces, expatiating on the Imperial clemency, acknowledging
the invincible obstinacy of the Christians, and directing the
officers of justice to cease their ineffectual prosecutions, and
to connive at the secret assemblies of those enthusiasts. In
consequence of these orders, great numbers of Christians were
released from prison, or delivered from the mines. The
confessors, singing hymns of triumph, returned into their own
countries; and those who had yielded to the violence of the
tempest, solicited with tears of repentance their readmission
into the bosom of the church.\textsuperscript{175}

\pagenote[175]{Eusebius, l. ix. c. 1. He inserts the epistle of
the præfect.}

But this treacherous calm was of short duration; nor could the
Christians of the East place any confidence in the character of
their sovereign. Cruelty and superstition were the ruling
passions of the soul of Maximin. The former suggested the means,
the latter pointed out the objects of persecution. The emperor
was devoted to the worship of the gods, to the study of magic,
and to the belief of oracles. The prophets or philosophers, whom
he revered as the favorites of Heaven, were frequently raised to
the government of provinces, and admitted into his most secret
councils. They easily convinced him that the Christians had been
indebted for their victories to their regular discipline, and
that the weakness of polytheism had principally flowed from a
want of union and subordination among the ministers of religion.
A system of government was therefore instituted, which was
evidently copied from the policy of the church. In all the great
cities of the empire, the temples were repaired and beautified by
the order of Maximin, and the officiating priests of the various
deities were subjected to the authority of a superior pontiff
destined to oppose the bishop, and to promote the cause of
paganism. These pontiffs acknowledged, in their turn, the supreme
jurisdiction of the metropolitans or high priests of the
province, who acted as the immediate vicegerents of the emperor
himself. A white robe was the ensign of their dignity; and these
new prelates were carefully selected from the most noble and
opulent families. By the influence of the magistrates, and of the
sacerdotal order, a great number of dutiful addresses were
obtained, particularly from the cities of Nicomedia, Antioch, and
Tyre, which artfully represented the well-known intentions of the
court as the general sense of the people; solicited the emperor
to consult the laws of justice rather than the dictates of his
clemency; expressed their abhorrence of the Christians, and
humbly prayed that those impious sectaries might at least be
excluded from the limits of their respective territories. The
answer of Maximin to the address which he obtained from the
citizens of Tyre is still extant. He praises their zeal and
devotion in terms of the highest satisfaction, descants on the
obstinate impiety of the Christians, and betrays, by the
readiness with which he consents to their banishment, that he
considered himself as receiving, rather than as conferring, an
obligation. The priests as well as the magistrates were empowered
to enforce the execution of his edicts, which were engraved on
tables of brass; and though it was recommended to them to avoid
the effusion of blood, the most cruel and ignominious punishments
were inflicted on the refractory Christians.\textsuperscript{176}

\pagenote[176]{See Eusebius, l. viii. c. 14, l. ix. c. 2—8.
Lactantius de M. P. c. 36. These writers agree in representing
the arts of Maximin; but the former relates the execution of
several martyrs, while the latter expressly affirms, occidi
servos Dei vetuit. * Note: It is easy to reconcile them; it is
sufficient to quote the entire text of Lactantius: Nam cum
clementiam specie tenus profiteretur, occidi servos Dei vetuit,
debilitari jussit. Itaque confessoribus effodiebantur oculi,
amputabantur manus, nares vel auriculæ desecabantur. Hæc ille
moliens Constantini litteris deterretur. Dissimulavit ergo, et
tamen, si quis inciderit. mari occulte mergebatur. This detail of
torments inflicted on the Christians easily reconciles Lactantius
and Eusebius. Those who died in consequence of their tortures,
those who were plunged into the sea, might well pass for martyrs.
The mutilation of the words of Lactantius has alone given rise to
the apparent contradiction.—G. ——Eusebius. ch. vi., relates the
public martyrdom of the aged bishop of Emesa, with two others,
who were thrown to the wild beasts, the beheading of Peter,
bishop of Alexandria, with several others, and the death of
Lucian, presbyter of Antioch, who was carried to Numidia, and put
to death in prison. The contradiction is direct and undeniable,
for although Eusebius may have misplaced the former martyrdoms,
it may be doubted whether the authority of Maximin extended to
Nicomedia till after the death of Galerius. The last edict of
toleration issued by Maximin and published by Eusebius himself,
Eccl. Hist. ix. 9. confirms the statement of Lactantius.—M.}

The Asiatic Christians had every thing to dread from the severity
of a bigoted monarch who prepared his measures of violence with
such deliberate policy. But a few months had scarcely elapsed
before the edicts published by the two Western emperors obliged
Maximin to suspend the prosecution of his designs: the civil war
which he so rashly undertook against Licinius employed all his
attention; and the defeat and death of Maximin soon delivered the
church from the last and most implacable of her enemies.\textsuperscript{177}

\pagenote[177]{A few days before his death, he published a very
ample edict of toleration, in which he imputes all the severities
which the Christians suffered to the judges and governors, who
had misunderstood his intentions.See the edict of Eusebius, l.
ix. c. 10.}

In this general view of the persecution, which was first
authorized by the edicts of Diocletian, I have purposely
refrained from describing the particular sufferings and deaths of
the Christian martyrs. It would have been an easy task, from the
history of Eusebius, from the declamations of Lactantius, and
from the most ancient acts, to collect a long series of horrid
and disgustful pictures, and to fill many pages with racks and
scourges, with iron hooks and red-hot beds, and with all the
variety of tortures which fire and steel, savage beasts, and more
savage executioners, could inflict upon the human body. These
melancholy scenes might be enlivened by a crowd of visions and
miracles destined either to delay the death, to celebrate the
triumph, or to discover the relics of those canonized saints who
suffered for the name of Christ. But I cannot determine what I
ought to transcribe, till I am satisfied how much I ought to
believe. The gravest of the ecclesiastical historians, Eusebius
himself, indirectly confesses, that he has related whatever might
redound to the glory, and that he has suppressed all that could
tend to the disgrace, of religion.\textsuperscript{178} Such an acknowledgment
will naturally excite a suspicion that a writer who has so openly
violated one of the fundamental laws of history, has not paid a
very strict regard to the observance of the other; and the
suspicion will derive additional credit from the character of
Eusebius,\textsuperscript{1781} which was less tinctured with credulity, and more
practised in the arts of courts, than that of almost any of his
contemporaries. On some particular occasions, when the
magistrates were exasperated by some personal motives of interest
or resentment, the rules of prudence, and perhaps of decency, to
overturn the altars, to pour out imprecations against the
emperors, or to strike the judge as he sat on his tribunal, it
may be presumed, that every mode of torture which cruelty could
invent, or constancy could endure, was exhausted on those devoted
victims.\textsuperscript{179} Two circumstances, however, have been unwarily
mentioned, which insinuate that the general treatment of the
Christians, who had been apprehended by the officers of justice,
was less intolerable than it is usually imagined to have been. 1.
The confessors who were condemned to work in the mines were
permitted by the humanity or the negligence of their keepers to
build chapels, and freely to profess their religion in the midst
of those dreary habitations.\textsuperscript{180} 2. The bishops were obliged to
check and to censure the forward zeal of the Christians, who
voluntarily threw themselves into the hands of the magistrates.
Some of these were persons oppressed by poverty and debts, who
blindly sought to terminate a miserable existence by a glorious
death. Others were allured by the hope that a short confinement
would expiate the sins of a whole life; and others again were
actuated by the less honorable motive of deriving a plentiful
subsistence, and perhaps a considerable profit, from the alms
which the charity of the faithful bestowed on the prisoners.\textsuperscript{181}
After the church had triumphed over all her enemies, the interest
as well as vanity of the captives prompted them to magnify the
merit of their respective sufferings. A convenient distance of
time or place gave an ample scope to the progress of fiction; and
the frequent instances which might be alleged of holy martyrs,
whose wounds had been instantly healed, whose strength had been
renewed, and whose lost members had miraculously been restored,
were extremely convenient for the purpose of removing every
difficulty, and of silencing every objection. The most
extravagant legends, as they conduced to the honor of the church,
were applauded by the credulous multitude, countenanced by the
power of the clergy, and attested by the suspicious evidence of
ecclesiastical history.

\pagenote[178]{Such is the \textit{fair} deduction from two remarkable
passages in Eusebius, l. viii. c. 2, and de Martyr. Palestin. c.
12. The prudence of the historian has exposed his own character
to censure and suspicion. It was well known that he himself had
been thrown into prison; and it was suggested that he had
purchased his deliverance by some dishonorable compliance. The
reproach was urged in his lifetime, and even in his presence, at
the council of Tyre. See Tillemont, Mémoires Ecclésiastiques,
tom. viii. part i. p. 67.}

\pagenote[1781]{Historical criticism does not consist in
rejecting indiscriminately all the facts which do not agree with
a particular system, as Gibbon does in this chapter, in which,
except at the last extremity, he will not consent to believe a
martyrdom. Authorities are to be weighed, not excluded from
examination. Now, the Pagan historians justify in many places the
detail which have been transmitted to us by the historians of the
church, concerning the tortures endured by the Christians. Celsus
reproaches the Christians with holding their assemblies in
secret, on account of the fear inspired by their sufferings, “for
when you are arrested,” he says, “you are dragged to punishment:
and, before you are put to death, you have to suffer all kinds of
tortures.” Origen cont. Cels. l. i. ii. vi. viii. passing.
Libanius, the panegyrist of Julian, says, while speaking of the
Christians. “Those who followed a corrupt religion were in
continual apprehensions; they feared lest Julian should invent
tortures still more refined than those to which they had been
exposed before, as mutilation, burning alive, \&c.; for the
emperors had inflicted upon them all these barbarities.” Lib.
Parent in Julian. ap. Fab. Bib. Græc. No. 9, No. 58, p. 283—G.
——This sentence of Gibbon has given rise to several learned
dissertation: Möller, de Fide Eusebii Cæsar, \&c., Havniæ, 1813.
Danzius, de Eusebio Cæs. Hist. Eccl. Scriptore, ejusque tide
historica recte æstimandâ, \&c., Jenæ, 1815. Kestner Commentatio
de Eusebii Hist. Eccles. conditoris auctoritate et fide, \&c. See
also Reuterdahl, de Fontibus Historiæ Eccles. Eusebianæ, Lond.
Goth., 1826. Gibbon’s inference may appear stronger than the text
will warrant, yet it is difficult, after reading the passages, to
dismiss all suspicion of partiality from the mind.—M.}

\pagenote[179]{The ancient, and perhaps authentic, account of the
sufferings of Tarachus and his companions, (Acta Sincera Ruinart,
p. 419—448,) is filled with strong expressions of resentment and
contempt, which could not fail of irritating the magistrate. The
behavior of Ædesius to Hierocles, præfect of Egypt, was still
more extraordinary. Euseb. de Martyr. Palestin. c. 5. * Note: M.
Guizot states, that the acts of Tarachus and his companion
contain nothing that appears dictated by violent feelings,
(sentiment outré.) Nothing can be more painful than the constant
attempt of Gibbon throughout this discussion, to find some flaw
in the virtue and heroism of the martyrs, some extenuation for
the cruelty of the persecutors. But truth must not be sacrificed
even to well-grounded moral indignation. Though the language of
these martyrs is in great part that of calm de fiance, of noble
firmness, yet there are many expressions which betray “resentment
and contempt.” “Children of Satan, worshippers of Devils,” is
their common appellation of the heathen. One of them calls the
judge another, one curses, and declares that he will curse the
Emperors, as pestilential and bloodthirsty tyrants, whom God will
soon visit in his wrath. On the other hand, though at first they
speak the milder language of persuasion, the cold barbarity of
the judges and officers might surely have called forth one
sentence of abhorrence from Gibbon. On the first unsatisfactory
answer, “Break his jaw,” is the order of the judge. They direct
and witness the most excruciating tortures; the people, as M.
Guizot observers, were so much revolted by the cruelty of Maximus
that when the martyrs appeared in the amphitheatre, fear seized
on all hearts, and general murmurs against the unjust judge rank
through the assembly. It is singular, at least, that Gibbon
should have quoted “as probably authentic,” acts so much
embellished with miracle as these of Tarachus are, particularly
towards the end.—M. * Note: Scarcely were the authorities
informed of this, than the president of the province, a man, says
Eusebius, harsh and cruel, banished the confessors, some to
Cyprus, others to different parts of Palestine, and ordered them
to be tormented by being set to the most painful labors. Four of
them, whom he required to abjure their faith and refused, were
burnt alive. Euseb. de Mart. Palest. c. xiii.—G. Two of these
were bishops; a fifth, Silvanus, bishop of Gaza, was the last
martyr; another, named John was blinded, but used to officiate,
and recite from memory long passages of the sacred writings—M.}

\pagenote[180]{Euseb. de Martyr. Palestin. c. 13.}

\pagenote[181]{Augustin. Collat. Carthagin. Dei, iii. c. 13, ap.
Tillanant, Mémoires Ecclésiastiques, tom. v. part i. p. 46. The
controversy with the Donatists, has reflected some, though
perhaps a partial, light on the history of the African church.}

\section{Part \thesection.}

The vague descriptions of exile and imprisonment, of pain and
torture, are so easily exaggerated or softened by the pencil of
an artful orator,\textsuperscript{1811} that we are naturally induced to inquire
into a fact of a more distinct and stubborn kind; the number of
persons who suffered death in consequence of the edicts published
by Diocletian, his associates, and his successors. The recent
legendaries record whole armies and cities, which were at once
swept away by the undistinguishing rage of persecution. The more
ancient writers content themselves with pouring out a liberal
effusion of loose and tragical invectives, without condescending
to ascertain the precise number of those persons who were
permitted to seal with their blood their belief of the gospel.
From the history of Eusebius, it may, however, be collected, that
only nine bishops were punished with death; and we are assured,
by his particular enumeration of the martyrs of Palestine,\textsuperscript{182}
that no more than ninety-two Christians were entitled to that
honorable appellation.\textsuperscript{1821} As we are unacquainted with the
degree of episcopal zeal and courage which prevailed at that
time, it is not in our power to draw any useful inferences from
the former of these facts: but the latter may serve to justify a
very important and probable conclusion. According to the
distribution of Roman provinces, Palestine may be considered as
the sixteenth part of the Eastern empire:\textsuperscript{183} and since there
were some governors, who from a real or affected clemency had
preserved their hands unstained with the blood of the faithful,\textsuperscript{184}
it is reasonable to believe, that the country which had given
birth to Christianity, produced at least the sixteenth part of
the martyrs who suffered death within the dominions of Galerius
and Maximin; the whole might consequently amount to about fifteen
hundred, a number which, if it is equally divided between the ten
years of the persecution, will allow an annual consumption of one
hundred and fifty martyrs. Allotting the same proportion to the
provinces of Italy, Africa, and perhaps Spain, where, at the end
of two or three years, the rigor of the penal laws was either
suspended or abolished, the multitude of Christians in the Roman
empire, on whom a capital punishment was inflicted by a judicia,
sentence, will be reduced to somewhat less than two thousand
persons. Since it cannot be doubted that the Christians were more
numerous, and their enemies more exasperated, in the time of
Diocletian, than they had ever been in any former persecution,
this probable and moderate computation may teach us to estimate
the number of primitive saints and martyrs who sacrificed their
lives for the important purpose of introducing Christianity into
the world.

\pagenote[1811]{Perhaps there never was an instance of an author
committing so deliberately the fault which he reprobates so
strongly in others. What is the dexterous management of the more
inartificial historians of Christianity, in exaggerating the
numbers of the martyrs, compared to the unfair address with which
Gibbon here quietly dismisses from the account all the horrible
and excruciating tortures which fell short of death? The reader
may refer to the xiith chapter (book viii.) of Eusebius for the
description and for the scenes of these tortures.—M.}

\pagenote[182]{Eusebius de Martyr. Palestin. c. 13. He closes his
narration by assuring us that these were the martyrdoms inflicted
in Palestine, during the \textit{whole} course of the persecution. The
9th chapter of his viiith book, which relates to the province of
Thebais in Egypt, may seem to contradict our moderate
computation; but it will only lead us to admire the artful
management of the historian. Choosing for the scene of the most
exquisite cruelty the most remote and sequestered country of the
Roman empire, he relates that in Thebais from ten to one hundred
persons had frequently suffered martyrdom in the same day. But
when he proceeds to mention his own journey into Egypt, his
language insensibly becomes more cautious and moderate. Instead
of a large, but definite number, he speaks of many Christians,
and most artfully selects two ambiguous words, which may signify
either what he had seen, or what he had heard; either the
expectation, or the execution of the punishment. Having thus
provided a secure evasion, he commits the equivocal passage to
his readers and translators; justly conceiving that their piety
would induce them to prefer the most favorable sense. There was
perhaps some malice in the remark of Theodorus Metochita, that
all who, like Eusebius, had been conversant with the Egyptians,
delighted in an obscure and intricate style. (See Valesius ad
loc.)}

\pagenote[1821]{This calculation is made from the martyrs, of
whom Eusebius speaks by name; but he recognizes a much greater
number. Thus the ninth and tenth chapters of his work are
entitled, “Of Antoninus, Zebinus, Germanus, and other martyrs; of
Peter the monk. of Asclepius the Maroionite, and other martyrs.”
[Are these vague contents of chapters very good authority?—M.]
Speaking of those who suffered under Diocletian, he says, “I will
only relate the death of one of these, from which, the reader may
divine what befell the rest.” Hist. Eccl. viii. 6. [This relates
only to the martyrs in the royal household.—M.] Dodwell had made,
before Gibbon, this calculation and these objections; but Ruinart
(Act. Mart. Pref p. 27, \textit{et seq}.) has answered him in a
peremptory manner: Nobis constat Eusebium in historia infinitos
passim martyres admisisse. quamvis revera paucorum nomina
recensuerit. Nec alium Eusebii interpretem quam ipsummet Eusebium
proferimus, qui (l. iii. c. 33) ait sub Trajano plurimosa ex
fidelibus martyrii certamen subiisse (l. v. init.) sub Antonino
et Vero innumerabiles prope martyres per universum orbem
enituisse affirmat. (L. vi. c. 1.) Severum persecutionem
concitasse refert, in qua per omnes ubique locorum Ecclesias, ab
athletis pro pietate certantibus, illustria confecta fuerunt
martyria. Sic de Decii, sic de Valeriani, persecutionibus
loquitur, quæ an Dodwelli faveant conjectionibus judicet æquus
lector. Even in the persecutions which Gibbon has represented as
much more mild than that of Diocletian, the number of martyrs
appears much greater than that to which he limits the martyrs of
the latter: and this number is attested by incontestable
monuments. I will quote but one example. We find among the
letters of St. Cyprian one from Lucianus to Celerinus, written
from the depth of a prison, in which Lucianus names seventeen of
his brethren dead, some in the quarries, some in the midst of
tortures some of starvation in prison. Jussi sumus (he proceeds)
secundum præ ceptum imperatoris, fame et siti necari, et reclusi
sumus in duabus cellis, ta ut nos afficerent fame et siti et
ignis vapore.—G.}

\pagenote[183]{When Palestine was divided into three, the
præfecture of the East contained forty-eight provinces. As the
ancient distinctions of nations were long since abolished, the
Romans distributed the provinces according to a general
proportion of their extent and opulence.}

\pagenote[184]{Ut gloriari possint nullam se innocentium
poremisse, nam et ipse audivi aloquos gloriantes, quia
administratio sua, in hac paris merit incruenta. Lactant.
Institur. Divin v. 11.}

We shall conclude this chapter by a melancholy truth, which
obtrudes itself on the reluctant mind; that even admitting,
without hesitation or inquiry, all that history has recorded, or
devotion has feigned, on the subject of martyrdoms, it must still
be acknowledged, that the Christians, in the course of their
intestine dissensions, have inflicted far greater severities on
each other, than they had experienced from the zeal of infidels.
During the ages of ignorance which followed the subversion of the
Roman empire in the West, the bishops of the Imperial city
extended their dominion over the laity as well as clergy of the
Latin church. The fabric of superstition which they had erected,
and which might long have defied the feeble efforts of reason,
was at length assaulted by a crowd of daring fanatics, who from
the twelfth to the sixteenth century assumed the popular
character of reformers. The church of Rome defended by violence
the empire which she had acquired by fraud; a system of peace and
benevolence was soon disgraced by proscriptions, war, massacres,
and the institution of the holy office. And as the reformers were
animated by the love of civil as well as of religious freedom,
the Catholic princes connected their own interest with that of
the clergy, and enforced by fire and the sword the terrors of
spiritual censures. In the Netherlands alone, more than one
hundred thousand of the subjects of Charles V. are said to have
suffered by the hand of the executioner; and this extraordinary
number is attested by Grotius,\textsuperscript{185} a man of genius and learning,
who preserved his moderation amidst the fury of contending sects,
and who composed the annals of his own age and country, at a time
when the invention of printing had facilitated the means of
intelligence, and increased the danger of detection.

If we are obliged to submit our belief to the authority of
Grotius, it must be allowed, that the number of Protestants, who
were executed in a single province and a single reign, far
exceeded that of the primitive martyrs in the space of three
centuries, and of the Roman empire. But if the improbability of
the fact itself should prevail over the weight of evidence; if
Grotius should be convicted of exaggerating the merit and
sufferings of the Reformers;\textsuperscript{186} we shall be naturally led to
inquire what confidence can be placed in the doubtful and
imperfect monuments of ancient credulity; what degree of credit
can be assigned to a courtly bishop, and a passionate declaimer,\textsuperscript{1861}
who, under the protection of Constantine, enjoyed the
exclusive privilege of recording the persecutions inflicted on
the Christians by the vanquished rivals or disregarded
predecessors of their gracious sovereign.

\pagenote[185]{Grot. Annal. de Rebus Belgicis, l. i. p. 12, edit.
fol.}

\pagenote[186]{Fra Paola (Istoria del Concilio Tridentino, l.
iii.) reduces the number of the Belgic martyrs to 50,000. In
learning and moderation Fra Paola was not inferior to Grotius.
The priority of time gives some advantage to the evidence of the
former, which he loses, on the other hand, by the distance of
Venice from the Netherlands.}

\pagenote[1861]{Eusebius and the author of the Treatise de
Mortibus Persecutorum. It is deeply to be regretted that the
history of this period rest so much on the loose and, it must be
admitted, by no means scrupulous authority of Eusebius.
Ecclesiastical history is a solemn and melancholy lesson that the
best, even the most sacred, cause will eventually the least
departure from truth!—M.}

