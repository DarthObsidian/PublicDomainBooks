\chapter{The Retreat And Death Of Julian.}
\section{Part \thesection.}

\textit{Residence Of Julian At Antioch. — His Successful Expedition Against
The Persians. — Passage Of The Tigris — The Retreat And Death Of
Julian. — Election Of Jovian. — He Saves The Roman Army By A
Disgraceful Treaty.}
\vspace{\onelineskip}

The philosophical fable which Julian composed under the name of
the Cæsars,\textsuperscript{1} is one of the most agreeable and instructive
productions of ancient wit.\textsuperscript{2} During the freedom and equality of
the days of the Saturnalia, Romulus prepared a feast for the
deities of Olympus, who had adopted him as a worthy associate,
and for the Roman princes, who had reigned over his martial
people, and the vanquished nations of the earth. The immortals
were placed in just order on their thrones of state, and the
table of the Cæsars was spread below the Moon in the upper region
of the air. The tyrants, who would have disgraced the society of
gods and men, were thrown headlong, by the inexorable Nemesis,
into the Tartarean abyss. The rest of the Cæsars successively
advanced to their seats; and as they passed, the vices, the
defects, the blemishes of their respective characters, were
maliciously noticed by old Silenus, a laughing moralist, who
disguised the wisdom of a philosopher under the mask of a
Bacchanal.\textsuperscript{3} As soon as the feast was ended, the voice of Mercury
proclaimed the will of Jupiter, that a celestial crown should be
the reward of superior merit. Julius Cæsar, Augustus, Trajan, and
Marcus Antoninus, were selected as the most illustrious
candidates; the effeminate Constantine\textsuperscript{4} was not excluded from
this honorable competition, and the great Alexander was invited
to dispute the prize of glory with the Roman heroes. Each of the
candidates was allowed to display the merit of his own exploits;
but, in the judgment of the gods, the modest silence of Marcus
pleaded more powerfully than the elaborate orations of his
haughty rivals. When the judges of this awful contest proceeded
to examine the heart, and to scrutinize the springs of action,
the superiority of the Imperial Stoic appeared still more
decisive and conspicuous.\textsuperscript{5} Alexander and Cæsar, Augustus,
Trajan, and Constantine, acknowledged, with a blush, that fame,
or power, or pleasure had been the important object of \textit{their}
labors: but the gods themselves beheld, with reverence and love,
a virtuous mortal, who had practised on the throne the lessons of
philosophy; and who, in a state of human imperfection, had
aspired to imitate the moral attributes of the Deity. The value
of this agreeable composition (the Cæsars of Julian) is enhanced
by the rank of the author. A prince, who delineates, with
freedom, the vices and virtues of his predecessors, subscribes,
in every line, the censure or approbation of his own conduct.

\pagenote[1]{See this fable or satire, p. 306-336 of the Leipsig
edition of Julian’s works. The French version of the learned
Ezekiel Spanheim (Paris, 1683) is coarse, languid, and correct;
and his notes, proofs, illustrations, \&c., are piled on each
other till they form a mass of 557 close-printed quarto pages.
The Abbé’ de la Bleterie (Vie de Jovien, tom. i. p. 241-393) has
more happily expressed the spirit, as well as the sense, of the
original, which he illustrates with some concise and curious
notes.}

\pagenote[2]{Spanheim (in his preface) has most learnedly
discussed the etymology, origin, resemblance, and disagreement of
the Greek \textit{satyrs}, a dramatic piece, which was acted after the
tragedy; and the Latin \textit{satires}, (from \textit{Satura},) a
\textit{miscellaneous} composition, either in prose or verse. But the
Cæsars of Julian are of such an original cast, that the critic is
perplexed to which class he should ascribe them. * Note: See also
Casaubon de Satira, with Rambach’s observations.—M.}

\pagenote[3]{This mixed character of Silenus is finely painted in
the sixth eclogue of Virgil.}

\pagenote[4]{Every impartial reader must perceive and condemn the
partiality of Julian against his uncle Constantine, and the
Christian religion. On this occasion, the interpreters are
compelled, by a most sacred interest, to renounce their
allegiance, and to desert the cause of their author.}

\pagenote[5]{Julian was secretly inclined to prefer a Greek to a
Roman. But when he seriously compared a hero with a philosopher,
he was sensible that mankind had much greater obligations to
Socrates than to Alexander, (Orat. ad Themistium, p. 264.)}

In the cool moments of reflection, Julian preferred the useful
and benevolent virtues of Antoninus; but his ambitious spirit was
inflamed by the glory of Alexander; and he solicited, with equal
ardor, the esteem of the wise, and the applause of the multitude.
In the season of life when the powers of the mind and body enjoy
the most active vigor, the emperor who was instructed by the
experience, and animated by the success, of the German war,
resolved to signalize his reign by some more splendid and
memorable achievement. The ambassadors of the East, from the
continent of India, and the Isle of Ceylon,\textsuperscript{6} had respectfully
saluted the Roman purple.\textsuperscript{7} The nations of the West esteemed and
dreaded the personal virtues of Julian, both in peace and war. He
despised the trophies of a Gothic victory, and was satisfied that
the rapacious Barbarians of the Danube would be restrained from
any future violation of the faith of treaties by the terror of
his name, and the additional fortifications with which he
strengthened the Thracian and Illyrian frontiers. The successor
of Cyrus and Artaxerxes was the only rival whom he deemed worthy
of his arms; and he resolved, by the final conquest of Persia, to
chastise the naughty nation which had so long resisted and
insulted the majesty of Rome.\textsuperscript{9} As soon as the Persian monarch
was informed that the throne of Constantius was filled by a
prince of a very different character, he condescended to make
some artful, or perhaps sincere, overtures towards a negotiation
of peace. But the pride of Sapor was astonished by the firmness
of Julian; who sternly declared, that he would never consent to
hold a peaceful conference among the flames and ruins of the
cities of Mesopotamia; and who added, with a smile of contempt,
that it was needless to treat by ambassadors, as he himself had
determined to visit speedily the court of Persia. The impatience
of the emperor urged the diligence of the military preparations.
The generals were named; and Julian, marching from Constantinople
through the provinces of Asia Minor, arrived at Antioch about
eight months after the death of his predecessor. His ardent
desire to march into the heart of Persia, was checked by the
indispensable duty of regulating the state of the empire; by his
zeal to revive the worship of the gods; and by the advice of his
wisest friends; who represented the necessity of allowing the
salutary interval of winter quarters, to restore the exhausted
strength of the legions of Gaul, and the discipline and spirit of
the Eastern troops. Julian was persuaded to fix, till the ensuing
spring, his residence at Antioch, among a people maliciously
disposed to deride the haste, and to censure the delays, of their
sovereign.\textsuperscript{10}

\pagenote[6]{Inde nationibus Indicis certatim cum aonis optimates
mittentibus.... ab usque Divis et \textit{Serendivis}. Ammian. xx. 7.
This island, to which the names of Taprobana, Serendib, and
Ceylon, have been successively applied, manifests how imperfectly
the seas and lands to the east of Cape Comorin were known to the
Romans. 1. Under the reign of Claudius, a freedman, who farmed
the customs of the Red Sea, was accidentally driven by the winds
upon this strange and undiscovered coast: he conversed six months
with the natives; and the king of Ceylon, who heard, for the
first time, of the power and justice of Rome, was persuaded to
send an embassy to the emperor. (Plin. Hist. Nat. vi. 24.) 2. The
geographers (and even Ptolemy) have magnified, above fifteen
times, the real size of this new world, which they extended as
far as the equator, and the neighborhood of China. * Note: The
name of Diva gens or Divorum regio, according to the probable
conjecture of M. Letronne, (Trois Mém. Acad. p. 127,) was applied
by the ancients to the whole eastern coast of the Indian
Peninsula, from Ceylon to the Canges. The name may be traced in
Devipatnam, Devidan, Devicotta, Divinelly, the point of Divy.——M.
Letronne, p.121, considers the freedman with his embassy from
Ceylon to have been an impostor.—M.}

\pagenote[7]{These embassies had been sent to Constantius.
Ammianus, who unwarily deviates into gross flattery, must have
forgotten the length of the way, and the short duration of the
reign of Julian. ——Gothos sæpe fallaces et perfidos; hostes
quærere se meliores aiebat: illis enim sufficere mercators
Galatas per quos ubique sine conditionis discrimine venumdantur.
(Ammian. xxii. 7.) Within less than fifteen years, these Gothic
slaves threatened and subdued their masters.}

\pagenote[9]{Alexander reminds his rival Cæsar, who depreciated
the fame and merit of an Asiatic victory, that Crassus and Antony
had felt the Persian arrows; and that the Romans, in a war of
three hundred years, had not yet subdued the single province of
Mesopotamia or Assyria, (Cæsares, p. 324.)}

\pagenote[10]{The design of the Persian war is declared by
Ammianus, (xxii. 7, 12,) Libanius, (Orat. Parent. c. 79, 80, p.
305, 306,) Zosimus, (l. iii. p. 158,) and Socrates, (l. iii. c.
19.)}

If Julian had flattered himself, that his personal connection
with the capital of the East would be productive of mutual
satisfaction to the prince and people, he made a very false
estimate of his own character, and of the manners of Antioch.\textsuperscript{11}
The warmth of the climate disposed the natives to the most
intemperate enjoyment of tranquillity and opulence; and the
lively licentiousness of the Greeks was blended with the
hereditary softness of the Syrians. Fashion was the only law,
pleasure the only pursuit, and the splendor of dress and
furniture was the only distinction of the citizens of Antioch.
The arts of luxury were honored; the serious and manly virtues
were the subject of ridicule; and the contempt for female modesty
and reverent age announced the universal corruption of the
capital of the East. The love of spectacles was the taste, or
rather passion, of the Syrians; the most skilful artists were
procured from the adjacent cities;\textsuperscript{12} a considerable share of the
revenue was devoted to the public amusements; and the
magnificence of the games of the theatre and circus was
considered as the happiness and as the glory of Antioch. The
rustic manners of a prince who disdained such glory, and was
insensible of such happiness, soon disgusted the delicacy of his
subjects; and the effeminate Orientals could neither imitate, nor
admire, the severe simplicity which Julian always maintained, and
sometimes affected. The days of festivity, consecrated, by
ancient custom, to the honor of the gods, were the only occasions
in which Julian relaxed his philosophic severity; and those
festivals were the only days in which the Syrians of Antioch
could reject the allurements of pleasure. The majority of the
people supported the glory of the Christian name, which had been
first invented by their ancestors:\textsuperscript{13} they contended themselves
with disobeying the moral precepts, but they were scrupulously
attached to the speculative doctrines of their religion. The
church of Antioch was distracted by heresy and schism; but the
Arians and the Athanasians, the followers of Meletius and those
of Paulinus,\textsuperscript{14} were actuated by the same pious hatred of their
common adversary.

\pagenote[11]{The Satire of Julian, and the Homilies of St.
Chrysostom, exhibit the same picture of Antioch. The miniature
which the Abbé de la Bleterie has copied from thence, (Vie de
Julian, p. 332,) is elegant and correct.}

\pagenote[12]{Laodicea furnished charioteers; Tyre and Berytus,
comedians; Cæsarea, pantomimes; Heliopolis, singers; Gaza,
gladiators, Ascalon, wrestlers; and Castabala, rope-dancers. See
the Expositio totius Mundi, p. 6, in the third tome of Hudson’s
Minor Geographers.}

\pagenote[13]{The people of Antioch ingenuously professed their
attachment to the \textit{Chi}, (Christ,) and the \textit{Kappa},
(Constantius.) Julian in Misopogon, p. 357.}

\pagenote[14]{The schism of Antioch, which lasted eighty-five
years, (A. D. 330-415,) was inflamed, while Julian resided in
that city, by the indiscreet ordination of Paulinus. See
Tillemont, Mém. Eccles. tom. iii. p. 803 of the quarto edition,
(Paris, 1701, \&c,) which henceforward I shall quote.}

The strongest prejudice was entertained against the character of
an apostate, the enemy and successor of a prince who had engaged
the affections of a very numerous sect; and the removal of St.
Babylas excited an implacable opposition to the person of Julian.
His subjects complained, with superstitious indignation, that
famine had pursued the emperor’s steps from Constantinople to
Antioch; and the discontent of a hungry people was exasperated by
the injudicious attempt to relieve their distress. The inclemency
of the season had affected the harvests of Syria; and the price
of bread,\textsuperscript{15} in the markets of Antioch, had naturally risen in
proportion to the scarcity of corn. But the fair and reasonable
proportion was soon violated by the rapacious arts of monopoly.
In this unequal contest, in which the produce of the land is
claimed by one party as his exclusive property, is used by
another as a lucrative object of trade, and is required by a
third for the daily and necessary support of life, all the
profits of the intermediate agents are accumulated on the head of
the defenceless customers. The hardships of their situation were
exaggerated and increased by their own impatience and anxiety;
and the apprehension of a scarcity gradually produced the
appearances of a famine. When the luxurious citizens of Antioch
complained of the high price of poultry and fish, Julian publicly
declared, that a frugal city ought to be satisfied with a regular
supply of wine, oil, and bread; but he acknowledged, that it was
the duty of a sovereign to provide for the subsistence of his
people. With this salutary view, the emperor ventured on a very
dangerous and doubtful step, of fixing, by legal authority, the
value of corn. He enacted, that, in a time of scarcity, it should
be sold at a price which had seldom been known in the most
plentiful years; and that his own example might strengthen his
laws, he sent into the market four hundred and twenty-two
thousand \textit{modii}, or measures, which were drawn by his order from
the granaries of Hierapolis, of Chalcis, and even of Egypt. The
consequences might have been foreseen, and were soon felt. The
Imperial wheat was purchased by the rich merchants; the
proprietors of land, or of corn, withheld from the city the
accustomed supply; and the small quantities that appeared in the
market were secretly sold at an advanced and illegal price.
Julian still continued to applaud his own policy, treated the
complaints of the people as a vain and ungrateful murmur, and
convinced Antioch that he had inherited the obstinacy, though not
the cruelty, of his brother Gallus.\textsuperscript{16} The remonstrances of the
municipal senate served only to exasperate his inflexible mind.
He was persuaded, perhaps with truth, that the senators of
Antioch who possessed lands, or were concerned in trade, had
themselves contributed to the calamities of their country; and he
imputed the disrespectful boldness which they assumed, to the
sense, not of public duty, but of private interest. The whole
body, consisting of two hundred of the most noble and wealthy
citizens, were sent, under a guard, from the palace to the
prison; and though they were permitted, before the close of
evening, to return to their respective houses,\textsuperscript{17} the emperor
himself could not obtain the forgiveness which he had so easily
granted. The same grievances were still the subject of the same
complaints, which were industriously circulated by the wit and
levity of the Syrian Greeks. During the licentious days of the
Saturnalia, the streets of the city resounded with insolent
songs, which derided the laws, the religion, the personal
conduct, and even the \textit{beard}, of the emperor; the spirit of
Antioch was manifested by the connivance of the magistrates, and
the applause of the multitude.\textsuperscript{18} The disciple of Socrates was
too deeply affected by these popular insults; but the monarch,
endowed with a quick sensibility, and possessed of absolute
power, refused his passions the gratification of revenge. A
tyrant might have proscribed, without distinction, the lives and
fortunes of the citizens of Antioch; and the unwarlike Syrians
must have patiently submitted to the lust, the rapaciousness and
the cruelty, of the faithful legions of Gaul. A milder sentence
might have deprived the capital of the East of its honors and
privileges; and the courtiers, perhaps the subjects, of Julian,
would have applauded an act of justice, which asserted the
dignity of the supreme magistrate of the republic.\textsuperscript{19} But instead
of abusing, or exerting, the authority of the state, to revenge
his personal injuries, Julian contented himself with an
inoffensive mode of retaliation, which it would be in the power
of few princes to employ. He had been insulted by satires and
libels; in his turn, he composed, under the title of the \textit{Enemy
of the Beard}, an ironical confession of his own faults, and a
severe satire on the licentious and effeminate manners of
Antioch. This Imperial reply was publicly exposed before the
gates of the palace; and the Misopogon\textsuperscript{20} still remains a
singular monument of the resentment, the wit, the humanity, and
the indiscretion of Julian. Though he affected to laugh, he could
not forgive.\textsuperscript{21} His contempt was expressed, and his revenge might
be gratified, by the nomination of a governor\textsuperscript{22} worthy only of
such subjects; and the emperor, forever renouncing the ungrateful
city, proclaimed his resolution to pass the ensuing winter at
Tarsus in Cilicia.\textsuperscript{23}

\pagenote[15]{Julian states three different proportions, of five,
ten, or fifteen \textit{modii} of wheat for one piece of gold, according
to the degrees of plenty and scarcity, (in Misopogon, p. 369.)
From this fact, and from some collateral examples, I conclude,
that under the successors of Constantine, the moderate price of
wheat was about thirty-two shillings the English quarter, which
is equal to the average price of the sixty-four first years of
the present century. See Arbuthnot’s Tables of Coins, Weights,
and Measures, p. 88, 89. Plin. Hist. Natur. xviii. 12. Mém. de
l’Académie des Inscriptions, tom. xxviii. p. 718-721. Smith’s
Inquiry into the Nature and Causes of the Wealth of Nations, vol.
i. p 246. This last I am proud to quote as the work of a sage and
a friend.}

\pagenote[16]{Nunquam a proposito declinabat, Galli similis
fratris, licet incruentus. Ammian. xxii. 14. The ignorance of the
most enlightened princes may claim some excuse; but we cannot be
satisfied with Julian’s own defence, (in Misopogon, p. 363, 369,)
or the elaborate apology of Libanius, (Orat. Parental c. xcvii.
p. 321.)}

\pagenote[17]{Their short and easy confinement is gently touched
by Libanius, (Orat. Parental. c. xcviii. p. 322, 323.)}

\pagenote[18]{Libanius, (ad Antiochenos de Imperatoris ira, c.
17, 18, 19, in Fabricius, Bibliot. Græc. tom. vii. p. 221-223,)
like a skilful advocate, severely censures the folly of the
people, who suffered for the crime of a few obscure and drunken
wretches.}

\pagenote[19]{Libanius (ad Antiochen. c. vii. p. 213) reminds
Antioch of the recent chastisement of Cæsarea; and even Julian
(in Misopogon, p. 355) insinuates how severely Tarentum had
expiated the insult to the Roman ambassadors.}

\pagenote[20]{On the subject of the Misopogon, see Ammianus,
(xxii. 14,) Libanius, (Orat. Parentalis, c. xcix. p. 323,)
Gregory Nazianzen, (Orat. iv. p. 133) and the Chronicle of
Antioch, by John Malala, (tom. ii. p. 15, 16.) I have essential
obligations to the translation and notes of the Abbé de la
Bleterie, (Vie de Jovien, tom. ii. p. 1-138.)}

\pagenote[21]{Ammianus very justly remarks, Coactus dissimulare
pro tempore ira sufflabatur interna. The elaborate irony of
Julian at length bursts forth into serious and direct invective.}

\pagenote[22]{Ipse autem Antiochiam egressurus, Heliopoliten
quendam Alexandrum Syriacæ jurisdictioni præfecit, turbulentum et
sævum; dicebatque non illum meruisse, sed Antiochensibus avaris
et contumeliosis hujusmodi judicem convenire. Ammian. xxiii. 2.
Libanius, (Epist. 722, p. 346, 347,) who confesses to Julian
himself, that he had shared the general discontent, pretends that
Alexander was a useful, though harsh, reformer of the manners and
religion of Antioch.}

\pagenote[23]{Julian, in Misopogon, p. 364. Ammian. xxiii. 2, and
Valesius, ad loc. Libanius, in a professed oration, invites him
to return to his loyal and penitent city of Antioch.}

Yet Antioch possessed one citizen, whose genius and virtues might
atone, in the opinion of Julian, for the vice and folly of his
country. The sophist Libanius was born in the capital of the
East; he publicly professed the arts of rhetoric and declamation
at Nice, Nicomedia, Constantinople, Athens, and, during the
remainder of his life, at Antioch. His school was assiduously
frequented by the Grecian youth; his disciples, who sometimes
exceeded the number of eighty, celebrated their incomparable
master; and the jealousy of his rivals, who persecuted him from
one city to another, confirmed the favorable opinion which
Libanius ostentatiously displayed of his superior merit. The
preceptors of Julian had extorted a rash but solemn assurance,
that he would never attend the lectures of their adversary: the
curiosity of the royal youth was checked and inflamed: he
secretly procured the writings of this dangerous sophist, and
gradually surpassed, in the perfect imitation of his style, the
most laborious of his domestic pupils.\textsuperscript{24} When Julian ascended
the throne, he declared his impatience to embrace and reward the
Syrian sophist, who had preserved, in a degenerate age, the
Grecian purity of taste, of manners, and of religion. The
emperor’s prepossession was increased and justified by the
discreet pride of his favorite. Instead of pressing, with the
foremost of the crowd, into the palace of Constantinople,
Libanius calmly expected his arrival at Antioch; withdrew from
court on the first symptoms of coldness and indifference;
required a formal invitation for each visit; and taught his
sovereign an important lesson, that he might command the
obedience of a subject, but that he must deserve the attachment
of a friend. The sophists of every age, despising, or affecting
to despise, the accidental distinctions of birth and fortune,\textsuperscript{25}
reserve their esteem for the superior qualities of the mind, with
which they themselves are so plentifully endowed. Julian might
disdain the acclamations of a venal court, who adored the
Imperial purple; but he was deeply flattered by the praise, the
admonition, the freedom, and the envy of an independent
philosopher, who refused his favors, loved his person, celebrated
his fame, and protected his memory. The voluminous writings of
Libanius still exist; for the most part, they are the vain and
idle compositions of an orator, who cultivated the science of
words; the productions of a recluse student, whose mind,
regardless of his contemporaries, was incessantly fixed on the
Trojan war and the Athenian commonwealth. Yet the sophist of
Antioch sometimes descended from this imaginary elevation; he
entertained a various and elaborate correspondence;\textsuperscript{26} he praised
the virtues of his own times; he boldly arraigned the abuse of
public and private life; and he eloquently pleaded the cause of
Antioch against the just resentment of Julian and Theodosius. It
is the common calamity of old age,\textsuperscript{27} to lose whatever might have
rendered it desirable; but Libanius experienced the peculiar
misfortune of surviving the religion and the sciences, to which
he had consecrated his genius. The friend of Julian was an
indignant spectator of the triumph of Christianity; and his
bigotry, which darkened the prospect of the visible world, did
not inspire Libanius with any lively hopes of celestial glory and
happiness.\textsuperscript{28}

\pagenote[24]{Libanius, Orat. Parent. c. vii. p. 230, 231.}

\pagenote[25]{Eunapius reports, that Libanius refused the
honorary rank of Prætorian præfect, as less illustrious than the
title of Sophist, (in Vit. Sophist. p. 135.) The critics have
observed a similar sentiment in one of the epistles (xviii. edit.
Wolf) of Libanius himself.}

\pagenote[26]{Near two thousand of his letters—a mode of
composition in which Libanius was thought to excel—are still
extant, and already published. The critics may praise their
subtle and elegant brevity; yet Dr. Bentley (Dissertation upon
Phalaris, p. 48) might justly, though quaintly observe, that “you
feel, by the emptiness and deadness of them, that you converse
with some dreaming pedant, with his elbow on his desk.”}

\pagenote[27]{His birth is assigned to the year 314. He mentions
the seventy-sixth year of his age, (A. D. 390,) and seems to
allude to some events of a still later date.}

\pagenote[28]{Libanius has composed the vain, prolix, but curious
narrative of his own life, (tom. ii. p. 1-84, edit. Morell,) of
which Eunapius (p. 130-135) has left a concise and unfavorable
account. Among the moderns, Tillemont, (Hist. des Empereurs, tom.
iv. p. 571-576,) Fabricius, (Bibliot. Græc. tom. vii. p.
376-414,) and Lardner, (Heathen Testimonies, tom. iv. p.
127-163,) have illustrated the character and writings of this
famous sophist.}

\section{Part \thesection.}

The martial impatience of Julian urged him to take the field in
the beginning of the spring; and he dismissed, with contempt and
reproach, the senate of Antioch, who accompanied the emperor
beyond the limits of their own territory, to which he was
resolved never to return. After a laborious march of two days,\textsuperscript{29}
he halted on the third at Beræa, or Aleppo, where he had the
mortification of finding a senate almost entirely Christian; who
received with cold and formal demonstrations of respect the
eloquent sermon of the apostle of paganism. The son of one of the
most illustrious citizens of Beræa, who had embraced, either from
interest or conscience, the religion of the emperor, was
disinherited by his angry parent. The father and the son were
invited to the Imperial table. Julian, placing himself between
them, attempted, without success, to inculcate the lesson and
example of toleration; supported, with affected calmness, the
indiscreet zeal of the aged Christian, who seemed to forget the
sentiments of nature, and the duty of a subject; and at length,
turning towards the afflicted youth, “Since you have lost a
father,” said he, “for my sake, it is incumbent on me to supply
his place.”\textsuperscript{30} The emperor was received in a manner much more
agreeable to his wishes at Batnæ,\textsuperscript{3011} a small town pleasantly
seated in a grove of cypresses, about twenty miles from the city
of Hierapolis. The solemn rites of sacrifice were decently
prepared by the inhabitants of Batnæ, who seemed attached to the
worship of their tutelar deities, Apollo and Jupiter; but the
serious piety of Julian was offended by the tumult of their
applause; and he too clearly discerned, that the smoke which
arose from their altars was the incense of flattery, rather than
of devotion. The ancient and magnificent temple which had
sanctified, for so many ages, the city of Hierapolis,\textsuperscript{31} no
longer subsisted; and the consecrated wealth, which afforded a
liberal maintenance to more than three hundred priests, might
hasten its downfall. Yet Julian enjoyed the satisfaction of
embracing a philosopher and a friend, whose religious firmness
had withstood the pressing and repeated solicitations of
Constantius and Gallus, as often as those princes lodged at his
house, in their passage through Hierapolis. In the hurry of
military preparation, and the careless confidence of a familiar
correspondence, the zeal of Julian appears to have been lively
and uniform. He had now undertaken an important and difficult
war; and the anxiety of the event rendered him still more
attentive to observe and register the most trifling presages,
from which, according to the rules of divination, any knowledge
of futurity could be derived.\textsuperscript{32} He informed Libanius of his
progress as far as Hierapolis, by an elegant epistle,\textsuperscript{33} which
displays the facility of his genius, and his tender friendship
for the sophist of Antioch.

\pagenote[29]{From Antioch to Litarbe, on the territory of
Chalcis, the road, over hills and through morasses, was extremely
bad; and the loose stones were cemented only with sand, (Julian.
epist. xxvii.) It is singular enough that the Romans should have
neglected the great communication between Antioch and the
Euphrates. See Wesseling Itinerar. p. 190 Bergier, Hist des
Grands Chemins, tom. ii. p. 100}

\pagenote[30]{Julian alludes to this incident, (epist. xxvii.,)
which is more distinctly related by Theodoret, (l. iii. c. 22.)
The intolerant spirit of the father is applauded by Tillemont,
(Hist. des Empereurs, tom. iv. p. 534.) and even by La Bleterie,
(Vie de Julien, p. 413.)}

\pagenote[3011]{This name, of Syriac origin, is found in the
Arabic, and means a place in a valley where waters meet. Julian
says, the name of the city is Barbaric, the situation Greek. The
geographer Abulfeda (tab. Syriac. p. 129, edit. Koehler) speaks
of it in a manner to justify the praises of Julian.—St. Martin.
Notes to Le Beau, iii. 56.—M.}

\pagenote[31]{See the curious treatise de Deâ Syriâ, inserted
among the works of Lucian, (tom. iii. p. 451-490, edit. Reitz.)
The singular appellation of \textit{Ninus vetus} (Ammian. xiv. 8) might
induce a suspicion, that Heirapolis had been the royal seat of
the Assyrians.}

\pagenote[32]{Julian (epist. xxviii.) kept a regular account of
all the fortunate omens; but he suppresses the inauspicious
signs, which Ammianus (xxiii. 2) has carefully recorded.}

\pagenote[33]{Julian. epist. xxvii. p. 399-402.}

Hierapolis,\textsuperscript{3311} situate almost on the banks of the Euphrates,\textsuperscript{34}
had been appointed for the general rendezvous of the Roman
troops, who immediately passed the great river on a bridge of
boats, which was previously constructed.\textsuperscript{35} If the inclinations
of Julian had been similar to those of his predecessor, he might
have wasted the active and important season of the year in the
circus of Samosata or in the churches of Edessa. But as the
warlike emperor, instead of Constantius, had chosen Alexander for
his model, he advanced without delay to Carrhæ,\textsuperscript{36} a very ancient
city of Mesopotamia, at the distance of fourscore miles from
Hierapolis. The temple of the Moon attracted the devotion of
Julian; but the halt of a few days was principally employed in
completing the immense preparations of the Persian war. The
secret of the expedition had hitherto remained in his own breast;
but as Carrhæ is the point of separation of the two great roads,
he could no longer conceal whether it was his design to attack
the dominions of Sapor on the side of the Tigris, or on that of
the Euphrates. The emperor detached an army of thirty thousand
men, under the command of his kinsman Procopius, and of
Sebastian, who had been duke of Egypt. They were ordered to
direct their march towards Nisibis, and to secure the frontier
from the desultory incursions of the enemy, before they attempted
the passage of the Tigris. Their subsequent operations were left
to the discretion of the generals; but Julian expected, that
after wasting with fire and sword the fertile districts of Media
and Adiabene, they might arrive under the walls of Ctesiphon at
the same time that he himself, advancing with equal steps along
the banks of the Euphrates, should besiege the capital of the
Persian monarchy. The success of this well-concerted plan
depended, in a great measure, on the powerful and ready
assistance of the king of Armenia, who, without exposing the
safety of his own dominions, might detach an army of four
thousand horse, and twenty thousand foot, to the assistance of
the Romans.\textsuperscript{37} But the feeble Arsaces Tiranus,\textsuperscript{38} king of
Armenia, had degenerated still more shamefully than his father
Chosroes, from the manly virtues of the great Tiridates; and as
the pusillanimous monarch was averse to any enterprise of danger
and glory, he could disguise his timid indolence by the more
decent excuses of religion and gratitude. He expressed a pious
attachment to the memory of Constantius, from whose hands he had
received in marriage Olympias, the daughter of the præfect
Ablavius; and the alliance of a female, who had been educated as
the destined wife of the emperor Constans, exalted the dignity of
a Barbarian king.\textsuperscript{39} Tiranus professed the Christian religion; he
reigned over a nation of Christians; and he was restrained, by
every principle of conscience and interest, from contributing to
the victory, which would consummate the ruin of the church. The
alienated mind of Tiranus was exasperated by the indiscretion of
Julian, who treated the king of Armenia as \textit{his} slave, and as
the enemy of the gods. The haughty and threatening style of the
Imperial mandates\textsuperscript{40} awakened the secret indignation of a prince,
who, in the humiliating state of dependence, was still conscious
of his royal descent from the Arsacides, the lords of the East,
and the rivals of the Roman power.\textsuperscript{4011}

\pagenote[3311]{Or Bambyce, now Bambouch; Manbedj Arab., or
Maboug, Syr. It was twenty-four Roman miles from the
Euphrates.—M.}

\pagenote[34]{I take the earliest opportunity of acknowledging my
obligations to M. d’Anville, for his recent geography of the
Euphrates and Tigris, (Paris, 1780, in 4to.,) which particularly
illustrates the expedition of Julian.}

\pagenote[35]{There are three passages within a few miles of each
other; 1. Zeugma, celebrated by the ancients; 2. Bir, frequented
by the moderns; and, 3. The bridge of Menbigz, or Hierapolis, at
the distance of four parasangs from the city. —— Djisr Manbedj is
the same with the ancient Zeugma. St. Martin, iii. 58—M.}

\pagenote[36]{Haran, or Carrhæ, was the ancient residence of the
Sabæans, and of Abraham. See the Index Geographicus of Schultens,
(ad calcem Vit. Saladin.,) a work from which I have obtained much
\textit{Oriental} knowledge concerning the ancient and modern geography
of Syria and the adjacent countries. ——On an inedited medal in
the collection of the late M. Tochon. of the Academy of
Inscriptions, it is read Xappan. St. Martin. iii 60—M.}

\pagenote[37]{See Xenophon. Cyropæd. l. iii. p. 189, edit.
Hutchinson. Artavasdes might have supplied Marc Antony with
16,000 horse, armed and disciplined after the Parthian manner,
(Plutarch, in M. Antonio. tom. v. p. 117.)}

\pagenote[38]{Moses of Chorene (Hist. Armeniac. l. iii. c. 11, p.
242) fixes his accession (A. D. 354) to the 17th year of
Constantius. ——Arsaces Tiranus, or Diran, had ceased to reign
twenty-five years before, in 337. The intermediate changes in
Armenia, and the character of this Arsaces, the son of Diran, are
traced by M. St. Martin, at considerable length, in his
supplement to Le Beau, ii. 208-242. As long as his Grecian queen
Olympias maintained her influence, Arsaces was faithful to the
Roman and \textit{Christian} alliance. On the accession of Julian, the
same influence made his fidelity to waver; but Olympias having
been poisoned in the sacramental bread by the agency of
Pharandcem, the former wife of Arsaces, another change took place
in Armenian politics unfavorable to the Christian interest. The
patriarch Narses retired from the impious court to a safe
seclusion. Yet Pharandsem was equally hostile to the Persian
influence, and Arsaces began to support with vigor the cause of
Julian. He made an inroad into the Persian dominions with a body
of Rans and Alans as auxiliaries; wasted Aderbidgan and Sapor,
who had been defeated near Tauriz, was engaged in making head
against his troops in Persarmenia, at the time of the death of
Julian. Such is M. St. Martin’s view, (ii. 276, et sqq.,) which
rests on the Armenian historians, Faustos of Byzantium, and
Mezrob the biographer of the Partriarch Narses. In the history of
Armenia by Father Chamitch, and translated by Avdall, Tiran is
still king of Armenia, at the time of Julian’s death. F. Chamitch
follows Moses of Chorene, The authority of Gibbon.—M.}

\pagenote[39]{Ammian. xx. 11. Athanasius (tom. i. p. 856) says,
in general terms, that Constantius gave to his brother’s widow,
an expression more suitable to a Roman than a Christian.}

\pagenote[40]{Ammianus (xxiii. 2) uses a word much too soft for
the occasion, \textit{monuerat}. Muratori (Fabricius, Bibliothec. Græc.
tom. vii. p. 86) has published an epistle from Julian to the
satrap Arsaces; fierce, vulgar, and (though it might deceive
Sozomen, l. vi. c. 5) most probably spurious. La Bleterie (Hist.
de Jovien, tom. ii. p. 339) translates and rejects it. Note: St.
Martin considers it genuine: the Armenian writers mention such a
letter, iii. 37.—M.}

\pagenote[4011]{Arsaces did not abandon the Roman alliance, but
gave it only feeble support. St. Martin, iii. 41—M.}

The military dispositions of Julian were skilfully contrived to
deceive the spies and to divert the attention of Sapor. The
legions appeared to direct their march towards Nisibis and the
Tigris. On a sudden they wheeled to the right; traversed the
level and naked plain of Carrhæ; and reached, on the third day,
the banks of the Euphrates, where the strong town of Nicephorium,
or Callinicum, had been founded by the Macedonian kings. From
thence the emperor pursued his march, above ninety miles, along
the winding stream of the Euphrates, till, at length, about one
month after his departure from Antioch, he discovered the towers
of Circesium,\textsuperscript{4012} the extreme limit of the Roman dominions. The
army of Julian, the most numerous that any of the Cæsars had ever
led against Persia, consisted of sixty-five thousand effective
and well-disciplined soldiers. The veteran bands of cavalry and
infantry, of Romans and Barbarians, had been selected from the
different provinces; and a just preëminence of loyalty and valor
was claimed by the hardy Gauls, who guarded the throne and person
of their beloved prince. A formidable body of Scythian
auxiliaries had been transported from another climate, and almost
from another world, to invade a distant country, of whose name
and situation they were ignorant. The love of rapine and war
allured to the Imperial standard several tribes of Saracens, or
roving Arabs, whose service Julian had commanded, while he
sternly refused the payment of the accustomed subsidies. The
broad channel of the Euphrates\textsuperscript{41} was crowded by a fleet of
eleven hundred ships, destined to attend the motions, and to
satisfy the wants, of the Roman army. The military strength of
the fleet was composed of fifty armed galleys; and these were
accompanied by an equal number of flat-bottomed boats, which
might occasionally be connected into the form of temporary
bridges. The rest of the ships, partly constructed of timber, and
partly covered with raw hides, were laden with an almost
inexhaustible supply of arms and engines, of utensils and
provisions. The vigilant humanity of Julian had embarked a very
large magazine of vinegar and biscuit for the use of the
soldiers, but he prohibited the indulgence of wine; and
rigorously stopped a long string of superfluous camels that
attempted to follow the rear of the army. The River Chaboras
falls into the Euphrates at Circesium;\textsuperscript{42} and as soon as the
trumpet gave the signal of march, the Romans passed the little
stream which separated two mighty and hostile empires. The custom
of ancient discipline required a military oration; and Julian
embraced every opportunity of displaying his eloquence. He
animated the impatient and attentive legions by the example of
the inflexible courage and glorious triumphs of their ancestors.
He excited their resentment by a lively picture of the insolence
of the Persians; and he exhorted them to imitate his firm
resolution, either to extirpate that perfidious nation, or to
devote his life in the cause of the republic. The eloquence of
Julian was enforced by a donative of one hundred and thirty
pieces of silver to every soldier; and the bridge of the Chaboras
was instantly cut away, to convince the troops that they must
place their hopes of safety in the success of their arms. Yet the
prudence of the emperor induced him to secure a remote frontier,
perpetually exposed to the inroads of the hostile Arabs. A
detachment of four thousand men was left at Circesium, which
completed, to the number of ten thousand, the regular garrison of
that important fortress.\textsuperscript{43}

\pagenote[4012]{Kirkesia the Carchemish of the Scriptures.—M.}

\pagenote[41]{Latissimum flumen Euphraten artabat. Ammian. xxiii.
3 Somewhat higher, at the fords of Thapsacus, the river is four
stadia or 800 yards, almost half an English mile, broad.
(Xenophon, Anabasis, l. i. p. 41, edit. Hutchinson, with Foster’s
Observations, p. 29, \&c., in the 2d volume of Spelman’s
translation.) If the breadth of the Euphrates at Bir and Zeugma
is no more than 130 yards, (Voyages de Niebuhr, tom. ii. p. 335,)
the enormous difference must chiefly arise from the depth of the
channel.}

\pagenote[42]{Munimentum tutissimum et fabre politum, Abora (the
Orientals aspirate Chaboras or Chabour) et Euphrates ambiunt
flumina, velut spatium insulare fingentes. Ammian. xxiii. 5.}

\pagenote[43]{The enterprise and armament of Julian are described
by himself, (Epist. xxvii.,) Ammianus Marcellinus, (xxiii. 3, 4,
5,) Libanius, (Orat. Parent. c. 108, 109, p. 332, 333,) Zosimus,
(l. iii. p. 160, 161, 162) Sozomen, (l. vi. c. l,) and John
Malala, (tom. ii. p. 17.)}

From the moment that the Romans entered the enemy’s country,\textsuperscript{44}
the country of an active and artful enemy, the order of march was
disposed in three columns.\textsuperscript{45} The strength of the infantry, and
consequently of the whole army was placed in the centre, under
the peculiar command of their master-general Victor. On the
right, the brave Nevitta led a column of several legions along
the banks of the Euphrates, and almost always in sight of the
fleet. The left flank of the army was protected by the column of
cavalry. Hormisdas and Arinthæus were appointed generals of the
horse; and the singular adventures of Hormisdas\textsuperscript{46} are not
undeserving of our notice. He was a Persian prince, of the royal
race of the Sassanides, who, in the troubles of the minority of
Sapor, had escaped from prison to the hospitable court of the
great Constantine. Hormisdas at first excited the compassion, and
at length acquired the esteem, of his new masters; his valor and
fidelity raised him to the military honors of the Roman service;
and though a Christian, he might indulge the secret satisfaction
of convincing his ungrateful country, that an oppressed subject
may prove the most dangerous enemy. Such was the disposition of
the three principal columns. The front and flanks of the army
were covered by Lucilianus with a flying detachment of fifteen
hundred light-armed soldiers, whose active vigilance observed the
most distant signs, and conveyed the earliest notice, of any
hostile approach. Dagalaiphus, and Secundinus duke of Osrhoene,
conducted the troops of the rear-guard; the baggage securely
proceeded in the intervals of the columns; and the ranks, from a
motive either of use or ostentation, were formed in such open
order, that the whole line of march extended almost ten miles.
The ordinary post of Julian was at the head of the centre column;
but as he preferred the duties of a general to the state of a
monarch, he rapidly moved, with a small escort of light cavalry,
to the front, the rear, the flanks, wherever his presence could
animate or protect the march of the Roman army. The country which
they traversed from the Chaboras, to the cultivated lands of
Assyria, may be considered as a part of the desert of Arabia, a
dry and barren waste, which could never be improved by the most
powerful arts of human industry. Julian marched over the same
ground which had been trod above seven hundred years before by
the footsteps of the younger Cyrus, and which is described by one
of the companions of his expedition, the sage and heroic
Xenophon.\textsuperscript{47} “The country was a plain throughout, as even as the
sea, and full of wormwood; and if any other kind of shrubs or
reeds grew there, they had all an aromatic smell, but no trees
could be seen. Bustards and ostriches, antelopes and wild asses,\textsuperscript{48}
appeared to be the only inhabitants of the desert; and the
fatigues of the march were alleviated by the amusements of the
chase.” The loose sand of the desert was frequently raised by the
wind into clouds of dust; and a great number of the soldiers of
Julian, with their tents, were suddenly thrown to the ground by
the violence of an unexpected hurricane.

\pagenote[44]{Before he enters Persia, Ammianus copiously
describes (xxiii. p. 396-419, edit. Gronov. in 4to.) the eighteen
great provinces, (as far as the Seric, or Chinese frontiers,)
which were subject to the Sassanides.}

\pagenote[45]{Ammianus (xxiv. 1) and Zosimus (l. iii. p. 162,
163) rately expressed the order of march.}

\pagenote[46]{The adventures of Hormisdas are related with some
mixture of fable, (Zosimus, l. ii. p. 100-102; Tillemont, Hist.
des Empereurs tom. iv. p. 198.) It is almost impossible that he
should be the brother (frater germanus) of an \textit{eldest} and
\textit{posthumous} child: nor do I recollect that Ammianus ever gives
him that title. * Note: St. Martin conceives that he was an elder
brother by another mother who had several children, ii. 24—M.}

\pagenote[47]{See the first book of the Anabasis, p. 45, 46. This
pleasing work is original and authentic. Yet Xenophon’s memory,
perhaps many years after the expedition, has sometimes betrayed
him; and the distances which he marks are often larger than
either a soldier or a geographer will allow.}

\pagenote[48]{Mr. Spelman, the English translator of the
Anabasis, (vol. i. p. 51,) confounds the antelope with the
roebuck, and the wild ass with the zebra.}

The sandy plains of Mesopotamia were abandoned to the antelopes
and wild asses of the desert; but a variety of populous towns and
villages were pleasantly situated on the banks of the Euphrates,
and in the islands which are occasionally formed by that river.
The city of Annah, or Anatho,\textsuperscript{49} the actual residence of an
Arabian emir, is composed of two long streets, which enclose,
within a natural fortification, a small island in the midst, and
two fruitful spots on either side, of the Euphrates. The warlike
inhabitants of Anatho showed a disposition to stop the march of a
Roman emperor; till they were diverted from such fatal
presumption by the mild exhortations of Prince Hormisdas, and the
approaching terrors of the fleet and army. They implored, and
experienced, the clemency of Julian, who transplanted the people
to an advantageous settlement, near Chalcis in Syria, and
admitted Pusæus, the governor, to an honorable rank in his
service and friendship. But the impregnable fortress of Thilutha
could scorn the menace of a siege; and the emperor was obliged to
content himself with an insulting promise, that, when he had
subdued the interior provinces of Persia, Thilutha would no
longer refuse to grace the triumph of the emperor. The
inhabitants of the open towns, unable to resist, and unwilling to
yield, fled with precipitation; and their houses, filled with
spoil and provisions, were occupied by the soldiers of Julian,
who massacred, without remorse and without punishment, some
defenceless women. During the march, the Surenas, 4911 or Persian
general, and Malek Rodosaces, the renowned emir of the tribe of
Gassan,\textsuperscript{50} incessantly hovered round the army; every straggler
was intercepted; every detachment was attacked; and the valiant
Hormisdas escaped with some difficulty from their hands. But the
Barbarians were finally repulsed; the country became every day
less favorable to the operations of cavalry; and when the Romans
arrived at Macepracta, they perceived the ruins of the wall,
which had been constructed by the ancient kings of Assyria, to
secure their dominions from the incursions of the Medes. These
preliminaries of the expedition of Julian appear to have employed
about fifteen days; and we may compute near three hundred miles
from the fortress of Circesium to the wall of Macepracta.\textsuperscript{51}

\pagenote[49]{See Voyages de Tavernier, part i. l. iii. p. 316,
and more especially Viaggi di Pietro della Valle, tom. i. lett.
xvii. p. 671, \&c. He was ignorant of the old name and condition
of Annah. Our blind travellers \textit{seldom} possess any previous
knowledge of the countries which they visit. Shaw and Tournefort
deserve an honorable exception.}

\pagenote[4911]{This is not a title, but the name of a great
Persian family. St. Martin, iii. 79.—M.}

\pagenote[50]{Famosi nominis latro, says Ammianus; a high
encomium for an Arab. The tribe of Gassan had settled on the edge
of Syria, and reigned some time in Damascus, under a dynasty of
thirty-one kings, or emirs, from the time of Pompey to that of
the Khalif Omar. D’Herbelot, Bibliothèque Orientale, p. 360.
Pococke, Specimen Hist. Arabicæ, p. 75-78. The name of Rodosaces
does not appear in the list. * Note: Rodosaces-malek is king. St.
Martin considers that Gibbon has fallen into an error in bringing
the tribe of Gassan to the Euphrates. In Ammianus it is Assan. M.
St. Martin would read Massanitarum, the same with the Mauzanitæ
of Malala.—M.}

\pagenote[51]{See Ammianus, (xxiv. 1, 2,) Libanius, (Orat.
Parental. c. 110, 111, p. 334,) Zosimus, (l. iii. p. 164-168.) *
Note: This Syriac or Chaldaic has relation to its position; it
easily bears the signification of the division of the waters. M.
St. M. considers it the Missice of Pliny, v. 26. St. Martin, iii.
83.—M.}

The fertile province of Assyria,\textsuperscript{52} which stretched beyond the
Tigris, as far as the mountains of Media,\textsuperscript{53} extended about four
hundred miles from the ancient wall of Macepracta, to the
territory of Basra, where the united streams of the Euphrates and
Tigris discharge themselves into the Persian Gulf.\textsuperscript{54} The whole
country might have claimed the peculiar name of Mesopotamia; as
the two rivers, which are never more distant than fifty,
approach, between Bagdad and Babylon, within twenty-five miles,
of each other. A multitude of artificial canals, dug without much
labor in a soft and yielding soil connected the rivers, and
intersected the plain of Assyria. The uses of these artificial
canals were various and important. They served to discharge the
superfluous waters from one river into the other, at the season
of their respective inundations. Subdividing themselves into
smaller and smaller branches, they refreshed the dry lands, and
supplied the deficiency of rain. They facilitated the intercourse
of peace and commerce; and, as the dams could be speedily broke
down, they armed the despair of the Assyrians with the means of
opposing a sudden deluge to the progress of an invading army. To
the soil and climate of Assyria, nature had denied some of her
choicest gifts, the vine, the olive, and the fig-tree;\textsuperscript{5411} but
the food which supports the life of man, and particularly wheat
and barley, were produced with inexhaustible fertility; and the
husbandman, who committed his seed to the earth, was frequently
rewarded with an increase of two, or even of three, hundred. The
face of the country was interspersed with groves of innumerable
palm-trees;\textsuperscript{55} and the diligent natives celebrated, either in
verse or prose, the three hundred and sixty uses to which the
trunk, the branches, the leaves, the juice, and the fruit, were
skilfully applied. Several manufactures, especially those of
leather and linen, employed the industry of a numerous people,
and afforded valuable materials for foreign trade; which appears,
however, to have been conducted by the hands of strangers.
Babylon had been converted into a royal park; but near the ruins
of the ancient capital, new cities had successively arisen, and
the populousness of the country was displayed in the multitude of
towns and villages, which were built of bricks dried in the sun,
and strongly cemented with bitumen; the natural and peculiar
production of the Babylonian soil. While the successors of Cyrus
reigned over Asia, the province of Syria alone maintained, during
a third part of the year, the luxurious plenty of the table and
household of the Great King. Four considerable villages were
assigned for the subsistence of his Indian dogs; eight hundred
stallions, and sixteen thousand mares, were constantly kept, at
the expense of the country, for the royal stables; and as the
daily tribute, which was paid to the satrap, amounted to one
English bushe of silver, we may compute the annual revenue of
Assyria at more than twelve hundred thousand pounds sterling.\textsuperscript{56}

\pagenote[52]{The description of Assyria, is furnished by
Herodotus, (l. i. c. 192, \&c.,) who sometimes writes for
children, and sometimes for philosophers; by Strabo, (l. xvi. p.
1070-1082,) and by Ammianus, (l.xxiii. c. 6.) The most useful of
the modern travellers are Tavernier, (part i. l. ii. p. 226-258,)
Otter, (tom. ii. p. 35-69, and 189-224,) and Niebuhr, (tom. ii.
p. 172-288.) Yet I much regret that the \textit{Irak Arabi} of Abulfeda
has not been translated.}

\pagenote[53]{Ammianus remarks, that the primitive Assyria, which
comprehended Ninus, (Nineveh,) and Arbela, had assumed the more
recent and peculiar appellation of Adiabene; and he seems to fix
Teredon, Vologesia, and Apollonia, as the \textit{extreme} cities of the
actual province of Assyria.}

\pagenote[54]{The two rivers unite at Apamea, or Corna, (one
hundred miles from the Persian Gulf,) into the broad stream of
the Pasitigris, or Shutul-Arab. The Euphrates formerly reached
the sea by a separate channel, which was obstructed and diverted
by the citizens of Orchoe, about twenty miles to the south-east
of modern Basra. (D’Anville, in the Mémoires de l’Acad. des
Inscriptions, tom.xxx. p. 171-191.)}

\pagenote[5411]{We are informed by Mr. Gibbon, that nature has
denied to the soil an climate of Assyria some of her choicest
gifts, the vine, the olive, and the fig-tree. This might have
been the case ir the age of Ammianus Marcellinus, but it is not
so at the present day; and it is a curious fact that the grape,
the olive, and the fig, are the most common fruits in the
province, and may be seen in every garden. Macdonald Kinneir,
Geogr. Mem. on Persia 239—M.}

\pagenote[55]{The learned Kæmpfer, as a botanist, an antiquary,
and a traveller, has exhausted (Amœnitat. Exoticæ, Fasicul. iv.
p. 660-764) the whole subject of palm-trees.}

\pagenote[56]{Assyria yielded to the Persian satrap an \textit{Artaba}
of silver each day. The well-known proportion of weights and
measures (see Bishop Hooper’s elaborate Inquiry,) the specific
gravity of water and silver, and the value of that metal, will
afford, after a short process, the annual revenue which I have
stated. Yet the Great King received no more than 1000 Euboic, or
Tyrian, talents (252,000l.) from Assyria. The comparison of two
passages in Herodotus, (l. i. c. 192, l. iii. c. 89-96) reveals
an important difference between the \textit{gross}, and the \textit{net},
revenue of Persia; the sums paid by the province, and the gold or
silver deposited in the royal treasure. The monarch might
annually save three millions six hundred thousand pounds, of the
seventeen or eighteen millions raised upon the people.}

\section{Part \thesection.}

The fields of Assyria were devoted by Julian to the calamities of
war; and the philosopher retaliated on a guiltless people the
acts of rapine and cruelty which had been committed by their
haughty master in the Roman provinces. The trembling Assyrians
summoned the rivers to their assistance; and completed, with
their own hands, the ruin of their country. The roads were
rendered impracticable; a flood of waters was poured into the
camp; and, during several days, the troops of Julian were obliged
to contend with the most discouraging hardships. But every
obstacle was surmounted by the perseverance of the legionaries,
who were inured to toil as well as to danger, and who felt
themselves animated by the spirit of their leader. The damage was
gradually repaired; the waters were restored to their proper
channels; whole groves of palm-trees were cut down, and placed
along the broken parts of the road; and the army passed over the
broad and deeper canals, on bridges of floating rafts, which were
supported by the help of bladders. Two cities of Assyria presumed
to resist the arms of a Roman emperor: and they both paid the
severe penalty of their rashness. At the distance of fifty miles
from the royal residence of Ctesiphon, Perisabor,\textsuperscript{5711} or Anbar,
held the second rank in the province; a city, large, populous,
and well fortified, surrounded with a double wall, almost
encompassed by a branch of the Euphrates, and defended by the
valor of a numerous garrison. The exhortations of Hormisdas were
repulsed with contempt; and the ears of the Persian prince were
wounded by a just reproach, that, unmindful of his royal birth,
he conducted an army of strangers against his king and country.
The Assyrians maintained their loyalty by a skilful, as well as
vigorous, defence; till the lucky stroke of a battering-ram,
having opened a large breach, by shattering one of the angles of
the wall, they hastily retired into the fortifications of the
interior citadel. The soldiers of Julian rushed impetuously into
the town, and after the full gratification of every military
appetite, Perisabor was reduced to ashes; and the engines which
assaulted the citadel were planted on the ruins of the smoking
houses. The contest was continued by an incessant and mutual
discharge of missile weapons; and the superiority which the
Romans might derive from the mechanical powers of their balistæ
and catapultæ was counterbalanced by the advantage of the ground
on the side of the besieged. But as soon as an \textit{Helepolis} had
been constructed, which could engage on equal terms with the
loftiest ramparts, the tremendous aspect of a moving turret, that
would leave no hope of resistance or mercy, terrified the
defenders of the citadel into an humble submission; and the place
was surrendered only two days after Julian first appeared under
the walls of Perisabor. Two thousand five hundred persons, of
both sexes, the feeble remnant of a flourishing people, were
permitted to retire; the plentiful magazines of corn, of arms,
and of splendid furniture, were partly distributed among the
troops, and partly reserved for the public service; the useless
stores were destroyed by fire or thrown into the stream of the
Euphrates; and the fate of Amida was revenged by the total ruin
of Perisabor.

\pagenote[5711]{Libanius says that it was a great city of
Assyria, called after the name of the reigning king. The orator
of Antioch is not mistaken. The Persians and Syrians called it
Fyrouz Schapour or Fyrouz Schahbour; in Persian, the victory of
Schahpour. It owed that name to Sapor the First. It was before
called Anbar St. Martin, iii. 85.—M.}

The city or rather fortress, of Maogamalcha, which was defended
by sixteen large towers, a deep ditch, and two strong and solid
walls of brick and bitumen, appears to have been constructed at
the distance of eleven miles, as the safeguard of the capital of
Persia. The emperor, apprehensive of leaving such an important
fortress in his rear, immediately formed the siege of
Maogamalcha; and the Roman army was distributed, for that
purpose, into three divisions. Victor, at the head of the
cavalry, and of a detachment of heavy-armed foot, was ordered to
clear the country, as far as the banks of the Tigris, and the
suburbs of Ctesiphon. The conduct of the attack was assumed by
Julian himself, who seemed to place his whole dependence in the
military engines which he erected against the walls; while he
secretly contrived a more efficacious method of introducing his
troops into the heart of the city. Under the direction of Nevitta
and Dagalaiphus, the trenches were opened at a considerable
distance, and gradually prolonged as far as the edge of the
ditch. The ditch was speedily filled with earth; and, by the
incessant labor of the troops, a mine was carried under the
foundations of the walls, and sustained, at sufficient intervals,
by props of timber. Three chosen cohorts, advancing in a single
file, silently explored the dark and dangerous passage; till
their intrepid leader whispered back the intelligence, that he
was ready to issue from his confinement into the streets of the
hostile city. Julian checked their ardor, that he might insure
their success; and immediately diverted the attention of the
garrison, by the tumult and clamor of a general assault. The
Persians, who, from their walls, contemptuously beheld the
progress of an impotent attack, celebrated with songs of triumph
the glory of Sapor; and ventured to assure the emperor, that he
might ascend the starry mansion of Ormusd, before he could hope
to take the impregnable city of Maogamalcha. The city was already
taken. History has recorded the name of a private soldier the
first who ascended from the mine into a deserted tower. The
passage was widened by his companions, who pressed forwards with
impatient valor. Fifteen hundred enemies were already in the
midst of the city. The astonished garrison abandoned the walls,
and their only hope of safety; the gates were instantly burst
open; and the revenge of the soldier, unless it were suspended by
lust or avarice, was satiated by an undistinguishing massacre.
The governor, who had yielded on a promise of mercy, was burnt
alive, a few days afterwards, on a charge of having uttered some
disrespectful words against the honor of Prince Hormisdas. The
fortifications were razed to the ground; and not a vestige was
left, that the city of Maogamalcha had ever existed. The
neighborhood of the capital of Persia was adorned with three
stately palaces, laboriously enriched with every production that
could gratify the luxury and pride of an Eastern monarch. The
pleasant situation of the gardens along the banks of the Tigris,
was improved, according to the Persian taste, by the symmetry of
flowers, fountains, and shady walks: and spacious parks were
enclosed for the reception of the bears, lions, and wild boars,
which were maintained at a considerable expense for the pleasure
of the royal chase. The park walls were broken down, the savage
game was abandoned to the darts of the soldiers, and the palaces
of Sapor were reduced to ashes, by the command of the Roman
emperor. Julian, on this occasion, showed himself ignorant, or
careless, of the laws of civility, which the prudence and
refinement of polished ages have established between hostile
princes. Yet these wanton ravages need not excite in our breasts
any vehement emotions of pity or resentment. A simple, naked
statue, finished by the hand of a Grecian artist, is of more
genuine value than all these rude and costly monuments of
Barbaric labor; and, if we are more deeply affected by the ruin
of a palace than by the conflagration of a cottage, our humanity
must have formed a very erroneous estimate of the miseries of
human life.\textsuperscript{57}

\pagenote[57]{The operations of the Assyrian war are
circumstantially related by Ammianus, (xxiv. 2, 3, 4, 5,)
Libanius, (Orat. Parent. c. 112-123, p. 335-347,) Zosimus, (l.
iii. p. 168-180,) and Gregory Nazianzen, (Orat iv. p. 113, 144.)
The \textit{military} criticisms of the saint are devoutly copied by
Tillemont, his faithful slave.}

Julian was an object of hatred and terror to the Persian and the
painters of that nation represented the invader of their country
under the emblem of a furious lion, who vomited from his mouth a
consuming fire.\textsuperscript{58} To his friends and soldiers the philosophic
hero appeared in a more amiable light; and his virtues were never
more conspicuously displayed, than in the last and most active
period of his life. He practised, without effort, and almost
without merit, the habitual qualities of temperance and sobriety.
According to the dictates of that artificial wisdom, which
assumes an absolute dominion over the mind and body, he sternly
refused himself the indulgence of the most natural appetites.\textsuperscript{59}
In the warm climate of Assyria, which solicited a luxurious
people to the gratification of every sensual desire,\textsuperscript{60} a
youthful conqueror preserved his chastity pure and inviolate; nor
was Julian ever tempted, even by a motive of curiosity, to visit
his female captives of exquisite beauty,\textsuperscript{61} who, instead of
resisting his power, would have disputed with each other the
honor of his embraces. With the same firmness that he resisted
the allurements of love, he sustained the hardships of war. When
the Romans marched through the flat and flooded country, their
sovereign, on foot, at the head of his legions, shared their
fatigues and animated their diligence. In every useful labor, the
hand of Julian was prompt and strenuous; and the Imperial purple
was wet and dirty as the coarse garment of the meanest soldier.
The two sieges allowed him some remarkable opportunities of
signalizing his personal valor, which, in the improved state of
the military art, can seldom be exerted by a prudent general. The
emperor stood before the citadel of Perisabor, insensible of his
extreme danger, and encouraged his troops to burst open the gates
of iron, till he was almost overwhelmed under a cloud of missile
weapons and huge stones, that were directed against his person.
As he examined the exterior fortifications of Maogamalcha, two
Persians, devoting themselves for their country, suddenly rushed
upon him with drawn cimeters: the emperor dexterously received
their blows on his uplifted shield; and, with a steady and
well-aimed thrust, laid one of his adversaries dead at his feet.
The esteem of a prince who possesses the virtues which he
approves, is the noblest recompense of a deserving subject; and
the authority which Julian derived from his personal merit,
enabled him to revive and enforce the rigor of ancient
discipline. He punished with death or ignominy the misbehavior of
three troops of horse, who, in a skirmish with the Surenas, had
lost their honor and one of their standards: and he distinguished
with \textit{obsidional}\textsuperscript{62} crowns the valor of the foremost soldiers,
who had ascended into the city of Maogamalcha.

After the siege of Perisabor, the firmness of the emperor was
exercised by the insolent avarice of the army, who loudly
complained, that their services were rewarded by a trifling
donative of one hundred pieces of silver. His just indignation
was expressed in the grave and manly language of a Roman. “Riches
are the object of your desires; those riches are in the hands of
the Persians; and the spoils of this fruitful country are
proposed as the prize of your valor and discipline. Believe me,”
added Julian, “the Roman republic, which formerly possessed such
immense treasures, is now reduced to want and wretchedness once
our princes have been persuaded, by weak and interested
ministers, to purchase with gold the tranquillity of the
Barbarians. The revenue is exhausted; the cities are ruined; the
provinces are dispeopled. For myself, the only inheritance that I
have received from my royal ancestors is a soul incapable of
fear; and as long as I am convinced that every real advantage is
seated in the mind, I shall not blush to acknowledge an honorable
poverty, which, in the days of ancient virtue, was considered as
the glory of Fabricius. That glory, and that virtue, may be your
own, if you will listen to the voice of Heaven and of your
leader. But if you will rashly persist, if you are determined to
renew the shameful and mischievous examples of old seditions,
proceed. As it becomes an emperor who has filled the first rank
among men, I am prepared to die, standing; and to despise a
precarious life, which, every hour, may depend on an accidental
fever. If I have been found unworthy of the command, there are
now among you, (I speak it with pride and pleasure,) there are
many chiefs whose merit and experience are equal to the conduct
of the most important war. Such has been the temper of my reign,
that I can retire, without regret, and without apprehension, to
the obscurity of a private station”\textsuperscript{63} The modest resolution of
Julian was answered by the unanimous applause and cheerful
obedience of the Romans, who declared their confidence of
victory, while they fought under the banners of their heroic
prince. Their courage was kindled by his frequent and familiar
asseverations, (for such wishes were the oaths of Julian,) “So
may I reduce the Persians under the yoke!” “Thus may I restore
the strength and splendor of the republic!” The love of fame was
the ardent passion of his soul: but it was not before he trampled
on the ruins of Maogamalcha, that he allowed himself to say, “We
have now provided some materials for the sophist of Antioch.”\textsuperscript{64}

\pagenote[58]{Libanius de ulciscenda Juliani nece, c. 13, p.
162.}

\pagenote[59]{The famous examples of Cyrus, Alexander, and
Scipio, were acts of justice. Julian’s chastity was voluntary,
and, in his opinion, meritorious.}

\pagenote[60]{Sallust (ap. Vet. Scholiast. Juvenal. Satir. i.
104) observes, that nihil corruptius moribus. The matrons and
virgins of Babylon freely mingled with the men in licentious
banquets; and as they felt the intoxication of wine and love,
they gradually, and almost completely, threw aside the
encumbrance of dress; ad ultimum ima corporum velamenta
projiciunt. Q. Curtius, v. 1.}

\pagenote[61]{Ex virginibus autem quæ speciosæ sunt captæ, et in
Perside, ubi fæminarum pulchritudo excellit, nec contrectare
aliquam votuit nec videre. Ammian. xxiv. 4. The native race of
Persians is small and ugly; but it has been improved by the
perpetual mixture of Circassian blood, (Herodot. l. iii. c. 97.
Buffon, Hist. Naturelle, tom. iii. p. 420.)}

\pagenote[62]{Obsidionalibus coronis donati. Ammian. xxiv. 4.
Either Julian or his historian were unskillful antiquaries. He
should have given mural crowns. The \textit{obsidional} were the reward
of a general who had delivered a besieged city, (Aulus Gellius,
Noct. Attic. v. 6.)}

\pagenote[63]{I give this speech as original and genuine.
Ammianus might hear, could transcribe, and was incapable of
inventing, it. I have used some slight freedoms, and conclude
with the most forcibic sentence.}

\pagenote[64]{Ammian. xxiv. 3. Libanius, Orat. Parent. c. 122, p.
346.}

The successful valor of Julian had triumphed over all the
obstacles that opposed his march to the gates of Ctesiphon. But
the reduction, or even the siege, of the capital of Persia, was
still at a distance: nor can the military conduct of the emperor
be clearly apprehended, without a knowledge of the country which
was the theatre of his bold and skilful operations.\textsuperscript{65} Twenty
miles to the south of Bagdad, and on the eastern bank of the
Tigris, the curiosity of travellers has observed some ruins of
the palaces of Ctesiphon, which, in the time of Julian, was a
great and populous city. The name and glory of the adjacent
Seleucia were forever extinguished; and the only remaining
quarter of that Greek colony had resumed, with the Assyrian
language and manners, the primitive appellation of Coche. Coche
was situate on the western side of the Tigris; but it was
naturally considered as a suburb of Ctesiphon, with which we may
suppose it to have been connected by a permanent bridge of boats.

The united parts contribute to form the common epithet of Al
Modain, the cities, which the Orientals have bestowed on the
winter residence of the Sassinadees; and the whole circumference
of the Persian capital was strongly fortified by the waters of
the river, by lofty walls, and by impracticable morasses. Near
the ruins of Seleucia, the camp of Julian was fixed, and secured,
by a ditch and rampart, against the sallies of the numerous and
enterprising garrison of Coche. In this fruitful and pleasant
country, the Romans were plentifully supplied with water and
forage: and several forts, which might have embarrassed the
motions of the army, submitted, after some resistance, to the
efforts of their valor. The fleet passed from the Euphrates into
an artificial derivation of that river, which pours a copious and
navigable stream into the Tigris, at a small distance \textit{below} the
great city. If they had followed this royal canal, which bore the
name of Nahar-Malcha,\textsuperscript{66} the intermediate situation of Coche
would have separated the fleet and army of Julian; and the rash
attempt of steering against the current of the Tigris, and
forcing their way through the midst of a hostile capital, must
have been attended with the total destruction of the Roman navy.
The prudence of the emperor foresaw the danger, and provided the
remedy. As he had minutely studied the operations of Trajan in
the same country, he soon recollected that his warlike
predecessor had dug a new and navigable canal, which, leaving
Coche on the right hand, conveyed the waters of the Nahar-Malcha
into the river Tigris, at some distance \textit{above} the cities. From
the information of the peasants, Julian ascertained the vestiges
of this ancient work, which were almost obliterated by design or
accident. By the indefatigable labor of the soldiers, a broad and
deep channel was speedily prepared for the reception of the
Euphrates. A strong dike was constructed to interrupt the
ordinary current of the Nahar-Malcha: a flood of waters rushed
impetuously into their new bed; and the Roman fleet, steering
their triumphant course into the Tigris, derided the vain and
ineffectual barriers which the Persians of Ctesiphon had erected
to oppose their passage.

\pagenote[65]{M. d’Anville, (Mém. de l’Académie des Inscriptions,
tom. xxxviii p. 246-259) has ascertained the true position and
distance of Babylon, Seleucia, Ctesiphon, Bagdad, \&c. The Roman
traveller, Pietro della Valle, (tom. i. lett. xvii. p. 650-780,)
seems to be the most intelligent spectator of that famous
province. He is a gentleman and a scholar, but intolerably vain
and prolix.}

\pagenote[66]{The Royal Canal (\textit{Nahar-Malcha}) might be
successively restored, altered, divided, \&c., (Cellarius,
Geograph. Antiq. tom. ii. p. 453;) and these changes may serve to
explain the seeming contradictions of antiquity. In the time of
Julian, it must have fallen into the Euphrates \textit{below}
Ctesiphon.}

As it became necessary to transport the Roman army over the
Tigris, another labor presented itself, of less toil, but of more
danger, than the preceding expedition. The stream was broad and
rapid; the ascent steep and difficult; and the intrenchments
which had been formed on the ridge of the opposite bank, were
lined with a numerous army of heavy cuirrasiers, dexterous
archers, and huge elephants; who (according to the extravagant
hyperbole of Libanius) could trample with the same ease a field
of corn, or a legion of Romans.\textsuperscript{67} In the presence of such an
enemy, the construction of a bridge was impracticable; and the
intrepid prince, who instantly seized the only possible
expedient, concealed his design, till the moment of execution,
from the knowledge of the Barbarians, of his own troops, and even
of his generals themselves. Under the specious pretence of
examining the state of the magazines, fourscore vessels\textsuperscript{6711} were
gradually unladen; and a select detachment, apparently destined
for some secret expedition, was ordered to stand to their arms on
the first signal. Julian disguised the silent anxiety of his own
mind with smiles of confidence and joy; and amused the hostile
nations with the spectacle of military games, which he
insultingly celebrated under the walls of Coche. The day was
consecrated to pleasure; but, as soon as the hour of supper was
passed, the emperor summoned the generals to his tent, and
acquainted them that he had fixed that night for the passage of
the Tigris. They stood in silent and respectful astonishment;
but, when the venerable Sallust assumed the privilege of his age
and experience, the rest of the chiefs supported with freedom the
weight of his prudent remonstrances.\textsuperscript{68} Julian contented himself
with observing, that conquest and safety depended on the attempt;
that instead of diminishing, the number of their enemies would be
increased, by successive reenforcements; and that a longer delay
would neither contract the breadth of the stream, nor level the
height of the bank. The signal was instantly given, and obeyed;
the most impatient of the legionaries leaped into five vessels
that lay nearest to the bank; and as they plied their oars with
intrepid diligence, they were lost, after a few moments, in the
darkness of the night. A flame arose on the opposite side; and
Julian, who too clearly understood that his foremost vessels, in
attempting to land, had been fired by the enemy, dexterously
converted their extreme danger into a presage of victory. “Our
fellow-soldiers,” he eagerly exclaimed, “are already masters of
the bank; see—they make the appointed signal; let us hasten to
emulate and assist their courage.” The united and rapid motion of
a great fleet broke the violence of the current, and they reached
the eastern shore of the Tigris with sufficient speed to
extinguish the flames, and rescue their adventurous companions.
The difficulties of a steep and lofty ascent were increased by
the weight of armor, and the darkness of the night. A shower of
stones, darts, and fire, was incessantly discharged on the heads
of the assailants; who, after an arduous struggle, climbed the
bank and stood victorious upon the rampart. As soon as they
possessed a more equal field, Julian, who, with his light
infantry, had led the attack,\textsuperscript{69} darted through the ranks a
skilful and experienced eye: his bravest soldiers, according to
the precepts of Homer,\textsuperscript{70} were distributed in the front and rear:
and all the trumpets of the Imperial army sounded to battle. The
Romans, after sending up a military shout, advanced in measured
steps to the animating notes of martial music; launched their
formidable javelins; and rushed forwards with drawn swords, to
deprive the Barbarians, by a closer onset, of the advantage of
their missile weapons. The whole engagement lasted above twelve
hours; till the gradual retreat of the Persians was changed into
a disorderly flight, of which the shameful example was given by
the principal leader, and the Surenas himself. They were pursued
to the gates of Ctesiphon; and the conquerors might have entered
the dismayed city,\textsuperscript{71} if their general, Victor, who was
dangerously wounded with an arrow, had not conjured them to
desist from a rash attempt, which must be fatal, if it were not
successful. On \textit{their} side, the Romans acknowledged the loss of
only seventy-five men; while they affirmed, that the Barbarians
had left on the field of battle two thousand five hundred, or
even six thousand, of their bravest soldiers. The spoil was such
as might be expected from the riches and luxury of an Oriental
camp; large quantities of silver and gold, splendid arms and
trappings, and beds and tables of massy silver.\textsuperscript{7111} The
victorious emperor distributed, as the rewards of valor, some
honorable gifts, civic, and mural, and naval crowns; which he,
and perhaps he alone, esteemed more precious than the wealth of
Asia. A solemn sacrifice was offered to the god of war, but the
appearances of the victims threatened the most inauspicious
events; and Julian soon discovered, by less ambiguous signs, that
he had now reached the term of his prosperity.\textsuperscript{72}

\pagenote[67]{Rien n’est beau que le vrai; a maxim which should
be inscribed on the desk of every rhetorician.}

\pagenote[6711]{This is a mistake; each vessel (according to
Zosimus two, according to Ammianus five) had eighty men. Amm.
xxiv. 6, with Wagner’s note. Gibbon must have read \textit{octogenas}
for \textit{octogenis}. The five vessels selected for this service were
remarkably large and strong provision transports. The strength of
the fleet remained with Julian to carry over the army—M.}

\pagenote[68]{Libanius alludes to the most powerful of the
generals. I have ventured to name \textit{Sallust}. Ammianus says, of
all the leaders, quod acri metû territ acrimetu territi duces
concordi precatû precaut fieri prohibere tentarent. * Note: It is
evident that Gibbon has mistaken the sense of Libanius; his words
can only apply to a commander of a detachment, not to so eminent
a person as the Præfect of the East. St. Martin, iii. 313.—M.}

\pagenote[69]{Hinc Imperator.... (says Ammianus) ipse cum levis
armaturæ auxiliis per prima postremaque discurrens, \&c. Yet
Zosimus, his friend, does not allow him to pass the river till
two days after the battle.}

\pagenote[70]{Secundum Homericam dispositionem. A similar
disposition is ascribed to the wise Nestor, in the fourth book of
the Iliad; and Homer was never absent from the mind of Julian.}

\pagenote[71]{Persas terrore subito miscuerunt, versisque
agminibus totius gentis, apertas Ctesiphontis portas victor miles
intrâsset, ni major prædarum occasio fuisset, quam cura victoriæ,
(Sextus Rufus de Provinciis c. 28.) Their avarice might dispose
them to hear the advice of Victor.}

\pagenote[7111]{The suburbs of Ctesiphon, according to a new
fragment of Eunapius, were so full of provisions, that the
soldiers were in danger of suffering from excess. Mai, p. 260.
Eunapius in Niebuhr. Nov. Byz. Coll. 68. Julian exhibited warlike
dances and games in his camp to recreate the soldiers Ibid.—M.}

\pagenote[72]{The labor of the canal, the passage of the Tigris,
and the victory, are described by Ammianus, (xxiv. 5, 6,)
Libanius, (Orat. Parent. c. 124-128, p. 347-353,) Greg.
Nazianzen, (Orat. iv. p. 115,) Zosimus, (l. iii. p. 181-183,) and
Sextus Rufus, (de Provinciis, c. 28.)}

On the second day after the battle, the domestic guards, the
Jovians and Herculians, and the remaining troops, which composed
near two thirds of the whole army, were securely wafted over the
Tigris.\textsuperscript{73} While the Persians beheld from the walls of Ctesiphon
the desolation of the adjacent country, Julian cast many an
anxious look towards the North, in full expectation, that as he
himself had victoriously penetrated to the capital of Sapor, the
march and junction of his lieutenants, Sebastian and Procopius,
would be executed with the same courage and diligence. His
expectations were disappointed by the treachery of the Armenian
king, who permitted, and most probably directed, the desertion of
his auxiliary troops from the camp of the Romans;\textsuperscript{74} and by the
dissensions of the two generals, who were incapable of forming or
executing any plan for the public service. When the emperor had
relinquished the hope of this important reenforcement, he
condescended to hold a council of war, and approved, after a full
debate, the sentiment of those generals, who dissuaded the siege
of Ctesiphon, as a fruitless and pernicious undertaking. It is
not easy for us to conceive, by what arts of fortification a city
thrice besieged and taken by the predecessors of Julian could be
rendered impregnable against an army of sixty thousand Romans,
commanded by a brave and experienced general, and abundantly
supplied with ships, provisions, battering engines, and military
stores. But we may rest assured, from the love of glory, and
contempt of danger, which formed the character of Julian, that he
was not discouraged by any trivial or imaginary obstacles.\textsuperscript{75} At
the very time when he declined the siege of Ctesiphon, he
rejected, with obstinacy and disdain, the most flattering offers
of a negotiation of peace. Sapor, who had been so long accustomed
to the tardy ostentation of Constantius, was surprised by the
intrepid diligence of his successor. As far as the confines of
India and Scythia, the satraps of the distant provinces were
ordered to assemble their troops, and to march, without delay, to
the assistance of their monarch. But their preparations were
dilatory, their motions slow; and before Sapor could lead an army
into the field, he received the melancholy intelligence of the
devastation of Assyria, the ruin of his palaces, and the
slaughter of his bravest troops, who defended the passage of the
Tigris. The pride of royalty was humbled in the dust; he took his
repasts on the ground; and the disorder of his hair expressed the
grief and anxiety of his mind. Perhaps he would not have refused
to purchase, with one half of his kingdom, the safety of the
remainder; and he would have gladly subscribed himself, in a
treaty of peace, the faithful and dependent ally of the Roman
conqueror. Under the pretence of private business, a minister of
rank and confidence was secretly despatched to embrace the knees
of Hormisdas, and to request, in the language of a suppliant,
that he might be introduced into the presence of the emperor. The
Sassanian prince, whether he listened to the voice of pride or
humanity, whether he consulted the sentiments of his birth, or
the duties of his situation, was equally inclined to promote a
salutary measure, which would terminate the calamities of Persia,
and secure the triumph of Rome. He was astonished by the
inflexible firmness of a hero, who remembered, most unfortunately
for himself and for his country, that Alexander had uniformly
rejected the propositions of Darius. But as Julian was sensible,
that the hope of a safe and honorable peace might cool the ardor
of his troops, he earnestly requested that Hormisdas would
privately dismiss the minister of Sapor, and conceal this
dangerous temptation from the knowledge of the camp.\textsuperscript{76}

\pagenote[73]{The fleet and army were formed in three divisions,
of which the first only had passed during the night.}

\pagenote[74]{Moses of Chorene (Hist. Armen. l. iii. c. 15, p.
246) supplies us with a national tradition, and a spurious
letter. I have borrowed only the leading circumstance, which is
consistent with truth, probability, and Libanius, (Orat. Parent.
c. 131, p. 355.)}

\pagenote[75]{Civitas inexpugnabilis, facinus audax et
importunum. Ammianus, xxiv. 7. His fellow-soldier, Eutropius,
turns aside from the difficulty, Assyriamque populatus, castra
apud Ctesiphontem stativa aliquandiu habuit: remeansbue victor,
\&c. x. 16. Zosimus is artful or ignorant, and Socrates
inaccurate.}

\pagenote[76]{Libanius, Orat. Parent. c. 130, p. 354, c. 139, p.
361. Socrates, l. iii. c. 21. The ecclesiastical historian
imputes the refusal of peace to the advice of Maximus. Such
advice was unworthy of a philosopher; but the philosopher was
likewise a magician, who flattered the hopes and passions of his
master.}

\section{Part \thesection.}

The honor, as well as interest, of Julian, forbade him to consume
his time under the impregnable walls of Ctesiphon and as often as
he defied the Barbarians, who defended the city, to meet him on
the open plain, they prudently replied, that if he desired to
exercise his valor, he might seek the army of the Great King. He
felt the insult, and he accepted the advice. Instead of confining
his servile march to the banks of the Euphrates and Tigris, he
resolved to imitate the adventurous spirit of Alexander, and
boldly to advance into the inland provinces, till he forced his
rival to contend with him, perhaps in the plains of Arbela, for
the empire of Asia. The magnanimity of Julian was applauded and
betrayed, by the arts of a noble Persian, who, in the cause of
his country, had generously submitted to act a part full of
danger, of falsehood, and of shame.\textsuperscript{77} With a train of faithful
followers, he deserted to the Imperial camp; exposed, in a
specious tale, the injuries which he had sustained; exaggerated
the cruelty of Sapor, the discontent of the people, and the
weakness of the monarchy; and confidently offered himself as the
hostage and guide of the Roman march. The most rational grounds
of suspicion were urged, without effect, by the wisdom and
experience of Hormisdas; and the credulous Julian, receiving the
traitor into his bosom, was persuaded to issue a hasty order,
which, in the opinion of mankind, appeared to arraign his
prudence, and to endanger his safety. He destroyed, in a single
hour, the whole navy, which had been transported above five
hundred miles, at so great an expense of toil, of treasure, and
of blood. Twelve, or, at the most, twenty-two small vessels were
saved, to accompany, on carriages, the march of the army, and to
form occasional bridges for the passage of the rivers. A supply
of twenty days’ provisions was reserved for the use of the
soldiers; and the rest of the magazines, with a fleet of eleven
hundred vessels, which rode at anchor in the Tigris, were
abandoned to the flames, by the absolute command of the emperor.
The Christian bishops, Gregory and Augustin, insult the madness
of the Apostate, who executed, with his own hands, the sentence
of divine justice. Their authority, of less weight, perhaps, in a
military question, is confirmed by the cool judgment of an
experienced soldier, who was himself spectator of the
conflagration, and who could not disapprove the reluctant murmurs
of the troops.\textsuperscript{78} Yet there are not wanting some specious, and
perhaps solid, reasons, which might justify the resolution of
Julian. The navigation of the Euphrates never ascended above
Babylon, nor that of the Tigris above Opis.\textsuperscript{79} The distance of
the last-mentioned city from the Roman camp was not very
considerable: and Julian must soon have renounced the vain and
impracticable attempt of forcing upwards a great fleet against
the stream of a rapid river,\textsuperscript{80} which in several places was
embarrassed by natural or artificial cataracts.\textsuperscript{81} The power of
sails and oars was insufficient; it became necessary to tow the
ships against the current of the river; the strength of twenty
thousand soldiers was exhausted in this tedious and servile
labor, and if the Romans continued to march along the banks of
the Tigris, they could only expect to return home without
achieving any enterprise worthy of the genius or fortune of their
leader. If, on the contrary, it was advisable to advance into the
inland country, the destruction of the fleet and magazines was
the only measure which could save that valuable prize from the
hands of the numerous and active troops which might suddenly be
poured from the gates of Ctesiphon. Had the arms of Julian been
victorious, we should now admire the conduct, as well as the
courage, of a hero, who, by depriving his soldiers of the hopes
of a retreat, left them only the alternative of death or
conquest.\textsuperscript{82}

\pagenote[77]{The arts of this new Zopyrus (Greg. Nazianzen,
Orat. iv. p. 115, 116) may derive some credit from the testimony
of two abbreviators, (Sextus Rufus and Victor,) and the casual
hints of Libanius (Orat. Parent. c. 134, p. 357) and Ammianus,
(xxiv. 7.) The course of genuine history is interrupted by a most
unseasonable chasm in the text of Ammianus.}

\pagenote[78]{See Ammianus, (xxiv. 7,) Libanius, (Orat.
Parentalis, c. 132, 133, p. 356, 357,) Zosimus, (l. iii. p. 183,)
Zonaras, (tom. ii. l. xiii. p. 26) Gregory, (Orat. iv. p. 116,)
and Augustin, (de Civitate Dei, l. iv. c. 29, l. v. c. 21.) Of
these Libanius alone attempts a faint apology for his hero; who,
according to Ammianus, pronounced his own condemnation by a tardy
and ineffectual attempt to extinguish the flames.}

\pagenote[79]{Consult Herodotus, (l. i. c. 194,) Strabo, (l. xvi.
p. 1074,) and Tavernier, (part i. l. ii. p. 152.)}

\pagenote[80]{A celeritate Tigris incipit vocari, ita appellant
Medi sagittam. Plin. Hist. Natur. vi. 31.}

\pagenote[81]{One of these dikes, which produces an artificial
cascade or cataract, is described by Tavernier (part i. l. ii. p.
226) and Thevenot, (part ii. l. i. p. 193.) The Persians, or
Assyrians, labored to interrupt the navigation of the river,
(Strabo, l. xv. p. 1075. D’Anville, l’Euphrate et le Tigre, p.
98, 99.)}

\pagenote[82]{Recollect the successful and applauded rashness of
Agathocles and Cortez, who burnt their ships on the coast of
Africa and Mexico.}

The cumbersome train of artillery and wagons, which retards the
operations of a modern army, were in a great measure unknown in
the camps of the Romans.\textsuperscript{83} Yet, in every age, the subsistence of
sixty thousand men must have been one of the most important cares
of a prudent general; and that subsistence could only be drawn
from his own or from the enemy’s country. Had it been possible
for Julian to maintain a bridge of communication on the Tigris,
and to preserve the conquered places of Assyria, a desolated
province could not afford any large or regular supplies, in a
season of the year when the lands were covered by the inundation
of the Euphrates,\textsuperscript{84} and the unwholesome air was darkened with
swarms of innumerable insects.\textsuperscript{85} The appearance of the hostile
country was far more inviting. The extensive region that lies
between the River Tigris and the mountains of Media, was filled
with villages and towns; and the fertile soil, for the most part,
was in a very improved state of cultivation. Julian might expect,
that a conqueror, who possessed the two forcible instruments of
persuasion, steel and gold, would easily procure a plentiful
subsistence from the fears or avarice of the natives. But, on the
approach of the Romans, the rich and smiling prospect was
instantly blasted. Wherever they moved, the inhabitants deserted
the open villages, and took shelter in the fortified towns; the
cattle was driven away; the grass and ripe corn were consumed
with fire; and, as soon as the flames had subsided which
interrupted the march of Julian, he beheld the melancholy face of
a smoking and naked desert. This desperate but effectual method
of defence can only be executed by the enthusiasm of a people who
prefer their independence to their property; or by the rigor of
an arbitrary government, which consults the public safety without
submitting to their inclinations the liberty of choice. On the
present occasion the zeal and obedience of the Persians seconded
the commands of Sapor; and the emperor was soon reduced to the
scanty stock of provisions, which continually wasted in his
hands. Before they were entirely consumed, he might still have
reached the wealthy and unwarlike cities of Ecbatana or Susa, by
the effort of a rapid and well-directed march;\textsuperscript{86} but he was
deprived of this last resource by his ignorance of the roads, and
by the perfidy of his guides. The Romans wandered several days in
the country to the eastward of Bagdad; the Persian deserter, who
had artfully led them into the snare, escaped from their
resentment; and his followers, as soon as they were put to the
torture, confessed the secret of the conspiracy. The visionary
conquests of Hyrcania and India, which had so long amused, now
tormented, the mind of Julian. Conscious that his own imprudence
was the cause of the public distress, he anxiously balanced the
hopes of safety or success, without obtaining a satisfactory
answer, either from gods or men. At length, as the only
practicable measure, he embraced the resolution of directing his
steps towards the banks of the Tigris, with the design of saving
the army by a hasty march to the confines of Corduene; a fertile
and friendly province, which acknowledged the sovereignty of
Rome. The desponding troops obeyed the signal of the retreat,
only seventy days after they had passed the Chaboras, with the
sanguine expectation of subverting the throne of Persia.\textsuperscript{87}

\pagenote[83]{See the judicious reflections of the author of the
Essai sur la Tactique, tom. ii. p. 287-353, and the learned
remarks of M. Guichardt Nouveaux Mémoires Militaires, tom. i. p.
351-382, on the baggage and subsistence of the Roman armies.}

\pagenote[84]{The Tigris rises to the south, the Euphrates to the
north, of the Armenian mountains. The former overflows in March,
the latter in July. These circumstances are well explained in the
Geographical Dissertation of Foster, inserted in Spelman’s
Expedition of Cyras, vol. ii. p. 26.}

\pagenote[85]{Ammianus (xxiv. 8) describes, as he had felt, the
inconveniency of the flood, the heat, and the insects. The lands
of Assyria, oppressed by the Turks, and ravaged by the Curds or
Arabs, yield an increase of ten, fifteen, and twenty fold, for
the seed which is cast into the ground by the wretched and
unskillful husbandmen. Voyage de Niebuhr, tom. ii. p. 279, 285.}

\pagenote[86]{Isidore of Charax (Mansion. Parthic. p. 5, 6, in
Hudson, Geograph. Minor. tom. ii.) reckons 129 schæni from
Seleucia, and Thevenot, (part i. l. i. ii. p. 209-245,) 128 hours
of march from Bagdad to Ecbatana, or Hamadan. These measures
cannot exceed an ordinary parasang, or three Roman miles.}

\pagenote[87]{The march of Julian from Ctesiphon is
circumstantially, but not clearly, described by Ammianus, (xxiv.
7, 8,) Libanius, (Orat. Parent. c. 134, p. 357,) and Zosimus, (l.
iii. p. 183.) The two last seem ignorant that their conqueror was
retreating; and Libanius absurdly confines him to the banks of
the Tigris.}

As long as the Romans seemed to advance into the country, their
march was observed and insulted from a distance, by several
bodies of Persian cavalry; who, showing themselves sometimes in
loose, and sometimes in close order, faintly skirmished with the
advanced guards. These detachments were, however, supported by a
much greater force; and the heads of the columns were no sooner
pointed towards the Tigris than a cloud of dust arose on the
plain. The Romans, who now aspired only to the permission of a
safe and speedy retreat, endeavored to persuade themselves, that
this formidable appearance was occasioned by a troop of wild
asses, or perhaps by the approach of some friendly Arabs. They
halted, pitched their tents, fortified their camp, passed the
whole night in continual alarms; and discovered at the dawn of
day, that they were surrounded by an army of Persians. This army,
which might be considered only as the van of the Barbarians, was
soon followed by the main body of cuirassiers, archers, and
elephants, commanded by Meranes, a general of rank and
reputation. He was accompanied by two of the king’s sons, and
many of the principal satraps; and fame and expectation
exaggerated the strength of the remaining powers, which slowly
advanced under the conduct of Sapor himself. As the Romans
continued their march, their long array, which was forced to bend
or divide, according to the varieties of the ground, afforded
frequent and favorable opportunities to their vigilant enemies.
The Persians repeatedly charged with fury; they were repeatedly
repulsed with firmness; and the action at Maronga, which almost
deserved the name of a battle, was marked by a considerable loss
of satraps and elephants, perhaps of equal value in the eyes of
their monarch. These splendid advantages were not obtained
without an adequate slaughter on the side of the Romans: several
officers of distinction were either killed or wounded; and the
emperor himself, who, on all occasions of danger, inspired and
guided the valor of his troops, was obliged to expose his person,
and exert his abilities. The weight of offensive and defensive
arms, which still constituted the strength and safety of the
Romans, disabled them from making any long or effectual pursuit;
and as the horsemen of the East were trained to dart their
javelins, and shoot their arrows, at full speed, and in every
possible direction,\textsuperscript{88} the cavalry of Persia was never more
formidable than in the moment of a rapid and disorderly flight.
But the most certain and irreparable loss of the Romans was that
of time. The hardy veterans, accustomed to the cold climate of
Gaul and Germany, fainted under the sultry heat of an Assyrian
summer; their vigor was exhausted by the incessant repetition of
march and combat; and the progress of the army was suspended by
the precautions of a slow and dangerous retreat, in the presence
of an active enemy. Every day, every hour, as the supply
diminished, the value and price of subsistence increased in the
Roman camp.\textsuperscript{89} Julian, who always contented himself with such
food as a hungry soldier would have disdained, distributed, for
the use of the troops, the provisions of the Imperial household,
and whatever could be spared, from the sumpter-horses, of the
tribunes and generals. But this feeble relief served only to
aggravate the sense of the public distress; and the Romans began
to entertain the most gloomy apprehensions that, before they
could reach the frontiers of the empire, they should all perish,
either by famine, or by the sword of the Barbarians.\textsuperscript{90}

\pagenote[88]{Chardin, the most judicious of modern travellers,
describes (tom. ii. p. 57, 58, \&c., edit. in 4to.) the education
and dexterity of the Persian horsemen. Brissonius (de Regno
Persico, p. 650 651, \&c.,) has collected the testimonies of
antiquity.}

\pagenote[89]{In Mark Antony’s retreat, an attic chœnix sold for
fifty drachmæ, or, in other words, a pound of flour for twelve or
fourteen shillings barley bread was sold for its weight in
silver. It is impossible to peruse the interesting narrative of
Plutarch, (tom. v. p. 102-116,) without perceiving that Mark
Antony and Julian were pursued by the same enemies, and involved
in the same distress.}

\pagenote[90]{Ammian. xxiv. 8, xxv. 1. Zosimus, l. iii. p. 184,
185, 186. Libanius, Orat. Parent. c. 134, 135, p. 357, 358, 359.
The sophist of Antioch appears ignorant that the troops were
hungry.}

While Julian struggled with the almost insuperable difficulties
of his situation, the silent hours of the night were still
devoted to study and contemplation. Whenever he closed his eyes
in short and interrupted slumbers, his mind was agitated with
painful anxiety; nor can it be thought surprising, that the
Genius of the empire should once more appear before him, covering
with a funeral veil his head, and his horn of abundance, and
slowly retiring from the Imperial tent. The monarch started from
his couch, and stepping forth to refresh his wearied spirits with
the coolness of the midnight air, he beheld a fiery meteor, which
shot athwart the sky, and suddenly vanished. Julian was convinced
that he had seen the menacing countenance of the god of war;\textsuperscript{91}
the council which he summoned, of Tuscan Haruspices,\textsuperscript{92}
unanimously pronounced that he should abstain from action; but on
this occasion, necessity and reason were more prevalent than
superstition; and the trumpets sounded at the break of day. The
army marched through a hilly country; and the hills had been
secretly occupied by the Persians. Julian led the van with the
skill and attention of a consummate general; he was alarmed by
the intelligence that his rear was suddenly attacked. The heat of
the weather had tempted him to lay aside his cuirass; but he
snatched a shield from one of his attendants, and hastened, with
a sufficient reenforcement, to the relief of the rear-guard. A
similar danger recalled the intrepid prince to the defence of the
front; and, as he galloped through the columns, the centre of the
left was attacked, and almost overpowered by the furious charge
of the Persian cavalry and elephants. This huge body was soon
defeated, by the well-timed evolution of the light infantry, who
aimed their weapons, with dexterity and effect, against the backs
of the horsemen, and the legs of the elephants. The Barbarians
fled; and Julian, who was foremost in every danger, animated the
pursuit with his voice and gestures. His trembling guards,
scattered and oppressed by the disorderly throng of friends and
enemies, reminded their fearless sovereign that he was without
armor; and conjured him to decline the fall of the impending
ruin. As they exclaimed,\textsuperscript{93} a cloud of darts and arrows was
discharged from the flying squadrons; and a javelin, after razing
the skin of his arm, transpierced the ribs, and fixed in the
inferior part of the liver. Julian attempted to draw the deadly
weapon from his side; but his fingers were cut by the sharpness
of the steel, and he fell senseless from his horse. His guards
flew to his relief; and the wounded emperor was gently raised
from the ground, and conveyed out of the tumult of the battle
into an adjacent tent. The report of the melancholy event passed
from rank to rank; but the grief of the Romans inspired them with
invincible valor, and the desire of revenge. The bloody and
obstinate conflict was maintained by the two armies, till they
were separated by the total darkness of the night. The Persians
derived some honor from the advantage which they obtained against
the left wing, where Anatolius, master of the offices, was slain,
and the præfect Sallust very narrowly escaped. But the event of
the day was adverse to the Barbarians. They abandoned the field;
their two generals, Meranes and Nohordates,\textsuperscript{94} fifty nobles or
satraps, and a multitude of their bravest soldiers; and the
success of the Romans, if Julian had survived, might have been
improved into a decisive and useful victory.

\pagenote[91]{Ammian. xxv. 2. Julian had sworn in a passion,
nunquam se Marti sacra facturum, (xxiv. 6.) Such whimsical
quarrels were not uncommon between the gods and their insolent
votaries; and even the prudent Augustus, after his fleet had been
twice shipwrecked, excluded Neptune from the honors of public
processions. See Hume’s Philosophical Reflections. Essays, vol.
ii. p. 418.}

\pagenote[92]{They still retained the monopoly of the vain but
lucrative science, which had been invented in Hetruria; and
professed to derive their knowledge of signs and omens from the
ancient books of Tarquitius, a Tuscan sage.}

\pagenote[93]{Clambant hinc inde \textit{candidati} (see the note of
Valesius) quos terror, ut fugientium molem tanquam ruinam male
compositi culminis declinaret. Ammian. xxv 3.}

\pagenote[94]{Sapor himself declared to the Romans, that it was
his practice to comfort the families of his deceased satraps, by
sending them, as a present, the heads of the guards and officers
who had not fallen by their master’s side. Libanius, de nece
Julian. ulcis. c. xiii. p. 163.}

The first words that Julian uttered, after his recovery from the
fainting fit into which he had been thrown by loss of blood, were
expressive of his martial spirit. He called for his horse and
arms, and was impatient to rush into the battle. His remaining
strength was exhausted by the painful effort; and the surgeons,
who examined his wound, discovered the symptoms of approaching
death. He employed the awful moments with the firm temper of a
hero and a sage; the philosophers who had accompanied him in this
fatal expedition, compared the tent of Julian with the prison of
Socrates; and the spectators, whom duty, or friendship, or
curiosity, had assembled round his couch, listened with
respectful grief to the funeral oration of their dying emperor.\textsuperscript{95}
“Friends and fellow-soldiers, the seasonable period of my
departure is now arrived, and I discharge, with the cheerfulness
of a ready debtor, the demands of nature. I have learned from
philosophy, how much the soul is more excellent than the body;
and that the separation of the nobler substance should be the
subject of joy, rather than of affliction. I have learned from
religion, that an early death has often been the reward of piety;\textsuperscript{96}
and I accept, as a favor of the gods, the mortal stroke that
secures me from the danger of disgracing a character, which has
hitherto been supported by virtue and fortitude. I die without
remorse, as I have lived without guilt. I am pleased to reflect
on the innocence of my private life; and I can affirm with
confidence, that the supreme authority, that emanation of the
Divine Power, has been preserved in my hands pure and immaculate.
Detesting the corrupt and destructive maxims of despotism, I have
considered the happiness of the people as the end of government.
Submitting my actions to the laws of prudence, of justice, and of
moderation, I have trusted the event to the care of Providence.
Peace was the object of my counsels, as long as peace was
consistent with the public welfare; but when the imperious voice
of my country summoned me to arms, I exposed my person to the
dangers of war, with the clear foreknowledge (which I had
acquired from the art of divination) that I was destined to fall
by the sword. I now offer my tribute of gratitude to the Eternal
Being, who has not suffered me to perish by the cruelty of a
tyrant, by the secret dagger of conspiracy, or by the slow
tortures of lingering disease. He has given me, in the midst of
an honorable career, a splendid and glorious departure from this
world; and I hold it equally absurd, equally base, to solicit, or
to decline, the stroke of fate. This much I have attempted to
say; but my strength fails me, and I feel the approach of death.
I shall cautiously refrain from any word that may tend to
influence your suffrages in the election of an emperor. My choice
might be imprudent or injudicious; and if it should not be
ratified by the consent of the army, it might be fatal to the
person whom I should recommend. I shall only, as a good citizen,
express my hopes, that the Romans may be blessed with the
government of a virtuous sovereign.” After this discourse, which
Julian pronounced in a firm and gentle tone of voice, he
distributed, by a military testament,\textsuperscript{97} the remains of his
private fortune; and making some inquiry why Anatolius was not
present, he understood, from the answer of Sallust, that
Anatolius was killed; and bewailed, with amiable inconsistency,
the loss of his friend. At the same time he reproved the
immoderate grief of the spectators; and conjured them not to
disgrace, by unmanly tears, the fate of a prince, who in a few
moments would be united with heaven, and with the stars.\textsuperscript{98} The
spectators were silent; and Julian entered into a metaphysical
argument with the philosophers Priscus and Maximus, on the nature
of the soul. The efforts which he made, of mind as well as body,
most probably hastened his death. His wound began to bleed with
fresh violence; his respiration was embarrassed by the swelling
of the veins; he called for a draught of cold water, and, as soon
as he had drank it, expired without pain, about the hour of
midnight. Such was the end of that extraordinary man, in the
thirty-second year of his age, after a reign of one year and
about eight months, from the death of Constantius. In his last
moments he displayed, perhaps with some ostentation, the love of
virtue and of fame, which had been the ruling passions of his
life.\textsuperscript{99}

\pagenote[95]{The character and situation of Julian might
countenance the suspicion that he had previously composed the
elaborate oration, which Ammianus heard, and has transcribed. The
version of the Abbé de la Bleterie is faithful and elegant. I
have followed him in expressing the Platonic idea of emanations,
which is darkly insinuated in the original.}

\pagenote[96]{Herodotus (l. i. c. 31,) has displayed that
doctrine in an agreeable tale. Yet the Jupiter, (in the 16th book
of the Iliad,) who laments with tears of blood the death of
Sarpedon his son, had a very imperfect notion of happiness or
glory beyond the grave.}

\pagenote[97]{The soldiers who made their verbal or nuncupatory
testaments, upon actual service, (in procinctu,) were exempted
from the formalities of the Roman law. See Heineccius, (Antiquit.
Jur. Roman. tom. i. p. 504,) and Montesquieu, (Esprit des Loix,
l. xxvii.)}

\pagenote[98]{This union of the human soul with the divine
æthereal substance of the universe, is the ancient doctrine of
Pythagoras and Plato: but it seems to exclude any personal or
conscious immortality. See Warburton’s learned and rational
observations. Divine Legation, vol ii. p. 199-216.}

\pagenote[99]{The whole relation of the death of Julian is given
by Ammianus, (xxv. 3,) an intelligent spectator. Libanius, who
turns with horror from the scene, has supplied some
circumstances, (Orat. Parental. c 136-140, p. 359-362.) The
calumnies of Gregory, and the legends of more recent saints, may
now be \textit{silently} despised. * Note: A very remarkable fragment of
Eunapius describes, not without spirit, the struggle between the
terror of the army on account of their perilous situation, and
their grief for the death of Julian. “Even the vulgar felt that
they would soon provide a general, but such a general as Julian
they would never find, even though a god in the form of
man—Julian, who, with a mind equal to the divinity, triumphed
over the evil propensities of human nature,—* * who held commerce
with immaterial beings while yet in the material body—who
condescended to rule because a ruler was necessary to the welfare
of mankind.” Mai, Nov. Coll. ii. 261. Eunapius in Niebuhr, 69.}

The triumph of Christianity, and the calamities of the empire,
may, in some measure, be ascribed to Julian himself, who had
neglected to secure the future execution of his designs, by the
timely and judicious nomination of an associate and successor.
But the royal race of Constantius Chlorus was reduced to his own
person; and if he entertained any serious thoughts of investing
with the purple the most worthy among the Romans, he was diverted
from his resolution by the difficulty of the choice, the jealousy
of power, the fear of ingratitude, and the natural presumption of
health, of youth, and of prosperity. His unexpected death left
the empire without a master, and without an heir, in a state of
perplexity and danger, which, in the space of fourscore years,
had never been experienced, since the election of Diocletian. In
a government which had almost forgotten the distinction of pure
and noble blood, the superiority of birth was of little moment;
the claims of official rank were accidental and precarious; and
the candidates, who might aspire to ascend the vacant throne
could be supported only by the consciousness of personal merit,
or by the hopes of popular favor. But the situation of a famished
army, encompassed on all sides by a host of Barbarians, shortened
the moments of grief and deliberation. In this scene of terror
and distress, the body of the deceased prince, according to his
own directions, was decently embalmed; and, at the dawn of day,
the generals convened a military senate, at which the commanders
of the legions, and the officers both of cavalry and infantry,
were invited to assist. Three or four hours of the night had not
passed away without some secret cabals; and when the election of
an emperor was proposed, the spirit of faction began to agitate
the assembly. Victor and Arinthæus collected the remains of the
court of Constantius; the friends of Julian attached themselves
to the Gallic chiefs, Dagalaiphus and Nevitta; and the most fatal
consequences might be apprehended from the discord of two
factions, so opposite in their character and interest, in their
maxims of government, and perhaps in their religious principles.
The superior virtues of Sallust could alone reconcile their
divisions, and unite their suffrages; and the venerable præfect
would immediately have been declared the successor of Julian, if
he himself, with sincere and modest firmness, had not alleged his
age and infirmities, so unequal to the weight of the diadem. The
generals, who were surprised and perplexed by his refusal, showed
some disposition to adopt the salutary advice of an inferior
officer,\textsuperscript{100} that they should act as they would have acted in the
absence of the emperor; that they should exert their abilities to
extricate the army from the present distress; and, if they were
fortunate enough to reach the confines of Mesopotamia, they
should proceed with united and deliberate counsels in the
election of a lawful sovereign. While they debated, a few voices
saluted Jovian, who was no more than \textit{first}\textsuperscript{101} of the
domestics, with the names of Emperor and Augustus. The tumultuary
acclamation\textsuperscript{10111} was instantly repeated by the guards who
surrounded the tent, and passed, in a few minutes, to the
extremities of the line. The new prince, astonished with his own
fortune was hastily invested with the Imperial ornaments, and
received an oath of fidelity from the generals, whose favor and
protection he so lately solicited. The strongest recommendation
of Jovian was the merit of his father, Count Varronian, who
enjoyed, in honorable retirement, the fruit of his long services.
In the obscure freedom of a private station, the son indulged his
taste for wine and women; yet he supported, with credit, the
character of a Christian\textsuperscript{102} and a soldier. Without being
conspicuous for any of the ambitious qualifications which excite
the admiration and envy of mankind, the comely person of Jovian,
his cheerful temper, and familiar wit, had gained the affection
of his fellow-soldiers; and the generals of both parties
acquiesced in a popular election, which had not been conducted by
the arts of their enemies. The pride of this unexpected elevation
was moderated by the just apprehension, that the same day might
terminate the life and reign of the new emperor. The pressing
voice of necessity was obeyed without delay; and the first orders
issued by Jovian, a few hours after his predecessor had expired,
were to prosecute a march, which could alone extricate the Romans
from their actual distress.\textsuperscript{103}

\pagenote[100]{Honoratior aliquis miles; perhaps Ammianus
himself. The modest and judicious historian describes the scene
of the election, at which he was undoubtedly present, (xxv. 5.)}

\pagenote[101]{The \textit{primus} or \textit{primicerius} enjoyed the dignity
of a senator, and though only a tribune, he ranked with the
military dukes. Cod. Theodosian. l. vi. tit. xxiv. These
privileges are perhaps more recent than the time of Jovian.}

\pagenote[10111]{The soldiers supposed that the acclamations
proclaimed the name of Julian, restored, as they fondly thought,
to health, not that of Jovian. loc.—M.}

\pagenote[102]{The ecclesiastical historians, Socrates, (l. iii.
c. 22,) Sozomen, (l. vi. c. 3,) and Theodoret, (l. iv. c. 1,)
ascribe to Jovian the merit of a confessor under the preceding
reign; and piously suppose that he refused the purple, till the
whole army unanimously exclaimed that they were Christians.
Ammianus, calmly pursuing his narrative, overthrows the legend by
a single sentence. Hostiis pro Joviano extisque inspectis,
pronuntiatum est, \&c., xxv. 6.}

\pagenote[103]{Ammianus (xxv. 10) has drawn from the life an
impartial portrait of Jovian; to which the younger Victor has
added some remarkable strokes. The Abbé de la Bleterie (Histoire
de Jovien, tom. i. p. 1-238) has composed an elaborate history of
his short reign; a work remarkably distinguished by elegance of
style, critical disquisition, and religious prejudice.}

\section{Part \thesection.}

The esteem of an enemy is most sincerely expressed by his fears;
and the degree of fear may be accurately measured by the joy with
which he celebrates his deliverance. The welcome news of the
death of Julian, which a deserter revealed to the camp of Sapor,
inspired the desponding monarch with a sudden confidence of
victory. He immediately detached the royal cavalry, perhaps the
ten thousand \textit{Immortals},\textsuperscript{104} to second and support the pursuit;
and discharged the whole weight of his united forces on the
rear-guard of the Romans. The rear-guard was thrown into
disorder; the renowned legions, which derived their titles from
Diocletian, and his warlike colleague, were broke and trampled
down by the elephants; and three tribunes lost their lives in
attempting to stop the flight of their soldiers. The battle was
at length restored by the persevering valor of the Romans; the
Persians were repulsed with a great slaughter of men and
elephants; and the army, after marching and fighting a long
summer’s day, arrived, in the evening, at Samara, on the banks of
the Tigris, about one hundred miles above Ctesiphon.\textsuperscript{105} On the
ensuing day, the Barbarians, instead of harassing the march,
attacked the camp, of Jovian; which had been seated in a deep and
sequestered valley. From the hills, the archers of Persia
insulted and annoyed the wearied legionaries; and a body of
cavalry, which had penetrated with desperate courage through the
Prætorian gate, was cut in pieces, after a doubtful conflict,
near the Imperial tent. In the succeeding night, the camp of
Carche was protected by the lofty dikes of the river; and the
Roman army, though incessantly exposed to the vexatious pursuit
of the Saracens, pitched their tents near the city of Dura,\textsuperscript{106}
four days after the death of Julian. The Tigris was still on
their left; their hopes and provisions were almost consumed; and
the impatient soldiers, who had fondly persuaded themselves that
the frontiers of the empire were not far distant, requested their
new sovereign, that they might be permitted to hazard the passage
of the river. With the assistance of his wisest officers, Jovian
endeavored to check their rashness; by representing, that if they
possessed sufficient skill and vigor to stem the torrent of a
deep and rapid stream, they would only deliver themselves naked
and defenceless to the Barbarians, who had occupied the opposite
banks, Yielding at length to their clamorous importunities, he
consented, with reluctance, that five hundred Gauls and Germans,
accustomed from their infancy to the waters of the Rhine and
Danube, should attempt the bold adventure, which might serve
either as an encouragement, or as a warning, for the rest of the
army. In the silence of the night, they swam the Tigris,
surprised an unguarded post of the enemy, and displayed at the
dawn of day the signal of their resolution and fortune. The
success of this trial disposed the emperor to listen to the
promises of his architects, who propose to construct a floating
bridge of the inflated skins of sheep, oxen, and goats, covered
with a floor of earth and fascines.\textsuperscript{107} Two important days were
spent in the ineffectual labor; and the Romans, who already
endured the miseries of famine, cast a look of despair on the
Tigris, and upon the Barbarians; whose numbers and obstinacy
increased with the distress of the Imperial army.\textsuperscript{108}

\pagenote[104]{Regius equitatus. It appears, from Irocopius, that
the Immortals, so famous under Cyrus and his successors, were
revived, if we may use that improper word, by the Sassanides.
Brisson de Regno Persico, p. 268, \&c.}

\pagenote[105]{The obscure villages of the inland country are
irrecoverably lost; nor can we name the field of battle where
Julian fell: but M. D’Anville has demonstrated the precise
situation of Sumere, Carche, and Dura, along the banks of the
Tigris, (Geographie Ancienne, tom. ii. p. 248 L’Euphrate et le
Tigre, p. 95, 97.) In the ninth century, Sumere, or Samara,
became, with a slight change of name, the royal residence of the
khalifs of the house of Abbas. * Note: Sormanray, called by the
Arabs Samira, where D’Anville placed Samara, is too much to the
south; and is a modern town built by Caliph Morasen.
Serra-man-rai means, in Arabic, it rejoices every one who sees
it. St. Martin, iii. 133.—M.}

\pagenote[106]{Dura was a fortified place in the wars of
Antiochus against the rebels of Media and Persia, (Polybius, l.
v. c. 48, 52, p. 548, 552 edit. Casaubon, in 8vo.)}

\pagenote[107]{A similar expedient was proposed to the leaders of
the ten thousand, and wisely rejected. Xenophon, Anabasis, l.
iii. p. 255, 256, 257. It appears, from our modern travellers,
that rafts floating on bladders perform the trade and navigation
of the Tigris.}

\pagenote[108]{The first military acts of the reign of Jovian are
related by Ammianus, (xxv. 6,) Libanius, (Orat. Parent. c. 146,
p. 364,) and Zosimus, (l. iii. p. 189, 190, 191.) Though we may
distrust the fairness of Libanius, the ocular testimony of
Eutropius (uno a Persis atque altero prœlio victus, x. 17) must
incline us to suspect that Ammianus had been too jealous of the
honor of the Roman arms.}

In this hopeless condition, the fainting spirits of the Romans
were revived by the sound of peace. The transient presumption of
Sapor had vanished: he observed, with serious concern, that, in
the repetition of doubtful combats, he had lost his most faithful
and intrepid nobles, his bravest troops, and the greatest part of
his train of elephants: and the experienced monarch feared to
provoke the resistance of despair, the vicissitudes of fortune,
and the unexhausted powers of the Roman empire; which might soon
advance to elieve, or to revenge, the successor of Julian. The
Surenas himself, accompanied by another satrap, appeared in the
camp of Jovian;\textsuperscript{109} and declared, that the clemency of his
sovereign was not averse to signify the conditions on which he
would consent to spare and to dismiss the Cæsar with the relics
of his captive army.\textsuperscript{10911} The hopes of safety subdued the
firmness of the Romans; the emperor was compelled, by the advice
of his council, and the cries of his soldiers, to embrace the
offer of peace; 10912 and the præfect Sallust was immediately
sent, with the general Arinthæus, to understand the pleasure of
the Great King. The crafty Persian delayed, under various
pretenses, the conclusion of the agreement; started difficulties,
required explanations, suggested expedients, receded from his
concessions, increased his demands, and wasted four days in the
arts of negotiation, till he had consumed the stock of provisions
which yet remained in the camp of the Romans. Had Jovian been
capable of executing a bold and prudent measure, he would have
continued his march, with unremitting diligence; the progress of
the treaty would have suspended the attacks of the Barbarians;
and, before the expiration of the fourth day, he might have
safely reached the fruitful province of Corduene, at the distance
only of one hundred miles.\textsuperscript{110} The irresolute emperor, instead of
breaking through the toils of the enemy, expected his fate with
patient resignation; and accepted the humiliating conditions of
peace, which it was no longer in his power to refuse. The five
provinces beyond the Tigris, which had been ceded by the
grandfather of Sapor, were restored to the Persian monarchy. He
acquired, by a single article, the impregnable city of Nisibis;
which had sustained, in three successive sieges, the effort of
his arms. Singara, and the castle of the Moors, one of the
strongest places of Mesopotamia, were likewise dismembered from
the empire. It was considered as an indulgence, that the
inhabitants of those fortresses were permitted to retire with
their effects; but the conqueror rigorously insisted, that the
Romans should forever abandon the king and kingdom of Armenia.\textsuperscript{11011}
A peace, or rather a long truce, of thirty years, was
stipulated between the hostile nations; the faith of the treaty
was ratified by solemn oaths and religious ceremonies; and
hostages of distinguished rank were reciprocally delivered to
secure the performance of the conditions.\textsuperscript{111}

\pagenote[109]{Sextus Rufus (de Provinciis, c. 29) embraces a
poor subterfuge of national vanity. Tanta reverentia nominis
Romani fuit, ut a Persis \textit{primus} de pace sermo haberetur. ——He
is called Junius by John Malala; the same, M. St. Martin
conjectures, with a satrap of Gordyene named Jovianus, or
Jovinianus; mentioned in Ammianus Marcellinus, xviii. 6.—M.}

\pagenote[10911]{The Persian historians couch the message of
Shah-pour in these Oriental terms: “I have reassembled my
numerous army. I am resolved to revenge my subjects, who have
been plundered, made captives, and slain. It is for this that I
have bared my arm, and girded my loins. If you consent to pay the
price of the blood which has been shed, to deliver up the booty
which has been plundered, and to restore the city of Nisibis,
which is in Irak, and belongs to our empire, though now in your
possession, I will sheathe the sword of war; but should you
refuse these terms, the hoofs of my horse, which are hard as
steel, shall efface the name of the Romans from the earth; and my
glorious cimeter, that destroys like fire, shall exterminate the
people of your empire.” These authorities do not mention the
death of Julian. Malcolm’s Persia, i. 87.—M.}

\pagenote[10912]{The Paschal chronicle, not, as M. St. Martin
says, supported by John Malala, places the mission of this
ambassador before the death of Julian. The king of Persia was
then in Persarmenia, ignorant of the death of Julian; he only
arrived at the army subsequent to that event. St. Martin adopts
this view, and finds or extorts support for it, from Libanius and
Ammianus, iii. 158.—M.}

\pagenote[110]{It is presumptuous to controvert the opinion of
Ammianus, a soldier and a spectator. Yet it is difficult to
understand \textit{how} the mountains of Corduene could extend over the
plains of Assyria, as low as the conflux of the Tigris and the
great Zab; or \textit{how} an army of sixty thousand men could march one
hundred miles in four days. Note: * Yet this appears to be the
case (in modern maps: ) the march is the difficulty.—M.}

\pagenote[11011]{Sapor availed himself, a few years after, of the
dissolution of the alliance between the Romans and the Armenians.
See St. M. iii. 163.—M.}

\pagenote[111]{The treaty of Dura is recorded with grief or
indignation by Ammianus, (xxv. 7,) Libanius, (Orat. Parent. c.
142, p. 364,) Zosimus, (l. iii. p. 190, 191,) Gregory Nazianzen,
(Orat. iv. p. 117, 118, who imputes the distress to Julian, the
deliverance to Jovian,) and Eutropius, (x. 17.) The
last-mentioned writer, who was present in military station,
styles this peace necessarium quidem sed ignoblem.}

The sophist of Antioch, who saw with indignation the sceptre of
his hero in the feeble hand of a Christian successor, professes
to admire the moderation of Sapor, in contenting himself with so
small a portion of the Roman empire. If he had stretched as far
as the Euphrates the claims of his ambition, he might have been
secure, says Libanius, of not meeting with a refusal. If he had
fixed, as the boundary of Persia, the Orontes, the Cydnus, the
Sangarius, or even the Thracian Bosphorus, flatterers would not
have been wanting in the court of Jovian to convince the timid
monarch, that his remaining provinces would still afford the most
ample gratifications of power and luxury.\textsuperscript{112} Without adopting in
its full force this malicious insinuation, we must acknowledge,
that the conclusion of so ignominious a treaty was facilitated by
the private ambition of Jovian. The obscure domestic, exalted to
the throne by fortune, rather than by merit, was impatient to
escape from the hands of the Persians, that he might prevent the
designs of Procopius, who commanded the army of Mesopotamia, and
establish his doubtful reign over the legions and provinces which
were still ignorant of the hasty and tumultuous choice of the
camp beyond the Tigris.\textsuperscript{113} In the neighborhood of the same
river, at no very considerable distance from the fatal station of
Dura,\textsuperscript{114} the ten thousand Greeks, without generals, or guides,
or provisions, were abandoned, above twelve hundred miles from
their native country, to the resentment of a victorious monarch.
The difference of \textit{their} conduct and success depended much more
on their character than on their situation. Instead of tamely
resigning themselves to the secret deliberations and private
views of a single person, the united councils of the Greeks were
inspired by the generous enthusiasm of a popular assembly; where
the mind of each citizen is filled with the love of glory, the
pride of freedom, and the contempt of death. Conscious of their
superiority over the Barbarians in arms and discipline, they
disdained to yield, they refused to capitulate: every obstacle
was surmounted by their patience, courage, and military skill;
and the memorable retreat of the ten thousand exposed and
insulted the weakness of the Persian monarchy.\textsuperscript{115}

\pagenote[112]{Libanius, Orat. Parent. c. 143, p. 364, 365.}

\pagenote[113]{Conditionibus..... dispendiosis Romanæ reipublicæ
impositis.... quibus cupidior regni quam gloriæ Jovianus, imperio
rudis, adquievit. Sextus Rufus de Provinciis, c. 29. La Bleterie
has expressed, in a long, direct oration, these specious
considerations of public and private interest, (Hist. de Jovien,
tom. i. p. 39, \&c.)}

\pagenote[114]{The generals were murdered on the bauks of the
Zabatus, (Ana basis, l. ii. p. 156, l. iii. p. 226,) or great
Zab, a river of Assyria, 400 feet broad, which falls into the
Tigris fourteen hours below Mosul. The error of the Greeks
bestowed on the greater and lesser Zab the names of the \textit{Wolf},
(Lycus,) and the \textit{Goat}, (Capros.) They created these animals to
attend the \textit{Tiger} of the East.}

\pagenote[115]{The \textit{Cyropædia} is vague and languid; the
\textit{Anabasis} circumstance and animated. Such is the eternal
difference between fiction and truth.}

As the price of his disgraceful concessions, the emperor might
perhaps have stipulated, that the camp of the hungry Romans
should be plentifully supplied;\textsuperscript{116} and that they should be
permitted to pass the Tigris on the bridge which was constructed
by the hands of the Persians. But, if Jovian presumed to solicit
those equitable terms, they were sternly refused by the haughty
tyrant of the East, whose clemency had pardoned the invaders of
his country. The Saracens sometimes intercepted the stragglers of
the march; but the generals and troops of Sapor respected the
cessation of arms; and Jovian was suffered to explore the most
convenient place for the passage of the river. The small vessels,
which had been saved from the conflagration of the fleet,
performed the most essential service. They first conveyed the
emperor and his favorites; and afterwards transported, in many
successive voyages, a great part of the army. But, as every man
was anxious for his personal safety, and apprehensive of being
left on the hostile shore, the soldiers, who were too impatient
to wait the slow returns of the boats, boldly ventured themselves
on light hurdles, or inflated skins; and, drawing after them
their horses, attempted, with various success, to swim across the
river. Many of these daring adventurers were swallowed by the
waves; many others, who were carried along by the violence of the
stream, fell an easy prey to the avarice or cruelty of the wild
Arabs: and the loss which the army sustained in the passage of
the Tigris, was not inferior to the carnage of a day of battle.
As soon as the Romans were landed on the western bank, they were
delivered from the hostile pursuit of the Barbarians; but, in a
laborious march of two hundred miles over the plains of
Mesopotamia, they endured the last extremities of thirst and
hunger. They were obliged to traverse the sandy desert, which, in
the extent of seventy miles, did not afford a single blade of
sweet grass, nor a single spring of fresh water; and the rest of
the inhospitable waste was untrod by the footsteps either of
friends or enemies. Whenever a small measure of flour could be
discovered in the camp, twenty pounds weight were greedily
purchased with ten pieces of gold:\textsuperscript{117} the beasts of burden were
slaughtered and devoured; and the desert was strewed with the
arms and baggage of the Roman soldiers, whose tattered garments
and meagre countenances displayed their past sufferings and
actual misery. A small convoy of provisions advanced to meet the
army as far as the castle of Ur; and the supply was the more
grateful, since it declared the fidelity of Sebastian and
Procopius. At Thilsaphata,\textsuperscript{118} the emperor most graciously
received the generals of Mesopotamia; and the remains of a once
flourishing army at length reposed themselves under the walls of
Nisibis. The messengers of Jovian had already proclaimed, in the
language of flattery, his election, his treaty, and his return;
and the new prince had taken the most effectual measures to
secure the allegiance of the armies and provinces of Europe, by
placing the military command in the hands of those officers, who,
from motives of interest, or inclination, would firmly support
the cause of their benefactor.\textsuperscript{119}

\pagenote[116]{According to Rufinus, an immediate supply of
provisions was stipulated by the treaty, and Theodoret affirms,
that the obligation was faithfully discharged by the Persians.
Such a fact is probable but undoubtedly false. See Tillemont,
Hist. des Empereurs, tom. iv. p. 702.}

\pagenote[117]{We may recollect some lines of Lucan, (Pharsal.
iv. 95,) who describes a similar distress of Cæsar’s army in
Spain:— ——Sæva fames aderat—Miles eget: toto censu non prodigus
emit Exiguam Cererem. Proh lucri pallida tabes! Non deest prolato
jejunus venditor auro. See Guichardt (Nouveaux Mémoires
Militaires, tom. i. p. 370-382.) His analysis of the two
campaigns in Spain and Africa is the noblest monument that has
ever been raised to the fame of Cæsar.}

\pagenote[118]{M. d’Anville (see his Maps, and l’Euphrate et le
Tigre, p. 92, 93) traces their march, and assigns the true
position of Hatra, Ur, and Thilsaphata, which Ammianus has
mentioned. ——He does not complain of the Samiel, the deadly hot
wind, which Thevenot (Voyages, part ii. l. i. p. 192) so much
dreaded. ——Hatra, now Kadhr. Ur, Kasr or Skervidgi. Thilsaphata
is unknown—M.}

\pagenote[119]{The retreat of Jovian is described by Ammianus,
(xxv. 9,) Libanius, (Orat. Parent. c. 143, p. 365,) and Zosimus,
(l. iii. p. 194.)}

The friends of Julian had confidently announced the success of
his expedition. They entertained a fond persuasion that the
temples of the gods would be enriched with the spoils of the
East; that Persia would be reduced to the humble state of a
tributary province, governed by the laws and magistrates of Rome;
that the Barbarians would adopt the dress, and manners, and
language of their conquerors; and that the youth of Ecbatana and
Susa would study the art of rhetoric under Grecian masters.\textsuperscript{120}
The progress of the arms of Julian interrupted his communication
with the empire; and, from the moment that he passed the Tigris,
his affectionate subjects were ignorant of the fate and fortunes
of their prince. Their contemplation of fancied triumphs was
disturbed by the melancholy rumor of his death; and they
persisted to doubt, after they could no longer deny, the truth of
that fatal event.\textsuperscript{121} The messengers of Jovian promulgated the
specious tale of a prudent and necessary peace; the voice of
fame, louder and more sincere, revealed the disgrace of the
emperor, and the conditions of the ignominious treaty. The minds
of the people were filled with astonishment and grief, with
indignation and terror, when they were informed, that the
unworthy successor of Julian relinquished the five provinces
which had been acquired by the victory of Galerius; and that he
shamefully surrendered to the Barbarians the important city of
Nisibis, the firmest bulwark of the provinces of the East.\textsuperscript{122}
The deep and dangerous question, how far the public faith should
be observed, when it becomes incompatible with the public safety,
was freely agitated in popular conversation; and some hopes were
entertained that the emperor would redeem his pusillanimous
behavior by a splendid act of patriotic perfidy. The inflexible
spirit of the Roman senate had always disclaimed the unequal
conditions which were extorted from the distress of their captive
armies; and, if it were necessary to satisfy the national honor,
by delivering the guilty general into the hands of the
Barbarians, the greatest part of the subjects of Jovian would
have cheerfully acquiesced in the precedent of ancient times.\textsuperscript{123}

\pagenote[120]{Libanius, (Orat. Parent. c. 145, p. 366.) Such
were the natural hopes and wishes of a rhetorician.}

\pagenote[121]{The people of Carrhæ, a city devoted to Paganism,
buried the inauspicious messenger under a pile of stones,
(Zosimus, l. iii. p. 196.) Libanius, when he received the fatal
intelligence, cast his eye on his sword; but he recollected that
Plato had condemned suicide, and that he must live to compose the
Panegyric of Julian, (Libanius de Vita sua, tom. ii. p. 45, 46.)}

\pagenote[122]{Ammianus and Eutropius may be admitted as fair and
credible witnesses of the public language and opinions. The
people of Antioch reviled an ignominious peace, which exposed
them to the Persians, on a naked and defenceless frontier,
(Excerpt. Valesiana, p. 845, ex Johanne Antiocheno.)}

\pagenote[123]{The Abbé de la Bleterie, (Hist. de Jovien, tom. i.
p. 212-227.) though a severe casuist, has pronounced that Jovian
was not bound to execute his promise; since he \textit{could not}
dismember the empire, nor alienate, without their consent, the
allegiance of his people. I have never found much delight or
instruction in such political metaphysics.}

But the emperor, whatever might be the limits of his
constitutional authority, was the absolute master of the laws and
arms of the state; and the same motives which had forced him to
subscribe, now pressed him to execute, the treaty of peace. He
was impatient to secure an empire at the expense of a few
provinces; and the respectable names of religion and honor
concealed the personal fears and ambition of Jovian.
Notwithstanding the dutiful solicitations of the inhabitants,
decency, as well as prudence, forbade the emperor to lodge in the
palace of Nisibis; but the next morning after his arrival,
Bineses, the ambassador of Persia, entered the place, displayed
from the citadel the standard of the Great King, and proclaimed,
in his name, the cruel alternative of exile or servitude. The
principal citizens of Nisibis, who, till that fatal moment, had
confided in the protection of their sovereign, threw themselves
at his feet. They conjured him not to abandon, or, at least, not
to deliver, a faithful colony to the rage of a Barbarian tyrant,
exasperated by the three successive defeats which he had
experienced under the walls of Nisibis. They still possessed arms
and courage to repel the invaders of their country: they
requested only the permission of using them in their own defence;
and, as soon as they had asserted their independence, they should
implore the favor of being again admitted into the ranks of his
subjects. Their arguments, their eloquence, their tears, were
ineffectual. Jovian alleged, with some confusion, the sanctity of
oaths; and, as the reluctance with which he accepted the present
of a crown of gold, convinced the citizens of their hopeless
condition, the advocate Sylvanus was provoked to exclaim, “O
emperor! may you thus be crowned by all the cities of your
dominions!” Jovian, who in a few weeks had assumed the habits of
a prince,\textsuperscript{124} was displeased with freedom, and offended with
truth: and as he reasonably supposed, that the discontent of the
people might incline them to submit to the Persian government, he
published an edict, under pain of death, that they should leave
the city within the term of three days. Ammianus has delineated
in lively colors the scene of universal despair, which he seems
to have viewed with an eye of compassion.\textsuperscript{125} The martial youth
deserted, with indignant grief, the walls which they had so
gloriously defended: the disconsolate mourner dropped a last tear
over the tomb of a son or husband, which must soon be profaned by
the rude hand of a Barbarian master; and the aged citizen kissed
the threshold, and clung to the doors, of the house where he had
passed the cheerful and careless hours of infancy. The highways
were crowded with a trembling multitude: the distinctions of
rank, and sex, and age, were lost in the general calamity. Every
one strove to bear away some fragment from the wreck of his
fortunes; and as they could not command the immediate service of
an adequate number of horses or wagons, they were obliged to
leave behind them the greatest part of their valuable effects.
The savage insensibility of Jovian appears to have aggravated the
hardships of these unhappy fugitives. They were seated, however,
in a new-built quarter of Amida; and that rising city, with the
reenforcement of a very considerable colony, soon recovered its
former splendor, and became the capital of Mesopotamia.\textsuperscript{126}
Similar orders were despatched by the emperor for the evacuation
of Singara and the castle of the Moors; and for the restitution
of the five provinces beyond the Tigris. Sapor enjoyed the glory
and the fruits of his victory; and this ignominious peace has
justly been considered as a memorable æra in the decline and fall
of the Roman empire. The predecessors of Jovian had sometimes
relinquished the dominion of distant and unprofitable provinces;
but, since the foundation of the city, the genius of Rome, the
god Terminus, who guarded the boundaries of the republic, had
never retired before the sword of a victorious enemy.\textsuperscript{127}

\pagenote[124]{At Nisibis he performed a \textit{royal} act. A brave
officer, his namesake, who had been thought worthy of the purple,
was dragged from supper, thrown into a well, and stoned to death
without any form of trial or evidence of guilt. Anomian. xxv. 8.}

\pagenote[125]{See xxv. 9, and Zosimus, l. iii. p. 194, 195.}

\pagenote[126]{Chron. Paschal. p. 300. The ecclesiastical Notitiæ
may be consulted.}

\pagenote[127]{Zosimus, l. iii. p. 192, 193. Sextus Rufus de
Provinciis, c. 29. Augustin de Civitat. Dei, l. iv. c. 29. This
general position must be applied and interpreted with some
caution.}

After Jovian had performed those engagements which the voice of
his people might have tempted him to violate, he hastened away
from the scene of his disgrace, and proceeded with his whole
court to enjoy the luxury of Antioch.\textsuperscript{128} Without consulting the
dictates of religious zeal, he was prompted, by humanity and
gratitude, to bestow the last honors on the remains of his
deceased sovereign:\textsuperscript{129} and Procopius, who sincerely bewailed the
loss of his kinsman, was removed from the command of the army,
under the decent pretence of conducting the funeral. The corpse
of Julian was transported from Nisibis to Tarsus, in a slow march
of fifteen days; and, as it passed through the cities of the
East, was saluted by the hostile factions, with mournful
lamentations and clamorous insults. The Pagans already placed
their beloved hero in the rank of those gods whose worship he had
restored; while the invectives of the Christians pursued the soul
of the Apostate to hell, and his body to the grave.\textsuperscript{130} One party
lamented the approaching ruin of their altars; the other
celebrated the marvellous deliverance of their church. The
Christians applauded, in lofty and ambiguous strains, the stroke
of divine vengeance, which had been so long suspended over the
guilty head of Julian. They acknowledge, that the death of the
tyrant, at the instant he expired beyond the Tigris, was
\textit{revealed} to the saints of Egypt, Syria, and Cappadocia;\textsuperscript{131} and
instead of suffering him to fall by the Persian darts, their
indiscretion ascribed the heroic deed to the obscure hand of some
mortal or immortal champion of the faith.\textsuperscript{132} Such imprudent
declarations were eagerly adopted by the malice, or credulity, of
their adversaries;\textsuperscript{133} who darkly insinuated, or confidently
asserted, that the governors of the church had instigated and
directed the fanaticism of a domestic assassin.\textsuperscript{134} Above sixteen
years after the death of Julian, the charge was solemnly and
vehemently urged, in a public oration, addressed by Libanius to
the emperor Theodosius. His suspicions are unsupported by fact or
argument; and we can only esteem the generous zeal of the sophist
of Antioch for the cold and neglected ashes of his friend.\textsuperscript{135}

\pagenote[128]{Ammianus, xxv. 9. Zosimus, l. iii. p. 196. He
might be edax, vino Venerique indulgens. But I agree with La
Bleterie (tom. i. p. 148-154) in rejecting the foolish report of
a Bacchanalian riot (ap. Suidam) celebrated at Antioch, by the
emperor, his \textit{wife}, and a troop of concubines.}

\pagenote[129]{The Abbé de la Bleterie (tom. i. p. 156-209)
handsomely exposes the brutal bigotry of Baronius, who would have
thrown Julian to the dogs, ne cespititia quidem sepultura
dignus.}

\pagenote[130]{Compare the sophist and the saint, (Libanius,
Monod. tom. ii. p. 251, and Orat. Parent. c. 145, p. 367, c. 156,
p. 377, with Gregory Nazianzen, Orat. iv. p. 125-132.) The
Christian orator faintly mutters some exhortations to modesty and
forgiveness; but he is well satisfied, that the real sufferings
of Julian will far exceed the fabulous torments of Ixion or
Tantalus.}

\pagenote[131]{Tillemont (Hist. des Empereurs, tom. iv. p. 549)
has collected these visions. Some saint or angel was observed to
be absent in the night, on a secret expedition, \&c.}

\pagenote[132]{Sozomen (l. vi. 2) applauds the Greek doctrine of
\textit{tyrannicide;} but the whole passage, which a Jesuit might have
translated, is prudently suppressed by the president Cousin.}

\pagenote[133]{Immediately after the death of Julian, an
uncertain rumor was scattered, telo cecidisse Romano. It was
carried, by some deserters to the Persian camp; and the Romans
were reproached as the assassins of the emperor by Sapor and his
subjects, (Ammian. xxv. 6. Libanius de ulciscenda Juliani nece,
c. xiii. p. 162, 163.) It was urged, as a decisive proof, that no
Persian had appeared to claim the promised reward, (Liban. Orat.
Parent. c. 141, p. 363.) But the flying horseman, who darted the
fatal javelin, might be ignorant of its effect; or he might be
slain in the same action. Ammianus neither feels nor inspires a
suspicion.}

\pagenote[134]{This dark and ambiguous expression may point to
Athanasius, the first, without a rival, of the Christian clergy,
(Libanius de ulcis. Jul. nece, c. 5, p. 149. La Bleterie, Hist.
de Jovien, tom. i. p. 179.)}

\pagenote[135]{The orator (Fabricius, Bibliot. Græc. tom. vii. p.
145-179) scatters suspicions, demands an inquiry, and insinuates,
that proofs might still be obtained. He ascribes the success of
the Huns to the criminal neglect of revenging Julian’s death.}

It was an ancient custom in the funerals, as well as in the
triumphs, of the Romans, that the voice of praise should be
corrected by that of satire and ridicule; and that, in the midst
of the splendid pageants, which displayed the glory of the living
or of the dead, their imperfections should not be concealed from
the eyes of the world.\textsuperscript{136} This custom was practised in the
funeral of Julian. The comedians, who resented his contempt and
aversion for the theatre, exhibited, with the applause of a
Christian audience, the lively and exaggerated representation of
the faults and follies of the deceased emperor. His various
character and singular manners afforded an ample scope for
pleasantry and ridicule.\textsuperscript{137} In the exercise of his uncommon
talents, he often descended below the majesty of his rank.
Alexander was transformed into Diogenes; the philosopher was
degraded into a priest. The purity of his virtue was sullied by
excessive vanity; his superstition disturbed the peace, and
endangered the safety, of a mighty empire; and his irregular
sallies were the less entitled to indulgence, as they appeared to
be the laborious efforts of art, or even of affectation. The
remains of Julian were interred at Tarsus in Cilicia; but his
stately tomb, which arose in that city, on the banks of the cold
and limpid Cydnus,\textsuperscript{138} was displeasing to the faithful friends,
who loved and revered the memory of that extraordinary man. The
philosopher expressed a very reasonable wish, that the disciple
of Plato might have reposed amidst the groves of the academy;\textsuperscript{139}
while the soldier exclaimed, in bolder accents, that the ashes of
Julian should have been mingled with those of Cæsar, in the field
of Mars, and among the ancient monuments of Roman virtue.\textsuperscript{140} The
history of princes does not very frequently renew the examples of
a similar competition.

\pagenote[136]{At the funeral of Vespasian, the comedian who
personated that frugal emperor, anxiously inquired how much it
cost. Fourscore thousand pounds, (centies.) Give me the tenth
part of the sum, and throw my body into the Tiber. Sueton, in
Vespasian, c. 19, with the notes of Casaubon and Gronovius.}

\pagenote[137]{Gregory (Orat. iv. p. 119, 120) compares this
supposed ignominy and ridicule to the funeral honors of
Constantius, whose body was chanted over Mount Taurus by a choir
of angels.}

\pagenote[138]{Quintus Curtius, l. iii. c. 4. The luxuriancy of
his descriptions has been often censured. Yet it was almost the
duty of the historian to describe a river, whose waters had
nearly proved fatal to Alexander.}

\pagenote[139]{Libanius, Orat. Parent. c. 156, p. 377. Yet he
acknowledges with gratitude the liberality of the two royal
brothers in decorating the tomb of Julian, (de ulcis. Jul. nece,
c. 7, p. 152.)}

\pagenote[140]{Cujus suprema et cineres, si qui tunc juste
consuleret, non Cydnus videre deberet, quamvis gratissimus amnis
et liquidus: sed ad perpetuandam gloriam recte factorum
præterlambere Tiberis, intersecans urbem æternam, divorumque
veterum monumenta præstringens Ammian. xxv. 10.}

