\chapter{Reigns Of Jovian And Valentinian, Division Of The Empire.}
\section{Part \thesection.}

\textit{The Government And Death Of Jovian. — Election Of Valentinian, Who
Associates His Brother Valens, And Makes The Final Division Of The
Eastern And Western Empires. — Revolt Of Procopius. — Civil And
Ecclesiastical Administration. — Germany. — Britain. — Africa. — The
East. — The Danube. — Death Of Valentinian. — His Two Sons, Gratian And
Valentinian II., Succeed To The Western Empire.}
\vspace{\onelineskip}

The death of Julian had left the public affairs of the empire in
a very doubtful and dangerous situation. The Roman army was saved
by an inglorious, perhaps a necessary treaty;\textsuperscript{1} and the first
moments of peace were consecrated by the pious Jovian to restore
the domestic tranquility of the church and state. The
indiscretion of his predecessor, instead of reconciling, had
artfully fomented the religious war: and the balance which he
affected to preserve between the hostile factions, served only to
perpetuate the contest, by the vicissitudes of hope and fear, by
the rival claims of ancient possession and actual favor. The
Christians had forgotten the spirit of the gospel; and the Pagans
had imbibed the spirit of the church. In private families, the
sentiments of nature were extinguished by the blind fury of zeal
and revenge: the majesty of the laws was violated or abused; the
cities of the East were stained with blood; and the most
implacable enemies of the Romans were in the bosom of their
country. Jovian was educated in the profession of Christianity;
and as he marched from Nisibis to Antioch, the banner of the
Cross, the Labarum of Constantine, which was again displayed at
the head of the legions, announced to the people the faith of
their new emperor. As soon as he ascended the throne, he
transmitted a circular epistle to all the governors of provinces;
in which he confessed the divine truth, and secured the legal
establishment, of the Christian religion. The insidious edicts of
Julian were abolished; the ecclesiastical immunities were
restored and enlarged; and Jovian condescended to lament, that
the distress of the times obliged him to diminish the measure of
charitable distributions.\textsuperscript{2} The Christians were unanimous in the
loud and sincere applause which they bestowed on the pious
successor of Julian. But they were still ignorant what creed, or
what synod, he would choose for the standard of orthodoxy; and
the peace of the church immediately revived those eager disputes
which had been suspended during the season of persecution. The
episcopal leaders of the contending sects, convinced, from
experience, how much their fate would depend on the earliest
impressions that were made on the mind of an untutored soldier,
hastened to the court of Edessa, or Antioch. The highways of the
East were crowded with Homoousian, and Arian, and Semi-Arian, and
Eunomian bishops, who struggled to outstrip each other in the
holy race: the apartments of the palace resounded with their
clamors; and the ears of the prince were assaulted, and perhaps
astonished, by the singular mixture of metaphysical argument and
passionate invective.\textsuperscript{3} The moderation of Jovian, who recommended
concord and charity, and referred the disputants to the sentence
of a future council, was interpreted as a symptom of
indifference: but his attachment to the Nicene creed was at
length discovered and declared, by the reverence which he
expressed for the \textit{celestial}\textsuperscript{4} virtues of the great Athanasius.
The intrepid veteran of the faith, at the age of seventy, had
issued from his retreat on the first intelligence of the tyrant’s
death. The acclamations of the people seated him once more on the
archiepiscopal throne; and he wisely accepted, or anticipated,
the invitation of Jovian. The venerable figure of Athanasius, his
calm courage, and insinuating eloquence, sustained the reputation
which he had already acquired in the courts of four successive
princes.\textsuperscript{5} As soon as he had gained the confidence, and secured
the faith, of the Christian emperor, he returned in triumph to
his diocese, and continued, with mature counsels and undiminished
vigor, to direct, ten years longer,\textsuperscript{6} the ecclesiastical
government of Alexandria, Egypt, and the Catholic church. Before
his departure from Antioch, he assured Jovian that his orthodox
devotion would be rewarded with a long and peaceful reign.
Athanasius, had reason to hope, that he should be allowed either
the merit of a successful prediction, or the excuse of a grateful
though ineffectual prayer.\textsuperscript{7}

\pagenote[1]{The medals of Jovian adorn him with victories,
laurel crowns, and prostrate captives. Ducange, Famil. Byzantin.
p. 52. Flattery is a foolish suicide; she destroys herself with
her own hands.}

\pagenote[2]{Jovian restored to the church a forcible and
comprehensive expression, (Philostorgius, l. viii. c. 5, with
Godefroy’s Dissertations, p. 329. Sozomen, l. vi. c. 3.) The new
law which condemned the rape or marriage of nuns (Cod. Theod. l.
ix. tit. xxv. leg. 2) is exaggerated by Sozomen; who supposes,
that an amorous glance, the adultery of the heart, was punished
with death by the evangelic legislator.}

\pagenote[3]{Compare Socrates, l. iii. c. 25, and Philostorgius,
l. viii. c. 6, with Godefroy’s Dissertations, p. 330.}

\pagenote[4]{The word \textit{celestial} faintly expresses the impious
and extravagant flattery of the emperor to the archbishop. (See
the original epistle in Athanasius, tom. ii. p. 33.) Gregory
Nazianzen (Orat. xxi. p. 392) celebrates the friendship of Jovian
and Athanasius. The primate’s journey was advised by the Egyptian
monks, (Tillemont, Mém. Eccles. tom. viii. p. 221.)}

\pagenote[5]{Athanasius, at the court of Antioch, is agreeably
represented by La Bleterie, (Hist. de Jovien, tom. i. p.
121-148;) he translates the singular and original conferences of
the emperor, the primate of Egypt, and the Arian deputies. The
Abbé is not satisfied with the coarse pleasantry of Jovian; but
his partiality for Athanasius assumes, in \textit{his} eyes, the
character of justice.}

\pagenote[6]{The true area of his death is perplexed with some
difficulties, (Tillemont, Mém. Eccles. tom. viii. p. 719-723.)
But the date (A. D. 373, May 2) which seems the most consistent
with history and reason, is ratified by his authentic life,
(Maffei Osservazioni Letterarie, tom. iii. p. 81.)}

\pagenote[7]{See the observations of Valesius and Jortin (Remarks
on Ecclesiastical History, vol. iv. p. 38) on the original letter
of Athanasius; which is preserved by Theodoret, (l. iv. c. 3.) In
some Mss. this indiscreet promise is omitted; perhaps by the
Catholics, jealous of the prophetic fame of their leader.}

The slightest force, when it is applied to assist and guide the
natural descent of its object, operates with irresistible weight;
and Jovian had the good fortune to embrace the religious opinions
which were supported by the spirit of the times, and the zeal and
numbers of the most powerful sect.\textsuperscript{8} Under his reign,
Christianity obtained an easy and lasting victory; and as soon as
the smile of royal patronage was withdrawn, the genius of
Paganism, which had been fondly raised and cherished by the arts
of Julian, sunk irrecoverably in the. In many cities, the temples
were shut or deserted: the philosophers who had abused their
transient favor, thought it prudent to shave their beards, and
disguise their profession; and the Christians rejoiced, that they
were now in a condition to forgive, or to revenge, the injuries
which they had suffered under the preceding reign.\textsuperscript{9} The
consternation of the Pagan world was dispelled by a wise and
gracious edict of toleration; in which Jovian explicitly
declared, that although he should severely punish the
sacrilegious rites of magic, his subjects might exercise, with
freedom and safety, the ceremonies of the ancient worship. The
memory of this law has been preserved by the orator Themistius,
who was deputed by the senate of Constantinople to express their
royal devotion for the new emperor. Themistius expatiates on the
clemency of the Divine Nature, the facility of human error, the
rights of conscience, and the independence of the mind; and, with
some eloquence, inculcates the principles of philosophical
toleration; whose aid Superstition herself, in the hour of her
distress, is not ashamed to implore. He justly observes, that in
the recent changes, both religions had been alternately disgraced
by the seeming acquisition of worthless proselytes, of those
votaries of the reigning purple, who could pass, without a
reason, and without a blush, from the church to the temple, and
from the altars of Jupiter to the sacred table of the Christians.\textsuperscript{10}

\pagenote[8]{Athanasius (apud Theodoret, l. iv. c. 3) magnifies
the number of the orthodox, who composed the whole world. This
assertion was verified in the space of thirty and forty years.}

\pagenote[9]{Socrates, l. iii. c. 24. Gregory Nazianzen (Orat.
iv. p. 131) and Libanius (Orat. Parentalis, c. 148, p. 369)
expresses the \textit{living} sentiments of their respective factions.}

\pagenote[10]{Themistius, Orat. v. p. 63-71, edit. Harduin,
Paris, 1684. The Abbé de la Bleterie judiciously remarks, (Hist.
de Jovien, tom. i. p. 199,) that Sozomen has forgot the general
toleration; and Themistius the establishment of the Catholic
religion. Each of them turned away from the object which he
disliked, and wished to suppress the part of the edict the least
honorable, in his opinion, to the emperor.}

In the space of seven months, the Roman troops, who were now
returned to Antioch, had performed a march of fifteen hundred
miles; in which they had endured all the hardships of war, of
famine, and of climate. Notwithstanding their services, their
fatigues, and the approach of winter, the timid and impatient
Jovian allowed only, to the men and horses, a respite of six
weeks. The emperor could not sustain the indiscreet and malicious
raillery of the people of Antioch.\textsuperscript{11} He was impatient to possess
the palace of Constantinople; and to prevent the ambition of some
competitor, who might occupy the vacant allegiance of Europe. But
he soon received the grateful intelligence, that his authority
was acknowledged from the Thracian Bosphorus to the Atlantic
Ocean. By the first letters which he despatched from the camp of
Mesopotamia, he had delegated the military command of Gaul and
Illyricum to Malarich, a brave and faithful officer of the nation
of the Franks; and to his father-in-law, Count Lucillian, who had
formerly distinguished his courage and conduct in the defence of
Nisibis. Malarich had declined an office to which he thought
himself unequal; and Lucillian was massacred at Rheims, in an
accidental mutiny of the Batavian cohorts.\textsuperscript{12} But the moderation
of Jovinus, master-general of the cavalry, who forgave the
intention of his disgrace, soon appeased the tumult, and
confirmed the uncertain minds of the soldiers. The oath of
fidelity was administered and taken, with loyal acclamations; and
the deputies of the Western armies\textsuperscript{13} saluted their new sovereign
as he descended from Mount Taurus to the city of Tyana in
Cappadocia. From Tyana he continued his hasty march to Ancyra,
capital of the province of Galatia; where Jovian assumed, with
his infant son, the name and ensigns of the consulship.\textsuperscript{14}
Dadastana,\textsuperscript{15} an obscure town, almost at an equal distance
between Ancyra and Nice, was marked for the fatal term of his
journey and life. After indulging himself with a plentiful,
perhaps an intemperate, supper, he retired to rest; and the next
morning the emperor Jovian was found dead in his bed. The cause
of this sudden death was variously understood. By some it was
ascribed to the consequences of an indigestion, occasioned either
by the quantity of the wine, or the quality of the mushrooms,
which he had swallowed in the evening. According to others, he
was suffocated in his sleep by the vapor of charcoal, which
extracted from the walls of the apartment the unwholesome
moisture of the fresh plaster.\textsuperscript{16} But the want of a regular
inquiry into the death of a prince, whose reign and person were
soon forgotten, appears to have been the only circumstance which
countenanced the malicious whispers of poison and domestic guilt.\textsuperscript{17}
The body of Jovian was sent to Constantinople, to be interred
with his predecessors, and the sad procession was met on the road
by his wife Charito, the daughter of Count Lucillian; who still
wept the recent death of her father, and was hastening to dry her
tears in the embraces of an Imperial husband. Her disappointment
and grief were imbittered by the anxiety of maternal tenderness.
Six weeks before the death of Jovian, his infant son had been
placed in the curule chair, adorned with the title of
\textit{Nobilissimus}, and the vain ensigns of the consulship.
Unconscious of his fortune, the royal youth, who, from his
grandfather, assumed the name of Varronian, was reminded only by
the jealousy of the government, that he was the son of an
emperor. Sixteen years afterwards he was still alive, but he had
already been deprived of an eye; and his afflicted mother
expected every hour, that the innocent victim would be torn from
her arms, to appease, with his blood, the suspicions of the
reigning prince.\textsuperscript{18}

\pagenote[11]{Johan. Antiochen. in Excerpt. Valesian. p. 845. The
libels of Antioch may be admitted on very slight evidence.}

\pagenote[12]{Compare Ammianus, (xxv. 10,) who omits the name of
the Batarians, with Zosimus, (l. iii. p. 197,) who removes the
scene of action from Rheims to Sirmium.}

\pagenote[13]{Quos capita scholarum ordo castrensis appellat.
Ammian. xxv. 10, and Vales. ad locum.}

\pagenote[14]{Cugus vagitus, pertinaciter reluctantis, ne in
curuli sella veheretur ex more, id quod mox accidit protendebat.
Augustus and his successors respectfully solicited a dispensation
of age for the sons or nephews whom they raised to the
consulship. But the curule chair of the first Brutus had never
been dishonored by an infant.}

\pagenote[15]{The Itinerary of Antoninus fixes Dadastana 125
Roman miles from Nice; 117 from Ancyra, (Wesseling, Itinerar. p.
142.) The pilgrim of Bourdeaux, by omitting some stages, reduces
the whole space from 242 to 181 miles. Wesseling, p. 574. * Note:
Dadastana is supposed to be Castabat.—M.}

\pagenote[16]{See Ammianus, (xxv. 10,) Eutropius, (x. 18.) who
might likewise be present, Jerom, (tom. i. p. 26, ad Heliodorum.)
Orosius, (vii. 31,) Sozomen, (l. vi. c. 6,) Zosimus, (l. iii. p.
197, 198,) and Zonaras, (tom. ii. l. xiii. p. 28, 29.) We cannot
expect a perfect agreement, and we shall not discuss minute
differences.}

\pagenote[17]{Ammianus, unmindful of his usual candor and good
sense, compares the death of the harmless Jovian to that of the
second Africanus, who had excited the fears and resentment of the
popular faction.}

\pagenote[18]{Chrysostom, tom. i. p. 336, 344, edit. Montfaucon.
The Christian orator attempts to comfort a widow by the examples
of illustrious misfortunes; and observes, that of nine emperors
(including the Cæsar Gallus) who had reigned in his time, only
two (Constantine and Constantius) died a natural death. Such
vague consolations have never wiped away a single tear.}

After the death of Jovian, the throne of the Roman world remained
ten days,\textsuperscript{19} without a master. The ministers and generals still
continued to meet in council; to exercise their respective
functions; to maintain the public order; and peaceably to conduct
the army to the city of Nice in Bithynia, which was chosen for
the place of the election.\textsuperscript{20} In a solemn assembly of the civil
and military powers of the empire, the diadem was again
unanimously offered to the præfect Sallust. He enjoyed the glory
of a second refusal: and when the virtues of the father were
alleged in favor of his son, the præfect, with the firmness of a
disinterested patriot, declared to the electors, that the feeble
age of the one, and the unexperienced youth of the other, were
equally incapable of the laborious duties of government. Several
candidates were proposed; and, after weighing the objections of
character or situation, they were successively rejected; but, as
soon as the name of Valentinian was pronounced, the merit of that
officer united the suffrages of the whole assembly, and obtained
the sincere approbation of Sallust himself. Valentinian\textsuperscript{21} was
the son of Count Gratian, a native of Cibalis, in Pannonia, who
from an obscure condition had raised himself, by matchless
strength and dexterity, to the military commands of Africa and
Britain; from which he retired with an ample fortune and
suspicious integrity. The rank and services of Gratian
contributed, however, to smooth the first steps of the promotion
of his son; and afforded him an early opportunity of displaying
those solid and useful qualifications, which raised his character
above the ordinary level of his fellow-soldiers. The person of
Valentinian was tall, graceful, and majestic. His manly
countenance, deeply marked with the impression of sense and
spirit, inspired his friends with awe, and his enemies with fear;
and to second the efforts of his undaunted courage, the son of
Gratian had inherited the advantages of a strong and healthy
constitution. By the habits of chastity and temperance, which
restrain the appetites and invigorate the faculties, Valentinian
preserved his own and the public esteem. The avocations of a
military life had diverted his youth from the elegant pursuits of
literature;\textsuperscript{2111} he was ignorant of the Greek language, and the
arts of rhetoric; but as the mind of the orator was never
disconcerted by timid perplexity, he was able, as often as the
occasion prompted him, to deliver his decided sentiments with
bold and ready elocution. The laws of martial discipline were the
only laws that he had studied; and he was soon distinguished by
the laborious diligence, and inflexible severity, with which he
discharged and enforced the duties of the camp. In the time of
Julian he provoked the danger of disgrace, by the contempt which
he publicly expressed for the reigning religion;\textsuperscript{22} and it should
seem, from his subsequent conduct, that the indiscreet and
unseasonable freedom of Valentinian was the effect of military
spirit, rather than of Christian zeal. He was pardoned, however,
and still employed by a prince who esteemed his merit;\textsuperscript{23} and in
the various events of the Persian war, he improved the reputation
which he had already acquired on the banks of the Rhine. The
celerity and success with which he executed an important
commission, recommended him to the favor of Jovian; and to the
honorable command of the second \textit{school}, or company, of
Targetiers, of the domestic guards. In the march from Antioch, he
had reached his quarters at Ancyra, when he was unexpectedly
summoned, without guilt and without intrigue, to assume, in the
forty-third year of his age, the absolute government of the Roman
empire.

\pagenote[19]{Ten days appear scarcely sufficient for the march
and election. But it may be observed, 1. That the generals might
command the expeditious use of the public posts for themselves,
their attendants, and messengers. 2. That the troops, for the
ease of the cities, marched in many divisions; and that the head
of the column might arrive at Nice, when the rear halted at
Ancyra.}

\pagenote[20]{Ammianus, xxvi. 1. Zosimus, l. iii. p. 198.
Philostorgius, l. viii. c. 8, and Godefroy, Dissertat. p. 334.
Philostorgius, who appears to have obtained some curious and
authentic intelligence, ascribes the choice of Valentinian to the
præfect Sallust, the master-general Arintheus, Dagalaiphus count
of the domestics, and the patrician Datianus, whose pressing
recommendations from Ancyra had a weighty influence in the
election.}

\pagenote[21]{Ammianus (xxx. 7, 9) and the younger Victor have
furnished the portrait of Valentinian, which naturally precedes
and illustrates the history of his reign. * Note: Symmachus, in a
fragment of an oration published by M. Mai, describes Valentinian
as born among the snows of Illyria, and habituated to military
labor amid the heat and dust of Libya: genitus in frigoribus,
educatus is solibus Sym. Orat. Frag. edit. Niebuhr, p. 5.—M.}

\pagenote[2111]{According to Ammianus, he wrote elegantly, and
was skilled in painting and modelling. Scribens decore,
venusteque pingens et fingens. xxx. 7.—M.}

\pagenote[22]{At Antioch, where he was obliged to attend the
emperor to the table, he struck a priest, who had presumed to
purify him with lustral water, (Sozomen, l. vi. c. 6. Theodoret,
l. iii. c. 15.) Such public defiance might become Valentinian;
but it could leave no room for the unworthy delation of the
philosopher Maximus, which supposes some more private offence,
(Zosimus, l. iv. p. 200, 201.)}

\pagenote[23]{Socrates, l. iv. A previous exile to Melitene, or
Thebais (the first might be possible,) is interposed by Sozomen
(l. vi. c. 6) and Philostorgius, (l. vii. c. 7, with Godefroy’s
Dissertations, p. 293.)}

The invitation of the ministers and generals at Nice was of
little moment, unless it were confirmed by the voice of the army.

The aged Sallust, who had long observed the irregular
fluctuations of popular assemblies, proposed, under pain of
death, that none of those persons, whose rank in the service
might excite a party in their favor, should appear in public on
the day of the inauguration. Yet such was the prevalence of
ancient superstition, that a whole day was voluntarily added to
this dangerous interval, because it happened to be the
intercalation of the Bissextile.\textsuperscript{24} At length, when the hour was
supposed to be propitious, Valentinian showed himself from a
lofty tribunal; the judicious choice was applauded; and the new
prince was solemnly invested with the diadem and the purple,
amidst the acclamation of the troops, who were disposed in
martial order round the tribunal. But when he stretched forth his
hand to address the armed multitude, a busy whisper was
accidentally started in the ranks, and insensibly swelled into a
loud and imperious clamor, that he should name, without delay, a
colleague in the empire. The intrepid calmness of Valentinian
obtained silence, and commanded respect; and he thus addressed
the assembly: “A few minutes since it was in \textit{your} power,
fellow-soldiers, to have left me in the obscurity of a private
station. Judging, from the testimony of my past life, that I
deserved to reign, you have placed me on the throne. It is now
\textit{my} duty to consult the safety and interest of the republic. The
weight of the universe is undoubtedly too great for the hands of
a feeble mortal. I am conscious of the limits of my abilities,
and the uncertainty of my life; and far from declining, I am
anxious to solicit, the assistance of a worthy colleague. But,
where discord may be fatal, the choice of a faithful friend
requires mature and serious deliberation. That deliberation shall
be \textit{my} care. Let \textit{your} conduct be dutiful and consistent.
Retire to your quarters; refresh your minds and bodies; and
expect the accustomed donative on the accession of a new
emperor.”\textsuperscript{25} The astonished troops, with a mixture of pride, of
satisfaction, and of terror, confessed the voice of their master.

Their angry clamors subsided into silent reverence; and
Valentinian, encompassed with the eagles of the legions, and the
various banners of the cavalry and infantry, was conducted, in
warlike pomp, to the palace of Nice. As he was sensible, however,
of the importance of preventing some rash declaration of the
soldiers, he consulted the assembly of the chiefs; and their real
sentiments were concisely expressed by the generous freedom of
Dagalaiphus. “Most excellent prince,” said that officer, “if you
consider only your family, you have a brother; if you love the
republic, look round for the most deserving of the Romans.”\textsuperscript{26}
The emperor, who suppressed his displeasure, without altering his
intention, slowly proceeded from Nice to Nicomedia and
Constantinople. In one of the suburbs of that capital,\textsuperscript{27} thirty
days after his own elevation, he bestowed the title of Augustus
on his brother Valens;\textsuperscript{2711} and as the boldest patriots were
convinced, that their opposition, without being serviceable to
their country, would be fatal to themselves, the declaration of
his absolute will was received with silent submission. Valens was
now in the thirty-sixth year of his age; but his abilities had
never been exercised in any employment, military or civil; and
his character had not inspired the world with any sanguine
expectations. He possessed, however, one quality, which
recommended him to Valentinian, and preserved the domestic peace
of the empire; devout and grateful attachment to his benefactor,
whose superiority of genius, as well as of authority, Valens
humbly and cheerfully acknowledged in every action of his life.\textsuperscript{28}

\pagenote[24]{Ammianus, in a long, because unseasonable,
digression, (xxvi. l, and Valesius, ad locum,) rashly supposes
that he understands an astronomical question, of which his
readers are ignorant. It is treated with more judgment and
propriety by Censorinus (de Die Natali, c. 20) and Macrobius,
(Saturnal. i. c. 12-16.) The appellation of \textit{Bissextile}, which
marks the inauspicious year, (Augustin. ad Januarium, Epist.
119,) is derived from the \textit{repetition} of the \textit{sixth} day of the
calends of March.}

\pagenote[25]{Valentinian’s first speech is in Ammianus, (xxvi.
2;) concise and sententious in Philostorgius, (l. viii. c. 8.)}

\pagenote[26]{Si tuos amas, Imperator optime, habes fratrem; si
Rempublicam quære quem vestias. Ammian. xxvi. 4. In the division
of the empire, Valentinian retained that sincere counsellor for
himself, (c.6.)}

\pagenote[27]{In suburbano, Ammian. xxvi. 4. The famous
\textit{Hebdomon}, or field of Mars, was distant from Constantinople
either seven stadia, or seven miles. See Valesius, and his
brother, ad loc., and Ducange, Const. l. ii. p. 140, 141, 172,
173.}

\pagenote[2711]{Symmachus praises the liberality of Valentinian
in raising his brother at once to the rank of Augustus, not
training him through the slow and probationary degree of Cæsar.
Exigui animi vices munerum partiuntur, liberalitas desideriis
nihil reliquit. Symm. Orat. p. 7. edit. Niebuhr, 1816, reprinted
from Mai.—M.}

\pagenote[28]{Participem quidem legitimum potestatis; sed in
modum apparitoris morigerum, ut progrediens aperiet textus.
Ammian. xxvi. 4.}

\section{Part \thesection.}

Before Valentinian divided the provinces, he reformed the
administration of the empire. All ranks of subjects, who had been
injured or oppressed under the reign of Julian, were invited to
support their public accusations. The silence of mankind attested
the spotless integrity of the præfect Sallust;\textsuperscript{29} and his own
pressing solicitations, that he might be permitted to retire from
the business of the state, were rejected by Valentinian with the
most honorable expressions of friendship and esteem. But among
the favorites of the late emperor, there were many who had abused
his credulity or superstition; and who could no longer hope to be
protected either by favor or justice.\textsuperscript{30} The greater part of the
ministers of the palace, and the governors of the provinces, were
removed from their respective stations; yet the eminent merit of
some officers was distinguished from the obnoxious crowd; and,
notwithstanding the opposite clamors of zeal and resentment, the
whole proceedings of this delicate inquiry appear to have been
conducted with a reasonable share of wisdom and moderation.\textsuperscript{31}
The festivity of a new reign received a short and suspicious
interruption from the sudden illness of the two princes; but as
soon as their health was restored, they left Constantinople in
the beginning of the spring. In the castle, or palace, of
Mediana, only three miles from Naissus, they executed the solemn
and final division of the Roman empire.\textsuperscript{32} Valentinian bestowed
on his brother the rich præfecture of the \textit{East}, from the Lower
Danube to the confines of Persia; whilst he reserved for his
immediate government the warlike\textsuperscript{3211} præfectures of \textit{Illyricum,
Italy}, and \textit{Gaul}, from the extremity of Greece to the
Caledonian rampart, and from the rampart of Caledonia to the foot
of Mount Atlas. The provincial administration remained on its
former basis; but a double supply of generals and magistrates was
required for two councils, and two courts: the division was made
with a just regard to their peculiar merit and situation, and
seven master-generals were soon created, either of the cavalry or
infantry. When this important business had been amicably
transacted, Valentinian and Valens embraced for the last time.
The emperor of the West established his temporary residence at
Milan; and the emperor of the East returned to Constantinople, to
assume the dominion of fifty provinces, of whose language he was
totally ignorant.\textsuperscript{33}

\pagenote[29]{Notwithstanding the evidence of Zonaras, Suidas,
and the Paschal Chronicle, M. de Tillemont (Hist. des Empereurs,
tom. v. p. 671) \textit{wishes} to disbelieve those stories, si
avantageuses à un payen.}

\pagenote[30]{Eunapius celebrates and exaggerates the sufferings
of Maximus. (p. 82, 83;) yet he allows that the sophist or
magician, the guilty favorite of Julian, and the personal enemy
of Valentinian, was dismissed on the payment of a small fine.}

\pagenote[31]{The loose assertions of a general disgrace
(Zosimus, l. iv. p. 201), are detected and refuted by Tillemont,
(tom. v. p. 21.)}

\pagenote[32]{Ammianus, xxvi. 5.}

\pagenote[3211]{Ipse supra impacati Rhen semibarbaras ripas
raptim vexilla constituens * * Princeps creatus ad difficilem
militiam revertisti. Symm. Orat. 81.—M.}

\pagenote[33]{Ammianus says, in general terms, subagrestis
ingenii, nec bellicis nec liberalibus studiis eruditus. Ammian.
xxxi. 14. The orator Themistius, with the genuine impertinence of
a Greek, wishes for the first time to speak the Latin language,
the dialect of his sovereign. Orat. vi. p. 71.}

The tranquility of the East was soon disturbed by rebellion; and
the throne of Valens was threatened by the daring attempts of a
rival whose affinity to the emperor Julian\textsuperscript{34} was his sole merit,
and had been his only crime. Procopius had been hastily promoted
from the obscure station of a tribune, and a notary, to the joint
command of the army of Mesopotamia; the public opinion already
named him as the successor of a prince who was destitute of
natural heirs; and a vain rumor was propagated by his friends, or
his enemies, that Julian, before the altar of the Moon at Carrhæ,
had privately invested Procopius with the Imperial purple.\textsuperscript{35} He
endeavored, by his dutiful and submissive behavior, to disarm the
jealousy of Jovian; resigned, without a contest, his military
command; and retired, with his wife and family, to cultivate the
ample patrimony which he possessed in the province of Cappadocia.
These useful and innocent occupations were interrupted by the
appearance of an officer with a band of soldiers, who, in the
name of his new sovereigns, Valentinian and Valens, was
despatched to conduct the unfortunate Procopius either to a
perpetual prison or an ignominious death. His presence of mind
procured him a longer respite, and a more splendid fate. Without
presuming to dispute the royal mandate, he requested the
indulgence of a few moments to embrace his weeping family; and
while the vigilance of his guards was relaxed by a plentiful
entertainment, he dexterously escaped to the sea-coast of the
Euxine, from whence he passed over to the country of Bosphorus.
In that sequestered region he remained many months, exposed to
the hardships of exile, of solitude, and of want; his melancholy
temper brooding over his misfortunes, and his mind agitated by
the just apprehension, that, if any accident should discover his
name, the faithless Barbarians would violate, without much
scruple, the laws of hospitality. In a moment of impatience and
despair, Procopius embarked in a merchant vessel, which made sail
for Constantinople; and boldly aspired to the rank of a
sovereign, because he was not allowed to enjoy the security of a
subject. At first he lurked in the villages of Bithynia,
continually changing his habitation and his disguise.\textsuperscript{36} By
degrees he ventured into the capital, trusted his life and
fortune to the fidelity of two friends, a senator and a eunuch,
and conceived some hopes of success, from the intelligence which
he obtained of the actual state of public affairs. The body of
the people was infected with a spirit of discontent: they
regretted the justice and the abilities of Sallust, who had been
imprudently dismissed from the præfecture of the East. They
despised the character of Valens, which was rude without vigor,
and feeble without mildness. They dreaded the influence of his
father-in-law, the patrician Petronius, a cruel and rapacious
minister, who rigorously exacted all the arrears of tribute that
might remain unpaid since the reign of the emperor Aurelian. The
circumstances were propitious to the designs of a usurper. The
hostile measures of the Persians required the presence of Valens
in Syria: from the Danube to the Euphrates the troops were in
motion; and the capital was occasionally filled with the soldiers
who passed or repassed the Thracian Bosphorus. Two cohorts of
Gaul were persuaded to listen to the secret proposals of the
conspirators; which were recommended by the promise of a liberal
donative; and, as they still revered the memory of Julian, they
easily consented to support the hereditary claim of his
proscribed kinsman. At the dawn of day they were drawn up near
the baths of Anastasia; and Procopius, clothed in a purple
garment, more suitable to a player than to a monarch, appeared,
as if he rose from the dead, in the midst of Constantinople. The
soldiers, who were prepared for his reception, saluted their
trembling prince with shouts of joy and vows of fidelity. Their
numbers were soon increased by a band of sturdy peasants,
collected from the adjacent country; and Procopius, shielded by
the arms of his adherents, was successively conducted to the
tribunal, the senate, and the palace. During the first moments of
his tumultuous reign, he was astonished and terrified by the
gloomy silence of the people; who were either ignorant of the
cause, or apprehensive of the event. But his military strength
was superior to any actual resistance: the malcontents flocked to
the standard of rebellion; the poor were excited by the hopes,
and the rich were intimidated by the fear, of a general pillage;
and the obstinate credulity of the multitude was once more
deceived by the promised advantages of a revolution. The
magistrates were seized; the prisons and arsenals broke open; the
gates, and the entrance of the harbor, were diligently occupied;
and, in a few hours, Procopius became the absolute, though
precarious, master of the Imperial city.\textsuperscript{3611} The usurper
improved this unexpected success with some degree of courage and
dexterity. He artfully propagated the rumors and opinions the
most favorable to his interest; while he deluded the populace by
giving audience to the frequent, but imaginary, ambassadors of
distant nations. The large bodies of troops stationed in the
cities of Thrace and the fortresses of the Lower Danube, were
gradually involved in the guilt of rebellion: and the Gothic
princes consented to supply the sovereign of Constantinople with
the formidable strength of several thousand auxiliaries. His
generals passed the Bosphorus, and subdued, without an effort,
the unarmed, but wealthy provinces of Bithynia and Asia. After an
honorable defence, the city and island of Cyzicus yielded to his
power; the renowned legions of the Jovians and Herculeans
embraced the cause of the usurper, whom they were ordered to
crush; and, as the veterans were continually augmented with new
levies, he soon appeared at the head of an army, whose valor, as
well as numbers, were not unequal to the greatness of the
contest. The son of Hormisdas,\textsuperscript{37} a youth of spirit and ability,
condescended to draw his sword against the lawful emperor of the
East; and the Persian prince was immediately invested with the
ancient and extraordinary powers of a Roman Proconsul. The
alliance of Faustina, the widow of the emperor Constantius, who
intrusted herself and her daughter to the hands of the usurper,
added dignity and reputation to his cause. The princess
Constantia, who was then about five years of age, accompanied, in
a litter, the march of the army. She was shown to the multitude
in the arms of her adopted father; and, as often as she passed
through the ranks, the tenderness of the soldiers was inflamed
into martial fury:\textsuperscript{38} they recollected the glories of the house
of Constantine, and they declared, with loyal acclamation, that
they would shed the last drop of their blood in the defence of
the royal infant.\textsuperscript{39}

\pagenote[34]{The uncertain degree of alliance, or consanguinity,
is expressed by the words, cognatus, consobrinus, (see Valesius
ad Ammian. xxiii. 3.) The mother of Procopius might be a sister
of Basilina and Count Julian, the mother and uncle of the
Apostate. Ducange, Fam. Byzantin. p. 49.}

\pagenote[35]{Ammian. xxiii. 3, xxvi. 6. He mentions the report
with much hesitation: susurravit obscurior fama; nemo enim dicti
auctor exstitit verus. It serves, however, to remark, that
Procopius was a Pagan. Yet his religion does not appear to have
promoted, or obstructed, his pretensions.}

\pagenote[36]{One of his retreats was a country-house of
Eunomius, the heretic. The master was absent, innocent, ignorant;
yet he narrowly escaped a sentence of death, and was banished
into the remote parts of Mauritania, (Philostorg. l. ix. c. 5, 8,
and Godefroy’s Dissert. p. 369-378.)}

\pagenote[3611]{It may be suspected, from a fragment of Eunapius,
that the heathen and philosophic party espoused the cause of
Procopius. Heraclius, the Cynic, a man who had been honored by a
philosophic controversy with Julian, striking the ground with his
staff, incited him to courage with the line of Homer Eunapius.
Mai, p. 207 or in Niebuhr’s edition, p. 73.—M.}

\pagenote[37]{Hormisdæ maturo juveni Hormisdæ regalis illius
filio, potestatem Proconsulis detulit; et civilia, more veterum,
et bella, recturo. Ammian. xxvi. 8. The Persian prince escaped
with honor and safety, and was afterwards (A. D. 380) restored to
the same extraordinary office of proconsul of Bithynia,
(Tillemont, Hist. des Empereurs, tom. v. p. 204) I am ignorant
whether the race of Sassan was propagated. I find (A. D. 514) a
pope Hormisdas; but he was a native of Frusino, in Italy, (Pagi
Brev. Pontific. tom. i. p. 247)}

\pagenote[38]{The infant rebel was afterwards the wife of the
emperor Gratian but she died young, and childless. See Ducange,
Fam. Byzantin. p. 48, 59.}

\pagenote[39]{Sequimini culminis summi prosapiam, was the
language of Procopius, who affected to despise the obscure birth,
and fortuitous election of the upstart Pannonian. Ammian. xxvi.
7.}

In the mean while Valentinian was alarmed and perplexed by the
doubtful intelligence of the revolt of the East.\textsuperscript{3911} The
difficulties of a German war forced him to confine his immediate
care to the safety of his own dominions; and, as every channel of
communication was stopped or corrupted, he listened, with
doubtful anxiety, to the rumors which were industriously spread,
that the defeat and death of Valens had left Procopius sole
master of the Eastern provinces. Valens was not dead: but on the
news of the rebellion, which he received at Cæsarea, he basely
despaired of his life and fortune; proposed to negotiate with the
usurper, and discovered his secret inclination to abdicate the
Imperial purple. The timid monarch was saved from disgrace and
ruin by the firmness of his ministers, and their abilities soon
decided in his favor the event of the civil war. In a season of
tranquillity, Sallust had resigned without a murmur; but as soon
as the public safety was attacked, he ambitiously solicited the
preëminence of toil and danger; and the restoration of that
virtuous minister to the præfecture of the East, was the first
step which indicated the repentance of Valens, and satisfied the
minds of the people. The reign of Procopius was apparently
supported by powerful armies and obedient provinces. But many of
the principal officers, military as well as civil, had been
urged, either by motives of duty or interest, to withdraw
themselves from the guilty scene; or to watch the moment of
betraying, and deserting, the cause of the usurper. Lupicinus
advanced by hasty marches, to bring the legions of Syria to the
aid of Valens. Arintheus, who, in strength, beauty, and valor,
excelled all the heroes of the age, attacked with a small troop a
superior body of the rebels. When he beheld the faces of the
soldiers who had served under his banner, he commanded them, with
a loud voice, to seize and deliver up their pretended leader; and
such was the ascendant of his genius, that this extraordinary
order was instantly obeyed.\textsuperscript{40} Arbetio, a respectable veteran of
the great Constantine, who had been distinguished by the honors
of the consulship, was persuaded to leave his retirement, and
once more to conduct an army into the field. In the heat of
action, calmly taking off his helmet, he showed his gray hairs
and venerable countenance: saluted the soldiers of Procopius by
the endearing names of children and companions, and exhorted them
no longer to support the desperate cause of a contemptible
tyrant; but to follow their old commander, who had so often led
them to honor and victory. In the two engagements of Thyatira\textsuperscript{41}
and Nacolia, the unfortunate Procopius was deserted by his
troops, who were seduced by the instructions and example of their
perfidious officers. After wandering some time among the woods
and mountains of Phyrgia, he was betrayed by his desponding
followers, conducted to the Imperial camp, and immediately
beheaded. He suffered the ordinary fate of an unsuccessful
usurper; but the acts of cruelty which were exercised by the
conqueror, under the forms of legal justice, excited the pity and
indignation of mankind.\textsuperscript{42}

\pagenote[3911]{Symmachus describes his embarrassment. “The
Germans are the common enemies of the state, Procopius the
private foe of the Emperor; his first care must be victory, his
second revenge.” Symm. Orat. p. 11.—M.}

\pagenote[40]{Et dedignatus hominem superare certamine
despicabilem, auctoritatis et celsi fiducia corporis ipsis
hostibus jussit, suum vincire rectorem: atque ita turmarum,
antesignanus umbratilis comprensus suorum manibus. The strength
and beauty of Arintheus, the new Hercules, are celebrated by St.
Basil, who supposed that God had created him as an inimitable
model of the human species. The painters and sculptors could not
express his figure: the historians appeared fabulous when they
related his exploits, (Ammian. xxvi. and Vales. ad loc.)}

\pagenote[41]{The same field of battle is placed by Ammianus in
Lycia, and by Zosimus at Thyatira, which are at the distance of
150 miles from each other. But Thyatira alluitur \textit{Lyco}, (Plin.
Hist. Natur. v. 31, Cellarius, Geograph. Antiq. tom. ii. p. 79;)
and the transcribers might easily convert an obscure river into a
well-known province. * Note: Ammianus and Zosimus place the last
battle at Nacolia in \textit{Phrygia;} Ammianus altogether omits the
former battle near Thyatira. Procopius was on his march (iter
tendebat) towards Lycia. See Wagner’s note, in c.—M.}

\pagenote[42]{The adventures, usurpation, and fall of Procopius,
are related, in a regular series, by Ammianus, (xxvi. 6, 7, 8, 9,
10,) and Zosimus, (l. iv. p. 203-210.) They often illustrate, and
seldom contradict, each other. Themistius (Orat. vii. p. 91, 92)
adds some base panegyric; and Euna pius (p. 83, 84) some
malicious satire. ——Symmachus joins with Themistius in praising
the clemency of Valens dic victoriæ moderatus est, quasi contra
se nemo pugnavit. Symm. Orat. p. 12.—M.}

Such indeed are the common and natural fruits of despotism and
rebellion. But the inquisition into the crime of magic,\textsuperscript{4211}
which, under the reign of the two brothers, was so rigorously
prosecuted both at Rome and Antioch, was interpreted as the fatal
symptom, either of the displeasure of Heaven, or of the depravity
of mankind.\textsuperscript{43} Let us not hesitate to indulge a liberal pride,
that, in the present age, the enlightened part of Europe has
abolished\textsuperscript{44} a cruel and odious prejudice, which reigned in every
climate of the globe, and adhered to every system of religious
opinions.\textsuperscript{45} The nations, and the sects, of the Roman world,
admitted with equal credulity, and similar abhorrence, the
reality of that infernal art,\textsuperscript{46} which was able to control the
eternal order of the planets, and the voluntary operations of the
human mind. They dreaded the mysterious power of spells and
incantations, of potent herbs, and execrable rites; which could
extinguish or recall life, inflame the passions of the soul,
blast the works of creation, and extort from the reluctant dæmons
the secrets of futurity. They believed, with the wildest
inconsistency, that this preternatural dominion of the air, of
earth, and of hell, was exercised, from the vilest motives of
malice or gain, by some wrinkled hags and itinerant sorcerers,
who passed their obscure lives in penury and contempt.\textsuperscript{47} The
arts of magic were equally condemned by the public opinion, and
by the laws of Rome; but as they tended to gratify the most
imperious passions of the heart of man, they were continually
proscribed, and continually practised.\textsuperscript{48} An imaginary cause was
capable of producing the most serious and mischievous effects.
The dark predictions of the death of an emperor, or the success
of a conspiracy, were calculated only to stimulate the hopes of
ambition, and to dissolve the ties of fidelity; and the
intentional guilt of magic was aggravated by the actual crimes of
treason and sacrilege.\textsuperscript{49} Such vain terrors disturbed the peace
of society, and the happiness of individuals; and the harmless
flame which insensibly melted a waxen image, might derive a
powerful and pernicious energy from the affrighted fancy of the
person whom it was maliciously designed to represent.\textsuperscript{50} From the
infusion of those herbs, which were supposed to possess a
supernatural influence, it was an easy step to the use of more
substantial poison; and the folly of mankind sometimes became the
instrument, and the mask, of the most atrocious crimes. As soon
as the zeal of informers was encouraged by the ministers of
Valens and Valentinian, they could not refuse to listen to
another charge, too frequently mingled in the scenes of domestic
guilt; a charge of a softer and less malignant nature, for which
the pious, though excessive, rigor of Constantine had recently
decreed the punishment of death.\textsuperscript{51} This deadly and incoherent
mixture of treason and magic, of poison and adultery, afforded
infinite gradations of guilt and innocence, of excuse and
aggravation, which in these proceedings appear to have been
confounded by the angry or corrupt passions of the judges. They
easily discovered that the degree of their industry and
discernment was estimated, by the Imperial court, according to
the number of executions that were furnished from the respective
tribunals. It was not without extreme reluctance that they
pronounced a sentence of acquittal; but they eagerly admitted
such evidence as was stained with perjury, or procured by
torture, to prove the most improbable charges against the most
respectable characters. The progress of the inquiry continually
opened new subjects of criminal prosecution; the audacious
informer, whose falsehood was detected, retired with impunity;
but the wretched victim, who discovered his real or pretended
accomplices, were seldom permitted to receive the price of his
infamy. From the extremity of Italy and Asia, the young, and the
aged, were dragged in chains to the tribunals of Rome and
Antioch. Senators, matrons, and philosophers, expired in
ignominious and cruel tortures. The soldiers, who were appointed
to guard the prisons, declared, with a murmur of pity and
indignation, that their numbers were insufficient to oppose the
flight, or resistance, of the multitude of captives. The
wealthiest families were ruined by fines and confiscations; the
most innocent citizens trembled for their safety; and we may form
some notion of the magnitude of the evil, from the extravagant
assertion of an ancient writer, that, in the obnoxious provinces,
the prisoners, the exiles, and the fugitives, formed the greatest
part of the inhabitants.\textsuperscript{52}

\pagenote[4211]{This infamous inquisition into sorcery and
witchcraft has been of greater influence on human affairs than is
commonly supposed. The persecutions against philosophers and
their libraries was carried on with so much fury, that from this
time (A. D. 374) the names of the Gentile philosophers became
almost extinct; and the Christian philosophy and religion,
particularly in the East, established their ascendency. I am
surprised that Gibbon has not made this observation. Heyne, Note
on Zosimus, l. iv. 14, p. 637. Besides vast heaps of manuscripts
publicly destroyed throughout the East, men of letters burned
their whole libraries, lest some fatal volume should expose them
to the malice of the informers and the extreme penalty of the
law. Amm. Marc. xxix. 11.—M.}

\pagenote[43]{Libanius de ulciscend. Julian. nece, c. ix. p. 158,
159. The sophist deplores the public frenzy, but he does not
(after their deaths) impeach the justice of the emperors.}

\pagenote[44]{The French and English lawyers, of the present age,
allow the \textit{theory}, and deny the \textit{practice}, of witchcraft,
(Denisart, Recueil de Decisions de Jurisprudence, au mot
\textit{Sorciers}, tom. iv. p. 553. Blackstone’s Commentaries, vol. iv.
p. 60.) As private reason always prevents, or outstrips, public
wisdom, the president Montesquieu (Esprit des Loix, l. xii. c. 5,
6) rejects the \textit{existence} of magic.}

\pagenote[45]{See Œuvres de Bayle, tom. iii. p. 567-589. The
sceptic of Rotterdam exhibits, according to his custom, a strange
medley of loose knowledge and lively wit.}

\pagenote[46]{The Pagans distinguished between good and bad
magic, the Theurgic and the Goetic, (Hist. de l’Académie, \&c.,
tom. vii. p. 25.) But they could not have defended this obscure
distinction against the acute logic of Bayle. In the Jewish and
Christian system, \textit{all} dæmons are infernal spirits; and \textit{all}
commerce with them is idolatry, apostasy \&c., which deserves
death and damnation.}

\pagenote[47]{The Canidia of Horace (Carm. l. v. Od. 5, with
Dacier’s and Sanadon’s illustrations) is a vulgar witch. The
Erictho of Lucan (Pharsal. vi. 430-830) is tedious, disgusting,
but sometimes sublime. She chides the delay of the Furies, and
threatens, with tremendous obscurity, to pronounce their real
names; to reveal the true infernal countenance of Hecate; to
invoke the secret powers that lie below hell, \&c.}

\pagenote[48]{Genus hominum potentibus infidum, sperantibus
fallax, quod in civitate nostrâ et vetabitur semper et
retinebitur. Tacit. Hist. i. 22. See Augustin. de Civitate Dei,
l. viii. c. 19, and the Theodosian Code l. ix. tit. xvi., with
Godefroy’s Commentary.}

\pagenote[49]{The persecution of Antioch was occasioned by a
criminal consultation. The twenty-four letters of the alphabet
were arranged round a magic tripod: and a dancing ring, which had
been placed in the centre, pointed to the four first letters in
the name of the future emperor, O. E. O Triangle. Theodorus
(perhaps with many others, who owned the fatal syllables) was
executed. Theodosius succeeded. Lardner (Heathen Testimonies,
vol. iv. p. 353-372) has copiously and fairly examined this dark
transaction of the reign of Valens.}

\pagenote[50]{Limus ut hic durescit, et hæc ut cera liquescit Uno eodemque
igni—Virgil. Bucolic. viii. 80.
Devovet absentes, simulacraque cerea figit. —Ovid. in Epist. Hypsil.
ad Jason 91.

Such vain incantations could affect the mind, and increase the
disease of Germanicus. Tacit. Annal. ii. 69.}

\pagenote[51]{See Heineccius, Antiquitat. Juris Roman. tom. ii.
p. 353, \&c. Cod. Theodosian. l. ix. tit. 7, with Godefroy’s
Commentary.}

\pagenote[52]{The cruel persecution of Rome and Antioch is
described, and most probably exaggerated, by Ammianus (xxvii. 1.
xxix. 1, 2) and Zosimus, (l. iv. p. 216-218.) The philosopher
Maximus, with some justice, was involved in the charge of magic,
(Eunapius in Vit. Sophist. p. 88, 89;) and young Chrysostom, who
had accidentally found one of the proscribed books, gave himself
up for lost, (Tillemont, Hist. des Empereurs, tom. v. p. 340.)}

When Tacitus describes the deaths of the innocent and illustrious
Romans, who were sacrificed to the cruelty of the first Cæsars,
the art of the historian, or the merit of the sufferers, excites
in our breast the most lively sensations of terror, of
admiration, and of pity. The coarse and undistinguishing pencil
of Ammianus has delineated his bloody figures with tedious and
disgusting accuracy. But as our attention is no longer engaged by
the contrast of freedom and servitude, of recent greatness and of
actual misery, we should turn with horror from the frequent
executions, which disgraced, both at Rome and Antioch, the reign
of the two brothers.\textsuperscript{53} Valens was of a timid,\textsuperscript{54} and Valentinian
of a choleric, disposition.\textsuperscript{55} An anxious regard to his personal
safety was the ruling principle of the administration of Valens.
In the condition of a subject, he had kissed, with trembling awe,
the hand of the oppressor; and when he ascended the throne, he
reasonably expected, that the same fears, which had subdued his
own mind, would secure the patient submission of his people. The
favorites of Valens obtained, by the privilege of rapine and
confiscation, the wealth which his economy would have refused.\textsuperscript{56}
They urged, with persuasive eloquence, \textit{that}, in all cases of
treason, suspicion is equivalent to proof; \textit{that} the power
supposes the intention, of mischief; \textit{that} the intention is not
less criminal than the act; and \textit{that} a subject no longer
deserves to live, if his life may threaten the safety, or disturb
the repose, of his sovereign. The judgment of Valentinian was
sometimes deceived, and his confidence abused; but he would have
silenced the informers with a contemptuous smile, had they
presumed to alarm his fortitude by the sound of danger. They
praised his inflexible love of justice; and, in the pursuit of
justice, the emperor was easily tempted to consider clemency as a
weakness, and passion as a virtue. As long as he wrestled with
his equals, in the bold competition of an active and ambitious
life, Valentinian was seldom injured, and never insulted, with
impunity: if his prudence was arraigned, his spirit was
applauded; and the proudest and most powerful generals were
apprehensive of provoking the resentment of a fearless soldier.
After he became master of the world, he unfortunately forgot,
that where no resistance can be made, no courage can be exerted;
and instead of consulting the dictates of reason and magnanimity,
he indulged the furious emotions of his temper, at a time when
they were disgraceful to himself, and fatal to the defenceless
objects of his displeasure. In the government of his household,
or of his empire, slight, or even imaginary, offences—a hasty
word, a casual omission, an involuntary delay—were chastised by a
sentence of immediate death. The expressions which issued the
most readily from the mouth of the emperor of the West were,
“Strike off his head;” “Burn him alive;” “Let him be beaten with
clubs till he expires;”\textsuperscript{57} and his most favored ministers soon
understood, that, by a rash attempt to dispute, or suspend, the
execution of his sanguinary commands, they might involve
themselves in the guilt and punishment of disobedience. The
repeated gratification of this savage justice hardened the mind
of Valentinian against pity and remorse; and the sallies of
passion were confirmed by the habits of cruelty.\textsuperscript{58} He could
behold with calm satisfaction the convulsive agonies of torture
and death; he reserved his friendship for those faithful servants
whose temper was the most congenial to his own. The merit of
Maximin, who had slaughtered the noblest families of Rome, was
rewarded with the royal approbation, and the præfecture of Gaul.

Two fierce and enormous bears, distinguished by the appellations
of \textit{Innocence}, and \textit{Mica Aurea}, could alone deserve to share
the favor of Maximin. The cages of those trusty guards were
always placed near the bed-chamber of Valentinian, who frequently
amused his eyes with the grateful spectacle of seeing them tear
and devour the bleeding limbs of the malefactors who were
abandoned to their rage. Their diet and exercises were carefully
inspected by the Roman emperor; and when \textit{Innocence} had earned
her discharge, by a long course of meritorious service, the
faithful animal was again restored to the freedom of her native
woods.\textsuperscript{59}

\pagenote[53]{Consult the six last books of Ammianus, and more
particularly the portraits of the two royal brothers, (xxx. 8, 9,
xxxi. 14.) Tillemont has collected (tom. v. p. 12-18, p. 127-133)
from all antiquity their virtues and vices.}

\pagenote[54]{The younger Victor asserts, that he was valde
timidus: yet he behaved, as almost every man would do, with
decent resolution at the \textit{head} of an army. The same historian
attempts to prove that his anger was harmless. Ammianus observes,
with more candor and judgment, incidentia crimina ad contemptam
vel læsam principis amplitudinem trahens, in sanguinem sæviebat.}

\pagenote[55]{Cum esset ad acerbitatem naturæ calore propensior.
.. pœnas perignes augebat et gladios. Ammian. xxx. 8. See xxvii.
7}

\pagenote[56]{I have transferred the reproach of avarice from
Valens to his servant. Avarice more properly belongs to ministers
than to kings; in whom that passion is commonly extinguished by
absolute possession.}

\pagenote[57]{He sometimes expressed a sentence of death with a
tone of pleasantry: “Abi, Comes, et muta ei caput, qui sibi
mutari provinciam cupit.” A boy, who had slipped too hastily a
Spartan bound; an armorer, who had made a polished cuirass that
wanted some grains of the legitimate weight, \&c., were the
victims of his fury.}

\pagenote[58]{The innocents of Milan were an agent and three
apparitors, whom Valentinian condemned for signifying a legal
summons. Ammianus (xxvii. 7) strangely supposes, that all who had
been unjustly executed were worshipped as martyrs by the
Christians. His impartial silence does not allow us to believe,
that the great chamberlain Rhodanus was burnt alive for an act of
oppression, (Chron. Paschal. p. 392.) * Note: Ammianus does not
say that they were worshipped as \textit{martyrs}. Quorum memoriam apud
Mediolanum colentes nunc usque Christiani loculos ubi sepulti
sunt, \textit{ad innocentes} appellant. Wagner’s note in loco. Yet if
the next paragraph refers to that transaction, which is not quite
clear. Gibbon is right.—M.}

\pagenote[59]{Ut bene meritam in sylvas jussit abire \textit{Innoxiam}.
Ammian. xxix. and Valesius ad locum.}

\section{Part \thesection.}

But in the calmer moments of reflection, when the mind of Valens
was not agitated by fear, or that of Valentinian by rage, the
tyrant resumed the sentiments, or at least the conduct, of the
father of his country. The dispassionate judgment of the Western
emperor could clearly perceive, and accurately pursue, his own
and the public interest; and the sovereign of the East, who
imitated with equal docility the various examples which he
received from his elder brother, was sometimes guided by the
wisdom and virtue of the præfect Sallust. Both princes invariably
retained, in the purple, the chaste and temperate simplicity
which had adorned their private life; and, under their reign, the
pleasures of the court never cost the people a blush or a sigh.
They gradually reformed many of the abuses of the times of
Constantius; judiciously adopted and improved the designs of
Julian and his successor; and displayed a style and spirit of
legislation which might inspire posterity with the most favorable
opinion of their character and government. It is not from the
master of \textit{Innocence}, that we should expect the tender regard
for the welfare of his subjects, which prompted Valentinian to
condemn the exposition of new-born infants;\textsuperscript{60} and to establish
fourteen skilful physicians, with stipends and privileges, in the
fourteen quarters of Rome. The good sense of an illiterate
soldier founded a useful and liberal institution for the
education of youth, and the support of declining science.\textsuperscript{61} It
was his intention, that the arts of rhetoric and grammar should
be taught in the Greek and Latin languages, in the metropolis of
every province; and as the size and dignity of the school was
usually proportioned to the importance of the city, the academies
of Rome and Constantinople claimed a just and singular
preëminence. The fragments of the literary edicts of Valentinian
imperfectly represent the school of Constantinople, which was
gradually improved by subsequent regulations. That school
consisted of thirty-one professors in different branches of
learning. One philosopher, and two lawyers; five sophists, and
ten grammarians for the Greek, and three orators, and ten
grammarians for the Latin tongue; besides seven scribes, or, as
they were then styled, antiquarians, whose laborious pens
supplied the public library with fair and correct copies of the
classic writers. The rule of conduct, which was prescribed to the
students, is the more curious, as it affords the first outlines
of the form and discipline of a modern university. It was
required, that they should bring proper certificates from the
magistrates of their native province. Their names, professions,
and places of abode, were regularly entered in a public register.

\pagenote[60]{See the Code of Justinian, l. viii. tit. lii. leg.
2. Unusquisque sabolem suam nutriat. Quod si exponendam putaverit
animadversioni quæ constituta est subjacebit. For the present I
shall not interfere in the dispute between Noodt and Binkershoek;
how far, or how long this unnatural practice had been condemned
or abolished by law philosophy, and the more civilized state of
society.}

\pagenote[61]{These salutary institutions are explained in the
Theodosian Code, l. xiii. tit. iii. \textit{De Professoribus et
Medicis}, and l. xiv. tit. ix. \textit{De Studiis liberalibus Urbis
Romæ}. Besides our usual guide, (Godefroy,) we may consult
Giannone, (Istoria di Napoli, tom. i. p. 105-111,) who has
treated the interesting subject with the zeal and curiosity of a
man of latters who studies his domestic history.}

The studious youth were severely prohibited from wasting their
time in feasts, or in the theatre; and the term of their
education was limited to the age of twenty. The præfect of the
city was empowered to chastise the idle and refractory by stripes
or expulsion; and he was directed to make an annual report to the
master of the offices, that the knowledge and abilities of the
scholars might be usefully applied to the public service. The
institutions of Valentinian contributed to secure the benefits of
peace and plenty; and the cities were guarded by the
establishment of the \textit{Defensors;}\textsuperscript{62} freely elected as the
tribunes and advocates of the people, to support their rights,
and to expose their grievances, before the tribunals of the civil
magistrates, or even at the foot of the Imperial throne. The
finances were diligently administered by two princes, who had
been so long accustomed to the rigid economy of a private
fortune; but in the receipt and application of the revenue, a
discerning eye might observe some difference between the
government of the East and of the West. Valens was persuaded,
that royal liberality can be supplied only by public oppression,
and his ambition never aspired to secure, by their actual
distress, the future strength and prosperity of his people.
Instead of increasing the weight of taxes, which, in the space of
forty years, had been gradually doubled, he reduced, in the first
years of his reign, one fourth of the tribute of the East.\textsuperscript{63}
Valentinian appears to have been less attentive and less anxious
to relieve the burdens of his people. He might reform the abuses
of the fiscal administration; but he exacted, without scruple, a
very large share of the private property; as he was convinced,
that the revenues, which supported the luxury of individuals,
would be much more advantageously employed for the defence and
improvement of the state. The subjects of the East, who enjoyed
the present benefit, applauded the indulgence of their prince.
The solid but less splendid, merit of Valentinian was felt and
acknowledged by the subsequent generation.\textsuperscript{64}

\pagenote[62]{Cod. Theodos. l. i. tit. xi. with Godefroy’s
\textit{Paratitlon}, which diligently gleans from the rest of the code.}

\pagenote[63]{Three lines of Ammianus (xxxi. 14) countenance a
whole oration of Themistius, (viii. p. 101-120,) full of
adulation, pedantry, and common-place morality. The eloquent M.
Thomas (tom. i. p. 366-396) has amused himself with celebrating
the virtues and genius of Themistius, who was not unworthy of the
age in which he lived.}

\pagenote[64]{Zosimus, l. iv. p. 202. Ammian. xxx. 9. His
reformation of costly abuses might entitle him to the praise of,
in provinciales admodum parcus, tributorum ubique molliens
sarcinas. By some his frugality was styled avarice, (Jerom.
Chron. p. 186)}

But the most honorable circumstance of the character of
Valentinian, is the firm and temperate impartiality which he
uniformly preserved in an age of religious contention. His strong
sense, unenlightened, but uncorrupted, by study, declined, with
respectful indifference, the subtle questions of theological
debate. The government of the \textit{Earth} claimed his vigilance, and
satisfied his ambition; and while he remembered that he was the
disciple of the church, he never forgot that he was the sovereign
of the clergy. Under the reign of an apostate, he had signalized
his zeal for the honor of Christianity: he allowed to his
subjects the privilege which he had assumed for himself; and they
might accept, with gratitude and confidence, the general
toleration which was granted by a prince addicted to passion, but
incapable of fear or of disguise.\textsuperscript{65} The Pagans, the Jews, and
all the various sects which acknowledged the divine authority of
Christ, were protected by the laws from arbitrary power or
popular insult; nor was any mode of worship prohibited by
Valentinian, except those secret and criminal practices, which
abused the name of religion for the dark purposes of vice and
disorder. The art of magic, as it was more cruelly punished, was
more strictly proscribed: but the emperor admitted a formal
distinction to protect the ancient methods of divination, which
were approved by the senate, and exercised by the Tuscan
haruspices. He had condemned, with the consent of the most
rational Pagans, the license of nocturnal sacrifices; but he
immediately admitted the petition of Prætextatus, proconsul of
Achaia, who represented, that the life of the Greeks would become
dreary and comfortless, if they were deprived of the invaluable
blessing of the Eleusinian mysteries. Philosophy alone can boast,
(and perhaps it is no more than the boast of philosophy,) that
her gentle hand is able to eradicate from the human mind the
latent and deadly principle of fanaticism. But this truce of
twelve years, which was enforced by the wise and vigorous
government of Valentinian, by suspending the repetition of mutual
injuries, contributed to soften the manners, and abate the
prejudices, of the religious factions.

\pagenote[65]{Testes sunt leges a me in exordio Imperii mei datæ;
quibus unicuique quod animo imbibisset colendi libera facultas
tributa est. Cod. Theodos. l. ix. tit. xvi. leg. 9. To this
declaration of Valentinian, we may add the various testimonies of
Ammianus, (xxx. 9,) Zosimus, (l. iv. p. 204,) and Sozomen, (l.
vi. c. 7, 21.) Baronius would naturally blame such rational
toleration, (Annal. Eccles A. D. 370, No. 129-132, A. D. 376, No.
3, 4.) ——Comme il s’était prescrit pour règle de ne point se
mêler de disputes de religion, son histoire est presque
entièrement dégagée des affaires ecclésiastiques. Le Beau. iii.
214.—M.}

The friend of toleration was unfortunately placed at a distance
from the scene of the fiercest controversies. As soon as the
Christians of the West had extricated themselves from the snares
of the creed of Rimini, they happily relapsed into the slumber of
orthodoxy; and the small remains of the Arian party, that still
subsisted at Sirmium or Milan, might be considered rather as
objects of contempt than of resentment. But in the provinces of
the East, from the Euxine to the extremity of Thebais, the
strength and numbers of the hostile factions were more equally
balanced; and this equality, instead of recommending the counsels
of peace, served only to perpetuate the horrors of religious war.
The monks and bishops supported their arguments by invectives;
and their invectives were sometimes followed by blows. Athanasius
still reigned at Alexandria; the thrones of Constantinople and
Antioch were occupied by Arian prelates, and every episcopal
vacancy was the occasion of a popular tumult. The Homoousians
were fortified by the reconciliation of fifty-nine Macelonian, or
Semi-Arian, bishops; but their secret reluctance to embrace the
divinity of the Holy Ghost, clouded the splendor of the triumph;
and the declaration of Valens, who, in the first years of his
reign, had imitated the impartial conduct of his brother, was an
important victory on the side of Arianism. The two brothers had
passed their private life in the condition of catechumens; but
the piety of Valens prompted him to solicit the sacrament of
baptism, before he exposed his person to the dangers of a Gothic
war. He naturally addressed himself to Eudoxus,\textsuperscript{66} \textsuperscript{6611} bishop of
the Imperial city; and if the ignorant monarch was instructed by
that Arian pastor in the principles of heterodox theology, his
misfortune, rather than his guilt, was the inevitable consequence
of his erroneous choice. Whatever had been the determination of
the emperor, he must have offended a numerous party of his
Christian subjects; as the leaders both of the Homoousians and of
the Arians believed, that, if they were not suffered to reign,
they were most cruelly injured and oppressed. After he had taken
this decisive step, it was extremely difficult for him to
preserve either the virtue, or the reputation of impartiality. He
never aspired, like Constantius, to the fame of a profound
theologian; but as he had received with simplicity and respect
the tenets of Euxodus, Valens resigned his conscience to the
direction of his ecclesiastical guides, and promoted, by the
influence of his authority, the reunion of the \textit{Athanasian
heretics} to the body of the Catholic church. At first, he pitied
their blindness; by degrees he was provoked at their obstinacy;
and he insensibly hated those sectaries to whom he was an object
of hatred.\textsuperscript{67} The feeble mind of Valens was always swayed by the
persons with whom he familiarly conversed; and the exile or
imprisonment of a private citizen are the favors the most readily
granted in a despotic court. Such punishments were frequently
inflicted on the leaders of the Homoousian party; and the
misfortune of fourscore ecclesiastics of Constantinople, who,
perhaps accidentally, were burned on shipboard, was imputed to
the cruel and premeditated malice of the emperor, and his Arian
ministers. In every contest, the Catholics (if we may anticipate
that name) were obliged to pay the penalty of their own faults,
and of those of their adversaries. In every election, the claims
of the Arian candidate obtained the preference; and if they were
opposed by the majority of the people, he was usually supported
by the authority of the civil magistrate, or even by the terrors
of a military force. The enemies of Athanasius attempted to
disturb the last years of his venerable age; and his temporary
retreat to his father’s sepulchre has been celebrated as a fifth
exile. But the zeal of a great people, who instantly flew to
arms, intimidated the præfect: and the archbishop was permitted
to end his life in peace and in glory, after a reign of
forty-seven years. The death of Athanasius was the signal of the
persecution of Egypt; and the Pagan minister of Valens, who
forcibly seated the worthless Lucius on the archiepiscopal
throne, purchased the favor of the reigning party, by the blood
and sufferings of their Christian brethren. The free toleration
of the heathen and Jewish worship was bitterly lamented, as a
circumstance which aggravated the misery of the Catholics, and
the guilt of the impious tyrant of the East.\textsuperscript{68}

\pagenote[66]{Eudoxus was of a mild and timid disposition. When
he baptized Valens, (A. D. 367,) he must have been extremely old;
since he had studied theology fifty-five years before, under
Lucian, a learned and pious martyr. Philostorg. l. ii. c. 14-16,
l. iv. c. 4, with Godefroy, p 82, 206, and Tillemont, Mém.
Eccles. tom. v. p. 471-480, \&c.}

\pagenote[6611]{Through the influence of his wife say the
ecclesiastical writers.—M.}

\pagenote[67]{Gregory Nazianzen (Orat. xxv. p. 432) insults the
persecuting spirit of the Arians, as an infallible symptom of
error and heresy.}

\pagenote[68]{This sketch of the ecclesiastical government of
Valens is drawn from Socrates, (l. iv.,) Sozomen, (l. vi.,)
Theodoret, (l. iv.,) and the immense compilations of Tillemont,
(particularly tom. vi. viii. and ix.)}

The triumph of the orthodox party has left a deep stain of
persecution on the memory of Valens; and the character of a
prince who derived his virtues, as well as his vices, from a
feeble understanding and a pusillanimous temper, scarcely
deserves the labor of an apology. Yet candor may discover some
reasons to suspect that the ecclesiastical ministers of Valens
often exceeded the orders, or even the intentions, of their
master; and that the real measure of facts has been very
liberally magnified by the vehement declamation and easy
credulity of his antagonists.\textsuperscript{69} 1. The silence of Valentinian
may suggest a probable argument that the partial severities,
which were exercised in the name and provinces of his colleague,
amounted only to some obscure and inconsiderable deviations from
the established system of religious toleration: and the judicious
historian, who has praised the equal temper of the elder brother,
has not thought himself obliged to contrast the tranquillity of
the West with the cruel persecution of the East.\textsuperscript{70} 2. Whatever
credit may be allowed to vague and distant reports, the
character, or at least the behavior, of Valens, may be most
distinctly seen in his personal transactions with the eloquent
Basil, archbishop of Cæsarea, who had succeeded Athanasius in the
management of the Trinitarian cause.\textsuperscript{71} The circumstantial
narrative has been composed by the friends and admirers of Basil;
and as soon as we have stripped away a thick coat of rhetoric and
miracle, we shall be astonished by the unexpected mildness of the
Arian tyrant, who admired the firmness of his character, or was
apprehensive, if he employed violence, of a general revolt in the
province of Cappadocia. The archbishop, who asserted, with
inflexible pride,\textsuperscript{72} the truth of his opinions, and the dignity
of his rank, was left in the free possession of his conscience
and his throne. The emperor devoutly assisted at the solemn
service of the cathedral; and, instead of a sentence of
banishment, subscribed the donation of a valuable estate for the
use of a hospital, which Basil had lately founded in the
neighborhood of Cæsarea.\textsuperscript{73} 3. I am not able to discover, that
any law (such as Theodosius afterwards enacted against the
Arians) was published by Valens against the Athanasian sectaries;
and the edict which excited the most violent clamors, may not
appear so extremely reprehensible. The emperor had observed, that
several of his subjects, gratifying their lazy disposition under
the pretence of religion, had associated themselves with the
monks of Egypt; and he directed the count of the East to drag
them from their solitude; and to compel these deserters of
society to accept the fair alternative of renouncing their
temporal possessions, or of discharging the public duties of men
and citizens.\textsuperscript{74} The ministers of Valens seem to have extended
the sense of this penal statute, since they claimed a right of
enlisting the young and ablebodied monks in the Imperial armies.
A detachment of cavalry and infantry, consisting of three
thousand men, marched from Alexandria into the adjacent desert of
Nitria,\textsuperscript{75} which was peopled by five thousand monks. The soldiers
were conducted by Arian priests; and it is reported, that a
considerable slaughter was made in the monasteries which
disobeyed the commands of their sovereign.\textsuperscript{76}

\pagenote[69]{Dr. Jortin (Remarks on Ecclesiastical History, vol.
iv. p. 78) has already conceived and intimated the same
suspicion.}

\pagenote[70]{This reflection is so obvious and forcible, that
Orosius (l. vii. c. 32, 33,) delays the persecution till after
the death of Valentinian. Socrates, on the other hand, supposes,
(l. iii. c. 32,) that it was appeased by a philosophical oration,
which Themistius pronounced in the year 374, (Orat. xii. p. 154,
in Latin only.) Such contradictions diminish the evidence, and
reduce the term, of the persecution of Valens.}

\pagenote[71]{Tillemont, whom I follow and abridge, has extracted
(Mém. Eccles. tom. viii. p. 153-167) the most authentic
circumstances from the Panegyrics of the two Gregories; the
brother, and the friend, of Basil. The letters of Basil himself
(Dupin, Bibliothèque, Ecclesiastique, tom. ii. p. 155-180) do not
present the image of a very lively persecution.}

\pagenote[72]{Basilius Cæsariensis episcopus Cappadociæ clarus
habetur... qui multa continentiæ et ingenii bona uno superbiæ
malo perdidit. This irreverent passage is perfectly in the style
and character of St. Jerom. It does not appear in Scaliger’s
edition of his Chronicle; but Isaac Vossius found it in some old
Mss. which had not been reformed by the monks.}

\pagenote[73]{This noble and charitable foundation (almost a new
city) surpassed in merit, if not in greatness, the pyramids, or
the walls of Babylon. It was principally intended for the
reception of lepers, (Greg. Nazianzen, Orat. xx. p. 439.)}

\pagenote[74]{Cod. Theodos. l. xii. tit. i. leg. 63. Godefroy
(tom. iv. p. 409-413) performs the duty of a commentator and
advocate. Tillemont (Mém. Eccles. tom. viii. p. 808) \textit{supposes} a
second law to excuse his orthodox friends, who had misrepresented
the edict of Valens, and suppressed the liberty of choice.}

\pagenote[75]{See D’Anville, Description de l’Egypte, p. 74.
Hereafter I shall consider the monastic institutions.}

\pagenote[76]{Socrates, l. iv. c. 24, 25. Orosius, l. vii. c. 33.
Jerom. in Chron. p. 189, and tom. ii. p. 212. The monks of Egypt
performed many miracles, which prove the truth of their faith.
Right, says Jortin, (Remarks, vol iv. p. 79,) but what proves the
truth of those miracles.}

The strict regulations which have been framed by the wisdom of
modern legislators to restrain the wealth and avarice of the
clergy, may be originally deduced from the example of the emperor
Valentinian. His edict,\textsuperscript{77} addressed to Damasus, bishop of Rome,
was publicly read in the churches of the city. He admonished the
ecclesiastics and monks not to frequent the houses of widows and
virgins; and menaced their disobedience with the animadversion of
the civil judge. The director was no longer permitted to receive
any gift, or legacy, or inheritance, from the liberality of his
spiritual-daughter: every testament contrary to this edict was
declared null and void; and the illegal donation was confiscated
for the use of the treasury. By a subsequent regulation, it
should seem, that the same provisions were extended to nuns and
bishops; and that all persons of the ecclesiastical order were
rendered incapable of receiving any testamentary gifts, and
strictly confined to the natural and legal rights of inheritance.
As the guardian of domestic happiness and virtue, Valentinian
applied this severe remedy to the growing evil. In the capital of
the empire, the females of noble and opulent houses possessed a
very ample share of independent property: and many of those
devout females had embraced the doctrines of Christianity, not
only with the cold assent of the understanding, but with the
warmth of affection, and perhaps with the eagerness of fashion.
They sacrificed the pleasures of dress and luxury; and renounced,
for the praise of chastity, the soft endearments of conjugal
society. Some ecclesiastic, of real or apparent sanctity, was
chosen to direct their timorous conscience, and to amuse the
vacant tenderness of their heart: and the unbounded confidence,
which they hastily bestowed, was often abused by knaves and
enthusiasts; who hastened from the extremities of the East, to
enjoy, on a splendid theatre, the privileges of the monastic
profession. By their contempt of the world, they insensibly
acquired its most desirable advantages; the lively attachment,
perhaps of a young and beautiful woman, the delicate plenty of an
opulent household, and the respectful homage of the slaves, the
freedmen, and the clients of a senatorial family. The immense
fortunes of the Roman ladies were gradually consumed in lavish
alms and expensive pilgrimages; and the artful monk, who had
assigned himself the first, or possibly the sole place, in the
testament of his spiritual daughter, still presumed to declare,
with the smooth face of hypocrisy, that \textit{he} was only the
instrument of charity, and the steward of the poor. The
lucrative, but disgraceful, trade,\textsuperscript{78} which was exercised by the
clergy to defraud the expectations of the natural heirs, had
provoked the indignation of a superstitious age: and two of the
most respectable of the Latin fathers very honestly confess, that
the ignominious edict of Valentinian was just and necessary; and
that the Christian priests had deserved to lose a privilege,
which was still enjoyed by comedians, charioteers, and the
ministers of idols. But the wisdom and authority of the
legislator are seldom victorious in a contest with the vigilant
dexterity of private interest; and Jerom, or Ambrose, might
patiently acquiesce in the justice of an ineffectual or salutary
law. If the ecclesiastics were checked in the pursuit of personal
emolument, they would exert a more laudable industry to increase
the wealth of the church; and dignify their covetousness with the
specious names of piety and patriotism.\textsuperscript{79}

\pagenote[77]{Cod. Theodos. l. xvi. tit. ii. leg. 20. Godefroy,
(tom. vi. p. 49,) after the example of Baronius, impartially
collects all that the fathers have said on the subject of this
important law; whose spirit was long afterwards revived by the
emperor Frederic II., Edward I. of England, and other Christian
princes who reigned after the twelfth century.}

\pagenote[78]{The expressions which I have used are temperate and
feeble, if compared with the vehement invectives of Jerom, (tom.
i. p. 13, 45, 144, \&c.) In \textit{his} turn he was reproached with the
guilt which he imputed to his brother monks; and the
\textit{Sceleratus}, the \textit{Versipellis}, was publicly accused as the
lover of the widow Paula, (tom. ii. p. 363.) He undoubtedly
possessed the affection, both of the mother and the daughter; but
he declares that he never abused his influence to any selfish or
sensual purpose.}

\pagenote[79]{Pudet dicere, sacerdotes idolorum, mimi et aurigæ,
et scorta, hæreditates capiunt: solis \textit{clericis} ac \textit{monachis}
hac lege prohibetur. Et non prohibetur a persecutoribus, sed a
principibus Christianis. Nec de lege queror; sed doleo cur
\textit{meruerimus} hanc legem. Jerom (tom. i. p. 13) discreetly
insinuates the secret policy of his patron Damasus.}

Damasus, bishop of Rome, who was constrained to stigmatize the
avarice of his clergy by the publication of the law of
Valentinian, had the good sense, or the good fortune, to engage
in his service the zeal and abilities of the learned Jerom; and
the grateful saint has celebrated the merit and purity of a very
ambiguous character.\textsuperscript{80} But the splendid vices of the church of
Rome, under the reign of Valentinian and Damasus, have been
curiously observed by the historian Ammianus, who delivers his
impartial sense in these expressive words: “The præfecture of
Juventius was accompanied with peace and plenty, but the
tranquillity of his government was soon disturbed by a bloody
sedition of the distracted people. The ardor of Damasus and
Ursinus, to seize the episcopal seat, surpassed the ordinary
measure of human ambition. They contended with the rage of party;
the quarrel was maintained by the wounds and death of their
followers; and the præfect, unable to resist or appease the
tumult, was constrained, by superior violence, to retire into the
suburbs. Damasus prevailed: the well-disputed victory remained on
the side of his faction; one hundred and thirty-seven dead bodies\textsuperscript{81}
were found in the \textit{Basilica} of Sicininus,\textsuperscript{82} where the
Christians hold their religious assemblies; and it was long
before the angry minds of the people resumed their accustomed
tranquillity. When I consider the splendor of the capital, I am
not astonished that so valuable a prize should inflame the
desires of ambitious men, and produce the fiercest and most
obstinate contests. The successful candidate is secure, that he
will be enriched by the offerings of matrons;\textsuperscript{83} that, as soon as
his dress is composed with becoming care and elegance, he may
proceed, in his chariot, through the streets of Rome;\textsuperscript{84} and that
the sumptuousness of the Imperial table will not equal the
profuse and delicate entertainments provided by the taste, and at
the expense, of the Roman pontiffs. How much more rationally
(continues the honest Pagan) would those pontiffs consult their
true happiness, if, instead of alleging the greatness of the city
as an excuse for their manners, they would imitate the exemplary
life of some provincial bishops, whose temperance and sobriety,
whose mean apparel and downcast looks, recommend their pure and
modest virtue to the Deity and his true worshippers!”\textsuperscript{85} The
schism of Damasus and Ursinus was extinguished by the exile of
the latter; and the wisdom of the præfect Prætextatus\textsuperscript{86} restored
the tranquillity of the city. Prætextatus was a philosophic
Pagan, a man of learning, of taste, and politeness; who disguised
a reproach in the form of a jest, when he assured Damasus, that
if he could obtain the bishopric of Rome, he himself would
immediately embrace the Christian religion.\textsuperscript{87} This lively
picture of the wealth and luxury of the popes in the fourth
century becomes the more curious, as it represents the
intermediate degree between the humble poverty of the apostolic
fishermen, and the royal state of a temporal prince, whose
dominions extend from the confines of Naples to the banks of the
Po.

\pagenote[80]{Three words of Jerom, \textit{sanctæ memoriæ Damasus}
(tom. ii. p. 109,) wash away all his stains, and blind the devout
eyes of Tillemont. (Mem Eccles. tom. viii. p. 386-424.)}

\pagenote[81]{Jerom himself is forced to allow, crudelissimæ
interfectiones diversi sexûs perpetratæ, (in Chron. p. 186.) But
an original \textit{libel}, or petition of two presbyters of the adverse
party, has unaccountably escaped. They affirm that the doors of
the Basilica were burnt, and that the roof was untiled; that
Damasus marched at the head of his own clergy, grave-diggers,
charioteers, and hired gladiators; that none of \textit{his} party were
killed, but that one hundred and sixty dead bodies were found.
This petition is published by the P. Sirmond, in the first volume
of his work.}

\pagenote[82]{The \textit{Basilica} of Sicininus, or Liberius, is
probably the church of Sancta Maria Maggiore, on the Esquiline
hill. Baronius, A. D. 367 No. 3; and Donatus, Roma Antiqua et
Nova, l. iv. c. 3, p. 462.}

\pagenote[83]{The enemies of Damasus styled him \textit{Auriscalpius
Matronarum} the ladies’ ear-scratcher.}

\pagenote[84]{Gregory Nazianzen (Orat. xxxii. p. 526) describes
the pride and luxury of the prelates who reigned in the Imperial
cities; their gilt car, fiery steeds, numerous train, \&c. The
crowd gave way as to a wild beast.}

\pagenote[85]{Ammian. xxvii. 3. Perpetuo Numini, \textit{verisque} ejus
cultoribus. The incomparable pliancy of a polytheist!}

\pagenote[86]{Ammianus, who makes a fair report of his præfecture
(xxvii. 9) styles him præclaræ indolis, gravitatisque senator,
(xxii. 7, and Vales. ad loc.) A curious inscription (Grutor MCII.
No. 2) records, in two columns, his religious and civil honors.
In one line he was Pontiff of the Sun, and of Vesta, Augur,
Quindecemvir, Hierophant, \&c., \&c. In the other, 1. Quæstor
candidatus, more probably titular. 2. Prætor. 3. Corrector of
Tuscany and Umbria. 4. Consular of Lusitania. 5. Proconsul of
Achaia. 6. Præfect of Rome. 7. Prætorian præfect of Italy. 8. Of
Illyricum. 9. Consul elect; but he died before the beginning of
the year 385. See Tillemont, Hist. des Empereurs, tom v. p. 241,
736.}

\pagenote[87]{Facite me Romanæ urbis episcopum; et ero protinus
Christianus (Jerom, tom. ii. p. 165.) It is more than probable
that Damasus would not have purchased his conversion at such a
price.}

\section{Part \thesection.}

When the suffrage of the generals and of the army committed the
sceptre of the Roman empire to the hands of Valentinian, his
reputation in arms, his military skill and experience, and his
rigid attachment to the forms, as well as spirit, of ancient
discipline, were the principal motives of their judicious choice.

The eagerness of the troops, who pressed him to nominate his
colleague, was justified by the dangerous situation of public
affairs; and Valentinian himself was conscious, that the
abilities of the most active mind were unequal to the defence of
the distant frontiers of an invaded monarchy. As soon as the
death of Julian had relieved the Barbarians from the terror of
his name, the most sanguine hopes of rapine and conquest excited
the nations of the East, of the North, and of the South. Their
inroads were often vexatious, and sometimes formidable; but,
during the twelve years of the reign of Valentinian, his firmness
and vigilance protected his own dominions; and his powerful
genius seemed to inspire and direct the feeble counsels of his
brother. Perhaps the method of annals would more forcibly express
the urgent and divided cares of the two emperors; but the
attention of the reader, likewise, would be distracted by a
tedious and desultory narrative. A separate view of the five
great theatres of war; I. Germany; II. Britain; III. Africa; IV.
The East; and, V. The Danube; will impress a more distinct image
of the military state of the empire under the reigns of
Valentinian and Valens.

I. The ambassadors of the Alemanni had been offended by the harsh
and haughty behavior of Ursacius, master of the offices;\textsuperscript{88} who
by an act of unseasonable parsimony, had diminished the value, as
well as the quantity, of the presents to which they were
entitled, either from custom or treaty, on the accession of a new
emperor. They expressed, and they communicated to their
countrymen, their strong sense of the national affront. The
irascible minds of the chiefs were exasperated by the suspicion
of contempt; and the martial youth crowded to their standard.
Before Valentinian could pass the Alps, the villages of Gaul were
in flames; before his general Degalaiphus could encounter the
Alemanni, they had secured the captives and the spoil in the
forests of Germany. In the beginning of the ensuing year, the
military force of the whole nation, in deep and solid columns,
broke through the barrier of the Rhine, during the severity of a
northern winter. Two Roman counts were defeated and mortally
wounded; and the standard of the Heruli and Batavians fell into
the hands of the conquerors, who displayed, with insulting shouts
and menaces, the trophy of their victory. The standard was
recovered; but the Batavians had not redeemed the shame of their
disgrace and flight in the eyes of their severe judge. It was the
opinion of Valentinian, that his soldiers must learn to fear
their commander, before they could cease to fear the enemy. The
troops were solemnly assembled; and the trembling Batavians were
enclosed within the circle of the Imperial army. Valentinian then
ascended his tribunal; and, as if he disdained to punish
cowardice with death, he inflicted a stain of indelible ignominy
on the officers, whose misconduct and pusillanimity were found to
be the first occasion of the defeat. The Batavians were degraded
from their rank, stripped of their arms, and condemned to be sold
for slaves to the highest bidder. At this tremendous sentence,
the troops fell prostrate on the ground, deprecated the
indignation of their sovereign, and protested, that, if he would
indulge them in another trial, they would approve themselves not
unworthy of the name of Romans, and of his soldiers. Valentinian,
with affected reluctance, yielded to their entreaties; the
Batavians resumed their arms, and with their arms, the invincible
resolution of wiping away their disgrace in the blood of the
Alemanni.\textsuperscript{89} The principal command was declined by Dagalaiphus;
and that experienced general, who had represented, perhaps with
too much prudence, the extreme difficulties of the undertaking,
had the mortification, before the end of the campaign, of seeing
his rival Jovinus convert those difficulties into a decisive
advantage over the scattered forces of the Barbarians. At the
head of a well-disciplined army of cavalry, infantry, and light
troops, Jovinus advanced, with cautious and rapid steps, to
Scarponna,\textsuperscript{90} \textsuperscript{9011} in the territory of Metz, where he surprised a
large division of the Alemanni, before they had time to run to
their arms; and flushed his soldiers with the confidence of an
easy and bloodless victory. Another division, or rather army, of
the enemy, after the cruel and wanton devastation of the adjacent
country, reposed themselves on the shady banks of the Moselle.
Jovinus, who had viewed the ground with the eye of a general,
made a silent approach through a deep and woody vale, till he
could distinctly perceive the indolent security of the Germans.
Some were bathing their huge limbs in the river; others were
combing their long and flaxen hair; others again were swallowing
large draughts of rich and delicious wine. On a sudden they heard
the sound of the Roman trumpet; they saw the enemy in their camp.
Astonishment produced disorder; disorder was followed by flight
and dismay; and the confused multitude of the bravest warriors
was pierced by the swords and javelins of the legionaries and
auxiliaries. The fugitives escaped to the third, and most
considerable, camp, in the Catalonian plains, near Châlons in
Champagne: the straggling detachments were hastily recalled to
their standard; and the Barbarian chiefs, alarmed and admonished
by the fate of their companions, prepared to encounter, in a
decisive battle, the victorious forces of the lieutenant of
Valentinian. The bloody and obstinate conflict lasted a whole
summer’s day, with equal valor, and with alternate success. The
Romans at length prevailed, with the loss of about twelve hundred
men. Six thousand of the Alemanni were slain, four thousand were
wounded; and the brave Jovinus, after chasing the flying remnant
of their host as far as the banks of the Rhine, returned to
Paris, to receive the applause of his sovereign, and the ensigns
of the consulship for the ensuing year.\textsuperscript{91} The triumph of the
Romans was indeed sullied by their treatment of the captive king,
whom they hung on a gibbet, without the knowledge of their
indignant general. This disgraceful act of cruelty, which might
be imputed to the fury of the troops, was followed by the
deliberate murder of Withicab, the son of Vadomair; a German
prince, of a weak and sickly constitution, but of a daring and
formidable spirit. The domestic assassin was instigated and
protected by the Romans;\textsuperscript{92} and the violation of the laws of
humanity and justice betrayed their secret apprehension of the
weakness of the declining empire. The use of the dagger is seldom
adopted in public councils, as long as they retain any confidence
in the power of the sword.

\pagenote[88]{Ammian, xxvi. 5. Valesius adds a long and good note
on the master of the offices.}

\pagenote[89]{Ammian. xxvii. 1. Zosimus, l. iv. p. 208. The
disgrace of the Batavians is suppressed by the contemporary
soldier, from a regard for military honor, which could not affect
a Greek rhetorician of the succeeding age.}

\pagenote[90]{See D’Anville, Notice de l’Ancienne Gaule, p. 587.
The name of the Moselle, which is not specified by Ammianus, is
clearly understood by Mascou, (Hist. of the Ancient Germans, vii.
2)}

\pagenote[9011]{Charpeigne on the Moselle. Mannert—M.}

\pagenote[91]{The battles are described by Ammianus, (xxvii. 2,)
and by Zosimus, (l. iv. p. 209,) who supposes Valentinian to have
been present.}

\pagenote[92]{Studio solicitante nostrorum, occubuit. Ammian
xxvii. 10.}

While the Alemanni appeared to be humbled by their recent
calamities, the pride of Valentinian was mortified by the
unexpected surprisal of Moguntiacum, or Mentz, the principal city
of the Upper Germany. In the unsuspicious moment of a Christian
festival,\textsuperscript{9211} Rando, a bold and artful chieftain, who had long
meditated his attempt, suddenly passed the Rhine; entered the
defenceless town, and retired with a multitude of captives of
either sex. Valentinian resolved to execute severe vengeance on
the whole body of the nation. Count Sebastian, with the bands of
Italy and Illyricum, was ordered to invade their country, most
probably on the side of Rhætia. The emperor in person,
accompanied by his son Gratian, passed the Rhine at the head of a
formidable army, which was supported on both flanks by Jovinus
and Severus, the two masters-general of the cavalry and infantry
of the West. The Alemanni, unable to prevent the devastation of
their villages, fixed their camp on a lofty, and almost
inaccessible, mountain, in the modern duchy of Wirtemberg, and
resolutely expected the approach of the Romans. The life of
Valentinian was exposed to imminent danger by the intrepid
curiosity with which he persisted to explore some secret and
unguarded path. A troop of Barbarians suddenly rose from their
ambuscade: and the emperor, who vigorously spurred his horse down
a steep and slippery descent, was obliged to leave behind him his
armor-bearer, and his helmet, magnificently enriched with gold
and precious stones. At the signal of the general assault, the
Roman troops encompassed and ascended the mountain of Solicinium
on three different sides.\textsuperscript{9212} Every step which they gained,
increased their ardor, and abated the resistance of the enemy:
and after their united forces had occupied the summit of the
hill, they impetuously urged the Barbarians down the northern
descent, where Count Sebastian was posted to intercept their
retreat. After this signal victory, Valentinian returned to his
winter quarters at Treves; where he indulged the public joy by
the exhibition of splendid and triumphal games.\textsuperscript{93} But the wise
monarch, instead of aspiring to the conquest of Germany, confined
his attention to the important and laborious defence of the
Gallic frontier, against an enemy whose strength was renewed by a
stream of daring volunteers, which incessantly flowed from the
most distant tribes of the North.\textsuperscript{94} The banks of the Rhine\textsuperscript{9411}
from its source to the straits of the ocean, were closely planted
with strong castles and convenient towers; new works, and new
arms, were invented by the ingenuity of a prince who was skilled
in the mechanical arts; and his numerous levies of Roman and
Barbarian youth were severely trained in all the exercises of
war. The progress of the work, which was sometimes opposed by
modest representations, and sometimes by hostile attempts,
secured the tranquillity of Gaul during the nine subsequent years
of the administration of Valentinian.\textsuperscript{95}

\pagenote[9211]{Probably Easter. Wagner.—M.}

\pagenote[9212]{Mannert is unable to fix the position of
Solicinium. Haefelin (in Comm Acad Elect. Palat. v. 14)
conjectures Schwetzingen, near Heidelberg. See Wagner’s note. St.
Martin, Sultz in Wirtemberg, near the sources of the Neckar St.
Martin, iii. 339.—M.}

\pagenote[93]{The expedition of Valentinian is related by
Ammianus, (xxvii. 10;) and celebrated by Ausonius, (Mosell. 421,
\&c.,) who foolishly supposes, that the Romans were ignorant of
the sources of the Danube.}

\pagenote[94]{Immanis enim natio, jam inde ab incunabulis primis
varietate casuum imminuta; ita sæpius adolescit, ut fuisse longis
sæculis æstimetur intacta. Ammianus, xxviii. 5. The Count de Buat
(Hist. des Peuples de l’Europe, tom. vi. p. 370) ascribes the
fecundity of the Alemanni to their easy adoption of strangers.
——Note: “This explanation,” says Mr. Malthus, “only removes the
difficulty a little farther off. It makes the earth rest upon the
tortoise, but does not tell us on what the tortoise rests. We may
still ask what northern reservoir supplied this incessant stream
of daring adventurers. Montesquieu’s solution of the problem
will, I think, hardly be admitted, (Grandeur et Décadence des
Romains, c. 16, p. 187.) * * * The whole difficulty, however, is
at once removed, if we apply to the German nations, at that time,
a fact which is so generally known to have occurred in America,
and suppose that, when not checked by wars and famine, they
increased at a rate that would double their numbers in
twenty-five or thirty years. The propriety, and even the
necessity, of applying this rate of increase to the inhabitants
of ancient Germany, will strikingly appear from that most
valuable picture of their manners which has been left us by
Tacitus, (Tac. de Mor. Germ. 16 to 20.) * * * With these manners,
and a habit of enterprise and emigration, which would naturally
remove all fears about providing for a family, it is difficult to
conceive a society with a stronger principle of increase in it,
and we see at once that prolific source of armies and colonies
against which the force of the Roman empire so long struggled
with difficulty, and under which it ultimately sunk. It is not
probable that, for two periods together, or even for one, the
population within the confines of Germany ever doubled itself in
twenty-five years. Their perpetual wars, the rude state of
agriculture, and particularly the very strange custom adopted by
most of the tribes of marking their barriers by extensive
deserts, would prevent any very great actual increase of numbers.
At no one period could the country be called well peopled, though
it was often redundant in population. * * * Instead of clearing
their forests, draining their swamps, and rendering their soil
fit to support an extended population, they found it more
congenial to their martial habits and impatient dispositions to
go in quest of food, of plunder, or of glory, into other
countries.” Malthus on Population, i. p. 128.—G.}

\pagenote[9411]{The course of the Neckar was likewise strongly
guarded. The hyperbolical eulogy of Symmachus asserts that the
Neckar first became known to the Romans by the conquests and
fortifications of Valentinian. Nunc primum victoriis tuis
externus fluvius publicatur. Gaudeat servitute, captivus
innotuit. Symm. Orat. p. 22.—M.}

\pagenote[95]{Ammian. xxviii. 2. Zosimus, l. iv. p. 214. The
younger Victor mentions the mechanical genius of Valentinian,
nova arma meditari fingere terra seu limo simulacra.}

That prudent emperor, who diligently practised the wise maxims of
Diocletian, was studious to foment and excite the intestine
divisions of the tribes of Germany. About the middle of the
fourth century, the countries, perhaps of Lusace and Thuringia,
on either side of the Elbe, were occupied by the vague dominion
of the Burgundians; a warlike and numerous people,\textsuperscript{9511} of the
Vandal race,\textsuperscript{96} whose obscure name insensibly swelled into a
powerful kingdom, and has finally settled on a flourishing
province. The most remarkable circumstance in the ancient manners
of the Burgundians appears to have been the difference of their
civil and ecclesiastical constitution. The appellation of
\textit{Hendinos} was given to the king or general, and the title of
\textit{Sinistus} to the high priest, of the nation. The person of the
priest was sacred, and his dignity perpetual; but the temporal
government was held by a very precarious tenure. If the events of
war accuses the courage or conduct of the king, he was
immediately deposed; and the injustice of his subjects made him
responsible for the fertility of the earth, and the regularity of
the seasons, which seemed to fall more properly within the
sacerdotal department.\textsuperscript{97} The disputed possession of some
salt-pits\textsuperscript{98} engaged the Alemanni and the Burgundians in frequent
contests: the latter were easily tempted, by the secret
solicitations and liberal offers of the emperor; and their
fabulous descent from the Roman soldiers, who had formerly been
left to garrison the fortresses of Drusus, was admitted with
mutual credulity, as it was conducive to mutual interest.\textsuperscript{99} An
army of fourscore thousand Burgundians soon appeared on the banks
of the Rhine; and impatiently required the support and subsidies
which Valentinian had promised: but they were amused with excuses
and delays, till at length, after a fruitless expectation, they
were compelled to retire. The arms and fortifications of the
Gallic frontier checked the fury of their just resentment; and
their massacre of the captives served to imbitter the hereditary
feud of the Burgundians and the Alemanni. The inconstancy of a
wise prince may, perhaps, be explained by some alteration of
circumstances; and perhaps it was the original design of
Valentinian to intimidate, rather than to destroy; as the balance
of power would have been equally overturned by the extirpation of
either of the German nations. Among the princes of the Alemanni,
Macrianus, who, with a Roman name, had assumed the arts of a
soldier and a statesman, deserved his hatred and esteem. The
emperor himself, with a light and unencumbered band, condescended
to pass the Rhine, marched fifty miles into the country, and
would infallibly have seized the object of his pursuit, if his
judicious measures had not been defeated by the impatience of the
troops. Macrianus was afterwards admitted to the honor of a
personal conference with the emperor; and the favors which he
received, fixed him, till the hour of his death, a steady and
sincere friend of the republic.\textsuperscript{100}

\pagenote[9511]{According to the general opinion, the Burgundians
formed a Gothic o Vandalic tribe, who, from the banks of the
Lower Vistula, made incursions, on one side towards Transylvania,
on the other towards the centre of Germany. All that remains of
the Burgundian language is Gothic. * * * Nothing in their customs
indicates a different origin. Malte Brun, Geog. tom. i. p. 396.
(edit. 1831.)—M.}

\pagenote[96]{Bellicosos et pubis immensæ viribus affluentes; et
ideo metuendos finitimis universis. Ammian. xxviii. 5.}

\pagenote[97]{I am always apt to suspect historians and
travellers of improving extraordinary facts into general laws.
Ammianus ascribes a similar custom to Egypt; and the Chinese have
imputed it to the Ta-tsin, or Roman empire, (De Guignes, Hist.
des Huns, tom. ii. part. 79.)}

\pagenote[98]{Salinarum finiumque causa Alemannis sæpe jurgabant.
Ammian xxviii. 5. Possibly they disputed the possession of the
\textit{Sala}, a river which produced salt, and which had been the
object of ancient contention. Tacit. Annal. xiii. 57, and Lipsius
ad loc.}

\pagenote[99]{Jam inde temporibus priscis sobolem se esse Romanam
Burgundii sciunt: and the vague tradition gradually assumed a
more regular form, (Oros. l. vii. c. 32.) It is annihilated by
the decisive authority of Pliny, who composed the History of
Drusus, and served in Germany, (Plin. Secund. Epist. iii. 5,)
within sixty years after the death of that hero. \textit{Germanorum
genera} quinque; Vindili, quorum pars \textit{Burgundiones}, \&c., (Hist.
Natur. iv. 28.)}

\pagenote[100]{The wars and negotiations relative to the
Burgundians and Alemanni, are distinctly related by Ammianus
Marcellinus, (xxviii. 5, xxix 4, xxx. 3.) Orosius, (l. vii. c.
32,) and the Chronicles of Jerom and Cassiodorus, fix some dates,
and add some circumstances.}

The land was covered by the fortifications of Valentinian; but
the sea-coast of Gaul and Britain was exposed to the depredations
of the Saxons. That celebrated name, in which we have a dear and
domestic interest, escaped the notice of Tacitus; and in the maps
of Ptolemy, it faintly marks the narrow neck of the Cimbric
peninsula, and three small islands towards the mouth of the Elbe.\textsuperscript{101}
This contracted territory, the present duchy of Sleswig, or
perhaps of Holstein, was incapable of pouring forth the
inexhaustible swarms of Saxons who reigned over the ocean, who
filled the British island with their language, their laws, and
their colonies; and who so long defended the liberty of the North
against the arms of Charlemagne.\textsuperscript{102} The solution of this
difficulty is easily derived from the similar manners, and loose
constitution, of the tribes of Germany; which were blended with
each other by the slightest accidents of war or friendship. The
situation of the native Saxons disposed them to embrace the
hazardous professions of fishermen and pirates; and the success
of their first adventures would naturally excite the emulation of
their bravest countrymen, who were impatient of the gloomy
solitude of their woods and mountains. Every tide might float
down the Elbe whole fleets of canoes, filled with hardy and
intrepid associates, who aspired to behold the unbounded prospect
of the ocean, and to taste the wealth and luxury of unknown
worlds. It should seem probable, however, that the most numerous
auxiliaries of the Saxons were furnished by the nations who dwelt
along the shores of the Baltic. They possessed arms and ships,
the art of navigation, and the habits of naval war; but the
difficulty of issuing through the northern columns of Hercules\textsuperscript{103}
(which, during several months of the year, are obstructed
with ice) confined their skill and courage within the limits of a
spacious lake. The rumor of the successful armaments which sailed
from the mouth of the Elbe, would soon provoke them to cross the
narrow isthmus of Sleswig, and to launch their vessels on the
great sea. The various troops of pirates and adventurers, who
fought under the same standard, were insensibly united in a
permanent society, at first of rapine, and afterwards of
government. A military confederation was gradually moulded into a
national body, by the gentle operation of marriage and
consanguinity; and the adjacent tribes, who solicited the
alliance, accepted the name and laws, of the Saxons. If the fact
were not established by the most unquestionable evidence, we
should appear to abuse the credulity of our readers, by the
description of the vessels in which the Saxon pirates ventured to
sport in the waves of the German Ocean, the British Channel, and
the Bay of Biscay. The keel of their large flat-bottomed boats
were framed of light timber, but the sides and upper works
consisted only of wicker, with a covering of strong hides.\textsuperscript{104} In
the course of their slow and distant navigations, they must
always have been exposed to the danger, and very frequently to
the misfortune, of shipwreck; and the naval annals of the Saxons
were undoubtedly filled with the accounts of the losses which
they sustained on the coasts of Britain and Gaul. But the daring
spirit of the pirates braved the perils both of the sea and of
the shore: their skill was confirmed by the habits of enterprise;
the meanest of their mariners was alike capable of handling an
oar, of rearing a sail, or of conducting a vessel, and the Saxons
rejoiced in the appearance of a tempest, which concealed their
design, and dispersed the fleets of the enemy.\textsuperscript{105} After they had
acquired an accurate knowledge of the maritime provinces of the
West, they extended the scene of their depredations, and the most
sequestered places had no reason to presume on their security.
The Saxon boats drew so little water that they could easily
proceed fourscore or a hundred miles up the great rivers; their
weight was so inconsiderable, that they were transported on
wagons from one river to another; and the pirates who had entered
the mouth of the Seine, or of the Rhine, might descend, with the
rapid stream of the Rhone, into the Mediterranean. Under the
reign of Valentinian, the maritime provinces of Gaul were
afflicted by the Saxons: a military count was stationed for the
defence of the sea-coast, or Armorican limit; and that officer,
who found his strength, or his abilities, unequal to the task,
implored the assistance of Severus, master-general of the
infantry. The Saxons, surrounded and outnumbered, were forced to
relinquish their spoil, and to yield a select band of their tall
and robust youth to serve in the Imperial armies. They stipulated
only a safe and honorable retreat; and the condition was readily
granted by the Roman general, who meditated an act of perfidy,\textsuperscript{106}
imprudent as it was inhuman, while a Saxon remained alive,
and in arms, to revenge the fate of their countrymen. The
premature eagerness of the infantry, who were secretly posted in
a deep valley, betrayed the ambuscade; and they would perhaps
have fallen the victims of their own treachery, if a large body
of cuirassiers, alarmed by the noise of the combat, had not
hastily advanced to extricate their companions, and to overwhelm
the undaunted valor of the Saxons. Some of the prisoners were
saved from the edge of the sword, to shed their blood in the
amphitheatre; and the orator Symmachus complains, that
twenty-nine of those desperate savages, by strangling themselves
with their own hands, had disappointed the amusement of the
public. Yet the polite and philosophic citizens of Rome were
impressed with the deepest horror, when they were informed, that
the Saxons consecrated to the gods the tithe of their \textit{human}
spoil; and that they ascertained by lot the objects of the
barbarous sacrifice.\textsuperscript{107}

\pagenote[101]{At the northern extremity of the peninsula, (the
Cimbric promontory of Pliny, iv. 27,) Ptolemy fixes the remnant
of the \textit{Cimbri}. He fills the interval between the \textit{Saxons} and
the Cimbri with six obscure tribes, who were united, as early as
the sixth century, under the national appellation of \textit{Danes}. See
Cluver. German. Antiq. l. iii. c. 21, 22, 23.}

\pagenote[102]{M. D’Anville (Establissement des Etats de
l’Europe, \&c., p. 19-26) has marked the extensive limits of the
Saxony of Charlemagne.}

\pagenote[103]{The fleet of Drusus had failed in their attempt to
pass, or even to approach, the \textit{Sound}, (styled, from an obvious
resemblance, the columns of Hercules,) and the naval enterprise
was never resumed, (Tacit. de Moribus German. c. 34.) The
knowledge which the Romans acquired of the naval powers of the
Baltic, (c. 44, 45) was obtained by their land journeys in search
of amber.}

\pagenote[104]{Quin et Aremoricus piratam \textit{Saxona} tractus
Sperabat; cui pelle salum sulcare Britannum
Ludus; et assuto glaucum mare findere lembo.
Sidon. in Panegyr. Avit. 369.

The genius of Cæsar imitated, for a particular service, these
rude, but light vessels, which were likewise used by the natives
of Britain. (Comment. de Bell. Civil. i. 51, and Guichardt,
Nouveaux Mémoires Militaires, tom. ii. p. 41, 42.) The British
vessels would now astonish the genius of Cæsar.}

\pagenote[105]{The best original account of the Saxon pirates may
be found in Sidonius Apollinaris, (l. viii. epist. 6, p. 223,
edit. Sirmond,) and the best commentary in the Abbé du Bos,
(Hist. Critique de la Monarchie Françoise, \&c. tom. i. l. i. c.
16, p. 148-155. See likewise p. 77, 78.)}

\pagenote[106]{Ammian. (xxviii. 5) justifies this breach of faith
to pirates and robbers; and Orosius (l. vii. c. 32) more clearly
expresses their real guilt; virtute atque agilitate terribeles.}

\pagenote[107]{Symmachus (l. ii. epist. 46) still presumes to
mention the sacred name of Socrates and philosophy. Sidonius,
bishop of Clermont, might condemn, (l. viii. epist. 6,) with
\textit{less} inconsistency, the human sacrifices of the Saxons.}

II. The fabulous colonies of Egyptians and Trojans, of
Scandinavians and Spaniards, which flattered the pride, and
amused the credulity, of our rude ancestors, have insensibly
vanished in the light of science and philosophy.\textsuperscript{108} The present
age is satisfied with the simple and rational opinion, that the
islands of Great Britain and Ireland were gradually peopled from
the adjacent continent of Gaul. From the coast of Kent, to the
extremity of Caithness and Ulster, the memory of a Celtic origin
was distinctly preserved, in the perpetual resemblance of
language, of religion, and of manners; and the peculiar
characters of the British tribes might be naturally ascribed to
the influence of accidental and local circumstances.\textsuperscript{109} The
Roman Province was reduced to the state of civilized and peaceful
servitude; the rights of savage freedom were contracted to the
narrow limits of Caledonia. The inhabitants of that northern
region were divided, as early as the reign of Constantine,
between the two great tribes of the Scots and of the Picts,\textsuperscript{110}
who have since experienced a very different fortune. The power,
and almost the memory, of the Picts have been extinguished by
their successful rivals; and the Scots, after maintaining for
ages the dignity of an independent kingdom, have multiplied, by
an equal and voluntary union, the honors of the English name. The
hand of nature had contributed to mark the ancient distinctions
of the Scots and Picts. The former were the men of the hills, and
the latter those of the plain. The eastern coast of Caledonia may
be considered as a level and fertile country, which, even in a
rude state of tillage, was capable of producing a considerable
quantity of corn; and the epithet of \textit{cruitnich}, or
wheat-eaters, expressed the contempt or envy of the carnivorous
highlander. The cultivation of the earth might introduce a more
accurate separation of property, and the habits of a sedentary
life; but the love of arms and rapine was still the ruling
passion of the Picts; and their warriors, who stripped themselves
for a day of battle, were distinguished, in the eyes of the
Romans, by the strange fashion of painting their naked bodies
with gaudy colors and fantastic figures. The western part of
Caledonia irregularly rises into wild and barren hills, which
scarcely repay the toil of the husbandman, and are most
profitably used for the pasture of cattle. The highlanders were
condemned to the occupations of shepherds and hunters; and, as
they seldom were fixed to any permanent habitation, they acquired
the expressive name of Scots, which, in the Celtic tongue, is
said to be equivalent to that of \textit{wanderers}, or \textit{vagrants}. The
inhabitants of a barren land were urged to seek a fresh supply of
food in the waters. The deep lakes and bays which intersect their
country, are plentifully supplied with fish; and they gradually
ventured to cast their nets in the waves of the ocean. The
vicinity of the Hebrides, so profusely scattered along the
western coast of Scotland, tempted their curiosity, and improved
their skill; and they acquired, by slow degrees, the art, or
rather the habit, of managing their boats in a tempestuous sea,
and of steering their nocturnal course by the light of the
well-known stars. The two bold headlands of Caledonia almost
touch the shores of a spacious island, which obtained, from its
luxuriant vegetation, the epithet of \textit{Green;} and has preserved,
with a slight alteration, the name of Erin, or Ierne, or Ireland.
It is \textit{probable}, that in some remote period of antiquity, the
fertile plains of Ulster received a colony of hungry Scots; and
that the strangers of the North, who had dared to encounter the
arms of the legions, spread their conquests over the savage and
unwarlike natives of a solitary island. It is \textit{certain}, that, in
the declining age of the Roman empire, Caledonia, Ireland, and
the Isle of Man, were inhabited by the Scots, and that the
kindred tribes, who were often associated in military enterprise,
were deeply affected by the various accidents of their mutual
fortunes. They long cherished the lively tradition of their
common name and origin; and the missionaries of the Isle of
Saints, who diffused the light of Christianity over North
Britain, established the vain opinion, that their Irish
countrymen were the natural, as well as spiritual, fathers of the
Scottish race. The loose and obscure tradition has been preserved
by the venerable Bede, who scattered some rays of light over the
darkness of the eighth century. On this slight foundation, a huge
superstructure of fable was gradually reared, by the bards and
the monks; two orders of men, who equally abused the privilege of
fiction. The Scottish nation, with mistaken pride, adopted their
Irish genealogy; and the annals of a long line of imaginary kings
have been adorned by the fancy of Boethius, and the classic
elegance of Buchanan.\textsuperscript{111}

\pagenote[108]{In the beginning of the last century, the learned
Camden was obliged to undermine, with respectful scepticism, the
romance of \textit{Brutus}, the Trojan; who is now buried in silent
oblivion with \textit{Scota}, the daughter of Pharaoh, and her numerous
progeny. Yet I am informed, that some champions of the \textit{Milesian
colony} may still be found among the original natives of Ireland.
A people dissatisfied with their present condition, grasp at any
visions of their past or future glory.}

\pagenote[109]{Tacitus, or rather his father-in-law, Agricola,
might remark the German or Spanish complexion of some British
tribes. But it was their sober, deliberate opinion: “In universum
tamen æstimanti Gallos cicinum solum occupâsse credibile est.
Eorum sacra deprehendas.... ermo haud multum diversus,” (in Vit.
Agricol. c. xi.) Cæsar had observed their common religion,
(Comment. de Bello Gallico, vi. 13;) and in his time the
emigration from the Belgic Gaul was a recent, or at least an
historical event, (v. 10.) Camden, the British Strabo, has
modestly ascertained our genuine antiquities, (Britannia, vol. i.
Introduction, p. ii.—xxxi.)}

\pagenote[110]{In the dark and doubtful paths of Caledonian
antiquity, I have chosen for my guides two learned and ingenious
Highlanders, whom their birth and education had peculiarly
qualified for that office. See Critical Dissertations on the
Origin and Antiquities, \&c., of the Caledonians, by Dr. John
Macpherson, London 1768, in 4to.; and Introduction to the History
of Great Britain and Ireland, by James Macpherson, Esq., London
1773, in 4to., third edit. Dr. Macpherson was a minister in the
Isle of Sky: and it is a circumstance honorable for the present
age, that a work, replete with erudition and criticism, should
have been composed in the most remote of the Hebrides.}

\pagenote[111]{The Irish descent of the Scots has been revived in
the last moments of its decay, and strenuously supported, by the
Rev. Mr. Whitaker, (Hist. of Manchester, vol. i. p. 430, 431; and
Genuine History of the Britons asserted, \&c., p. 154-293) Yet he
acknowledges, 1. \textit{That} the Scots of Ammianus Marcellinus (A.D.
340) were already settled in Caledonia; and that the Roman
authors do not afford any hints of their emigration from another
country. 2. \textit{That} all the accounts of such emigrations, which
have been asserted or received, by Irish bards, Scotch
historians, or English antiquaries, (Buchanan, Camden, Usher,
Stillingfleet, \&c.,) are totally fabulous. 3. \textit{That} three of the
Irish tribes, which are mentioned by Ptolemy, (A.D. 150,) were of
Caledonian extraction. 4. \textit{That} a younger branch of Caledonian
princes, of the house of Fingal, acquired and possessed the
monarchy of Ireland. After these concessions, the remaining
difference between Mr. Whitaker and his adversaries is minute and
obscure. The \textit{genuine history}, which he produces, of a Fergus,
the cousin of Ossian, who was transplanted (A.D. 320) from
Ireland to Caledonia, is built on a conjectural supplement to the
Erse poetry, and the feeble evidence of Richard of Cirencester, a
monk of the fourteenth century. The lively spirit of the learned
and ingenious antiquarian has tempted him to forget the nature of
a question, which he so \textit{vehemently} debates, and so \textit{absolutely}
decides. * Note: This controversy has not slumbered since the
days of Gibbon. We have strenuous advocates of the Phœnician
origin of the Irish, and each of the old theories, with several
new ones, maintains its partisans. It would require several pages
fairly to bring down the dispute to our own days, and perhaps we
should be no nearer to any satisfactory theory than Gibbon
was.—M.}

\section{Part \thesection.}

Six years after the death of Constantine, the destructive inroads
of the Scots and Picts required the presence of his youngest son,
who reigned in the Western empire. Constans visited his British
dominions: but we may form some estimate of the importance of his
achievements, by the language of panegyric, which celebrates only
his triumph over the elements or, in other words, the good
fortune of a safe and easy passage from the port of Boulogne to
the harbor of Sandwich.\textsuperscript{112} The calamities which the afflicted
provincials continued to experience, from foreign war and
domestic tyranny, were aggravated by the feeble and corrupt
administration of the eunuchs of Constantius; and the transient
relief which they might obtain from the virtues of Julian, was
soon lost by the absence and death of their benefactor. The sums
of gold and silver, which had been painfully collected, or
liberally transmitted, for the payment of the troops, were
intercepted by the avarice of the commanders; discharges, or, at
least, exemptions, from the military service, were publicly sold;
the distress of the soldiers, who were injuriously deprived of
their legal and scanty subsistence, provoked them to frequent
desertion; the nerves of discipline were relaxed, and the
highways were infested with robbers.\textsuperscript{113} The oppression of the
good, and the impunity of the wicked, equally contributed to
diffuse through the island a spirit of discontent and revolt; and
every ambitious subject, every desperate exile, might entertain a
reasonable hope of subverting the weak and distracted government
of Britain. The hostile tribes of the North, who detested the
pride and power of the King of the World, suspended their
domestic feuds; and the Barbarians of the land and sea, the
Scots, the Picts, and the Saxons, spread themselves with rapid
and irresistible fury, from the wall of Antoninus to the shores
of Kent. Every production of art and nature, every object of
convenience and luxury, which they were incapable of creating by
labor or procuring by trade, was accumulated in the rich and
fruitful province of Britain.\textsuperscript{114} A philosopher may deplore the
eternal discords of the human race, but he will confess, that the
desire of spoil is a more rational provocation than the vanity of
conquest. From the age of Constantine to the Plantagenets, this
rapacious spirit continued to instigate the poor and hardy
Caledonians; but the same people, whose generous humanity seems
to inspire the songs of Ossian, was disgraced by a savage
ignorance of the virtues of peace, and of the laws of war. Their
southern neighbors have felt, and perhaps exaggerated, the cruel
depredations of the Scots and Picts;\textsuperscript{115} and a valiant tribe of
Caledonia, the Attacotti,\textsuperscript{116} the enemies, and afterwards the
soldiers, of Valentinian, are accused, by an eye-witness, of
delighting in the taste of human flesh. When they hunted the
woods for prey, it is said, that they attacked the shepherd
rather than his flock; and that they curiously selected the most
delicate and brawny parts, both of males and females, which they
prepared for their horrid repasts.\textsuperscript{117} If, in the neighborhood of
the commercial and literary town of Glasgow, a race of cannibals
has really existed, we may contemplate, in the period of the
Scottish history, the opposite extremes of savage and civilized
life. Such reflections tend to enlarge the circle of our ideas;
and to encourage the pleasing hope, that New Zealand may produce,
in some future age, the Hume of the Southern Hemisphere.

\pagenote[112]{Hyeme tumentes ac sævientes undas calcâstis Oceani
sub remis vestris;... insperatam imperatoris faciem Britannus
expavit. Julius Fermicus Maternus de Errore Profan. Relig. p.
464. edit. Gronov. ad calcem Minuc. Fæl. See Tillemont, (Hist.
des Empereurs, tom. iv. p. 336.)}

\pagenote[113]{Libanius, Orat. Parent. c. xxxix. p. 264. This
curious passage has escaped the diligence of our British
antiquaries.}

\pagenote[114]{The Caledonians praised and coveted the gold, the
steeds, the lights, \&c., of the \textit{stranger}. See Dr. Blair’s
Dissertation on Ossian, vol ii. p. 343; and Mr. Macpherson’s
Introduction, p. 242-286.}

\pagenote[115]{Lord Lyttelton has circumstantially related,
(History of Henry II. vol. i. p. 182,) and Sir David Dalrymple
has slightly mentioned, (Annals of Scotland, vol. i. p. 69,) a
barbarous inroad of the Scots, at a time (A.D. 1137) when law,
religion, and society must have softened their primitive
manners.}

\pagenote[116]{Attacotti bellicosa hominum natio. Ammian. xxvii.
8. Camden (Introduct. p. clii.) has restored their true name in
the text of Jerom. The bands of Attacotti, which Jerom had seen
in Gaul, were afterwards stationed in Italy and Illyricum,
(Notitia, S. viii. xxxix. xl.)}

\pagenote[117]{Cum ipse adolescentulus in Gallia viderim
Attacottos (or Scotos) gentem Britannicam humanis vesci carnibus;
et cum per silvas porcorum greges, et armentorum percudumque
reperiant, pastorum \textit{nates} et feminarum \textit{papillas} solere
abscindere; et has solas ciborum delicias arbitrari. Such is the
evidence of Jerom, (tom. ii. p. 75,) whose veracity I find no
reason to question. * Note: See Dr. Parr’s works, iii. 93, where
he questions the propriety of Gibbon’s translation of this
passage. The learned doctor approves of the version proposed by a
Mr. Gaches, who would make out that it was the delicate parts of
the swine and the cattle, which were eaten by these ancestors of
the Scotch nation. I confess that even to acquit them of this
charge. I cannot agree to the new version, which, in my opinion,
is directly contrary both to the meaning of the words, and the
general sense of the passage. But I would suggest, did Jerom, as
a boy, accompany these savages in any of their hunting
expeditions? If he did not, how could he be an eye-witness of
this practice? The Attacotti in Gaul must have been in the
service of Rome. Were they permitted to indulge these cannibal
propensities at the expense, not of the flocks, but of the
shepherds of the provinces? These sanguinary trophies of plunder
would scarce’y have been publicly exhibited in a Roman city or a
Roman camp. I must leave the hereditary pride of our northern
neighbors at issue with the veracity of St. Jerom.—M.}

Every messenger who escaped across the British Channel, conveyed
the most melancholy and alarming tidings to the ears of
Valentinian; and the emperor was soon informed that the two
military commanders of the province had been surprised and cut
off by the Barbarians. Severus, count of the domestics, was
hastily despatched, and as suddenly recalled, by the court of
Treves. The representations of Jovinus served only to indicate
the greatness of the evil; and, after a long and serious
consultation, the defence, or rather the recovery, of Britain was
intrusted to the abilities of the brave Theodosius. The exploits
of that general, the father of a line of emperors, have been
celebrated, with peculiar complacency, by the writers of the age:
but his real merit deserved their applause; and his nomination
was received, by the army and province, as a sure presage of
approaching victory. He seized the favorable moment of
navigation, and securely landed the numerous and veteran bands of
the Heruli and Batavians, the Jovians and the Victors. In his
march from Sandwich to London, Theodosius defeated several
parties of the Barbarians, released a multitude of captives, and,
after distributing to his soldiers a small portion of the spoil,
established the fame of disinterested justice, by the restitution
of the remainder to the rightful proprietors. The citizens of
London, who had almost despaired of their safety, threw open
their gates; and as soon as Theodosius had obtained from the
court of Treves the important aid of a military lieutenant, and a
civil governor, he executed, with wisdom and vigor, the laborious
task of the deliverance of Britain. The vagrant soldiers were
recalled to their standard; an edict of amnesty dispelled the
public apprehensions; and his cheerful example alleviated the
rigor of martial discipline. The scattered and desultory warfare
of the Barbarians, who infested the land and sea, deprived him of
the glory of a signal victory; but the prudent spirit, and
consummate art, of the Roman general, were displayed in the
operations of two campaigns, which successively rescued every
part of the province from the hands of a cruel and rapacious
enemy. The splendor of the cities, and the security of the
fortifications, were diligently restored, by the paternal care of
Theodosius; who with a strong hand confined the trembling
Caledonians to the northern angle of the island; and perpetuated,
by the name and settlement of the new province of \textit{Valentia}, the
glories of the reign of Valentinian.\textsuperscript{118} The voice of poetry and
panegyric may add, perhaps with some degree of truth, that the
unknown regions of Thule were stained with the blood of the
Picts; that the oars of Theodosius dashed the waves of the
Hyperborean ocean; and that the distant Orkneys were the scene of
his naval victory over the Saxon pirates.\textsuperscript{119} He left the
province with a fair, as well as splendid, reputation; and was
immediately promoted to the rank of master-general of the
cavalry, by a prince who could applaud, without envy, the merit
of his servants. In the important station of the Upper Danube,
the conqueror of Britain checked and defeated the armies of the
Alemanni, before he was chosen to suppress the revolt of Africa.

\pagenote[118]{Ammianus has concisely represented (xx. l. xxvi.
4, xxvii. 8 xxviii. 3) the whole series of the British war.}

\pagenote[119]{Horrescit.... ratibus.... impervia Thule. Ille....
nec falso nomine Pictos Edomuit. Scotumque vago mucrone secutus,
Fregit Hyperboreas remis audacibus undas. Claudian, in iii. Cons.
Honorii, ver. 53, \&c—Madurunt Saxone fuso Orcades: incaluit
Pictorum sanguine Thule, Scotorum cumulos flevit glacialis Ierne.
In iv. Cons. Hon. ver. 31, \&c. ——See likewise Pacatus, (in
Panegyr. Vet. xii. 5.) But it is not easy to appreciate the
intrinsic value of flattery and metaphor. Compare the \textit{British}
victories of Bolanus (Statius, Silv. v. 2) with his real
character, (Tacit. in Vit. Agricol. c. 16.)}

III. The prince who refuses to be the judge, instructs the people
to consider him as the accomplice, of his ministers. The military
command of Africa had been long exercised by Count Romanus, and
his abilities were not inadequate to his station; but, as sordid
interest was the sole motive of his conduct, he acted, on most
occasions, as if he had been the enemy of the province, and the
friend of the Barbarians of the desert. The three flourishing
cities of Oea, Leptis, and Sobrata, which, under the name of
Tripoli, had long constituted a federal union,\textsuperscript{120} were obliged,
for the first time, to shut their gates against a hostile
invasion; several of their most honorable citizens were surprised
and massacred; the villages, and even the suburbs, were pillaged;
and the vines and fruit trees of that rich territory were
extirpated by the malicious savages of Getulia. The unhappy
provincials implored the protection of Romanus; but they soon
found that their military governor was not less cruel and
rapacious than the Barbarians. As they were incapable of
furnishing the four thousand camels, and the exorbitant present,
which he required, before he would march to the assistance of
Tripoli; his demand was equivalent to a refusal, and he might
justly be accused as the author of the public calamity. In the
annual assembly of the three cities, they nominated two deputies,
to lay at the feet of Valentinian the customary offering of a
gold victory; and to accompany this tribute of duty, rather than
of gratitude, with their humble complaint, that they were ruined
by the enemy, and betrayed by their governor. If the severity of
Valentinian had been rightly directed, it would have fallen on
the guilty head of Romanus. But the count, long exercised in the
arts of corruption, had despatched a swift and trusty messenger
to secure the venal friendship of Remigius, master of the
offices. The wisdom of the Imperial council was deceived by
artifice; and their honest indignation was cooled by delay. At
length, when the repetition of complaint had been justified by
the repetition of public misfortunes, the notary Palladius was
sent from the court of Treves, to examine the state of Africa,
and the conduct of Romanus. The rigid impartiality of Palladius
was easily disarmed: he was tempted to reserve for himself a part
of the public treasure, which he brought with him for the payment
of the troops; and from the moment that he was conscious of his
own guilt, he could no longer refuse to attest the innocence and
merit of the count. The charge of the Tripolitans was declared to
be false and frivolous; and Palladius himself was sent back from
Treves to Africa, with a special commission to discover and
prosecute the authors of this impious conspiracy against the
representatives of the sovereign. His inquiries were managed with
so much dexterity and success, that he compelled the citizens of
Leptis, who had sustained a recent siege of eight days, to
contradict the truth of their own decrees, and to censure the
behavior of their own deputies. A bloody sentence was pronounced,
without hesitation, by the rash and headstrong cruelty of
Valentinian. The president of Tripoli, who had presumed to pity
the distress of the province, was publicly executed at Utica;
four distinguished citizens were put to death, as the accomplices
of the imaginary fraud; and the tongues of two others were cut
out, by the express order of the emperor. Romanus, elated by
impunity, and irritated by resistance, was still continued in the
military command; till the Africans were provoked, by his
avarice, to join the rebellious standard of Firmus, the Moor.\textsuperscript{121}

\pagenote[120]{Ammianus frequently mentions their concilium
annuum, legitimum, \&c. Leptis and Sabrata are long since ruined;
but the city of Oea, the native country of Apuleius, still
flourishes under the provincial denomination of \textit{Tripoli}. See
Cellarius (Geograph. Antiqua, tom. ii. part ii. p. 81,)
D’Anville, (Geographie Ancienne, tom. iii. p. 71, 72,) and
Marmol, (Arrique, tom. ii. p. 562.)}

\pagenote[121]{Ammian. xviii. 6. Tillemont (Hist. des Empereurs,
tom. v. p 25, 676) has discussed the chronological difficulties
of the history of Count Romanus.}

His father Nabal was one of the richest and most powerful of the
Moorish princes, who acknowledged the supremacy of Rome. But as
he left, either by his wives or concubines, a very numerous
posterity, the wealthy inheritance was eagerly disputed; and
Zamma, one of his sons, was slain in a domestic quarrel by his
brother Firmus. The implacable zeal, with which Romanus
prosecuted the legal revenge of this murder, could be ascribed
only to a motive of avarice, or personal hatred; but, on this
occasion, his claims were just; his influence was weighty; and
Firmus clearly understood, that he must either present his neck
to the executioner, or appeal from the sentence of the Imperial
consistory, to his sword, and to the people.\textsuperscript{122} He was received
as the deliverer of his country; and, as soon as it appeared that
Romanus was formidable only to a submissive province, the tyrant
of Africa became the object of universal contempt. The ruin of
Cæsarea, which was plundered and burnt by the licentious
Barbarians, convinced the refractory cities of the danger of
resistance; the power of Firmus was established, at least in the
provinces of Mauritania and Numidia; and it seemed to be his only
doubt whether he should assume the diadem of a Moorish king, or
the purple of a Roman emperor. But the imprudent and unhappy
Africans soon discovered, that, in this rash insurrection, they
had not sufficiently consulted their own strength, or the
abilities of their leader. Before he could procure any certain
intelligence, that the emperor of the West had fixed the choice
of a general, or that a fleet of transports was collected at the
mouth of the Rhone, he was suddenly informed that the great
Theodosius, with a small band of veterans, had landed near
Igilgilis, or Gigeri, on the African coast; and the timid usurper
sunk under the ascendant of virtue and military genius. Though
Firmus possessed arms and treasures, his despair of victory
immediately reduced him to the use of those arts, which, in the
same country, and in a similar situation, had formerly been
practised by the crafty Jugurtha. He attempted to deceive, by an
apparent submission, the vigilance of the Roman general; to
seduce the fidelity of his troops; and to protract the duration
of the war, by successively engaging the independent tribes of
Africa to espouse his quarrel, or to protect his flight.
Theodosius imitated the example, and obtained the success, of his
predecessor Metellus. When Firmus, in the character of a
suppliant, accused his own rashness, and humbly solicited the
clemency of the emperor, the lieutenant of Valentinian received
and dismissed him with a friendly embrace: but he diligently
required the useful and substantial pledges of a sincere
repentance; nor could he be persuaded, by the assurances of
peace, to suspend, for an instant, the operations of an active
war. A dark conspiracy was detected by the penetration of
Theodosius; and he satisfied, without much reluctance, the public
indignation, which he had secretly excited. Several of the guilty
accomplices of Firmus were abandoned, according to ancient
custom, to the tumult of a military execution; many more, by the
amputation of both their hands, continued to exhibit an
instructive spectacle of horror; the hatred of the rebels was
accompanied with fear; and the fear of the Roman soldiers was
mingled with respectful admiration. Amidst the boundless plains
of Getulia, and the innumerable valleys of Mount Atlas, it was
impossible to prevent the escape of Firmus; and if the usurper
could have tired the patience of his antagonist, he would have
secured his person in the depth of some remote solitude, and
expected the hopes of a future revolution. He was subdued by the
perseverance of Theodosius; who had formed an inflexible
determination, that the war should end only by the death of the
tyrant; and that every nation of Africa, which presumed to
support his cause, should be involved in his ruin. At the head of
a small body of troops, which seldom exceeded three thousand five
hundred men, the Roman general advanced, with a steady prudence,
devoid of rashness or of fear, into the heart of a country, where
he was sometimes attacked by armies of twenty thousand Moors. The
boldness of his charge dismayed the irregular Barbarians; they
were disconcerted by his seasonable and orderly retreats; they
were continually baffled by the unknown resources of the military
art; and they felt and confessed the just superiority which was
assumed by the leader of a civilized nation. When Theodosius
entered the extensive dominions of Igmazen, king of the
Isaflenses, the haughty savage required, in words of defiance,
his name, and the object of his expedition. “I am,” replied the
stern and disdainful count, “I am the general of Valentinian, the
lord of the world; who has sent me hither to pursue and punish a
desperate robber. Deliver him instantly into my hands; and be
assured, that if thou dost not obey the commands of my invincible
sovereign, thou, and the people over whom thou reignest, shall be
utterly extirpated.”\textsuperscript{12211} As soon as Igmazen was satisfied, that
his enemy had strength and resolution to execute the fatal
menace, he consented to purchase a necessary peace by the
sacrifice of a guilty fugitive. The guards that were placed to
secure the person of Firmus deprived him of the hopes of escape;
and the Moorish tyrant, after wine had extinguished the sense of
danger, disappointed the insulting triumph of the Romans, by
strangling himself in the night. His dead body, the only present
which Igmazen could offer to the conqueror, was carelessly thrown
upon a camel; and Theodosius, leading back his victorious troops
to Sitifi, was saluted by the warmest acclamations of joy and
loyalty.\textsuperscript{123}

\pagenote[122]{The Chronology of Ammianus is loose and obscure;
and Orosius (i. vii. c. 33, p. 551, edit. Havercamp) seems to
place the revolt of Firmus after the deaths of Valentinian and
Valens. Tillemont (Hist. des. Emp. tom. v. p. 691) endeavors to
pick his way. The patient and sure-foot mule of the Alps may be
trusted in the most slippery paths.}

\pagenote[12211]{The war was longer protracted than this sentence
would lead us to suppose: it was not till defeated more than once
that Igmazen yielded Amm. xxix. 5.—M}

\pagenote[123]{Ammian xxix. 5. The text of this long chapter
(fifteen quarto pages) is broken and corrupted; and the narrative
is perplexed by the want of chronological and geographical
landmarks.}

Africa had been lost by the vices of Romanus; it was restored by
the virtues of Theodosius; and our curiosity may be usefully
directed to the inquiry of the respective treatment which the two
generals received from the Imperial court. The authority of Count
Romanus had been suspended by the master-general of the cavalry;
and he was committed to safe and honorable custody till the end
of the war. His crimes were proved by the most authentic
evidence; and the public expected, with some impatience, the
decree of severe justice. But the partial and powerful favor of
Mellobaudes encouraged him to challenge his legal judges, to
obtain repeated delays for the purpose of procuring a crowd of
friendly witnesses, and, finally, to cover his guilty conduct, by
the additional guilt of fraud and forgery. About the same time,
the restorer of Britain and Africa, on a vague suspicion that his
name and services were superior to the rank of a subject, was
ignominiously beheaded at Carthage. Valentinian no longer
reigned; and the death of Theodosius, as well as the impunity of
Romanus, may justly be imputed to the arts of the ministers, who
abused the confidence, and deceived the inexperienced youth, of
his sons.\textsuperscript{124}

\pagenote[124]{Ammian xxviii. 4. Orosius, l. vii. c. 33, p. 551,
552. Jerom. in Chron. p. 187.}

If the geographical accuracy of Ammianus had been fortunately
bestowed on the British exploits of Theodosius, we should have
traced, with eager curiosity, the distinct and domestic footsteps
of his march. But the tedious enumeration of the unknown and
uninteresting tribes of Africa may be reduced to the general
remark, that they were all of the swarthy race of the Moors; that
they inhabited the back settlements of the Mauritanian and
Numidian province, the country, as they have since been termed by
the Arabs, of dates and of locusts;\textsuperscript{125} and that, as the Roman
power declined in Africa, the boundary of civilized manners and
cultivated land was insensibly contracted. Beyond the utmost
limits of the Moors, the vast and inhospitable desert of the
South extends above a thousand miles to the banks of the Niger.
The ancients, who had a very faint and imperfect knowledge of the
great peninsula of Africa, were sometimes tempted to believe,
that the torrid zone must ever remain destitute of inhabitants;\textsuperscript{126}
and they sometimes amused their fancy by filling the vacant
space with headless men, or rather monsters;\textsuperscript{127} with horned and
cloven-footed satyrs;\textsuperscript{128} with fabulous centaurs;\textsuperscript{129} and with
human pygmies, who waged a bold and doubtful warfare against the
cranes.\textsuperscript{130} Carthage would have trembled at the strange
intelligence that the countries on either side of the equator
were filled with innumerable nations, who differed only in their
color from the ordinary appearance of the human species: and the
subjects of the Roman empire might have anxiously expected, that
the swarms of Barbarians, which issued from the North, would soon
be encountered from the South by new swarms of Barbarians,
equally fierce and equally formidable. These gloomy terrors would
indeed have been dispelled by a more intimate acquaintance with
the character of their African enemies. The inaction of the
negroes does not seem to be the effect either of their virtue or
of their pusillanimity. They indulge, like the rest of mankind,
their passions and appetites; and the adjacent tribes are engaged
in frequent acts of hostility.\textsuperscript{131} But their rude ignorance has
never invented any effectual weapons of defence, or of
destruction; they appear incapable of forming any extensive plans
of government, or conquest; and the obvious inferiority of their
mental faculties has been discovered and abused by the nations of
the temperate zone. Sixty thousand blacks are annually embarked
from the coast of Guinea, never to return to their native
country; but they are embarked in chains;\textsuperscript{132} and this constant
emigration, which, in the space of two centuries, might have
furnished armies to overrun the globe, accuses the guilt of
Europe, and the weakness of Africa.

\pagenote[125]{Leo Africanus (in the Viaggi di Ramusio, tom. i.
fol. 78-83) has traced a curious picture of the people and the
country; which are more minutely described in the Afrique de
Marmol, tom. iii. p. 1-54.}

\pagenote[126]{This uninhabitable zone was gradually reduced by
the improvements of ancient geography, from forty-five to
twenty-four, or even sixteen degrees of latitude. See a learned
and judicious note of Dr. Robertson, Hist. of America, vol. i. p.
426.}

\pagenote[127]{Intra, si credere libet, vix jam homines et magis
semiferi... Blemmyes, Satyri, \&c. Pomponius Mela, i. 4, p. 26,
edit. Voss. in 8vo. Pliny \textit{philosophically} explains (vi. 35) the
irregularities of nature, which he had \textit{credulously} admitted,
(v. 8.)}

\pagenote[128]{If the satyr was the Orang-outang, the great human
ape, (Buffon, Hist. Nat. tom. xiv. p. 43, \&c.,) one of that
species might actually be shown alive at Alexandria, in the reign
of Constantine. Yet some difficulty will still remain about the
conversation which St. Anthony held with one of these pious
savages, in the desert of Thebais. (Jerom. in Vit. Paul. Eremit.
tom. i. p. 238.)}

\pagenote[129]{St. Anthony likewise met one of \textit{these} monsters;
whose existence was seriously asserted by the emperor Claudius.
The public laughed; but his præfect of Egypt had the address to
send an artful preparation, the embalmed corpse of a
\textit{Hippocentaur}, which was preserved almost a century afterwards
in the Imperial palace. See Pliny, (Hist. Natur. vii. 3,) and the
judicious observations of Freret. (Mémoires de l’Acad. tom. vii.
p. 321, \&c.)}

\pagenote[130]{The fable of the pygmies is as old as Homer,
(Iliad. iii. 6) The pygmies of India and Æthiopia were
(trispithami) twenty-seven inches high. Every spring their
cavalry (mounted on rams and goats) marched, in battle array, to
destroy the cranes’ eggs, aliter (says Pliny) futuris gregibus
non resisti. Their houses were built of mud, feathers, and
egg-shells. See Pliny, (vi. 35, vii. 2,) and Strabo, (l. ii. p.
121.)}

\pagenote[131]{The third and fourth volumes of the valuable
Histoire des Voyages describe the present state of the Negroes.
The nations of the sea-coast have been polished by European
commerce; and those of the inland country have been improved by
Moorish colonies. * Note: The martial tribes in chain armor,
discovered by Denham, are Mahometan; the great question of the
inferiority of the African tribes in their mental faculties will
probably be experimentally resolved before the close of the
century; but the Slave Trade still continues, and will, it is to
be feared, till the spirit of gain is subdued by the spirit of
Christian humanity.—M.}

\pagenote[132]{Histoire Philosophique et Politique, \&c., tom. iv.
p. 192.}

\section{Part \thesection.}

IV. The ignominious treaty, which saved the army of Jovian, had
been faithfully executed on the side of the Romans; and as they
had solemnly renounced the sovereignty and alliance of Armenia
and Iberia, those tributary kingdoms were exposed, without
protection, to the arms of the Persian monarch.\textsuperscript{133} Sapor entered
the Armenian territories at the head of a formidable host of
cuirassiers, of archers, and of mercenary foot; but it was the
invariable practice of Sapor to mix war and negotiation, and to
consider falsehood and perjury as the most powerful instruments
of regal policy. He affected to praise the prudent and moderate
conduct of the king of Armenia; and the unsuspicious Tiranus was
persuaded, by the repeated assurances of insidious friendship, to
deliver his person into the hands of a faithless and cruel enemy.
In the midst of a splendid entertainment, he was bound in chains
of silver, as an honor due to the blood of the Arsacides; and,
after a short confinement in the Tower of Oblivion at Ecbatana,
he was released from the miseries of life, either by his own
dagger, or by that of an assassin.\textsuperscript{13311} The kingdom of Armenia
was reduced to the state of a Persian province; the
administration was shared between a distinguished satrap and a
favorite eunuch; and Sapor marched, without delay, to subdue the
martial spirit of the Iberians. Sauromaces, who reigned in that
country by the permission of the emperors, was expelled by a
superior force; and, as an insult on the majesty of Rome, the
king of kings placed a diadem on the head of his abject vassal
Aspacuras. The city of Artogerassa\textsuperscript{134} was the only place of
Armenia\textsuperscript{13411} which presumed to resist the efforts of his arms.
The treasure deposited in that strong fortress tempted the
avarice of Sapor; but the danger of Olympias, the wife or widow
of the Armenian king, excited the public compassion, and animated
the desperate valor of her subjects and soldiers.\textsuperscript{13412} The
Persians were surprised and repulsed under the walls of
Artogerassa, by a bold and well-concerted sally of the besieged.
But the forces of Sapor were continually renewed and increased;
the hopeless courage of the garrison was exhausted; the strength
of the walls yielded to the assault; and the proud conqueror,
after wasting the rebellious city with fire and sword, led away
captive an unfortunate queen; who, in a more auspicious hour, had
been the destined bride of the son of Constantine.\textsuperscript{135} Yet if
Sapor already triumphed in the easy conquest of two dependent
kingdoms, he soon felt, that a country is unsubdued as long as
the minds of the people are actuated by a hostile and
contumacious spirit. The satraps, whom he was obliged to trust,
embraced the first opportunity of regaining the affection of
their countrymen, and of signalizing their immortal hatred to the
Persian name. Since the conversion of the Armenians and Iberians,
these nations considered the Christians as the favorites, and the
Magians as the adversaries, of the Supreme Being: the influence
of the clergy, over a superstitious people was uniformly exerted
in the cause of Rome; and as long as the successors of
Constantine disputed with those of Artaxerxes the sovereignty of
the intermediate provinces, the religious connection always threw
a decisive advantage into the scale of the empire. A numerous and
active party acknowledged Para, the son of Tiranus, as the lawful
sovereign of Armenia, and his title to the throne was deeply
rooted in the hereditary succession of five hundred years. By the
unanimous consent of the Iberians, the country was equally
divided between the rival princes; and Aspacuras, who owed his
diadem to the choice of Sapor, was obliged to declare, that his
regard for his children, who were detained as hostages by the
tyrant, was the only consideration which prevented him from
openly renouncing the alliance of Persia. The emperor Valens, who
respected the obligations of the treaty, and who was apprehensive
of involving the East in a dangerous war, ventured, with slow and
cautious measures, to support the Roman party in the kingdoms of
Iberia and Armenia.\textsuperscript{13511} Twelve legions established the
authority of Sauromaces on the banks of the Cyrus. The Euphrates
was protected by the valor of Arintheus. A powerful army, under
the command of Count Trajan, and of Vadomair, king of the
Alemanni, fixed their camp on the confines of Armenia. But they
were strictly enjoined not to commit the first hostilities, which
might be understood as a breach of the treaty: and such was the
implicit obedience of the Roman general, that they retreated,
with exemplary patience, under a shower of Persian arrows till
they had clearly acquired a just title to an honorable and
legitimate victory. Yet these appearances of war insensibly
subsided in a vain and tedious negotiation. The contending
parties supported their claims by mutual reproaches of perfidy
and ambition; and it should seem, that the original treaty was
expressed in very obscure terms, since they were reduced to the
necessity of making their inconclusive appeal to the partial
testimony of the generals of the two nations, who had assisted at
the negotiations.\textsuperscript{136} The invasion of the Goths and Huns which
soon afterwards shook the foundations of the Roman empire,
exposed the provinces of Asia to the arms of Sapor. But the
declining age, and perhaps the infirmities, of the monarch
suggested new maxims of tranquillity and moderation. His death,
which happened in the full maturity of a reign of seventy years,
changed in a moment the court and councils of Persia; and their
attention was most probably engaged by domestic troubles, and the
distant efforts of a Carmanian war.\textsuperscript{137} The remembrance of
ancient injuries was lost in the enjoyment of peace. The kingdoms
of Armenia and Iberia were permitted, by the mutual,though tacit
consent of both empires, to resume their doubtful neutrality. In
the first years of the reign of Theodosius, a Persian embassy
arrived at Constantinople, to excuse the unjustifiable measures
of the former reign; and to offer, as the tribute of friendship,
or even of respect, a splendid present of gems, of silk, and of
Indian elephants.\textsuperscript{138}

\pagenote[133]{The evidence of Ammianus is original and decisive,
(xxvii. 12.) Moses of Chorene, (l. iii. c. 17, p. 249, and c. 34,
p. 269,) and Procopius, (de Bell. Persico, l. i. c. 5, p. 17,
edit. Louvre,) have been consulted: but those historians who
confound distinct facts, repeat the same events, and introduce
strange stories, must be used with diffidence and caution. Note:
The statement of Ammianus is more brief and succinct, but
harmonizes with the more complicated history developed by M. St.
Martin from the Armenian writers, and from Procopius, who wrote,
as he states from Armenian authorities.—M.}

\pagenote[13311]{According to M. St. Martin, Sapor, though
supported by the two apostate Armenian princes, Meroujan the
Ardzronnian and Vahan the Mamigonian, was gallantly resisted by
Arsaces, and his brave though impious wife Pharandsem. His troops
were defeated by Vasag, the high constable of the kingdom. (See
M. St. Martin.) But after four years’ courageous defence of his
kingdom, Arsaces was abandoned by his nobles, and obliged to
accept the perfidious hospitality of Sapor. He was blinded and
imprisoned in the “Castle of Oblivion;” his brave general Vasag
was flayed alive; his skin stuffed and placed near the king in
his lonely prison. It was not till many years after (A.D. 371)
that he stabbed himself, according to the romantic story, (St. M.
iii. 387, 389,) in a paroxysm of excitement at his restoration to
royal honors. St. Martin, Additions to Le Beau, iii. 283,
296.—M.}

\pagenote[134]{Perhaps Artagera, or Ardis; under whose walls
Caius, the grandson of Augustus, was wounded. This fortress was
situate above Amida, near one of the sources of the Tigris. See
D’Anville, Geographie Ancienue, tom. ii. p. 106. * Note: St.
Martin agrees with Gibbon, that it was the same fortress with
Ardis Note, p. 373.—M.}

\pagenote[13411]{Artaxata, Vagharschabad, or Edchmiadzin,
Erovantaschad, and many other cities, in all of which there was a
considerable Jewish population were taken and destroyed.—M.}

\pagenote[13412]{Pharandsem, not Olympias, refusing the orders of
her captive husband to surrender herself to Sapor, threw herself
into Artogerassa St. Martin, iii. 293, 302. She defended herself
for fourteen months, till famine and disease had left few
survivors out of 11,000 soldiers and 6000 women who had taken
refuge in the fortress. She then threw open the gates with her
own hand. M. St. Martin adds, what even the horrors of Oriental
warfare will scarcely permit us to credit, that she was exposed
by Sapor on a public scaffold to the brutal lusts of his
soldiery, and afterwards empaled, iii. 373, \&c.—M.}

\pagenote[135]{Tillemont (Hist. des Empereurs, tom. v. p. 701)
proves, from chronology, that Olympias must have been the mother
of Para. Note *: An error according to St. M. 273.—M.}

\pagenote[13511]{According to Themistius, quoted by St. Martin,
he once advanced to the Tigris, iii. 436.—M.}

\pagenote[136]{Ammianus (xxvii. 12, xix. 1. xxx. 1, 2) has
described the events, without the dates, of the Persian war.
Moses of Chorene (Hist. Armen. l. iii. c. 28, p. 261, c. 31, p.
266, c. 35, p. 271) affords some additional facts; but it is
extremely difficult to separate truth from fable.}

\pagenote[137]{Artaxerxes was the successor and brother (\textit{the
cousin-german}) of the great Sapor; and the guardian of his son,
Sapor III. (Agathias, l. iv. p. 136, edit. Louvre.) See the
Universal History, vol. xi. p. 86, 161. The authors of that
unequal work have compiled the Sassanian dynasty with erudition
and diligence; but it is a preposterous arrangement to divide the
Roman and Oriental accounts into two distinct histories. * Note:
On the war of Sapor with the Bactrians, which diverted from
Armenia, see St. M. iii. 387.—M.}

\pagenote[138]{Pacatus in Panegyr. Vet. xii. 22, and Orosius, l.
vii. c. 34. Ictumque tum fœdus est, quo universus Oriens usque ad
num (A. D. 416) tranquillissime fruitur.}

In the general picture of the affairs of the East under the reign
of Valens, the adventures of Para form one of the most striking
and singular objects. The noble youth, by the persuasion of his
mother Olympias, had escaped through the Persian host that
besieged Artogerassa, and implored the protection of the emperor
of the East. By his timid councils, Para was alternately
supported, and recalled, and restored, and betrayed. The hopes of
the Armenians were sometimes raised by the presence of their
natural sovereign,\textsuperscript{13811} and the ministers of Valens were
satisfied, that they preserved the integrity of the public faith,
if their vassal was not suffered to assume the diadem and title
of King. But they soon repented of their own rashness. They were
confounded by the reproaches and threats of the Persian monarch.
They found reason to distrust the cruel and inconstant temper of
Para himself; who sacrificed, to the slightest suspicions, the
lives of his most faithful servants, and held a secret and
disgraceful correspondence with the assassin of his father and
the enemy of his country. Under the specious pretence of
consulting with the emperor on the subject of their common
interest, Para was persuaded to descend from the mountains of
Armenia, where his party was in arms, and to trust his
independence and safety to the discretion of a perfidious court.
The king of Armenia, for such he appeared in his own eyes and in
those of his nation, was received with due honors by the
governors of the provinces through which he passed; but when he
arrived at Tarsus in Cilicia, his progress was stopped under
various pretences; his motions were watched with respectful
vigilance, and he gradually discovered, that he was a prisoner in
the hands of the Romans. Para suppressed his indignation,
dissembled his fears, and after secretly preparing his escape,
mounted on horseback with three hundred of his faithful
followers. The officer stationed at the door of his apartment
immediately communicated his flight to the consular of Cilicia,
who overtook him in the suburbs, and endeavored without success,
to dissuade him from prosecuting his rash and dangerous design. A
legion was ordered to pursue the royal fugitive; but the pursuit
of infantry could not be very alarming to a body of light
cavalry; and upon the first cloud of arrows that was discharged
into the air, they retreated with precipitation to the gates of
Tarsus. After an incessant march of two days and two nights, Para
and his Armenians reached the banks of the Euphrates; but the
passage of the river which they were obliged to swim,\textsuperscript{13812} was
attended with some delay and some loss. The country was alarmed;
and the two roads, which were only separated by an interval of
three miles had been occupied by a thousand archers on horseback,
under the command of a count and a tribune. Para must have
yielded to superior force, if the accidental arrival of a
friendly traveller had not revealed the danger and the means of
escape. A dark and almost impervious path securely conveyed the
Armenian troop through the thicket; and Para had left behind him
the count and the tribune, while they patiently expected his
approach along the public highways. They returned to the Imperial
court to excuse their want of diligence or success; and seriously
alleged, that the king of Armenia, who was a skilful magician,
had transformed himself and his followers, and passed before
their eyes under a borrowed shape.\textsuperscript{13813} After his return to his
native kingdom, Para still continued to profess himself the
friend and ally of the Romans: but the Romans had injured him too
deeply ever to forgive, and the secret sentence of his death was
signed in the council of Valens. The execution of the bloody deed
was committed to the subtle prudence of Count Trajan; and he had
the merit of insinuating himself into the confidence of the
credulous prince, that he might find an opportunity of stabbing
him to the heart Para was invited to a Roman banquet, which had
been prepared with all the pomp and sensuality of the East; the
hall resounded with cheerful music, and the company was already
heated with wine; when the count retired for an instant, drew his
sword, and gave the signal of the murder. A robust and desperate
Barbarian instantly rushed on the king of Armenia; and though he
bravely defended his life with the first weapon that chance
offered to his hand, the table of the Imperial general was
stained with the royal blood of a guest, and an ally. Such were
the weak and wicked maxims of the Roman administration, that, to
attain a doubtful object of political interest the laws of
nations, and the sacred rights of hospitality were inhumanly
violated in the face of the world.\textsuperscript{139}

\pagenote[13811]{On the reconquest of Armenia by Para, or rather
by Mouschegh, the Mamigonian see St. M. iii. 375, 383.—M.}

\pagenote[13812]{On planks floated by bladders.—M.}

\pagenote[13813]{It is curious enough that the Armenian
historian, Faustus of Byzandum, represents Para as a magician.
His impious mother Pharandac had devoted him to the demons on his
birth. St. M. iv. 23.—M.}

\pagenote[139]{See in Ammianus (xxx. 1) the adventures of Para.
Moses of Chorene calls him Tiridates; and tells a long, and not
improbable story of his son Gnelus, who afterwards made himself
popular in Armenia, and provoked the jealousy of the reigning
king, (l. iii. c 21, \&c., p. 253, \&c.) * Note: This note is a
tissue of mistakes. Tiridates and Para are two totally different
persons. Tiridates was the father of Gnel first husband of
Pharandsem, the mother of Para. St. Martin, iv. 27—M.}

V. During a peaceful interval of thirty years, the Romans secured
their frontiers, and the Goths extended their dominions. The
victories of the great Hermanric,\textsuperscript{140} king of the Ostrogoths, and
the most noble of the race of the Amali, have been compared, by
the enthusiasm of his countrymen, to the exploits of Alexander;
with this singular, and almost incredible, difference, that the
martial spirit of the Gothic hero, instead of being supported by
the vigor of youth, was displayed with glory and success in the
extreme period of human life, between the age of fourscore and
one hundred and ten years. The independent tribes were persuaded,
or compelled, to acknowledge the king of the Ostrogoths as the
sovereign of the Gothic nation: the chiefs of the Visigoths, or
Thervingi, renounced the royal title, and assumed the more humble
appellation of \textit{Judges;} and, among those judges, Athanaric,
Fritigern, and Alavivus, were the most illustrious, by their
personal merit, as well as by their vicinity to the Roman
provinces. These domestic conquests, which increased the military
power of Hermanric, enlarged his ambitious designs. He invaded
the adjacent countries of the North; and twelve considerable
nations, whose names and limits cannot be accurately defined,
successively yielded to the superiority of the Gothic arms.\textsuperscript{141}
The Heruli, who inhabited the marshy lands near the lake Mæotis,
were renowned for their strength and agility; and the assistance
of their light infantry was eagerly solicited, and highly
esteemed, in all the wars of the Barbarians. But the active
spirit of the Heruli was subdued by the slow and steady
perseverance of the Goths; and, after a bloody action, in which
the king was slain, the remains of that warlike tribe became a
useful accession to the camp of Hermanric.

He then marched against the Venedi; unskilled in the use of arms,
and formidable only by their numbers, which filled the wide
extent of the plains of modern Poland. The victorious Goths, who
were not inferior in numbers, prevailed in the contest, by the
decisive advantages of exercise and discipline. After the
submission of the Venedi, the conqueror advanced, without
resistance, as far as the confines of the Æstii;\textsuperscript{142} an ancient
people, whose name is still preserved in the province of
Esthonia. Those distant inhabitants of the Baltic coast were
supported by the labors of agriculture, enriched by the trade of
amber, and consecrated by the peculiar worship of the Mother of
the Gods. But the scarcity of iron obliged the Æstian warriors to
content themselves with wooden clubs; and the reduction of that
wealthy country is ascribed to the prudence, rather than to the
arms, of Hermanric. His dominions, which extended from the Danube
to the Baltic, included the native seats, and the recent
acquisitions, of the Goths; and he reigned over the greatest part
of Germany and Scythia with the authority of a conqueror, and
sometimes with the cruelty of a tyrant. But he reigned over a
part of the globe incapable of perpetuating and adorning the
glory of its heroes. The name of Hermanric is almost buried in
oblivion; his exploits are imperfectly known; and the Romans
themselves appeared unconscious of the progress of an aspiring
power which threatened the liberty of the North, and the peace of
the empire.\textsuperscript{143}

\pagenote[140]{The concise account of the reign and conquests of
Hermanric seems to be one of the valuable fragments which
Jornandes (c 28) borrowed from the Gothic histories of Ablavius,
or Cassiodorus.}

\pagenote[141]{M. d. Buat. (Hist. des Peuples de l’Europe, tom.
vi. p. 311-329) investigates, with more industry than success,
the nations subdued by the arms of Hermanric. He denies the
existence of the \textit{Vasinobroncæ}, on account of the immoderate
length of their name. Yet the French envoy to Ratisbon, or
Dresden, must have traversed the country of the \textit{Mediomatrici}.}

\pagenote[142]{The edition of Grotius (Jornandes, p. 642)
exhibits the name of \textit{Æstri}. But reason and the Ambrosian MS.
have restored the \textit{Æstii}, whose manners and situation are
expressed by the pencil of Tacitus, (Germania, c. 45.)}

\pagenote[143]{Ammianus (xxxi. 3) observes, in general terms,
Ermenrichi.... nobilissimi Regis, et per multa variaque fortiter
facta, vicinigentibus formidati, \&c.}

The Goths had contracted an hereditary attachment for the
Imperial house of Constantine, of whose power and liberality they
had received so many signal proofs. They respected the public
peace; and if a hostile band sometimes presumed to pass the Roman
limit, their irregular conduct was candidly ascribed to the
ungovernable spirit of the Barbarian youth. Their contempt for
two new and obscure princes, who had been raised to the throne by
a popular election, inspired the Goths with bolder hopes; and,
while they agitated some design of marching their confederate
force under the national standard,\textsuperscript{144} they were easily tempted
to embrace the party of Procopius; and to foment, by their
dangerous aid, the civil discord of the Romans. The public treaty
might stipulate no more than ten thousand auxiliaries; but the
design was so zealously adopted by the chiefs of the Visigoths,
that the army which passed the Danube amounted to the number of
thirty thousand men.\textsuperscript{145} They marched with the proud confidence,
that their invincible valor would decide the fate of the Roman
empire; and the provinces of Thrace groaned under the weight of
the Barbarians, who displayed the insolence of masters and the
licentiousness of enemies. But the intemperance which gratified
their appetites, retarded their progress; and before the Goths
could receive any certain intelligence of the defeat and death of
Procopius, they perceived, by the hostile state of the country,
that the civil and military powers were resumed by his successful
rival. A chain of posts and fortifications, skilfully disposed by
Valens, or the generals of Valens, resisted their march,
prevented their retreat, and intercepted their subsistence. The
fierceness of the Barbarians was tamed and suspended by hunger;
they indignantly threw down their arms at the feet of the
conqueror, who offered them food and chains: the numerous
captives were distributed in all the cities of the East; and the
provincials, who were soon familiarized with their savage
appearance, ventured, by degrees, to measure their own strength
with these formidable adversaries, whose name had so long been
the object of their terror. The king of Scythia (and Hermanric
alone could deserve so lofty a title) was grieved and exasperated
by this national calamity. His ambassadors loudly complained, at
the court of Valens, of the infraction of the ancient and solemn
alliance, which had so long subsisted between the Romans and the
Goths. They alleged, that they had fulfilled the duty of allies,
by assisting the kinsman and successor of the emperor Julian;
they required the immediate restitution of the noble captives;
and they urged a very singular claim, that the Gothic generals
marching in arms, and in hostile array, were entitled to the
sacred character and privileges of ambassadors. The decent, but
peremptory, refusal of these extravagant demands, was signified
to the Barbarians by Victor, master-general of the cavalry; who
expressed, with force and dignity, the just complaints of the
emperor of the East.\textsuperscript{146} The negotiation was interrupted; and the
manly exhortations of Valentinian encouraged his timid brother to
vindicate the insulted majesty of the empire.\textsuperscript{147}

\pagenote[144]{Valens. ... docetur relationibus Ducum, gentem
Gothorum, ea tempestate intactam ideoque sævissimam, conspirantem
in unum, ad pervadenda parari collimitia Thraciarum. Ammian. xxi.
6.}

\pagenote[145]{M. de Buat (Hist. des Peuples de l’Europe, tom.
vi. p. 332) has curiously ascertained the real number of these
auxiliaries. The 3000 of Ammianus, and the 10,000 of Zosimus,
were only the first divisions of the Gothic army. * Note: M. St.
Martin (iii. 246) denies that there is any authority for these
numbers.—M.}

\pagenote[146]{The march, and subsequent negotiation, are
described in the Fragments of Eunapius, (Excerpt. Legat. p. 18,
edit. Louvre.) The provincials who afterwards became familiar
with the Barbarians, found that their strength was more apparent
than real. They were tall of stature; but their legs were clumsy,
and their shoulders were narrow.}

\pagenote[147]{Valens enim, ut consulto placuerat fratri, cujus
regebatur arbitrio, arma concussit in Gothos ratione justâ
permotus. Ammianus (xxvii. 4) then proceeds to describe, not the
country of the Goths, but the peaceful and obedient province of
Thrace, which was not affected by the war.}

The splendor and magnitude of this Gothic war are celebrated by a
contemporary historian:\textsuperscript{148} but the events scarcely deserve the
attention of posterity, except as the preliminary steps of the
approaching decline and fall of the empire. Instead of leading
the nations of Germany and Scythia to the banks of the Danube, or
even to the gates of Constantinople, the aged monarch of the
Goths resigned to the brave Athanaric the danger and glory of a
defensive war, against an enemy, who wielded with a feeble hand
the powers of a mighty state. A bridge of boats was established
upon the Danube; the presence of Valens animated his troops; and
his ignorance of the art of war was compensated by personal
bravery, and a wise deference to the advice of Victor and
Arintheus, his masters-general of the cavalry and infantry. The
operations of the campaign were conducted by their skill and
experience; but they found it impossible to drive the Visigoths
from their strong posts in the mountains; and the devastation of
the plains obliged the Romans themselves to repass the Danube on
the approach of winter. The incessant rains, which swelled the
waters of the river, produced a tacit suspension of arms, and
confined the emperor Valens, during the whole course of the
ensuing summer, to his camp of Marcianopolis. The third year of
the war was more favorable to the Romans, and more pernicious to
the Goths. The interruption of trade deprived the Barbarians of
the objects of luxury, which they already confounded with the
necessaries of life; and the desolation of a very extensive tract
of country threatened them with the horrors of famine. Athanaric
was provoked, or compelled, to risk a battle, which he lost, in
the plains; and the pursuit was rendered more bloody by the cruel
precaution of the victorious generals, who had promised a large
reward for the head of every Goth that was brought into the
Imperial camp. The submission of the Barbarians appeased the
resentment of Valens and his council: the emperor listened with
satisfaction to the flattering and eloquent remonstrance of the
senate of Constantinople, which assumed, for the first time, a
share in the public deliberations; and the same generals, Victor
and Arintheus, who had successfully directed the conduct of the
war, were empowered to regulate the conditions of peace. The
freedom of trade, which the Goths had hitherto enjoyed, was
restricted to two cities on the Danube; the rashness of their
leaders was severely punished by the suppression of their
pensions and subsidies; and the exception, which was stipulated
in favor of Athanaric alone, was more advantageous than honorable
to the Judge of the Visigoths. Athanaric, who, on this occasion,
appears to have consulted his private interest, without expecting
the orders of his sovereign, supported his own dignity, and that
of his tribe, in the personal interview which was proposed by the
ministers of Valens. He persisted in his declaration, that it was
impossible for him, without incurring the guilt of perjury, ever
to set his foot on the territory of the empire; and it is more
than probable, that his regard for the sanctity of an oath was
confirmed by the recent and fatal examples of Roman treachery.
The Danube, which separated the dominions of the two independent
nations, was chosen for the scene of the conference. The emperor
of the East, and the Judge of the Visigoths, accompanied by an
equal number of armed followers, advanced in their respective
barges to the middle of the stream. After the ratification of the
treaty, and the delivery of hostages, Valens returned in triumph
to Constantinople; and the Goths remained in a state of
tranquillity about six years; till they were violently impelled
against the Roman empire by an innumerable host of Scythians, who
appeared to issue from the frozen regions of the North.\textsuperscript{149}

\pagenote[148]{Eunapius, in Excerpt. Legat. p. 18, 19. The Greek
sophist must have considered as \textit{one} and the \textit{same} war, the
whole series of Gothic history till the victories and peace of
Theodosius.}

\pagenote[149]{The Gothic war is described by Ammianus, (xxvii.
6,) Zosimus, (l. iv. p. 211-214,) and Themistius, (Orat. x. p.
129-141.) The orator Themistius was sent from the senate of
Constantinople to congratulate the victorious emperor; and his
servile eloquence compares Valens on the Danube to Achilles in
the Scamander. Jornandes forgets a war peculiar to the
\textit{Visi}-Goths, and inglorious to the Gothic name, (Mascon’s Hist.
of the Germans, vii. 3.)}

The emperor of the West, who had resigned to his brother the
command of the Lower Danube, reserved for his immediate care the
defence of the Rhætian and Illyrian provinces, which spread so
many hundred miles along the greatest of the European rivers. The
active policy of Valentinian was continually employed in adding
new fortifications to the security of the frontier: but the abuse
of this policy provoked the just resentment of the Barbarians.
The Quadi complained, that the ground for an intended fortress
had been marked out on their territories; and their complaints
were urged with so much reason and moderation, that Equitius,
master-general of Illyricum, consented to suspend the prosecution
of the work, till he should be more clearly informed of the will
of his sovereign. This fair occasion of injuring a rival, and of
advancing the fortune of his son, was eagerly embraced by the
inhuman Maximin, the præfect, or rather tyrant, of Gaul. The
passions of Valentinian were impatient of control; and he
credulously listened to the assurances of his favorite, that if
the government of Valeria, and the direction of the work, were
intrusted to the zeal of his son Marcellinus, the emperor should
no longer be importuned with the audacious remonstrances of the
Barbarians. The subjects of Rome, and the natives of Germany,
were insulted by the arrogance of a young and worthless minister,
who considered his rapid elevation as the proof and reward of his
superior merit. He affected, however, to receive the modest
application of Gabinius, king of the Quadi, with some attention
and regard: but this artful civility concealed a dark and bloody
design, and the credulous prince was persuaded to accept the
pressing invitation of Marcellinus. I am at a loss how to vary
the narrative of similar crimes; or how to relate, that, in the
course of the same year, but in remote parts of the empire, the
inhospitable table of two Imperial generals was stained with the
royal blood of two guests and allies, inhumanly murdered by their
order, and in their presence. The fate of Gabinius, and of Para,
was the same: but the cruel death of their sovereign was resented
in a very different manner by the servile temper of the
Armenians, and the free and daring spirit of the Germans. The
Quadi were much declined from that formidable power, which, in
the time of Marcus Antoninus, had spread terror to the gates of
Rome. But they still possessed arms and courage; their courage
was animated by despair, and they obtained the usual
reenforcement of the cavalry of their Sarmatian allies. So
improvident was the assassin Marcellinus, that he chose the
moment when the bravest veterans had been drawn away, to suppress
the revolt of Firmus; and the whole province was exposed, with a
very feeble defence, to the rage of the exasperated Barbarians.
They invaded Pannonia in the season of harvest; unmercifully
destroyed every object of plunder which they could not easily
transport; and either disregarded, or demolished, the empty
fortifications. The princess Constantia, the daughter of the
emperor Constantius, and the granddaughter of the great
Constantine, very narrowly escaped. That royal maid, who had
innocently supported the revolt of Procopius, was now the
destined wife of the heir of the Western empire. She traversed
the peaceful province with a splendid and unarmed train. Her
person was saved from danger, and the republic from disgrace, by
the active zeal of Messala, governor of the provinces. As soon as
he was informed that the village, where she stopped only to dine,
was almost encompassed by the Barbarians, he hastily placed her
in his own chariot, and drove full speed till he reached the
gates of Sirmium, which were at the distance of six-and-twenty
miles. Even Sirmium might not have been secure, if the Quadi and
Sarmatians had diligently advanced during the general
consternation of the magistrates and people. Their delay allowed
Probus, the Prætorian præfect, sufficient time to recover his own
spirits, and to revive the courage of the citizens. He skilfully
directed their strenuous efforts to repair and strengthen the
decayed fortifications; and procured the seasonable and effectual
assistance of a company of archers, to protect the capital of the
Illyrian provinces. Disappointed in their attempts against the
walls of Sirmium, the indignant Barbarians turned their arms
against the master general of the frontier, to whom they unjustly
attributed the murder of their king. Equitius could bring into
the field no more than two legions; but they contained the
veteran strength of the Mæsian and Pannonian bands. The obstinacy
with which they disputed the vain honors of rank and precedency,
was the cause of their destruction; and while they acted with
separate forces and divided councils, they were surprised and
slaughtered by the active vigor of the Sarmatian horse. The
success of this invasion provoked the emulation of the bordering
tribes; and the province of Mæsia would infallibly have been
lost, if young Theodosius, the duke, or military commander, of
the frontier, had not signalized, in the defeat of the public
enemy, an intrepid genius, worthy of his illustrious father, and
of his future greatness.\textsuperscript{150}

\pagenote[150]{Ammianus (xxix. 6) and Zosimus (I. iv. p. 219,
220) carefully mark the origin and progress of the Quadic and
Sarmatian war.}

\section{Part \thesection.}

The mind of Valentinian, who then resided at Treves, was deeply
affected by the calamities of Illyricum; but the lateness of the
season suspended the execution of his designs till the ensuing
spring. He marched in person, with a considerable part of the
forces of Gaul, from the banks of the Moselle: and to the
suppliant ambassadors of the Sarmatians, who met him on the way,
he returned a doubtful answer, that, as soon as he reached the
scene of action, he should examine, and pronounce. When he
arrived at Sirmium, he gave audience to the deputies of the
Illyrian provinces; who loudly congratulated their own felicity
under the auspicious government of Probus, his Prætorian præfect.\textsuperscript{151}
Valentinian, who was flattered by these demonstrations of
their loyalty and gratitude, imprudently asked the deputy of
Epirus, a Cynic philosopher of intrepid sincerity,\textsuperscript{152} whether he
was freely sent by the wishes of the province. “With tears and
groans am I sent,” replied Iphicles, “by a reluctant people.” The
emperor paused: but the impunity of his ministers established the
pernicious maxim, that they might oppress his subjects, without
injuring his service. A strict inquiry into their conduct would
have relieved the public discontent. The severe condemnation of
the murder of Gabinius, was the only measure which could restore
the confidence of the Germans, and vindicate the honor of the
Roman name. But the haughty monarch was incapable of the
magnanimity which dares to acknowledge a fault. He forgot the
provocation, remembered only the injury, and advanced into the
country of the Quadi with an insatiate thirst of blood and
revenge. The extreme devastation, and promiscuous massacre, of a
savage war, were justified, in the eyes of the emperor, and
perhaps in those of the world, by the cruel equity of
retaliation:\textsuperscript{153} and such was the discipline of the Romans, and
the consternation of the enemy, that Valentinian repassed the
Danube without the loss of a single man. As he had resolved to
complete the destruction of the Quadi by a second campaign, he
fixed his winter quarters at Bregetio, on the Danube, near the
Hungarian city of Presburg. While the operations of war were
suspended by the severity of the weather, the Quadi made an
humble attempt to deprecate the wrath of their conqueror; and, at
the earnest persuasion of Equitius, their ambassadors were
introduced into the Imperial council. They approached the throne
with bended bodies and dejected countenances; and without daring
to complain of the murder of their king, they affirmed, with
solemn oaths, that the late invasion was the crime of some
irregular robbers, which the public council of the nation
condemned and abhorred. The answer of the emperor left them but
little to hope from his clemency or compassion. He reviled, in
the most intemperate language, their baseness, their ingratitude,
their insolence. His eyes, his voice, his color, his gestures,
expressed the violence of his ungoverned fury; and while his
whole frame was agitated with convulsive passion, a large blood
vessel suddenly burst in his body; and Valentinian fell
speechless into the arms of his attendants. Their pious care
immediately concealed his situation from the crowd; but, in a few
minutes, the emperor of the West expired in an agony of pain,
retaining his senses till the last; and struggling, without
success, to declare his intentions to the generals and ministers,
who surrounded the royal couch. Valentinian was about fifty-four
years of age; and he wanted only one hundred days to accomplish
the twelve years of his reign.\textsuperscript{154}

\pagenote[151]{Ammianus, (xxx. 5,) who acknowledges the merit,
has censured, with becoming asperity, the oppressive
administration of Petronius Probus. When Jerom translated and
continued the Chronicle of Eusebius, (A. D. 380; see Tillemont,
Mém. Eccles. tom. xii. p. 53, 626,) he expressed the truth, or at
least the public opinion of his country, in the following words:
“Probus P. P. Illyrici inquissimus tributorum exactionibus, ante
provincias quas regebat, quam a Barbaris vastarentur, \textit{erasit}.”
(Chron. edit. Scaliger, p. 187. Animadvers p. 259.) The Saint
afterwards formed an intimate and tender friendship with the
widow of Probus; and the name of Count Equitius with less
propriety, but without much injustice, has been substituted in
the text.}

\pagenote[152]{Julian (Orat. vi. p. 198) represents his friend
Iphicles, as a man of virtue and merit, who had made himself
ridiculous and unhappy by adopting the extravagant dress and
manners of the Cynics.}

\pagenote[153]{Ammian. xxx. v. Jerom, who exaggerates the
misfortune of Valentinian, refuses him even this last consolation
of revenge. Genitali vastato solo et \textit{inultam} patriam
derelinquens, (tom. i. p. 26.)}

\pagenote[154]{See, on the death of Valentinian, Ammianus, (xxx.
6,) Zosimus, (l. iv. p. 221,) Victor, (in Epitom.,) Socrates, (l.
iv. c. 31,) and Jerom, (in Chron. p. 187, and tom. i. p. 26, ad
Heliodor.) There is much variety of circumstances among them; and
Ammianus is so eloquent, that he writes nonsense.}

The polygamy of Valentinian is seriously attested by an
ecclesiastical historian.\textsuperscript{155} “The empress Severa (I relate the
fable) admitted into her familiar society the lovely Justina, the
daughter of an Italian governor: her admiration of those naked
charms, which she had often seen in the bath, was expressed with
such lavish and imprudent praise, that the emperor was tempted to
introduce a second wife into his bed; and his public edict
extended to all the subjects of the empire the same domestic
privilege which he had assumed for himself.” But we may be
assured, from the evidence of reason as well as history, that the
two marriages of Valentinian, with Severa, and with Justina, were
\textit{successively} contracted; and that he used the ancient
permission of divorce, which was still allowed by the laws,
though it was condemned by the church. Severa was the mother of
Gratian, who seemed to unite every claim which could entitle him
to the undoubted succession of the Western empire. He was the
eldest son of a monarch whose glorious reign had confirmed the
free and honorable choice of his fellow-soldiers. Before he had
attained the ninth year of his age, the royal youth received from
the hands of his indulgent father the purple robe and diadem,
with the title of Augustus; the election was solemnly ratified by
the consent and applause of the armies of Gaul;\textsuperscript{156} and the name
of Gratian was added to the names of Valentinian and Valens, in
all the legal transactions of the Roman government. By his
marriage with the granddaughter of Constantine, the son of
Valentinian acquired all the hereditary rights of the Flavian
family; which, in a series of three Imperial generations, were
sanctified by time, religion, and the reverence of the people. At
the death of his father, the royal youth was in the seventeenth
year of his age; and his virtues already justified the favorable
opinion of the army and the people. But Gratian resided, without
apprehension, in the palace of Treves; whilst, at the distance of
many hundred miles, Valentinian suddenly expired in the camp of
Bregetio. The passions, which had been so long suppressed by the
presence of a master, immediately revived in the Imperial
council; and the ambitious design of reigning in the name of an
infant, was artfully executed by Mellobaudes and Equitius, who
commanded the attachment of the Illyrian and Italian bands. They
contrived the most honorable pretences to remove the popular
leaders, and the troops of Gaul, who might have asserted the
claims of the lawful successor; they suggested the necessity of
extinguishing the hopes of foreign and domestic enemies, by a
bold and decisive measure. The empress Justina, who had been left
in a palace about one hundred miles from Bregetio, was
respectively invited to appear in the camp, with the son of the
deceased emperor. On the sixth day after the death of
Valentinian, the infant prince of the same name, who was only
four years old, was shown, in the arms of his mother, to the
legions; and solemnly invested, by military acclamation, with the
titles and ensigns of supreme power. The impending dangers of a
civil war were seasonably prevented by the wise and moderate
conduct of the emperor Gratian. He cheerfully accepted the choice
of the army; declared that he should always consider the son of
Justina as a brother, not as a rival; and advised the empress,
with her son Valentinian to fix their residence at Milan, in the
fair and peaceful province of Italy; while he assumed the more
arduous command of the countries beyond the Alps. Gratian
dissembled his resentment till he could safely punish, or
disgrace, the authors of the conspiracy; and though he uniformly
behaved with tenderness and regard to his infant colleague, he
gradually confounded, in the administration of the Western
empire, the office of a guardian with the authority of a
sovereign. The government of the Roman world was exercised in the
united names of Valens and his two nephews; but the feeble
emperor of the East, who succeeded to the rank of his elder
brother, never obtained any weight or influence in the councils
of the West.\textsuperscript{157}

\pagenote[155]{Socrates (l. iv. c. 31) is the only original
witness of this foolish story, so repugnant to the laws and
manners of the Romans, that it scarcely deserved the formal and
elaborate dissertation of M. Bonamy, (Mém. de l’Académie, tom.
xxx. p. 394-405.) Yet I would preserve the natural circumstance
of the bath; instead of following Zosimus who represents Justina
as an old woman, the widow of Magnentius.}

\pagenote[156]{Ammianus (xxvii. 6) describes the form of this
military election, and \textit{august} investiture. Valentinian does not
appear to have consulted, or even informed, the senate of Rome.}

\pagenote[157]{Ammianus, xxx. 10. Zosimus, l. iv. p. 222, 223.
Tillemont has proved (Hist. des Empereurs, tom. v. p. 707-709)
that Gratian \textit{reigned} in Italy, Africa, and Illyricum. I have
endeavored to express his authority over his brother’s dominions,
as he used it, in an ambiguous style.}

