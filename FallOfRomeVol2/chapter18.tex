\chapter{Character Of Constantine And His Sons.}
\section{Part \thesection.}

\textit{Character Of Constantine. — Gothic War. — Death Of
Constantine. — Division Of The Empire Among His Three Sons. — Persian
War. — Tragic Deaths Of Constantine The Younger And
Constans. — Usurpation Of Magnentius. — Civil War. — Victory Of
Constantius.}
\vspace{\onelineskip}

The character of the prince who removed the seat of empire, and
introduced such important changes into the civil and religious
constitution of his country, has fixed the attention, and divided
the opinions, of mankind. By the grateful zeal of the Christians,
the deliverer of the church has been decorated with every
attribute of a hero, and even of a saint; while the discontent of
the vanquished party has compared Constantine to the most
abhorred of those tyrants, who, by their vice and weakness,
dishonored the Imperial purple. The same passions have in some
degree been perpetuated to succeeding generations, and the
character of Constantine is considered, even in the present age,
as an object either of satire or of panegyric. By the impartial
union of those defects which are confessed by his warmest
admirers, and of those virtues which are acknowledged by his
most-implacable enemies, we might hope to delineate a just
portrait of that extraordinary man, which the truth and candor of
history should adopt without a blush.\textsuperscript{1} But it would soon appear,
that the vain attempt to blend such discordant colors, and to
reconcile such inconsistent qualities, must produce a figure
monstrous rather than human, unless it is viewed in its proper
and distinct lights, by a careful separation of the different
periods of the reign of Constantine.

\pagenote[1]{On ne se trompera point sur Constantin, en croyant
tout le mal ru’en dit Eusebe, et tout le bien qu’en dit Zosime.
Fleury, Hist. Ecclesiastique, tom. iii. p. 233. Eusebius and
Zosimus form indeed the two extremes of flattery and invective.
The intermediate shades are expressed by those writers, whose
character or situation variously tempered the influence of their
religious zeal.}

The person, as well as the mind, of Constantine, had been
enriched by nature with her choicest endowments. His stature was
lofty, his countenance majestic, his deportment graceful; his
strength and activity were displayed in every manly exercise, and
from his earliest youth, to a very advanced season of life, he
preserved the vigor of his constitution by a strict adherence to
the domestic virtues of chastity and temperance. He delighted in
the social intercourse of familiar conversation; and though he
might sometimes indulge his disposition to raillery with less
reserve than was required by the severe dignity of his station,
the courtesy and liberality of his manners gained the hearts of
all who approached him. The sincerity of his friendship has been
suspected; yet he showed, on some occasions, that he was not
incapable of a warm and lasting attachment. The disadvantage of
an illiterate education had not prevented him from forming a just
estimate of the value of learning; and the arts and sciences
derived some encouragement from the munificent protection of
Constantine. In the despatch of business, his diligence was
indefatigable; and the active powers of his mind were almost
continually exercised in reading, writing, or meditating, in
giving audiences to ambassadors, and in examining the complaints
of his subjects. Even those who censured the propriety of his
measures were compelled to acknowledge, that he possessed
magnanimity to conceive, and patience to execute, the most
arduous designs, without being checked either by the prejudices
of education, or by the clamors of the multitude. In the field,
he infused his own intrepid spirit into the troops, whom he
conducted with the talents of a consummate general; and to his
abilities, rather than to his fortune, we may ascribe the signal
victories which he obtained over the foreign and domestic foes of
the republic. He loved glory as the reward, perhaps as the
motive, of his labors. The boundless ambition, which, from the
moment of his accepting the purple at York, appears as the ruling
passion of his soul, may be justified by the dangers of his own
situation, by the character of his rivals, by the consciousness
of superior merit, and by the prospect that his success would
enable him to restore peace and order to the distracted empire.
In his civil wars against Maxentius and Licinius, he had engaged
on his side the inclinations of the people, who compared the
undissembled vices of those tyrants with the spirit of wisdom and
justice which seemed to direct the general tenor of the
administration of Constantine.\textsuperscript{2}

\pagenote[2]{The virtues of Constantine are collected for the
most part from Eutropius and the younger Victor, two sincere
pagans, who wrote after the extinction of his family. Even
Zosimus, and the \textit{Emperor} Julian, acknowledge his personal
courage and military achievements.}

Had Constantine fallen on the banks of the Tyber, or even in the
plains of Hadrianople, such is the character which, with a few
exceptions, he might have transmitted to posterity. But the
conclusion of his reign (according to the moderate and indeed
tender sentence of a writer of the same age) degraded him from
the rank which he had acquired among the most deserving of the
Roman princes.\textsuperscript{3} In the life of Augustus, we behold the tyrant of
the republic, converted, almost by imperceptible degrees, into
the father of his country, and of human kind. In that of
Constantine, we may contemplate a hero, who had so long inspired
his subjects with love, and his enemies with terror, degenerating
into a cruel and dissolute monarch, corrupted by his fortune, or
raised by conquest above the necessity of dissimulation. The
general peace which he maintained during the last fourteen years
of his reign, was a period of apparent splendor rather than of
real prosperity; and the old age of Constantine was disgraced by
the opposite yet reconcilable vices of rapaciousness and
prodigality. The accumulated treasures found in the palaces of
Maxentius and Licinius, were lavishly consumed; the various
innovations introduced by the conqueror, were attended with an
increasing expense; the cost of his buildings, his court, and his
festivals, required an immediate and plentiful supply; and the
oppression of the people was the only fund which could support
the magnificence of the sovereign.\textsuperscript{4} His unworthy favorites,
enriched by the boundless liberality of their master, usurped
with impunity the privilege of rapine and corruption.\textsuperscript{5} A secret
but universal decay was felt in every part of the public
administration, and the emperor himself, though he still retained
the obedience, gradually lost the esteem, of his subjects. The
dress and manners, which, towards the decline of life, he chose
to affect, served only to degrade him in the eyes of mankind. The
Asiatic pomp, which had been adopted by the pride of Diocletian,
assumed an air of softness and effeminacy in the person of
Constantine. He is represented with false hair of various colors,
laboriously arranged by the skilful artists to the times; a
diadem of a new and more expensive fashion; a profusion of gems
and pearls, of collars and bracelets, and a variegated flowing
robe of silk, most curiously embroidered with flowers of gold. In
such apparel, scarcely to be excused by the youth and folly of
Elagabalus, we are at a loss to discover the wisdom of an aged
monarch, and the simplicity of a Roman veteran.\textsuperscript{6} A mind thus
relaxed by prosperity and indulgence, was incapable of rising to
that magnanimity which disdains suspicion, and dares to forgive.
The deaths of Maximian and Licinius may perhaps be justified by
the maxims of policy, as they are taught in the schools of
tyrants; but an impartial narrative of the executions, or rather
murders, which sullied the declining age of Constantine, will
suggest to our most candid thoughts the idea of a prince who
could sacrifice without reluctance the laws of justice, and the
feelings of nature, to the dictates either of his passions or of
his interest.

\pagenote[3]{See Eutropius, x. 6. In primo Imperii tempore
optimis principibus, ultimo mediis comparandus. From the ancient
Greek version of Poeanius, (edit. Havercamp. p. 697,) I am
inclined to suspect that Eutropius had originally written \textit{vix}
mediis; and that the offensive monosyllable was dropped by the
wilful inadvertency of transcribers. Aurelius Victor expresses
the general opinion by a vulgar and indeed obscure proverb.
\textit{Trachala} decem annis præstantissimds; duodecim sequentibus
\textit{latro;} decem novissimis textit{pupillus} ob immouicas profusiones.}

\pagenote[4]{Julian, Orat. i. p. 8, in a flattering discourse
pronounced before the son of Constantine; and Cæsares, p. 336.
Zosimus, p. 114, 115. The stately buildings of Constantinople,
\&c., may be quoted as a lasting and unexceptionable proof of the
profuseness of their founder.}

\pagenote[5]{The impartial Ammianus deserves all our confidence.
Proximorum fauces aperuit primus omnium Constantinus. L. xvi. c.
8. Eusebius himself confesses the abuse, (Vit. Constantin. l. iv.
c. 29, 54;) and some of the Imperial laws feebly point out the
remedy. See above, p. 146 of this volume.}

\pagenote[6]{Julian, in the Cæsars, attempts to ridicule his
uncle. His suspicious testimony is confirmed, however, by the
learned Spanheim, with the authority of medals, (see Commentaire,
p. 156, 299, 397, 459.) Eusebius (Orat. c. 5) alleges, that
Constantine dressed for the public, not for himself. Were this
admitted, the vainest coxcomb could never want an excuse.}

The same fortune which so invariably followed the standard of
Constantine, seemed to secure the hopes and comforts of his
domestic life. Those among his predecessors who had enjoyed the
longest and most prosperous reigns, Augustus Trajan, and
Diocletian, had been disappointed of posterity; and the frequent
revolutions had never allowed sufficient time for any Imperial
family to grow up and multiply under the shade of the purple. But
the royalty of the Flavian line, which had been first ennobled by
the Gothic Claudius, descended through several generations; and
Constantine himself derived from his royal father the hereditary
honors which he transmitted to his children. The emperor had been
twice married. Minervina, the obscure but lawful object of his
youthful attachment,\textsuperscript{7} had left him only one son, who was called
Crispus. By Fausta, the daughter of Maximian, he had three
daughters, and three sons known by the kindred names of
Constantine, Constantius, and Constans. The unambitious brothers
of the great Constantine, Julius Constantius, Dalmatius, and
Hannibalianus,\textsuperscript{8} were permitted to enjoy the most honorable rank,
and the most affluent fortune, that could be consistent with a
private station. The youngest of the three lived without a name,
and died without posterity. His two elder brothers obtained in
marriage the daughters of wealthy senators, and propagated new
branches of the Imperial race. Gallus and Julian afterwards
became the most illustrious of the children of Julius
Constantius, the \textit{Patrician}. The two sons of Dalmatius, who had
been decorated with the vain title of \textit{Censor}, were named
Dalmatius and Hannibalianus. The two sisters of the great
Constantine, Anastasia and Eutropia, were bestowed on Optatus and
Nepotianus, two senators of noble birth and of consular dignity.
His third sister, Constantia, was distinguished by her
preëminence of greatness and of misery. She remained the widow of
the vanquished Licinius; and it was by her entreaties, that an
innocent boy, the offspring of their marriage, preserved, for
some time, his life, the title of Cæsar, and a precarious hope of
the succession. Besides the females, and the allies of the
Flavian house, ten or twelve males, to whom the language of
modern courts would apply the title of princes of the blood,
seemed, according to the order of their birth, to be destined
either to inherit or to support the throne of Constantine. But in
less than thirty years, this numerous and increasing family was
reduced to the persons of Constantius and Julian, who alone had
survived a series of crimes and calamities, such as the tragic
poets have deplored in the devoted lines of Pelops and of Cadmus.

\pagenote[7]{Zosimus and Zonaras agree in representing Minervina
as the concubine of Constantine; but Ducange has very gallantly
rescued her character, by producing a decisive passage from one
of the panegyrics: “Ab ipso fine pueritiæ te matrimonii legibus
dedisti.”}

\pagenote[8]{Ducange (Familiæ Byzantinæ, p. 44) bestows on him,
after Zosimus, the name of Constantine; a name somewhat unlikely,
as it was already occupied by the elder brother. That of
Hannibalianus is mentioned in the Paschal Chronicle, and is
approved by Tillemont. Hist. des Empereurs, tom. iv. p. 527.}

Crispus, the eldest son of Constantine, and the presumptive heir
of the empire, is represented by impartial historians as an
amiable and accomplished youth. The care of his education, or at
least of his studies, was intrusted to Lactantius, the most
eloquent of the Christians; a preceptor admirably qualified to
form the taste, and the excite the virtues, of his illustrious
disciple.\textsuperscript{9} At the age of seventeen, Crispus was invested with
the title of Cæsar, and the administration of the Gallic
provinces, where the inroads of the Germans gave him an early
occasion of signalizing his military prowess. In the civil war
which broke out soon afterwards, the father and son divided their
powers; and this history has already celebrated the valor as well
as conduct displayed by the latter, in forcing the straits of the
Hellespont, so obstinately defended by the superior fleet of
Lacinius. This naval victory contributed to determine the event
of the war; and the names of Constantine and of Crispus were
united in the joyful acclamations of their eastern subjects; who
loudly proclaimed, that the world had been subdued, and was now
governed, by an emperor endowed with every virtue; and by his
illustrious son, a prince beloved of Heaven, and the lively image
of his father’s perfections. The public favor, which seldom
accompanies old age, diffused its lustre over the youth of
Crispus. He deserved the esteem, and he engaged the affections,
of the court, the army, and the people. The experienced merit of
a reigning monarch is acknowledged by his subjects with
reluctance, and frequently denied with partial and discontented
murmurs; while, from the opening virtues of his successor, they
fondly conceive the most unbounded hopes of private as well as
public felicity.\textsuperscript{10}

\pagenote[9]{Jerom. in Chron. The poverty of Lactantius may be
applied either to the praise of the disinterested philosopher, or
to the shame of the unfeeling patron. See Tillemont, Mém.
Ecclesiast. tom. vi. part 1. p. 345. Dupin, Bibliothèque
Ecclesiast. tom. i. p. 205. Lardner’s Credibility of the Gospel
History, part ii. vol. vii. p. 66.}

\pagenote[10]{Euseb. Hist. Ecclesiast. l. x. c. 9. Eutropius (x.
6) styles him “egregium virum;” and Julian (Orat. i.) very
plainly alludes to the exploits of Crispus in the civil war. See
Spanheim, Comment. p. 92.}

This dangerous popularity soon excited the attention of
Constantine, who, both as a father and as a king, was impatient
of an equal. Instead of attempting to secure the allegiance of
his son by the generous ties of confidence and gratitude, he
resolved to prevent the mischiefs which might be apprehended from
dissatisfied ambition. Crispus soon had reason to complain, that
while his infant brother Constantius was sent, with the title of
Cæsar, to reign over his peculiar department of the Gallic
provinces,\textsuperscript{11} \textit{he}, a prince of mature years, who had performed
such recent and signal services, instead of being raised to the
superior rank of Augustus, was confined almost a prisoner to his
father’s court; and exposed, without power or defence, to every
calumny which the malice of his enemies could suggest. Under such
painful circumstances, the royal youth might not always be able
to compose his behavior, or suppress his discontent; and we may
be assured, that he was encompassed by a train of indiscreet or
perfidious followers, who assiduously studied to inflame, and who
were perhaps instructed to betray, the unguarded warmth of his
resentment. An edict of Constantine, published about this time,
manifestly indicates his real or affected suspicions, that a
secret conspiracy had been formed against his person and
government. By all the allurements of honors and rewards, he
invites informers of every degree to accuse without exception his
magistrates or ministers, his friends or his most intimate
favorites, protesting, with a solemn asseveration, that he
himself will listen to the charge, that he himself will revenge
his injuries; and concluding with a prayer, which discovers some
apprehension of danger, that the providence of the Supreme Being
may still continue to protect the safety of the emperor and of
the empire.\textsuperscript{12}

\pagenote[11]{Compare Idatius and the Paschal Chronicle, with
Ammianus, (l, xiv. c. 5.) The \textit{year} in which Constantius was
created Cæsar seems to be more accurately fixed by the two
chronologists; but the historian who lived in his court could not
be ignorant of the \textit{day} of the anniversary. For the appointment
of the new Cæsar to the provinces of Gaul, see Julian, Orat. i.
p. 12, Godefroy, Chronol. Legum, p. 26. and Blondel, de Primauté
de l’Eglise, p. 1183.}

\pagenote[12]{Cod. Theod. l. ix. tit. iv. Godefroy suspected the
secret motives of this law. Comment. tom. iii. p. 9.}

The informers, who complied with so liberal an invitation, were
sufficiently versed in the arts of courts to select the friends
and adherents of Crispus as the guilty persons; nor is there any
reason to distrust the veracity of the emperor, who had promised
an ample measure of revenge and punishment. The policy of
Constantine maintained, however, the same appearances of regard
and confidence towards a son, whom he began to consider as his
most irreconcilable enemy. Medals were struck with the customary
vows for the long and auspicious reign of the young Cæsar;\textsuperscript{13} and
as the people, who were not admitted into the secrets of the
palace, still loved his virtues, and respected his dignity, a
poet who solicits his recall from exile, adores with equal
devotion the majesty of the father and that of the son.\textsuperscript{14} The
time was now arrived for celebrating the august ceremony of the
twentieth year of the reign of Constantine; and the emperor, for
that purpose, removed his court from Nicomedia to Rome, where the
most splendid preparations had been made for his reception. Every
eye, and every tongue, affected to express their sense of the
general happiness, and the veil of ceremony and dissimulation was
drawn for a while over the darkest designs of revenge and murder.\textsuperscript{15}
In the midst of the festival, the unfortunate Crispus was
apprehended by order of the emperor, who laid aside the
tenderness of a father, without assuming the equity of a judge.
The examination was short and private;\textsuperscript{16} and as it was thought
decent to conceal the fate of the young prince from the eyes of
the Roman people, he was sent under a strong guard to Pola, in
Istria, where, soon afterwards, he was put to death, either by
the hand of the executioner, or by the more gentle operations of
poison.\textsuperscript{17} The Cæsar Licinius, a youth of amiable manners, was
involved in the ruin of Crispus:\textsuperscript{18} and the stern jealousy of
Constantine was unmoved by the prayers and tears of his favorite
sister, pleading for the life of a son, whose rank was his only
crime, and whose loss she did not long survive. The story of
these unhappy princes, the nature and evidence of their guilt,
the forms of their trial, and the circumstances of their death,
were buried in mysterious obscurity; and the courtly bishop, who
has celebrated in an elaborate work the virtues and piety of his
hero, observes a prudent silence on the subject of these tragic
events.\textsuperscript{19} Such haughty contempt for the opinion of mankind,
whilst it imprints an indelible stain on the memory of
Constantine, must remind us of the very different behavior of one
of the greatest monarchs of the present age. The Czar Peter, in
the full possession of despotic power, submitted to the judgment
of Russia, of Europe, and of posterity, the reasons which had
compelled him to subscribe the condemnation of a criminal, or at
least of a degenerate son.\textsuperscript{20}

\pagenote[13]{Ducange, Fam. Byzant. p. 28. Tillemont, tom. iv. p.
610.}

\pagenote[14]{His name was Porphyrius Optatianus. The date of his
panegyric, written, according to the taste of the age, in vile
acrostics, is settled by Scaliger ad Euseb. p. 250, Tillemont,
tom. iv. p. 607, and Fabricius, Biblioth. Latin, l. iv. c. 1.}

\pagenote[15]{Zosim. l. ii. p. 103. Godefroy, Chronol. Legum, p.
28.}

\pagenote[16]{The elder Victor, who wrote under the next reign,
speaks with becoming caution. “Natu grandior incertum qua causa,
patris judicio occidisset.” If we consult the succeeding writers,
Eutropius, the younger Victor, Orosius, Jerom, Zosimus,
Philostorgius, and Gregory of Tours, their knowledge will appear
gradually to increase, as their means of information must have
diminished—a circumstance which frequently occurs in historical
disquisition.}

\pagenote[17]{Ammianus (l. xiv. c. 11) uses the general
expression of peremptum Codinus (p. 34) beheads the young prince;
but Sidonius Apollinaris (Epistol. v. 8,) for the sake perhaps of
an antithesis to Fausta’s \textit{warm} bath, chooses to administer a
draught of \textit{cold} poison.}

\pagenote[18]{Sororis filium, commodæ indolis juvenem. Eutropius,
x. 6 May I not be permitted to conjecture that Crispus had
married Helena the daughter of the emperor Licinius, and that on
the happy delivery of the princess, in the year 322, a general
pardon was granted by Constantine? See Ducange, Fam. Byzant. p.
47, and the law (l. ix. tit. xxxvii.) of the Theodosian code,
which has so much embarrassed the interpreters. Godefroy, tom.
iii. p. 267 * Note: This conjecture is very doubtful. The
obscurity of the law quoted from the Theodosian code scarcely
allows any inference, and there is extant but one meda which can
be attributed to a Helena, wife of Crispus.}

\pagenote[19]{See the life of Constantine, particularly l. ii. c.
19, 20. Two hundred and fifty years afterwards Evagrius (l. iii.
c. 41) deduced from the silence of Eusebius a vain argument
against the reality of the fact.}

\pagenote[20]{Histoire de Pierre le Grand, par Voltaire, part ii.
c. 10.}

The innocence of Crispus was so universally acknowledged, that
the modern Greeks, who adore the memory of their founder, are
reduced to palliate the guilt of a parricide, which the common
feelings of human nature forbade them to justify. They pretend,
that as soon as the afflicted father discovered the falsehood of
the accusation by which his credulity had been so fatally misled,
he published to the world his repentance and remorse; that he
mourned forty days, during which he abstained from the use of the
bath, and all the ordinary comforts of life; and that, for the
lasting instruction of posterity, he erected a golden statue of
Crispus, with this memorable inscription: To my son, whom I
unjustly condemned.\textsuperscript{21} A tale so moral and so interesting would
deserve to be supported by less exceptionable authority; but if
we consult the more ancient and authentic writers, they will
inform us, that the repentance of Constantine was manifested only
in acts of blood and revenge; and that he atoned for the murder
of an innocent son, by the execution, perhaps, of a guilty wife.
They ascribe the misfortunes of Crispus to the arts of his
step-mother Fausta, whose implacable hatred, or whose
disappointed love, renewed in the palace of Constantine the
ancient tragedy of Hippolitus and of Phædra.\textsuperscript{22} Like the daughter
of Minos, the daughter of Maximian accused her son-in-law of an
incestuous attempt on the chastity of his father’s wife; and
easily obtained, from the jealousy of the emperor, a sentence of
death against a young prince, whom she considered with reason as
the most formidable rival of her own children. But Helena, the
aged mother of Constantine, lamented and revenged the untimely
fate of her grandson Crispus; nor was it long before a real or
pretended discovery was made, that Fausta herself entertained a
criminal connection with a slave belonging to the Imperial
stables.\textsuperscript{23} Her condemnation and punishment were the instant
consequences of the charge; and the adulteress was suffocated by
the steam of a bath, which, for that purpose, had been heated to
an extraordinary degree.\textsuperscript{24} By some it will perhaps be thought,
that the remembrance of a conjugal union of twenty years, and the
honor of their common offspring, the destined heirs of the
throne, might have softened the obdurate heart of Constantine,
and persuaded him to suffer his wife, however guilty she might
appear, to expiate her offences in a solitary prison. But it
seems a superfluous labor to weigh the propriety, unless we could
ascertain the truth, of this singular event, which is attended
with some circumstances of doubt and perplexity. Those who have
attacked, and those who have defended, the character of
Constantine, have alike disregarded two very remarkable passages
of two orations pronounced under the succeeding reign. The former
celebrates the virtues, the beauty, and the fortune of the
empress Fausta, the daughter, wife, sister, and mother of so many
princes.\textsuperscript{25} The latter asserts, in explicit terms, that the
mother of the younger Constantine, who was slain three years
after his father’s death, survived to weep over the fate of her
son.\textsuperscript{26} Notwithstanding the positive testimony of several writers
of the Pagan as well as of the Christian religion, there may
still remain some reason to believe, or at least to suspect, that
Fausta escaped the blind and suspicious cruelty of her husband.\textsuperscript{2611}
The deaths of a son and a nephew, with the execution of a
great number of respectable, and perhaps innocent friends,\textsuperscript{27} who
were involved in their fall, may be sufficient, however, to
justify the discontent of the Roman people, and to explain the
satirical verses affixed to the palace gate, comparing the
splendid and bloody reigns of Constantine and Nero.\textsuperscript{28}

\pagenote[21]{In order to prove that the statue was erected by
Constantine, and afterwards concealed by the malice of the
Arians, Codinus very readily creates (p. 34) two witnesses,
Hippolitus, and the younger Herodotus, to whose imaginary
histories he appeals with unblushing confidence.}

\pagenote[22]{Zosimus (l. ii. p. 103) may be considered as our
original. The ingenuity of the moderns, assisted by a few hints
from the ancients, has illustrated and improved his obscure and
imperfect narrative.}

\pagenote[23]{Philostorgius, l. ii. c. 4. Zosimus (l. ii. p. 104,
116) imputes to Constantine the death of two wives, of the
innocent Fausta, and of an adulteress, who was the mother of his
three successors. According to Jerom, three or four years elapsed
between the death of Crispus and that of Fausta. The elder Victor
is prudently silent.}

\pagenote[24]{If Fausta was put to death, it is reasonable to
believe that the private apartments of the palace were the scene
of her execution. The orator Chrysostom indulges his fancy by
exposing the naked desert mountain to be devoured by wild
beasts.}

\pagenote[25]{Julian. Orat. i. He seems to call her the mother of
Crispus. She might assume that title by adoption. At least, she
was not considered as his mortal enemy. Julian compares the
fortune of Fausta with that of Parysatis, the Persian queen. A
Roman would have more naturally recollected the second Agrippina:
Et moi, qui sur le trone ai suivi mes ancêtres: Moi, fille,
femme,sœur, et mere de vos maitres.}

\pagenote[26]{Monod. in Constantin. Jun. c. 4, ad Calcem Eutrop.
edit. Havercamp. The orator styles her the most divine and pious
of queens.}

\pagenote[2611]{Manso (Leben Constantins, p. 65) treats this
inference o: Gibbon, and the authorities to which he appeals,
with too much contempt, considering the general scantiness of
proof on this curious question.—M.}

\pagenote[27]{Interfecit numerosos amicos. Eutrop. xx. 6.}

\pagenote[28]{Saturni aurea sæcula quis requirat? Sunt hæc
gemmea, sed Neroniana. Sidon. Apollinar. v. 8. ——It is somewhat
singular that these satirical lines should be attributed, not to
an obscure libeller, or a disappointed patriot, but to Ablavius,
prime minister and favorite of the emperor. We may now perceive
that the imprecations of the Roman people were dictated by
humanity, as well as by superstition. Zosim. l. ii. p. 105.}

\section{Part \thesection.}

By the death of Crispus, the inheritance of the empire seemed to
devolve on the three sons of Fausta, who have been already
mentioned under the names of Constantine, of Constantius, and of
Constans. These young princes were successively invested with the
title of Cæsar; and the dates of their promotion may be referred
to the tenth, the twentieth, and the thirtieth years of the reign
of their father.\textsuperscript{29} This conduct, though it tended to multiply
the future masters of the Roman world, might be excused by the
partiality of paternal affection; but it is not so easy to
understand the motives of the emperor, when he endangered the
safety both of his family and of his people, by the unnecessary
elevation of his two nephews, Dalmatius and Hannibalianus. The
former was raised, by the title of Cæsar, to an equality with his
cousins. In favor of the latter, Constantine invented the new and
singular appellation of \textit{Nobilissimus;}\textsuperscript{30} to which he annexed
the flattering distinction of a robe of purple and gold. But of
the whole series of Roman princes in any age of the empire,
Hannibalianus alone was distinguished by the title of King; a
name which the subjects of Tiberius would have detested, as the
profane and cruel insult of capricious tyranny. The use of such a
title, even as it appears under the reign of Constantine, is a
strange and unconnected fact, which can scarcely be admitted on
the joint authority of Imperial medals and contemporary writers.\textsuperscript{31} \textsuperscript{3111}

\pagenote[29]{Euseb. Orat. in Constantin. c. 3. These dates are
sufficiently correct to justify the orator.}

\pagenote[30]{Zosim. l. ii. p. 117. Under the predecessors of
Constantine, \textit{Nobilissimus} was a vague epithet, rather than a
legal and determined title.}

\pagenote[31]{Adstruunt nummi veteres ac singulares. Spanheim de
Usu Numismat. Dissertat. xii. vol. ii. p. 357. Ammianus speaks of
this Roman king (l. xiv. c. l, and Valesius ad loc.) The Valesian
fragment styles him King of kings; and the Paschal Chronicle
acquires the weight of Latin evidence.}

\pagenote[3111]{Hannibalianus is always designated in these
authors by the title of king. There still exist medals struck to
his honor, on which the same title is found, Fl. Hannibaliano
Regi. See Eckhel, Doct. Num. t. viii. 204. Armeniam nationesque
circum socias habebat, says Aur. Victor, p. 225. The writer means
the Lesser Armenia. Though it is not possible to question a fact
supported by such respectable authorities, Gibbon considers it
inexplicable and incredible. It is a strange abuse of the
privilege of doubting, to refuse all belief in a fact of such
little importance in itself, and attested thus formally by
contemporary authors and public monuments. St. Martin note to Le
Beau i. 341.—M.}

The whole empire was deeply interested in the education of these
five youths, the acknowledged successors of Constantine. The
exercise of the body prepared them for the fatigues of war and
the duties of active life. Those who occasionally mention the
education or talents of Constantius, allow that he excelled in
the gymnastic arts of leaping and running that he was a dexterous
archer, a skilful horseman, and a master of all the different
weapons used in the service either of the cavalry or of the
infantry.\textsuperscript{32} The same assiduous cultivation was bestowed, though
not perhaps with equal success, to improve the minds of the sons
and nephews of Constantine.\textsuperscript{33} The most celebrated professors of
the Christian faith, of the Grecian philosophy, and of the Roman
jurisprudence, were invited by the liberality of the emperor, who
reserved for himself the important task of instructing the royal
youths in the science of government, and the knowledge of
mankind. But the genius of Constantine himself had been formed by
adversity and experience. In the free intercourse of private
life, and amidst the dangers of the court of Galerius, he had
learned to command his own passions, to encounter those of his
equals, and to depend for his present safety and future greatness
on the prudence and firmness of his personal conduct. His
destined successors had the misfortune of being born and educated
in the imperial purple. Incessantly surrounded with a train of
flatterers, they passed their youth in the enjoyment of luxury,
and the expectation of a throne; nor would the dignity of their
rank permit them to descend from that elevated station from
whence the various characters of human nature appear to wear a
smooth and uniform aspect. The indulgence of Constantine admitted
them, at a very tender age, to share the administration of the
empire; and they studied the art of reigning, at the expense of
the people intrusted to their care. The younger Constantine was
appointed to hold his court in Gaul; and his brother Constantius
exchanged that department, the ancient patrimony of their father,
for the more opulent, but less martial, countries of the East.
Italy, the Western Illyricum, and Africa, were accustomed to
revere Constans, the third of his sons, as the representative of
the great Constantine. He fixed Dalmatius on the Gothic frontier,
to which he annexed the government of Thrace, Macedonia, and
Greece. The city of Cæsarea was chosen for the residence of
Hannibalianus; and the provinces of Pontus, Cappadocia, and the
Lesser Armenia, were destined to form the extent of his new
kingdom. For each of these princes a suitable establishment was
provided. A just proportion of guards, of legions, and of
auxiliaries, was allotted for their respective dignity and
defence. The ministers and generals, who were placed about their
persons, were such as Constantine could trust to assist, and even
to control, these youthful sovereigns in the exercise of their
delegated power. As they advanced in years and experience, the
limits of their authority were insensibly enlarged: but the
emperor always reserved for himself the title of Augustus; and
while he showed the \textit{Cæsars} to the armies and provinces, he
maintained every part of the empire in equal obedience to its
supreme head.\textsuperscript{34} The tranquillity of the last fourteen years of
his reign was scarcely interrupted by the contemptible
insurrection of a camel-driver in the Island of Cyprus,\textsuperscript{35} or by
the active part which the policy of Constantine engaged him to
assume in the wars of the Goths and Sarmatians.

\pagenote[32]{His dexterity in martial exercises is celebrated by
Julian, (Orat. i. p. 11, Orat. ii. p. 53,) and allowed by
Ammianus, (l. xxi. c. 16.)}

\pagenote[33]{Euseb. in Vit. Constantin. l. iv. c. 51. Julian,
Orat. i. p. 11-16, with Spanheim’s elaborate Commentary.
Libanius, Orat. iii. p. 109. Constantius studied with laudable
diligence; but the dulness of his fancy prevented him from
succeeding in the art of poetry, or even of rhetoric.}

\pagenote[34]{Eusebius, (l. iv. c. 51, 52,) with a design of
exalting the authority and glory of Constantine, affirms, that he
divided the Roman empire as a private citizen might have divided
his patrimony. His distribution of the provinces may be collected
from Eutropius, the two Victors and the Valesian fragment.}

\pagenote[35]{Calocerus, the obscure leader of this rebellion, or
rather tumult, was apprehended and burnt alive in the
market-place of Tarsus, by the vigilance of Dalmatius. See the
elder Victor, the Chronicle of Jerom, and the doubtful traditions
of Theophanes and Cedrenus.}

Among the different branches of the human race, the Sarmatians
form a very remarkable shade; as they seem to unite the manners
of the Asiatic barbarians with the figure and complexion of the
ancient inhabitants of Europe. According to the various accidents
of peace and war, of alliance or conquest, the Sarmatians were
sometimes confined to the banks of the Tanais; and they sometimes
spread themselves over the immense plains which lie between the
Vistula and the Volga.\textsuperscript{36} The care of their numerous flocks and
herds, the pursuit of game, and the exercises of war, or rather
of rapine, directed the vagrant motions of the Sarmatians. The
movable camps or cities, the ordinary residence of their wives
and children, consisted only of large wagons drawn by oxen, and
covered in the form of tents. The military strength of the nation
was composed of cavalry; and the custom of their warriors, to
lead in their hand one or two spare horses, enabled them to
advance and to retreat with a rapid diligence, which surprised
the security, and eluded the pursuit, of a distant enemy.\textsuperscript{37}
Their poverty of iron prompted their rude industry to invent a
sort of cuirass, which was capable of resisting a sword or
javelin, though it was formed only of horses’ hoofs, cut into
thin and polished slices, carefully laid over each other in the
manner of scales or feathers, and strongly sewed upon an under
garment of coarse linen.\textsuperscript{38} The offensive arms of the Sarmatians
were short daggers, long lances, and a weighty bow with a quiver
of arrows. They were reduced to the necessity of employing
fish-bones for the points of their weapons; but the custom of
dipping them in a venomous liquor, that poisoned the wounds which
they inflicted, is alone sufficient to prove the most savage
manners, since a people impressed with a sense of humanity would
have abhorred so cruel a practice, and a nation skilled in the
arts of war would have disdained so impotent a resource.\textsuperscript{39}
Whenever these Barbarians issued from their deserts in quest of
prey, their shaggy beards, uncombed locks, the furs with which
they were covered from head to foot, and their fierce
countenances, which seemed to express the innate cruelty of their
minds, inspired the more civilized provincials of Rome with
horror and dismay.

\pagenote[36]{Cellarius has collected the opinions of the
ancients concerning the European and Asiatic Sarmatia; and M.
D’Anville has applied them to modern geography with the skill and
accuracy which always distinguish that excellent writer.}

\pagenote[37]{Ammian. l. xvii. c. 12. The Sarmatian horses were
castrated to prevent the mischievous accidents which might happen
from the noisy and ungovernable passions of the males.}

\pagenote[38]{Pausanius, l. i. p. 50,. edit. Kuhn. That
inquisitive traveller had carefully examined a Sarmatian cuirass,
which was preserved in the temple of Æsculapius at Athens.}

\pagenote[39]{Aspicis et mitti sub adunco toxica ferro, Et telum
causas mortis habere duas. Ovid, ex Ponto, l. iv. ep. 7, ver.
7.——See in the Recherches sur les Americains, tom. ii. p.
236—271, a very curious dissertation on poisoned darts. The venom
was commonly extracted from the vegetable reign: but that
employed by the Scythians appears to have been drawn from the
viper, and a mixture of human blood.}

The use of poisoned arms, which has been spread over both worlds,
never preserved a savage tribe from the arms of a disciplined
enemy. The tender Ovid, after a youth spent in the enjoyment of
fame and luxury, was condemned to a hopeless exile on the frozen
banks of the Danube, where he was exposed, almost without
defence, to the fury of these monsters of the desert, with whose
stern spirits he feared that his gentle shade might hereafter be
confounded. In his pathetic, but sometimes unmanly lamentations,\textsuperscript{40}
he describes in the most lively colors the dress and manners,
the arms and inroads, of the Getæ and Sarmatians, who were
associated for the purposes of destruction; and from the accounts
of history there is some reason to believe that these Sarmatians
were the Jazygæ, one of the most numerous and warlike tribes of
the nation. The allurements of plenty engaged them to seek a
permanent establishment on the frontiers of the empire. Soon
after the reign of Augustus, they obliged the Dacians, who
subsisted by fishing on the banks of the River Teyss or Tibiscus,
to retire into the hilly country, and to abandon to the
victorious Sarmatians the fertile plains of the Upper Hungary,
which are bounded by the course of the Danube and the
semicircular enclosure of the Carpathian Mountains.\textsuperscript{41} In this
advantageous position, they watched or suspended the moment of
attack, as they were provoked by injuries or appeased by
presents; they gradually acquired the skill of using more
dangerous weapons, and although the Sarmatians did not illustrate
their name by any memorable exploits, they occasionally assisted
their eastern and western neighbors, the Goths and the Germans,
with a formidable body of cavalry. They lived under the irregular
aristocracy of their chieftains:\textsuperscript{42} but after they had received
into their bosom the fugitive Vandals, who yielded to the
pressure of the Gothic power, they seem to have chosen a king
from that nation, and from the illustrious race of the Astingi,
who had formerly dwelt on the hores of the northern ocean.\textsuperscript{43}

\pagenote[40]{The nine books of Poetical Epistles which Ovid
composed during the seven first years of his melancholy exile,
possess, beside the merit of elegance, a double value. They
exhibit a picture of the human mind under very singular
circumstances; and they contain many curious observations, which
no Roman except Ovid, could have an opportunity of making. Every
circumstance which tends to illustrate the history of the
Barbarians, has been drawn together by the very accurate Count de
Buat. Hist. Ancienne des Peuples de l’Europe, tom. iv. c. xvi. p.
286-317}

\pagenote[41]{The Sarmatian Jazygæ were settled on the banks of
Pathissus or Tibiscus, when Pliny, in the year 79, published his
Natural History. See l. iv. c. 25. In the time of Strabo and
Ovid, sixty or seventy years before, they appear to have
inhabited beyond the Getæ, along the coast of the Euxine.}

\pagenote[42]{Principes Sarmaturum Jazygum penes quos civitatis
regimen plebem quoque et vim equitum, qua sola valent,
offerebant. Tacit. Hist. iii. p. 5. This offer was made in the
civil war between Vitellino and Vespasian.}

\pagenote[43]{This hypothesis of a Vandal king reigning over
Sarmatian subjects, seems necessary to reconcile the Goth
Jornandes with the Greek and Latin historians of Constantine. It
may be observed that Isidore, who lived in Spain under the
dominion of the Goths, gives them for enemies, not the Vandals,
but the Sarmatians. See his Chronicle in Grotius, p. 709. Note: I
have already noticed the confusion which must necessarily arise
in history, when names purely \textit{geographical}, as this of
Sarmatia, are taken for \textit{historical} names belonging to a single
nation. We perceive it here; it has forced Gibbon to suppose,
without any reason but the necessity of extricating himself from
his perplexity, that the Sarmatians had taken a king from among
the Vandals; a supposition entirely contrary to the usages of
Barbarians Dacia, at this period, was occupied, not by
Sarmatians, who have never formed a distinct race, but by
Vandals, whom the ancients have often confounded under the
general term Sarmatians. See Gatterer’s Welt-Geschiehte p.
464—G.}

This motive of enmity must have inflamed the subjects of
contention, which perpetually arise on the confines of warlike
and independent nations. The Vandal princes were stimulated by
fear and revenge; the Gothic kings aspired to extend their
dominion from the Euxine to the frontiers of Germany; and the
waters of the Maros, a small river which falls into the Teyss,
were stained with the blood of the contending Barbarians. After
some experience of the superior strength and numbers of their
adversaries, the Sarmatians implored the protection of the Roman
monarch, who beheld with pleasure the discord of the nations, but
who was justly alarmed by the progress of the Gothic arms. As
soon as Constantine had declared himself in favor of the weaker
party, the haughty Araric, king of the Goths, instead of
expecting the attack of the legions, boldly passed the Danube,
and spread terror and devastation through the province of Mæsia.

To oppose the inroad of this destroying host, the aged emperor
took the field in person; but on this occasion either his conduct
or his fortune betrayed the glory which he had acquired in so
many foreign and domestic wars. He had the mortification of
seeing his troops fly before an inconsiderable detachment of the
Barbarians, who pursued them to the edge of their fortified camp,
and obliged him to consult his safety by a precipitate and
ignominious retreat.\textsuperscript{4311} The event of a second and more
successful action retrieved the honor of the Roman name; and the
powers of art and discipline prevailed, after an obstinate
contest, over the efforts of irregular valor. The broken army of
the Goths abandoned the field of battle, the wasted province, and
the passage of the Danube: and although the eldest of the sons of
Constantine was permitted to supply the place of his father, the
merit of the victory, which diffused universal joy, was ascribed
to the auspicious counsels of the emperor himself.

\pagenote[4311]{Gibbon states, that Constantine was defeated by
the Goths in a first battle. No ancient author mentions such an
event. It is, no doubt, a mistake in Gibbon. St Martin, note to
Le Beau. i. 324.—M.}

He contributed at least to improve this advantage, by his
negotiations with the free and warlike people of Chersonesus,\textsuperscript{44}
whose capital, situate on the western coast of the Tauric or
Crimæan peninsula, still retained some vestiges of a Grecian
colony, and was governed by a perpetual magistrate, assisted by a
council of senators, emphatically styled the Fathers of the City.

The Chersonites were animated against the Goths, by the memory of
the wars, which, in the preceding century, they had maintained
with unequal forces against the invaders of their country. They
were connected with the Romans by the mutual benefits of
commerce; as they were supplied from the provinces of Asia with
corn and manufactures, which they purchased with their only
productions, salt, wax, and hides. Obedient to the requisition of
Constantine, they prepared, under the conduct of their magistrate
Diogenes, a considerable army, of which the principal strength
consisted in cross-bows and military chariots. The speedy march
and intrepid attack of the Chersonites, by diverting the
attention of the Goths, assisted the operations of the Imperial
generals. The Goths, vanquished on every side, were driven into
the mountains, where, in the course of a severe campaign, above a
hundred thousand were computed to have perished by cold and
hunger. Peace was at length granted to their humble
supplications; the eldest son of Araric was accepted as the most
valuable hostage; and Constantine endeavored to convince their
chiefs, by a liberal distribution of honors and rewards, how far
the friendship of the Romans was preferable to their enmity. In
the expressions of his gratitude towards the faithful
Chersonites, the emperor was still more magnificent. The pride of
the nation was gratified by the splendid and almost royal
decorations bestowed on their magistrate and his successors. A
perpetual exemption from all duties was stipulated for their
vessels which traded to the ports of the Black Sea. A regular
subsidy was promised, of iron, corn, oil, and of every supply
which could be useful either in peace or war. But it was thought
that the Sarmatians were sufficiently rewarded by their
deliverance from impending ruin; and the emperor, perhaps with
too strict an economy, deducted some part of the expenses of the
war from the customary gratifications which were allowed to that
turbulent nation.

\pagenote[44]{I may stand in need of some apology for having
used, without scruple, the authority of Constantine
Porphyrogenitus, in all that relates to the wars and negotiations
of the Chersonites. I am aware that he was a Greek of the tenth
century, and that his accounts of ancient history are frequently
confused and fabulous. But on this occasion his narrative is, for
the most part, consistent and probable nor is there much
difficulty in conceiving that an emperor might have access to
some secret archives, which had escaped the diligence of meaner
historians. For the situation and history of Chersone, see
Peyssonel, des Peuples barbares qui ont habite les Bords du
Danube, c. xvi. 84-90. ——Gibbon has confounded the inhabitants of
the city of Cherson, the ancient Chersonesus, with the people of
the Chersonesus Taurica. If he had read with more attention the
chapter of Constantius Porphyrogenitus, from which this narrative
is derived, he would have seen that the author clearly
distinguishes the republic of Cherson from the rest of the Tauric
Peninsula, then possessed by the kings of the Cimmerian
Bosphorus, and that the city of Cherson alone furnished succors
to the Romans. The English historian is also mistaken in saying
that the Stephanephoros of the Chersonites was a perpetual
magistrate; since it is easy to discover from the great number of
Stephanephoroi mentioned by Constantine Porphyrogenitus, that
they were annual magistrates, like almost all those which
governed the Grecian republics. St. Martin, note to Le Beau i.
326.—M.}

Exasperated by this apparent neglect, the Sarmatians soon forgot,
with the levity of barbarians, the services which they had so
lately received, and the dangers which still threatened their
safety. Their inroads on the territory of the empire provoked the
indignation of Constantine to leave them to their fate; and he no
longer opposed the ambition of Geberic, a renowned warrior, who
had recently ascended the Gothic throne. Wisumar, the Vandal
king, whilst alone, and unassisted, he defended his dominions
with undaunted courage, was vanquished and slain in a decisive
battle, which swept away the flower of the Sarmatian youth.\textsuperscript{4411}
The remainder of the nation embraced the desperate expedient of
arming their slaves, a hardy race of hunters and herdsmen, by
whose tumultuary aid they revenged their defeat, and expelled the
invader from their confines. But they soon discovered that they
had exchanged a foreign for a domestic enemy, more dangerous and
more implacable. Enraged by their former servitude, elated by
their present glory, the slaves, under the name of Limigantes,
claimed and usurped the possession of the country which they had
saved. Their masters, unable to withstand the ungoverned fury of
the populace, preferred the hardships of exile to the tyranny of
their servants. Some of the fugitive Sarmatians solicited a less
ignominious dependence, under the hostile standard of the Goths.
A more numerous band retired beyond the Carpathian Mountains,
among the Quadi, their German allies, and were easily admitted to
share a superfluous waste of uncultivated land. But the far
greater part of the distressed nation turned their eyes towards
the fruitful provinces of Rome. Imploring the protection and
forgiveness of the emperor, they solemnly promised, as subjects
in peace, and as soldiers in war, the most inviolable fidelity to
the empire which should graciously receive them into its bosom.
According to the maxims adopted by Probus and his successors, the
offers of this barbarian colony were eagerly accepted; and a
competent portion of lands in the provinces of Pannonia, Thrace,
Macedonia, and Italy, were immediately assigned for the
habitation and subsistence of three hundred thousand Sarmatians.\textsuperscript{45} \textsuperscript{4511}

\pagenote[4411]{Gibbon supposes that this war took place because
Constantine had deducted a part of the customary gratifications,
granted by his predecessors to the Sarmatians. Nothing of this
kind appears in the authors. We see, on the contrary, that after
his victory, and to punish the Sarmatia is for the ravages they
had committed, he withheld the sums which it had been the custom
to bestow. St. Martin, note to Le Beau, i. 327.—M.}

\pagenote[45]{The Gothic and Sarmatian wars are related in so
broken and imperfect a manner, that I have been obliged to
compare the following writers, who mutually supply, correct, and
illustrate each other. Those who will take the same trouble, may
acquire a right of criticizing my narrative. Ammianus, l. xvii.
c. 12. Anonym. Valesian. p. 715. Eutropius, x. 7. Sextus Rufus de
Provinciis, c. 26. Julian Orat. i. p. 9, and Spanheim, Comment.
p. 94. Hieronym. in Chron. Euseb. in Vit. Constantin. l. iv. c.
6. Socrates, l. i. c. 18. Sozomen, l. i. c. 8. Zosimus, l. ii. p.
108. Jornandes de Reb. Geticis, c. 22. Isidorus in Chron. p. 709;
in Hist. Gothorum Grotii. Constantin. Porphyrogenitus de
Administrat. Imperii, c. 53, p. 208, edit. Meursii.}

\pagenote[4511]{Compare, on this very obscure but remarkable war,
Manso, Leben Coa xantius, p. 195—M.}

By chastising the pride of the Goths, and by accepting the homage
of a suppliant nation, Constantine asserted the majesty of the
Roman empire; and the ambassadors of Æthiopia, Persia, and the
most remote countries of India, congratulated the peace and
prosperity of his government.\textsuperscript{46} If he reckoned, among the favors
of fortune, the death of his eldest son, of his nephew, and
perhaps of his wife, he enjoyed an uninterrupted flow of private
as well as public felicity, till the thirtieth year of his reign;
a period which none of his predecessors, since Augustus, had been
permitted to celebrate. Constantine survived that solemn festival
about ten months; and at the mature age of sixty-four, after a
short illness, he ended his memorable life at the palace of
Aquyrion, in the suburbs of Nicomedia, whither he had retired for
the benefit of the air, and with the hope of recruiting his
exhausted strength by the use of the warm baths. The excessive
demonstrations of grief, or at least of mourning, surpassed
whatever had been practised on any former occasion.
Notwithstanding the claims of the senate and people of ancient
Rome, the corpse of the deceased emperor, according to his last
request, was transported to the city, which was destined to
preserve the name and memory of its founder. The body of
Constantine adorned with the vain symbols of greatness, the
purple and diadem, was deposited on a golden bed in one of the
apartments of the palace, which for that purpose had been
splendidly furnished and illuminated. The forms of the court were
strictly maintained. Every day, at the appointed hours, the
principal officers of the state, the army, and the household,
approaching the person of their sovereign with bended knees and a
composed countenance, offered their respectful homage as
seriously as if he had been still alive. From motives of policy,
this theatrical representation was for some time continued; nor
could flattery neglect the opportunity of remarking that
Constantine alone, by the peculiar indulgence of Heaven, had
reigned after his death.\textsuperscript{47}

\pagenote[46]{Eusebius (in Vit. Const. l. iv. c. 50) remarks
three circumstances relative to these Indians. 1. They came from
the shores of the eastern ocean; a description which might be
applied to the coast of China or Coromandel. 2. They presented
shining gems, and unknown animals. 3. They protested their kings
had erected statues to represent the supreme majesty of
Constantine.}

\pagenote[47]{Funus relatum in urbem sui nominis, quod sane P. R.
ægerrime tulit. Aurelius Victor. Constantine prepared for himself
a stately tomb in the church of the Holy Apostles. Euseb. l. iv.
c. 60. The best, and indeed almost the only account of the
sickness, death, and funeral of Constantine, is contained in the
fourth book of his Life by Eusebius.}

But this reign could subsist only in empty pageantry; and it was
soon discovered that the will of the most absolute monarch is
seldom obeyed, when his subjects have no longer anything to hope
from his favor, or to dread from his resentment. The same
ministers and generals, who bowed with such referential awe
before the inanimate corpse of their deceased sovereign, were
engaged in secret consultations to exclude his two nephews,
Dalmatius and Hannibalianus, from the share which he had assigned
them in the succession of the empire. We are too imperfectly
acquainted with the court of Constantine to form any judgment of
the real motives which influenced the leaders of the conspiracy;
unless we should suppose that they were actuated by a spirit of
jealousy and revenge against the præfect Ablavius, a proud
favorite, who had long directed the counsels and abused the
confidence of the late emperor. The arguments, by which they
solicited the concurrence of the soldiers and people, are of a
more obvious nature; and they might with decency, as well as
truth, insist on the superior rank of the children of
Constantine, the danger of multiplying the number of sovereigns,
and the impending mischiefs which threatened the republic, from
the discord of so many rival princes, who were not connected by
the tender sympathy of fraternal affection. The intrigue was
conducted with zeal and secrecy, till a loud and unanimous
declaration was procured from the troops, that they would suffer
none except the sons of their lamented monarch to reign over the
Roman empire.\textsuperscript{48} The younger Dalmatius, who was united with his
collateral relations by the ties of friendship and interest, is
allowed to have inherited a considerable share of the abilities
of the great Constantine; but, on this occasion, he does not
appear to have concerted any measure for supporting, by arms, the
just claims which himself and his royal brother derived from the
liberality of their uncle. Astonished and overwhelmed by the tide
of popular fury, they seem to have remained, without the power of
flight or of resistance, in the hands of their implacable
enemies. Their fate was suspended till the arrival of
Constantius, the second, and perhaps the most favored, of the
sons of Constantine.\textsuperscript{49}

\pagenote[48]{Eusebius (l. iv. c. 6) terminates his narrative by
this loyal declaration of the troops, and avoids all the
invidious circumstances of the subsequent massacre.}

\pagenote[49]{The character of Dalmatius is advantageously,
though concisely drawn by Eutropius. (x. 9.) Dalmatius Cæsar
prosperrimâ indole, neque patrou absimilis, \textit{haud multo} post
oppressus est factione militari. As both Jerom and the
Alexandrian Chronicle mention the third year of the Cæsar, which
did not commence till the 18th or 24th of September, A. D. 337,
it is certain that these military factions continued above four
months.}

\section{Part \thesection.}

The voice of the dying emperor had recommended the care of his
funeral to the piety of Constantius; and that prince, by the
vicinity of his eastern station, could easily prevent the
diligence of his brothers, who resided in their distant
government of Italy and Gaul. As soon as he had taken possession
of the palace of Constantinople, his first care was to remove the
apprehensions of his kinsmen, by a solemn oath which he pledged
for their security. His next employment was to find some specious
pretence which might release his conscience from the obligation
of an imprudent promise. The arts of fraud were made subservient
to the designs of cruelty; and a manifest forgery was attested by
a person of the most sacred character. From the hands of the
Bishop of Nicomedia, Constantius received a fatal scroll,
affirmed to be the genuine testament of his father; in which the
emperor expressed his suspicions that he had been poisoned by his
brothers; and conjured his sons to revenge his death, and to
consult their own safety, by the punishment of the guilty.\textsuperscript{50}
Whatever reasons might have been alleged by these unfortunate
princes to defend their life and honor against so incredible an
accusation, they were silenced by the furious clamors of the
soldiers, who declared themselves, at once, their enemies, their
judges, and their executioners. The spirit, and even the forms of
legal proceedings were repeatedly violated in a promiscuous
massacre; which involved the two uncles of Constantius, seven of
his cousins, of whom Dalmatius and Hannibalianus were the most
illustrious, the Patrician Optatus, who had married a sister of
the late emperor, and the Præfect Ablavius, whose power and
riches had inspired him with some hopes of obtaining the purple.
If it were necessary to aggravate the horrors of this bloody
scene, we might add, that Constantius himself had espoused the
daughter of his uncle Julius, and that he had bestowed his sister
in marriage on his cousin Hannibalianus. These alliances, which
the policy of Constantine, regardless of the public prejudice,\textsuperscript{51}
had formed between the several branches of the Imperial house,
served only to convince mankind, that these princes were as cold
to the endearments of conjugal affection, as they were insensible
to the ties of consanguinity, and the moving entreaties of youth
and innocence. Of so numerous a family, Gallus and Julian alone,
the two youngest children of Julius Constantius, were saved from
the hands of the assassins, till their rage, satiated with
slaughter, had in some measure subsided. The emperor Constantius,
who, in the absence of his brothers, was the most obnoxious to
guilt and reproach, discovered, on some future occasions, a faint
and transient remorse for those cruelties which the perfidious
counsels of his ministers, and the irresistible violence of the
troops, had extorted from his unexperienced youth.\textsuperscript{52}

\pagenote[50]{I have related this singular anecdote on the
authority of Philostorgius, l. ii. c. 16. But if such a pretext
was ever used by Constantius and his adherents, it was laid aside
with contempt, as soon as it served their immediate purpose.
Athanasius (tom. i. p. 856) mention the oath which Constantius
had taken for the security of his kinsmen. ——The authority of
Philostorgius is so suspicious, as not to be sufficient to
establish this fact, which Gibbon has inserted in his history as
certain, while in the note he appears to doubt it.—G.}

\pagenote[51]{Conjugia sobrinarum diu ignorata, tempore addito
percrebuisse. Tacit. Annal. xii. 6, and Lipsius ad loc. The
repeal of the ancient law, and the practice of five hundred
years, were insufficient to eradicate the prejudices of the
Romans, who still considered the marriages of cousins-german as a
species of imperfect incest. (Augustin de Civitate Dei, xv. 6;)
and Julian, whose mind was biased by superstition and resentment,
stigmatizes these unnatural alliances between his own cousins
with the opprobrious epithet (Orat. vii. p. 228.). The
jurisprudence of the canons has since received and enforced this
prohibition, without being able to introduce it either into the
civil or the common law of Europe. See on the subject of these
marriages, Taylor’s Civil Law, p. 331. Brouer de Jure Connub. l.
ii. c. 12. Hericourt des Loix Ecclésiastiques, part iii. c. 5.
Fleury, Institutions du Droit Canonique, tom. i. p. 331. Paris,
1767, and Fra Paolo, Istoria del Concilio Trident, l. viii.}

\pagenote[52]{Julian (ad S. P.. Q. Athen. p. 270) charges his
cousin Constantius with the whole guilt of a massacre, from which
he himself so narrowly escaped. His assertion is confirmed by
Athanasius, who, for reasons of a very different nature, was not
less an enemy of Constantius, (tom. i. p. 856.) Zosimus joins in
the same accusation. But the three abbreviators, Eutropius and
the Victors, use very qualifying expressions: “sinente potius
quam jubente;” “incertum quo suasore;” “vi militum.”}

The massacre of the Flavian race was succeeded by a new division
of the provinces; which was ratified in a personal interview of
the three brothers. Constantine, the eldest of the Cæsars,
obtained, with a certain preëminence of rank, the possession of
the new capital, which bore his own name and that of his father.
Thrace, and the countries of the East, were allotted for the
patrimony of Constantius; and Constans was acknowledged as the
lawful sovereign of Italy, Africa, and the Western Illyricum. The
armies submitted to their hereditary right; and they
condescended, after some delay, to accept from the Roman senate
the title of \textit{Augustus}. When they first assumed the reins of
government, the eldest of these princes was twenty-one, the
second twenty, and the third only seventeen, years of age.\textsuperscript{53}

\pagenote[53]{Euseb. in Vit. Constantin. l. iv. c. 69. Zosimus,
l. ii. p. 117. Idat. in Chron. See two notes of Tillemont, Hist.
des Empereurs, tom. iv. p. 1086-1091. The reign of the eldest
brother at Constantinople is noticed only in the Alexandrian
Chronicle.}

While the martial nations of Europe followed the standards of his
brothers, Constantius, at the head of the effeminate troops of
Asia, was left to sustain the weight of the Persian war. At the
decease of Constantine, the throne of the East was filled by
Sapor, son of Hormouz, or Hormisdas, and grandson of Narses, who,
after the victory of Galerius, had humbly confessed the
superiority of the Roman power. Although Sapor was in the
thirtieth year of his long reign, he was still in the vigor of
youth, as the date of his accession, by a very strange fatality,
had preceded that of his birth. The wife of Hormouz remained
pregnant at the time of her husband’s death; and the uncertainty
of the sex, as well as of the event, excited the ambitious hopes
of the princes of the house of Sassan. The apprehensions of civil
war were at length removed, by the positive assurance of the
Magi, that the widow of Hormouz had conceived, and would safely
produce a son. Obedient to the voice of superstition, the
Persians prepared, without delay, the ceremony of his coronation.

A royal bed, on which the queen lay in state, was exhibited in
the midst of the palace; the diadem was placed on the spot, which
might be supposed to conceal the future heir of Artaxerxes, and
the prostrate satraps adored the majesty of their invisible and
insensible sovereign.\textsuperscript{54} If any credit can be given to this
marvellous tale, which seems, however, to be countenanced by the
manners of the people, and by the extraordinary duration of his
reign, we must admire not only the fortune, but the genius, of
Sapor. In the soft, sequestered education of a Persian harem, the
royal youth could discover the importance of exercising the vigor
of his mind and body; and, by his personal merit, deserved a
throne, on which he had been seated, while he was yet unconscious
of the duties and temptations of absolute power. His minority was
exposed to the almost inevitable calamities of domestic discord;
his capital was surprised and plundered by Thair, a powerful king
of Yemen, or Arabia; and the majesty of the royal family was
degraded by the captivity of a princess, the sister of the
deceased king. But as soon as Sapor attained the age of manhood,
the presumptuous Thair, his nation, and his country, fell beneath
the first effort of the young warrior; who used his victory with
so judicious a mixture of rigor and clemency, that he obtained
from the fears and gratitude of the Arabs the title of
\textit{Dhoulacnaf}, or protector of the nation.\textsuperscript{55} \textsuperscript{5511}

\pagenote[54]{Agathias, who lived in the sixth century, is the
author of this story, (l. iv. p. 135, edit. Louvre.) He derived
his information from some extracts of the Persian Chronicles,
obtained and translated by the interpreter Sergius, during his
embassy at that country. The coronation of the mother of Sapor is
likewise mentioned by Snikard, (Tarikh. p. 116,) and D’Herbelot
(Bibliothèque Orientale, p. 703.) ——The author of the
Zenut-ul-Tarikh states, that the lady herself affirmed her belief
of this from the extraordinary liveliness of the infant, and its
lying on the right side. Those who are sage on such subjects must
determine what right she had to be positive from these symptoms.
Malcolm, Hist. of Persia, i 83.—M.}

\pagenote[55]{D’Herbelot, Bibliothèque Orientale, p. 764.}

\pagenote[5511]{Gibbon, according to Sir J. Malcolm, has greatly
mistaken the derivation of this name; it means Zoolaktaf, the
Lord of the Shoulders, from his directing the shoulders of his
captives to be pierced and then dislocated by a string passed
through them. Eastern authors are agreed with respect to the
origin of this title. Malcolm, i. 84. Gibbon took his derivation
from D’Herbelot, who gives both, the latter on the authority of
the Leb. Tarikh.—M.}

The ambition of the Persian, to whom his enemies ascribe the
virtues of a soldier and a statesman, was animated by the desire
of revenging the disgrace of his fathers, and of wresting from
the hands of the Romans the five provinces beyond the Tigris. The
military fame of Constantine, and the real or apparent strength
of his government, suspended the attack; and while the hostile
conduct of Sapor provoked the resentment, his artful negotiations
amused the patience of the Imperial court. The death of
Constantine was the signal of war,\textsuperscript{56} and the actual condition of
the Syrian and Armenian frontier seemed to encourage the Persians
by the prospect of a rich spoil and an easy conquest. The example
of the massacres of the palace diffused a spirit of
licentiousness and sedition among the troops of the East, who
were no longer restrained by their habits of obedience to a
veteran commander. By the prudence of Constantius, who, from the
interview with his brothers in Pannonia, immediately hastened to
the banks of the Euphrates, the legions were gradually restored
to a sense of duty and discipline; but the season of anarchy had
permitted Sapor to form the siege of Nisibis, and to occupy
several of the mo st important fortresses of Mesopotamia.\textsuperscript{57} In
Armenia, the renowned Tiridates had long enjoyed the peace and
glory which he deserved by his valor and fidelity to the cause of
Rome.\textsuperscript{5711} The firm alliance which he maintained with Constantine
was productive of spiritual as well as of temporal benefits; by
the conversion of Tiridates, the character of a saint was applied
to that of a hero, the Christian faith was preached and
established from the Euphrates to the shores of the Caspian, and
Armenia was attached to the empire by the double ties of policy
and religion. But as many of the Armenian nobles still refused to
abandon the plurality of their gods and of their wives, the
public tranquillity was disturbed by a discontented faction,
which insulted the feeble age of their sovereign, and impatiently
expected the hour of his death. He died at length after a reign
of fifty-six years, and the fortune of the Armenian monarchy
expired with Tiridates. His lawful heir was driven into exile,
the Christian priests were either murdered or expelled from their
churches, the barbarous tribes of Albania were solicited to
descend from their mountains; and two of the most powerful
governors, usurping the ensigns or the powers of royalty,
implored the assistance of Sapor, and opened the gates of their
cities to the Persian garrisons. The Christian party, under the
guidance of the Archbishop of Artaxata, the immediate successor
of St. Gregory the Illuminator, had recourse to the piety of
Constantius. After the troubles had continued about three years,
Antiochus, one of the officers of the household, executed with
success the Imperial commission of restoring Chosroes,\textsuperscript{5712} the
son of Tiridates, to the throne of his fathers, of distributing
honors and rewards among the faithful servants of the house of
Arsaces, and of proclaiming a general amnesty, which was accepted
by the greater part of the rebellious satraps. But the Romans
derived more honor than advantage from this revolution. Chosroes
was a prince of a puny stature and a pusillanimous spirit.
Unequal to the fatigues of war, averse to the society of mankind,
he withdrew from his capital to a retired palace, which he built
on the banks of the River Eleutherus, and in the centre of a
shady grove; where he consumed his vacant hours in the rural
sports of hunting and hawking. To secure this inglorious ease, he
submitted to the conditions of peace which Sapor condescended to
impose; the payment of an annual tribute, and the restitution of
the fertile province of Atropatene, which the courage of
Tiridates, and the victorious arms of Galerius, had annexed to
the Armenian monarchy.\textsuperscript{58} \textsuperscript{5811}

\pagenote[56]{Sextus Rufus, (c. 26,) who on this occasion is no
contemptible authority, affirms, that the Persians sued in vain
for peace, and that Constantine was preparing to march against
them: yet the superior weight of the testimony of Eusebius
obliges us to admit the preliminaries, if not the ratification,
of the treaty. See Tillemont, Hist. des Empereurs, tom. iv. p.
420. ——Constantine had endeavored to allay the fury of the
prosecutions, which, at the instigation of the Magi and the Jews,
Sapor had commenced against the Christians. Euseb Vit. Hist.
Theod. i. 25. Sozom. ii. c. 8, 15.—M.}

\pagenote[57]{Julian. Orat. i. p. 20.}

\pagenote[5711]{Tiridates had sustained a war against Maximin.
caused by the hatred of the latter against Christianity. Armenia
was the first \textit{nation} which embraced Christianity. About the
year 276 it was the religion of the king, the nobles, and the
people of Armenia. From St. Martin, Supplement to Le Beau, v. i.
p. 78.——Compare Preface to History of Vartan by Professor
Neumann, p ix.—M.}

\pagenote[5712]{Chosroes was restored probably by Licinius,
between 314 and 319. There was an Antiochus who was præfectus
vigilum at Rome, as appears from the Theodosian Code, (l. iii. de
inf. his quæ sub ty.,) in 326, and from a fragment of the same
work published by M. Amedee Peyron, in 319. He may before this
have been sent into Armenia. St. M. p. 407. [Is it not more
probable that Antiochus was an officer in the service of the
Cæsar who ruled in the East?—M.] Chosroes was succeeded in the
year 322 by his son Diran. Diran was a weak prince, and in the
sixteenth year of his reign. A. D. 337. was betrayed into the
power of the Persians by the treachery of his chamberlain and the
Persian governor of Atropatene or Aderbidjan. He was blinded: his
wife and his son Arsaces shared his captivity, but the princes
and nobles of Armenia claimed the protection of Rome; and this
was the cause of Constantine’s declaration of war against the
Persians.—The king of Persia attempted to make himself master of
Armenia; but the brave resistance of the people, the advance of
Constantius, and a defeat which his army suffered at Oskha in
Armenia, and the failure before Nisibis, forced Shahpour to
submit to terms of peace. Varaz-Shahpour, the perfidious governor
of Atropatene, was flayed alive; Diran and his son were released
from captivity; Diran refused to ascend the throne, and retired
to an obscure retreat: his son Arsaces was crowned king of
Armenia. Arsaces pursued a vacillating policy between the
influence of Rome and Persia, and the war recommenced in the year
345. At least, that was the period of the expedition of
Constantius to the East. See St. Martin, additions to Le Beau, i.
442. The Persians have made an extraordinary romance out of the
history of Shahpour, who went as a spy to Constantinople, was
taken, harnessed like a horse, and carried to witness the
devastation of his kingdom. Malcolm. 84—M.}

\pagenote[58]{Julian. Orat. i. p. 20, 21. Moses of Chorene, l.
ii. c. 89, l. iii. c. 1—9, p. 226—240. The perfect agreement
between the vague hints of the contemporary orator, and the
circumstantial narrative of the national historian, gives light
to the former, and weight to the latter. For the credit of Moses,
it may be likewise observed, that the name of Antiochus is found
a few years before in a civil office of inferior dignity. See
Godefroy, Cod. Theod. tom. vi. p. 350.}

\pagenote[5811]{Gibbon has endeavored, in his History, to make
use of the information furnished by Moses of Chorene, the only
Armenian historian then translated into Latin. Gibbon has not
perceived all the chronological difficulties which occur in the
narrative of that writer. He has not thought of all the critical
discussions which his text ought to undergo before it can be
combined with the relations of the western writers. From want of
this attention, Gibbon has made the facts which he has drawn from
this source more erroneous than they are in the original. This
judgment applies to all which the English historian has derived
from the Armenian author. I have made the History of Moses a
subject of particular attention; and it is with confidence that I
offer the results, which I insert here, and which will appear in
the course of my notes. In order to form a judgment of the
difference which exists between me and Gibbon, I will content
myself with remarking, that throughout he has committed an
anachronism of thirty years, from whence it follows, that he
assigns to the reign of Constantius many events which took place
during that of Constantine. He could not, therefore, discern the
true connection which exists between the Roman history and that
of Armenia, or form a correct notion of the reasons which induced
Constantine, at the close of his life, to make war upon the
Persians, or of the motives which detained Constantius so long in
the East; he does not even mention them. St. Martin, note on Le
Beau, i. 406. I have inserted M. St. Martin’s observations, but I
must add, that the chronology which he proposes, is not generally
received by Armenian scholars, not, I believe, by Professor
Neumann.—M.}

During the long period of the reign of Constantius, the provinces
of the East were afflicted by the calamities of the Persian war.\textsuperscript{5813}
The irregular incursions of the light troops alternately
spread terror and devastation beyond the Tigris and beyond the
Euphrates, from the gates of Ctesiphon to those of Antioch; and
this active service was performed by the Arabs of the desert, who
were divided in their interest and affections; some of their
independent chiefs being enlisted in the party of Sapor, whilst
others had engaged their doubtful fidelity to the emperor.\textsuperscript{59} The
more grave and important operations of the war were conducted
with equal vigor; and the armies of Rome and Persia encountered
each other in nine bloody fields, in two of which Constantius
himself commanded in person.\textsuperscript{60} The event of the day was most
commonly adverse to the Romans, but in the battle of Singara,
their imprudent valor had almost achieved a signal and decisive
victory. The stationary troops of Singara\textsuperscript{6011} retired on the
approach of Sapor, who passed the Tigris over three bridges, and
occupied near the village of Hilleh an advantageous camp, which,
by the labor of his numerous pioneers, he surrounded in one day
with a deep ditch and a lofty rampart. His formidable host, when
it was drawn out in order of battle, covered the banks of the
river, the adjacent heights, and the whole extent of a plain of
above twelve miles, which separated the two armies. Both were
alike impatient to engage; but the Barbarians, after a slight
resistance, fled in disorder; unable to resist, or desirous to
weary, the strength of the heavy legions, who, fainting with heat
and thirst, pursued them across the plain, and cut in pieces a
line of cavalry, clothed in complete armor, which had been posted
before the gates of the camp to protect their retreat.
Constantius, who was hurried along in the pursuit, attempted,
without effect, to restrain the ardor of his troops, by
representing to them the dangers of the approaching night, and
the certainty of completing their success with the return of day.
As they depended much more on their own valor than on the
experience or the abilities of their chief, they silenced by
their clamors his timid remonstrances; and rushing with fury to
the charge, filled up the ditch, broke down the rampart, and
dispersed themselves through the tents to recruit their exhausted
strength, and to enjoy the rich harvest of their labors. But the
prudent Sapor had watched the moment of victory. His army, of
which the greater part, securely posted on the heights, had been
spectators of the action, advanced in silence, and under the
shadow of the night; and his Persian archers, guided by the
illumination of the camp, poured a shower of arrows on a disarmed
and licentious crowd. The sincerity of history\textsuperscript{61} declares, that
the Romans were vanquished with a dreadful slaughter, and that
the flying remnant of the legions was exposed to the most
intolerable hardships. Even the tenderness of panegyric,
confessing that the glory of the emperor was sullied by the
disobedience of his soldiers, chooses to draw a veil over the
circumstances of this melancholy retreat. Yet one of those venal
orators, so jealous of the fame of Constantius, relates, with
amazing coolness, an act of such incredible cruelty, as, in the
judgment of posterity, must imprint a far deeper stain on the
honor of the Imperial name. The son of Sapor, the heir of his
crown, had been made a captive in the Persian camp. The unhappy
youth, who might have excited the compassion of the most savage
enemy, was scourged, tortured, and publicly executed by the
inhuman Romans.\textsuperscript{62}

\pagenote[5813]{It was during this war that a bold flatterer
(whose name is unknown) published the Itineraries of Alexander
and Trajan, in order to direct the \textit{victorious} Constantius in
the footsteps of those great conquerors of the East. The former
of these has been published for the first time by M. Angelo Mai
(Milan, 1817, reprinted at Frankfort, 1818.) It adds so little to
our knowledge of Alexander’s campaigns, that it only excites our
regret that it is not the Itinerary of Trajan, of whose eastern
victories we have no distinct record—M}

\pagenote[59]{Ammianus (xiv. 4) gives a lively description of the
wandering and predatory life of the Saracens, who stretched from
the confines of Assyria to the cataracts of the Nile. It appears
from the adventures of Malchus, which Jerom has related in so
entertaining a manner, that the high road between Beræa and
Edessa was infested by these robbers. See Hieronym. tom. i. p.
256.}

\pagenote[60]{We shall take from Eutropius the general idea of
the war. A Persis enim multa et gravia perpessus, sæpe captis,
oppidis, obsessis urbibus, cæsis exercitibus, nullumque ei contra
Saporem prosperum prælium fuit, nisi quod apud Singaram, \&c. This
honest account is confirmed by the hints of Ammianus, Rufus, and
Jerom. The two first orations of Julian, and the third oration of
Libanius, exhibit a more flattering picture; but the recantation
of both those orators, after the death of Constantius, while it
restores us to the possession of the truth, degrades their own
character, and that of the emperor. The Commentary of Spanheim on
the first oration of Julian is profusely learned. See likewise
the judicious observations of Tillemont, Hist. des Empereurs,
tom. iv. p. 656.}

\pagenote[6011]{Now Sinjar, or the River Claboras.—M.}

\pagenote[61]{Acerrimâ nocturnâ concertatione pugnatum est,
nostrorum copiis ngenti strage confossis. Ammian. xviii. 5. See
likewise Eutropius, x. 10, and S. Rufus, c. 27. ——The Persian
historians, or romancers, do not mention the battle of Singara,
but make the captive Shahpour escape, defeat, and take prisoner,
the Roman emperor. The Roman captives were forced to repair all
the ravages they had committed, even to replanting the smallest
trees. Malcolm. i. 82.—M.}

\pagenote[62]{Libanius, Orat. iii. p. 133, with Julian. Orat. i.
p. 24, and Spanneism’s Commentary, p. 179.}

Whatever advantages might attend the arms of Sapor in the field,
though nine repeated victories diffused among the nations the
fame of his valor and conduct, he could not hope to succeed in
the execution of his designs, while the fortified towns of
Mesopotamia, and, above all, the strong and ancient city of
Nisibis, remained in the possession of the Romans. In the space
of twelve years, Nisibis, which, since the time of Lucullus, had
been deservedly esteemed the bulwark of the East, sustained three
memorable sieges against the power of Sapor; and the disappointed
monarch, after urging his attacks above sixty, eighty, and a
hundred days, was thrice repulsed with loss and ignominy.\textsuperscript{63} This
large and populous city was situate about two days’ journey from
the Tigris, in the midst of a pleasant and fertile plain at the
foot of Mount Masius. A treble enclosure of brick walls was
defended by a deep ditch;\textsuperscript{64} and the intrepid resistance of Count
Lucilianus, and his garrison, was seconded by the desperate
courage of the people. The citizens of Nisibis were animated by
the exhortations of their bishop,\textsuperscript{65} inured to arms by the
presence of danger, and convinced of the intentions of Sapor to
plant a Persian colony in their room, and to lead them away into
distant and barbarous captivity. The event of the two former
sieges elated their confidence, and exasperated the haughty
spirit of the Great King, who advanced a third time towards
Nisibis, at the head of the united forces of Persia and India.
The ordinary machines, invented to batter or undermine the walls,
were rendered ineffectual by the superior skill of the Romans;
and many days had vainly elapsed, when Sapor embraced a
resolution worthy of an eastern monarch, who believed that the
elements themselves were subject to his power. At the stated
season of the melting of the snows in Armenia, the River
Mygdonius, which divides the plain and the city of Nisibis,
forms, like the Nile,\textsuperscript{66} an inundation over the adjacent country.
By the labor of the Persians, the course of the river was stopped
below the town, and the waters were confined on every side by
solid mounds of earth. On this artificial lake, a fleet of armed
vessels filled with soldiers, and with engines which discharged
stones of five hundred pounds weight, advanced in order of
battle, and engaged, almost upon a level, the troops which
defended the ramparts.\textsuperscript{6611} The irresistible force of the waters
was alternately fatal to the contending parties, till at length a
portion of the walls, unable to sustain the accumulated pressure,
gave way at once, and exposed an ample breach of one hundred and
fifty feet. The Persians were instantly driven to the assault,
and the fate of Nisibis depended on the event of the day. The
heavy-armed cavalry, who led the van of a deep column, were
embarrassed in the mud, and great numbers were drowned in the
unseen holes which had been filled by the rushing waters. The
elephants, made furious by their wounds, increased the disorder,
and trampled down thousands of the Persian archers. The Great
King, who, from an exalted throne, beheld the misfortunes of his
arms, sounded, with reluctant indignation, the signal of the
retreat, and suspended for some hours the prosecution of the
attack. But the vigilant citizens improved the opportunity of the
night; and the return of day discovered a new wall of six feet in
height, rising every moment to fill up the interval of the
breach. Notwithstanding the disappointment of his hopes, and the
loss of more than twenty thousand men, Sapor still pressed the
reduction of Nisibis, with an obstinate firmness, which could
have yielded only to the necessity of defending the eastern
provinces of Persia against a formidable invasion of the
Massagetæ.\textsuperscript{67} Alarmed by this intelligence, he hastily
relinquished the siege, and marched with rapid diligence from the
banks of the Tigris to those of the Oxus. The danger and
difficulties of the Scythian war engaged him soon afterwards to
conclude, or at least to observe, a truce with the Roman emperor,
which was equally grateful to both princes; as Constantius
himself, after the death of his two brothers, was involved, by
the revolutions of the West, in a civil contest, which required
and seemed to exceed the most vigorous exertion of his undivided
strength.

\pagenote[63]{See Julian. Orat. i. p. 27, Orat. ii. p. 62, \&c.,
with the Commentary of Spanheim, (p. 188-202,) who illustrates
the circumstances, and ascertains the time of the three sieges of
Nisibis. Their dates are likewise examined by Tillemont, (Hist.
des Empereurs, tom. iv. p. 668, 671, 674.) Something is added
from Zosimus, l. iii. p. 151, and the Alexandrine Chronicle, p.
290.}

\pagenote[64]{Sallust. Fragment. lxxxiv. edit. Brosses, and
Plutarch in Lucull. tom. iii. p. 184. Nisibis is now reduced to
one hundred and fifty houses: the marshy lands produce rice, and
the fertile meadows, as far as Mosul and the Tigris, are covered
with the ruins of towns and allages. See Niebuhr, Voyages, tom.
ii. p. 300-309.}

\pagenote[65]{The miracles which Theodoret (l. ii. c. 30)
ascribes to St. James, Bishop of Edessa, were at least performed
in a worthy cause, the defence of his couutry. He appeared on the
walls under the figure of the Roman emperor, and sent an army of
gnats to sting the trunks of the elephants, and to discomfit the
host of the new Sennacherib.}

\pagenote[66]{Julian. Orat. i. p. 27. Though Niebuhr (tom. ii. p.
307) allows a very considerable swell to the Mygdonius, over
which he saw a bridge of \textit{twelve} arches: it is difficult,
however, to understand this parallel of a trifling rivulet with a
mighty river. There are many circumstances obscure, and almost
unintelligible, in the description of these stupendous
water-works.}

\pagenote[6611]{Macdonald Kinnier observes on these floating
batteries, “As the elevation of place is considerably above the
level of the country in its immediate vicinity, and the Mygdonius
is a very insignificant stream, it is difficult to imagine how
this work could have been accomplished, even with the wonderful
resources which the king must have had at his disposal”
Geographical Memoir. p. 262.—M.}

\pagenote[67]{We are obliged to Zonaras (tom. ii. l. xiii. p. 11)
for this invasion of the Massagetæ, which is perfectly consistent
with the general series of events to which we are darkly led by
the broken history of Ammianus.}

After the partition of the empire, three years had scarcely
elapsed before the sons of Constantine seemed impatient to
convince mankind that they were incapable of contenting
themselves with the dominions which they were unqualified to
govern. The eldest of those princes soon complained, that he was
defrauded of his just proportion of the spoils of their murdered
kinsmen; and though he might yield to the superior guilt and
merit of Constantius, he exacted from Constans the cession of the
African provinces, as an equivalent for the rich countries of
Macedonia and Greece, which his brother had acquired by the death
of Dalmatius. The want of sincerity, which Constantine
experienced in a tedious and fruitless negotiation, exasperated
the fierceness of his temper; and he eagerly listened to those
favorites, who suggested to him that his honor, as well as his
interest, was concerned in the prosecution of the quarrel. At the
head of a tumultuary band, suited for rapine rather than for
conquest, he suddenly broke onto the dominions of Constans, by
the way of the Julian Alps, and the country round Aquileia felt
the first effects of his resentment. The measures of Constans,
who then resided in Dacia, were directed with more prudence and
ability. On the news of his brother’s invasion, he detached a
select and disciplined body of his Illyrian troops, proposing to
follow them in person, with the remainder of his forces. But the
conduct of his lieutenants soon terminated the unnatural contest.

By the artful appearances of flight, Constantine was betrayed
into an ambuscade, which had been concealed in a wood, where the
rash youth, with a few attendants, was surprised, surrounded, and
slain. His body, after it had been found in the obscure stream of
the Alsa, obtained the honors of an Imperial sepulchre; but his
provinces transferred their allegiance to the conqueror, who,
refusing to admit his elder brother Constantius to any share in
these new acquisitions, maintained the undisputed possession of
more than two thirds of the Roman empire.\textsuperscript{68}

\pagenote[68]{The causes and the events of this civil war are
related with much perplexity and contradiction. I have chiefly
followed Zonaras and the younger Victor. The monody (ad Calcem
Eutrop. edit. Havercamp.) pronounced on the death of Constantine,
might have been very instructive; but prudence and false taste
engaged the orator to involve himself in vague declamation.}

\section{Part \thesection.}

The fate of Constans himself was delayed about ten years longer,
and the revenge of his brother’s death was reserved for the more
ignoble hand of a domestic traitor. The pernicious tendency of
the system introduced by Constantine was displayed in the feeble
administration of his sons; who, by their vices and weakness,
soon lost the esteem and affections of their people. The pride
assumed by Constans, from the unmerited success of his arms, was
rendered more contemptible by his want of abilities and
application. His fond partiality towards some German captives,
distinguished only by the charms of youth, was an object of
scandal to the people;\textsuperscript{69} and Magnentius, an ambitious soldier,
who was himself of Barbarian extraction, was encouraged by the
public discontent to assert the honor of the Roman name.\textsuperscript{70} The
chosen bands of Jovians and Herculians, who acknowledged
Magnentius as their leader, maintained the most respectable and
important station in the Imperial camp. The friendship of
Marcellinus, count of the sacred largesses, supplied with a
liberal hand the means of seduction. The soldiers were convinced
by the most specious arguments, that the republic summoned them
to break the bonds of hereditary servitude; and, by the choice of
an active and vigilant prince, to reward the same virtues which
had raised the ancestors of the degenerate Constans from a
private condition to the throne of the world. As soon as the
conspiracy was ripe for execution, Marcellinus, under the
pretence of celebrating his son’s birthday, gave a splendid
entertainment to the \textit{illustrious} and \textit{honorable} persons of the
court of Gaul, which then resided in the city of Autun. The
intemperance of the feast was artfully protracted till a very
late hour of the night; and the unsuspecting guests were tempted
to indulge themselves in a dangerous and guilty freedom of
conversation. On a sudden the doors were thrown open, and
Magnentius, who had retired for a few moments, returned into the
apartment, invested with the diadem and purple. The conspirators
instantly saluted him with the titles of Augustus and Emperor.
The surprise, the terror, the intoxication, the ambitious hopes,
and the mutual ignorance of the rest of the assembly, prompted
them to join their voices to the general acclamation. The guards
hastened to take the oath of fidelity; the gates of the town were
shut; and before the dawn of day, Magnentius became master of the
troops and treasure of the palace and city of Autun. By his
secrecy and diligence he entertained some hopes of surprising the
person of Constans, who was pursuing in the adjacent forest his
favorite amusement of hunting, or perhaps some pleasures of a
more private and criminal nature. The rapid progress of fame
allowed him, however, an instant for flight, though the desertion
of his soldiers and subjects deprived him of the power of
resistance. Before he could reach a seaport in Spain, where he
intended to embark, he was overtaken near Helena,\textsuperscript{71} at the foot
of the Pyrenees, by a party of light cavalry, whose chief,
regardless of the sanctity of a temple, executed his commission
by the murder of the son of Constantine.\textsuperscript{72}

\pagenote[69]{Quarum (\textit{gentium}) obsides pretio quæsitos pueros
venustiore quod cultius habuerat libidine hujusmodi arsisse \textit{pro
certo} habet. Had not the depraved taste of Constans been
publicly avowed, the elder Victor, who held a considerable office
in his brother’s reign, would not have asserted it in such
positive terms.}

\pagenote[70]{Julian. Orat. i. and ii. Zosim. l. ii. p. 134.
Victor in Epitome. There is reason to believe that Magnentius was
born in one of those Barbarian colonies which Constantius Chlorus
had established in Gaul, (see this History, vol. i. p. 414.) His
behavior may remind us of the patriot earl of Leicester, the
famous Simon de Montfort, who could persuade the good people of
England, that he, a Frenchman by birth had taken arms to deliver
them from foreign favorites.}

\pagenote[71]{This ancient city had once flourished under the
name of Illiberis (Pomponius Mela, ii. 5.) The munificence of
Constantine gave it new splendor, and his mother’s name. Helena
(it is still called Elne) became the seat of a bishop, who long
afterwards transferred his residence to Perpignan, the capital of
modern Rousillon. See D’Anville. Notice de l’Ancienne Gaule, p.
380. Longuerue, Description de la France, p. 223, and the Marca
Hispanica, l. i. c. 2.}

\pagenote[72]{Zosimus, l. ii. p. 119, 120. Zonaras, tom. ii. l.
xiii. p. 13, and the Abbreviators.}

As soon as the death of Constans had decided this easy but
important revolution, the example of the court of Autun was
imitated by the provinces of the West. The authority of
Magnentius was acknowledged through the whole extent of the two
great præfectures of Gaul and Italy; and the usurper prepared, by
every act of oppression, to collect a treasure, which might
discharge the obligation of an immense donative, and supply the
expenses of a civil war. The martial countries of Illyricum, from
the Danube to the extremity of Greece, had long obeyed the
government of Vetranio, an aged general, beloved for the
simplicity of his manners, and who had acquired some reputation
by his experience and services in war.\textsuperscript{73} Attached by habit, by
duty, and by gratitude, to the house of Constantine, he
immediately gave the strongest assurances to the only surviving
son of his late master, that he would expose, with unshaken
fidelity, his person and his troops, to inflict a just revenge on
the traitors of Gaul. But the legions of Vetranio were seduced,
rather than provoked, by the example of rebellion; their leader
soon betrayed a want of firmness, or a want of sincerity; and his
ambition derived a specious pretence from the approbation of the
princess Constantina. That cruel and aspiring woman, who had
obtained from the great Constantine, her father, the rank of
\textit{Augusta}, placed the diadem with her own hands on the head of
the Illyrian general; and seemed to expect from his victory the
accomplishment of those unbounded hopes, of which she had been
disappointed by the death of her husband Hannibalianus. Perhaps
it was without the consent of Constantina, that the new emperor
formed a necessary, though dishonorable, alliance with the
usurper of the West, whose purple was so recently stained with
her brother’s blood.\textsuperscript{74}

\pagenote[73]{Eutropius (x. 10) describes Vetranio with more
temper, and probably with more truth, than either of the two
Victors. Vetranio was born of obscure parents in the wildest
parts of Mæsia; and so much had his education been neglected,
that, after his elevation, he studied the alphabet.}

\pagenote[74]{The doubtful, fluctuating conduct of Vetranio is
described by Julian in his first oration, and accurately
explained by Spanheim, who discusses the situation and behavior
of Constantina.}

The intelligence of these important events, which so deeply
affected the honor and safety of the Imperial house, recalled the
arms of Constantius from the inglorious prosecution of the
Persian war. He recommended the care of the East to his
lieutenants, and afterwards to his cousin Gallus, whom he raised
from a prison to a throne; and marched towards Europe, with a
mind agitated by the conflict of hope and fear, of grief and
indignation. On his arrival at Heraclea in Thrace, the emperor
gave audience to the ambassadors of Magnentius and Vetranio. The
first author of the conspiracy Marcellinus, who in some measure
had bestowed the purple on his new master, boldly accepted this
dangerous commission; and his three colleagues were selected from
the illustrious personages of the state and army. These deputies
were instructed to soothe the resentment, and to alarm the fears,
of Constantius. They were empowered to offer him the friendship
and alliance of the western princes, to cement their union by a
double marriage; of Constantius with the daughter of Magnentius,
and of Magnentius himself with the ambitious Constantina; and to
acknowledge in the treaty the preëminence of rank, which might
justly be claimed by the emperor of the East. Should pride and
mistaken piety urge him to refuse these equitable conditions, the
ambassadors were ordered to expatiate on the inevitable ruin
which must attend his rashness, if he ventured to provoke the
sovereigns of the West to exert their superior strength; and to
employ against him that valor, those abilities, and those
legions, to which the house of Constantine had been indebted for
so many triumphs. Such propositions and such arguments appeared
to deserve the most serious attention; the answer of Constantius
was deferred till the next day; and as he had reflected on the
importance of justifying a civil war in the opinion of the
people, he thus addressed his council, who listened with real or
affected credulity: “Last night,” said he, “after I retired to
rest, the shade of the great Constantine, embracing the corpse of
my murdered brother, rose before my eyes; his well-known voice
awakened me to revenge, forbade me to despair of the republic,
and assured me of the success and immortal glory which would
crown the justice of my arms.” The authority of such a vision, or
rather of the prince who alleged it, silenced every doubt, and
excluded all negotiation. The ignominious terms of peace were
rejected with disdain. One of the ambassadors of the tyrant was
dismissed with the haughty answer of Constantius; his colleagues,
as unworthy of the privileges of the law of nations, were put in
irons; and the contending powers prepared to wage an implacable
war.\textsuperscript{75}

\pagenote[75]{See Peter the Patrician, in the Excerpta Legationem
p. 27.}

Such was the conduct, and such perhaps was the duty, of the
brother of Constans towards the perfidious usurper of Gaul. The
situation and character of Vetranio admitted of milder measures;
and the policy of the Eastern emperor was directed to disunite
his antagonists, and to separate the forces of Illyricum from the
cause of rebellion. It was an easy task to deceive the frankness
and simplicity of Vetranio, who, fluctuating some time between
the opposite views of honor and interest, displayed to the world
the insincerity of his temper, and was insensibly engaged in the
snares of an artful negotiation. Constantius acknowledged him as
a legitimate and equal colleague in the empire, on condition that
he would renounce his disgraceful alliance with Magnentius, and
appoint a place of interview on the frontiers of their respective
provinces; where they might pledge their friendship by mutual
vows of fidelity, and regulate by common consent the future
operations of the civil war. In consequence of this agreement,
Vetranio advanced to the city of Sardica,\textsuperscript{76} at the head of
twenty thousand horse, and of a more numerous body of infantry; a
power so far superior to the forces of Constantius, that the
Illyrian emperor appeared to command the life and fortunes of his
rival, who, depending on the success of his private negotiations,
had seduced the troops, and undermined the throne, of Vetranio.
The chiefs, who had secretly embraced the party of Constantius,
prepared in his favor a public spectacle, calculated to discover
and inflame the passions of the multitude.\textsuperscript{77} The united armies
were commanded to assemble in a large plain near the city. In the
centre, according to the rules of ancient discipline, a military
tribunal, or rather scaffold, was erected, from whence the
emperors were accustomed, on solemn and important occasions, to
harangue the troops. The well-ordered ranks of Romans and
Barbarians, with drawn swords, or with erected spears, the
squadrons of cavalry, and the cohorts of infantry, distinguished
by the variety of their arms and ensigns, formed an immense
circle round the tribunal; and the attentive silence which they
preserved was sometimes interrupted by loud bursts of clamor or
of applause. In the presence of this formidable assembly, the two
emperors were called upon to explain the situation of public
affairs: the precedency of rank was yielded to the royal birth of
Constantius; and though he was indifferently skilled in the arts
of rhetoric, he acquitted himself, under these difficult
circumstances, with firmness, dexterity, and eloquence. The first
part of his oration seemed to be pointed only against the tyrant
of Gaul; but while he tragically lamented the cruel murder of
Constans, he insinuated, that none, except a brother, could claim
a right to the succession of his brother. He displayed, with some
complacency, the glories of his Imperial race; and recalled to
the memory of the troops the valor, the triumphs, the liberality
of the great Constantine, to whose sons they had engaged their
allegiance by an oath of fidelity, which the ingratitude of his
most favored servants had tempted them to violate. The officers,
who surrounded the tribunal, and were instructed to act their
part in this extraordinary scene, confessed the irresistible
power of reason and eloquence, by saluting the emperor
Constantius as their lawful sovereign. The contagion of loyalty
and repentance was communicated from rank to rank; till the plain
of Sardica resounded with the universal acclamation of “Away with
these upstart usurpers! Long life and victory to the son of
Constantine! Under his banners alone we will fight and conquer.”
The shout of thousands, their menacing gestures, the fierce
clashing of their arms, astonished and subdued the courage of
Vetranio, who stood, amidst the defection of his followers, in
anxious and silent suspense. Instead of embracing the last refuge
of generous despair, he tamely submitted to his fate; and taking
the diadem from his head, in the view of both armies fell
prostrate at the feet of his conqueror. Constantius used his
victory with prudence and moderation; and raising from the ground
the aged suppliant, whom he affected to style by the endearing
name of Father, he gave him his hand to descend from the throne.
The city of Prusa was assigned for the exile or retirement of the
abdicated monarch, who lived six years in the enjoyment of ease
and affluence. He often expressed his grateful sense of the
goodness of Constantius, and, with a very amiable simplicity,
advised his benefactor to resign the sceptre of the world, and to
seek for content (where alone it could be found) in the peaceful
obscurity of a private condition.\textsuperscript{78}

\pagenote[76]{Zonaras, tom. ii. l. xiii. p. 16. The position of
Sardica, near the modern city of Sophia, appears better suited to
this interview than the situation of either Naissus or Sirmium,
where it is placed by Jerom, Socrates, and Sozomen.}

\pagenote[77]{See the two first orations of Julian, particularly
p. 31; and Zosimus, l. ii. p. 122. The distinct narrative of the
historian serves to illustrate the diffuse but vague descriptions
of the orator.}

\pagenote[78]{The younger Victor assigns to his exile the
emphatical appellation of “Voluptarium otium.” Socrates (l. ii.
c. 28) is the voucher for the correspondence with the emperor,
which would seem to prove that Vetranio was indeed, prope ad
stultitiam simplicissimus.}

The behavior of Constantius on this memorable occasion was
celebrated with some appearance of justice; and his courtiers
compared the studied orations which a Pericles or a Demosthenes
addressed to the populace of Athens, with the victorious
eloquence which had persuaded an armed multitude to desert and
depose the object of their partial choice.\textsuperscript{79} The approaching
contest with Magnentius was of a more serious and bloody kind.
The tyrant advanced by rapid marches to encounter Constantius, at
the head of a numerous army, composed of Gauls and Spaniards, of
Franks and Saxons; of those provincials who supplied the strength
of the legions, and of those barbarians who were dreaded as the
most formidable enemies of the republic. The fertile plains\textsuperscript{80} of
the Lower Pannonia, between the Drave, the Save, and the Danube,
presented a spacious theatre; and the operations of the civil war
were protracted during the summer months by the skill or timidity
of the combatants.\textsuperscript{81} Constantius had declared his intention of
deciding the quarrel in the fields of Cibalis, a name that would
animate his troops by the remembrance of the victory, which, on
the same auspicious ground, had been obtained by the arms of his
father Constantine. Yet by the impregnable fortifications with
which the emperor encompassed his camp, he appeared to decline,
rather than to invite, a general engagement.

It was the object of Magnentius to tempt or to compel his
adversary to relinquish this advantageous position; and he
employed, with that view, the various marches, evolutions, and
stratagems, which the knowledge of the art of war could suggest
to an experienced officer. He carried by assault the important
town of Siscia; made an attack on the city of Sirmium, which lay
in the rear of the Imperial camp, attempted to force a passage
over the Save into the eastern provinces of Illyricum; and cut in
pieces a numerous detachment, which he had allured into the
narrow passes of Adarne. During the greater part of the summer,
the tyrant of Gaul showed himself master of the field. The troops
of Constantius were harassed and dispirited; his reputation
declined in the eye of the world; and his pride condescended to
solicit a treaty of peace, which would have resigned to the
assassin of Constans the sovereignty of the provinces beyond the
Alps. These offers were enforced by the eloquence of Philip the
Imperial ambassador; and the council as well as the army of
Magnentius were disposed to accept them. But the haughty usurper,
careless of the remonstrances of his friends, gave orders that
Philip should be detained as a captive, or, at least, as a
hostage; while he despatched an officer to reproach Constantius
with the weakness of his reign, and to insult him by the promise
of a pardon if he would instantly abdicate the purple. “That he
should confide in the justice of his cause, and the protection of
an avenging Deity,” was the only answer which honor permitted the
emperor to return. But he was so sensible of the difficulties of
his situation, that he no longer dared to retaliate the indignity
which had been offered to his representative. The negotiation of
Philip was not, however, ineffectual, since he determined
Sylvanus the Frank, a general of merit and reputation, to desert
with a considerable body of cavalry, a few days before the battle
of Mursa.

\pagenote[79]{Eum Constantius..... facundiæ vi dejectum Imperio
in pri vatum otium removit. Quæ gloria post natum Imperium soli
proces sit eloquio clementiâque, \&c. Aurelius Victor, Julian, and
Themistius (Orat. iii. and iv.) adorn this exploit with all the
artificial and gaudy coloring of their rhetoric.}

\pagenote[80]{Busbequius (p. 112) traversed the Lower Hungary and
Sclavonia at a time when they were reduced almost to a desert, by
the reciprocal hostilities of the Turks and Christians. Yet he
mentions with admiration the unconquerable fertility of the soil;
and observes that the height of the grass was sufficient to
conceal a loaded wagon from his sight. See likewise Browne’s
Travels, in Harris’s Collection, vol ii. p. 762 \&c.}

\pagenote[81]{Zosimus gives a very large account of the war, and
the negotiation, (l. ii. p. 123-130.) But as he neither shows
himself a soldier nor a politician, his narrative must be weighed
with attention, and received with caution.}

The city of Mursa, or Essek, celebrated in modern times for a
bridge of boats, five miles in length, over the River Drave, and
the adjacent morasses,\textsuperscript{82} has been always considered as a place
of importance in the wars of Hungary. Magnentius, directing his
march towards Mursa, set fire to the gates, and, by a sudden
assault, had almost scaled the walls of the town. The vigilance
of the garrison extinguished the flames; the approach of
Constantius left him no time to continue the operations of the
siege; and the emperor soon removed the only obstacle that could
embarrass his motions, by forcing a body of troops which had
taken post in an adjoining amphitheatre. The field of battle
round Mursa was a naked and level plain: on this ground the army
of Constantius formed, with the Drave on their right; while their
left, either from the nature of their disposition, or from the
superiority of their cavalry, extended far beyond the right flank
of Magnentius.\textsuperscript{83} The troops on both sides remained under arms,
in anxious expectation, during the greatest part of the morning;
and the son of Constantine, after animating his soldiers by an
eloquent speech, retired into a church at some distance from the
field of battle, and committed to his generals the conduct of
this decisive day.\textsuperscript{84} They deserved his confidence by the valor
and military skill which they exerted. They wisely began the
action upon the left; and advancing their whole wing of cavalry
in an oblique line, they suddenly wheeled it on the right flank
of the enemy, which was unprepared to resist the impetuosity of
their charge. But the Romans of the West soon rallied, by the
habits of discipline; and the Barbarians of Germany supported the
renown of their national bravery. The engagement soon became
general; was maintained with various and singular turns of
fortune; and scarcely ended with the darkness of the night. The
signal victory which Constantius obtained is attributed to the
arms of his cavalry. His cuirassiers are described as so many
massy statues of steel, glittering with their scaly armor, and
breaking with their ponderous lances the firm array of the Gallic
legions. As soon as the legions gave way, the lighter and more
active squadrons of the second line rode sword in hand into the
intervals, and completed the disorder. In the mean while, the
huge bodies of the Germans were exposed almost naked to the
dexterity of the Oriental archers; and whole troops of those
Barbarians were urged by anguish and despair to precipitate
themselves into the broad and rapid stream of the Drave.\textsuperscript{85} The
number of the slain was computed at fifty-four thousand men, and
the slaughter of the conquerors was more considerable than that
of the vanquished;\textsuperscript{86} a circumstance which proves the obstinacy
of the contest, and justifies the observation of an ancient
writer, that the forces of the empire were consumed in the fatal
battle of Mursa, by the loss of a veteran army, sufficient to
defend the frontiers, or to add new triumphs to the glory of
Rome.\textsuperscript{87} Notwithstanding the invectives of a servile orator,
there is not the least reason to believe that the tyrant deserted
his own standard in the beginning of the engagement. He seems to
have displayed the virtues of a general and of a soldier till the
day was irrecoverably lost, and his camp in the possession of the
enemy. Magnentius then consulted his safety, and throwing away
the Imperial ornaments, escaped with some difficulty from the
pursuit of the light horse, who incessantly followed his rapid
flight from the banks of the Drave to the foot of the Julian
Alps.\textsuperscript{88}

\pagenote[82]{This remarkable bridge, which is flanked with
towers, and supported on large wooden piles, was constructed A.
D. 1566, by Sultan Soliman, to facilitate the march of his armies
into Hungary.}

\pagenote[83]{This position, and the subsequent evolutions, are
clearly, though concisely, described by Julian, Orat. i. p. 36.}

\pagenote[84]{Sulpicius Severus, l. ii. p. 405. The emperor
passed the day in prayer with Valens, the Arian bishop of Mursa,
who gained his confidence by announcing the success of the
battle. M. de Tillemont (Hist. des Empereurs, tom. iv. p. 1110)
very properly remarks the silence of Julian with regard to the
personal prowess of Constantius in the battle of Mursa. The
silence of flattery is sometimes equal to the most positive and
authentic evidence.}

\pagenote[85]{Julian. Orat. i. p. 36, 37; and Orat. ii. p. 59,
60. Zonaras, tom ii. l. xiii. p. 17. Zosimus, l. ii. p. 130-133.
The last of these celebrates the dexterity of the archer
Menelaus, who could discharge three arrows at the same time; an
advantage which, according to his apprehension of military
affairs, materially contributed to the victory of Constantius.}

\pagenote[86]{According to Zonaras, Constantius, out of 80,000
men, lost 30,000; and Magnentius lost 24,000 out of 36,000. The
other articles of this account seem probable and authentic, but
the numbers of the tyrant’s army must have been mistaken, either
by the author or his transcribers. Magnentius had collected the
whole force of the West, Romans and Barbarians, into one
formidable body, which cannot fairly be estimated at less than
100,000 men. Julian. Orat. i. p. 34, 35.}

\pagenote[87]{Ingentes R. I. vires eâ dimicatione consumptæ sunt,
ad quælibet bella externa idoneæ, quæ multum triumphorum possent
securitatisque conferre. Eutropius, x. 13. The younger Victor
expresses himself to the same effect.}

\pagenote[88]{On this occasion, we must prefer the unsuspected
testimony of Zosimus and Zonaras to the flattering assertions of
Julian. The younger Victor paints the character of Magnentius in
a singular light: “Sermonis acer, animi tumidi, et immodice
timidus; artifex tamen ad occultandam audaciæ specie formidinem.”
Is it most likely that in the battle of Mursa his behavior was
governed by nature or by art should incline for the latter.}

The approach of winter supplied the indolence of Constantius with
specious reasons for deferring the prosecution of the war till
the ensuing spring. Magnentius had fixed his residence in the
city of Aquileia, and showed a seeming resolution to dispute the
passage of the mountains and morasses which fortified the
confines of the Venetian province. The surprisal of a castle in
the Alps by the secret march of the Imperialists, could scarcely
have determined him to relinquish the possession of Italy, if the
inclinations of the people had supported the cause of their
tyrant.\textsuperscript{89} But the memory of the cruelties exercised by his
ministers, after the unsuccessful revolt of Nepotian, had left a
deep impression of horror and resentment on the minds of the
Romans. That rash youth, the son of the princess Eutropia, and
the nephew of Constantine, had seen with indignation the sceptre
of the West usurped by a perfidious barbarian. Arming a desperate
troop of slaves and gladiators, he overpowered the feeble guard
of the domestic tranquillity of Rome, received the homage of the
senate, and assuming the title of Augustus, precariously reigned
during a tumult of twenty-eight days. The march of some regular
forces put an end to his ambitious hopes: the rebellion was
extinguished in the blood of Nepotian, of his mother Eutropia,
and of his adherents; and the proscription was extended to all
who had contracted a fatal alliance with the name and family of
Constantine.\textsuperscript{90} But as soon as Constantius, after the battle of
Mursa, became master of the sea-coast of Dalmatia, a band of
noble exiles, who had ventured to equip a fleet in some harbor of
the Adriatic, sought protection and revenge in his victorious
camp. By their secret intelligence with their countrymen, Rome
and the Italian cities were persuaded to display the banners of
Constantius on their walls. The grateful veterans, enriched by
the liberality of the father, signalized their gratitude and
loyalty to the son. The cavalry, the legions, and the auxiliaries
of Italy, renewed their oath of allegiance to Constantius; and
the usurper, alarmed by the general desertion, was compelled,
with the remains of his faithful troops, to retire beyond the
Alps into the provinces of Gaul. The detachments, however, which
were ordered either to press or to intercept the flight of
Magnentius, conducted themselves with the usual imprudence of
success; and allowed him, in the plains of Pavia, an opportunity
of turning on his pursuers, and of gratifying his despair by the
carnage of a useless victory.\textsuperscript{91}

\pagenote[89]{Julian. Orat. i. p. 38, 39. In that place, however,
as well as in Oration ii. p. 97, he insinuates the general
disposition of the senate, the people, and the soldiers of Italy,
towards the party of the emperor.}

\pagenote[90]{The elder Victor describes, in a pathetic manner,
the miserable condition of Rome: “Cujus stolidum ingenium adeo P.
R. patribusque exitio fuit, uti passim domus, fora, viæ,
templaque, cruore, cadaveri busque opplerentur bustorum modo.”
Athanasius (tom. i. p. 677) deplores the fate of several
illustrious victims, and Julian (Orat. ii p 58) execrates the
cruelty of Marcellinus, the implacable enemy of the house of
Constantine.}

\pagenote[91]{Zosim. l. ii. p. 133. Victor in Epitome. The
panegyrists of Constantius, with their usual candor, forget to
mention this accidental defeat.}

The pride of Magnentius was reduced, by repeated misfortunes, to
sue, and to sue in vain, for peace. He first despatched a
senator, in whose abilities he confided, and afterwards several
bishops, whose holy character might obtain a more favorable
audience, with the offer of resigning the purple, and the promise
of devoting the remainder of his life to the service of the
emperor. But Constantius, though he granted fair terms of pardon
and reconciliation to all who abandoned the standard of
rebellion,\textsuperscript{92} avowed his inflexible resolution to inflict a just
punishment on the crimes of an assassin, whom he prepared to
overwhelm on every side by the effort of his victorious arms. An
Imperial fleet acquired the easy possession of Africa and Spain,
confirmed the wavering faith of the Moorish nations, and landed a
considerable force, which passed the Pyrenees, and advanced
towards Lyons, the last and fatal station of Magnentius.\textsuperscript{93} The
temper of the tyrant, which was never inclined to clemency, was
urged by distress to exercise every act of oppression which could
extort an immediate supply from the cities of Gaul.\textsuperscript{94} Their
patience was at length exhausted; and Treves, the seat of
Prætorian government, gave the signal of revolt, by shutting her
gates against Decentius, who had been raised by his brother to
the rank either of Cæsar or of Augustus.\textsuperscript{95} From Treves,
Decentius was obliged to retire to Sens, where he was soon
surrounded by an army of Germans, whom the pernicious arts of
Constantius had introduced into the civil dissensions of Rome.\textsuperscript{96}
In the mean time, the Imperial troops forced the passages of the
Cottian Alps, and in the bloody combat of Mount Seleucus
irrevocably fixed the title of rebels on the party of Magnentius.\textsuperscript{97}
He was unable to bring another army into the field; the
fidelity of his guards was corrupted; and when he appeared in
public to animate them by his exhortations, he was saluted with a
unanimous shout of “Long live the emperor Constantius!” The
tyrant, who perceived that they were preparing to deserve pardon
and rewards by the sacrifice of the most obnoxious criminal,
prevented their design by falling on his sword;\textsuperscript{98} a death more
easy and more honorable than he could hope to obtain from the
hands of an enemy, whose revenge would have been colored with the
specious pretence of justice and fraternal piety. The example of
suicide was imitated by Decentius, who strangled himself on the
news of his brother’s death. The author of the conspiracy,
Marcellinus, had long since disappeared in the battle of Mursa,\textsuperscript{99}
and the public tranquillity was confirmed by the execution of
the surviving leaders of a guilty and unsuccessful faction. A
severe inquisition was extended over all who, either from choice
or from compulsion, had been involved in the cause of rebellion.
Paul, surnamed Catena from his superior skill in the judicial
exercise of tyranny,\textsuperscript{9911} was sent to explore the latent remains
of the conspiracy in the remote province of Britain. The honest
indignation expressed by Martin, vice-præfect of the island, was
interpreted as an evidence of his own guilt; and the governor was
urged to the necessity of turning against his breast the sword
with which he had been provoked to wound the Imperial minister.
The most innocent subjects of the West were exposed to exile and
confiscation, to death and torture; and as the timid are always
cruel, the mind of Constantius was inaccessible to mercy.\textsuperscript{100}

\pagenote[92]{Zonaras, tom. ii. l. xiii. p. 17. Julian, in
several places of the two orations, expatiates on the clemency of
Constantius to the rebels.}

\pagenote[93]{Zosim. l. ii. p. 133. Julian. Orat. i. p. 40, ii.
p. 74.}

\pagenote[94]{Ammian. xv. 6. Zosim. l. ii. p. 123. Julian, who
(Orat. i. p. 40) unveighs against the cruel effects of the
tyrant’s despair, mentions (Orat. i. p. 34) the oppressive edicts
which were dictated by his necessities, or by his avarice. His
subjects were compelled to purchase the Imperial demesnes; a
doubtful and dangerous species of property, which, in case of a
revolution, might be imputed to them as a treasonable
usurpation.}

\pagenote[95]{The medals of Magnentius celebrate the victories of
the \textit{two} Augusti, and of the Cæsar. The Cæsar was another
brother, named Desiderius. See Tillemont, Hist. des Empereurs,
tom. iv. p. 757.}

\pagenote[96]{Julian. Orat. i. p. 40, ii. p. 74; with Spanheim,
p. 263. His Commentary illustrates the transactions of this civil
war. Mons Seleuci was a small place in the Cottian Alps, a few
miles distant from Vapincum, or Gap, an episcopal city of
Dauphine. See D’Anville, Notice de la Gaule, p. 464; and
Longuerue, Description de la France, p. 327.—— The Itinerary of
Antoninus (p. 357, ed. Wess.) places Mons Seleucu twenty-four
miles from Vapinicum, (Gap,) and twenty-six from Lucus. (le Luc,)
on the road to Die, (Dea Vocontiorum.) The situation answers to
Mont Saleon, a little place on the right of the small river
Buech, which falls into the Durance. Roman antiquities have been
found in this place. St. Martin. Note to Le Beau, ii. 47.—M.}

\pagenote[97]{Zosimus, l. ii. p. 134. Liban. Orat. x. p. 268,
269. The latter most vehemently arraigns this cruel and selfish
policy of Constantius.}

\pagenote[98]{Julian. Orat. i. p. 40. Zosimus, l. ii. p. 134.
Socrates, l. ii. c. 32. Sozomen, l. iv. c. 7. The younger Victor
describes his death with some horrid circumstances: Transfosso
latere, ut erat vasti corporis, vulnere naribusque et ore cruorem
effundens, exspiravit. If we can give credit to Zonaras, the
tyrant, before he expired, had the pleasure of murdering, with
his own hand, his mother and his brother Desiderius.}

\pagenote[99]{Julian (Orat. i. p. 58, 59) seems at a loss to
determine, whether he inflicted on himself the punishment of his
crimes, whether he was drowned in the Drave, or whether he was
carried by the avenging dæmons from the field of battle to his
destined place of eternal tortures.}

\pagenote[9911]{This is scarcely correct, ut erat in complicandis
negotiis artifex dirum made ei Catenæ inditum est cognomentum.
Amm. Mar. loc. cit.—M.}

\pagenote[100]{Ammian. xiv. 5, xxi. 16.}

