\chapter{Conversion Of Constantine.}
\section{Part \thesection.}

\textit{The Motives, Progress, And Effects Of The Conversion Of
Constantine. — Legal Establishment And Constitution Of The Christian Or
Catholic Church.}
\vspace{\onelineskip}

The public establishment of Christianity may be considered as one of
those important and domestic revolutions which excite the most lively
curiosity, and afford the most valuable instruction. The victories and
the civil policy of Constantine no longer influence the state of
Europe; but a considerable portion of the globe still retains the
impression which it received from the conversion of that monarch; and
the ecclesiastical institutions of his reign are still connected, by an
indissoluble chain, with the opinions, the passions, and the interests
of the present generation. In the consideration of a subject which may
be examined with impartiality, but cannot be viewed with indifference,
a difficulty immediately arises of a very unexpected nature; that of
ascertaining the real and precise date of the conversion of
Constantine. The eloquent Lactantius, in the midst of his court, seems
impatient\textsuperscript{1} to proclaim to the world the glorious example of the
sovereign of Gaul; who, in the first moments of his reign, acknowledged
and adored the majesty of the true and only God.\textsuperscript{2} The learned Eusebius
has ascribed the faith of Constantine to the miraculous sign which was
displayed in the heavens whilst he meditated and prepared the Italian
expedition.\textsuperscript{3} The historian Zosimus maliciously asserts, that the
emperor had imbrued his hands in the blood of his eldest son, before he
publicly renounced the gods of Rome and of his ancestors.\textsuperscript{4} The
perplexity produced by these discordant authorities is derived from the
behavior of Constantine himself. According to the strictness of
ecclesiastical language, the first of the \textit{Christian} emperors was
unworthy of that name, till the moment of his death; since it was only
during his last illness that he received, as a catechumen, the
imposition of hands,\textsuperscript{5} and was afterwards admitted, by the initiatory
rites of baptism, into the number of the faithful.\textsuperscript{6} The Christianity
of Constantine must be allowed in a much more vague and qualified
sense; and the nicest accuracy is required in tracing the slow and
almost imperceptible gradations by which the monarch declared himself
the protector, and at length the proselyte, of the church. It was an
arduous task to eradicate the habits and prejudices of his education,
to acknowledge the divine power of Christ, and to understand that the
truth of \textit{his} revelation was incompatible with the worship of the
gods. The obstacles which he had probably experienced in his own mind,
instructed him to proceed with caution in the momentous change of a
national religion; and he insensibly discovered his new opinions, as
far as he could enforce them with safety and with effect. During the
whole course of his reign, the stream of Christianity flowed with a
gentle, though accelerated, motion: but its general direction was
sometimes checked, and sometimes diverted, by the accidental
circumstances of the times, and by the prudence, or possibly by the
caprice, of the monarch. His ministers were permitted to signify the
intentions of their master in the various language which was best
adapted to their respective principles;\textsuperscript{7} and he artfully balanced the
hopes and fears of his subjects, by publishing in the same year two
edicts; the first of which enjoined the solemn observance of Sunday,\textsuperscript{8}
and the second directed the regular consultation of the Aruspices.\textsuperscript{9}
While this important revolution yet remained in suspense, the
Christians and the Pagans watched the conduct of their sovereign with
the same anxiety, but with very opposite sentiments. The former were
prompted by every motive of zeal, as well as vanity, to exaggerate the
marks of his favor, and the evidences of his faith. The latter, till
their just apprehensions were changed into despair and resentment,
attempted to conceal from the world, and from themselves, that the gods
of Rome could no longer reckon the emperor in the number of their
votaries. The same passions and prejudices have engaged the partial
writers of the times to connect the public profession of Christianity
with the most glorious or the most ignominious æra of the reign of
Constantine.

\pagenote[1]{The date of the Divine Institutions of Lactantius has been
accurately discussed, difficulties have been started, solutions
proposed, and an expedient imagined of two \textit{original} editions; the
former published during the persecution of Diocletian, the latter under
that of Licinius. See Dufresnoy, Prefat. p. v. Tillemont, Mém.
Ecclesiast. tom. vi. p. 465-470. Lardner’s Credibility, part ii. vol.
vii. p. 78-86. For my own part, I am \textit{almost} convinced that Lactantius
dedicated his Institutions to the sovereign of Gaul, at a time when
Galerius, Maximin, and even Licinius, persecuted the Christians; that
is, between the years 306 and 311.}

\pagenote[2]{Lactant. Divin. Instit. i. l. vii. 27. The first and most
important of these passages is indeed wanting in twenty-eight
manuscripts; but it is found in nineteen. If we weigh the comparative
value of these manuscripts, one of 900 years old, in the king of
France’s library may be alleged in its favor; but the passage is
omitted in the correct manuscript of Bologna, which the P. de
Montfaucon ascribes to the sixth or seventh century (Diarium Italic. p.
489.) The taste of most of the editors (except Isæus; see Lactant.
edit. Dufresnoy, tom. i. p. 596) has felt the genuine style of
Lactantius.}

\pagenote[3]{Euseb. in Vit. Constant. l. i. c. 27-32.}

\pagenote[4]{Zosimus, l. ii. p. 104.}

\pagenote[5]{That rite was \textit{always} used in making a catechumen, (see
Bingham’s Antiquities. l. x. c. i. p. 419. Dom Chardon, Hist. des
Sacramens, tom. i. p. 62,) and Constantine received it for the \textit{first}
time (Euseb. in Vit Constant. l. iv. c. 61) immediately before his
baptism and death. From the connection of these two facts, Valesius (ad
loc. Euseb.) has drawn the conclusion which is reluctantly admitted by
Tillemont, (Hist. des Empereurs, tom. iv. p. 628,) and opposed with
feeble arguments by Mosheim, (p. 968.)}

\pagenote[6]{Euseb. in Vit. Constant. l. iv. c. 61, 62, 63. The legend
of Constantine’s baptism at Rome, thirteen years before his death, was
invented in the eighth century, as a proper motive for his \textit{donation}.
Such has been the gradual progress of knowledge, that a story, of which
Cardinal Baronius (Annual Ecclesiast. A. D. 324, No. 43-49) declared
himself the unblushing advocate, is now feebly supported, even within
the verge of the Vatican. See the Antiquitates Christianæ, tom. ii. p.
232; a work published with six approbations at Rome, in the year 1751
by Father Mamachi, a learned Dominican.}

\pagenote[7]{The quæstor, or secretary, who composed the law of the
Theodosian Code, makes his master say with indifference, “hominibus
supradictæ religionis,” (l. xvi. tit. ii. leg. 1.) The minister of
ecclesiastical affairs was allowed a more devout and respectful style,
[**Greek] the legal, most holy, and Catholic worship.}

\pagenote[8]{Cod. Theodos. l. ii. viii. tit. leg. 1. Cod. Justinian. l.
iii. tit. xii. leg. 3. Constantine styles the Lord’s day \textit{dies solis},
a name which could not offend the ears of his pagan subjects.}

\pagenote[9]{Cod. Theodos. l. xvi. tit. x. leg. l. Godefroy, in the
character of a commentator, endeavors (tom. vi. p. 257) to excuse
Constantine; but the more zealous Baronius (Annal. Eccles. A. D. 321,
No. 17) censures his profane conduct with truth and asperity.}

Whatever symptoms of Christian piety might transpire in the discourses
or actions of Constantine, he persevered till he was near forty years
of age in the practice of the established religion;\textsuperscript{10} and the same
conduct which in the court of Nicomedia might be imputed to his fear,
could be ascribed only to the inclination or policy of the sovereign of
Gaul. His liberality restored and enriched the temples of the gods; the
medals which issued from his Imperial mint are impressed with the
figures and attributes of Jupiter and Apollo, of Mars and Hercules; and
his filial piety increased the council of Olympus by the solemn
apotheosis of his father Constantius.\textsuperscript{11} But the devotion of
Constantine was more peculiarly directed to the genius of the Sun, the
Apollo of Greek and Roman mythology; and he was pleased to be
represented with the symbols of the God of Light and Poetry. The
unerring shafts of that deity, the brightness of his eyes, his laurel
wreath, immortal beauty, and elegant accomplishments, seem to point him
out as the patron of a young hero. The altars of Apollo were crowned
with the votive offerings of Constantine; and the credulous multitude
were taught to believe, that the emperor was permitted to behold with
mortal eyes the visible majesty of their tutelar deity; and that,
either walking or in a vision, he was blessed with the auspicious omens
of a long and victorious reign. The Sun was universally celebrated as
the invincible guide and protector of Constantine; and the Pagans might
reasonably expect that the insulted god would pursue with unrelenting
vengeance the impiety of his ungrateful favorite.\textsuperscript{12}

\pagenote[10]{Theodoret. (l. i. c. 18) seems to insinuate that Helena
gave her son a Christian education; but we may be assured, from the
superior authority of Eusebius, (in Vit. Constant. l. iii. c. 47,) that
she herself was indebted to Constantine for the knowledge of
Christianity.}

\pagenote[11]{See the medals of Constantine in Ducange and Banduri. As
few cities had retained the privilege of coining, almost all the medals
of that age issued from the mint under the sanction of the Imperial
authority.}

\pagenote[12]{The panegyric of Eumenius, (vii. inter Panegyr. Vet.,)
which was pronounced a few months before the Italian war, abounds with
the most unexceptionable evidence of the Pagan superstition of
Constantine, and of his particular veneration for Apollo, or the Sun;
to which Julian alludes.}

As long as Constantine exercised a limited sovereignty over the
provinces of Gaul, his Christian subjects were protected by the
authority, and perhaps by the laws, of a prince, who wisely left to the
gods the care of vindicating their own honor. If we may credit the
assertion of Constantine himself, he had been an indignant spectator of
the savage cruelties which were inflicted, by the hands of Roman
soldiers, on those citizens whose religion was their only crime.\textsuperscript{13} In
the East and in the West, he had seen the different effects of severity
and indulgence; and as the former was rendered still more odious by the
example of Galerius, his implacable enemy, the latter was recommended
to his imitation by the authority and advice of a dying father. The son
of Constantius immediately suspended or repealed the edicts of
persecution, and granted the free exercise of their religious
ceremonies to all those who had already professed themselves members of
the church. They were soon encouraged to depend on the favor as well as
on the justice of their sovereign, who had imbibed a secret and sincere
reverence for the name of Christ, and for the God of the Christians.\textsuperscript{14}

\pagenote[13]{Constantin. Orat. ad Sanctos, c. 25. But it might easily
be shown, that the Greek translator has improved the sense of the Latin
original; and the aged emperor might recollect the persecution of
Diocletian with a more lively abhorrence than he had actually felt to
the days of his youth and Paganism.}

\pagenote[14]{See Euseb. Hist. Eccles. l. viii. 13, l. ix. 9, and in
Vit. Const. l. i. c. 16, 17 Lactant. Divin. Institut. i. l. Cæcilius de
Mort. Persecut. c. 25.}

About five months after the conquest of Italy, the emperor made a
solemn and authentic declaration of his sentiments by the celebrated
edict of Milan, which restored peace to the Catholic church. In the
personal interview of the two western princes, Constantine, by the
ascendant of genius and power, obtained the ready concurrence of his
colleague, Licinius; the union of their names and authority disarmed
the fury of Maximin; and after the death of the tyrant of the East, the
edict of Milan was received as a general and fundamental law of the
Roman world.\textsuperscript{15}

\pagenote[15]{Cæcilius (de Mort. Persecut. c. 48) has preserved the
Latin original; and Eusebius (Hist. Eccles. l. x. c. 5) has given a
Greek translation of this perpetual edict, which refers to some
provisional regulations.}

The wisdom of the emperors provided for the restitution of all the
civil and religious rights of which the Christians had been so unjustly
deprived. It was enacted that the places of worship, and public lands,
which had been confiscated, should be restored to the church, without
dispute, without delay, and without expense; and this severe injunction
was accompanied with a gracious promise, that if any of the purchasers
had paid a fair and adequate price, they should be indemnified from the
Imperial treasury. The salutary regulations which guard the future
tranquillity of the faithful are framed on the principles of enlarged
and equal toleration; and such an equality must have been interpreted
by a recent sect as an advantageous and honorable distinction. The two
emperors proclaim to the world, that they have granted a free and
absolute power to the Christians, and to all others, of following the
religion which each individual thinks proper to prefer, to which he has
addicted his mind, and which he may deem the best adapted to his own
use. They carefully explain every ambiguous word, remove every
exception, and exact from the governors of the provinces a strict
obedience to the true and simple meaning of an edict, which was
designed to establish and secure, without any limitation, the claims of
religious liberty. They condescend to assign two weighty reasons which
have induced them to allow this universal toleration: the humane
intention of consulting the peace and happiness of their people; and
the pious hope, that, by such a conduct, they shall appease and
propitiate \textit{the Deity}, whose seat is in heaven. They gratefully
acknowledge the many signal proofs which they have received of the
divine favor; and they trust that the same Providence will forever
continue to protect the prosperity of the prince and people. From these
vague and indefinite expressions of piety, three suppositions may be
deduced, of a different, but not of an incompatible nature. The mind of
Constantine might fluctuate between the Pagan and the Christian
religions. According to the loose and complying notions of Polytheism,
he might acknowledge the God of the Christians as \textit{one} of the \textit{many}
deities who compose the hierarchy of heaven. Or perhaps he might
embrace the philosophic and pleasing idea, that, notwithstanding the
variety of names, of rites, and of opinions, all the sects, and all the
nations of mankind, are united in the worship of the common Father and
Creator of the universe.\textsuperscript{16}

\pagenote[16]{A panegyric of Constantine, pronounced seven or eight
months after the edict of Milan, (see Gothofred. Chronolog. Legum, p.
7, and Tillemont, Hist. des Empereurs, tom. iv. p. 246,) uses the
following remarkable expression: “Summe rerum sator, cujus tot nomina
sant, quot linguas gentium esse voluisti, quem enim te ipse dici velin,
scire non possumus.” (Panegyr. Vet. ix. 26.) In explaining
Constantine’s progress in the faith, Mosheim (p. 971, \&c.) is
ingenious, subtle, prolix.}

But the counsels of princes are more frequently influenced by views of
temporal advantage, than by considerations of abstract and speculative
truth. The partial and increasing favor of Constantine may naturally be
referred to the esteem which he entertained for the moral character of
the Christians; and to a persuasion, that the propagation of the gospel
would inculcate the practice of private and public virtue. Whatever
latitude an absolute monarch may assume in his own conduct, whatever
indulgence he may claim for his own passions, it is undoubtedly his
interest that all his subjects should respect the natural and civil
obligations of society. But the operation of the wisest laws is
imperfect and precarious. They seldom inspire virtue, they cannot
always restrain vice. Their power is insufficient to prohibit all that
they condemn, nor can they always punish the actions which they
prohibit. The legislators of antiquity had summoned to their aid the
powers of education and of opinion. But every principle which had once
maintained the vigor and purity of Rome and Sparta, was long since
extinguished in a declining and despotic empire. Philosophy still
exercised her temperate sway over the human mind, but the cause of
virtue derived very feeble support from the influence of the Pagan
superstition. Under these discouraging circumstances, a prudent
magistrate might observe with pleasure the progress of a religion which
diffused among the people a pure, benevolent, and universal system of
ethics, adapted to every duty and every condition of life; recommended
as the will and reason of the supreme Deity, and enforced by the
sanction of eternal rewards or punishments. The experience of Greek and
Roman history could not inform the world how far the system of national
manners might be reformed and improved by the precepts of a divine
revelation; and Constantine might listen with some confidence to the
flattering, and indeed reasonable, assurances of Lactantius. The
eloquent apologist seemed firmly to expect, and almost ventured to
promise, \textit{that} the establishment of Christianity would restore the
innocence and felicity of the primitive age; \textit{that} the worship of the
true God would extinguish war and dissension among those who mutually
considered themselves as the children of a common parent; \textit{that} every
impure desire, every angry or selfish passion, would be restrained by
the knowledge of the gospel; and \textit{that} the magistrates might sheath
the sword of justice among a people who would be universally actuated
by the sentiments of truth and piety, of equity and moderation, of
harmony and universal love.\textsuperscript{17}

\pagenote[17]{See the elegant description of Lactantius, (Divin
Institut. v. 8,) who is much more perspicuous and positive than becomes
a discreet prophet.}

The passive and unresisting obedience, which bows under the yoke of
authority, or even of oppression, must have appeared, in the eyes of an
absolute monarch, the most conspicuous and useful of the evangelic
virtues.\textsuperscript{18} The primitive Christians derived the institution of civil
government, not from the consent of the people, but from the decrees of
Heaven. The reigning emperor, though he had usurped the sceptre by
treason and murder, immediately assumed the sacred character of
vicegerent of the Deity. To the Deity alone he was accountable for the
abuse of his power; and his subjects were indissolubly bound, by their
oath of fidelity, to a tyrant, who had violated every law of nature and
society. The humble Christians were sent into the world as sheep among
wolves; and since they were not permitted to employ force even in the
defence of their religion, they should be still more criminal if they
were tempted to shed the blood of their fellow-creatures in disputing
the vain privileges, or the sordid possessions, of this transitory
life. Faithful to the doctrine of the apostle, who in the reign of Nero
had preached the duty of unconditional submission, the Christians of
the three first centuries preserved their conscience pure and innocent
of the guilt of secret conspiracy, or open rebellion. While they
experienced the rigor of persecution, they were never provoked either
to meet their tyrants in the field, or indignantly to withdraw
themselves into some remote and sequestered corner of the globe.\textsuperscript{19} The
Protestants of France, of Germany, and of Britain, who asserted with
such intrepid courage their civil and religious freedom, have been
insulted by the invidious comparison between the conduct of the
primitive and of the reformed Christians.\textsuperscript{20} Perhaps, instead of
censure, some applause may be due to the superior sense and spirit of
our ancestors, who had convinced themselves that religion cannot
abolish the unalienable rights of human nature.\textsuperscript{21} Perhaps the patience
of the primitive church may be ascribed to its weakness, as well as to
its virtue.

A sect of unwarlike plebeians, without leaders, without arms, without
fortifications, must have encountered inevitable destruction in a rash
and fruitless resistance to the master of the Roman legions. But the
Christians, when they deprecated the wrath of Diocletian, or solicited
the favor of Constantine, could allege, with truth and confidence, that
they held the principle of passive obedience, and that, in the space of
three centuries, their conduct had always been conformable to their
principles. They might add, that the throne of the emperors would be
established on a fixed and permanent basis, if all their subjects,
embracing the Christian doctrine, should learn to suffer and to obey.

\pagenote[18]{The political system of the Christians is explained by
Grotius, de Jure Belli et Pacis, l. i. c. 3, 4. Grotius was a
republican and an exile, but the mildness of his temper inclined him to
support the established powers.}

\pagenote[19]{Tertullian. Apolog. c. 32, 34, 35, 36. Tamen nunquam
Albiniani, nec Nigriani vel Cassiani inveniri potuerunt Christiani. Ad
Scapulam, c. 2. If this assertion be strictly true, it excludes the
Christians of that age from all civil and military employments, which
would have compelled them to take an active part in the service of
their respective governors. See Moyle’s Works, vol. ii. p. 349.}

\pagenote[20]{See the artful Bossuet, (Hist. des Variations des Eglises
Protestantes, tom. iii. p. 210-258.) and the malicious Bayle, (tom ii.
p. 820.) I \textit{name} Bayle, for he was certainly the author of the Avis
aux Refugies; consult the Dictionnaire Critique de Chauffepié, tom. i.
part ii. p. 145.}

\pagenote[21]{Buchanan is the earliest, or at least the most
celebrated, of the reformers, who has justified the theory of
resistance. See his Dialogue de Jure Regni apud Scotos, tom. ii. p. 28,
30, edit. fol. Rudiman.}

In the general order of Providence, princes and tyrants are considered
as the ministers of Heaven, appointed to rule or to chastise the
nations of the earth. But sacred history affords many illustrious
examples of the more immediate interposition of the Deity in the
government of his chosen people. The sceptre and the sword were
committed to the hands of Moses, of Joshua, of Gideon, of David, of the
Maccabees; the virtues of those heroes were the motive or the effect of
the divine favor, the success of their arms was destined to achieve the
deliverance or the triumph of the church. If the judges of Israel were
occasional and temporary magistrates, the kings of Judah derived from
the royal unction of their great ancestor an hereditary and
indefeasible right, which could not be forfeited by their own vices,
nor recalled by the caprice of their subjects. The same extraordinary
providence, which was no longer confined to the Jewish people, might
elect Constantine and his family as the protectors of the Christian
world; and the devout Lactantius announces, in a prophetic tone, the
future glories of his long and universal reign.\textsuperscript{22} Galerius and
Maximin, Maxentius and Licinius, were the rivals who shared with the
favorite of heaven the provinces of the empire. The tragic deaths of
Galerius and Maximin soon gratified the resentment, and fulfilled the
sanguine expectations, of the Christians. The success of Constantine
against Maxentius and Licinius removed the two formidable competitors
who still opposed the triumph of the second David, and his cause might
seem to claim the peculiar interposition of Providence. The character
of the Roman tyrant disgraced the purple and human nature; and though
the Christians might enjoy his precarious favor, they were exposed,
with the rest of his subjects, to the effects of his wanton and
capricious cruelty. The conduct of Licinius soon betrayed the
reluctance with which he had consented to the wise and humane
regulations of the edict of Milan. The convocation of provincial synods
was prohibited in his dominions; his Christian officers were
ignominiously dismissed; and if he avoided the guilt, or rather danger,
of a general persecution, his partial oppressions were rendered still
more odious by the violation of a solemn and voluntary engagement.\textsuperscript{23}
While the East, according to the lively expression of Eusebius, was
involved in the shades of infernal darkness, the auspicious rays of
celestial light warmed and illuminated the provinces of the West. The
piety of Constantine was admitted as an unexceptionable proof of the
justice of his arms; and his use of victory confirmed the opinion of
the Christians, that their hero was inspired, and conducted, by the
Lord of Hosts. The conquest of Italy produced a general edict of
toleration; and as soon as the defeat of Licinius had invested
Constantine with the sole dominion of the Roman world, he immediately,
by circular letters, exhorted all his subjects to imitate, without
delay, the example of their sovereign, and to embrace the divine truth
of Christianity.\textsuperscript{24}

\pagenote[22]{Lactant Divin. Institut. i. l. Eusebius in the course of
his history, his life, and his oration, repeatedly inculcates the
divine right of Constantine to the empire.}

\pagenote[23]{Our imperfect knowledge of the persecution of Licinius is
derived from Eusebius, (Hist. l. x. c. 8. Vit. Constantin. l. i. c.
49-56, l. ii. c. 1, 2.) Aurelius Victor mentions his cruelty in general
terms.}

\pagenote[24]{Euseb. in Vit. Constant. l. ii. c. 24-42 48-60.}

\section{Part \thesection.}

The assurance that the elevation of Constantine was intimately
connected with the designs of Providence, instilled into the minds of
the Christians two opinions, which, by very different means, assisted
the accomplishment of the prophecy. Their warm and active loyalty
exhausted in his favor every resource of human industry; and they
confidently expected that their strenuous efforts would be seconded by
some divine and miraculous aid. The enemies of Constantine have imputed
to interested motives the alliance which he insensibly contracted with
the Catholic church, and which apparently contributed to the success of
his ambition. In the beginning of the fourth century, the Christians
still bore a very inadequate proportion to the inhabitants of the
empire; but among a degenerate people, who viewed the change of masters
with the indifference of slaves, the spirit and union of a religious
party might assist the popular leader, to whose service, from a
principle of conscience, they had devoted their lives and fortunes.\textsuperscript{25}
The example of his father had instructed Constantine to esteem and to
reward the merit of the Christians; and in the distribution of public
offices, he had the advantage of strengthening his government, by the
choice of ministers or generals, in whose fidelity he could repose a
just and unreserved confidence. By the influence of these dignified
missionaries, the proselytes of the new faith must have multiplied in
the court and army; the Barbarians of Germany, who filled the ranks of
the legions, were of a careless temper, which acquiesced without
resistance in the religion of their commander; and when they passed the
Alps, it may fairly be presumed, that a great number of the soldiers
had already consecrated their swords to the service of Christ and of
Constantine.\textsuperscript{26} The habits of mankind and the interests of religion
gradually abated the horror of war and bloodshed, which had so long
prevailed among the Christians; and in the councils which were
assembled under the gracious protection of Constantine, the authority
of the bishops was seasonably employed to ratify the obligation of the
military oath, and to inflict the penalty of excommunication on those
soldiers who threw away their arms during the peace of the church.\textsuperscript{27}
While Constantine, in his own dominions, increased the number and zeal
of his faithful adherents, he could depend on the support of a powerful
faction in those provinces which were still possessed or usurped by his
rivals. A secret disaffection was diffused among the Christian subjects
of Maxentius and Licinius; and the resentment, which the latter did not
attempt to conceal, served only to engage them still more deeply in the
interest of his competitor. The regular correspondence which connected
the bishops of the most distant provinces, enabled them freely to
communicate their wishes and their designs, and to transmit without
danger any useful intelligence, or any pious contributions, which might
promote the service of Constantine, who publicly declared that he had
taken up arms for the deliverance of the church.\textsuperscript{28}

\pagenote[25]{In the beginning of the last century, the Papists of
England were only a \textit{thirtieth}, and the Protestants of France only a
\textit{fifteenth}, part of the respective nations, to whom their spirit and
power were a constant object of apprehension. See the relations which
Bentivoglio (who was then nuncio at Brussels, and afterwards cardinal)
transmitted to the court of Rome, (Relazione, tom. ii. p. 211, 241.)
Bentivoglio was curious, well informed, but somewhat partial.}

\pagenote[26]{This careless temper of the Germans appears almost
uniformly on the history of the conversion of each of the tribes. The
legions of Constantine were recruited with Germans, (Zosimus, l. ii. p.
86;) and the court even of his father had been filled with Christians.
See the first book of the Life of Constantine, by Eusebius.}

\pagenote[27]{De his qui arma projiciunt in \textit{pace}, placuit eos
abstinere a communione. Council. Arelat. Canon. iii. The best critics
apply these words to the \textit{peace of the church}.}

\pagenote[28]{Eusebius always considers the second civil war against
Licinius as a sort of religious crusade. At the invitation of the
tyrant, some Christian officers had resumed their \textit{zones;} or, in other
words, had returned to the military service. Their conduct was
afterwards censured by the twelfth canon of the Council of Nice; if
this particular application may be received, instead of the lo se and
general sense of the Greek interpreters, Balsamor Zonaras, and Alexis
Aristenus. See Beveridge, Pandect. Eccles. Græc. tom. i. p. 72, tom.
ii. p. 73 Annotation.}

The enthusiasm which inspired the troops, and perhaps the emperor
himself, had sharpened their swords while it satisfied their
conscience. They marched to battle with the full assurance, that the
same God, who had formerly opened a passage to the Israelites through
the waters of Jordan, and had thrown down the walls of Jericho at the
sound of the trumpets of Joshua, would display his visible majesty and
power in the victory of Constantine. The evidence of ecclesiastical
history is prepared to affirm, that their expectations were justified
by the conspicuous miracle to which the conversion of the first
Christian emperor has been almost unanimously ascribed. The real or
imaginary cause of so important an event, deserves and demands the
attention of posterity; and I shall endeavor to form a just estimate of
the famous vision of Constantine, by a distinct consideration of the
\textit{standard}, the \textit{dream}, and the \textit{celestial sign;} by separating the
historical, the natural, and the marvellous parts of this extraordinary
story, which, in the composition of a specious argument, have been
artfully confounded in one splendid and brittle mass.

I. An instrument of the tortures which were inflicted only on slaves
and strangers, became on object of horror in the eyes of a Roman
citizen; and the ideas of guilt, of pain, and of ignominy, were closely
united with the idea of the cross.\textsuperscript{29} The piety, rather than the
humanity, of Constantine soon abolished in his dominions the punishment
which the Savior of mankind had condescended to suffer;\textsuperscript{30} but the
emperor had already learned to despise the prejudices of his education,
and of his people, before he could erect in the midst of Rome his own
statue, bearing a cross in its right hand; with an inscription which
referred the victory of his arms, and the deliverance of Rome, to the
virtue of that salutary sign, the true symbol of force and courage.\textsuperscript{31}
The same symbol sanctified the arms of the soldiers of Constantine; the
cross glittered on their helmet, was engraved on their shields, was
interwoven into their banners; and the consecrated emblems which
adorned the person of the emperor himself, were distinguished only by
richer materials and more exquisite workmanship.\textsuperscript{32} But the principal
standard which displayed the triumph of the cross was styled the
Labarum,\textsuperscript{33} an obscure, though celebrated name, which has been vainly
derived from almost all the languages of the world. It is described\textsuperscript{34}
as a long pike intersected by a transversal beam. The silken veil,
which hung down from the beam, was curiously inwrought with the images
of the reigning monarch and his children. The summit of the pike
supported a crown of gold which enclosed the mysterious monogram, at
once expressive of the figure of the cross, and the initial letters, of
the name of Christ.\textsuperscript{35} The safety of the labarum was intrusted to fifty
guards, of approved valor and fidelity; their station was marked by
honors and emoluments; and some fortunate accidents soon introduced an
opinion, that as long as the guards of the labarum were engaged in the
execution of their office, they were secure and invulnerable amidst the
darts of the enemy. In the second civil war, Licinius felt and dreaded
the power of this consecrated banner, the sight of which, in the
distress of battle, animated the soldiers of Constantine with an
invincible enthusiasm, and scattered terror and dismay through the
ranks of the adverse legions.\textsuperscript{36} The Christian emperors, who respected
the example of Constantine, displayed in all their military expeditions
the standard of the cross; but when the degenerate successors of
Theodosius had ceased to appear in person at the head of their armies,
the labarum was deposited as a venerable but useless relic in the
palace of Constantinople.\textsuperscript{37} Its honors are still preserved on the
medals of the Flavian family. Their grateful devotion has placed the
monogram of Christ in the midst of the ensigns of Rome. The solemn
epithets of, safety of the republic, glory of the army, restoration of
public happiness, are equally applied to the religious and military
trophies; and there is still extant a medal of the emperor Constantius,
where the standard of the labarum is accompanied with these memorable
words, BY THIS SIGN THOU SHALT CONQUER.\textsuperscript{38}

\pagenote[29]{Nomen ipsum \textit{crucis} absit non modo a corpore civium
Romano rum, sed etiam a cogitatione, oculis, auribus. Cicero pro
Raberio, c. 5. The Christian writers, Justin, Minucius Felix,
Tertullian, Jerom, and Maximus of Turin, have investigated with
tolerable success the figure or likeness of a cross in almost every
object of nature or art; in the intersection of the meridian and
equator, the human face, a bird flying, a man swimming, a mast and
yard, a plough, a \textit{standard}, \&c., \&c., \&c. See Lipsius de Cruce, l. i.
c. 9.}

\pagenote[30]{See Aurelius Victor, who considers this law as one of the
examples of Constantine’s piety. An edict so honorable to Christianity
deserved a place in the Theodosian Code, instead of the indirect
mention of it, which seems to result from the comparison of the fifth
and eighteenth titles of the ninth book.}

\pagenote[31]{Eusebius, in Vit. Constantin. l. i. c. 40. This statue,
or at least the cross and inscription, may be ascribed with more
probability to the second, or even third, visit of Constantine to Rome.
Immediately after the defeat of Maxentius, the minds of the senate and
people were scarcely ripe for this public monument.}

\pagenote[32]{Agnoscas, regina, libens mea signa necesse est;
In quibus effigies crucis aut gemmata refulget
Aut longis solido ex auro præfertur in hastis.
Hoc signo invictus, transmissis Alpibus Ultor
Servitium solvit miserabile Constantinus.

Christus \textit{purpureum} gemmanti textus in auro
Signabat \textit{Labarum}, clypeorum insignia Christus
Scripserat; ardebat summis crux addita cristis.

Prudent. in Symmachum, l. ii. 464, 486.}

\pagenote[33]{The derivation and meaning of the word \textit{Labarum} or
\textit{Laborum}, which is employed by Gregory Nazianzen, Ambrose, Prudentius,
\&c., still remain totally unknown, in spite of the efforts of the
critics, who have ineffectually tortured the Latin, Greek, Spanish,
Celtic, Teutonic, Illyric, Armenian, \&c., in search of an etymology.
See Ducange, in Gloss. Med. et infim. Latinitat. sub voce \textit{Labarum},
and Godefroy, ad Cod. Theodos. tom. ii. p. 143.}

\pagenote[34]{Euseb. in Vit. Constantin. l. i. c. 30, 31. Baronius
(Annal. Eccles. A. D. 312, No. 26) has engraved a representation of the
Labarum.}

\pagenote[35]{Transversâ X literâ, summo capite circumflexo, Christum
in scutis notat. Cæcilius de M. P. c. 44, Cuper, (ad M. P. in edit.
Lactant. tom. ii. p. 500,) and Baronius (A. D. 312, No. 25) have
engraved from ancient monuments several specimens (as thus of these
monograms) which became extremely fashionable in the Christian world.}

\pagenote[36]{Euseb. in Vit. Constantin. l. ii. c. 7, 8, 9. He
introduces the Labarum before the Italian expedition; but his narrative
seems to indicate that it was never shown at the head of an army till
Constantine above ten years afterwards, declared himself the enemy of
Licinius, and the deliverer of the church.}

\pagenote[37]{See Cod. Theod. l. vi. tit. xxv. Sozomen, l. i. c. 2.
Theophan. Chronograph. p. 11. Theophanes lived towards the end of the
eighth century, almost five hundred years after Constantine. The modern
Greeks were not inclined to display in the field the standard of the
empire and of Christianity; and though they depended on every
superstitious hope of \textit{defence}, the promise of \textit{victory} would have
appeared too bold a fiction.}

\pagenote[38]{The Abbé du Voisín, p. 103, \&c., alleges several of these
medals, and quotes a particular dissertation of a Jesuit the Père de
Grainville, on this subject.}

II. In all occasions of danger and distress, it was the practice of the
primitive Christians to fortify their minds and bodies by the sign of
the cross, which they used, in all their ecclesiastical rites, in all
the daily occurrences of life, as an infallible preservative against
every species of spiritual or temporal evil.\textsuperscript{39} The authority of the
church might alone have had sufficient weight to justify the devotion
of Constantine, who in the same prudent and gradual progress
acknowledged the truth, and assumed the symbol, of Christianity. But
the testimony of a contemporary writer, who in a formal treatise has
avenged the cause of religion, bestows on the piety of the emperor a
more awful and sublime character. He affirms, with the most perfect
confidence, that in the night which preceded the last battle against
Maxentius, Constantine was admonished in a dream\textsuperscript{39a} to inscribe the
shields of his soldiers with the \textit{celestial sign of God}, the sacred
monogram of the name of Christ; that he executed the commands of
Heaven, and that his valor and obedience were rewarded by the decisive
victory of the Milvian Bridge. Some considerations might perhaps
incline a sceptical mind to suspect the judgment or the veracity of the
rhetorician, whose pen, either from zeal or interest, was devoted to
the cause of the prevailing faction.\textsuperscript{40} He appears to have published
his deaths of the persecutors at Nicomedia about three years after the
Roman victory; but the interval of a thousand miles, and a thousand
days, will allow an ample latitude for the invention of declaimers, the
credulity of party, and the tacit approbation of the emperor himself
who might listen without indignation to a marvellous tale, which
exalted his fame, and promoted his designs. In favor of Licinius, who
still dissembled his animosity to the Christians, the same author has
provided a similar vision, of a form of prayer, which was communicated
by an angel, and repeated by the whole army before they engaged the
legions of the tyrant Maximin. The frequent repetition of miracles
serves to provoke, where it does not subdue, the reason of mankind;\textsuperscript{41}
but if the dream of Constantine is separately considered, it may be
naturally explained either by the policy or the enthusiasm of the
emperor. Whilst his anxiety for the approaching day, which must decide
the fate of the empire, was suspended by a short and interrupted
slumber, the venerable form of Christ, and the well-known symbol of his
religion, might forcibly offer themselves to the active fancy of a
prince who reverenced the name, and had perhaps secretly implored the
power, of the God of the Christians. As readily might a consummate
statesman indulge himself in the use of one of those military
stratagems, one of those pious frauds, which Philip and Sertorius had
employed with such art and effect.\textsuperscript{42} The præternatural origin of
dreams was universally admitted by the nations of antiquity, and a
considerable part of the Gallic army was already prepared to place
their confidence in the salutary sign of the Christian religion. The
secret vision of Constantine could be disproved only by the event; and
the intrepid hero who had passed the Alps and the Apennine, might view
with careless despair the consequences of a defeat under the walls of
Rome. The senate and people, exulting in their own deliverance from an
odious tyrant, acknowledged that the victory of Constantine surpassed
the powers of man, without daring to insinuate that it had been
obtained by the protection of the \textit{Gods}. The triumphal arch, which was
erected about three years after the event, proclaims, in ambiguous
language, that by the greatness of his own mind, and by an \textit{instinct}
or impulse of the Divinity, he had saved and avenged the Roman
republic.\textsuperscript{43} The Pagan orator, who had seized an earlier opportunity of
celebrating the virtues of the conqueror, supposes that he alone
enjoyed a secret and intimate commerce with the Supreme Being, who
delegated the care of mortals to his subordinate deities; and thus
assigns a very plausible reason why the subjects of Constantine should
not presume to embrace the new religion of their sovereign.\textsuperscript{44}

\pagenote[39]{Tertullian de Corona, c. 3. Athanasius, tom. i. p. 101.
The learned Jesuit Petavius (Dogmata Theolog. l. xv. c. 9, 10) has
collected many similar passages on the virtues of the cross, which in
the last age embarrassed our Protestant disputants.}

\pagenote[39a]{Manso has observed, that Gibbon ought not to have
separated the vision of Constantine from the wonderful apparition in
the sky, as the two wonders are closely connected in Eusebius. Manso,
Leben Constantine, p. 82—M.}

\pagenote[40]{Cæcilius de M. P. c. 44. It is certain, that this
historical declamation was composed and published while Licinius,
sovereign of the East, still preserved the friendship of Constantine
and of the Christians. Every reader of taste must perceive that the
style is of a very different and inferior character to that of
Lactantius; and such indeed is the judgment of Le Clerc and Lardner,
(Bibliothèque Ancienne et Moderne, tom. iii. p. 438. Credibility of the
Gospel, \&c., part ii. vol. vii. p. 94.) Three arguments from the title
of the book, and from the names of Donatus and Cæcilius, are produced
by the advocates for Lactantius. (See the P. Lestocq, tom. ii. p.
46-60.) Each of these proofs is singly weak and defective; but their
concurrence has great weight. I have often fluctuated, and shall
\textit{tamely} follow the Colbert Ms. in calling the author (whoever he was)
Cæcilius.}

\pagenote[41]{Cæcilius de M. P. c. 46. There seems to be some reason in
the observation of M. de Voltaire, (Œuvres, tom. xiv. p. 307.) who
ascribes to the success of Constantine the superior fame of his Labarum
above the angel of Licinius. Yet even this angel is favorably
entertained by Pagi, Tillemont, Fleury, \&c., who are fond of increasing
their stock of miracles.}

\pagenote[42]{Besides these well-known examples, Tollius (Preface to
Boileau’s translation of Longinus) has discovered a vision of
Antigonus, who assured his troops that he had seen a pentagon (the
symbol of safety) with these words, “In this conquer.” But Tollius has
most inexcusably omitted to produce his authority, and his own
character, literary as well as moral, is not free from reproach. (See
Chauffepié, Dictionnaire Critique, tom. iv. p. 460.) Without insisting
on the silence of Diodorus Plutarch, Justin, \&c., it may be observed
that Polyænus, who in a separate chapter (l. iv. c. 6) has collected
nineteen military stratagems of Antigonus, is totally ignorant of this
remarkable vision.}

\pagenote[43]{Instinctu Divinitatis, mentis magnitudine. The
inscription on the triumphal arch of Constantine, which has been copied
by Baronius, Gruter, \&c., may still be perused by every curious
traveller.}

\pagenote[44]{Habes profecto aliquid cum illa mente Divinâ secretum;
quæ delegatâ nostrâ Diis Minoribus curâ uni se tibi dignatur ostendere
Panegyr. Vet. ix. 2.}

III. The philosopher, who with calm suspicion examines the dreams and
omens, the miracles and prodigies, of profane or even of ecclesiastical
history, will probably conclude, that if the eyes of the spectators
have sometimes been deceived by fraud, the understanding of the readers
has much more frequently been insulted by fiction. Every event, or
appearance, or accident, which seems to deviate from the ordinary
course of nature, has been rashly ascribed to the immediate action of
the Deity; and the astonished fancy of the multitude has sometimes
given shape and color, language and motion, to the fleeting but
uncommon meteors of the air.\textsuperscript{45} Nazarius and Eusebius are the two most
celebrated orators, who, in studied panegyrics, have labored to exalt
the glory of Constantine. Nine years after the Roman victory, Nazarius\textsuperscript{46}
describes an army of divine warriors, who seemed to fall from the
sky: he marks their beauty, their spirit, their gigantic forms, the
stream of light which beamed from their celestial armor, their patience
in suffering themselves to be heard, as well as seen, by mortals; and
their declaration that they were sent, that they flew, to the
assistance of the great Constantine. For the truth of this prodigy, the
Pagan orator appeals to the whole Gallic nation, in whose presence he
was then speaking; and seems to hope that the ancient apparitions\textsuperscript{47}
would now obtain credit from this recent and public event. The
Christian fable of Eusebius, which, in the space of twenty-six years,
might arise from the original dream, is cast in a much more correct and
elegant mould. In one of the marches of Constantine, he is reported to
have seen with his own eyes the luminous trophy of the cross, placed
above the meridian sun and inscribed with the following words: BY THIS
CONQUER. This amazing object in the sky astonished the whole army, as
well as the emperor himself, who was yet undetermined in the choice of
a religion: but his astonishment was converted into faith by the vision
of the ensuing night. Christ appeared before his eyes; and displaying
the same celestial sign of the cross, he directed Constantine to frame
a similar standard, and to march, with an assurance of victory, against
Maxentius and all his enemies.\textsuperscript{48} The learned bishop of Cæsarea appears
to be sensible, that the recent discovery of this marvellous anecdote
would excite some surprise and distrust among the most pious of his
readers. Yet, instead of ascertaining the precise circumstances of time
and place, which always serve to detect falsehood or establish truth;\textsuperscript{49}
instead of collecting and recording the evidence of so many living
witnesses who must have been spectators of this stupendous miracle;\textsuperscript{50}
Eusebius contents himself with alleging a very singular testimony; that
of the deceased Constantine, who, many years after the event, in the
freedom of conversation, had related to him this extraordinary incident
of his own life, and had attested the truth of it by a solemn oath. The
prudence and gratitude of the learned prelate forbade him to suspect
the veracity of his victorious master; but he plainly intimates, that
in a fact of such a nature, he should have refused his assent to any
meaner authority. This motive of credibility could not survive the
power of the Flavian family; and the celestial sign, which the Infidels
might afterwards deride,\textsuperscript{51} was disregarded by the Christians of the
age which immediately followed the conversion of Constantine.\textsuperscript{52} But
the Catholic church, both of the East and of the West, has adopted a
prodigy which favors, or seems to favor, the popular worship of the
cross. The vision of Constantine maintained an honorable place in the
legend of superstition, till the bold and sagacious spirit of criticism
presumed to depreciate the triumph, and to arraign the truth, of the
first Christian emperor.\textsuperscript{53}

\pagenote[45]{M. Freret (Mémoires de l’Académie des Inscriptions, tom.
iv. p. 411-437) explains, by physical causes, many of the prodigies of
antiquity; and Fabricius, who is abused by both parties, vainly tries
to introduce the celestial cross of Constantine among the solar halos.
Bibliothec. Græc. tom. iv. p. 8-29. * Note: The great difficulty in
resolving it into a natural phenomenon, arises from the inscription;
even the most heated or awe-struck imagination would hardly discover
distinct and legible letters in a solar halo. But the inscription may
have been a later embellishment, or an interpretation of the meaning
which the sign was construed to convey. Compare Heirichen, Excur in
locum Eusebii, and the authors quoted.}

\pagenote[46]{Nazarius inter Panegyr. Vet. x. 14, 15. It is unnecessary
to name the moderns, whose undistinguishing and ravenous appetite has
swallowed even the Pagan bait of Nazarius.}

\pagenote[47]{The apparitions of Castor and Pollux, particularly to
announce the Macedonian victory, are attested by historians and public
monuments. See Cicero de Natura Deorum, ii. 2, iii. 5, 6. Florus, ii.
12. Valerius Maximus, l. i. c. 8, No. 1. Yet the most recent of these
miracles is omitted, and indirectly denied, by Livy, (xlv. i.)}

\pagenote[48]{Eusebius, l. i. c. 28, 29, 30. The silence of the same
Eusebius, in his Ecclesiastical History, is deeply felt by those
advocates for the miracle who are not absolutely callous.}

\pagenote[49]{The narrative of Constantine seems to indicate, that he
saw the cross in the sky before he passed the Alps against Maxentius.
The scene has been fixed by provincial vanity at Trèves, Besançon, \&c.
See Tillemont, Hist. des Empereurs, tom. iv. p. 573.}

\pagenote[50]{The pious Tillemont (Mém. Eccles. tom. vii. p. 1317)
rejects with a sigh the useful Acts of Artemius, a veteran and a
martyr, who attests as an eye-witness to the vision of Constantine.}

\pagenote[51]{Gelasius Cyzic. in Act. Concil. Nicen. l. i. c. 4.}

\pagenote[52]{The advocates for the vision are unable to produce a
single testimony from the Fathers of the fourth and fifth centuries,
who, in their voluminous writings, repeatedly celebrate the triumph of
the church and of Constantine. As these venerable men had not any
dislike to a miracle, we may suspect, (and the suspicion is confirmed
by the ignorance of Jerom,) that they were all unacquainted with the
life of Constantine by Eusebius. This tract was recovered by the
diligence of those who translated or continued his Ecclesiastical
History, and who have represented in various colors the vision of the
cross.}

\pagenote[53]{Godefroy was the first, who, in the year 1643, (Not ad
Philostorgium, l. i. c. 6, p. 16,) expressed any doubt of a miracle
which had been supported with equal zeal by Cardinal Baronius, and the
Centuriators of Magdeburgh. Since that time, many of the Protestant
critics have inclined towards doubt and disbelief. The objections are
urged, with great force, by M. Chauffepié, (Dictionnaire Critique, tom.
iv. p. 6–11;) and, in the year 1774, a doctor of Sorbonne, the Abbé du
Voisin published an apology, which deserves the praise of learning and
moderation. * Note: The first Excursus of Heinichen (in Vitam
Constantini, p. 507) contains a full summary of the opinions and
arguments of the later writers who have discussed this interminable
subject. As to his conversion, where interest and inclination, state
policy, and, if not a sincere conviction of its truth, at least a
respect, an esteem, an awe of Christianity, thus coincided, Constantine
himself would probably have been unable to trace the actual history of
the workings of his own mind, or to assign its real influence to each
concurrent motive.—M}

The Protestant and philosophic readers of the present age will incline
to believe, that in the account of his own conversion, Constantine
attested a wilful falsehood by a solemn and deliberate perjury. They
may not hesitate to pronounce, that in the choice of a religion, his
mind was determined only by a sense of interest; and that (according to
the expression of a profane poet) 54 he used the altars of the church
as a convenient footstool to the throne of the empire. A conclusion so
harsh and so absolute is not, however, warranted by our knowledge of
human nature, of Constantine, or of Christianity. In an age of
religious fervor, the most artful statesmen are observed to feel some
part of the enthusiasm which they inspire, and the most orthodox saints
assume the dangerous privilege of defending the cause of truth by the
arms of deceit and falsehood.

Personal interest is often the standard of our belief, as well as of
our practice; and the same motives of temporal advantage which might
influence the public conduct and professions of Constantine, would
insensibly dispose his mind to embrace a religion so propitious to his
fame and fortunes. His vanity was gratified by the flattering
assurance, that \textit{he} had been chosen by Heaven to reign over the earth;
success had justified his divine title to the throne, and that title
was founded on the truth of the Christian revelation. As real virtue is
sometimes excited by undeserved applause, the specious piety of
Constantine, if at first it was only specious, might gradually, by the
influence of praise, of habit, and of example, be matured into serious
faith and fervent devotion. The bishops and teachers of the new sect,
whose dress and manners had not qualified them for the residence of a
court, were admitted to the Imperial table; they accompanied the
monarch in his expeditions; and the ascendant which one of them, an
Egyptian or a Spaniard,\textsuperscript{55} acquired over his mind, was imputed by the
Pagans to the effect of magic.\textsuperscript{56} Lactantius, who has adorned the
precepts of the gospel with the eloquence of Cicero,\textsuperscript{57} and Eusebius,
who has consecrated the learning and philosophy of the Greeks to the
service of religion,\textsuperscript{58} were both received into the friendship and
familiarity of their sovereign; and those able masters of controversy
could patiently watch the soft and yielding moments of persuasion, and
dexterously apply the arguments which were the best adapted to his
character and understanding. Whatever advantages might be derived from
the acquisition of an Imperial proselyte, he was distinguished by the
splendor of his purple, rather than by the superiority of wisdom, or
virtue, from the many thousands of his subjects who had embraced the
doctrines of Christianity. Nor can it be deemed incredible, that the
mind of an unlettered soldier should have yielded to the weight of
evidence, which, in a more enlightened age, has satisfied or subdued
the reason of a Grotius, a Pascal, or a Locke. In the midst of the
incessant labors of his great office, this soldier employed, or
affected to employ, the hours of the night in the diligent study of the
Scriptures, and the composition of theological discourses; which he
afterwards pronounced in the presence of a numerous and applauding
audience. In a very long discourse, which is still extant, the royal
preacher expatiates on the various proofs still extant, the royal
preacher expatiates on the various proofs of religion; but he dwells
with peculiar complacency on the Sibylline verses,\textsuperscript{59} and the fourth
eclogue of Virgil.\textsuperscript{60} Forty years before the birth of Christ, the
Mantuan bard, as if inspired by the celestial muse of Isaiah, had
celebrated, with all the pomp of oriental metaphor, the return of the
Virgin, the fall of the serpent, the approaching birth of a godlike
child, the offspring of the great Jupiter, who should expiate the guilt
of human kind, and govern the peaceful universe with the virtues of his
father; the rise and appearance of a heavenly race, primitive nation
throughout the world; and the gradual restoration of the innocence and
felicity of the golden age. The poet was perhaps unconscious of the
secret sense and object of these sublime predictions, which have been
so unworthily applied to the infant son of a consul, or a triumvir;\textsuperscript{61}
but if a more splendid, and indeed specious interpretation of the
fourth eclogue contributed to the conversion of the first Christian
emperor, Virgil may deserve to be ranked among the most successful
missionaries of the gospel.\textsuperscript{62}

\pagenote[54]{\begin{verse}
Lors Constantin dit ces propres paroles:\\
J’ai renversé le culte des idoles:\\
Sur les debris de leurs temples fumans\\
Au Dieu du Ciel j’ai prodigue l’encens.\\
Mais tous mes soins pour sa grandeur supreme\\
\vinphantom{00}N’eurent jamais d’autre objêt que moi-même;\\
\vspace{\onelineskip}
Les saints autels n’etoient à mes regards\\
Qu’un marchepié du trone des Césars.\\
L’ambition, la fureur, les delices\\
Etoient mes Dieux, avoient mes sacrifices.\\
L’or des Chrêtiens, leur intrigues, leur sang\\
\vinphantom{0}Ont cimenté ma fortune et mon rang.\\
\end{verse}

The poem which contains these lines may be read with pleasure, but
cannot be named with decency.}

\pagenote[55]{This favorite was probably the great Osius, bishop of
Cordova, who preferred the pastoral care of the whole church to the
government of a particular diocese. His character is magnificently,
though concisely, expressed by Athanasius, (tom. i. p. 703.) See
Tillemont, Mém. Eccles. tom. vii. p. 524-561. Osius was accused,
perhaps unjustly, of retiring from court with a very ample fortune.}

\pagenote[56]{See Eusebius (in Vit. Constant. passim) and Zosimus, l.
ii. p. 104.}

\pagenote[57]{The Christianity of Lactantius was of a moral rather than
of a mysterious cast. “Erat pæne rudis (says the orthodox Bull)
disciplinæ Christianæ, et in rhetorica melius quam in theologia
versatus.” Defensio Fidei Nicenæ, sect. ii. c. 14.}

\pagenote[58]{Fabricius, with his usual diligence, has collected a list
of between three and four hundred authors quoted in the Evangelical
Preparation of Eusebius. See Bibl. Græc. l. v. c. 4, tom. vi. p.
37-56.}

\pagenote[59]{See Constantin. Orat. ad Sanctos, c. 19 20. He chiefly
depends on a mysterious acrostic, composed in the sixth age after the
Deluge, by the Erythræan Sibyl, and translated by Cicero into Latin.
The initial letters of the thirty-four Greek verses form this prophetic
sentence: Jesus Christ, Son of God, Savior of the World.}

\pagenote[60]{In his paraphrase of Virgil, the emperor has frequently
assisted and improved the literal sense of the Latin ext. See Blondel
des Sibylles, l. i. c. 14, 15, 16.}

\pagenote[61]{The different claims of an elder and younger son of
Pollio, of Julia, of Drusus, of Marcellus, are found to be incompatible
with chronology, history, and the good sense of Virgil.}

\pagenote[62]{See Lowth de Sacra Poesi Hebræorum Prælect. xxi. p. 289-
293. In the examination of the fourth eclogue, the respectable bishop
of London has displayed learning, taste, ingenuity, and a temperate
enthusiasm, which exalts his fancy without degrading his judgment.}

\section{Part \thesection.}

The awful mysteries of the Christian faith and worship were concealed
from the eyes of strangers, and even of catechu mens, with an affected
secrecy, which served to excite their wonder and curiosity.\textsuperscript{63} But the
severe rules of discipline which the prudence of the bishops had
instituted, were relaxed by the same prudence in favor of an Imperial
proselyte, whom it was so important to allure, by every gentle
condescension, into the pale of the church; and Constantine was
permitted, at least by a tacit dispensation, to enjoy \textit{most} of the
privileges, before he had contracted \textit{any} of the obligations, of a
Christian. Instead of retiring from the congregation, when the voice of
the deacon dismissed the profane multitude, he prayed with the
faithful, disputed with the bishops, preached on the most sublime and
intricate subjects of theology, celebrated with sacred rites the vigil
of Easter, and publicly declared himself, not only a partaker, but, in
some measure, a priest and hierophant of the Christian mysteries.\textsuperscript{64}
The pride of Constantine might assume, and his services had deserved,
some extraordinary distinction: and ill-timed rigor might have blasted
the unripened fruits of his conversion; and if the doors of the church
had been strictly closed against a prince who had deserted the altars
of the gods, the master of the empire would have been left destitute of
any form of religious worship. In his last visit to Rome, he piously
disclaimed and insulted the superstition of his ancestors, by refusing
to lead the military procession of the equestrian order, and to offer
the public vows to the Jupiter of the Capitoline Hill.\textsuperscript{65} Many years
before his baptism and death, Constantine had proclaimed to the world,
that neither his person nor his image should ever more be seen within
the walls of an idolatrous temple; while he distributed through the
provinces a variety of medals and pictures, which represented the
emperor in an humble and suppliant posture of Christian devotion.\textsuperscript{66}

\pagenote[63]{The distinction between the public and the secret parts
of divine service, the \textit{missa catechumenorum} and the \textit{missa fidelium},
and the mysterious veil which piety or policy had cast over the latter,
are very judiciously explained by Thiers, Exposition du Saint
Sacrament, l. i. c. 8- 12, p. 59-91: but as, on this subject, the
Papists may reasonably be suspected, a Protestant reader will depend
with more confidence on the learned Bingham, Antiquities, l. x. c. 5.}

\pagenote[64]{See Eusebius in Vit. Const. l. iv. c. 15-32, and the
whole tenor of Constantine’s Sermon. The faith and devotion of the
emperor has furnished Batonics with a specious argument in favor of his
early baptism. Note: Compare Heinichen, Excursus iv. et v., where these
questions are examined with candor and acuteness, and with constant
reference to the opinions of more modern writers.—M.}

\pagenote[65]{Zosimus, l. ii. p. 105.}

\pagenote[66]{Eusebius in Vit. Constant. l. iv. c. 15, 16.}

The pride of Constantine, who refused the privileges of a catechumen,
cannot easily be explained or excused; but the delay of his baptism may
be justified by the maxims and the practice of ecclesiastical
antiquity. The sacrament of baptism\textsuperscript{67} was regularly administered by
the bishop himself, with his assistant clergy, in the cathedral church
of the diocese, during the fifty days between the solemn festivals of
Easter and Pentecost; and this holy term admitted a numerous band of
infants and adult persons into the bosom of the church. The discretion
of parents often suspended the baptism of their children till they
could understand the obligations which they contracted: the severity of
ancient bishops exacted from the new converts a novitiate of two or
three years; and the catechumens themselves, from different motives of
a temporal or a spiritual nature, were seldom impatient to assume the
character of perfect and initiated Christians. The sacrament of baptism
was supposed to contain a full and absolute expiation of sin; and the
soul was instantly restored to its original purity, and entitled to the
promise of eternal salvation. Among the proselytes of Christianity,
there are many who judged it imprudent to precipitate a salutary rite,
which could not be repeated; to throw away an inestimable privilege,
which could never be recovered. By the delay of their baptism, they
could venture freely to indulge their passions in the enjoyments of
this world, while they still retained in their own hands the means of a
sure and easy absolution.\textsuperscript{68} The sublime theory of the gospel had made
a much fainter impression on the heart than on the understanding of
Constantine himself. He pursued the great object of his ambition
through the dark and bloody paths of war and policy; and, after the
victory, he abandoned himself, without moderation, to the abuse of his
fortune. Instead of asserting his just superiority above the imperfect
heroism and profane philosophy of Trajan and the Antonines, the mature
age of Constantine forfeited the reputation which he had acquired in
his youth. As he gradually advanced in the knowledge of truth, he
proportionally declined in the practice of virtue; and the same year of
his reign in which he convened the council of Nice, was polluted by the
execution, or rather murder, of his eldest son. This date is alone
sufficient to refute the ignorant and malicious suggestions of Zosimus,\textsuperscript{69}
who affirms, that, after the death of Crispus, the remorse of his
father accepted from the ministers of christianity the expiation which
he had vainly solicited from the Pagan pontiffs. At the time of the
death of Crispus, the emperor could no longer hesitate in the choice of
a religion; he could no longer be ignorant that the church was
possessed of an infallible remedy, though he chose to defer the
application of it till the approach of death had removed the temptation
and danger of a relapse. The bishops whom he summoned, in his last
illness, to the palace of Nicomedia, were edified by the fervor with
which he requested and received the sacrament of baptism, by the solemn
protestation that the remainder of his life should be worthy of a
disciple of Christ, and by his humble refusal to wear the Imperial
purple after he had been clothed in the white garment of a Neophyte.
The example and reputation of Constantine seemed to countenance the
delay of baptism.\textsuperscript{70} Future tyrants were encouraged to believe, that
the innocent blood which they might shed in a long reign would
instantly be washed away in the waters of regeneration; and the abuse
of religion dangerously undermined the foundations of moral virtue.

\pagenote[67]{The theory and practice of antiquity, with regard to the
sacrament of baptism, have been copiously explained by Dom Chardon,
Hist. des Sacremens, tom. i. p. 3-405; Dom Martenne de Ritibus Ecclesiæ
Antiquis, tom. i.; and by Bingham, in the tenth and eleventh books of
his Christian Antiquities. One circumstance may be observed, in which
the modern churches have materially departed from the ancient custom.
The sacrament of baptism (even when it was administered to infants) was
immediately followed by confirmation and the holy communion.}

\pagenote[68]{The Fathers, who censured this criminal delay, could not
deny the certain and victorious efficacy even of a death-bed baptism.
The ingenious rhetoric of Chrysostom could find only three arguments
against these prudent Christians. 1. That we should love and pursue
virtue for her own sake, and not merely for the reward. 2. That we may
be surprised by death without an opportunity of baptism. 3. That
although we shall be placed in heaven, we shall only twinkle like
little stars, when compared to the suns of righteousness who have run
their appointed course with labor, with success, and with glory.
Chrysos tom in Epist. ad Hebræos, Homil. xiii. apud Chardon, Hist. des
Sacremens, tom. i. p. 49. I believe that this delay of baptism, though
attended with the most pernicious consequences, was never condemned by
any general or provincial council, or by any public act or declaration
of the church. The zeal of the bishops was easily kindled on much
slighter occasion. * Note: This passage of Chrysostom, though not in
his more forcible manner, is not quite fairly represented. He is
stronger in other places, in Act. Hom. xxiii.—and Hom. i. Compare,
likewise, the sermon of Gregory of Nysea on this subject, and Gregory
Nazianzen. After all, to those who believed in the efficacy of baptism,
what argument could be more conclusive, than the danger of dying
without it? Orat. xl.—M.}

\pagenote[69]{Zosimus, l. ii. p. 104. For this disingenuous falsehood
he has deserved and experienced the harshest treatment from all the
ecclesiastical writers, except Cardinal Baronius, (A. D. 324, No.
15-28,) who had occasion to employ the infidel on a particular service
against the Arian Eusebius. Note: Heyne, in a valuable note on this
passage of Zosimus, has shown decisively that this malicious way of
accounting for the conversion of Constantine was not an invention of
Zosimus. It appears to have been the current calumny eagerly adopted
and propagated by the exasperated Pagan party. Reitemeter, a later
editor of Zosimus, whose notes are retained in the recent edition, in
the collection of the Byzantine historians, has a disquisition on the
passage, as candid, but not more conclusive than some which have
preceded him—M.}

\pagenote[70]{Eusebius, l. iv. c. 61, 62, 63. The bishop of Cæsarea
supposes the salvation of Constantine with the most perfect
confidence.}

The gratitude of the church has exalted the virtues and excused the
failings of a generous patron, who seated Christianity on the throne of
the Roman world; and the Greeks, who celebrate the festival of the
Imperial saint, seldom mention the name of Constantine without adding
the title of \textit{equal to the Apostles}.\textsuperscript{71} Such a comparison, if it
allude to the character of those divine missionaries, must be imputed
to the extravagance of impious flattery. But if the parallel be
confined to the extent and number of their evangelic victories the
success of Constantine might perhaps equal that of the Apostles
themselves. By the edicts of toleration, he removed the temporal
disadvantages which had hitherto retarded the progress of Christianity;
and its active and numerous ministers received a free permission, a
liberal encouragement, to recommend the salutary truths of revelation
by every argument which could affect the reason or piety of mankind.
The exact balance of the two religions continued but a moment; and the
piercing eye of ambition and avarice soon discovered, that the
profession of Christianity might contribute to the interest of the
present, as well as of a future life.\textsuperscript{72} The hopes of wealth and
honors, the example of an emperor, his exhortations, his irresistible
smiles, diffused conviction among the venal and obsequious crowds which
usually fill the apartments of a palace. The cities which signalized a
forward zeal by the voluntary destruction of their temples, were
distinguished by municipal privileges, and rewarded with popular
donatives; and the new capital of the East gloried in the singular
advantage that Constantinople was never profaned by the worship of
idols.\textsuperscript{73} As the lower ranks of society are governed by imitation, the
conversion of those who possessed any eminence of birth, of power, or
of riches, was soon followed by dependent multitudes.\textsuperscript{74} The salvation
of the common people was purchased at an easy rate, if it be true that,
in one year, twelve thousand men were baptized at Rome, besides a
proportionable number of women and children, and that a white garment,
with twenty pieces of gold, had been promised by the emperor to every
convert.\textsuperscript{75} The powerful influence of Constantine was not circumscribed
by the narrow limits of his life, or of his dominions. The education
which he bestowed on his sons and nephews secured to the empire a race
of princes, whose faith was still more lively and sincere, as they
imbibed, in their earliest infancy, the spirit, or at least the
doctrine, of Christianity. War and commerce had spread the knowledge of
the gospel beyond the confines of the Roman provinces; and the
Barbarians, who had disdained as humble and proscribed sect, soon
learned to esteem a religion which had been so lately embraced by the
greatest monarch, and the most civilized nation, of the globe.\textsuperscript{76} The
Goths and Germans, who enlisted under the standard of Rome, revered the
cross which glittered at the head of the legions, and their fierce
countrymen received at the same time the lessons of faith and of
humanity. The kings of Iberia and Armenia\textsuperscript{76a} worshipped the god of
their protector; and their subjects, who have invariably preserved the
name of Christians, soon formed a sacred and perpetual connection with
their Roman brethren. The Christians of Persia were suspected, in time
of war, of preferring their religion to their country; but as long as
peace subsisted between the two empires, the persecuting spirit of the
Magi was effectually restrained by the interposition of Constantine.\textsuperscript{77}
The rays of the gospel illuminated the coast of India. The colonies of
Jews, who had penetrated into Arabia and Ethiopia,\textsuperscript{78} opposed the
progress of Christianity; but the labor of the missionaries was in some
measure facilitated by a previous knowledge of the Mosaic revelation;
and Abyssinia still reveres the memory of Frumentius,\textsuperscript{78a} who, in the
time of Constantine, devoted his life to the conversion of those
sequestered regions. Under the reign of his son Constantius,
Theophilus,\textsuperscript{79} who was himself of Indian extraction, was invested with
the double character of ambassador and bishop. He embarked on the Red
Sea with two hundred horses of the purest breed of Cappadocia, which
were sent by the emperor to the prince of the Sabæans, or Homerites.
Theophilus was intrusted with many other useful or curious presents,
which might raise the admiration, and conciliate the friendship, of the
Barbarians; and he successfully employed several years in a pastoral
visit to the churches of the torrid zone.\textsuperscript{80}

\pagenote[71]{See Tillemont, Hist. des Empereurs, tom. iv. p. 429. The
Greeks, the Russians, and, in the darker ages, the Latins themselves,
have been desirous of placing Constantine in the catalogue of saints.}

\pagenote[72]{See the third and fourth books of his life. He was
accustomed to say, that whether Christ was preached in pretence, or in
truth, he should still rejoice, (l. iii. c. 58.)}

\pagenote[73]{M. de Tillemont (Hist. des Empereurs, tom. iv. p. 374,
616) has defended, with strength and spirit, the virgin purity of
Constantinople against some malevolent insinuations of the Pagan
Zosimus.}

\pagenote[74]{The author of the Histoire Politique et Philosophique des
deux Indes (tom. i. p. 9) condemns a law of Constantine, which gave
freedom to all the slaves who should embrace Christianity. The emperor
did indeed publish a law, which restrained the Jews from circumcising,
perhaps from keeping, any Christian slave. (See Euseb. in Vit.
Constant. l. iv. c. 27, and Cod. Theod. l. xvi. tit. ix., with
Godefroy’s Commentary, tom. vi. p. 247.) But this imperfect exception
related only to the Jews, and the great body of slaves, who were the
property of Christian or Pagan masters, could not improve their
temporal condition by changing their religion. I am ignorant by what
guides the Abbé Raynal was deceived; as the total absence of quotations
is the unpardonable blemish of his entertaining history.}

\pagenote[75]{See Acta Sti Silvestri, and Hist. Eccles. Nicephor.
Callist. l. vii. c. 34, ap. Baronium Annal. Eccles. A. D. 324, No. 67,
74. Such evidence is contemptible enough; but these circumstances are
in themselves so probable, that the learned Dr. Howell (History of the
World, vol. iii. p. 14) has not scrupled to adopt them.}

\pagenote[76]{The conversion of the Barbarians under the reign of
Constantine is celebrated by the ecclesiastical historians. (See
Sozomen, l. ii. c. 6, and Theodoret, l. i. c. 23, 24.) But Rufinus, the
Latin translator of Eusebius, deserves to be considered as an original
authority. His information was curiously collected from one of the
companions of the Apostle of Æthiopia, and from Bacurius, an Iberian
prince, who was count of the domestics. Father Mamachi has given an
ample compilation on the progress of Christianity, in the first and
second volumes of his great but imperfect work.}

\pagenote[76a]{According to the Georgian chronicles, Iberia (Georgia)
was converted by the virgin Nino, who effected an extraordinary cure on
the wife of the king Mihran. The temple of the god Aramazt, or Armaz,
not far from the capital Mtskitha, was destroyed, and the cross erected
in its place. Le Beau, i. 202, with St. Martin’s Notes.—St. Martin has
likewise clearly shown (St. Martin, Add. to Le Beau, i. 291) Armenia
was the first \textit{nation} which embraced Christianity, (Addition to Le
Beau, i. 76. and Mémoire sur l’Armenie, i. 305.) Gibbon himself
suspected this truth.—“Instead of maintaining that the conversion of
Armenia was not attempted with any degree of success, till the sceptre
was in the hands of an orthodox emperor,” I ought to have said, that
the seeds of the faith were deeply sown during the season of the last
and greatest persecution, that many Roman exiles might assist the
labors of Gregory, and that the renowned Tiridates, the hero of the
East, may dispute with Constantine the honor of being the first
sovereign who embraced the Christian religion Vindication}

\pagenote[77]{See, in Eusebius, (in Vit. l. iv. c. 9,) the pressing and
pathetic epistle of Constantine in favor of his Christian brethren of
Persia.}

\pagenote[78]{See Basnage, Hist. des Juifs, tom. vii. p. 182, tom.
viii. p. 333, tom. ix. p. 810. The curious diligence of this writer
pursues the Jewish exiles to the extremities of the globe.}

\pagenote[78a]{Abba Salama, or Fremonatus, is mentioned in the Tareek
Negushti, chronicle of the kings of Abyssinia. Salt’s Travels, vol. ii.
p. 464.—M.}

\pagenote[79]{Theophilus had been given in his infancy as a hostage by
his countrymen of the Isle of Diva, and was educated by the Romans in
learning and piety. The Maldives, of which Male, or Diva, may be the
capital, are a cluster of 1900 or 2000 minute islands in the Indian
Ocean. The ancients were imperfectly acquainted with the Maldives; but
they are described in the two Mahometan travellers of the ninth
century, published by Renaudot, Geograph. Nubiensis, p. 30, 31
D’Herbelot, Bibliothèque Orientale p. 704. Hist. Generale des Voy ages,
tom. viii.—See the dissertation of M. Letronne on this question. He
conceives that Theophilus was born in the island of Dahlak, in the
Arabian Gulf. His embassy was to Abyssinia rather than to India.
Letronne, Materiaux pour l’Hist. du Christianisme en Egypte Indie, et
Abyssinie. Paris, 1832 3d Dissert.—M.}

\pagenote[80]{Philostorgius, l. iii. c. 4, 5, 6, with Godefroy’s
learned observations. The historical narrative is soon lost in an
inquiry concerning the seat of Paradise, strange monsters, \&c.}

The irresistible power of the Roman emperors was displayed in the
important and dangerous change of the national religion. The terrors of
a military force silenced the faint and unsupported murmurs of the
Pagans, and there was reason to expect, that the cheerful submission of
the Christian clergy, as well as people, would be the result of
conscience and gratitude. It was long since established, as a
fundamental maxim of the Roman constitution, that every rank of
citizens was alike subject to the laws, and that the care of religion
was the right as well as duty of the civil magistrate. Constantine and
his successors could not easily persuade themselves that they had
forfeited, by their conversion, any branch of the Imperial
prerogatives, or that they were incapable of giving laws to a religion
which they had protected and embraced. The emperors still continued to
exercise a supreme jurisdiction over the ecclesiastical order, and the
sixteenth book of the Theodosian code represents, under a variety of
titles, the authority which they assumed in the government of the
Catholic church. But the distinction of the spiritual and temporal
powers,\textsuperscript{81} which had never been imposed on the free spirit of Greece
and Rome, was introduced and confirmed by the legal establishment of
Christianity. The office of supreme pontiff, which, from the time of
Numa to that of Augustus, had always been exercised by one of the most
eminent of the senators, was at length united to the Imperial dignity.
The first magistrate of the state, as often as he was prompted by
superstition or policy, performed with his own hands the sacerdotal
functions;\textsuperscript{82} nor was there any order of priests, either at Rome or in
the provinces, who claimed a more sacred character among men, or a more
intimate communication with the gods. But in the Christian church,
which instrusts the service of the altar to a perpetual succession of
consecrated ministers, the monarch, whose spiritual rank is less
honorable than that of the meanest deacon, was seated below the rails
of the sanctuary, and confounded with the rest of the faithful
multitude.\textsuperscript{83} The emperor might be saluted as the father of his people,
but he owed a filial duty and reverence to the fathers of the church;
and the same marks of respect, which Constantine had paid to the
persons of saints and confessors, were soon exacted by the pride of the
episcopal order.\textsuperscript{84} A secret conflict between the civil and
ecclesiastical jurisdictions embarrassed the operation of the Roman
government; and a pious emperor was alarmed by the guilt and danger of
touching with a profane hand the ark of the covenant. The separation of
men into the two orders of the clergy and of the laity was, indeed,
familiar to many nations of antiquity; and the priests of India, of
Persia, of Assyria, of Judea, of Æthiopia, of Egypt, and of Gaul,
derived from a celestial origin the temporal power and possessions
which they had acquired. These venerable institutions had gradually
assimilated themselves to the manners and government of their
respective countries;\textsuperscript{85} but the opposition or contempt of the civil
power served to cement the discipline of the primitive church. The
Christians had been obliged to elect their own magistrates, to raise
and distribute a peculiar revenue, and to regulate the internal policy
of their republic by a code of laws, which were ratified by the consent
of the people and the practice of three hundred years. When Constantine
embraced the faith of the Christians, he seemed to contract a perpetual
alliance with a distinct and independent society; and the privileges
granted or confirmed by that emperor, or by his successors, were
accepted, not as the precarious favors of the court, but as the just
and inalienable rights of the ecclesiastical order.

\pagenote[81]{See the epistle of Osius, ap. Athanasium, vol. i. p. 840.
The public remonstrance which Osius was forced to address to the son,
contained the same principles of ecclesiastical and civil government
which he had secretly instilled into the mind of the father.}

\pagenote[82]{M. de la Bastiel has evidently proved, that Augustus and
his successors exercised in person all the sacred functions of pontifex
maximus, of high priest, of the Roman empire.}

\pagenote[83]{Something of a contrary practice had insensibly prevailed
in the church of Constantinople; but the rigid Ambrose commanded
Theodosius to retire below the rails, and taught him to know the
difference between a king and a priest. See Theodoret, l. v. c. 18.}

\pagenote[84]{At the table of the emperor Maximus, Martin, bishop of
Tours, received the cup from an attendant, and gave it to the
presbyter, his companion, before he allowed the emperor to drink; the
empress waited on Martin at table. Sulpicius Severus, in Vit. Sti
Martin, c. 23, and Dialogue ii. 7. Yet it may be doubted, whether these
extraordinary compliments were paid to the bishop or the saint. The
honors usually granted to the former character may be seen in Bingham’s
Antiquities, l. ii. c. 9, and Vales ad Theodoret, l. iv. c. 6. See the
haughty ceremonial which Leontius, bishop of Tripoli, imposed on the
empress. Tillemont, Hist. des Empereurs, tom. iv. p. 754. (Patres
Apostol. tom. ii. p. 179.)}

\pagenote[85]{Plutarch, in his treatise of Isis and Osiris, informs us
that the kings of Egypt, who were not already priests, were initiated,
after their election, into the sacerdotal order.}

The Catholic church was administered by the spiritual and legal
jurisdiction of eighteen hundred bishops;\textsuperscript{86} of whom one thousand were
seated in the Greek, and eight hundred in the Latin, provinces of the
empire. The extent and boundaries of their respective dioceses had been
variously and accidentally decided by the zeal and success of the first
missionaries, by the wishes of the people, and by the propagation of
the gospel. Episcopal churches were closely planted along the banks of
the Nile, on the sea-coast of Africa, in the proconsular Asia, and
through the southern provinces of Italy. The bishops of Gaul and Spain,
of Thrace and Pontus, reigned over an ample territory, and delegated
their rural suffragans to execute the subordinate duties of the
pastoral office.\textsuperscript{87} A Christian diocese might be spread over a
province, or reduced to a village; but all the bishops possessed an
equal and indelible character: they all derived the same powers and
privileges from the apostles, from the people, and from the laws. While
the \textit{civil} and \textit{military} professions were separated by the policy of
Constantine, a new and perpetual order of \textit{ecclesiastical} ministers,
always respectable, sometimes dangerous, was established in the church
and state. The important review of their station and attributes may be
distributed under the following heads: I. Popular Election. II.
Ordination of the Clergy. III. Property. IV. Civil Jurisdiction. V.
Spiritual censures. VI. Exercise of public oratory. VII. Privilege of
legislative assemblies.

\pagenote[86]{The numbers are not ascertained by any ancient writer or
original catalogue; for the partial lists of the eastern churches are
comparatively modern. The patient diligence of Charles a Sto Paolo, of
Luke Holstentius, and of Bingham, has laboriously investigated all the
episcopal sees of the Catholic church, which was almost commensurate
with the Roman empire. The ninth book of the Christian antiquities is a
very accurate map of ecclesiastical geography.}

\pagenote[87]{On the subject of rural bishops, or \textit{Chorepiscopi}, who
voted in tynods, and conferred the minor orders, See Thomassin,
Discipline de l’Eglise, tom. i. p. 447, \&c., and Chardon, Hist. des
Sacremens, tom. v. p. 395, \&c. They do not appear till the fourth
century; and this equivocal character, which had excited the jealousy
of the prelates, was abolished before the end of the tenth, both in the
East and the West.}

I. The freedom of election subsisted long after the legal establishment
of Christianity;\textsuperscript{88} and the subjects of Rome enjoyed in the church the
privilege which they had lost in the republic, of choosing the
magistrates whom they were bound to obey. As soon as a bishop had
closed his eyes, the metropolitan issued a commission to one of his
suffragans to administer the vacant see, and prepare, within a limited
time, the future election. The right of voting was vested in the
inferior clergy, who were best qualified to judge of the merit of the
candidates; in the senators or nobles of the city, all those who were
distinguished by their rank or property; and finally in the whole body
of the people, who, on the appointed day, flocked in multitudes from
the most remote parts of the diocese,\textsuperscript{89} and sometimes silenced by
their tumultuous acclamations, the voice of reason and the laws of
discipline. These acclamations might accidentally fix on the head of
the most deserving competitor; of some ancient presbyter, some holy
monk, or some layman, conspicuous for his zeal and piety. But the
episcopal chair was solicited, especially in the great and opulent
cities of the empire, as a temporal rather than as a spiritual dignity.
The interested views, the selfish and angry passions, the arts of
perfidy and dissimulation, the secret corruption, the open and even
bloody violence which had formerly disgraced the freedom of election in
the commonwealths of Greece and Rome, too often influenced the choice
of the successors of the apostles. While one of the candidates boasted
the honors of his family, a second allured his judges by the delicacies
of a plentiful table, and a third, more guilty than his rivals, offered
to share the plunder of the church among the accomplices of his
sacrilegious hopes\textsuperscript{90} The civil as well as ecclesiastical laws
attempted to exclude the populace from this solemn and important
transaction. The canons of ancient discipline, by requiring several
episcopal qualifications, of age, station, \&c., restrained, in some
measure, the indiscriminate caprice of the electors. The authority of
the provincial bishops, who were assembled in the vacant church to
consecrate the choice of the people, was interposed to moderate their
passions and to correct their mistakes. The bishops could refuse to
ordain an unworthy candidate, and the rage of contending factions
sometimes accepted their impartial mediation. The submission, or the
resistance, of the clergy and people, on various occasions, afforded
different precedents, which were insensibly converted into positive
laws and provincial customs;\textsuperscript{91} but it was every where admitted, as a
fundamental maxim of religious policy, that no bishop could be imposed
on an orthodox church, without the consent of its members. The
emperors, as the guardians of the public peace, and as the first
citizens of Rome and Constantinople, might effectually declare their
wishes in the choice of a primate; but those absolute monarchs
respected the freedom of ecclesiastical elections; and while they
distributed and resumed the honors of the state and army, they allowed
eighteen hundred perpetual magistrates to receive their important
offices from the free suffrages of the people.\textsuperscript{92} It was agreeable to
the dictates of justice, that these magistrates should not desert an
honorable station from which they could not be removed; but the wisdom
of councils endeavored, without much success, to enforce the residence,
and to prevent the translation, of bishops. The discipline of the West
was indeed less relaxed than that of the East; but the same passions
which made those regulations necessary, rendered them ineffectual. The
reproaches which angry prelates have so vehemently urged against each
other, serve only to expose their common guilt, and their mutual
indiscretion.

\pagenote[88]{Thomassin (Discipline de l’Eglise, tom, ii. l. ii. c.
1-8, p. 673-721) has copiously treated of the election of bishops
during the five first centuries, both in the East and in the West; but
he shows a very partial bias in favor of the episcopal aristocracy.
Bingham, (l. iv. c. 2) is moderate; and Chardon (Hist. des Sacremens
tom. v. p. 108-128) is very clear and concise. * Note: This freedom was
extremely limited, and soon annihilated; already, from the third
century, the deacons were no longer nominated by the members of the
community, but by the bishops. Although it appears by the letters of
Cyprian, that even in his time, no priest could be elected without the
consent of the community. (Ep. 68,) that election was far from being
altogether free. The bishop proposed to his parishioners the candidate
whom he had chosen, and they were permitted to make such objections as
might be suggested by his conduct and morals. (St. Cyprian, Ep. 33.)
They lost this last right towards the middle of the fourth century.—G}

\pagenote[89]{Incredibilis multitudo, non solum ex eo oppido,
(\textit{Tours},) sed etiam ex vicinis urbibus ad suffragia ferenda
convenerat, \&c. Sulpicius Severus, in Vit. Martin. c. 7. The council of
Laodicea, (canon xiii.) prohibits mobs and tumults; and Justinian
confines confined the right of election to the nobility. Novel. cxxiii.
l.}

\pagenote[90]{The epistles of Sidonius Apollinaris (iv. 25, vii. 5, 9)
exhibit some of the scandals of the Gallican church; and Gaul was less
polished and less corrupt than the East.}

\pagenote[91]{A compromise was sometimes introduced by law or by
consent; either the bishops or the people chose one of the three
candidates who had been named by the other party.}

\pagenote[92]{All the examples quoted by Thomassin (Discipline de
l’Eglise, tom. ii. l. iii. c. vi. p. 704-714) appear to be
extraordinary acts of power, and even of oppression. The confirmation
of the bishop of Alexandria is mentioned by Philostorgius as a more
regular proceeding. (Hist Eccles. l. ii. ll.) * Note: The statement of
Planck is more consistent with history: “From the middle of the fourth
century, the bishops of some of the larger churches, particularly those
of the Imperial residence, were almost always chosen under the
influence of the court, and often directly and immediately nominated by
the emperor.” Planck, Geschichte der Christlich-kirchlichen
Gesellschafteverfassung, verfassung, vol. i p 263.—M.}

II. The bishops alone possessed the faculty of \textit{spiritual} generation:
and this extraordinary privilege might compensate, in some degree, for
the painful celibacy\textsuperscript{93} which was imposed as a virtue, as a duty, and
at length as a positive obligation. The religions of antiquity, which
established a separate order of priests, dedicated a holy race, a tribe
or family, to the perpetual service of the gods.\textsuperscript{94} Such institutions
were founded for possession, rather than conquest. The children of the
priests enjoyed, with proud and indolent security, their sacred
inheritance; and the fiery spirit of enthusiasm was abated by the
cares, the pleasures, and the endearments of domestic life. But the
Christian sanctuary was open to every ambitious candidate, who aspired
to its heavenly promises or temporal possessions. This office of
priests, like that of soldiers or magistrates, was strenuously
exercised by those men, whose temper and abilities had prompted them to
embrace the ecclesiastical profession, or who had been selected by a
discerning bishop, as the best qualified to promote the glory and
interest of the church. The bishops\textsuperscript{95} (till the abuse was restrained
by the prudence of the laws) might constrain the reluctant, and protect
the distressed; and the imposition of hands forever bestowed some of
the most valuable privileges of civil society. The whole body of the
Catholic clergy, more numerous perhaps than the legions, was exempted\textsuperscript{95a}
by the emperors from all service, private or public, all
municipal offices, and all personal taxes and contributions, which
pressed on their fellow- citizens with intolerable weight; and the
duties of their holy profession were accepted as a full discharge of
their obligations to the republic.\textsuperscript{96} Each bishop acquired an absolute
and indefeasible right to the perpetual obedience of the clerk whom he
ordained: the clergy of each episcopal church, with its dependent
parishes, formed a regular and permanent society; and the cathedrals of
Constantinople\textsuperscript{97} and Carthage\textsuperscript{98} maintained their peculiar
establishment of five hundred ecclesiastical ministers. Their ranks\textsuperscript{99}
and numbers were insensibly multiplied by the superstition of the
times, which introduced into the church the splendid ceremonies of a
Jewish or Pagan temple; and a long train of priests, deacons,
sub-deacons, acolythes, exorcists, readers, singers, and doorkeepers,
contributed, in their respective stations, to swell the pomp and
harmony of religious worship. The clerical name and privileges were
extended to many pious fraternities, who devoutly supported the
ecclesiastical throne.\textsuperscript{100} Six hundred \textit{parabolani}, or adventurers,
visited the sick at Alexandria; eleven hundred \textit{copiatæ}, or
grave-diggers, buried the dead at Constantinople; and the swarms of
monks, who arose from the Nile, overspread and darkened the face of the
Christian world.

\pagenote[93]{The celibacy of the clergy during the first five or six
centuries, is a subject of discipline, and indeed of controversy, which
has been very diligently examined. See in particular, Thomassin,
Discipline de l’Eglise, tom. i. l. ii. c. lx. lxi. p. 886-902, and
Bingham’s Antiquities, l. iv. c. 5. By each of these learned but
partial critics, one half of the truth is produced, and the other is
concealed.—Note: Compare Planck, (vol. i. p. 348.) This century, the
third, first brought forth the monks, or the spirit of monkery, the
celibacy of the clergy. Planck likewise observes, that from the history
of Eusebius alone, names of married bishops and presbyters may be
adduced by dozens.—M.}

\pagenote[94]{Diodorus Siculus attests and approves the hereditary
succession of the priesthood among the Egyptians, the Chaldeans, and
the Indians, (l. i. p. 84, l. ii. p. 142, 153, edit. Wesseling.) The
magi are described by Ammianus as a very numerous family: “Per sæcula
multa ad præsens unâ eâdemque prosapiâ multitudo creata, Deorum
cultibus dedicata.” (xxiii. 6.) Ausonius celebrates the \textit{Stirps
Druidarum}, (De Professorib. Burdigal. iv.;) but we may infer from the
remark of Cæsar, (vi. 13,) that in the Celtic hierarchy, some room was
left for choice and emulation.}

\pagenote[95]{The subject of the vocation, ordination, obedience, \&c.,
of the clergy, is laboriously discussed by Thomassin (Discipline de
l’Eglise, tom. ii. p. 1-83) and Bingham, (in the 4th book of his
Antiquities, more especially the 4th, 6th, and 7th chapters.) When the
brother of St. Jerom was ordained in Cyprus, the deacons forcibly
stopped his mouth, lest he should make a solemn protestation, which
might invalidate the holy rites.

[This exemption was very much limited. The municipal offices were of
two kinds; the one attached to the individual in his character of
inhabitant, the other in that of \textit{proprietor}. Constantine had
exempted ecclesiastics from offices of the first description. (Cod.
Theod. xvi. t. ii. leg. 1, 2 Eusebius, Hist. Eccles. l. x. c. vii.)
They sought, also, to be exempted from those of the second, (munera
patrimoniorum.) The rich, to obtain this privilege, obtained
subordinate situations among the clergy. Constantine published in 320
an edict, by which he prohibited the more opulent citizens (decuriones
and curiales) from embracing the ecclesiastical profession, and the
bishops from admitting new ecclesiastics, before a place should be
vacant by the death of the occupant, (Godefroy ad Cod. Theod.t. xii.
t. i. de Decur.) Valentinian the First, by a rescript still more
general enacted that no rich citizen should obtain a situation in the
church, (De Episc 1. lxvii.) He also enacted that ecclesiastics, who
wished to be exempt from offices which they were bound to discharge as
proprietors, should be obliged to give up their property to their
relations. Cod Theodos l. xii t. i. leb. 49—G.]}

\pagenote[96]{The charter of immunities, which the clergy obtained from
the Christian emperors, is contained in the 16th book of the Theodosian
code; and is illustrated with tolerable candor by the learned Godefroy,
whose mind was balanced by the opposite prejudices of a civilian and a
Protestant.}

\pagenote[97]{Justinian. Novell. ciii. Sixty presbyters, or priests,
one hundred deacons, forty deaconesses, ninety sub-deacons, one hundred
and ten readers, twenty-five chanters, and one hundred door-keepers; in
all, five hundred and twenty-five. This moderate number was fixed by
the emperor to relieve the distress of the church, which had been
involved in debt and usury by the expense of a much higher
establishment.}

\pagenote[98]{Universus clerus ecclesiæ Carthaginiensis.... fere
\textit{quingenti} vel amplius; inter quos quamplurima erant lectores
infantuli. Victor Vitensis, de Persecut. Vandal. v. 9, p. 78, edit.
Ruinart. This remnant of a more prosperous state still subsisted under
the oppression of the Vandals.}

\pagenote[99]{The number of \textit{seven} orders has been fixed in the Latin
church, exclusive of the episcopal character. But the four inferior
ranks, the minor orders, are now reduced to empty and useless titles.}

\pagenote[100]{See Cod. Theodos. l. xvi. tit. ii. leg. 42, 43.
Godefroy’s Commentary, and the Ecclesiastical History of Alexandria,
show the danger of these pious institutions, which often disturbed the
peace of that turbulent capital.}

\section{Part \thesection.}

III. The edict of Milan secured the revenue as well as the peace of the
church.\textsuperscript{101} The Christians not only recovered the lands and houses of
which they had been stripped by the persecuting laws of Diocletian, but
they acquired a perfect title to all the possessions which they had
hitherto enjoyed by the connivance of the magistrate. As soon as
Christianity became the religion of the emperor and the empire, the
national clergy might claim a decent and honorable maintenance; and the
payment of an annual tax might have delivered the people from the more
oppressive tribute, which superstition imposes on her votaries. But as
the wants and expenses of the church increased with her prosperity, the
ecclesiastical order was still supported and enriched by the voluntary
oblations of the faithful. Eight years after the edict of Milan,
Constantine granted to all his subjects the free and universal
permission of bequeathing their fortunes to the holy Catholic church;\textsuperscript{102}
and their devout liberality, which during their lives was checked
by luxury or avarice, flowed with a profuse stream at the hour of their
death. The wealthy Christians were encouraged by the example of their
sovereign. An absolute monarch, who is rich without patrimony, may be
charitable without merit; and Constantine too easily believed that he
should purchase the favor of Heaven, if he maintained the idle at the
expense of the industrious; and distributed among the saints the wealth
of the republic. The same messenger who carried over to Africa the head
of Maxentius, might be intrusted with an epistle to Cæcilian, bishop of
Carthage. The emperor acquaints him, that the treasurers of the
province are directed to pay into his hands the sum of three thousand
\textit{folles}, or eighteen thousand pounds sterling, and to obey his further
requisitions for the relief of the churches of Africa, Numidia, and
Mauritania.\textsuperscript{103} The liberality of Constantine increased in a just
proportion to his faith, and to his vices. He assigned in each city a
regular allowance of corn, to supply the fund of ecclesiastical
charity; and the persons of both sexes who embraced the monastic life
became the peculiar favorites of their sovereign. The Christian temples
of Antioch, Alexandria, Jerusalem, Constantinople \&c., displayed the
ostentatious piety of a prince, ambitious in a declining age to equal
the perfect labors of antiquity.\textsuperscript{104} The form of these religious
edifices was simple and oblong; though they might sometimes swell into
the shape of a dome, and sometimes branch into the figure of a cross.
The timbers were framed for the most part of cedars of Libanus; the
roof was covered with tiles, perhaps of gilt brass; and the walls, the
columns, the pavement, were encrusted with variegated marbles. The most
precious ornaments of gold and silver, of silk and gems, were profusely
dedicated to the service of the altar; and this specious magnificence
was supported on the solid and perpetual basis of landed property. In
the space of two centuries, from the reign of Constantine to that of
Justinian, the eighteen hundred churches of the empire were enriched by
the frequent and unalienable gifts of the prince and people. An annual
income of six hundred pounds sterling may be reasonably assigned to the
bishops, who were placed at an equal distance between riches and
poverty,\textsuperscript{105} but the standard of their wealth insensibly rose with the
dignity and opulence of the cities which they governed. An authentic
but imperfect\textsuperscript{106} rent-roll specifies some houses, shops, gardens, and
farms, which belonged to the three \textit{Basilicæ} of Rome, St. Peter, St.
Paul, and St. John Lateran, in the provinces of Italy, Africa, and the
East. They produce, besides a reserved rent of oil, linen, paper,
aromatics, \&c., a clear annual revenue of twenty-two thousand pieces of
gold, or twelve thousand pounds sterling. In the age of Constantine and
Justinian, the bishops no longer possessed, perhaps they no longer
deserved, the unsuspecting confidence of their clergy and people. The
ecclesiastical revenues of each diocese were divided into four parts
for the respective uses of the bishop himself, of his inferior clergy,
of the poor, and of the public worship; and the abuse of this sacred
trust was strictly and repeatedly checked.\textsuperscript{107} The patrimony of the
church was still subject to all the public compositions of the state.\textsuperscript{108}
The clergy of Rome, Alexandria, Chessaionica, \&c., might solicit
and obtain some partial exemptions; but the premature attempt of the
great council of Rimini, which aspired to universal freedom, was
successfully resisted by the son of Constantine.\textsuperscript{109}

\pagenote[101]{The edict of Milan (de M. P. c. 48) acknowledges, by
reciting, that there existed a species of landed property, ad jus
corporis eorum, id est, ecclesiarum non hominum singulorum pertinentia.
Such a solemn declaration of the supreme magistrate must have been
received in all the tribunals as a maxim of civil law.}

\pagenote[102]{Habeat unusquisque licentiam sanctissimo Catholicæ
(\textit{ecclesiæ}) venerabilique concilio, decedens bonorum quod optavit
relinquere. Cod. Theodos. l. xvi. tit. ii. leg. 4. This law was
published at Rome, A. D. 321, at a time when Constantine might foresee
the probability of a rupture with the emperor of the East.}

\pagenote[103]{Eusebius, Hist. Eccles. l. x. 6; in Vit. Constantin. l.
iv. c. 28. He repeatedly expatiates on the liberality of the Christian
hero, which the bishop himself had an opportunity of knowing, and even
of lasting.}

\pagenote[104]{Eusebius, Hist. Eccles. l. x. c. 2, 3, 4. The bishop of
Cæsarea who studied and gratified the taste of his master, pronounced
in public an elaborate description of the church of Jerusalem, (in Vit
Cons. l. vi. c. 46.) It no longer exists, but he has inserted in the
life of Constantine (l. iii. c. 36) a short account of the architecture
and ornaments. He likewise mentions the church of the Holy Apostles at
Constantinople, (l. iv. c. 59.)}

\pagenote[105]{See Justinian. Novell. cxxiii. 3. The revenue of the
patriarchs, and the most wealthy bishops, is not expressed: the highest
annual valuation of a bishopric is stated at \textit{thirty}, and the lowest
at \textit{two}, pounds of gold; the medium might be taken at \textit{sixteen}, but
these valuations are much below the real value.}

\pagenote[106]{See Baronius, (Annal. Eccles. A. D. 324, No. 58, 65, 70,
71.) Every record which comes from the Vatican is justly suspected; yet
these rent-rolls have an ancient and authentic color; and it is at
least evident, that, if forged, they were forged in a period when
\textit{farms} not \textit{kingdoms}, were the objects of papal avarice.}

\pagenote[107]{See Thomassin, Discipline de l’Eglise, tom. iii. l. ii.
c. 13, 14, 15, p. 689-706. The legal division of the ecclesiastical
revenue does not appear to have been established in the time of Ambrose
and Chrysostom. Simplicius and Gelasius, who were bishops of Rome in
the latter part of the fifth century, mention it in their pastoral
letters as a general law, which was already confirmed by the custom of
Italy.}

\pagenote[108]{Ambrose, the most strenuous assertor of ecclesiastical
privileges, submits without a murmur to the payment of the land tax.
“Si tri butum petit Imperator, non negamus; agri ecclesiæ solvunt
tributum solvimus quæ sunt Cæsaris Cæsari, et quæ sunt Dei Deo;
tributum Cæsaris est; non negatur.” Baronius labors to interpret this
tribute as an act of charity rather than of duty, (Annal. Eccles. A. D.
387;) but the words, if not the intentions of Ambrose are more candidly
explained by Thomassin, Discipline de l’Eglise, tom. iii. l. i. c. 34.
p. 668.}

\pagenote[109]{In Ariminense synodo super ecclesiarum et clericorum
privilegiis tractatu habito, usque eo dispositio progressa est, ut juqa
quæ viderentur ad ecclesiam pertinere, a publica functione cessarent
inquietudine desistente; quod nostra videtur dudum sanctio repulsisse.
Cod. Theod. l. xvi. tit. ii. leg. 15. Had the synod of Rimini carried
this point, such practical merit might have atoned for some speculative
heresies.}

IV. The Latin clergy, who erected their tribunal on the ruins of the
civil and common law, have modestly accepted, as the gift of
Constantine,\textsuperscript{110} the independent jurisdiction, which was the fruit of
time, of accident, and of their own industry. But the liberality of the
Christian emperors had actually endowed them with some legal
prerogatives, which secured and dignified the sacerdotal character.\textsuperscript{111}
1. Under a despotic government, the bishops alone enjoyed and asserted
the inestimable privilege of being tried only by their \textit{peers}, and
even in a capital accusation, a synod of their brethren were the sole
judges of their guilt or innocence. Such a tribunal, unless it was
inflamed by personal resentment or religious discord, might be
favorable, or even partial, to the sacerdotal order: but Constantine
was satisfied,\textsuperscript{112} that secret impunity would be less pernicious than
public scandal: and the Nicene council was edited by his public
declaration, that if he surprised a bishop in the act of adultery, he
should cast his Imperial mantle over the episcopal sinner. 2. The
domestic jurisdiction of the bishops was at once a privilege and a
restraint of the ecclesiastical order, whose civil causes were decently
withdrawn from the cognizance of a secular judge. Their venial offences
were not exposed to the shame of a public trial or punishment; and the
gentle correction which the tenderness of youth may endure from its
parents or instructors, was inflicted by the temperate severity of the
bishops. But if the clergy were guilty of any crime which could not be
sufficiently expiated by their degradation from an honorable and
beneficial profession, the Roman magistrate drew the sword of justice,
without any regard to ecclesiastical immunities. 3. The arbitration of
the bishops was ratified by a positive law; and the judges were
instructed to execute, without appeal or delay, the episcopal decrees,
whose validity had hitherto depended on the consent of the parties. The
conversion of the magistrates themselves, and of the whole empire,
might gradually remove the fears and scruples of the Christians. But
they still resorted to the tribunal of the bishops, whose abilities and
integrity they esteemed; and the venerable Austin enjoyed the
satisfaction of complaining that his spiritual functions were
perpetually interrupted by the invidious labor of deciding the claim or
the possession of silver and gold, of lands and cattle. 4. The ancient
privilege of sanctuary was transferred to the Christian temples, and
extended, by the liberal piety of the younger Theodosius, to the
precincts of consecrated ground.\textsuperscript{113} The fugitive, and even guilty
suppliants,were permitted to implore either the justice, or the mercy,
of the Deity and his ministers. The rash violence of despotism was
suspended by the mild interposition of the church; and the lives or
fortunes of the most eminent subjects might be protected by the
mediation of the bishop.

\pagenote[110]{From Eusebius (in Vit. Constant. l. iv. c. 27) and
Sozomen (l. i. c. 9) we are assured that the episcopal jurisdiction was
extended and confirmed by Constantine; but the forgery of a famous
edict, which was never fairly inserted in the Theodosian Code (see at
the end, tom. vi. p. 303,) is demonstrated by Godefroy in the most
satisfactory manner. It is strange that M. de Montesquieu, who was a
lawyer as well as a philosopher, should allege this edict of
Constantine (Esprit des Loix, l. xxix. c. 16) without intimating any
suspicion.}

\pagenote[111]{The subject of ecclesiastical jurisdiction has been
involved in a mist of passion, of prejudice, and of interest. Two of
the fairest books which have fallen into my hands, are the Institutes
of Canon Law, by the Abbé de Fleury, and the Civil History of Naples,
by Giannone. Their moderation was the effect of situation as well as of
temper. Fleury was a French ecclesiastic, who respected the authority
of the parliaments; Giannone was an Italian lawyer, who dreaded the
power of the church. And here let me observe, that as the general
propositions which I advance are the result of \textit{many} particular and
imperfect facts, I must either refer the reader to those modern authors
who have expressly treated the subject, or swell these notes
disproportioned size.}

\pagenote[112]{Tillemont has collected from Rufinus, Theodoret, \&c.,
the sentiments and language of Constantine. Mém Eccles tom. iii p. 749,
759.}

\pagenote[113]{See Cod. Theod. l. ix. tit. xlv. leg. 4. In the works of
Fra Paolo. (tom. iv. p. 192, \&c.,) there is an excellent discourse on
the origin, claims, abuses, and limits of sanctuaries. He justly
observes, that ancient Greece might perhaps contain fifteen or twenty
\textit{azyla} or sanctuaries; a number which at present may be found in Italy
within the walls of a single city.}

V. The bishop was the perpetual censor of the morals of his people The
discipline of penance was digested into a system of canonical
jurisprudence,\textsuperscript{114} which accurately defined the duty of private or
public confession, the rules of evidence, the degrees of guilt, and the
measure of punishment. It was impossible to execute this spiritual
censure, if the Christian pontiff, who punished the obscure sins of the
multitude, respected the conspicuous vices and destructive crimes of
the magistrate: but it was impossible to arraign the conduct of the
magistrate, without, controlling the administration of civil
government. Some considerations of religion, or loyalty, or fear,
protected the sacred persons of the emperors from the zeal or
resentment of the bishops; but they boldly censured and excommunicated
the subordinate tyrants, who were not invested with the majesty of the
purple. St. Athanasius excommunicated one of the ministers of Egypt;
and the interdict which he pronounced, of fire and water, was solemnly
transmitted to the churches of Cappadocia.\textsuperscript{115} Under the reign of the
younger Theodosius, the polite and eloquent Synesius, one of the
descendants of Hercules,\textsuperscript{116} filled the episcopal seat of Ptolemais,
near the ruins of ancient Cyrene,\textsuperscript{117} and the philosophic bishop
supported with dignity the character which he had assumed with
reluctance.\textsuperscript{118} He vanquished the monster of Libya, the president
Andronicus, who abused the authority of a venal office, invented new
modes of rapine and torture, and aggravated the guilt of oppression by
that of sacrilege.\textsuperscript{119} After a fruitless attempt to reclaim the haughty
magistrate by mild and religious admonition, Synesius proceeds to
inflict the last sentence of ecclesiastical justice,\textsuperscript{120} which devotes
Andronicus, with his associates and their \textit{families}, to the abhorrence
of earth and heaven. The impenitent sinners, more cruel than Phalaris
or Sennacherib, more destructive than war, pestilence, or a cloud of
locusts, are deprived of the name and privileges of Christians, of the
participation of the sacraments, and of the hope of Paradise. The
bishop exhorts the clergy, the magistrates, and the people, to renounce
all society with the enemies of Christ; to exclude them from their
houses and tables; and to refuse them the common offices of life, and
the decent rites of burial. The church of Ptolemais, obscure and
contemptible as she may appear, addresses this declaration to all her
sister churches of the world; and the profane who reject her decrees,
will be involved in the guilt and punishment of Andronicus and his
impious followers. These spiritual terrors were enforced by a dexterous
application to the Byzantine court; the trembling president implored
the mercy of the church; and the descendants of Hercules enjoyed the
satisfaction of raising a prostrate tyrant from the ground.\textsuperscript{121} Such
principles and such examples insensibly prepared the triumph of the
Roman pontiffs, who have trampled on the necks of kings.

\pagenote[114]{The penitential jurisprudence was continually improved
by the canons of the councils. But as many cases were still left to the
discretion of the bishops, they occasionally published, after the
example of the Roman Prætor, the rules of discipline which they
proposed to observe. Among the canonical epistles of the fourth
century, those of Basil the Great were the most celebrated. They are
inserted in the Pandects of Beveridge, (tom. ii. p. 47-151,) and are
translated by Chardon, Hist. des Sacremens, tom. iv. p. 219-277.}

\pagenote[115]{Basil, Epistol. xlvii. in Baronius, (Annal. Eccles. A.
D. 370. N. 91,) who declares that he purposely relates it, to convince
govern that they were not exempt from a sentence of excommunication his
opinion, even a royal head is not safe from the thunders of the
Vatican; and the cardinal shows himself much more consistent than the
lawyers and theologians of the Gallican church.}

\pagenote[116]{The long series of his ancestors, as high as
Eurysthenes, the first Doric king of Sparta, and the fifth in lineal
descent from Hercules, was inscribed in the public registers of Cyrene,
a Lacedæmonian colony. (Synes. Epist. lvii. p. 197, edit. Petav.) Such
a pure and illustrious pedigree of seventeen hundred years, without
adding the royal ancestors of Hercules, cannot be equalled in the
history of mankind.}

\pagenote[117]{Synesius (de Regno, p. 2) pathetically deplores the
fallen and ruined state of Cyrene, [**Greek]. Ptolemais, a new city, 82
miles to the westward of Cyrene, assumed the metropolitan honors of the
Pentapolis, or Upper Libya, which were afterwards transferred to
Sozusa.}

\pagenote[118]{Synesius had previously represented his own
disqualifications. He loved profane studies and profane sports; he was
incapable of supporting a life of celibacy; he disbelieved the
resurrection; and he refused to preach \textit{fables} to the people unless he
might be permitted to \textit{philosophize} at home. Theophilus primate of
Egypt, who knew his merit, accepted this extraordinary compromise.}

\pagenote[119]{The promotion of Andronicus was illegal; since he was a
native of Berenice, in the same province. The instruments of torture
are curiously specified; the press that variously pressed on distended
the fingers, the feet, the nose, the ears, and the lips of the
victims.}

\pagenote[120]{The sentence of excommunication is expressed in a
rhetorical style. (Synesius, Epist. lviii. p. 201-203.) The method of
involving whole families, though somewhat unjust, was improved into
national interdicts.}

\pagenote[121]{See Synesius, Epist. xlvii. p. 186, 187. Epist. lxxii.
p. 218, 219 Epist. lxxxix. p. 230, 231.}

VI. Every popular government has experienced the effects of rude or
artificial eloquence. The coldest nature is animated, the firmest
reason is moved, by the rapid communication of the prevailing impulse;
and each hearer is affected by his own passions, and by those of the
surrounding multitude. The ruin of civil liberty had silenced the
demagogues of Athens, and the tribunes of Rome; the custom of preaching
which seems to constitute a considerable part of Christian devotion,
had not been introduced into the temples of antiquity; and the ears of
monarchs were never invaded by the harsh sound of popular eloquence,
till the pulpits of the empire were filled with sacred orators, who
possessed some advantages unknown to their profane predecessors.\textsuperscript{122}
The arguments and rhetoric of the tribune were instantly opposed with
equal arms, by skilful and resolute antagonists; and the cause of truth
and reason might derive an accidental support from the conflict of
hostile passions. The bishop, or some distinguished presbyter, to whom
he cautiously delegated the powers of preaching, harangued, without the
danger of interruption or reply, a submissive multitude, whose minds
had been prepared and subdued by the awful ceremonies of religion. Such
was the strict subordination of the Catholic church, that the same
concerted sounds might issue at once from a hundred pulpits of Italy or
Egypt, if they were \textit{tuned}\textsuperscript{123} by the master hand of the Roman or
Alexandrian primate. The design of this institution was laudable, but
the fruits were not always salutary. The preachers recommended the
practice of the social duties; but they exalted the perfection of
monastic virtue, which is painful to the individual, and useless to
mankind. Their charitable exhortations betrayed a secret wish that the
clergy might be permitted to manage the wealth of the faithful, for the
benefit of the poor. The most sublime representations of the attributes
and laws of the Deity were sullied by an idle mixture of metaphysical
subleties, puerile rites, and fictitious miracles: and they expatiated,
with the most fervent zeal, on the religious merit of hating the
adversaries, and obeying the ministers of the church. When the public
peace was distracted by heresy and schism, the sacred orators sounded
the trumpet of discord, and, perhaps, of sedition. The understandings
of their congregations were perplexed by mystery, their passions were
inflamed by invectives; and they rushed from the Christian temples of
Antioch or Alexandria, prepared either to suffer or to inflict
martyrdom. The corruption of taste and language is strongly marked in
the vehement declamations of the Latin bishops; but the compositions of
Gregory and Chrysostom have been compared with the most splendid models
of Attic, or at least of Asiatic, eloquence.\textsuperscript{124}

\pagenote[122]{See Thomassin (Discipline de l’Eglise, tom. ii. l. iii.
c. 83, p. 1761-1770,) and Bingham, (Antiquities, vol. i. l. xiv. c. 4,
p. 688- 717.) Preaching was considered as the most important office of
the bishop but this function was sometimes intrusted to such presbyters
as Chrysostom and Augustin.}

\pagenote[123]{Queen Elizabeth used this expression, and practised this
art whenever she wished to prepossess the minds of her people in favor
of any extraordinary measure of government. The hostile effects of this
\textit{music} were apprehended by her successor, and severely felt by his
son. “When pulpit, drum ecclesiastic,” \&c. See Heylin’s Life of
Archbishop Laud, p. 153.}

\pagenote[124]{Those modest orators acknowledged, that, as they were
destitute of the gift of miracles, they endeavored to acquire the arts
of eloquence.}

VII. The representatives of the Christian republic were regularly
assembled in the spring and autumn of each year; and these synods
diffused the spirit of ecclesiastical discipline and legislation
through the hundred and twenty provinces of the Roman world.\textsuperscript{125} The
archbishop or metropolitan was empowered, by the laws, to summon the
suffragan bishops of his province; to revise their conduct, to
vindicate their rights, to declare their faith, and to examine the
merits of the candidates who were elected by the clergy and people to
supply the vacancies of the episcopal college. The primates of Rome,
Alexandria, Antioch, Carthage, and afterwards Constantinople, who
exercised a more ample jurisdiction, convened the numerous assembly of
their dependent bishops. But the convocation of great and extraordinary
synods was the prerogative of the emperor alone. Whenever the
emergencies of the church required this decisive measure, he despatched
a peremptory summons to the bishops, or the deputies of each province,
with an order for the use of post-horses, and a competent allowance for
the expenses of their journey. At an early period, when Constantine was
the protector, rather than the proselyte, of Christianity, he referred
the African controversy to the council of Arles; in which the bishops
of York of Trèves, of Milan, and of Carthage, met as friends and
brethren, to debate in their native tongue on the common interest of
the Latin or Western church.\textsuperscript{126} Eleven years afterwards, a more
numerous and celebrated assembly was convened at Nice in Bithynia, to
extinguish, by their final sentence, the subtle disputes which had
arisen in Egypt on the subject of the Trinity. Three hundred and
eighteen bishops obeyed the summons of their indulgent master; the
ecclesiastics of every rank, and sect, and denomination, have been
computed at two thousand and forty-eight persons;\textsuperscript{127} the Greeks
appeared in person; and the consent of the Latins was expressed by the
legates of the Roman pontiff. The session, which lasted about two
months, was frequently honored by the presence of the emperor. Leaving
his guards at the door, he seated himself (with the permission of the
council) on a low stool in the midst of the hall. Constantine listened
with patience, and spoke with modesty: and while he influenced the
debates, he humbly professed that he was the minister, not the judge,
of the successors of the apostles, who had been established as priests
and as gods upon earth.\textsuperscript{128} Such profound reverence of an absolute
monarch towards a feeble and unarmed assembly of his own subjects, can
only be compared to the respect with which the senate had been treated
by the Roman princes who adopted the policy of Augustus. Within the
space of fifty years, a philosophic spectator of the vicissitudes of
human affairs might have contemplated Tacitus in the senate of Rome,
and Constantine in the council of Nice. The fathers of the Capitol and
those of the church had alike degenerated from the virtues of their
founders; but as the bishops were more deeply rooted in the public
opinion, they sustained their dignity with more decent pride, and
sometimes opposed with a manly spirit the wishes of their sovereign.
The progress of time and superstition erased the memory of the
weakness, the passion, the ignorance, which disgraced these
ecclesiastical synods; and the Catholic world has unanimously submitted\textsuperscript{129}
to the \textit{infallible} decrees of the general councils.\textsuperscript{130}

\pagenote[125]{The council of Nice, in the fourth, fifth, sixth, and
seventh canons, has made some fundamental regulations concerning
synods, metropolitan, and primates. The Nicene canons have been
variously tortured, abused, interpolated, or forged, according to the
interest of the clergy. The \textit{Suburbicarian} churches, assigned (by
Rufinus) to the bishop of Rome, have been made the subject of vehement
controversy (See Sirmond, Opera, tom. iv. p. 1-238.)}

\pagenote[126]{We have only thirty-three or forty-seven episcopal
subscriptions: but Addo, a writer indeed of small account, reckons six
hundred bishops in the council of Arles. Tillemont, Mém. Eccles. tom.
vi. p. 422.}

\pagenote[127]{See Tillemont, tom. vi. p. 915, and Beausobre, Hist. du
Mani cheisme, tom i p. 529. The name of \textit{bishop}, which is given by
Eusychius to the 2048 ecclesiastics, (Annal. tom. i. p. 440, vers.
Pocock,) must be extended far beyond the limits of an orthodox or even
episcopal ordination.}

\pagenote[128]{See Euseb. in Vit. Constantin. l. iii. c. 6-21.
Tillemont, Mém. Ecclésiastiques, tom. vi. p. 669-759.}

\pagenote[129]{Sancimus igitur vicem legum obtinere, quæ a quatuor
Sanctis Conciliis.... expositæ sunt act firmatæ. Prædictarum enim quat
uor synodorum dogmata sicut sanctas Scripturas et regulas sicut leges
observamus. Justinian. Novell. cxxxi. Beveridge (ad Pandect. proleg. p.
2) remarks, that the emperors never made new laws in ecclesiastical
matters; and Giannone observes, in a very different spirit, that they
gave a legal sanction to the canons of councils. Istoria Civile di
Napoli, tom. i. p. 136.}

\pagenote[130]{See the article Concile in the Eucyclopedie, tom. iii.
p. 668-879, edition de Lucques. The author, M. de docteur Bouchaud, has
discussed, according to the principles of the Gallican church, the
principal questions which relate to the form and constitution of
general, national, and provincial councils. The editors (see Preface,
p. xvi.) have reason to be proud of \textit{this} article. Those who consult
their immense compilation, seldom depart so well satisfied.}

