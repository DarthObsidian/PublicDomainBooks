\chapter{Constantius Sole Emperor.}
\section{Part \thesection.}

\textit{Constantius Sole Emperor. — Elevation And Death Of Gallus. — Danger
And Elevation Of Julian. — Sarmatian And Persian Wars. — Victories Of
Julian In Gaul.}
\vspace{\onelineskip}

The divided provinces of the empire were again united by the
victory of Constantius; but as that feeble prince was destitute
of personal merit, either in peace or war; as he feared his
generals, and distrusted his ministers; the triumph of his arms
served only to establish the reign of the \textit{eunuchs} over the
Roman world. Those unhappy beings, the ancient production of
Oriental jealousy and despotism,\textsuperscript[1] were introduced into Greece
and Rome by the contagion of Asiatic luxury.\textsuperscript[2] Their progress was
rapid; and the eunuchs, who, in the time of Augustus, had been
abhorred, as the monstrous retinue of an Egyptian queen,\textsuperscript[3] were
gradually admitted into the families of matrons, of senators, and
of the emperors themselves.\textsuperscript[4] Restrained by the severe edicts of
Domitian and Nerva, cherished by the pride of Diocletian, reduced
to an humble station by the prudence of Constantine,\textsuperscript[6] they
multiplied in the palaces of his degenerate sons, and insensibly
acquired the knowledge, and at length the direction, of the
secret councils of Constantius. The aversion and contempt which
mankind had so uniformly entertained for that imperfect species,
appears to have degraded their character, and to have rendered
them almost as incapable as they were supposed to be, of
conceiving any generous sentiment, or of performing any worthy
action.\textsuperscript[7] But the eunuchs were skilled in the arts of flattery
and intrigue; and they alternately governed the mind of
Constantius by his fears, his indolence, and his vanity.\textsuperscript[8] Whilst
he viewed in a deceitful mirror the fair appearance of public
prosperity, he supinely permitted them to intercept the
complaints of the injured provinces, to accumulate immense
treasures by the sale of justice and of honors; to disgrace the
most important dignities, by the promotion of those who had
purchased at their hands the powers of oppression,\textsuperscript[9] and to
gratify their resentment against the few independent spirits, who
arrogantly refused to solicit the protection of slaves. Of these
slaves the most distinguished was the chamberlain Eusebius, who
ruled the monarch and the palace with such absolute sway, that
Constantius, according to the sarcasm of an impartial historian,
possessed some credit with this haughty favorite.\textsuperscript[10] By his
artful suggestions, the emperor was persuaded to subscribe the
condemnation of the unfortunate Gallus, and to add a new crime to
the long list of unnatural murders which pollute the honor of the
house of Constantine.

\pagenote[1]{Ammianus (l. xiv. c. 6) imputes the first practice
of castration to the cruel ingenuity of Semiramis, who is
supposed to have reigned above nineteen hundred years before
Christ. The use of eunuchs is of high antiquity, both in Asia and
Egypt. They are mentioned in the law of Moses, Deuteron. xxxiii.
1. See Goguet, Origines des Loix, \&c., Part i. l. i. c. 3.}

\pagenote[2]{Eunuchum dixti velle te; Quia solæ utuntur his
reginæ—Terent. Eunuch. act i. scene 2. This play is translated
from Meander, and the original must have appeared soon after the
eastern conquests of Alexander.}

\pagenote[3]{Miles.... spadonibus Servire rugosis potest. Horat.
Carm. v. 9, and Dacier ad loe. By the word \textit{spado}, the Romans
very forcibly expressed their abhorrence of this mutilated
condition. The Greek appellation of eunuchs, which insensibly
prevailed, had a milder sound, and a more ambiguous sense.}

\pagenote[4]{We need only mention Posides, a freedman and eunuch
of Claudius, in whose favor the emperor prostituted some of the
most honorable rewards of military valor. See Sueton. in Claudio,
c. 28. Posides employed a great part of his wealth in building.

[Ut \textit{Spado} vincebat Capitolia Nostra Posides. Juvenal. Sat. xiv.

Castrari mares vetuit. Sueton. in Domitian. c. 7. See Dion
Cassius, l. lxvii. p. 1107, l. lxviii. p. 1119.}

\pagenote[6]{There is a passage in the Augustan History, p. 137,
in which Lampridius, whilst he praises Alexander Severus and
Constantine for restraining the tyranny of the eunuchs, deplores
the mischiefs which they occasioned in other reigns. Huc accedit
quod eunuchos nec in consiliis nec in ministeriis habuit; qui
soli principes perdunt, dum eos more gentium aut regum Persarum
volunt vivere; qui a populo etiam amicissimum semovent; qui
internuntii sunt, aliud quam respondetur, referentes; claudentes
principem suum, et agentes ante omnia ne quid sciat.}

\pagenote[7]{Xenophon (Cyropædia, l. viii. p. 540) has stated the
specious reasons which engaged Cyrus to intrust his person to the
guard of eunuchs. He had observed in animals, that although the
practice of castration might tame their ungovernable fierceness,
it did not diminish their strength or spirit; and he persuaded
himself, that those who were separated from the rest of human
kind, would be more firmly attached to the person of their
benefactor. But a long experience has contradicted the judgment
of Cyrus. Some particular instances may occur of eunuchs
distinguished by their fidelity, their valor, and their
abilities; but if we examine the general history of Persia,
India, and China, we shall find that the power of the eunuchs has
uniformly marked the decline and fall of every dynasty.}

\pagenote[8]{See Ammianus Marcellinus, l. xxi. c. 16, l. xxii. c.
4. The whole tenor of his impartial history serves to justify the
invectives of Mamertinus, of Libanius, and of Julian himself, who
have insulted the vices of the court of Constantius.}

\pagenote[9]{Aurelius Victor censures the negligence of his
sovereign in choosing the governors of the provinces, and the
generals of the army, and concludes his history with a very bold
observation, as it is much more dangerous under a feeble reign to
attack the ministers than the master himself. “Uti verum absolvam
brevi, ut Imperatore ipso clarius ita apparitorum plerisque magis
atrox nihil.”}

\pagenote[10]{Apud quem (si vere dici debeat) multum Constantius
potuit. Ammian. l. xviii. c. 4.}

When the two nephews of Constantine, Gallus and Julian, were
saved from the fury of the soldiers, the former was about twelve,
and the latter about six, years of age; and, as the eldest was
thought to be of a sickly constitution, they obtained with the
less difficulty a precarious and dependent life, from the
affected pity of Constantius, who was sensible that the execution
of these helpless orphans would have been esteemed, by all
mankind, an act of the most deliberate cruelty.\textsuperscript[11] Different
cities of Ionia and Bithynia were assigned for the places of
their exile and education; but as soon as their growing years
excited the jealousy of the emperor, he judged it more prudent to
secure those unhappy youths in the strong castle of Macellum,
near Cæsarea. The treatment which they experienced during a six
years’ confinement, was partly such as they could hope from a
careful guardian, and partly such as they might dread from a
suspicious tyrant.\textsuperscript[12] Their prison was an ancient palace, the
residence of the kings of Cappadocia; the situation was pleasant,
the buildings stately, the enclosure spacious. They pursued their
studies, and practised their exercises, under the tuition of the
most skilful masters; and the numerous household appointed to
attend, or rather to guard, the nephews of Constantine, was not
unworthy of the dignity of their birth. But they could not
disguise to themselves that they were deprived of fortune, of
freedom, and of safety; secluded from the society of all whom
they could trust or esteem, and condemned to pass their
melancholy hours in the company of slaves devoted to the commands
of a tyrant who had already injured them beyond the hope of
reconciliation. At length, however, the emergencies of the state
compelled the emperor, or rather his eunuchs, to invest Gallus,
in the twenty-fifth year of his age, with the title of Cæsar, and
to cement this political connection by his marriage with the
princess Constantina. After a formal interview, in which the two
princes mutually engaged their faith never to undertake any thing
to the prejudice of each other, they repaired without delay to
their respective stations. Constantius continued his march
towards the West, and Gallus fixed his residence at Antioch; from
whence, with a delegated authority, he administered the five
great dioceses of the eastern præfecture.\textsuperscript[13] In this fortunate
change, the new Cæsar was not unmindful of his brother Julian,
who obtained the honors of his rank, the appearances of liberty,
and the restitution of an ample patrimony.\textsuperscript[14]

\pagenote[11]{Gregory Nazianzen (Orat. iii. p. 90) reproaches the
apostate with his ingratitude towards Mark, bishop of Arethusa,
who had contributed to save his life; and we learn, though from a
less respectable authority, (Tillemont, Hist. des Empereurs, tom.
iv. p. 916,) that Julian was concealed in the sanctuary of a
church. * Note: Gallus and Julian were not sons of the same
mother. Their father, Julius Constantius, had had Gallus by his
first wife, named Galla: Julian was the son of Basilina, whom he
had espoused in a second marriage. Tillemont. Hist. des Emp. Vie
de Constantin. art. 3.—G.}

\pagenote[12]{The most authentic account of the education and
adventures of Julian is contained in the epistle or manifesto
which he himself addressed to the senate and people of Athens.
Libanius, (Orat. Parentalis,) on the side of the Pagans, and
Socrates, (l. iii. c. 1,) on that of the Christians, have
preserved several interesting circumstances.}

\pagenote[13]{For the promotion of Gallus, see Idatius, Zosimus,
and the two Victors. According to Philostorgius, (l. iv. c. 1,)
Theophilus, an Arian bishop, was the witness, and, as it were,
the guarantee of this solemn engagement. He supported that
character with generous firmness; but M. de Tillemont (Hist. des
Empereurs, tom. iv. p. 1120) thinks it very improbable that a
heretic should have possessed such virtue.}

\pagenote[14]{Julian was at first permitted to pursue his studies
at Constantinople, but the reputation which he acquired soon
excited the jealousy of Constantius; and the young prince was
advised to withdraw himself to the less conspicuous scenes of
Bithynia and Ionia.}

The writers the most indulgent to the memory of Gallus, and even
Julian himself, though he wished to cast a veil over the
frailties of his brother, are obliged to confess that the Cæsar
was incapable of reigning. Transported from a prison to a throne,
he possessed neither genius nor application, nor docility to
compensate for the want of knowledge and experience. A temper
naturally morose and violent, instead of being corrected, was
soured by solitude and adversity; the remembrance of what he had
endured disposed him to retaliation rather than to sympathy; and
the ungoverned sallies of his rage were often fatal to those who
approached his person, or were subject to his power.\textsuperscript[15]
Constantina, his wife, is described, not as a woman, but as one
of the infernal furies tormented with an insatiate thirst of
human blood.\textsuperscript[16] Instead of employing her influence to insinuate
the mild counsels of prudence and humanity, she exasperated the
fierce passions of her husband; and as she retained the vanity,
though she had renounced, the gentleness of her sex, a pearl
necklace was esteemed an equivalent price for the murder of an
innocent and virtuous nobleman.\textsuperscript[17] The cruelty of Gallus was
sometimes displayed in the undissembled violence of popular or
military executions; and was sometimes disguised by the abuse of
law, and the forms of judicial proceedings. The private houses of
Antioch, and the places of public resort, were besieged by spies
and informers; and the Cæsar himself, concealed in a a plebeian
habit, very frequently condescended to assume that odious
character. Every apartment of the palace was adorned with the
instruments of death and torture, and a general consternation was
diffused through the capital of Syria. The prince of the East, as
if he had been conscious how much he had to fear, and how little
he deserved to reign, selected for the objects of his resentment
the provincials accused of some imaginary treason, and his own
courtiers, whom with more reason he suspected of incensing, by
their secret correspondence, the timid and suspicious mind of
Constantius. But he forgot that he was depriving himself of his
only support, the affection of the people; whilst he furnished
the malice of his enemies with the arms of truth, and afforded
the emperor the fairest pretence of exacting the forfeit of his
purple, and of his life.\textsuperscript[18]

\pagenote[15]{See Julian. ad S. P. Q. A. p. 271. Jerom. in Chron.
Aurelius Victor, Eutropius, x. 14. I shall copy the words of
Eutropius, who wrote his abridgment about fifteen years after the
death of Gallus, when there was no longer any motive either to
flatter or to depreciate his character. “Multis incivilibus
gestis Gallus Cæsar.... vir natura ferox et ad tyrannidem
pronior, si suo jure imperare licuisset.”}

\pagenote[16]{Megæra quidem mortalis, inflammatrix sævientis
assidua, humani cruoris avida, \&c. Ammian. Marcellin. l. xiv. c.
1. The sincerity of Ammianus would not suffer him to misrepresent
facts or characters, but his love of \textit{ambitious} ornaments
frequently betrayed him into an unnatural vehemence of
expression.}

\pagenote[17]{His name was Clematius of Alexandria, and his only
crime was a refusal to gratify the desires of his mother-in-law;
who solicited his death, because she had been disappointed of his
love. Ammian. xiv. c. i.}

\pagenote[18]{See in Ammianus (l. xiv. c. 1, 7) a very ample
detail of the cruelties of Gallus. His brother Julian (p. 272)
insinuates, that a secret conspiracy had been formed against him;
and Zosimus names (l. ii. p. 135) the persons engaged in it; a
minister of considerable rank, and two obscure agents, who were
resolved to make their fortune.}

As long as the civil war suspended the fate of the Roman world,
Constantius dissembled his knowledge of the weak and cruel
administration to which his choice had subjected the East; and
the discovery of some assassins, secretly despatched to Antioch
by the tyrant of Gaul, was employed to convince the public, that
the emperor and the Cæsar were united by the same interest, and
pursued by the same enemies.\textsuperscript[19] But when the victory was decided
in favor of Constantius, his dependent colleague became less
useful and less formidable. Every circumstance of his conduct was
severely and suspiciously examined, and it was privately
resolved, either to deprive Gallus of the purple, or at least to
remove him from the indolent luxury of Asia to the hardships and
dangers of a German war. The death of Theophilus, consular of the
province of Syria, who in a time of scarcity had been massacred
by the people of Antioch, with the connivance, and almost at the
instigation, of Gallus, was justly resented, not only as an act
of wanton cruelty, but as a dangerous insult on the supreme
majesty of Constantius. Two ministers of illustrious rank,
Domitian the Oriental præfect, and Montius, quæstor of the
palace, were empowered by a special commission\textsuperscript[1911] to visit and
reform the state of the East. They were instructed to behave
towards Gallus with moderation and respect, and, by the gentlest
arts of persuasion, to engage him to comply with the invitation
of his brother and colleague. The rashness of the præfect
disappointed these prudent measures, and hastened his own ruin,
as well as that of his enemy. On his arrival at Antioch, Domitian
passed disdainfully before the gates of the palace, and alleging
a slight pretence of indisposition, continued several days in
sullen retirement, to prepare an inflammatory memorial, which he
transmitted to the Imperial court. Yielding at length to the
pressing solicitations of Gallus, the præfect condescended to
take his seat in council; but his first step was to signify a
concise and haughty mandate, importing that the Cæsar should
immediately repair to Italy, and threatening that he himself
would punish his delay or hesitation, by suspending the usual
allowance of his household. The nephew and daughter of
Constantine, who could ill brook the insolence of a subject,
expressed their resentment by instantly delivering Domitian to
the custody of a guard. The quarrel still admitted of some terms
of accommodation. They were rendered impracticable by the
imprudent behavior of Montius, a statesman whose arts and
experience were frequently betrayed by the levity of his
disposition.\textsuperscript[20] The quæstor reproached Gallus in a haughty
language, that a prince who was scarcely authorized to remove a
municipal magistrate, should presume to imprison a Prætorian
præfect; convoked a meeting of the civil and military officers;
and required them, in the name of their sovereign, to defend the
person and dignity of his representatives. By this rash
declaration of war, the impatient temper of Gallus was provoked
to embrace the most desperate counsels. He ordered his guards to
stand to their arms, assembled the populace of Antioch, and
recommended to their zeal the care of his safety and revenge. His
commands were too fatally obeyed. They rudely seized the præfect
and the quæstor, and tying their legs together with ropes, they
dragged them through the streets of the city, inflicted a
thousand insults and a thousand wounds on these unhappy victims,
and at last precipitated their mangled and lifeless bodies into
the stream of the Orontes.\textsuperscript[21]

\pagenote[19]{Zonaras, l. xiii. tom. ii. p. 17, 18. The assassins
had seduced a great number of legionaries; but their designs were
discovered and revealed by an old woman in whose cottage they
lodged.}

\pagenote[1911]{The commission seems to have been granted to
Domitian alone. Montius interfered to support his authority. Amm.
Marc. loc. cit.—M}

\pagenote[20]{In the present text of Ammianus, we read \textit{Asper},
quidem, sed ad \textit{lenitatem} propensior; which forms a sentence of
contradictory nonsense. With the aid of an old manuscript,
Valesius has rectified the first of these corruptions, and we
perceive a ray of light in the substitution of the word \textit{vafer}.
If we venture to change \textit{lenitatem} into \textit{levitatem}, this
alteration of a single letter will render the whole passage clear
and consistent.}

\pagenote[21]{Instead of being obliged to collect scattered and
imperfect hints from various sources, we now enter into the full
stream of the history of Ammianus, and need only refer to the
seventh and ninth chapters of his fourteenth book. Philostorgius,
however, (l. iii. c. 28) though partial to Gallus, should not be
entirely overlooked.}

After such a deed, whatever might have been the designs of
Gallus, it was only in a field of battle that he could assert his
innocence with any hope of success. But the mind of that prince
was formed of an equal mixture of violence and weakness. Instead
of assuming the title of Augustus, instead of employing in his
defence the troops and treasures of the East, he suffered himself
to be deceived by the affected tranquillity of Constantius, who,
leaving him the vain pageantry of a court, imperceptibly recalled
the veteran legions from the provinces of Asia. But as it still
appeared dangerous to arrest Gallus in his capital, the slow and
safer arts of dissimulation were practised with success. The
frequent and pressing epistles of Constantius were filled with
professions of confidence and friendship; exhorting the Cæsar to
discharge the duties of his high station, to relieve his
colleague from a part of the public cares, and to assist the West
by his presence, his counsels, and his arms. After so many
reciprocal injuries, Gallus had reason to fear and to distrust.
But he had neglected the opportunities of flight and of
resistance; he was seduced by the flattering assurances of the
tribune Scudilo, who, under the semblance of a rough soldier,
disguised the most artful insinuation; and he depended on the
credit of his wife Constantina, till the unseasonable death of
that princess completed the ruin in which he had been involved by
her impetuous passions.\textsuperscript[22]

\pagenote[22]{She had preceded her husband, but died of a fever
on the road at a little place in Bithynia, called Coenum
Gallicanum.}

\section{Part \thesection.}

After a long delay, the reluctant Cæsar set forwards on his
journey to the Imperial court. From Antioch to Hadrianople, he
traversed the wide extent of his dominions with a numerous and
stately train; and as he labored to conceal his apprehensions
from the world, and perhaps from himself, he entertained the
people of Constantinople with an exhibition of the games of the
circus. The progress of the journey might, however, have warned
him of the impending danger. In all the principal cities he was
met by ministers of confidence, commissioned to seize the offices
of government, to observe his motions, and to prevent the hasty
sallies of his despair. The persons despatched to secure the
provinces which he left behind, passed him with cold salutations,
or affected disdain; and the troops, whose station lay along the
public road, were studiously removed on his approach, lest they
might be tempted to offer their swords for the service of a civil
war.\textsuperscript[23] After Gallus had been permitted to repose himself a few
days at Hadrianople, he received a mandate, expressed in the most
haughty and absolute style, that his splendid retinue should halt
in that city, while the Cæsar himself, with only ten
post-carriages, should hasten to the Imperial residence at Milan.

In this rapid journey, the profound respect which was due to the
brother and colleague of Constantius, was insensibly changed into
rude familiarity; and Gallus, who discovered in the countenances
of the attendants that they already considered themselves as his
guards, and might soon be employed as his executioners, began to
accuse his fatal rashness, and to recollect, with terror and
remorse, the conduct by which he had provoked his fate. The
dissimulation which had hitherto been preserved, was laid aside
at Petovio,\textsuperscript[2311] in Pannonia. He was conducted to a palace in the
suburbs, where the general Barbatio, with a select band of
soldiers, who could neither be moved by pity, nor corrupted by
rewards, expected the arrival of his illustrious victim. In the
close of the evening he was arrested, ignominiously stripped of
the ensigns of Cæsar, and hurried away to Pola, [23b] in Istria,
a sequestered prison, which had been so recently polluted with
royal blood. The horror which he felt was soon increased by the
appearance of his implacable enemy the eunuch Eusebius, who, with
the assistance of a notary and a tribune, proceeded to
interrogate him concerning the administration of the East. The
Cæsar sank under the weight of shame and guilt, confessed all the
criminal actions and all the treasonable designs with which he
was charged; and by imputing them to the advice of his wife,
exasperated the indignation of Constantius, who reviewed with
partial prejudice the minutes of the examination. The emperor was
easily convinced, that his own safety was incompatible with the
life of his cousin: the sentence of death was signed, despatched,
and executed; and the nephew of Constantine, with his hands tied
behind his back, was beheaded in prison like the vilest
malefactor.\textsuperscript[24] Those who are inclined to palliate the cruelties
of Constantius, assert that he soon relented, and endeavored to
recall the bloody mandate; but that the second messenger,
intrusted with the reprieve, was detained by the eunuchs, who
dreaded the unforgiving temper of Gallus, and were desirous of
reuniting to \textit{their} empire the wealthy provinces of the East.\textsuperscript[25]

\pagenote[23]{The Thebæan legions, which were then quartered at
Hadrianople, sent a deputation to Gallus, with a tender of their
services. Ammian. l. xiv. c. 11. The Notitia (s. 6, 20, 38, edit.
Labb.) mentions three several legions which bore the name of
Thebæan. The zeal of M. de Voltaire to destroy a despicable
though celebrated legion, has tempted him on the slightest
grounds to deny the existence of a Thebæan legion in the Roman
armies. See Œuvres de Voltaire, tom. xv. p. 414, quarto edition.}

\pagenote[2311]{Pettau in Styria.—M ---- Rather to Flanonia. now
Fianone, near Pola. St. Martin.—M.}

\pagenote[24]{See the complete narrative of the journey and death
of Gallus in Ammianus, l. xiv. c. 11. Julian complains that his
brother was put to death without a trial; attempts to justify, or
at least to excuse, the cruel revenge which he had inflicted on
his enemies; but seems at last to acknowledge that he might
justly have been deprived of the purple.}

\pagenote[25]{Philostorgius, l. iv. c. 1. Zonaras, l. xiii. tom.
ii. p. 19. But the former was partial towards an Arian monarch,
and the latter transcribed, without choice or criticism, whatever
he found in the writings of the ancients.}

Besides the reigning emperor, Julian alone survived, of all the
numerous posterity of Constantius Chlorus. The misfortune of his
royal birth involved him in the disgrace of Gallus. From his
retirement in the happy country of Ionia, he was conveyed under a
strong guard to the court of Milan; where he languished above
seven months, in the continual apprehension of suffering the same
ignominious death, which was daily inflicted almost before his
eyes, on the friends and adherents of his persecuted family. His
looks, his gestures, his silence, were scrutinized with malignant
curiosity, and he was perpetually assaulted by enemies whom he
had never offended, and by arts to which he was a stranger.\textsuperscript[26]
But in the school of adversity, Julian insensibly acquired the
virtues of firmness and discretion. He defended his honor, as
well as his life, against the insnaring subtleties of the
eunuchs, who endeavored to extort some declaration of his
sentiments; and whilst he cautiously suppressed his grief and
resentment, he nobly disdained to flatter the tyrant, by any
seeming approbation of his brother’s murder. Julian most devoutly
ascribes his miraculous deliverance to the protection of the
gods, who had exempted his innocence from the sentence of
destruction pronounced by their justice against the impious house
of Constantine.\textsuperscript[27] As the most effectual instrument of their
providence, he gratefully acknowledges the steady and generous
friendship of the empress Eusebia,\textsuperscript[28] a woman of beauty and
merit, who, by the ascendant which she had gained over the mind
of her husband, counterbalanced, in some measure, the powerful
conspiracy of the eunuchs. By the intercession of his patroness,
Julian was admitted into the Imperial presence: he pleaded his
cause with a decent freedom, he was heard with favor; and,
notwithstanding the efforts of his enemies, who urged the danger
of sparing an avenger of the blood of Gallus, the milder
sentiment of Eusebia prevailed in the council. But the effects of
a second interview were dreaded by the eunuchs; and Julian was
advised to withdraw for a while into the neighborhood of Milan,
till the emperor thought proper to assign the city of Athens for
the place of his honorable exile. As he had discovered, from his
earliest youth, a propensity, or rather passion, for the
language, the manners, the learning, and the religion of the
Greeks, he obeyed with pleasure an order so agreeable to his
wishes. Far from the tumult of arms, and the treachery of courts,
he spent six months under the groves of the academy, in a free
intercourse with the philosophers of the age, who studied to
cultivate the genius, to encourage the vanity, and to inflame the
devotion of their royal pupil. Their labors were not
unsuccessful; and Julian inviolably preserved for Athens that
tender regard which seldom fails to arise in a liberal mind, from
the recollection of the place where it has discovered and
exercised its growing powers. The gentleness and affability of
manners, which his temper suggested and his situation imposed,
insensibly engaged the affections of the strangers, as well as
citizens, with whom he conversed. Some of his fellow-students
might perhaps examine his behavior with an eye of prejudice and
aversion; but Julian established, in the schools of Athens, a
general prepossession in favor of his virtues and talents, which
was soon diffused over the Roman world.\textsuperscript[29]

\pagenote[26]{See Ammianus Marcellin. l. xv. c. 1, 3, 8. Julian
himself in his epistle to the Athenians, draws a very lively and
just picture of his own danger, and of his sentiments. He shows,
however, a tendency to exaggerate his sufferings, by insinuating,
though in obscure terms, that they lasted above a year; a period
which cannot be reconciled with the truth of chronology.}

\pagenote[27]{Julian has worked the crimes and misfortunes of the
family of Constantine into an allegorical fable, which is happily
conceived and agreeably related. It forms the conclusion of the
seventh Oration, from whence it has been detached and translated
by the Abbé de la Bleterie, Vie de Jovien, tom. ii. p. 385-408.}

\pagenote[28]{She was a native of Thessalonica, in Macedonia, of
a noble family, and the daughter, as well as sister, of consuls.
Her marriage with the emperor may be placed in the year 352. In a
divided age, the historians of all parties agree in her praises.
See their testimonies collected by Tillemont, Hist. des
Empereurs, tom. iv. p. 750-754.}

\pagenote[29]{Libanius and Gregory Nazianzen have exhausted the
arts as well as the powers of their eloquence, to represent
Julian as the first of heroes, or the worst of tyrants. Gregory
was his fellow-student at Athens; and the symptoms which he so
tragically describes, of the future wickedness of the apostate,
amount only to some bodily imperfections, and to some
peculiarities in his speech and manner. He protests, however,
that he \textit{then} foresaw and foretold the calamities of the church
and state. (Greg. Nazianzen, Orat. iv. p. 121, 122.)}

Whilst his hours were passed in studious retirement, the empress,
resolute to achieve the generous design which she had undertaken,
was not unmindful of the care of his fortune. The death of the
late Cæsar had left Constantius invested with the sole command,
and oppressed by the accumulated weight, of a mighty empire.
Before the wounds of civil discord could be healed, the provinces
of Gaul were overwhelmed by a deluge of Barbarians. The
Sarmatians no longer respected the barrier of the Danube. The
impunity of rapine had increased the boldness and numbers of the
wild Isaurians: those robbers descended from their craggy
mountains to ravage the adjacent country, and had even presumed,
though without success, to besiege the important city of
Seleucia, which was defended by a garrison of three Roman
legions. Above all, the Persian monarch, elated by victory, again
threatened the peace of Asia, and the presence of the emperor was
indispensably required, both in the West and in the East. For the
first time, Constantius sincerely acknowledged, that his single
strength was unequal to such an extent of care and of dominion.\textsuperscript[30]
Insensible to the voice of flattery, which assured him that
his all-powerful virtue, and celestial fortune, would still
continue to triumph over every obstacle, he listened with
complacency to the advice of Eusebia, which gratified his
indolence, without offending his suspicious pride. As she
perceived that the remembrance of Gallus dwelt on the emperor’s
mind, she artfully turned his attention to the opposite
characters of the two brothers, which from their infancy had been
compared to those of Domitian and of Titus.\textsuperscript[31] She accustomed her
husband to consider Julian as a youth of a mild, unambitious
disposition, whose allegiance and gratitude might be secured by
the gift of the purple, and who was qualified to fill with honor
a subordinate station, without aspiring to dispute the commands,
or to shade the glories, of his sovereign and benefactor. After
an obstinate, though secret struggle, the opposition of the
favorite eunuchs submitted to the ascendency of the empress; and
it was resolved that Julian, after celebrating his nuptials with
Helena, sister of Constantius, should be appointed, with the
title of Cæsar, to reign over the countries beyond the Alps.\textsuperscript[32]

\pagenote[30]{Succumbere tot necessitatibus tamque crebris unum
se, quod nunquam fecerat, aperte demonstrans. Ammian. l. xv. c.
8. He then expresses, in their own words, the fattering
assurances of the courtiers.}

\pagenote[31]{Tantum a temperatis moribus Juliani differens
fratris quantum inter Vespasiani filios fuit, Domitianum et
Titum. Ammian. l. xiv. c. 11. The circumstances and education of
the two brothers, were so nearly the same, as to afford a strong
example of the innate difference of characters.}

\pagenote[32]{Ammianus, l. xv. c. 8. Zosimus, l. iii. p. 137,
138.}

Although the order which recalled him to court was probably
accompanied by some intimation of his approaching greatness, he
appeals to the people of Athens to witness his tears of
undissembled sorrow, when he was reluctantly torn away from his
beloved retirement.\textsuperscript[33] He trembled for his life, for his fame,
and even for his virtue; and his sole confidence was derived from
the persuasion, that Minerva inspired all his actions, and that
he was protected by an invisible guard of angels, whom for that
purpose she had borrowed from the Sun and Moon. He approached,
with horror, the palace of Milan; nor could the ingenuous youth
conceal his indignation, when he found himself accosted with
false and servile respect by the assassins of his family.
Eusebia, rejoicing in the success of her benevolent schemes,
embraced him with the tenderness of a sister; and endeavored, by
the most soothing caresses, to dispel his terrors, and reconcile
him to his fortune. But the ceremony of shaving his beard, and
his awkward demeanor, when he first exchanged the cloak of a
Greek philosopher for the military habit of a Roman prince,
amused, during a few days, the levity of the Imperial court.\textsuperscript[34]

\pagenote[33]{Julian. ad S. P. Q. A. p. 275, 276. Libanius, Orat.
x. p. 268. Julian did not yield till the gods had signified their
will by repeated visions and omens. His piety then forbade him to
resist.}

\pagenote[34]{Julian himself relates, (p. 274) with some humor,
the circumstances of his own metamorphoses, his downcast looks,
and his perplexity at being thus suddenly transported into a new
world, where every object appeared strange and hostile.}

The emperors of the age of Constantine no longer deigned to
consult with the senate in the choice of a colleague; but they
were anxious that their nomination should be ratified by the
consent of the army. On this solemn occasion, the guards, with
the other troops whose stations were in the neighborhood of
Milan, appeared under arms; and Constantius ascended his lofty
tribunal, holding by the hand his cousin Julian, who entered the
same day into the twenty-fifth year of his age.\textsuperscript[35] In a studied
speech, conceived and delivered with dignity, the emperor
represented the various dangers which threatened the prosperity
of the republic, the necessity of naming a Cæsar for the
administration of the West, and his own intention, if it was
agreeable to their wishes, of rewarding with the honors of the
purple the promising virtues of the nephew of Constantine. The
approbation of the soldiers was testified by a respectful murmur;
they gazed on the manly countenance of Julian, and observed with
pleasure, that the fire which sparkled in his eyes was tempered
by a modest blush, on being thus exposed, for the first time, to
the public view of mankind. As soon as the ceremony of his
investiture had been performed, Constantius addressed him with
the tone of authority which his superior age and station
permitted him to assume; and exhorting the new Cæsar to deserve,
by heroic deeds, that sacred and immortal name, the emperor gave
his colleague the strongest assurances of a friendship which
should never be impaired by time, nor interrupted by their
separation into the most distant climes. As soon as the speech
was ended, the troops, as a token of applause, clashed their
shields against their knees;\textsuperscript[36] while the officers who surrounded
the tribunal expressed, with decent reserve, their sense of the
merits of the representative of Constantius.

\pagenote[35]{See Ammian. Marcellin. l. xv. c. 8. Zosimus, l.
iii. p. 139. Aurelius Victor. Victor Junior in Epitom. Eutrop. x.
14.}

\pagenote[36]{Militares omnes horrendo fragore scuta genibus
illidentes; quod est prosperitatis indicium plenum; nam contra
cum hastis clypei feriuntur, iræ documentum est et doloris... ...
Ammianus adds, with a nice distinction, Eumque ut potiori
reverentia servaretur, nec supra modum laudabant nec infra quam
decebat.}

The two princes returned to the palace in the same chariot; and
during the slow procession, Julian repeated to himself a verse of
his favorite Homer, which he might equally apply to his fortune
and to his fears.\textsuperscript[37] The four-and-twenty days which the Cæsar
spent at Milan after his investiture, and the first months of his
Gallic reign, were devoted to a splendid but severe captivity;
nor could the acquisition of honor compensate for the loss of
freedom.\textsuperscript[38] His steps were watched, his correspondence was
intercepted; and he was obliged, by prudence, to decline the
visits of his most intimate friends. Of his former domestics,
four only were permitted to attend him; two pages, his physician,
and his librarian; the last of whom was employed in the care of a
valuable collection of books, the gift of the empress, who
studied the inclinations as well as the interest of her friend.
In the room of these faithful servants, a household was formed,
such indeed as became the dignity of a Cæsar; but it was filled
with a crowd of slaves, destitute, and perhaps incapable, of any
attachment for their new master, to whom, for the most part, they
were either unknown or suspected. His want of experience might
require the assistance of a wise council; but the minute
instructions which regulated the service of his table, and the
distribution of his hours, were adapted to a youth still under
the discipline of his preceptors, rather than to the situation of
a prince intrusted with the conduct of an important war. If he
aspired to deserve the esteem of his subjects, he was checked by
the fear of displeasing his sovereign; and even the fruits of his
marriage-bed were blasted by the jealous artifices of Eusebia\textsuperscript[39]
herself, who, on this occasion alone, seems to have been
unmindful of the tenderness of her sex, and the generosity of her
character. The memory of his father and of his brothers reminded
Julian of his own danger, and his apprehensions were increased by
the recent and unworthy fate of Sylvanus. In the summer which
preceded his own elevation, that general had been chosen to
deliver Gaul from the tyranny of the Barbarians; but Sylvanus
soon discovered that he had left his most dangerous enemies in
the Imperial court. A dexterous informer, countenanced by several
of the principal ministers, procured from him some recommendatory
letters; and erasing the whole of the contents, except the
signature, filled up the vacant parchment with matters of high
and treasonable import. By the industry and courage of his
friends, the fraud was however detected, and in a great council
of the civil and military officers, held in the presence of the
emperor himself, the innocence of Sylvanus was publicly
acknowledged. But the discovery came too late; the report of the
calumny, and the hasty seizure of his estate, had already
provoked the indignant chief to the rebellion of which he was so
unjustly accused. He assumed the purple at his head- quarters of
Cologne, and his active powers appeared to menace Italy with an
invasion, and Milan with a siege. In this emergency, Ursicinus, a
general of equal rank, regained, by an act of treachery, the
favor which he had lost by his eminent services in the East.
Exasperated, as he might speciously allege, by the injuries of a
similar nature, he hastened with a few followers to join the
standard, and to betray the confidence, of his too credulous
friend. After a reign of only twenty-eight days, Sylvanus was
assassinated: the soldiers who, without any criminal intention,
had blindly followed the example of their leader, immediately
returned to their allegiance; and the flatterers of Constantius
celebrated the wisdom and felicity of the monarch who had
extinguished a civil war without the hazard of a battle.\textsuperscript[40]

\pagenote[37]{The word \textit{purple} which Homer had used as a vague
but common epithet for death, was applied by Julian to express,
very aptly, the nature and object of his own apprehensions.}

\pagenote[38]{He represents, in the most pathetic terms, (p.
277,) the distress of his new situation. The provision for his
table was, however, so elegant and sumptuous, that the young
philosopher rejected it with disdain. Quum legeret libellum
assidue, quem Constantius ut privignum ad studia mittens manû suâ
conscripserat, prælicenter disponens quid in convivio Cæsaris
impendi deberit: Phasianum, et vulvam et sumen exigi vetuit et
inferri. Ammian. Marcellin. l. xvi. c. 5.}

\pagenote[39]{If we recollect that Constantine, the father of
Helena, died above eighteen years before, in a mature old age, it
will appear probable, that the daughter, though a virgin, could
not be very young at the time of her marriage. She was soon
afterwards delivered of a son, who died immediately, quod
obstetrix corrupta mercede, mox natum præsecto plusquam
convenerat umbilico necavit. She accompanied the emperor and
empress in their journey to Rome, and the latter, quæsitum
venenum bibere per fraudem illexit, ut quotiescunque concepisset,
immaturum abjicerit partum. Ammian. l. xvi. c. 10. Our physicians
will determine whether there exists such a poison. For my own
part I am inclined to hope that the public malignity imputed the
effects of accident as the guilt of Eusebia.}

\pagenote[40]{Ammianus (xv. v.) was perfectly well informed of
the conduct and fate of Sylvanus. He himself was one of the few
followers who attended Ursicinus in his dangerous enterprise.}

The protection of the Rhætian frontier, and the persecution of
the Catholic church, detained Constantius in Italy above eighteen
months after the departure of Julian. Before the emperor returned
into the East, he indulged his pride and curiosity in a visit to
the ancient capital.\textsuperscript[41] He proceeded from Milan to Rome along the
Æmilian and Flaminian ways, and as soon as he approached within
forty miles of the city, the march of a prince who had never
vanquished a foreign enemy, assumed the appearance of a triumphal
procession. His splendid train was composed of all the ministers
of luxury; but in a time of profound peace, he was encompassed by
the glittering arms of the numerous squadrons of his guards and
cuirassiers. Their streaming banners of silk, embossed with gold,
and shaped in the form of dragons, waved round the person of the
emperor. Constantius sat alone in a lofty car, resplendent with
gold and precious gems; and, except when he bowed his head to
pass under the gates of the cities, he affected a stately
demeanor of inflexible, and, as it might seem, of insensible
gravity. The severe discipline of the Persian youth had been
introduced by the eunuchs into the Imperial palace; and such were
the habits of patience which they had inculcated, that during a
slow and sultry march, he was never seen to move his hand towards
his face, or to turn his eyes either to the right or to the left.
He was received by the magistrates and senate of Rome; and the
emperor surveyed, with attention, the civil honors of the
republic, and the consular images of the noble families. The
streets were lined with an innumerable multitude. Their repeated
acclamations expressed their joy at beholding, after an absence
of thirty-two years, the sacred person of their sovereign, and
Constantius himself expressed, with some pleasantry, he affected
surprise that the human race should thus suddenly be collected on
the same spot. The son of Constantine was lodged in the ancient
palace of Augustus: he presided in the senate, harangued the
people from the tribunal which Cicero had so often ascended,
assisted with unusual courtesy at the games of the Circus, and
accepted the crowns of gold, as well as the Panegyrics which had
been prepared for the ceremony by the deputies of the principal
cities. His short visit of thirty days was employed in viewing
the monuments of art and power which were scattered over the
seven hills and the interjacent valleys. He admired the awful
majesty of the Capitol, the vast extent of the baths of Caracalla
and Diocletian, the severe simplicity of the Pantheon, the massy
greatness of the amphitheatre of Titus, the elegant architecture
of the theatre of Pompey and the Temple of Peace, and, above all,
the stately structure of the Forum and column of Trajan;
acknowledging that the voice of fame, so prone to invent and to
magnify, had made an inadequate report of the metropolis of the
world. The traveller, who has contemplated the ruins of ancient
Rome, may conceive some imperfect idea of the sentiments which
they must have inspired when they reared their heads in the
splendor of unsullied beauty.

[See The Pantheon: The severe simplicity of the Pantheon]

\pagenote[41]{For the particulars of the visit of Constantius to
Rome, see Ammianus, l. xvi. c. 10. We have only to add, that
Themistius was appointed deputy from Constantinople, and that he
composed his fourth oration for his ceremony.}

The satisfaction which Constantius had received from this journey
excited him to the generous emulation of bestowing on the Romans
some memorial of his own gratitude and munificence. His first
idea was to imitate the equestrian and colossal statue which he
had seen in the Forum of Trajan; but when he had maturely weighed
the difficulties of the execution,\textsuperscript[42] he chose rather to
embellish the capital by the gift of an Egyptian obelisk. In a
remote but polished age, which seems to have preceded the
invention of alphabetical writing, a great number of these
obelisks had been erected, in the cities of Thebes and
Heliopolis, by the ancient sovereigns of Egypt, in a just
confidence that the simplicity of their form, and the hardness of
their substance, would resist the injuries of time and violence.\textsuperscript[43]
Several of these extraordinary columns had been transported to
Rome by Augustus and his successors, as the most durable
monuments of their power and victory;\textsuperscript[44] but there remained one
obelisk, which, from its size or sanctity, escaped for a long
time the rapacious vanity of the conquerors. It was designed by
Constantine to adorn his new city;\textsuperscript[45] and, after being removed by
his order from the pedestal where it stood before the Temple of
the Sun at Heliopolis, was floated down the Nile to Alexandria.
The death of Constantine suspended the execution of his purpose,
and this obelisk was destined by his son to the ancient capital
of the empire. A vessel of uncommon strength and capaciousness
was provided to convey this enormous weight of granite, at least
a hundred and fifteen feet in length, from the banks of the Nile
to those of the Tyber. The obelisk of Constantius was landed
about three miles from the city, and elevated, by the efforts of
art and labor, in the great Circus of Rome.\textsuperscript[46] \textsuperscript[4611]

\pagenote[42]{Hormisdas, a fugitive prince of Persia, observed to
the emperor, that if he made such a horse, he must think of
preparing a similar stable, (the Forum of Trajan.) Another saying
of Hormisdas is recorded, “that one thing only had \textit{displeased}
him, to find that men died at Rome as well as elsewhere.” If we
adopt this reading of the text of Ammianus, (\textit{displicuisse},
instead of \textit{placuisse},) we may consider it as a reproof of Roman
vanity. The contrary sense would be that of a misanthrope.}

\pagenote[43]{When Germanicus visited the ancient monuments of
Thebes, the eldest of the priests explained to him the meaning of
these hiero glyphics. Tacit. Annal. ii. c. 60. But it seems
probable, that before the useful invention of an alphabet, these
natural or arbitrary signs were the common characters of the
Egyptian nation. See Warburton’s Divine Legation of Moses, vol.
iii. p. 69-243.}

\pagenote[44]{See Plin. Hist. Natur. l. xxxvi. c. 14, 15.}

\pagenote[45]{Ammian. Marcellin l. xvii. c. 4. He gives us a
Greek interpretation of the hieroglyphics, and his commentator
Lindenbrogius adds a Latin inscription, which, in twenty verses
of the age of Constantius, contain a short history of the
obelisk.}

\pagenote[46]{See Donat. Roma. Antiqua, l. iii. c. 14, l. iv. c.
12, and the learned, though confused, Dissertation of Bargæus on
Obelisks, inserted in the fourth volume of Grævius’s Roman
Antiquities, p. 1897- 1936. This dissertation is dedicated to
Pope Sixtus V., who erected the obelisk of Constantius in the
square before the patriarchal church of at. John Lateran.}

\pagenote[4611]{It is doubtful whether the obelisk transported by
Constantius to Rome now exists. Even from the text of Ammianus,
it is uncertain whether the interpretation of Hermapion refers to
the older obelisk, (obelisco incisus est veteri quem videmus in
Circo,) raised, as he himself states, in the Circus Maximus, long
before, by Augustus, or to the one brought by Constantius. The
obelisk in the square before the church of St. John Lateran is
ascribed not to Rameses the Great but to Thoutmos II.
Champollion, 1. Lettre a M. de Blacas, p. 32.—M}

The departure of Constantius from Rome was hastened by the
alarming intelligence of the distress and danger of the Illyrian
provinces. The distractions of civil war, and the irreparable
loss which the Roman legions had sustained in the battle of
Mursa, exposed those countries, almost without defence, to the
light cavalry of the Barbarians; and particularly to the inroads
of the Quadi, a fierce and powerful nation, who seem to have
exchanged the institutions of Germany for the arms and military
arts of their Sarmatian allies.\textsuperscript[47] The garrisons of the frontiers
were insufficient to check their progress; and the indolent
monarch was at length compelled to assemble, from the extremities
of his dominions, the flower of the Palatine troops, to take the
field in person, and to employ a whole campaign, with the
preceding autumn and the ensuing spring, in the serious
prosecution of the war. The emperor passed the Danube on a bridge
of boats, cut in pieces all that encountered his march,
penetrated into the heart of the country of the Quadi, and
severely retaliated the calamities which they had inflicted on
the Roman province. The dismayed Barbarians were soon reduced to
sue for peace: they offered the restitution of his captive
subjects as an atonement for the past, and the noblest hostages
as a pledge of their future conduct. The generous courtesy which
was shown to the first among their chieftains who implored the
clemency of Constantius, encouraged the more timid, or the more
obstinate, to imitate their example; and the Imperial camp was
crowded with the princes and ambassadors of the most distant
tribes, who occupied the plains of the Lesser Poland, and who
might have deemed themselves secure behind the lofty ridge of the
Carpathian Mountains. While Constantius gave laws to the
Barbarians beyond the Danube, he distinguished, with specious
compassion, the Sarmatian exiles, who had been expelled from
their native country by the rebellion of their slaves, and who
formed a very considerable accession to the power of the Quadi.
The emperor, embracing a generous but artful system of policy,
released the Sarmatians from the bands of this humiliating
dependence, and restored them, by a separate treaty, to the
dignity of a nation united under the government of a king, the
friend and ally of the republic. He declared his resolution of
asserting the justice of their cause, and of securing the peace
of the provinces by the extirpation, or at least the banishment,
of the Limigantes, whose manners were still infected with the
vices of their servile origin. The execution of this design was
attended with more difficulty than glory. The territory of the
Limigantes was protected against the Romans by the Danube,
against the hostile Barbarians by the Teyss. The marshy lands
which lay between those rivers, and were often covered by their
inundations, formed an intricate wilderness, pervious only to the
inhabitants, who were acquainted with its secret paths and
inaccessible fortresses. On the approach of Constantius, the
Limigantes tried the efficacy of prayers, of fraud, and of arms;
but he sternly rejected their supplications, defeated their rude
stratagems, and repelled with skill and firmness the efforts of
their irregular valor. One of their most warlike tribes,
established in a small island towards the conflux of the Teyss
and the Danube, consented to pass the river with the intention of
surprising the emperor during the security of an amicable
conference. They soon became the victims of the perfidy which
they meditated. Encompassed on every side, trampled down by the
cavalry, slaughtered by the swords of the legions, they disdained
to ask for mercy; and with an undaunted countenance, still
grasped their weapons in the agonies of death. After this
victory, a considerable body of Romans was landed on the opposite
banks of the Danube; the Taifalæ, a Gothic tribe engaged in the
service of the empire, invaded the Limigantes on the side of the
Teyss; and their former masters, the free Sarmatians, animated by
hope and revenge, penetrated through the hilly country, into the
heart of their ancient possessions. A general conflagration
revealed the huts of the Barbarians, which were seated in the
depth of the wilderness; and the soldier fought with confidence
on marshy ground, which it was dangerous for him to tread. In
this extremity, the bravest of the Limigantes were resolved to
die in arms, rather than to yield: but the milder sentiment,
enforced by the authority of their elders, at length prevailed;
and the suppliant crowd, followed by their wives and children,
repaired to the Imperial camp, to learn their fate from the mouth
of the conqueror. After celebrating his own clemency, which was
still inclined to pardon their repeated crimes, and to spare the
remnant of a guilty nation, Constantius assigned for the place of
their exile a remote country, where they might enjoy a safe and
honorable repose. The Limigantes obeyed with reluctance; but
before they could reach, at least before they could occupy, their
destined habitations, they returned to the banks of the Danube,
exaggerating the hardships of their situation, and requesting,
with fervent professions of fidelity, that the emperor would
grant them an undisturbed settlement within the limits of the
Roman provinces. Instead of consulting his own experience of
their incurable perfidy, Constantius listened to his flatterers,
who were ready to represent the honor and advantage of accepting
a colony of soldiers, at a time when it was much easier to obtain
the pecuniary contributions than the military service of the
subjects of the empire. The Limigantes were permitted to pass the
Danube; and the emperor gave audience to the multitude in a large
plain near the modern city of Buda. They surrounded the tribunal,
and seemed to hear with respect an oration full of mildness and
dignity when one of the Barbarians, casting his shoe into the
air, exclaimed with a loud voice, \textit{Marha! Marha!}\textsuperscript[4711] a word of
defiance, which was received as a signal of the tumult. They
rushed with fury to seize the person of the emperor; his royal
throne and golden couch were pillaged by these rude hands; but
the faithful defence of his guards, who died at his feet, allowed
him a moment to mount a fleet horse, and to escape from the
confusion. The disgrace which had been incurred by a treacherous
surprise was soon retrieved by the numbers and discipline of the
Romans; and the combat was only terminated by the extinction of
the name and nation of the Limigantes. The free Sarmatians were
reinstated in the possession of their ancient seats; and although
Constantius distrusted the levity of their character, he
entertained some hopes that a sense of gratitude might influence
their future conduct. He had remarked the lofty stature and
obsequious demeanor of Zizais, one of the noblest of their
chiefs. He conferred on him the title of King; and Zizais proved
that he was not unworthy to reign, by a sincere and lasting
attachment to the interests of his benefactor, who, after this
splendid success, received the name of \textit{Sarmaticus} from the
acclamations of his victorious army.\textsuperscript[48]

\pagenote[47]{The events of this Quadian and Sarmatian war are
related by Ammianus, xvi. 10, xvii. 12, 13, xix. 11}

\pagenote[4711]{Reinesius reads Warrha, Warrha, Guerre, War.
Wagner note as a mm. Marc xix. ll.—M.}

\pagenote[48]{Genti Sarmatarum magno decori confidens apud eos
regem dedit. Aurelius Victor. In a pompous oration pronounced by
Constantius himself, he expatiates on his own exploits with much
vanity, and some truth}

\section{Part \thesection.}

While the Roman emperor and the Persian monarch, at the distance
of three thousand miles, defended their extreme limits against
the Barbarians of the Danube and of the Oxus, their intermediate
frontier experienced the vicissitudes of a languid war, and a
precarious truce. Two of the eastern ministers of Constantius,
the Prætorian præfect Musonian, whose abilities were disgraced by
the want of truth and integrity, and Cassian, duke of
Mesopotamia, a hardy and veteran soldier, opened a secret
negotiation with the satrap Tamsapor.\textsuperscript[49] \textsuperscript[4911] These overtures of
peace, translated into the servile and flattering language of
Asia, were transmitted to the camp of the Great King; who
resolved to signify, by an ambassador, the terms which he was
inclined to grant to the suppliant Romans. Narses, whom he
invested with that character, was honorably received in his
passage through Antioch and Constantinople: he reached Sirmium
after a long journey, and, at his first audience, respectfully
unfolded the silken veil which covered the haughty epistle of his
sovereign. Sapor, King of Kings, and Brother of the Sun and Moon,
(such were the lofty titles affected by Oriental vanity,)
expressed his satisfaction that his brother, Constantius Cæsar,
had been taught wisdom by adversity. As the lawful successor of
Darius Hystaspes, Sapor asserted, that the River Strymon, in
Macedonia, was the true and ancient boundary of his empire;
declaring, however, that as an evidence of his moderation, he
would content himself with the provinces of Armenia and
Mesopotamia, which had been fraudulently extorted from his
ancestors. He alleged, that, without the restitution of these
disputed countries, it was impossible to establish any treaty on
a solid and permanent basis; and he arrogantly threatened, that
if his ambassador returned in vain, he was prepared to take the
field in the spring, and to support the justice of his cause by
the strength of his invincible arms. Narses, who was endowed with
the most polite and amiable manners, endeavored, as far as was
consistent with his duty, to soften the harshness of the message.\textsuperscript[50]
Both the style and substance were maturely weighed in the
Imperial council, and he was dismissed with the following answer:
“Constantius had a right to disclaim the officiousness of his
ministers, who had acted without any specific orders from the
throne: he was not, however, averse to an equal and honorable
treaty; but it was highly indecent, as well as absurd, to propose
to the sole and victorious emperor of the Roman world, the same
conditions of peace which he had indignantly rejected at the time
when his power was contracted within the narrow limits of the
East: the chance of arms was uncertain; and Sapor should
recollect, that if the Romans had sometimes been vanquished in
battle, they had almost always been successful in the event of
the war.” A few days after the departure of Narses, three
ambassadors were sent to the court of Sapor, who was already
returned from the Scythian expedition to his ordinary residence
of Ctesiphon. A count, a notary, and a sophist, had been selected
for this important commission; and Constantius, who was secretly
anxious for the conclusion of the peace, entertained some hopes
that the dignity of the first of these ministers, the dexterity
of the second, and the rhetoric of the third,\textsuperscript[51] would persuade
the Persian monarch to abate of the rigor of his demands. But the
progress of their negotiation was opposed and defeated by the
hostile arts of Antoninus,\textsuperscript[52] a Roman subject of Syria, who had
fled from oppression, and was admitted into the councils of
Sapor, and even to the royal table, where, according to the
custom of the Persians, the most important business was
frequently discussed.\textsuperscript[53] The dexterous fugitive promoted his
interest by the same conduct which gratified his revenge. He
incessantly urged the ambition of his new master to embrace the
favorable opportunity when the bravest of the Palatine troops
were employed with the emperor in a distant war on the Danube. He
pressed Sapor to invade the exhausted and defenceless provinces
of the East, with the numerous armies of Persia, now fortified by
the alliance and accession of the fiercest Barbarians. The
ambassadors of Rome retired without success, and a second
embassy, of a still more honorable rank, was detained in strict
confinement, and threatened either with death or exile.

\pagenote[49]{Ammian. xvi. 9.}

\pagenote[4911]{In Persian, Ten-schah-pour. St. Martin, ii.
177.—M.}

\pagenote[50]{Ammianus (xvii. 5) transcribes the haughty letter.
Themistius (Orat. iv. p. 57, edit. Petav.) takes notice of the
silken covering. Idatius and Zonaras mention the journey of the
ambassador; and Peter the Patrician (in Excerpt. Legat. p. 58)
has informed us of his behavior.}

\pagenote[51]{Ammianus, xvii. 5, and Valesius ad loc. The
sophist, or philosopher, (in that age these words were almost
synonymous,) was Eustathius the Cappadocian, the disciple of
Jamblichus, and the friend of St. Basil. Eunapius (in Vit.
Ædesii, p. 44-47) fondly attributes to this philosophic
ambassador the glory of enchanting the Barbarian king by the
persuasive charms of reason and eloquence. See Tillemont, Hist.
des Empereurs, tom. iv. p. 828, 1132.}

\pagenote[52]{Ammian. xviii. 5, 6, 8. The decent and respectful
behavior of Antoninus towards the Roman general, sets him in a
very interesting light; and Ammianus himself speaks of the
traitor with some compassion and esteem.}

\pagenote[53]{This circumstance, as it is noticed by Ammianus,
serves to prove the veracity of Herodotus, (l. i. c. 133,) and
the permanency of the Persian manners. In every age the Persians
have been addicted to intemperance, and the wines of Shiraz have
triumphed over the law of Mahomet. Brisson de Regno Pers. l. ii.
p. 462-472, and Voyages en Perse, tom, iii. p. 90.}

The military historian,\textsuperscript[54] who was himself despatched to observe
the army of the Persians, as they were preparing to construct a
bridge of boats over the Tigris, beheld from an eminence the
plain of Assyria, as far as the edge of the horizon, covered with
men, with horses, and with arms. Sapor appeared in the front,
conspicuous by the splendor of his purple. On his left hand, the
place of honor among the Orientals, Grumbates, king of the
Chionites, displayed the stern countenance of an aged and
renowned warrior. The monarch had reserved a similar place on his
right hand for the king of the Albanians, who led his independent
tribes from the shores of the Caspian.\textsuperscript[5411] The satraps and
generals were distributed according to their several ranks, and
the whole army, besides the numerous train of Oriental luxury,
consisted of more than one hundred thousand effective men, inured
to fatigue, and selected from the bravest nations of Asia. The
Roman deserter, who in some measure guided the councils of Sapor,
had prudently advised, that, instead of wasting the summer in
tedious and difficult sieges, he should march directly to the
Euphrates, and press forwards without delay to seize the feeble
and wealthy metropolis of Syria. But the Persians were no sooner
advanced into the plains of Mesopotamia, than they discovered
that every precaution had been used which could retard their
progress, or defeat their design. The inhabitants, with their
cattle, were secured in places of strength, the green forage
throughout the country was set on fire, the fords of the rivers
were fortified by sharp stakes; military engines were planted on
the opposite banks, and a seasonable swell of the waters of the
Euphrates deterred the Barbarians from attempting the ordinary
passage of the bridge of Thapsacus. Their skilful guide, changing
his plan of operations, then conducted the army by a longer
circuit, but through a fertile territory, towards the head of the
Euphrates, where the infant river is reduced to a shallow and
accessible stream. Sapor overlooked, with prudent disdain, the
strength of Nisibis; but as he passed under the walls of Amida,
he resolved to try whether the majesty of his presence would not
awe the garrison into immediate submission. The sacrilegious
insult of a random dart, which glanced against the royal tiara,
convinced him of his error; and the indignant monarch listened
with impatience to the advice of his ministers, who conjured him
not to sacrifice the success of his ambition to the gratification
of his resentment. The following day Grumbates advanced towards
the gates with a select body of troops, and required the instant
surrender of the city, as the only atonement which could be
accepted for such an act of rashness and insolence. His proposals
were answered by a general discharge, and his only son, a
beautiful and valiant youth, was pierced through the heart by a
javelin, shot from one of the balistæ. The funeral of the prince
of the Chionites was celebrated according to the rites of the
country; and the grief of his aged father was alleviated by the
solemn promise of Sapor, that the guilty city of Amida should
serve as a funeral pile to expiate the death, and to perpetuate
the memory, of his son.

\pagenote[54]{Ammian. lxviii. 6, 7, 8, 10.}

\pagenote[5411]{These perhaps were the barbarous tribes who
inhabit the northern part of the present Schirwan, the Albania of
the ancients. This country, now inhabited by the Lezghis, the
terror of the neighboring districts, was then occupied by the
same people, called by the ancients Legæ, by the Armenians Gheg,
or Leg. The latter represent them as constant allies of the
Persians in their wars against Armenia and the Empire. A little
after this period, a certain Schergir was their king, and it is
of him doubtless Ammianus Marcellinus speaks. St. Martin, ii.
285.—M.}

The ancient city of Amid or Amida,\textsuperscript[55] which sometimes assumes the
provincial appellation of Diarbekir,\textsuperscript[56] is advantageously situate
in a fertile plain, watered by the natural and artificial
channels of the Tigris, of which the least inconsiderable stream
bends in a semicircular form round the eastern part of the city.
The emperor Constantius had recently conferred on Amida the honor
of his own name, and the additional fortifications of strong
walls and lofty towers. It was provided with an arsenal of
military engines, and the ordinary garrison had been reenforced
to the amount of seven legions, when the place was invested by
the arms of Sapor.\textsuperscript[57] His first and most sanguine hopes depended
on the success of a general assault. To the several nations which
followed his standard, their respective posts were assigned; the
south to the Vertæ; the north to the Albanians; the east to the
Chionites, inflamed with grief and indignation; the west to the
Segestans, the bravest of his warriors, who covered their front
with a formidable line of Indian elephants.\textsuperscript[58] The Persians, on
every side, supported their efforts, and animated their courage;
and the monarch himself, careless of his rank and safety,
displayed, in the prosecution of the siege, the ardor of a
youthful soldier. After an obstinate combat, the Barbarians were
repulsed; they incessantly returned to the charge; they were
again driven back with a dreadful slaughter, and two rebel
legions of Gauls, who had been banished into the East, signalized
their undisciplined courage by a nocturnal sally into the heart
of the Persian camp. In one of the fiercest of these repeated
assaults, Amida was betrayed by the treachery of a deserter, who
indicated to the Barbarians a secret and neglected staircase,
scooped out of the rock that hangs over the stream of the Tigris.
Seventy chosen archers of the royal guard ascended in silence to
the third story of a lofty tower, which commanded the precipice;
they elevated on high the Persian banner, the signal of
confidence to the assailants, and of dismay to the besieged; and
if this devoted band could have maintained their post a few
minutes longer, the reduction of the place might have been
purchased by the sacrifice of their lives. After Sapor had tried,
without success, the efficacy of force and of stratagem, he had
recourse to the slower but more certain operations of a regular
siege, in the conduct of which he was instructed by the skill of
the Roman deserters. The trenches were opened at a convenient
distance, and the troops destined for that service advanced under
the portable cover of strong hurdles, to fill up the ditch, and
undermine the foundations of the walls. Wooden towers were at the
same time constructed, and moved forwards on wheels, till the
soldiers, who were provided with every species of missile
weapons, could engage almost on level ground with the troops who
defended the rampart. Every mode of resistance which art could
suggest, or courage could execute, was employed in the defence of
Amida, and the works of Sapor were more than once destroyed by
the fire of the Romans. But the resources of a besieged city may
be exhausted. The Persians repaired their losses, and pushed
their approaches; a large preach was made by the battering-ram,
and the strength of the garrison, wasted by the sword and by
disease, yielded to the fury of the assault. The soldiers, the
citizens, their wives, their children, all who had not time to
escape through the opposite gate, were involved by the conquerors
in a promiscuous massacre.

\pagenote[55]{For the description of Amida, see D’Herbelot,
Bebliotheque Orientale, p. Bibliothèque Orientale, p. 108.
Histoire de Timur Bec, par Cherefeddin Ali, l. iii. c. 41. Ahmed
Arabsiades, tom. i. p. 331, c. 43. Voyages de Tavernier, tom. i.
p. 301. Voyages d’Otter, tom. ii. p. 273, and Voyages de Niebuhr,
tom. ii. p. 324-328. The last of these travellers, a learned and
accurate Dane, has given a plan of Amida, which illustrates the
operations of the siege.}

\pagenote[56]{Diarbekir, which is styled Amid, or Kara Amid, in
the public writings of the Turks, contains above 16,000 houses,
and is the residence of a pacha with three tails. The epithet of
\textit{Kara} is derived from the \textit{blackness} of the stone which
composes the strong and ancient wall of Amida. ——In my Mém. Hist.
sur l’Armenie, l. i. p. 166, 173, I conceive that I have proved
this city, still called, by the Armenians, Dirkranagerd, the city
of Tigranes, to be the same with the famous Tigranocerta, of
which the situation was unknown. St. Martin, i. 432. On the siege
of Amida, see St. Martin’s Notes, ii. 290. Faustus of Byzantium,
nearly a contemporary, (Armenian,) states that the Persians, on
becoming masters of it, destroyed 40,000 houses though Ammianus
describes the city as of no great extent, (civitatis ambitum non
nimium amplæ.) Besides the ordinary population, and those who
took refuge from the country, it contained 20,000 soldiers. St.
Martin, ii. 290. This interpretation is extremely doubtful.
Wagner (note on Ammianus) considers the whole population to
amount only to—M.}

\pagenote[57]{The operations of the siege of Amida are very
minutely described by Ammianus, (xix. 1-9,) who acted an
honorable part in the defence, and escaped with difficulty when
the city was stormed by the Persians.}

\pagenote[58]{Of these four nations, the Albanians are too well
known to require any description. The Segestans [\textit{Sacastenè. St.
Martin.}] inhabited a large and level country, which still
preserves their name, to the south of Khorasan, and the west of
Hindostan. (See Geographia Nubiensis. p. 133, and D’Herbelot,
Bibliothèque Orientale, p. 797.) Notwithstanding the boasted
victory of Bahram, (vol. i. p. 410,) the Segestans, above
fourscore years afterwards, appear as an independent nation, the
ally of Persia. We are ignorant of the situation of the Vertæ and
Chionites, but I am inclined to place them (at least the latter)
towards the confines of India and Scythia. See Ammian. ——Klaproth
considers the real Albanians the same with the ancient Alani, and
quotes a passage of the emperor Julian in support of his opinion.
They are the Ossetæ, now inhabiting part of Caucasus. Tableaux
Hist. de l’Asie, p. 179, 180.—M. ——The Vertæ are still unknown.
It is possible that the Chionites are the same as the Huns. These
people were already known; and we find from Armenian authors that
they were making, at this period, incursions into Asia. They were
often at war with the Persians. The name was perhaps pronounced
differently in the East and in the West, and this prevents us
from recognizing it. St. Martin, ii. 177.—M.}

But the ruin of Amida was the safety of the Roman provinces.

As soon as the first transports of victory had subsided, Sapor
was at leisure to reflect, that to chastise a disobedient city,
he had lost the flower of his troops, and the most favorable
season for conquest.\textsuperscript[59] Thirty thousand of his veterans had
fallen under the walls of Amida, during the continuance of a
siege, which lasted seventy-three days; and the disappointed
monarch returned to his capital with affected triumph and secret
mortification. It is more than probable, that the inconstancy of
his Barbarian allies was tempted to relinquish a war in which
they had encountered such unexpected difficulties; and that the
aged king of the Chionites, satiated with revenge, turned away
with horror from a scene of action where he had been deprived of
the hope of his family and nation. The strength as well as the
spirit of the army with which Sapor took the field in the ensuing
spring was no longer equal to the unbounded views of his
ambition. Instead of aspiring to the conquest of the East, he was
obliged to content himself with the reduction of two fortified
cities of Mesopotamia, Singara and Bezabde;\textsuperscript[60] the one situate in
the midst of a sandy desert, the other in a small peninsula,
surrounded almost on every side by the deep and rapid stream of
the Tigris. Five Roman legions, of the diminutive size to which
they had been reduced in the age of Constantine, were made
prisoners, and sent into remote captivity on the extreme confines
of Persia. After dismantling the walls of Singara, the conqueror
abandoned that solitary and sequestered place; but he carefully
restored the fortifications of Bezabde, and fixed in that
important post a garrison or colony of veterans; amply supplied
with every means of defence, and animated by high sentiments of
honor and fidelity. Towards the close of the campaign, the arms
of Sapor incurred some disgrace by an unsuccessful enterprise
against Virtha, or Tecrit, a strong, or, as it was universally
esteemed till the age of Tamerlane, an impregnable fortress of
the independent Arabs.\textsuperscript[61] \textsuperscript[6111]

\pagenote[59]{Ammianus has marked the chronology of this year by
three signs, which do not perfectly coincide with each other, or
with the series of the history. 1 The corn was ripe when Sapor
invaded Mesopotamia; “Cum jam stipula flaveate turgerent;” a
circumstance, which, in the latitude of Aleppo, would naturally
refer us to the month of April or May. See Harmer’s Observations
on Scripture vol. i. p. 41. Shaw’s Travels, p. 335, edit 4to. 2.
The progress of Sapor was checked by the overflowing of the
Euphrates, which generally happens in July and August. Plin.
Hist. Nat. v. 21. Viaggi di Pietro della Valle, tom. i. p. 696.
3. When Sapor had taken Amida, after a siege of seventy-three
days, the autumn was far advanced. “Autumno præcipiti hædorumque
improbo sidere exorto.” To reconcile these apparent
contradictions, we must allow for some delay in the Persian king,
some inaccuracy in the historian, and some disorder in the
seasons.}

\pagenote[60]{The account of these sieges is given by Ammianus,
xx. 6, 7. ——The Christian bishop of Bezabde went to the camp of
the king of Persia, to persuade him to check the waste of human
blood Amm. Mare xx. 7.—M.}

\pagenote[61]{For the identity of Virtha and Tecrit, see
D’Anville, Geographie. For the siege of that castle by Timur Bec
or Tamerlane, see Cherefeddin, l. iii. c. 33. The Persian
biographer exaggerates the merit and difficulty of this exploit,
which delivered the caravans of Bagdad from a formidable gang of
robbers.}

\pagenote[6111]{St. Martin doubts whether it lay so much to the
south. “The word Girtha means in Syriac a castle or fortress, and
might be applied to many places.”}

The defence of the East against the arms of Sapor required and
would have exercised, the abilities of the most consummate
general; and it seemed fortunate for the state, that it was the
actual province of the brave Ursicinus, who alone deserved the
confidence of the soldiers and people. In the hour of danger,\textsuperscript[62]
Ursicinus was removed from his station by the intrigues of the
eunuchs; and the military command of the East was bestowed, by
the same influence, on Sabinian, a wealthy and subtle veteran,
who had attained the infirmities, without acquiring the
experience, of age. By a second order, which issued from the same
jealous and inconstant councils, Ursicinus was again despatched
to the frontier of Mesopotamia, and condemned to sustain the
labors of a war, the honors of which had been transferred to his
unworthy rival. Sabinian fixed his indolent station under the
walls of Edessa; and while he amused himself with the idle parade
of military exercise, and moved to the sound of flutes in the
Pyrrhic dance, the public defence was abandoned to the boldness
and diligence of the former general of the East. But whenever
Ursicinus recommended any vigorous plan of operations; when he
proposed, at the head of a light and active army, to wheel round
the foot of the mountains, to intercept the convoys of the enemy,
to harass the wide extent of the Persian lines, and to relieve
the distress of Amida; the timid and envious commander alleged,
that he was restrained by his positive orders from endangering
the safety of the troops. Amida was at length taken; its bravest
defenders, who had escaped the sword of the Barbarians, died in
the Roman camp by the hand of the executioner: and Ursicinus
himself, after supporting the disgrace of a partial inquiry, was
punished for the misconduct of Sabinian by the loss of his
military rank. But Constantius soon experienced the truth of the
prediction which honest indignation had extorted from his injured
lieutenant, that as long as such maxims of government were
suffered to prevail, the emperor himself would find it is no easy
task to defend his eastern dominions from the invasion of a
foreign enemy. When he had subdued or pacified the Barbarians of
the Danube, Constantius proceeded by slow marches into the East;
and after he had wept over the smoking ruins of Amida, he formed,
with a powerful army, the siege of Becabde. The walls were shaken
by the reiterated efforts of the most enormous of the
battering-rams; the town was reduced to the last extremity; but
it was still defended by the patient and intrepid valor of the
garrison, till the approach of the rainy season obliged the
emperor to raise the siege, and ingloviously to retreat into his
winter quarters at Antioch.\textsuperscript[63] The pride of Constantius, and the
ingenuity of his courtiers, were at a loss to discover any
materials for panegyric in the events of the Persian war; while
the glory of his cousin Julian, to whose military command he had
intrusted the provinces of Gaul, was proclaimed to the world in
the simple and concise narrative of his exploits.

\pagenote[62]{Ammianus (xviii. 5, 6, xix. 3, xx. 2) represents
the merit and disgrace of Ursicinus with that faithful attention
which a soldier owed to his general. Some partiality may be
suspected, yet the whole account is consistent and probable.}

\pagenote[63]{Ammian. xx. 11. Omisso vano incepto, hiematurus
Antiochiæ redit in Syriam ærumnosam, perpessus et ulcerum sed et
atrocia, diuque deflenda. It is \textit{thus} that James Gronovius has
restored an obscure passage; and he thinks that this correction
alone would have deserved a new edition of his author: whose
sense may now be darkly perceived. I expected some additional
light from the recent labors of the learned Ernestus. (Lipsiæ,
1773.) * Note: The late editor (Wagner) has nothing better to
suggest, and le menta with Gibbon, the silence of Ernesti.—M.}

In the blind fury of civil discord, Constantius had abandoned to
the Barbarians of Germany the countries of Gaul, which still
acknowledged the authority of his rival. A numerous swarm of
Franks and Alemanni were invited to cross the Rhine by presents
and promises, by the hopes of spoil, and by a perpetual grant of
all the territories which they should be able to subdue.\textsuperscript[64] But
the emperor, who for a temporary service had thus imprudently
provoked the rapacious spirit of the Barbarians, soon discovered
and lamented the difficulty of dismissing these formidable
allies, after they had tasted the richness of the Roman soil.
Regardless of the nice distinction of loyalty and rebellion,
these undisciplined robbers treated as their natural enemies all
the subjects of the empire, who possessed any property which they
were desirous of acquiring Forty-five flourishing cities,
Tongres, Cologne, Treves, Worms, Spires, Strasburgh, \&c., besides
a far greater number of towns and villages, were pillaged, and
for the most part reduced to ashes. The Barbarians of Germany,
still faithful to the maxims of their ancestors, abhorred the
confinement of walls, to which they applied the odious names of
prisons and sepulchres; and fixing their independent habitations
on the banks of rivers, the Rhine, the Moselle, and the Meuse,
they secured themselves against the danger of a surprise, by a
rude and hasty fortification of large trees, which were felled
and thrown across the roads. The Alemanni were established in the
modern countries of Alsace and Lorraine; the Franks occupied the
island of the Batavians, together with an extensive district of
Brabant, which was then known by the appellation of Toxandria,\textsuperscript[65]
and may deserve to be considered as the original seat of their
Gallic monarchy.\textsuperscript[66] From the sources, to the mouth, of the Rhine,
the conquests of the Germans extended above forty miles to the
west of that river, over a country peopled by colonies of their
own name and nation: and the scene of their devastations was
three times more extensive than that of their conquests. At a
still greater distance the open towns of Gaul were deserted, and
the inhabitants of the fortified cities, who trusted to their
strength and vigilance, were obliged to content themselves with
such supplies of corn as they could raise on the vacant land
within the enclosure of their walls. The diminished legions,
destitute of pay and provisions, of arms and discipline, trembled
at the approach, and even at the name, of the Barbarians.

\pagenote[64]{The ravages of the Germans, and the distress of
Gaul, may be collected from Julian himself. Orat. ad S. P. Q.
Athen. p. 277. Ammian. xv. ll. Libanius, Orat. x. Zosimus, l.
iii. p. 140. Sozomen, l. iii. c. l. (Mamertin. Grat. Art. c.
iv.)}

\pagenote[65]{Ammianus, xvi. 8. This name seems to be derived
from the Toxandri of Pliny, and very frequently occurs in the
histories of the middle age. Toxandria was a country of woods and
morasses, which extended from the neighborhood of Tongres to the
conflux of the Vahal and the Rhine. See Valesius, Notit. Galliar.
p. 558.}

\pagenote[66]{The paradox of P. Daniel, that the Franks never
obtained any permanent settlement on this side of the Rhine
before the time of Clovis, is refuted with much learning and good
sense by M. Biet, who has proved by a chain of evidence, their
uninterrupted possession of Toxandria, one hundred and thirty
years before the accession of Clovis. The Dissertation of M. Biet
was crowned by the Academy of Soissons, in the year 1736, and
seems to have been justly preferred to the discourse of his more
celebrated competitor, the Abbé le Bœuf, an antiquarian, whose
name was happily expressive of his talents.}

\section{Part \thesection.}

Under these melancholy circumstances, an unexperienced youth was
appointed to save and to govern the provinces of Gaul, or rather,
as he expressed it himself, to exhibit the vain image of Imperial
greatness. The retired scholastic education of Julian, in which
he had been more conversant with books than with arms, with the
dead than with the living, left him in profound ignorance of the
practical arts of war and government; and when he awkwardly
repeated some military exercise which it was necessary for him to
learn, he exclaimed with a sigh, “O Plato, Plato, what a task for
a philosopher!” Yet even this speculative philosophy, which men
of business are too apt to despise, had filled the mind of Julian
with the noblest precepts and the most shining examples; had
animated him with the love of virtue, the desire of fame, and the
contempt of death. The habits of temperance recommended in the
schools, are still more essential in the severe discipline of a
camp. The simple wants of nature regulated the measure of his
food and sleep. Rejecting with disdain the delicacies provided
for his table, he satisfied his appetite with the coarse and
common fare which was allotted to the meanest soldiers. During
the rigor of a Gallic winter, he never suffered a fire in his
bed-chamber; and after a short and interrupted slumber, he
frequently rose in the middle of the night from a carpet spread
on the floor, to despatch any urgent business, to visit his
rounds, or to steal a few moments for the prosecution of his
favorite studies.\textsuperscript[67] The precepts of eloquence, which he had
hitherto practised on fancied topics of declamation, were more
usefully applied to excite or to assuage the passions of an armed
multitude: and although Julian, from his early habits of
conversation and literature, was more familiarly acquainted with
the beauties of the Greek language, he had attained a competent
knowledge of the Latin tongue.\textsuperscript[68] Since Julian was not originally
designed for the character of a legislator, or a judge, it is
probable that the civil jurisprudence of the Romans had not
engaged any considerable share of his attention: but he derived
from his philosophic studies an inflexible regard for justice,
tempered by a disposition to clemency; the knowledge of the
general principles of equity and evidence, and the faculty of
patiently investigating the most intricate and tedious questions
which could be proposed for his discussion. The measures of
policy, and the operations of war, must submit to the various
accidents of circumstance and character, and the unpractised
student will often be perplexed in the application of the most
perfect theory.

But in the acquisition of this important science, Julian was
assisted by the active vigor of his own genius, as well as by the
wisdom and experience of Sallust, and officer of rank, who soon
conceived a sincere attachment for a prince so worthy of his
friendship; and whose incorruptible integrity was adorned by the
talent of insinuating the harshest truths without wounding the
delicacy of a royal ear.\textsuperscript[69]

\pagenote[67]{The private life of Julian in Gaul, and the severe
discipline which he embraced, are displayed by Ammianus, (xvi.
5,) who professes to praise, and by Julian himself, who affects
to ridicule, (Misopogon, p. 340,) a conduct, which, in a prince
of the house of Constantine, might justly excite the surprise of
mankind.}

\pagenote[68]{Aderat Latine quoque disserenti sufficiens sermo.
Ammianus xvi. 5. But Julian, educated in the schools of Greece,
always considered the language of the Romans as a foreign and
popular dialect which he might use on necessary occasions.}

\pagenote[69]{We are ignorant of the actual office of this
excellent minister, whom Julian afterwards created præfect of
Gaul. Sallust was speedly recalled by the jealousy of the
emperor; and we may still read a sensible but pedantic discourse,
(p. 240-252,) in which Julian deplores the loss of so valuable a
friend, to whom he acknowledges himself indebted for his
reputation. See La Bleterie, Preface a la Vie de lovien, p. 20.}

Immediately after Julian had received the purple at Milan, he was
sent into Gaul with a feeble retinue of three hundred and sixty
soldiers. At Vienna, where he passed a painful and anxious winter
in the hands of those ministers to whom Constantius had intrusted
the direction of his conduct, the Cæsar was informed of the siege
and deliverance of Autun. That large and ancient city, protected
only by a ruined wall and pusillanimous garrison, was saved by
the generous resolution of a few veterans, who resumed their arms
for the defence of their country. In his march from Autun,
through the heart of the Gallic provinces, Julian embraced with
ardor the earliest opportunity of signalizing his courage. At the
head of a small body of archers and heavy cavalry, he preferred
the shorter but the more dangerous of two roads;\textsuperscript[6911] and
sometimes eluding, and sometimes resisting, the attacks of the
Barbarians, who were masters of the field, he arrived with honor
and safety at the camp near Rheims, where the Roman troops had
been ordered to assemble. The aspect of their young prince
revived the drooping spirits of the soldiers, and they marched
from Rheims in search of the enemy, with a confidence which had
almost proved fatal to them. The Alemanni, familiarized to the
knowledge of the country, secretly collected their scattered
forces, and seizing the opportunity of a dark and rainy day,
poured with unexpected fury on the rear-guard of the Romans.
Before the inevitable disorder could be remedied, two legions
were destroyed; and Julian was taught by experience that caution
and vigilance are the most important lessons of the art of war.
In a second and more successful action, he recovered and
established his military fame; but as the agility of the
Barbarians saved them from the pursuit, his victory was neither
bloody nor decisive. He advanced, however, to the banks of the
Rhine, surveyed the ruins of Cologne, convinced himself of the
difficulties of the war, and retreated on the approach of winter,
discontented with the court, with his army, and with his own
success.\textsuperscript[70] The power of the enemy was yet unbroken; and the
Cæsar had no sooner separated his troops, and fixed his own
quarters at Sens, in the centre of Gaul, than he was surrounded
and besieged, by a numerous host of Germans. Reduced, in this
extremity, to the resources of his own mind, he displayed a
prudent intrepidity, which compensated for all the deficiencies
of the place and garrison; and the Barbarians, at the end of
thirty days, were obliged to retire with disappointed rage.

\pagenote[6911]{Aliis per Arbor—quibusdam per Sedelaucum et Coram
in debere firrantibus. Amm. Marc. xvi. 2. I do not know what
place can be meant by the mutilated name Arbor. Sedelanus is
Saulieu, a small town of the department of the Cote d’Or, six
leagues from Autun. Cora answers to the village of Cure, on the
river of the same name, between Autun and Nevera 4; Martin, ii.
162.—M. ——Note: At Brocomages, Brumat, near Strasburgh. St.
Martin, ii. 184.—M.}

\pagenote[70]{Ammianus (xvi. 2, 3) appears much better satisfied
with the success of his first campaign than Julian himself; who
very fairly owns that he did nothing of consequence, and that he
fled before the enemy.}

The conscious pride of Julian, who was indebted only to his sword
for this signal deliverance, was imbittered by the reflection,
that he was abandoned, betrayed, and perhaps devoted to
destruction, by those who were bound to assist him, by every tie
of honor and fidelity. Marcellus, master-general of the cavalry
in Gaul, interpreting too strictly the jealous orders of the
court, beheld with supine indifference the distress of Julian,
and had restrained the troops under his command from marching to
the relief of Sens. If the Cæsar had dissembled in silence so
dangerous an insult, his person and authority would have been
exposed to the contempt of the world; and if an action so
criminal had been suffered to pass with impunity, the emperor
would have confirmed the suspicions, which received a very
specious color from his past conduct towards the princes of the
Flavian family. Marcellus was recalled, and gently dismissed from
his office.\textsuperscript[71] In his room Severus was appointed general of the
cavalry; an experienced soldier, of approved courage and
fidelity, who could advise with respect, and execute with zeal;
and who submitted, without reluctance to the supreme command
which Julian, by the inrerest of his patroness Eusebia, at length
obtained over the armies of Gaul.\textsuperscript[72] A very judicious plan of
operations was adopted for the approaching campaign. Julian
himself, at the head of the remains of the veteran bands, and of
some new levies which he had been permitted to form, boldly
penetrated into the centre of the German cantonments, and
carefully reestablished the fortifications of Saverne, in an
advantageous post, which would either check the incursions, or
intercept the retreat, of the enemy. At the same time, Barbatio,
general of the infantry, advanced from Milan with an army of
thirty thousand men, and passing the mountains, prepared to throw
a bridge over the Rhine, in the neighborhood of Basil. It was
reasonable to expect that the Alemanni, pressed on either side by
the Roman arms, would soon be forced to evacuate the provinces of
Gaul, and to hasten to the defence of their native country. But
the hopes of the campaign were defeated by the incapacity, or the
envy, or the secret instructions, of Barbatio; who acted as if he
had been the enemy of the Cæsar, and the secret ally of the
Barbarians. The negligence with which he permitted a troop of
pillagers freely to pass, and to return almost before the gates
of his camp, may be imputed to his want of abilities; but the
treasonable act of burning a number of boats, and a superfluous
stock of provisions, which would have been of the most essential
service to the army of Gaul, was an evidence of his hostile and
criminal intentions. The Germans despised an enemy who appeared
destitute either of power or of inclination to offend them; and
the ignominious retreat of Barbatio deprived Julian of the
expected support; and left him to extricate himself from a
hazardous situation, where he could neither remain with safety,
nor retire with honor.\textsuperscript[73]

\pagenote[71]{Ammian. xvi. 7. Libanius speaks rather more
advantageously of the military talents of Marcellus, Orat. x. p.
272. And Julian insinuates, that he would not have been so easily
recalled, unless he had given other reasons of offence to the
court, p. 278.}

\pagenote[72]{Severus, non discors, non arrogans, sed longa
militiæ frugalitate compertus; et eum recta præeuntem secuturus,
ut duetorem morigeran miles. Ammian xvi. 11. Zosimus, l. iii. p.
140.}

\pagenote[73]{On the design and failure of the cooperation
between Julian and Barbatio, see Ammianus (xvi. 11) and Libanius,
(Orat. x. p. 273.) Note: Barbatio seems to have allowed himself
to be surprised and defeated—M.}

As soon as they were delivered from the fears of invasion, the
Alemanni prepared to chastise the Roman youth, who presumed to
dispute the possession of that country, which they claimed as
their own by the right of conquest and of treaties. They employed
three days, and as many nights, in transporting over the Rhine
their military powers. The fierce Chnodomar, shaking the
ponderous javelin which he had victoriously wielded against the
brother of Magnentius, led the van of the Barbarians, and
moderated by his experience the martial ardor which his example
inspired.\textsuperscript[74] He was followed by six other kings, by ten princes
of regal extraction, by a long train of high-spirited nobles, and
by thirty-five thousand of the bravest warriors of the tribes of
Germany. The confidence derived from the view of their own
strength, was increased by the intelligence which they received
from a deserter, that the Cæsar, with a feeble army of thirteen
thousand men, occupied a post about one-and-twenty miles from
their camp of Strasburgh. With this inadequate force, Julian
resolved to seek and to encounter the Barbarian host; and the
chance of a general action was preferred to the tedious and
uncertain operation of separately engaging the dispersed parties
of the Alemanni. The Romans marched in close order, and in two
columns; the cavalry on the right, the infantry on the left; and
the day was so far spent when they appeared in sight of the
enemy, that Julian was desirous of deferring the battle till the
next morning, and of allowing his troops to recruit their
exhausted strength by the necessary refreshments of sleep and
food. Yielding, however, with some reluctance, to the clamors of
the soldiers, and even to the opinion of his council, he exhorted
them to justify by their valor the eager impatience, which, in
case of a defeat, would be universally branded with the epithets
of rashness and presumption. The trumpets sounded, the military
shout was heard through the field, and the two armies rushed with
equal fury to the charge. The Cæsar, who conducted in person his
right wing, depended on the dexterity of his archers, and the
weight of his cuirassiers. But his ranks were instantly broken by
an irregular mixture of light horse and of light infantry, and he
had the mortification of beholding the flight of six hundred of
his most renowned cuirassiers.\textsuperscript[75] The fugitives were stopped and
rallied by the presence and authority of Julian, who, careless of
his own safety, threw himself before them, and urging every
motive of shame and honor, led them back against the victorious
enemy. The conflict between the two lines of infantry was
obstinate and bloody. The Germans possessed the superiority of
strength and stature, the Romans that of discipline and temper;
and as the Barbarians, who served under the standard of the
empire, united the respective advantages of both parties, their
strenuous efforts, guided by a skilful leader, at length
determined the event of the day. The Romans lost four tribunes,
and two hundred and forty-three soldiers, in this memorable
battle of Strasburgh, so glorious to the Cæsar,\textsuperscript[76] and so
salutary to the afflicted provinces of Gaul. Six thousand of the
Alemanni were slain in the field, without including those who
were drowned in the Rhine, or transfixed with darts while they
attempted to swim across the river.\textsuperscript[77] Chnodomar himself was
surrounded and taken prisoner, with three of his brave
companions, who had devoted themselves to follow in life or death
the fate of their chieftain. Julian received him with military
pomp in the council of his officers; and expressing a generous
pity for the fallen state, dissembled his inward contempt for the
abject humiliation, of his captive. Instead of exhibiting the
vanquished king of the Alemanni, as a grateful spectacle to the
cities of Gaul, he respectfully laid at the feet of the emperor
this splendid trophy of his victory. Chnodomar experienced an
honorable treatment: but the impatient Barbarian could not long
survive his defeat, his confinement, and his exile.\textsuperscript[78]

\pagenote[74]{Ammianus (xvi. 12) describes with his inflated
eloquence the figure and character of Chnodomar. Audax et fidens
ingenti robore lacertorum, ubi ardor prœlii sperabatur immanis,
equo spumante sublimior, erectus in jaculum formidandæ
vastitatis, armorumque nitore conspicuus: antea strenuus et
miles, et utilis præter cæteros ductor... Decentium Cæsarem
superavit æquo marte congressus.}

\pagenote[75]{After the battle, Julian ventured to revive the
rigor of ancient discipline, by exposing these fugitives in
female apparel to the derision of the whole camp. In the next
campaign, these troops nobly retrieved their honor. Zosimus, l.
iii. p. 142.}

\pagenote[76]{Julian himself (ad S. P. Q. Athen. p. 279) speaks
of the battle of Strasburgh with the modesty of conscious merit;
Zosimus compares it with the victory of Alexander over Darius;
and yet we are at a loss to discover any of those strokes of
military genius which fix the attention of ages on the conduct
and success of a single day.}

\pagenote[77]{Ammianus, xvi. 12. Libanius adds 2000 more to the
number of the slain, (Orat. x. p. 274.) But these trifling
differences disappear before the 60,000 Barbarians, whom Zosimus
has sacrificed to the glory of his hero, (l. iii. p. 141.) We
might attribute this extravagant number to the carelessness of
transcribers, if this credulous or partial historian had not
swelled the army of 35,000 Alemanni to an innumerable multitude
of Barbarians,. It is our own fault if this detection does not
inspire us with proper distrust on similar occasions.}

\pagenote[78]{Ammian. xvi. 12. Libanius, Orat. x. p. 276.}

After Julian had repulsed the Alemanni from the provinces of the
Upper Rhine, he turned his arms against the Franks, who were
seated nearer to the ocean, on the confines of Gaul and Germany;
and who, from their numbers, and still more from their intrepid
valor, had ever been esteemed the most formidable of the
Barbarians.\textsuperscript[79] Although they were strongly actuated by the
allurements of rapine, they professed a disinterested love of
war; which they considered as the supreme honor and felicity of
human nature; and their minds and bodies were so completely
hardened by perpetual action, that, according to the lively
expression of an orator, the snows of winter were as pleasant to
them as the flowers of spring. In the month of December, which
followed the battle of Strasburgh, Julian attacked a body of six
hundred Franks, who had thrown themselves into two castles on the
Meuse.\textsuperscript[80] In the midst of that severe season they sustained, with
inflexible constancy, a siege of fifty-four days; till at length,
exhausted by hunger, and satisfied that the vigilance of the
enemy, in breaking the ice of the river, left them no hopes of
escape, the Franks consented, for the first time, to dispense
with the ancient law which commanded them to conquer or to die.
The Cæsar immediately sent his captives to the court of
Constantius, who, accepting them as a valuable present,\textsuperscript[81]
rejoiced in the opportunity of adding so many heroes to the
choicest troops of his domestic guards. The obstinate resistance
of this handful of Franks apprised Julian of the difficulties of
the expedition which he meditated for the ensuing spring, against
the whole body of the nation. His rapid diligence surprised and
astonished the active Barbarians. Ordering his soldiers to
provide themselves with biscuit for twenty days, he suddenly
pitched his camp near Tongres, while the enemy still supposed him
in his winter quarters of Paris, expecting the slow arrival of
his convoys from Aquitain. Without allowing the Franks to unite
or deliberate, he skilfully spread his legions from Cologne to
the ocean; and by the terror, as well as by the success, of his
arms, soon reduced the suppliant tribes to implore the clemency,
and to obey the commands, of their conqueror. The Chamavians
submissively retired to their former habitations beyond the
Rhine; but the Salians were permitted to possess their new
establishment of Toxandria, as the subjects and auxiliaries of
the Roman empire.\textsuperscript[82] The treaty was ratified by solemn oaths; and
perpetual inspectors were appointed to reside among the Franks,
with the authority of enforcing the strict observance of the
conditions. An incident is related, interesting enough in itself,
and by no means repugnant to the character of Julian, who
ingeniously contrived both the plot and the catastrophe of the
tragedy. When the Chamavians sued for peace, he required the son
of their king, as the only hostage on whom he could rely. A
mournful silence, interrupted by tears and groans, declared the
sad perplexity of the Barbarians; and their aged chief lamented
in pathetic language, that his private loss was now imbittered by
a sense of public calamity. While the Chamavians lay prostrate at
the foot of his throne, the royal captive, whom they believed to
have been slain, unexpectedly appeared before their eyes; and as
soon as the tumult of joy was hushed into attention, the Cæsar
addressed the assembly in the following terms: “Behold the son,
the prince, whom you wept. You had lost him by your fault. God
and the Romans have restored him to you. I shall still preserve
and educate the youth, rather as a monument of my own virtue,
than as a pledge of your sincerity. Should you presume to violate
the faith which you have sworn, the arms of the republic will
avenge the perfidy, not on the innocent, but on the guilty.” The
Barbarians withdrew from his presence, impressed with the warmest
sentiments of gratitude and admiration.\textsuperscript[83]

\pagenote[79]{Libanius (Orat. iii. p. 137) draws a very lively
picture of the manners of the Franks.}

\pagenote[80]{Ammianus, xvii. 2. Libanius, Orat. x. p. 278. The
Greek orator, by misapprehending a passage of Julian, has been
induced to represent the Franks as consisting of a thousand men;
and as his head was always full of the Peloponnesian war, he
compares them to the Lacedæmonians, who were besieged and taken
in the Island of Sphatoria.}

\pagenote[81]{Julian. ad S. P. Q. Athen. p. 280. Libanius, Orat.
x. p. 278. According to the expression of Libanius, the emperor,
which La Bleterie understands (Vie de Julien, p. 118) as an
honest confession, and Valesius (ad Ammian. xvii. 2) as a mean
evasion, of the truth. Dom Bouquet, (Historiens de France, tom.
i. p. 733,) by substituting another word, would suppress both the
difficulty and the spirit of this passage.}

\pagenote[82]{Ammian. xvii. 8. Zosimus, l. iii. p. 146-150, (his
narrative is darkened by a mixture of fable,) and Julian. ad S.
P. Q. Athen. p. 280. His expression. This difference of treatment
confirms the opinion that the Salian Franks were permitted to
retain the settlements in Toxandria. Note: A newly discovered
fragment of Eunapius, whom Zosimus probably transcribed,
illustrates this transaction. “Julian commanded the Romans to
abstain from all hostile measures against the Salians, neither to
waste or ravage \textit{their own} country, for he called every country
\textit{their own} which was surrendered without resistance or toil on
the part of the conquerors.” Mai, Script. Vez Nov. Collect. ii.
256, and Eunapius in Niebuhr, Byzant. Hist.}

\pagenote[83]{This interesting story, which Zosimus has abridged,
is related by Eunapius, (in Excerpt. Legationum, p. 15, 16, 17,)
with all the amplifications of Grecian rhetoric: but the silence
of Libanius, of Ammianus, and of Julian himself, renders the
truth of it extremely suspicious.}

It was not enough for Julian to have delivered the provinces of
Gaul from the Barbarians of Germany. He aspired to emulate the
glory of the first and most illustrious of the emperors; after
whose example, he composed his own commentaries of the Gallic
war.\textsuperscript[84] Cæsar has related, with conscious pride, the manner in
which he \textit{twice} passed the Rhine. Julian could boast, that
before he assumed the title of Augustus, he had carried the Roman
eagles beyond that great river in \textit{three} successful expeditions.\textsuperscript[85]
The consternation of the Germans, after the battle of
Strasburgh, encouraged him to the first attempt; and the
reluctance of the troops soon yielded to the persuasive eloquence
of a leader, who shared the fatigues and dangers which he imposed
on the meanest of the soldiers. The villages on either side of
the Meyn, which were plentifully stored with corn and cattle,
felt the ravages of an invading army. The principal houses,
constructed with some imitation of Roman elegance, were consumed
by the flames; and the Cæsar boldly advanced about ten miles,
till his progress was stopped by a dark and impenetrable forest,
undermined by subterraneous passages, which threatened with
secret snares and ambush every step of the assailants. The ground
was already covered with snow; and Julian, after repairing an
ancient castle which had been erected by Trajan, granted a truce
of ten months to the submissive Barbarians. At the expiration of
the truce, Julian undertook a second expedition beyond the Rhine,
to humble the pride of Surmar and Hortaire, two of the kings of
the Alemanni, who had been present at the battle of Strasburgh.
They promised to restore all the Roman captives who yet remained
alive; and as the Cæsar had procured an exact account from the
cities and villages of Gaul, of the inhabitants whom they had
lost, he detected every attempt to deceive him, with a degree of
readiness and accuracy, which almost established the belief of
his supernatural knowledge. His third expedition was still more
splendid and important than the two former. The Germans had
collected their military powers, and moved along the opposite
banks of the river, with a design of destroying the bridge, and
of preventing the passage of the Romans. But this judicious plan
of defence was disconcerted by a skilful diversion. Three hundred
light-armed and active soldiers were detached in forty small
boats, to fall down the stream in silence, and to land at some
distance from the posts of the enemy. They executed their orders
with so much boldness and celerity, that they had almost
surprised the Barbarian chiefs, who returned in the fearless
confidence of intoxication from one of their nocturnal festivals.
Without repeating the uniform and disgusting tale of slaughter
and devastation, it is sufficient to observe, that Julian
dictated his own conditions of peace to six of the haughtiest
kings of the Alemanni, three of whom were permitted to view the
severe discipline and martial pomp of a Roman camp. Followed by
twenty thousand captives, whom he had rescued from the chains of
the Barbarians, the Cæsar repassed the Rhine, after terminating a
war, the success of which has been compared to the ancient
glories of the Punic and Cimbric victories.

\pagenote[84]{Libanius, the friend of Julian, clearly insinuates
(Orat. ix. p. 178) that his hero had composed the history of his
Gallic campaigns But Zosimus (l. iii. p, 140) seems to have
derived his information only from the Orations and the Epistles
of Julian. The discourse which is addressed to the Athenians
contains an accurate, though general, account of the war against
the Germans.}

\pagenote[85]{See Ammian. xvii. 1, 10, xviii. 2, and Zosim. l.
iii. p. 144. Julian ad S. P. Q. Athen. p. 280.}

As soon as the valor and conduct of Julian had secured an
interval of peace, he applied himself to a work more congenial to
his humane and philosophic temper. The cities of Gaul, which had
suffered from the inroads of the Barbarians, he diligently
repaired; and seven important posts, between Mentz and the mouth
of the Rhine, are particularly mentioned, as having been rebuilt
and fortified by the order of Julian.\textsuperscript[86] The vanquished Germans
had submitted to the just but humiliating condition of preparing
and conveying the necessary materials. The active zeal of Julian
urged the prosecution of the work; and such was the spirit which
he had diffused among the troops, that the auxiliaries
themselves, waiving their exemption from any duties of fatigue,
contended in the most servile labors with the diligence of the
Roman soldiers. It was incumbent on the Cæsar to provide for the
subsistence, as well as for the safety, of the inhabitants and of
the garrisons. The desertion of the former, and the mutiny of the
latter, must have been the fatal and inevitable consequences of
famine. The tillage of the provinces of Gaul had been interrupted
by the calamities of war; but the scanty harvests of the
continent were supplied, by his paternal care, from the plenty of
the adjacent island. Six hundred large barks, framed in the
forest of the Ardennes, made several voyages to the coast of
Britain; and returning from thence, laden with corn, sailed up
the Rhine, and distributed their cargoes to the several towns and
fortresses along the banks of the river.\textsuperscript[87] The arms of Julian
had restored a free and secure navigation, which Constantinius
had offered to purchase at the expense of his dignity, and of a
tributary present of two thousand pounds of silver. The emperor
parsimoniously refused to his soldiers the sums which he granted
with a lavish and trembling hand to the Barbarians. The
dexterity, as well as the firmness, of Julian was put to a severe
trial, when he took the field with a discontented army, which had
already served two campaigns, without receiving any regular pay
or any extraordinary donative.\textsuperscript[88]

\pagenote[86]{Ammian. xviii. 2. Libanius, Orat. x. p. 279, 280.
Of these seven posts, four are at present towns of some
consequence; Bingen, Andernach, Bonn, and Nuyss. The other three,
Tricesimæ, Quadriburgium, and Castra Herculis, or Heraclea, no
longer subsist; but there is room to believe, that on the ground
of Quadriburgium the Dutch have constructed the fort of Schenk, a
name so offensive to the fastidious delicacy of Boileau. See
D’Anville, Notice de l’Ancienne Gaule, p. 183. Boileau, Epitre
iv. and the notes. Note: Tricesimæ, Kellen, Mannert, quoted by
Wagner. Heraclea, Erkeleus in the district of Juliers. St.
Martin, ii. 311.—M.}

\pagenote[87]{We may credit Julian himself, (Orat. ad S. P. Q.
Atheniensem, p. 280,) who gives a very particular account of the
transaction. Zosimus adds two hundred vessels more, (l. iii. p.
145.) If we compute the 600 corn ships of Julian at only seventy
tons each, they were capable of exporting 120,000 quarters, (see
Arbuthnot’s Weights and Measures, p. 237;) and the country which
could bear so large an exportation, must already have attained an
improved state of agriculture.}

\pagenote[88]{The troops once broke out into a mutiny,
immediately before the second passage of the Rhine. Ammian. xvii.
9.}

A tender regard for the peace and happiness of his subjects was
the ruling principle which directed, or seemed to direct, the
administration of Julian.\textsuperscript[89] He devoted the leisure of his winter
quarters to the offices of civil government; and affected to
assume, with more pleasure, the character of a magistrate than
that of a general. Before he took the field, he devolved on the
provincial governors most of the public and private causes which
had been referred to his tribunal; but, on his return, he
carefully revised their proceedings, mitigated the rigor of the
law, and pronounced a second judgment on the judges themselves.
Superior to the last temptation of virtuous minds, an indiscreet
and intemperate zeal for justice, he restrained, with calmness
and dignity, the warmth of an advocate, who prosecuted, for
extortion, the president of the Narbonnese province. “Who will
ever be found guilty,” exclaimed the vehement Delphidius, “if it
be enough to deny?” “And who,” replied Julian, “will ever be
innocent, if it be sufficient to affirm?” In the general
administration of peace and war, the interest of the sovereign is
commonly the same as that of his people; but Constantius would
have thought himself deeply injured, if the virtues of Julian had
defrauded him of any part of the tribute which he extorted from
an oppressed and exhausted country. The prince who was invested
with the ensigns of royalty, might sometimes presume to correct
the rapacious insolence of his inferior agents, to expose their
corrupt arts, and to introduce an equal and easier mode of
collection. But the management of the finances was more safely
intrusted to Florentius, prætorian præfect of Gaul, an effeminate
tyrant, incapable of pity or remorse: and the haughty minister
complained of the most decent and gentle opposition, while Julian
himself was rather inclined to censure the weakness of his own
behavior. The Cæsar had rejected, with abhorrence, a mandate for
the levy of an extraordinary tax; a new superindiction, which the
præfect had offered for his signature; and the faithful picture
of the public misery, by which he had been obliged to justify his
refusal, offended the court of Constantius. We may enjoy the
pleasure of reading the sentiments of Julian, as he expresses
them with warmth and freedom in a letter to one of his most
intimate friends. After stating his own conduct, he proceeds in
the following terms: “Was it possible for the disciple of Plato
and Aristotle to act otherwise than I have done? Could I abandon
the unhappy subjects intrusted to my care? Was I not called upon
to defend them from the repeated injuries of these unfeeling
robbers? A tribune who deserts his post is punished with death,
and deprived of the honors of burial. With what justice could I
pronounce \textit{his} sentence, if, in the hour of danger, I myself
neglected a duty far more sacred and far more important? God has
placed me in this elevated post; his providence will guard and
support me. Should I be condemned to suffer, I shall derive
comfort from the testimony of a pure and upright conscience.
Would to Heaven that I still possessed a counsellor like Sallust!
If they think proper to send me a successor, I shall submit
without reluctance; and had much rather improve the short
opportunity of doing good, than enjoy a long and lasting impunity
of evil.”\textsuperscript[90] The precarious and dependent situation of Julian
displayed his virtues and concealed his defects. The young hero
who supported, in Gaul, the throne of Constantius, was not
permitted to reform the vices of the government; but he had
courage to alleviate or to pity the distress of the people.
Unless he had been able to revive the martial spirit of the
Romans, or to introduce the arts of industry and refinement among
their savage enemies, he could not entertain any rational hopes
of securing the public tranquillity, either by the peace or
conquest of Germany. Yet the victories of Julian suspended, for a
short time, the inroads of the Barbarians, and delayed the ruin
of the Western Empire.

\pagenote[89]{Ammian. xvi. 5, xviii. 1. Mamertinus in Panegyr.
Vet. xi. 4}

\pagenote[90]{Ammian. xvii. 3. Julian. Epistol. xv. edit.
Spanheim. Such a conduct almost justifies the encomium of
Mamertinus. Ita illi anni spatia divisa sunt, ut aut Barbaros
domitet, aut civibus jura restituat, perpetuum professus, aut
contra hostem, aut contra vitia, certamen.}

His salutary influence restored the cities of Gaul, which had
been so long exposed to the evils of civil discord, Barbarian
war, and domestic tyranny; and the spirit of industry was revived
with the hopes of enjoyment. Agriculture, manufactures, and
commerce, again flourished under the protection of the laws; and
the \textit{curiæ}, or civil corporations, were again filled with useful
and respectable members: the youth were no longer apprehensive of
marriage; and married persons were no longer apprehensive of
posterity: the public and private festivals were celebrated with
customary pomp; and the frequent and secure intercourse of the
provinces displayed the image of national prosperity.\textsuperscript[91] A mind
like that of Julian must have felt the general happiness of which
he was the author; but he viewed, with particular satisfaction
and complacency, the city of Paris; the seat of his winter
residence, and the object even of his partial affection.\textsuperscript[92] That
splendid capital, which now embraces an ample territory on either
side of the Seine, was originally confined to the small island in
the midst of the river, from whence the inhabitants derived a
supply of pure and salubrious water. The river bathed the foot of
the walls; and the town was accessible only by two wooden
bridges. A forest overspread the northern side of the Seine, but
on the south, the ground, which now bears the name of the
University, was insensibly covered with houses, and adorned with
a palace and amphitheatre, baths, an aqueduct, and a field of
Mars for the exercise of the Roman troops. The severity of the
climate was tempered by the neighborhood of the ocean; and with
some precautions, which experience had taught, the vine and
fig-tree were successfully cultivated. But in remarkable winters,
the Seine was deeply frozen; and the huge pieces of ice that
floated down the stream, might be compared, by an Asiatic, to the
blocks of white marble which were extracted from the quarries of
Phrygia. The licentiousness and corruption of Antioch recalled to
the memory of Julian the severe and simple manners of his beloved
Lutetia;\textsuperscript[93] where the amusements of the theatre were unknown or
despised. He indignantly contrasted the effeminate Syrians with
the brave and honest simplicity of the Gauls, and almost forgave
the intemperance, which was the only stain of the Celtic
character.\textsuperscript[94] If Julian could now revisit the capital of France,
he might converse with men of science and genius, capable of
understanding and of instructing a disciple of the Greeks; he
might excuse the lively and graceful follies of a nation, whose
martial spirit has never been enervated by the indulgence of
luxury; and he must applaud the perfection of that inestimable
art, which softens and refines and embellishes the intercourse of
social life.

\pagenote[91]{Libanius, Orat. Parental. in Imp. Julian. c. 38, in
Fabricius Bibliothec. Græc. tom. vii. p. 263, 264.}

\pagenote[92]{See Julian. in Misopogon, p. 340, 341. The
primitive state of Paris is illustrated by Henry Valesius, (ad
Ammian. xx. 4,) his brother Hadrian Valesius, or de Valois, and
M. D’Anville, (in their respective Notitias of ancient Gaul,) the
Abbé de Longuerue, (Description de la France, tom. i. p. 12, 13,)
and M. Bonamy, (in the Mém. de l’Académie des Inscriptions, tom.
xv. p. 656-691.)}

\pagenote[93]{Julian, in Misopogon, p. 340. Leuce tia, or
Lutetia, was the ancient name of the city, which, according to
the fashion of the fourth century, assumed the territorial
appellation of \textit{Parisii}.}

\pagenote[94]{Julian in Misopogon, p. 359, 360.}

