\chapter{Julian Declared Emperor.}
\section{Part \thesection.}

\textit{Julian Is Declared Emperor By The Legions Of Gaul. — His March And
Success. — The Death Of Constantius. — Civil Administration Of Julian.}
\vspace{\onelineskip}

While the Romans languished under the ignominious tyranny of
eunuchs and bishops, the praises of Julian were repeated with
transport in every part of the empire, except in the palace of
Constantius. The barbarians of Germany had felt, and still
dreaded, the arms of the young Cæsar; his soldiers were the
companions of his victory; the grateful provincials enjoyed the
blessings of his reign; but the favorites, who had opposed his
elevation, were offended by his virtues; and they justly
considered the friend of the people as the enemy of the court. As
long as the fame of Julian was doubtful, the buffoons of the
palace, who were skilled in the language of satire, tried the
efficacy of those arts which they had so often practised with
success. They easily discovered, that his simplicity was not
exempt from affectation: the ridiculous epithets of a hairy
savage, of an ape invested with the purple, were applied to the
dress and person of the philosophic warrior; and his modest
despatches were stigmatized as the vain and elaborate fictions of
a loquacious Greek, a speculative soldier, who had studied the
art of war amidst the groves of the academy.\textsuperscript{1} The voice of
malicious folly was at length silenced by the shouts of victory;
the conqueror of the Franks and Alemanni could no longer be
painted as an object of contempt; and the monarch himself was
meanly ambitious of stealing from his lieutenant the honorable
reward of his labors. In the letters crowned with laurel, which,
according to ancient custom, were addressed to the provinces, the
name of Julian was omitted. “Constantius had made his
dispositions in person; \textit{he} had signalized his valor in the
foremost ranks; \textit{his} military conduct had secured the victory;
and the captive king of the barbarians was presented to \textit{him} on
the field of battle,” from which he was at that time distant
about forty days’ journey.\textsuperscript{2} So extravagant a fable was
incapable, however, of deceiving the public credulity, or even of
satisfying the pride of the emperor himself. Secretly conscious
that the applause and favor of the Romans accompanied the rising
fortunes of Julian, his discontented mind was prepared to receive
the subtle poison of those artful sycophants, who colored their
mischievous designs with the fairest appearances of truth and
candor.\textsuperscript{3} Instead of depreciating the merits of Julian, they
acknowledged, and even exaggerated, his popular fame, superior
talents, and important services. But they darkly insinuated, that
the virtues of the Cæsar might instantly be converted into the
most dangerous crimes, if the inconstant multitude should prefer
their inclinations to their duty; or if the general of a
victorious army should be tempted from his allegiance by the
hopes of revenge and independent greatness. The personal fears of
Constantius were interpreted by his council as a laudable anxiety
for the public safety; whilst in private, and perhaps in his own
breast, he disguised, under the less odious appellation of fear,
the sentiments of hatred and envy, which he had secretly
conceived for the inimitable virtues of Julian.

\pagenote[1]{Omnes qui plus poterant in palatio, adulandi
professores jam docti, recte consulta, prospereque completa
vertebant in deridiculum: talia sine modo strepentes insulse; in
odium venit cum victoriis suis; capella, non homo; ut hirsutum
Julianum carpentes, appellantesque loquacem talpam, et purpuratam
simiam, et litterionem Græcum: et his congruentia plurima atque
vernacula principi resonantes, audire hæc taliaque gestienti,
virtutes ejus obruere verbis impudentibus conabantur, et segnem
incessentes et timidum et umbratilem, gestaque secus verbis
comptioribus exornantem. Ammianus, s. xvii. 11. * Note: The
philosophers retaliated on the courtiers. Marius (says Eunapius
in a newly-discovered fragment) was wont to call his antagonist
Sylla a beast half lion and half fox. Constantius had nothing of
the lion, but was surrounded by a whole litter of foxes. Mai.
Script. Byz. Nov. Col. ii. 238. Niebuhr. Byzant. Hist. 66.—M.}

\pagenote[2]{Ammian. xvi. 12. The orator Themistius (iv. p. 56,
57) believed whatever was contained in the Imperial letters,
which were addressed to the senate of Constantinople Aurelius
Victor, who published his Abridgment in the last year of
Constantius, ascribes the German victories to the \textit{wisdom} of the
emperor, and the \textit{fortune} of the Cæsar. Yet the historian, soon
afterwards, was indebted to the favor or esteem of Julian for the
honor of a brass statue, and the important offices of consular of
the second Pannonia, and præfect of the city, Ammian. xxi. 10.}

\pagenote[3]{Callido nocendi artificio, accusatoriam diritatem
laudum titulis peragebant. .. Hæ voces fuerunt ad inflammanda
odia probria omnibus potentiores. See Mamertin, in Actione
Gratiarum in Vet Panegyr. xi. 5, 6.}

The apparent tranquillity of Gaul, and the imminent danger of the
eastern provinces, offered a specious pretence for the design
which was artfully concerted by the Imperial ministers. They
resolved to disarm the Cæsar; to recall those faithful troops who
guarded his person and dignity; and to employ, in a distant war
against the Persian monarch, the hardy veterans who had
vanquished, on the banks of the Rhine, the fiercest nations of
Germany. While Julian used the laborious hours of his winter
quarters at Paris in the administration of power, which, in his
hands, was the exercise of virtue, he was surprised by the hasty
arrival of a tribune and a notary, with positive orders, from the
emperor, which \textit{they} were directed to execute, and \textit{he} was
commanded not to oppose. Constantius signified his pleasure, that
four entire legions, the Celtæ, and Petulants, the Heruli, and
the Batavians, should be separated from the standard of Julian,
under which they had acquired their fame and discipline; that in
each of the remaining bands three hundred of the bravest youths
should be selected; and that this numerous detachment, the
strength of the Gallic army, should instantly begin their march,
and exert their utmost diligence to arrive, before the opening of
the campaign, on the frontiers of Persia.\textsuperscript{4} The Cæsar foresaw and
lamented the consequences of this fatal mandate. Most of the
auxiliaries, who engaged their voluntary service, had stipulated,
that they should never be obliged to pass the Alps. The public
faith of Rome, and the personal honor of Julian, had been pledged
for the observance of this condition. Such an act of treachery
and oppression would destroy the confidence, and excite the
resentment, of the independent warriors of Germany, who
considered truth as the noblest of their virtues, and freedom as
the most valuable of their possessions. The legionaries, who
enjoyed the title and privileges of Romans, were enlisted for the
general defence of the republic; but those mercenary troops heard
with cold indifference the antiquated names of the republic and
of Rome. Attached, either from birth or long habit, to the
climate and manners of Gaul, they loved and admired Julian; they
despised, and perhaps hated, the emperor; they dreaded the
laborious march, the Persian arrows, and the burning deserts of
Asia. They claimed as their own the country which they had saved;
and excused their want of spirit, by pleading the sacred and more
immediate duty of protecting their families and friends.

The apprehensions of the Gauls were derived from the knowledge of
the impending and inevitable danger. As soon as the provinces
were exhausted of their military strength, the Germans would
violate a treaty which had been imposed on their fears; and
notwithstanding the abilities and valor of Julian, the general of
a nominal army, to whom the public calamities would be imputed,
must find himself, after a vain resistance, either a prisoner in
the camp of the barbarians, or a criminal in the palace of
Constantius. If Julian complied with the orders which he had
received, he subscribed his own destruction, and that of a people
who deserved his affection. But a positive refusal was an act of
rebellion, and a declaration of war. The inexorable jealousy of
the emperor, the peremptory, and perhaps insidious, nature of his
commands, left not any room for a fair apology, or candid
interpretation; and the dependent station of the Cæsar scarcely
allowed him to pause or to deliberate. Solitude increased the
perplexity of Julian; he could no longer apply to the faithful
counsels of Sallust, who had been removed from his office by the
judicious malice of the eunuchs: he could not even enforce his
representations by the concurrence of the ministers, who would
have been afraid or ashamed to approve the ruin of Gaul. The
moment had been chosen, when Lupicinus,\textsuperscript{5} the general of the
cavalry, was despatched into Britain, to repulse the inroads of
the Scots and Picts; and Florentius was occupied at Vienna by the
assessment of the tribute. The latter, a crafty and corrupt
statesman, declining to assume a responsible part on this
dangerous occasion, eluded the pressing and repeated invitations
of Julian, who represented to him, that in every important
measure, the presence of the præfect was indispensable in the
council of the prince. In the mean while the Cæsar was oppressed
by the rude and importunate solicitations of the Imperial
messengers, who presumed to suggest, that if he expected the
return of his ministers, he would charge himself with the guilt
of the delay, and reserve for them the merit of the execution.
Unable to resist, unwilling to comply, Julian expressed, in the
most serious terms, his wish, and even his intention, of
resigning the purple, which he could not preserve with honor, but
which he could not abdicate with safety.

\pagenote[4]{The minute interval, which may be interposed,
between the \textit{hyeme adultâ} and the \textit{primo vere} of Ammianus, (xx.
l. 4,) instead of allowing a sufficient space for a march of
three thousand miles, would render the orders of Constantius as
extravagant as they were unjust. The troops of Gaul could not
have reached Syria till the end of autumn. The memory of Ammianus
must have been inaccurate, and his language incorrect. * Note:
The late editor of Ammianus attempts to vindicate his author from
the charge of inaccuracy. “It is clear, from the whole course of
the narrative, that Constantius entertained this design of
demanding his troops from Julian, immediately after the taking of
Amida, in the autumn of the preceding year, and had transmitted
his orders into Gaul, before it was known that Lupicinus had gone
into Britain with the Herulians and Batavians.” Wagner, note to
Amm. xx. 4. But it seems also clear that the troops were in
winter quarters (hiemabant) when the orders arrived. Ammianus can
scarcely be acquitted of incorrectness in his language at
least.—M}

\pagenote[5]{Ammianus, xx. l. The valor of Lupicinus, and his
military skill, are acknowledged by the historian, who, in his
affected language, accuses the general of exalting the horns of
his pride, bellowing in a tragic tone, and exciting a doubt
whether he was more cruel or avaricious. The danger from the
Scots and Picts was so serious that Julian himself had some
thoughts of passing over into the island.}

After a painful conflict, Julian was compelled to acknowledge,
that obedience was the virtue of the most eminent subject, and
that the sovereign alone was entitled to judge of the public
welfare. He issued the necessary orders for carrying into
execution the commands of Constantius; a part of the troops began
their march for the Alps; and the detachments from the several
garrisons moved towards their respective places of assembly. They
advanced with difficulty through the trembling and affrighted
crowds of provincials, who attempted to excite their pity by
silent despair, or loud lamentations, while the wives of the
soldiers, holding their infants in their arms, accused the
desertion of their husbands, in the mixed language of grief, of
tenderness, and of indignation. This scene of general distress
afflicted the humanity of the Cæsar; he granted a sufficient
number of post-wagons to transport the wives and families of the
soldiers,\textsuperscript{6} endeavored to alleviate the hardships which he was
constrained to inflict, and increased, by the most laudable arts,
his own popularity, and the discontent of the exiled troops. The
grief of an armed multitude is soon converted into rage; their
licentious murmurs, which every hour were communicated from tent
to tent with more boldness and effect, prepared their minds for
the most daring acts of sedition; and by the connivance of their
tribunes, a seasonable libel was secretly dispersed, which
painted in lively colors the disgrace of the Cæsar, the
oppression of the Gallic army, and the feeble vices of the tyrant
of Asia. The servants of Constantius were astonished and alarmed
by the progress of this dangerous spirit. They pressed the Cæsar
to hasten the departure of the troops; but they imprudently
rejected the honest and judicious advice of Julian; who proposed
that they should not march through Paris, and suggested the
danger and temptation of a last interview.

\pagenote[6]{He granted them the permission of the \textit{cursus
clavularis}, or \textit{clabularis}. These post-wagons are often
mentioned in the Code, and were supposed to carry fifteen hundred
pounds weight. See Vales. ad Ammian. xx. 4.}

As soon as the approach of the troops was announced, the Cæsar
went out to meet them, and ascended his tribunal, which had been
erected in a plain before the gates of the city. After
distinguishing the officers and soldiers, who by their rank or
merit deserved a peculiar attention, Julian addressed himself in
a studied oration to the surrounding multitude: he celebrated
their exploits with grateful applause; encouraged them to accept,
with alacrity, the honor of serving under the eye of a powerful
and liberal monarch; and admonished them, that the commands of
Augustus required an instant and cheerful obedience. The
soldiers, who were apprehensive of offending their general by an
indecent clamor, or of belying their sentiments by false and
venal acclamations, maintained an obstinate silence; and after a
short pause, were dismissed to their quarters. The principal
officers were entertained by the Cæsar, who professed, in the
warmest language of friendship, his desire and his inability to
reward, according to their deserts, the brave companions of his
victories. They retired from the feast, full of grief and
perplexity; and lamented the hardship of their fate, which tore
them from their beloved general and their native country. The
only expedient which could prevent their separation was boldly
agitated and approved; the popular resentment was insensibly
moulded into a regular conspiracy; their just reasons of
complaint were heightened by passion, and their passions were
inflamed by wine; as, on the eve of their departure, the troops
were indulged in licentious festivity. At the hour of midnight,
the impetuous multitude, with swords, and bows, and torches in
their hands, rushed into the suburbs; encompassed the palace;\textsuperscript{7}
and, careless of future dangers, pronounced the fatal and
irrevocable words, Julian Augustus! The prince, whose anxious
suspense was interrupted by their disorderly acclamations,
secured the doors against their intrusion; and as long as it was
in his power, secluded his person and dignity from the accidents
of a nocturnal tumult. At the dawn of day, the soldiers, whose
zeal was irritated by opposition, forcibly entered the palace,
seized, with respectful violence, the object of their choice,
guarded Julian with drawn swords through the streets of Paris,
placed him on the tribunal, and with repeated shouts saluted him
as their emperor. Prudence, as well as loyalty, inculcated the
propriety of resisting their treasonable designs; and of
preparing, for his oppressed virtue, the excuse of violence.
Addressing himself by turns to the multitude and to individuals,
he sometimes implored their mercy, and sometimes expressed his
indignation; conjured them not to sully the fame of their
immortal victories; and ventured to promise, that if they would
immediately return to their allegiance, he would undertake to
obtain from the emperor not only a free and gracious pardon, but
even the revocation of the orders which had excited their
resentment. But the soldiers, who were conscious of their guilt,
chose rather to depend on the gratitude of Julian, than on the
clemency of the emperor. Their zeal was insensibly turned into
impatience, and their impatience into rage. The inflexible Cæsar
sustained, till the third hour of the day, their prayers, their
reproaches, and their menaces; nor did he yield, till he had been
repeatedly assured, that if he wished to live, he must consent to
reign. He was exalted on a shield in the presence, and amidst the
unanimous acclamations, of the troops; a rich military collar,
which was offered by chance, supplied the want of a diadem;\textsuperscript{8} the
ceremony was concluded by the promise of a moderate donative; and
the new emperor, overwhelmed with real or affected grief retired
into the most secret recesses of his apartment.\textsuperscript{10}

\pagenote[7]{Most probably the palace of the baths,
(\textit{Thermarum},) of which a solid and lofty hall still subsists in
the \textit{Rue de la Harpe}. The buildings covered a considerable space
of the modern quarter of the university; and the gardens, under
the Merovingian kings, communicated with the abbey of St. Germain
des Prez. By the injuries of time and the Normans, this ancient
palace was reduced, in the twelfth century, to a maze of ruins,
whose dark recesses were the scene of licentious love.

Explicat aula sinus montemque amplectitur alis; Multiplici latebra
scelerum tersura ruborem. .... pereuntis sæpe pudoris Celatura
nefas, Venerisque accommoda furtis.

(These lines are quoted from the Architrenius, l. iv. c. 8, a
poetical work of John de Hauteville, or Hanville, a monk of St.
Alban’s, about the year 1190. See Warton’s History of English
Poetry, vol. i. dissert. ii.) Yet such \textit{thefts} might be less
pernicious to mankind than the theological disputes of the
Sorbonne, which have been since agitated on the same ground.
Bonamy, Mém. de l’Académie, tom. xv. p. 678-632}

\pagenote[8]{Even in this tumultuous moment, Julian attended to
the forms of superstitious ceremony, and obstinately refused the
inauspicious use of a female necklace, or a horse collar, which
the impatient soldiers would have employed in the room of a
diadem. ----An equal proportion of gold and silver, five pieces
of the former one pound of the latter; the whole amounting to
about five pounds ten shillings of our money.}

\pagenote[10]{For the whole narrative of this revolt, we may
appeal to authentic and original materials; Julian himself, (ad
S. P. Q. Atheniensem, p. 282, 283, 284,) Libanius, (Orat.
Parental. c. 44-48, in Fabricius, Bibliot. Græc. tom. vii. p.
269-273,) Ammianus, (xx. 4,) and Zosimus, (l. iii. p. 151, 152,
153.) who, in the reign of Julian, appears to follow the more
respectable authority of Eunapius. With such guides we \textit{might}
neglect the abbreviators and ecclesiastical historians.}

The grief of Julian could proceed only from his innocence; out
his innocence must appear extremely doubtful\textsuperscript{11} in the eyes of
those who have learned to suspect the motives and the professions
of princes. His lively and active mind was susceptible of the
various impressions of hope and fear, of gratitude and revenge,
of duty and of ambition, of the love of fame, and of the fear of
reproach. But it is impossible for us to calculate the respective
weight and operation of these sentiments; or to ascertain the
principles of action which might escape the observation, while
they guided, or rather impelled, the steps of Julian himself. The
discontent of the troops was produced by the malice of his
enemies; their tumult was the natural effect of interest and of
passion; and if Julian had tried to conceal a deep design under
the appearances of chance, he must have employed the most
consummate artifice without necessity, and probably without
success. He solemnly declares, in the presence of Jupiter, of the
Sun, of Mars, of Minerva, and of all the other deities, that till
the close of the evening which preceded his elevation, he was
utterly ignorant of the designs of the soldiers;\textsuperscript{12} and it may
seem ungenerous to distrust the honor of a hero and the truth of
a philosopher. Yet the superstitious confidence that Constantius
was the enemy, and that he himself was the favorite, of the gods,
might prompt him to desire, to solicit, and even to hasten the
auspicious moment of his reign, which was predestined to restore
the ancient religion of mankind. When Julian had received the
intelligence of the conspiracy, he resigned himself to a short
slumber; and afterwards related to his friends that he had seen
the genius of the empire waiting with some impatience at his
door, pressing for admittance, and reproaching his want of spirit
and ambition.\textsuperscript{13} Astonished and perplexed, he addressed his
prayers to the great Jupiter, who immediately signified, by a
clear and manifest omen, that he should submit to the will of
heaven and of the army. The conduct which disclaims the ordinary
maxims of reason, excites our suspicion and eludes our inquiry.
Whenever the spirit of fanaticism, at once so credulous and so
crafty, has insinuated itself into a noble mind, it insensibly
corrodes the vital principles of virtue and veracity.

\pagenote[11]{Eutropius, a respectable witness, uses a doubtful
expression, “consensu militum.” (x. 15.) Gregory Nazianzen, whose
ignorance night excuse his fanaticism, directly charges the
apostate with presumption, madness, and impious rebellion, Orat.
iii. p. 67.}

\pagenote[12]{Julian. ad S. P. Q. Athen. p. 284. The \textit{devout}
Abbé de la Bleterie (Vie de Julien, p. 159) is almost inclined to
respect the \textit{devout} protestations of a Pagan.}

\pagenote[13]{Ammian. xx. 5, with the note of Lindenbrogius on
the Genius of the empire. Julian himself, in a confidential
letter to his friend and physician, Oribasius, (Epist. xvii. p.
384,) mentions another dream, to which, before the event, he gave
credit; of a stately tree thrown to the ground, of a small plant
striking a deep root into the earth. Even in his sleep, the mind
of the Cæsar must have been agitated by the hopes and fears of
his fortune. Zosimus (l. iii. p. 155) relates a subsequent
dream.}

To moderate the zeal of his party, to protect the persons of his
enemies,\textsuperscript{14} to defeat and to despise the secret enterprises which
were formed against his life and dignity, were the cares which
employed the first days of the reign of the new emperor. Although
he was firmly resolved to maintain the station which he had
assumed, he was still desirous of saving his country from the
calamities of civil war, of declining a contest with the superior
forces of Constantius, and of preserving his own character from
the reproach of perfidy and ingratitude. Adorned with the ensigns
of military and imperial pomp, Julian showed himself in the field
of Mars to the soldiers, who glowed with ardent enthusiasm in the
cause of their pupil, their leader, and their friend. He
recapitulated their victories, lamented their sufferings,
applauded their resolution, animated their hopes, and checked
their impetuosity; nor did he dismiss the assembly, till he had
obtained a solemn promise from the troops, that if the emperor of
the East would subscribe an equitable treaty, they would renounce
any views of conquest, and satisfy themselves with the tranquil
possession of the Gallic provinces. On this foundation he
composed, in his own name, and in that of the army, a specious
and moderate epistle,\textsuperscript{15} which was delivered to Pentadius, his
master of the offices, and to his chamberlain Eutherius; two
ambassadors whom he appointed to receive the answer, and observe
the dispositions of Constantius. This epistle is inscribed with
the modest appellation of Cæsar; but Julian solicits in a
peremptory, though respectful, manner, the confirmation of the
title of Augustus. He acknowledges the irregularity of his own
election, while he justifies, in some measure, the resentment and
violence of the troops which had extorted his reluctant consent.
He allows the supremacy of his brother Constantius; and engages
to send him an annual present of Spanish horses, to recruit his
army with a select number of barbarian youths, and to accept from
his choice a Prætorian præfect of approved discretion and
fidelity. But he reserves for himself the nomination of his other
civil and military officers, with the troops, the revenue, and
the sovereignty of the provinces beyond the Alps. He admonishes
the emperor to consult the dictates of justice; to distrust the
arts of those venal flatterers, who subsist only by the discord
of princes; and to embrace the offer of a fair and honorable
treaty, equally advantageous to the republic and to the house of
Constantine. In this negotiation Julian claimed no more than he
already possessed. The delegated authority which he had long
exercised over the provinces of Gaul, Spain, and Britain, was
still obeyed under a name more independent and august. The
soldiers and the people rejoiced in a revolution which was not
stained even with the blood of the guilty. Florentius was a
fugitive; Lupicinus a prisoner. The persons who were disaffected
to the new government were disarmed and secured; and the vacant
offices were distributed, according to the recommendation of
merit, by a prince who despised the intrigues of the palace, and
the clamors of the soldiers.\textsuperscript{16}

\pagenote[14]{The difficult situation of the prince of a
rebellious army is finely described by Tacitus, (Hist. 1, 80-85.)
But Otho had much more guilt, and much less abilities, than
Julian.}

\pagenote[15]{To this ostensible epistle he added, says Ammianus,
private letters, objurgatorias et mordaces, which the historian
had not seen, and would not have published. Perhaps they never
existed.}

\pagenote[16]{See the first transactions of his reign, in Julian.
ad S. P. Q. Athen. p. 285, 286. Ammianus, xx. 5, 8. Liban. Orat.
Parent. c. 49, 50, p. 273-275.}

The negotiations of peace were accompanied and supported by the
most vigorous preparations for war. The army, which Julian held
in readiness for immediate action, was recruited and augmented by
the disorders of the times. The cruel persecutions of the faction
of Magnentius had filled Gaul with numerous bands of outlaws and
robbers. They cheerfully accepted the offer of a general pardon
from a prince whom they could trust, submitted to the restraints
of military discipline, and retained only their implacable hatred
to the person and government of Constantius.\textsuperscript{17} As soon as the
season of the year permitted Julian to take the field, he
appeared at the head of his legions; threw a bridge over the
Rhine in the neighborhood of Cleves; and prepared to chastise the
perfidy of the Attuarii, a tribe of Franks, who presumed that
they might ravage, with impunity, the frontiers of a divided
empire. The difficulty, as well as glory, of this enterprise,
consisted in a laborious march; and Julian had conquered, as soon
as he could penetrate into a country, which former princes had
considered as inaccessible. After he had given peace to the
Barbarians, the emperor carefully visited the fortifications
along the Qhine from Cleves to Basil; surveyed, with peculiar
attention, the territories which he had recovered from the hands
of the Alemanni, passed through Besançon,\textsuperscript{18} which had severely
suffered from their fury, and fixed his headquarters at Vienna
for the ensuing winter. The barrier of Gaul was improved and
strengthened with additional fortifications; and Julian
entertained some hopes that the Germans, whom he had so often
vanquished, might, in his absence, be restrained by the terror of
his name. Vadomair\textsuperscript{19} was the only prince of the Alemanni whom he
esteemed or feared and while the subtle Barbarian affected to
observe the faith of treaties, the progress of his arms
threatened the state with an unseasonable and dangerous war. The
policy of Julian condescended to surprise the prince of the
Alemanni by his own arts: and Vadomair, who, in the character of
a friend, had incautiously accepted an invitation from the Roman
governors, was seized in the midst of the entertainment, and sent
away prisoner into the heart of Spain. Before the Barbarians were
recovered from their amazement, the emperor appeared in arms on
the banks of the Rhine, and, once more crossing the river,
renewed the deep impressions of terror and respect which had been
already made by four preceding expeditions.\textsuperscript{20}

\pagenote[17]{Liban. Orat. Parent. c. 50, p. 275, 276. A strange
disorder, since it continued above seven years. In the factions
of the Greek republics, the exiles amounted to 20,000 persons;
and Isocrates assures Philip, that it would be easier to raise an
army from the vagabonds than from the cities. See Hume’s Essays,
tom. i. p. 426, 427.}

\pagenote[18]{Julian (Epist. xxxviii. p. 414) gives a short
description of Vesontio, or Besançon; a rocky peninsula almost
encircled by the River Doux; once a magnificent city, filled with
temples, \&c., now reduced to a small town, emerging, however,
from its ruins.}

\pagenote[19]{Vadomair entered into the Roman service, and was
promoted from a barbarian kingdom to the military rank of duke of
Phœnicia. He still retained the same artful character, (Ammian.
xxi. 4;) but under the reign of Valens, he signalized his valor
in the Armenian war, (xxix. 1.)}

\pagenote[20]{Ammian. xx. 10, xxi. 3, 4. Zosimus, l. iii. p.
155.}

\section{Part \thesection.}

The ambassadors of Julian had been instructed to execute, with
the utmost diligence, their important commission. But, in their
passage through Italy and Illyricum, they were detained by the
tedious and affected delays of the provincial governors; they
were conducted by slow journeys from Constantinople to Cæsarea in
Cappadocia; and when at length they were admitted to the presence
of Constantius, they found that he had already conceived, from
the despatches of his own officers, the most unfavorable opinion
of the conduct of Julian, and of the Gallic army. The letters
were heard with impatience; the trembling messengers were
dismissed with indignation and contempt; and the looks, gestures,
the furious language of the monarch, expressed the disorder of
his soul. The domestic connection, which might have reconciled
the brother and the husband of Helena, was recently dissolved by
the death of that princess, whose pregnancy had been several
times fruitless, and was at last fatal to herself.\textsuperscript{21} The empress
Eusebia had preserved, to the last moment of her life, the warm,
and even jealous, affection which she had conceived for Julian;
and her mild influence might have moderated the resentment of a
prince, who, since her death, was abandoned to his own passions,
and to the arts of his eunuchs. But the terror of a foreign
invasion obliged him to suspend the punishment of a private
enemy: he continued his march towards the confines of Persia, and
thought it sufficient to signify the conditions which might
entitle Julian and his guilty followers to the clemency of their
offended sovereign. He required, that the presumptuous Cæsar
should expressly renounce the appellation and rank of Augustus,
which he had accepted from the rebels; that he should descend to
his former station of a limited and dependent minister; that he
should vest the powers of the state and army in the hands of
those officers who were appointed by the Imperial court; and that
he should trust his safety to the assurances of pardon, which
were announced by Epictetus, a Gallic bishop, and one of the
Arian favorites of Constantius. Several months were ineffectually
consumed in a treaty which was negotiated at the distance of
three thousand miles between Paris and Antioch; and, as soon as
Julian perceived that his modest and respectful behavior served
only to irritate the pride of an implacable adversary, he boldly
resolved to commit his life and fortune to the chance of a civil
war. He gave a public and military audience to the quæstor
Leonas: the haughty epistle of Constantius was read to the
attentive multitude; and Julian protested, with the most
flattering deference, that he was ready to resign the title of
Augustus, if he could obtain the consent of those whom he
acknowledged as the authors of his elevation. The faint proposal
was impetuously silenced; and the acclamations of “Julian
Augustus, continue to reign, by the authority of the army, of the
people, of the republic which you have saved,” thundered at once
from every part of the field, and terrified the pale ambassador
of Constantius. A part of the letter was afterwards read, in
which the emperor arraigned the ingratitude of Julian, whom he
had invested with the honors of the purple; whom he had educated
with so much care and tenderness; whom he had preserved in his
infancy, when he was left a helpless orphan.

“An orphan!” interrupted Julian, who justified his cause by
indulging his passions: “does the assassin of my family reproach
me that I was left an orphan? He urges me to revenge those
injuries which I have long studied to forget.” The assembly was
dismissed; and Leonas, who, with some difficulty, had been
protected from the popular fury, was sent back to his master with
an epistle, in which Julian expressed, in a strain of the most
vehement eloquence, the sentiments of contempt, of hatred, and of
resentment, which had been suppressed and imbittered by the
dissimulation of twenty years. After this message, which might be
considered as a signal of irreconcilable war, Julian, who, some
weeks before, had celebrated the Christian festival of the
Epiphany,\textsuperscript{22} made a public declaration that he committed the care
of his safety to the Immortal Gods; and thus publicly renounced
the religion as well as the friendship of Constantius.\textsuperscript{23}

\pagenote[21]{Her remains were sent to Rome, and interred near
those of her sister Constantina, in the suburb of the \textit{Via
Nomentana}. Ammian. xxi. 1. Libanius has composed a very weak
apology, to justify his hero from a very absurd charge of
poisoning his wife, and rewarding her physician with his mother’s
jewels. (See the seventh of seventeen new orations, published at
Venice, 1754, from a MS. in St. Mark’s Library, p. 117-127.)
Elpidius, the Prætorian præfect of the East, to whose evidence
the accuser of Julian appeals, is arraigned by Libanius, as
\textit{effeminate} and ungrateful; yet the religion of Elpidius is
praised by Jerom, (tom. i. p. 243,) and his Ammianus (xxi. 6.)}

\pagenote[22]{Feriarum die quem celebrantes mense Januario,
Christiani \textit{Epiphania} dictitant, progressus in eorum ecclesiam,
solemniter numine orato discessit. Ammian. xxi. 2. Zonaras
observes, that it was on Christmas day, and his assertion is not
inconsistent; since the churches of Egypt, Asia, and perhaps
Gaul, celebrated on the same day (the sixth of January) the
nativity and the baptism of their Savior. The Romans, as ignorant
as their brethren of the real date of his birth, fixed the solemn
festival to the 25th of December, the \textit{Brumalia}, or winter
solstice, when the Pagans annually celebrated the birth of the
sun. See Bingham’s Antiquities of the Christian Church, l. xx. c.
4, and Beausobre, Hist. Critique du Manicheismo tom. ii. p.
690-700.}

\pagenote[23]{The public and secret negotiations between
Constantius and Julian must be extracted, with some caution, from
Julian himself. (Orat. ad S. P. Q. Athen. p. 286.) Libanius,
(Orat. Parent. c. 51, p. 276,) Ammianus, (xx. 9,) Zosimus, (l.
iii. p. 154,) and even Zonaras, (tom. ii. l. xiii. p. 20, 21,
22,) who, on this occasion, appears to have possessed and used
some valuable materials.}

The situation of Julian required a vigorous and immediate
resolution. He had discovered, from intercepted letters, that his
adversary, sacrificing the interest of the state to that of the
monarch, had again excited the Barbarians to invade the provinces
of the West. The position of two magazines, one of them collected
on the banks of the Lake of Constance, the other formed at the
foot of the Cottian Alps, seemed to indicate the march of two
armies; and the size of those magazines, each of which consisted
of six hundred thousand quarters of wheat, or rather flour,\textsuperscript{24}
was a threatening evidence of the strength and numbers of the
enemy who prepared to surround him. But the Imperial legions were
still in their distant quarters of Asia; the Danube was feebly
guarded; and if Julian could occupy, by a sudden incursion, the
important provinces of Illyricum, he might expect that a people
of soldiers would resort to his standard, and that the rich mines
of gold and silver would contribute to the expenses of the civil
war. He proposed this bold enterprise to the assembly of the
soldiers; inspired them with a just confidence in their general,
and in themselves; and exhorted them to maintain their reputation
of being terrible to the enemy, moderate to their
fellow-citizens, and obedient to their officers. His spirited
discourse was received with the loudest acclamations, and the
same troops which had taken up arms against Constantius, when he
summoned them to leave Gaul, now declared with alacrity, that
they would follow Julian to the farthest extremities of Europe or
Asia. The oath of fidelity was administered; and the soldiers,
clashing their shields, and pointing their drawn swords to their
throats, devoted themselves, with horrid imprecations, to the
service of a leader whom they celebrated as the deliverer of Gaul
and the conqueror of the Germans.\textsuperscript{25} This solemn engagement,
which seemed to be dictated by affection rather than by duty, was
singly opposed by Nebridius, who had been admitted to the office
of Prætorian præfect. That faithful minister, alone and
unassisted, asserted the rights of Constantius, in the midst of
an armed and angry multitude, to whose fury he had almost fallen
an honorable, but useless sacrifice. After losing one of his
hands by the stroke of a sword, he embraced the knees of the
prince whom he had offended. Julian covered the præfect with his
Imperial mantle, and, protecting him from the zeal of his
followers, dismissed him to his own house, with less respect than
was perhaps due to the virtue of an enemy.\textsuperscript{26} The high office of
Nebridius was bestowed on Sallust; and the provinces of Gaul,
which were now delivered from the intolerable oppression of
taxes, enjoyed the mild and equitable administration of the
friend of Julian, who was permitted to practise those virtues
which he had instilled into the mind of his pupil.\textsuperscript{27}

\pagenote[24]{Three hundred myriads, or three millions of
\textit{medimni}, a corn measure familiar to the Athenians, and which
contained six Roman \textit{modii}. Julian explains, like a soldier and
a statesman, the danger of his situation, and the necessity and
advantages of an offensive war, (ad S. P. Q. Athen. p. 286,
287.)}

\pagenote[25]{See his oration, and the behavior of the troops, in
Ammian. xxi. 5.}

\pagenote[26]{He sternly refused his hand to the suppliant
præfect, whom he sent into Tuscany. (Ammian. xxi. 5.) Libanius,
with savage fury, insults Nebridius, applauds the soldiers, and
almost censures the humanity of Julian. (Orat. Parent. c. 53, p.
278.)}

\pagenote[27]{Ammian. xxi. 8. In this promotion, Julian obeyed
the law which he publicly imposed on himself. Neque civilis
quisquam judex nec militaris rector, alio quodam præter merita
suffragante, ad potiorem veniat gradum. (Ammian. xx. 5.) Absence
did not weaken his regard for Sallust, with whose name (A. D.
363) he honored the consulship.}

The hopes of Julian depended much less on the number of his
troops, than on the celerity of his motions. In the execution of
a daring enterprise, he availed himself of every precaution, as
far as prudence could suggest; and where prudence could no longer
accompany his steps, he trusted the event to valor and to
fortune. In the neighborhood of Basil he assembled and divided
his army.\textsuperscript{28} One body, which consisted of ten thousand men, was
directed under the command of Nevitta, general of the cavalry, to
advance through the midland parts of Rhætia and Noricum. A
similar division of troops, under the orders of Jovius and
Jovinus, prepared to follow the oblique course of the highways,
through the Alps, and the northern confines of Italy. The
instructions to the generals were conceived with energy and
precision: to hasten their march in close and compact columns,
which, according to the disposition of the ground, might readily
be changed into any order of battle; to secure themselves against
the surprises of the night by strong posts and vigilant guards;
to prevent resistance by their unexpected arrival; to elude
examination by their sudden departure; to spread the opinion of
their strength, and the terror of his name; and to join their
sovereign under the walls of Sirmium. For himself Julian had
reserved a more difficult and extraordinary part. He selected
three thousand brave and active volunteers, resolved, like their
leader, to cast behind them every hope of a retreat; at the head
of this faithful band, he fearlessly plunged into the recesses of
the Marcian, or Black Forest, which conceals the sources of the
Danube;\textsuperscript{29} and, for many days, the fate of Julian was unknown to
the world. The secrecy of his march, his diligence, and vigor,
surmounted every obstacle; he forced his way over mountains and
morasses, occupied the bridges or swam the rivers, pursued his
direct course,\textsuperscript{30} without reflecting whether he traversed the
territory of the Romans or of the Barbarians, and at length
emerged, between Ratisbon and Vienna, at the place where he
designed to embark his troops on the Danube. By a well-concerted
stratagem, he seized a fleet of light brigantines,\textsuperscript{31} as it lay
at anchor; secured a apply of coarse provisions sufficient to
satisfy the indelicate, and voracious, appetite of a Gallic army;
and boldly committed himself to the stream of the Danube. The
labors of the mariners, who plied their oars with incessant
diligence, and the steady continuance of a favorable wind,
carried his fleet above seven hundred miles in eleven days;\textsuperscript{32}
and he had already disembarked his troops at Bononia,\textsuperscript{3211} only
nineteen miles from Sirmium, before his enemies could receive any
certain intelligence that he had left the banks of the Rhine. In
the course of this long and rapid navigation, the mind of Julian
was fixed on the object of his enterprise; and though he accepted
the deputations of some cities, which hastened to claim the merit
of an early submission, he passed before the hostile stations,
which were placed along the river, without indulging the
temptation of signalizing a useless and ill-timed valor. The
banks of the Danube were crowded on either side with spectators,
who gazed on the military pomp, anticipated the importance of the
event, and diffused through the adjacent country the fame of a
young hero, who advanced with more than mortal speed at the head
of the innumerable forces of the West. Lucilian, who, with the
rank of general of the cavalry, commanded the military powers of
Illyricum, was alarmed and perplexed by the doubtful reports,
which he could neither reject nor believe. He had taken some slow
and irresolute measures for the purpose of collecting his troops,
when he was surprised by Dagalaiphus, an active officer, whom
Julian, as soon as he landed at Bononia, had pushed forwards with
some light infantry. The captive general, uncertain of his life
or death, was hastily thrown upon a horse, and conducted to the
presence of Julian; who kindly raised him from the ground, and
dispelled the terror and amazement which seemed to stupefy his
faculties. But Lucilian had no sooner recovered his spirits, than
he betrayed his want of discretion, by presuming to admonish his
conqueror that he had rashly ventured, with a handful of men, to
expose his person in the midst of his enemies. “Reserve for your
master Constantius these timid remonstrances,” replied Julian,
with a smile of contempt: “when I gave you my purple to kiss, I
received you not as a counsellor, but as a suppliant.” Conscious
that success alone could justify his attempt, and that boldness
only could command success, he instantly advanced, at the head of
three thousand soldiers, to attack the strongest and most
populous city of the Illyrian provinces. As he entered the long
suburb of Sirmium, he was received by the joyful acclamations of
the army and people; who, crowned with flowers, and holding
lighted tapers in their hands, conducted their acknowledged
sovereign to his Imperial residence. Two days were devoted to the
public joy, which was celebrated by the games of the circus; but,
early on the morning of the third day, Julian marched to occupy
the narrow pass of Succi, in the defiles of Mount Hæmus; which,
almost in the midway between Sirmium and Constantinople,
separates the provinces of Thrace and Dacia, by an abrupt descent
towards the former, and a gentle declivity on the side of the
latter.\textsuperscript{33} The defence of this important post was intrusted to
the brave Nevitta; who, as well as the generals of the Italian
division, successfully executed the plan of the march and
junction which their master had so ably conceived.\textsuperscript{34}

\pagenote[28]{Ammianus (xxi. 8) ascribes the same practice, and
the same motive, to Alexander the Great and other skilful
generals.}

\pagenote[29]{This wood was a part of the great Hercynian forest,
which, is the time of Cæsar, stretched away from the country of
the Rauraci (Basil) into the boundless regions of the north. See
Cluver, Germania Antiqua. l. iii. c. 47.}

\pagenote[30]{Compare Libanius, Orat. Parent. c. 53, p. 278, 279,
with Gregory Nazianzen, Orat. iii. p. 68. Even the saint admires
the speed and secrecy of this march. A modern divine might apply
to the progress of Julian the lines which were originally
designed for another apostate:—

—So eagerly the fiend, O’er bog, or steep, through strait, rough,
dense, or rare, With head, hands, wings, or feet, pursues his way,
And swims, or sinks, or wades, or creeps, or flies.}

\pagenote[31]{In that interval the \textit{Notitia} places two or three
fleets, the Lauriacensis, (at Lauriacum, or Lorch,) the
Arlapensis, the Maginensis; and mentions five legions, or
cohorts, of Libernarii, who should be a sort of marines. Sect.
lviii. edit. Labb.}

\pagenote[32]{Zosimus alone (l. iii. p. 156) has specified this
interesting circumstance. Mamertinus, (in Panegyr. Vet. xi. 6, 7,
8,) who accompanied Julian, as count of the sacred largesses,
describes this voyage in a florid and picturesque manner,
challenges Triptolemus and the Argonauts of Greece, \&c.}

\pagenote[3211]{Banostar. \textit{Mannert}.—M.}

\pagenote[33]{The description of Ammianus, which might be
supported by collateral evidence, ascertains the precise
situation of the \textit{Angustiæ Succorum}, or passes of \textit{Succi}. M.
d’Anville, from the trifling resemblance of names, has placed
them between Sardica and Naissus. For my own justification I am
obliged to mention the \textit{only} error which I have discovered in
the maps or writings of that admirable geographer.}

\pagenote[34]{Whatever circumstances we may borrow elsewhere,
Ammianus (xx. 8, 9, 10) still supplies the series of the
narrative.}

The homage which Julian obtained, from the fears or the
inclination of the people, extended far beyond the immediate
effect of his arms.\textsuperscript{35} The præfectures of Italy and Illyricum
were administered by Taurus and Florentius, who united that
important office with the vain honors of the consulship; and as
those magistrates had retired with precipitation to the court of
Asia, Julian, who could not always restrain the levity of his
temper, stigmatized their flight by adding, in all the Acts of
the Year, the epithet of \textit{fugitive} to the names of the two
consuls. The provinces which had been deserted by their first
magistrates acknowledged the authority of an emperor, who,
conciliating the qualities of a soldier with those of a
philosopher, was equally admired in the camps of the Danube and
in the cities of Greece. From his palace, or, more properly, from
his head-quarters of Sirmium and Naissus, he distributed to the
principal cities of the empire, a labored apology for his own
conduct; published the secret despatches of Constantius; and
solicited the judgment of mankind between two competitors, the
one of whom had expelled, and the other had invited, the
Barbarians.\textsuperscript{36} Julian, whose mind was deeply wounded by the
reproach of ingratitude, aspired to maintain, by argument as well
as by arms, the superior merits of his cause; and to excel, not
only in the arts of war, but in those of composition. His epistle
to the senate and people of Athens\textsuperscript{37} seems to have been dictated
by an elegant enthusiasm; which prompted him to submit his
actions and his motives to the degenerate Athenians of his own
times, with the same humble deference as if he had been pleading,
in the days of Aristides, before the tribunal of the Areopagus.
His application to the senate of Rome, which was still permitted
to bestow the titles of Imperial power, was agreeable to the
forms of the expiring republic. An assembly was summoned by
Tertullus, præfect of the city; the epistle of Julian was read;
and, as he appeared to be master of Italy his claims were
admitted without a dissenting voice. His oblique censure of the
innovations of Constantine, and his passionate invective against
the vices of Constantius, were heard with less satisfaction; and
the senate, as if Julian had been present, unanimously exclaimed,
“Respect, we beseech you, the author of your own fortune.”\textsuperscript{38} An
artful expression, which, according to the chance of war, might
be differently explained; as a manly reproof of the ingratitude
of the usurper, or as a flattering confession, that a single act
of such benefit to the state ought to atone for all the failings
of Constantius.

\pagenote[35]{Ammian. xxi. 9, 10. Libanius, Orat. Parent. c. 54,
p. 279, 280. Zosimus, l. iii. p. 156, 157.}

\pagenote[36]{Julian (ad S. P. Q. Athen. p. 286) positively
asserts, that he intercepted the letters of Constantius to the
Barbarians; and Libanius as positively affirms, that he read them
on his march to the troops and the cities. Yet Ammianus (xxi. 4)
expresses himself with cool and candid hesitation, si \textit{famæ
solius} admittenda est fides. He specifies, however, an
intercepted letter from Vadomair to Constantius, which supposes
an intimate correspondence between them. “disciplinam non
habet.”}

\pagenote[37]{Zosimus mentions his epistles to the Athenians, the
Corinthians, and the Lacedæmonians. The substance was probably
the same, though the address was properly varied. The epistle to
the Athenians is still extant, (p. 268-287,) and has afforded
much valuable information. It deserves the praises of the Abbé de
la Bleterie, (Pref. a l’Histoire de Jovien, p. 24, 25,) and is
one of the best manifestoes to be found in any language.}

\pagenote[38]{\textit{Auctori tuo reverentiam rogamus}. Ammian. xxi. 10.
It is amusing enough to observe the secret conflicts of the
senate between flattery and fear. See Tacit. Hist. i. 85.}

The intelligence of the march and rapid progress of Julian was
speedily transmitted to his rival, who, by the retreat of Sapor,
had obtained some respite from the Persian war. Disguising the
anguish of his soul under the semblance of contempt, Constantius
professed his intention of returning into Europe, and of giving
chase to Julian; for he never spoke of his military expedition in
any other light than that of a hunting party.\textsuperscript{39} In the camp of
Hierapolis, in Syria, he communicated this design to his army;
slightly mentioned the guilt and rashness of the Cæsar; and
ventured to assure them, that if the mutineers of Gaul presumed
to meet them in the field, they would be unable to sustain the
fire of their eyes, and the irresistible weight of their shout of
onset. The speech of the emperor was received with military
applause, and Theodotus, the president of the council of
Hierapolis, requested, with tears of adulation, that \textit{his} city
might be adorned with the head of the vanquished rebel.\textsuperscript{40} A
chosen detachment was despatched away in post-wagons, to secure,
if it were yet possible, the pass of Succi; the recruits, the
horses, the arms, and the magazines, which had been prepared
against Sapor, were appropriated to the service of the civil war;
and the domestic victories of Constantius inspired his partisans
with the most sanguine assurances of success. The notary
Gaudentius had occupied in his name the provinces of Africa; the
subsistence of Rome was intercepted; and the distress of Julian
was increased by an unexpected event, which might have been
productive of fatal consequences. Julian had received the
submission of two legions and a cohort of archers, who were
stationed at Sirmium; but he suspected, with reason, the fidelity
of those troops which had been distinguished by the emperor; and
it was thought expedient, under the pretence of the exposed state
of the Gallic frontier, to dismiss them from the most important
scene of action. They advanced, with reluctance, as far as the
confines of Italy; but as they dreaded the length of the way, and
the savage fierceness of the Germans, they resolved, by the
instigation of one of their tribunes, to halt at Aquileia, and to
erect the banners of Constantius on the walls of that impregnable
city. The vigilance of Julian perceived at once the extent of the
mischief, and the necessity of applying an immediate remedy. By
his order, Jovinus led back a part of the army into Italy; and
the siege of Aquileia was formed with diligence, and prosecuted
with vigor. But the legionaries, who seemed to have rejected the
yoke of discipline, conducted the defence of the place with skill
and perseverance; vited the rest of Italy to imitate the example
of their courage and loyalty; and threatened the retreat of
Julian, if he should be forced to yield to the superior numbers
of the armies of the East.\textsuperscript{41}

\pagenote[39]{Tanquam venaticiam prædam caperet: hoc enim ad
Jeniendum suorum metum subinde prædicabat. Ammian. xxii. 7.}

\pagenote[40]{See the speech and preparations in Ammianus, xxi.
13. The vile Theodotus afterwards implored and obtained his
pardon from the merciful conqueror, who signified his wish of
diminishing his enemies and increasing the numbers of his
friends, (xxii. 14.)}

\pagenote[41]{Ammian. xxi. 7, 11, 12. He seems to describe, with
superfluous labor, the operations of the siege of Aquileia,
which, on this occasion, maintained its impregnable fame. Gregory
Nazianzen (Orat. iii. p. 68) ascribes this accidental revolt to
the wisdom of Constantius, whose assured victory he announces
with some appearance of truth. Constantio quem credebat procul
dubio fore victorem; nemo enim omnium tunc ab hac constanti
sententia discrepebat. Ammian. xxi. 7.}

But the humanity of Julian was preserved from the cruel
alternative which he pathetically laments, of destroying or of
being himself destroyed: and the seasonable death of Constantius
delivered the Roman empire from the calamities of civil war. The
approach of winter could not detain the monarch at Antioch; and
his favorites durst not oppose his impatient desire of revenge. A
slight fever, which was perhaps occasioned by the agitation of
his spirits, was increased by the fatigues of the journey; and
Constantius was obliged to halt at the little town of Mopsucrene,
twelve miles beyond Tarsus, where he expired, after a short
illness, in the forty-fifth year of his age, and the
twenty-fourth of his reign.\textsuperscript{42} His genuine character, which was
composed of pride and weakness, of superstition and cruelty, has
been fully displayed in the preceding narrative of civil and
ecclesiastical events. The long abuse of power rendered him a
considerable object in the eyes of his contemporaries; but as
personal merit can alone deserve the notice of posterity, the
last of the sons of Constantine may be dismissed from the world,
with the remark, that he inherited the defects, without the
abilities, of his father. Before Constantius expired, he is said
to have named Julian for his successor; nor does it seem
improbable, that his anxious concern for the fate of a young and
tender wife, whom he left with child, may have prevailed, in his
last moments, over the harsher passions of hatred and revenge.
Eusebius, and his guilty associates, made a faint attempt to
prolong the reign of the eunuchs, by the election of another
emperor; but their intrigues were rejected with disdain, by an
army which now abhorred the thought of civil discord; and two
officers of rank were instantly despatched, to assure Julian,
that every sword in the empire would be drawn for his service.
The military designs of that prince, who had formed three
different attacks against Thrace, were prevented by this
fortunate event. Without shedding the blood of his
fellow-citizens, he escaped the dangers of a doubtful conflict,
and acquired the advantages of a complete victory. Impatient to
visit the place of his birth, and the new capital of the empire,
he advanced from Naissus through the mountains of Hæmus, and the
cities of Thrace. When he reached Heraclea, at the distance of
sixty miles, all Constantinople was poured forth to receive him;
and he made his triumphal entry amidst the dutiful acclamations
of the soldiers, the people, and the senate. An innumerable
multitude pressed around him with eager respect and were perhaps
disappointed when they beheld the small stature and simple garb
of a hero, whose unexperienced youth had vanquished the
Barbarians of Germany, and who had now traversed, in a successful
career, the whole continent of Europe, from the shores of the
Atlantic to those of the Bosphorus.\textsuperscript{43} A few days afterwards,
when the remains of the deceased emperor were landed in the
harbor, the subjects of Julian applauded the real or affected
humanity of their sovereign. On foot, without his diadem, and
clothed in a mourning habit, he accompanied the funeral as far as
the church of the Holy Apostles, where the body was deposited:
and if these marks of respect may be interpreted as a selfish
tribute to the birth and dignity of his Imperial kinsman, the
tears of Julian professed to the world that he had forgot the
injuries, and remembered only the obligations, which he had
received from Constantius.\textsuperscript{44} As soon as the legions of Aquileia
were assured of the death of the emperor, they opened the gates
of the city, and, by the sacrifice of their guilty leaders,
obtained an easy pardon from the prudence or lenity of Julian;
who, in the thirty-second year of his age, acquired the
undisputed possession of the Roman empire.\textsuperscript{45}

\pagenote[42]{His death and character are faithfully delineated
by Ammianus, (xxi. 14, 15, 16;) and we are authorized to despise
and detest the foolish calumny of Gregory, (Orat. iii. p. 68,)
who accuses Julian of contriving the death of his benefactor. The
private repentance of the emperor, that he had spared and
promoted Julian, (p. 69, and Orat. xxi. p. 389,) is not
improbable in itself, nor incompatible with the public verbal
testament which prudential considerations might dictate in the
last moments of his life. Note: Wagner thinks this sudden change
of sentiment altogether a fiction of the attendant courtiers and
chiefs of the army. who up to this time had been hostile to
Julian. Note in loco Ammian.—M.}

\pagenote[43]{In describing the triumph of Julian, Ammianus
(xxii. l, 2) assumes the lofty tone of an orator or poet; while
Libanius (Orat. Parent, c. 56, p. 281) sinks to the grave
simplicity of an historian.}

\pagenote[44]{The funeral of Constantius is described by
Ammianus, (xxi. 16.) Gregory Nazianzen, (Orat. iv. p. 119,)
Mamertinus, in (Panegyr. Vet. xi. 27,) Libanius, (Orat. Parent.
c. lvi. p. 283,) and Philostorgius, (l. vi. c. 6, with Godefroy’s
Dissertations, p. 265.) These writers, and their followers,
Pagans, Catholics, Arians, beheld with very different eyes both
the dead and the living emperor.}

\pagenote[45]{The day and year of the birth of Julian are not
perfectly ascertained. The day is probably the sixth of November,
and the year must be either 331 or 332. Tillemont, Hist. des
Empereurs, tom. iv. p. 693. Ducange, Fam. Byzantin. p. 50. I have
preferred the earlier date.}

\section{Part \thesection.}

Philosophy had instructed Julian to compare the advantages of
action and retirement; but the elevation of his birth, and the
accidents of his life, never allowed him the freedom of choice.
He might perhaps sincerely have preferred the groves of the
academy, and the society of Athens; but he was constrained, at
first by the will, and afterwards by the injustice, of
Constantius, to expose his person and fame to the dangers of
Imperial greatness; and to make himself accountable to the world,
and to posterity, for the happiness of millions.\textsuperscript{46} Julian
recollected with terror the observation of his master Plato,\textsuperscript{47}
that the government of our flocks and herds is always committed
to beings of a superior species; and that the conduct of nations
requires and deserves the celestial powers of the gods or of the
genii. From this principle he justly concluded, that the man who
presumes to reign, should aspire to the perfection of the divine
nature; that he should purify his soul from her mortal and
terrestrial part; that he should extinguish his appetites,
enlighten his understanding, regulate his passions, and subdue
the wild beast, which, according to the lively metaphor of
Aristotle,\textsuperscript{48} seldom fails to ascend the throne of a despot. The
throne of Julian, which the death of Constantius fixed on an
independent basis, was the seat of reason, of virtue, and perhaps
of vanity. He despised the honors, renounced the pleasures, and
discharged with incessant diligence the duties, of his exalted
station; and there were few among his subjects who would have
consented to relieve him from the weight of the diadem, had they
been obliged to submit their time and their actions to the
rigorous laws which that philosophic emperor imposed on himself.
One of his most intimate friends,\textsuperscript{49} who had often shared the
frugal simplicity of his table, has remarked, that his light and
sparing diet (which was usually of the vegetable kind) left his
mind and body always free and active, for the various and
important business of an author, a pontiff, a magistrate, a
general, and a prince. In one and the same day, he gave audience
to several ambassadors, and wrote, or dictated, a great number of
letters to his generals, his civil magistrates, his private
friends, and the different cities of his dominions. He listened
to the memorials which had been received, considered the subject
of the petitions, and signified his intentions more rapidly than
they could be taken in short-hand by the diligence of his
secretaries. He possessed such flexibility of thought, and such
firmness of attention, that he could employ his hand to write,
his ear to listen, and his voice to dictate; and pursue at once
three several trains of ideas without hesitation, and without
error. While his ministers reposed, the prince flew with agility
from one labor to another, and, after a hasty dinner, retired
into his library, till the public business, which he had
appointed for the evening, summoned him to interrupt the
prosecution of his studies. The supper of the emperor was still
less substantial than the former meal; his sleep was never
clouded by the fumes of indigestion; and except in the short
interval of a marriage, which was the effect of policy rather
than love, the chaste Julian never shared his bed with a female
companion.\textsuperscript{50} He was soon awakened by the entrance of fresh
secretaries, who had slept the preceding day; and his servants
were obliged to wait alternately while their indefatigable master
allowed himself scarcely any other refreshment than the change of
occupation. The predecessors of Julian, his uncle, his brother,
and his cousin, indulged their puerile taste for the games of the
Circus, under the specious pretence of complying with the
inclinations of the people; and they frequently remained the
greatest part of the day as idle spectators, and as a part of the
splendid spectacle, till the ordinary round of twenty-four races\textsuperscript{51}
was completely finished. On solemn festivals, Julian, who felt
and professed an unfashionable dislike to these frivolous
amusements, condescended to appear in the Circus; and after
bestowing a careless glance at five or six of the races, he
hastily withdrew with the impatience of a philosopher, who
considered every moment as lost that was not devoted to the
advantage of the public or the improvement of his own mind.\textsuperscript{52} By
this avarice of time, he seemed to protract the short duration of
his reign; and if the dates were less securely ascertained, we
should refuse to believe, that only sixteen months elapsed
between the death of Constantius and the departure of his
successor for the Persian war. The actions of Julian can only be
preserved by the care of the historian; but the portion of his
voluminous writings, which is still extant, remains as a monument
of the application, as well as of the genius, of the emperor. The
Misopogon, the Cæsars, several of his orations, and his elaborate
work against the Christian religion, were composed in the long
nights of the two winters, the former of which he passed at
Constantinople, and the latter at Antioch.

\pagenote[46]{Julian himself (p. 253-267) has expressed these
philosophical ideas with much eloquence and some affectation, in
a very elaborate epistle to Themistius. The Abbé de la Bleterie,
(tom. ii. p. 146-193,) who has given an elegant translation, is
inclined to believe that it was the celebrated Themistius, whose
orations are still extant.}

\pagenote[47]{Julian. ad Themist. p. 258. Petavius (not. p. 95)
observes that this passage is taken from the fourth book De
Legibus; but either Julian quoted from memory, or his MSS. were
different from ours Xenophon opens the Cyropædia with a similar
reflection.}

\pagenote[48]{Aristot. ap. Julian. p. 261. The MS. of Vossius,
unsatisfied with the single beast, affords the stronger reading
of which the experience of despotism may warrant.}

\pagenote[49]{Libanius (Orat. Parentalis, c. lxxxiv. lxxxv. p.
310, 311, 312) has given this interesting detail of the private
life of Julian. He himself (in Misopogon, p. 350) mentions his
vegetable diet, and upbraids the gross and sensual appetite of
the people of Antioch.}

\pagenote[50]{Lectulus... Vestalium toris purior, is the praise
which Mamertinus (Panegyr. Vet. xi. 13) addresses to Julian
himself. Libanius affirms, in sober peremptory language, that
Julian never knew a woman before his marriage, or after the death
of his wife, (Orat. Parent. c. lxxxviii. p. 313.) The chastity of
Julian is confirmed by the impartial testimony of Ammianus, (xxv.
4,) and the partial silence of the Christians. Yet Julian
ironically urges the reproach of the people of Antioch, that he
\textit{almost always} (in Misopogon, p. 345) lay alone. This suspicious
expression is explained by the Abbé de la Bleterie (Hist. de
Jovien, tom. ii. p. 103-109) with candor and ingenuity.}

\pagenote[51]{See Salmasius ad Sueton in Claud. c. xxi. A
twenty-fifth race, or \textit{missus}, was added, to complete the number
of one hundred chariots, four of which, the four colors, started
each heat.

Centum quadrijugos agitabo ad flumina currus.

It appears, that they ran five or seven times round the \textit{Meta}
(Sueton. in Domitian. c. 4;) and (from the measure of the Circus
Maximus at Rome, the Hippodrome at Constantinople, \&c.) it might
be about a four mile course.}

\pagenote[52]{Julian. in Misopogon, p. 340. Julius Cæsar had
offended the Roman people by reading his despatches during the
actual race. Augustus indulged their taste, or his own, by his
constant attention to the important business of the Circus, for
which he professed the warmest inclination. Sueton. in August. c.
xlv.}

The reformation of the Imperial court was one of the first and
most necessary acts of the government of Julian.\textsuperscript{53} Soon after
his entrance into the palace of Constantinople, he had occasion
for the service of a barber. An officer, magnificently dressed,
immediately presented himself. “It is a barber,” exclaimed the
prince, with affected surprise, “that I want, and not a
receiver-general of the finances.”\textsuperscript{54} He questioned the man
concerning the profits of his employment and was informed, that
besides a large salary, and some valuable perquisites, he enjoyed
a daily allowance for twenty servants, and as many horses. A
thousand barbers, a thousand cup-bearers, a thousand cooks, were
distributed in the several offices of luxury; and the number of
eunuchs could be compared only with the insects of a summer’s
day. The monarch who resigned to his subjects the superiority of
merit and virtue, was distinguished by the oppressive
magnificence of his dress, his table, his buildings, and his
train. The stately palaces erected by Constantine and his sons,
were decorated with many colored marbles, and ornaments of massy
gold. The most exquisite dainties were procured, to gratify their
pride, rather than their taste; birds of the most distant
climates, fish from the most remote seas, fruits out of their
natural season, winter roses, and summer snows.\textsuperscript{56} The domestic
crowd of the palace surpassed the expense of the legions; yet the
smallest part of this costly multitude was subservient to the
use, or even to the splendor, of the throne. The monarch was
disgraced, and the people was injured, by the creation and sale
of an infinite number of obscure, and even titular employments;
and the most worthless of mankind might purchase the privilege of
being maintained, without the necessity of labor, from the public
revenue. The waste of an enormous household, the increase of fees
and perquisites, which were soon claimed as a lawful debt, and
the bribes which they extorted from those who feared their
enmity, or solicited their favor, suddenly enriched these haughty
menials. They abused their fortune, without considering their
past, or their future, condition; and their rapine and venality
could be equalled only by the extravagance of their dissipations.
Their silken robes were embroidered with gold, their tables were
served with delicacy and profusion; the houses which they built
for their own use, would have covered the farm of an ancient
consul; and the most honorable citizens were obliged to dismount
from their horses, and respectfully to salute a eunuch whom they
met on the public highway. The luxury of the palace excited the
contempt and indignation of Julian, who usually slept on the
ground, who yielded with reluctance to the indispensable calls of
nature; and who placed his vanity, not in emulating, but in
despising, the pomp of royalty.

\pagenote[53]{The reformation of the palace is described by
Ammianus, (xxii. 4,) Libanius, Orat. (Parent. c. lxii. p. 288,
\&c.,) Mamertinus, in Panegyr. (Vet. xi. 11,) Socrates, (l. iii.
c. l.,) and Zonaras, (tom. ii. l. xiii. p. 24.)}

\pagenote[54]{Ego non \textit{rationalem} jussi sed tonsorem acciri.
Zonaras uses the less natural image of a senator. Yet an officer
of the finances, who was satisfied with wealth, might desire and
obtain the honors of the senate.}

\pagenote[56]{The expressions of Mamertinus are lively and
forcible. Quis etiam prandiorum et cænarum laboratas magnitudines
Romanus populus sensit; cum quæsitissimæ dapes non gustu sed
difficultatibus æstimarentur; miracula avium, longinqui maris
pisces, aheni temporis poma, æstivæ nives, hybernæ rosæ}

By the total extirpation of a mischief which was magnified even
beyond its real extent, he was impatient to relieve the distress,
and to appease the murmurs of the people; who support with less
uneasiness the weight of taxes, if they are convinced that the
fruits of their industry are appropriated to the service of the
state. But in the execution of this salutary work, Julian is
accused of proceeding with too much haste and inconsiderate
severity. By a single edict, he reduced the palace of
Constantinople to an immense desert, and dismissed with ignominy
the whole train of slaves and dependants,\textsuperscript{57} without providing
any just, or at least benevolent, exceptions, for the age, the
services, or the poverty, of the faithful domestics of the
Imperial family. Such indeed was the temper of Julian, who seldom
recollected the fundamental maxim of Aristotle, that true virtue
is placed at an equal distance between the opposite vices.

The splendid and effeminate dress of the Asiatics, the curls and
paint, the collars and bracelets, which had appeared so
ridiculous in the person of Constantine, were consistently
rejected by his philosophic successor. But with the fopperies,
Julian affected to renounce the decencies of dress; and seemed to
value himself for his neglect of the laws of cleanliness. In a
satirical performance, which was designed for the public eye, the
emperor descants with pleasure, and even with pride, on the
length of his nails, and the inky blackness of his hands;
protests, that although the greatest part of his body was covered
with hair, the use of the razor was confined to his head alone;
and celebrates, with visible complacency, the shaggy and
\textit{populous}\textsuperscript{58} beard, which he fondly cherished, after the example
of the philosophers of Greece. Had Julian consulted the simple
dictates of reason, the first magistrate of the Romans would have
scorned the affectation of Diogenes, as well as that of Darius.

\pagenote[57]{Yet Julian himself was accused of bestowing whole
towns on the eunuchs, (Orat. vii. against Polyclet. p. 117-127.)
Libanius contents himself with a cold but positive denial of the
fact, which seems indeed to belong more properly to Constantius.
This charge, however, may allude to some unknown circumstance.}

\pagenote[58]{In the Misopogon (p. 338, 339) he draws a very
singular picture of himself, and the following words are
strangely characteristic. The friends of the Abbé de la Bleterie
adjured him, in the name of the French nation, not to translate
this passage, so offensive to their delicacy, (Hist. de Jovien,
tom. ii. p. 94.) Like him, I have contented myself with a
transient allusion; but the little animal which Julian \textit{names},
is a beast familiar to man, and signifies love.}

But the work of public reformation would have remained imperfect,
if Julian had only corrected the abuses, without punishing the
crimes, of his predecessor’s reign. “We are now delivered,” says
he, in a familiar letter to one of his intimate friends, “we are
now surprisingly delivered from the voracious jaws of the Hydra.\textsuperscript{59}
I do not mean to apply the epithet to my brother Constantius.
He is no more; may the earth lie light on his head! But his
artful and cruel favorites studied to deceive and exasperate a
prince, whose natural mildness cannot be praised without some
efforts of adulation. It is not, however, my intention, that even
those men should be oppressed: they are accused, and they shall
enjoy the benefit of a fair and impartial trial.” To conduct this
inquiry, Julian named six judges of the highest rank in the state
and army; and as he wished to escape the reproach of condemning
his personal enemies, he fixed this extraordinary tribunal at
Chalcedon, on the Asiatic side of the Bosphorus; and transferred
to the commissioners an absolute power to pronounce and execute
their final sentence, without delay, and without appeal. The
office of president was exercised by the venerable præfect of the
East, a \textit{second} Sallust,\textsuperscript{60} whose virtues conciliated the esteem
of Greek sophists, and of Christian bishops. He was assisted by
the eloquent Mamertinus,\textsuperscript{61} one of the consuls elect, whose merit
is loudly celebrated by the doubtful evidence of his own
applause. But the civil wisdom of two magistrates was
overbalanced by the ferocious violence of four generals, Nevitta,
Agilo, Jovinus, and Arbetio. Arbetio, whom the public would have
seen with less surprise at the bar than on the bench, was
supposed to possess the secret of the commission; the armed and
angry leaders of the Jovian and Herculian bands encompassed the
tribunal; and the judges were alternately swayed by the laws of
justice, and by the clamors of faction.\textsuperscript{62}

\pagenote[59]{Julian, epist. xxiii. p. 389. He uses the words in
writing to his friend Hermogenes, who, like himself, was
conversant with the Greek poets.}

\pagenote[60]{The two Sallusts, the præfect of Gaul, and the
præfect of the East, must be carefully distinguished, (Hist. des
Empereurs, tom. iv. p. 696.) I have used the surname of
\textit{Secundus}, as a convenient epithet. The second Sallust extorted
the esteem of the Christians themselves; and Gregory Nazianzen,
who condemned his religion, has celebrated his virtues, (Orat.
iii. p. 90.) See a curious note of the Abbé de la Bleterie, Vie
de Julien, p. 363. Note: Gibbonus secundum habet pro numero, quod
tamen est viri agnomen Wagner, nota in loc. Amm. It is not a
mistake; it is rather an error in taste. Wagner inclines to
transfer the chief guilt to Arbetio.—M.}

\pagenote[61]{Mamertinus praises the emperor (xi. l.) for
bestowing the offices of Treasurer and Præfect on a man of
wisdom, firmness, integrity, \&c., like himself. Yet Ammianus
ranks him (xxi. l.) among the ministers of Julian, quorum merita
norat et fidem.}

\pagenote[62]{The proceedings of this chamber of justice are
related by Ammianus, (xxii. 3,) and praised by Libanius, (Orat.
Parent. c. 74, p. 299, 300.)}

The chamberlain Eusebius, who had so long abused the favor of
Constantius, expiated, by an ignominious death, the insolence,
the corruption, and cruelty of his servile reign. The executions
of Paul and Apodemius (the former of whom was burnt alive) were
accepted as an inadequate atonement by the widows and orphans of
so many hundred Romans, whom those legal tyrants had betrayed and
murdered. But justice herself (if we may use the pathetic
expression of Ammianus)\textsuperscript{63} appeared to weep over the fate of
Ursulus, the treasurer of the empire; and his blood accused the
ingratitude of Julian, whose distress had been seasonably
relieved by the intrepid liberality of that honest minister. The
rage of the soldiers, whom he had provoked by his indiscretion,
was the cause and the excuse of his death; and the emperor,
deeply wounded by his own reproaches and those of the public,
offered some consolation to the family of Ursulus, by the
restitution of his confiscated fortunes. Before the end of the
year in which they had been adorned with the ensigns of the
prefecture and consulship,\textsuperscript{64} Taurus and Florentius were reduced
to implore the clemency of the inexorable tribunal of Chalcedon.
The former was banished to Vercellæ in Italy, and a sentence of
death was pronounced against the latter. A wise prince should
have rewarded the crime of Taurus: the faithful minister, when he
was no longer able to oppose the progress of a rebel, had taken
refuge in the court of his benefactor and his lawful sovereign.
But the guilt of Florentius justified the severity of the judges;
and his escape served to display the magnanimity of Julian, who
nobly checked the interested diligence of an informer, and
refused to learn what place concealed the wretched fugitive from
his just resentment.\textsuperscript{65} Some months after the tribunal of
Chalcedon had been dissolved, the prætorian vicegerent of Africa,
the notary Gaudentius, and Artemius\textsuperscript{66} duke of Egypt, were
executed at Antioch. Artemius had reigned the cruel and corrupt
tyrant of a great province; Gaudentius had long practised the
arts of calumny against the innocent, the virtuous, and even the
person of Julian himself. Yet the circumstances of their trial
and condemnation were so unskillfully managed, that these wicked
men obtained, in the public opinion, the glory of suffering for
the obstinate loyalty with which they had supported the cause of
Constantius. The rest of his servants were protected by a general
act of oblivion; and they were left to enjoy with impunity the
bribes which they had accepted, either to defend the oppressed,
or to oppress the friendless. This measure, which, on the
soundest principles of policy, may deserve our approbation, was
executed in a manner which seemed to degrade the majesty of the
throne. Julian was tormented by the importunities of a multitude,
particularly of Egyptians, who loudly redemanded the gifts which
they had imprudently or illegally bestowed; he foresaw the
endless prosecution of vexatious suits; and he engaged a promise,
which ought always to have been sacred, that if they would repair
to Chalcedon, he would meet them in person, to hear and determine
their complaints. But as soon as they were landed, he issued an
absolute order, which prohibited the watermen from transporting
any Egyptian to Constantinople; and thus detained his
disappointed clients on the Asiatic shore till, their patience
and money being utterly exhausted, they were obliged to return
with indignant murmurs to their native country.\textsuperscript{67}

\pagenote[63]{Ursuli vero necem ipsa mihi videtur flesse
justitia. Libanius, who imputes his death to the soldiers,
attempts to criminate the court of the largesses.}

\pagenote[64]{Such respect was still entertained for the
venerable names of the commonwealth, that the public was
surprised and scandalized to hear Taurus summoned as a criminal
under the consulship of Taurus. The summons of his colleague
Florentius was probably delayed till the commencement of the
ensuing year.}

\pagenote[65]{Ammian. xx. 7.}

\pagenote[66]{For the guilt and punishment of Artemius, see
Julian (Epist. x. p. 379) and Ammianus, (xxii. 6, and Vales, ad
hoc.) The merit of Artemius, who demolished temples, and was put
to death by an apostate, has tempted the Greek and Latin churches
to honor him as a martyr. But as ecclesiastical history attests
that he was not only a tyrant, but an Arian, it is not altogether
easy to justify this indiscreet promotion. Tillemont, Mém.
Eccles. tom. vii. p. 1319.}

\pagenote[67]{See Ammian. xxii. 6, and Vales, ad locum; and the
Codex Theodosianus, l. ii. tit. xxxix. leg. i.; and Godefroy’s
Commentary, tom. i. p. 218, ad locum.}

\section{Part \thesection.}

The numerous army of spies, of agents, and informers enlisted by
Constantius to secure the repose of one man, and to interrupt
that of millions, was immediately disbanded by his generous
successor. Julian was slow in his suspicions, and gentle in his
punishments; and his contempt of treason was the result of
judgment, of vanity, and of courage. Conscious of superior merit,
he was persuaded that few among his subjects would dare to meet
him in the field, to attempt his life, or even to seat themselves
on his vacant throne. The philosopher could excuse the hasty
sallies of discontent; and the hero could despise the ambitious
projects which surpassed the fortune or the abilities of the rash
conspirators. A citizen of Ancyra had prepared for his own use a
purple garment; and this indiscreet action, which, under the
reign of Constantius, would have been considered as a capital
offence,\textsuperscript{68} was reported to Julian by the officious importunity
of a private enemy. The monarch, after making some inquiry into
the rank and character of his rival, despatched the informer with
a present of a pair of purple slippers, to complete the
magnificence of his Imperial habit. A more dangerous conspiracy
was formed by ten of the domestic guards, who had resolved to
assassinate Julian in the field of exercise near Antioch. Their
intemperance revealed their guilt; and they were conducted in
chains to the presence of their injured sovereign, who, after a
lively representation of the wickedness and folly of their
enterprise, instead of a death of torture, which they deserved
and expected, pronounced a sentence of exile against the two
principal offenders. The only instance in which Julian seemed to
depart from his accustomed clemency, was the execution of a rash
youth, who, with a feeble hand, had aspired to seize the reins of
empire. But that youth was the son of Marcellus, the general of
cavalry, who, in the first campaign of the Gallic war, had
deserted the standard of the Cæsar and the republic. Without
appearing to indulge his personal resentment, Julian might easily
confound the crime of the son and of the father; but he was
reconciled by the distress of Marcellus, and the liberality of
the emperor endeavored to heal the wound which had been inflicted
by the hand of justice.\textsuperscript{69}

\pagenote[68]{The president Montesquieu (Considerations sur la
Grandeur, \&c., des Romains, c. xiv. in his works, tom. iii. p.
448, 449,) excuses this minute and absurd tyranny, by supposing
that actions the most indifferent in our eyes might excite, in a
Roman mind, the idea of guilt and danger. This strange apology is
supported by a strange misapprehension of the English laws, “chez
une nation.... où il est défendu de boire à la santé d’une
certaine personne.”}

\pagenote[69]{The clemency of Julian, and the conspiracy which
was formed against his life at Antioch, are described by Ammianus
(xxii. 9, 10, and Vales, ad loc.) and Libanius, (Orat. Parent. c.
99, p. 323.)}

Julian was not insensible of the advantages of freedom.\textsuperscript{70} From
his studies he had imbibed the spirit of ancient sages and
heroes; his life and fortunes had depended on the caprice of a
tyrant; and when he ascended the throne, his pride was sometimes
mortified by the reflection, that the slaves who would not dare
to censure his defects were not worthy to applaud his virtues.\textsuperscript{71}
He sincerely abhorred the system of Oriental despotism, which
Diocletian, Constantine, and the patient habits of fourscore
years, had established in the empire. A motive of superstition
prevented the execution of the design, which Julian had
frequently meditated, of relieving his head from the weight of a
costly diadem;\textsuperscript{72} but he absolutely refused the title of
\textit{Dominus}, or \textit{Lord},\textsuperscript{73} a word which was grown so familiar to
the ears of the Romans, that they no longer remembered its
servile and humiliating origin. The office, or rather the name,
of consul, was cherished by a prince who contemplated with
reverence the ruins of the republic; and the same behavior which
had been assumed by the prudence of Augustus was adopted by
Julian from choice and inclination. On the calends of January, at
break of day, the new consuls, Mamertinus and Nevitta, hastened
to the palace to salute the emperor. As soon as he was informed
of their approach, he leaped from his throne, eagerly advanced to
meet them, and compelled the blushing magistrates to receive the
demonstrations of his affected humility. From the palace they
proceeded to the senate. The emperor, on foot, marched before
their litters; and the gazing multitude admired the image of
ancient times, or secretly blamed a conduct, which, in their
eyes, degraded the majesty of the purple.\textsuperscript{74} But the behavior of
Julian was uniformly supported. During the games of the Circus,
he had, imprudently or designedly, performed the manumission of a
slave in the presence of the consul. The moment he was reminded
that he had trespassed on the jurisdiction of \textit{another}
magistrate, he condemned himself to pay a fine of ten pounds of
gold; and embraced this public occasion of declaring to the
world, that he was subject, like the rest of his fellow-citizens,
to the laws,\textsuperscript{75} and even to the forms, of the republic. The
spirit of his administration, and his regard for the place of his
nativity, induced Julian to confer on the senate of
Constantinople the same honors, privileges, and authority, which
were still enjoyed by the senate of ancient Rome.\textsuperscript{76} A legal
fiction was introduced, and gradually established, that one half
of the national council had migrated into the East; and the
despotic successors of Julian, accepting the title of Senators,
acknowledged themselves the members of a respectable body, which
was permitted to represent the majesty of the Roman name. From
Constantinople, the attention of the monarch was extended to the
municipal senates of the provinces. He abolished, by repeated
edicts, the unjust and pernicious exemptions which had withdrawn
so many idle citizens from the services of their country; and by
imposing an equal distribution of public duties, he restored the
strength, the splendor, or, according to the glowing expression
of Libanius,\textsuperscript{77} the soul of the expiring cities of his empire.
The venerable age of Greece excited the most tender compassion in
the mind of Julian, which kindled into rapture when he
recollected the gods, the heroes, and the men superior to heroes
and to gods, who have bequeathed to the latest posterity the
monuments of their genius, or the example of their virtues. He
relieved the distress, and restored the beauty, of the cities of
Epirus and Peloponnesus.\textsuperscript{78} Athens acknowledged him for her
benefactor; Argos, for her deliverer. The pride of Corinth, again
rising from her ruins with the honors of a Roman colony, exacted
a tribute from the adjacent republics, for the purpose of
defraying the games of the Isthmus, which were celebrated in the
amphitheatre with the hunting of bears and panthers. From this
tribute the cities of Elis, of Delphi, and of Argos, which had
inherited from their remote ancestors the sacred office of
perpetuating the Olympic, the Pythian, and the Nemean games,
claimed a just exemption. The immunity of Elis and Delphi was
respected by the Corinthians; but the poverty of Argos tempted
the insolence of oppression; and the feeble complaints of its
deputies were silenced by the decree of a provincial magistrate,
who seems to have consulted only the interest of the capital in
which he resided. Seven years after this sentence, Julian\textsuperscript{79}
allowed the cause to be referred to a superior tribunal; and his
eloquence was interposed, most probably with success, in the
defence of a city, which had been the royal seat of Agamemnon,\textsuperscript{80}
and had given to Macedonia a race of kings and conquerors.\textsuperscript{81}

\pagenote[70]{According to some, says Aristotle, (as he is quoted
by Julian ad Themist. p. 261,) the form of absolute government is
contrary to nature. Both the prince and the philosopher choose,
how ever to involve this eternal truth in artful and labored
obscurity.}

\pagenote[71]{That sentiment is expressed almost in the words of
Julian himself. Ammian. xxii. 10.}

\pagenote[72]{Libanius, (Orat. Parent. c. 95, p. 320,) who
mentions the wish and design of Julian, insinuates, in mysterious
language that the emperor was restrained by some particular
revelation.}

\pagenote[73]{Julian in Misopogon, p. 343. As he never abolished,
by any public law, the proud appellations of \textit{Despot}, or
\textit{Dominus}, they are still extant on his medals, (Ducange, Fam.
Byzantin. p. 38, 39;) and the private displeasure which he
affected to express, only gave a different tone to the servility
of the court. The Abbé de la Bleterie (Hist. de Jovien, tom. ii.
p. 99-102) has curiously traced the origin and progress of the
word \textit{Dominus} under the Imperial government.}

\pagenote[74]{Ammian. xxii. 7. The consul Mamertinus (in Panegyr.
Vet. xi. 28, 29, 30) celebrates the auspicious day, like an
elegant slave, astonished and intoxicated by the condescension of
his master.}

\pagenote[75]{Personal satire was condemned by the laws of the
twelve tables: Si male condiderit in quem quis carmina, jus est
Judiciumque—Horat. Sat. ii. 1. 82. ——Julian (in Misopogon, p.
337) owns himself subject to the law; and the Abbé de la Bleterie
(Hist. de Jovien, tom. ii. p. 92) has eagerly embraced a
declaration so agreeable to his own system, and, indeed, to the
true spirit of the Imperial constitution.}

\pagenote[76]{Zosimus, l. iii. p. 158.}

\pagenote[77]{See Libanius, (Orat. Parent. c. 71, p. 296,)
Ammianus, (xxii. 9,) and the Theodosian Code (l. xii. tit. i.
leg. 50-55.) with Godefroy’s Commentary, (tom. iv. p. 390-402.)
Yet the whole subject of the \textit{Curia}, notwithstanding very ample
materials, still remains the most obscure in the legal history of
the empire.}

\pagenote[78]{Quæ paulo ante arida et siti anhelantia visebantur,
ea nunc perlui, mundari, madere; Fora, Deambulacra, Gymnasia,
lætis et gaudentibus populis frequentari; dies festos, et
celebrari veteres, et novos in honorem principis consecrari,
(Mamertin. xi. 9.) He particularly restored the city of Nicopolis
and the Actiac games, which had been instituted by Augustus.}

\pagenote[79]{Julian. Epist. xxxv. p. 407-411. This epistle,
which illustrates the declining age of Greece, is omitted by the
Abbé de la Bleterie, and strangely disfigured by the Latin
translator, who, by rendering \textit{tributum}, and \textit{populus}, directly
contradicts the sense of the original.}

\pagenote[80]{He reigned in Mycenæ at the distance of fifty
stadia, or six miles from Argos: but these cities, which
alternately flourished, are confounded by the Greek poets.
Strabo, l. viii. p. 579, edit. Amstel. 1707.}

\pagenote[81]{Marsham, Canon. Chron. p. 421. This pedigree from
Temenus and Hercules may be suspicious; yet it was allowed, after
a strict inquiry, by the judges of the Olympic games, (Herodot.
l. v. c. 22,) at a time when the Macedonian kings were obscure
and unpopular in Greece. When the Achæan league declared against
Philip, it was thought decent that the deputies of Argos should
retire, (T. Liv. xxxii. 22.)}

The laborious administration of military and civil affairs, which
were multiplied in proportion to the extent of the empire,
exercised the abilities of Julian; but he frequently assumed the
two characters of Orator\textsuperscript{82} and of Judge,\textsuperscript{83} which are almost
unknown to the modern sovereigns of Europe. The arts of
persuasion, so diligently cultivated by the first Cæsars, were
neglected by the military ignorance and Asiatic pride of their
successors; and if they condescended to harangue the soldiers,
whom they feared, they treated with silent disdain the senators,
whom they despised. The assemblies of the senate, which
Constantius had avoided, were considered by Julian as the place
where he could exhibit, with the most propriety, the maxims of a
republican, and the talents of a rhetorician. He alternately
practised, as in a school of declamation, the several modes of
praise, of censure, of exhortation; and his friend Libanius has
remarked, that the study of Homer taught him to imitate the
simple, concise style of Menelaus, the copiousness of Nestor,
whose words descended like the flakes of a winter’s snow, or the
pathetic and forcible eloquence of Ulysses. The functions of a
judge, which are sometimes incompatible with those of a prince,
were exercised by Julian, not only as a duty, but as an
amusement; and although he might have trusted the integrity and
discernment of his Prætorian præfects, he often placed himself by
their side on the seat of judgment. The acute penetration of his
mind was agreeably occupied in detecting and defeating the
chicanery of the advocates, who labored to disguise the truths of
facts, and to pervert the sense of the laws. He sometimes forgot
the gravity of his station, asked indiscreet or unseasonable
questions, and betrayed, by the loudness of his voice, and the
agitation of his body, the earnest vehemence with which he
maintained his opinion against the judges, the advocates, and
their clients. But his knowledge of his own temper prompted him
to encourage, and even to solicit, the reproof of his friends and
ministers; and whenever they ventured to oppose the irregular
sallies of his passions, the spectators could observe the shame,
as well as the gratitude, of their monarch. The decrees of Julian
were almost always founded on the principles of justice; and he
had the firmness to resist the two most dangerous temptations,
which assault the tribunal of a sovereign, under the specious
forms of compassion and equity. He decided the merits of the
cause without weighing the circumstances of the parties; and the
poor, whom he wished to relieve, were condemned to satisfy the
just demands of a wealthy and noble adversary. He carefully
distinguished the judge from the legislator;\textsuperscript{84} and though he
meditated a necessary reformation of the Roman jurisprudence, he
pronounced sentence according to the strict and literal
interpretation of those laws, which the magistrates were bound to
execute, and the subjects to obey.

\pagenote[82]{His eloquence is celebrated by Libanius, (Orat.
Parent. c. 75, 76, p. 300, 301,) who distinctly mentions the
orators of Homer. Socrates (l. iii. c. 1) has rashly asserted
that Julian was the only prince, since Julius Cæsar, who
harangued the senate. All the predecessors of Nero, (Tacit.
Annal. xiii. 3,) and many of his successors, possessed the
faculty of speaking in public; and it might be proved by various
examples, that they frequently exercised it in the senate.}

\pagenote[83]{Ammianus (xxi. 10) has impartially stated the
merits and defects of his judicial proceedings. Libanius (Orat.
Parent. c. 90, 91, p. 315, \&c.) has seen only the fair side, and
his picture, if it flatters the person, expresses at least the
duties, of the judge. Gregory Nazianzen, (Orat. iv. p. 120,) who
suppresses the virtues, and exaggerates even the venial faults of
the Apostate, triumphantly asks, whether such a judge was fit to
be seated between Minos and Rhadamanthus, in the Elysian Fields.}

\pagenote[84]{Of the laws which Julian enacted in a reign of
sixteen months, fifty-four have been admitted into the codes of
Theodosius and Justinian. (Gothofred. Chron. Legum, p. 64-67.)
The Abbé de la Bleterie (tom. ii. p. 329-336) has chosen one of
these laws to give an idea of Julian’s Latin style, which is
forcible and elaborate, but less pure than his Greek.}

The generality of princes, if they were stripped of their purple,
and cast naked into the world, would immediately sink to the
lowest rank of society, without a hope of emerging from their
obscurity. But the personal merit of Julian was, in some measure,
independent of his fortune. Whatever had been his choice of life,
by the force of intrepid courage, lively wit, and intense
application, he would have obtained, or at least he would have
deserved, the highest honors of his profession; and Julian might
have raised himself to the rank of minister, or general, of the
state in which he was born a private citizen. If the jealous
caprice of power had disappointed his expectations, if he had
prudently declined the paths of greatness, the employment of the
same talents in studious solitude would have placed beyond the
reach of kings his present happiness and his immortal fame. When
we inspect, with minute, or perhaps malevolent attention, the
portrait of Julian, something seems wanting to the grace and
perfection of the whole figure. His genius was less powerful and
sublime than that of Cæsar; nor did he possess the consummate
prudence of Augustus. The virtues of Trajan appear more steady
and natural, and the philosophy of Marcus is more simple and
consistent. Yet Julian sustained adversity with firmness, and
prosperity with moderation. After an interval of one hundred and
twenty years from the death of Alexander Severus, the Romans
beheld an emperor who made no distinction between his duties and
his pleasures; who labored to relieve the distress, and to revive
the spirit, of his subjects; and who endeavored always to connect
authority with merit, and happiness with virtue. Even faction,
and religious faction, was constrained to acknowledge the
superiority of his genius, in peace as well as in war, and to
confess, with a sigh, that the apostate Julian was a lover of his
country, and that he deserved the empire of the world.\textsuperscript{85}

\pagenote[85]{... Ductor fortissimus armis; Conditor et legum celeberrimus; ore
manûque Consultor patriæ; sed non consultor habendæ Religionis;
amans tercentum millia Divûm. Pertidus ille Deo, sed non et
perfidus orbi. Prudent. Apotheosis, 450, \&c.

The consciousness of a generous sentiment seems to have raised
the Christian post above his usual mediocrity.}

