\chapter{Progress of The Huns.}
\section{Part \thesection.}

\textit{Manners Of The Pastoral Nations.—Progress Of The Huns, From China
To Europe. — Flight Of The Goths. — They Pass The Danube. — Gothic
War. — Defeat And Death Of Valens. — Gratian Invests Theodosius With
The Eastern Empire. — His Character And Success. — Peace And
Settlement Of The Goths.}
\vspace{\onelineskip}

In the second year of the reign of Valentinian and Valens, on the
morning of the twenty-first day of July, the greatest part of the
Roman world was shaken by a violent and destructive earthquake.
The impression was communicated to the waters; the shores of the
Mediterranean were left dry, by the sudden retreat of the sea;
great quantities of fish were caught with the hand; large vessels
were stranded on the mud; and a curious spectator\textsuperscript{1} amused his
eye, or rather his fancy, by contemplating the various appearance
of valleys and mountains, which had never, since the formation of
the globe, been exposed to the sun. But the tide soon returned,
with the weight of an immense and irresistible deluge, which was
severely felt on the coasts of Sicily, of Dalmatia, of Greece,
and of Egypt: large boats were transported, and lodged on the
roofs of houses, or at the distance of two miles from the shore;
the people, with their habitations, were swept away by the
waters; and the city of Alexandria annually commemorated the
fatal day, on which fifty thousand persons had lost their lives
in the inundation. This calamity, the report of which was
magnified from one province to another, astonished and terrified
the subjects of Rome; and their affrighted imagination enlarged
the real extent of a momentary evil. They recollected the
preceding earthquakes, which had subverted the cities of
Palestine and Bithynia: they considered these alarming strokes as
the prelude only of still more dreadful calamities, and their
fearful vanity was disposed to confound the symptoms of a
declining empire and a sinking world.\textsuperscript{2} It was the fashion of the
times to attribute every remarkable event to the particular will
of the Deity; the alterations of nature were connected, by an
invisible chain, with the moral and metaphysical opinions of the
human mind; and the most sagacious divines could distinguish,
according to the color of their respective prejudices, that the
establishment of heresy tended to produce an earthquake; or that
a deluge was the inevitable consequence of the progress of sin
and error. Without presuming to discuss the truth or propriety of
these lofty speculations, the historian may content himself with
an observation, which seems to be justified by experience, that
man has much more to fear from the passions of his
fellow-creatures, than from the convulsions of the elements.\textsuperscript{3}
The mischievous effects of an earthquake, or deluge, a hurricane,
or the eruption of a volcano, bear a very inconsiderable portion
to the ordinary calamities of war, as they are now moderated by
the prudence or humanity of the princes of Europe, who amuse
their own leisure, and exercise the courage of their subjects, in
the practice of the military art. But the laws and manners of
modern nations protect the safety and freedom of the vanquished
soldier; and the peaceful citizen has seldom reason to complain,
that his life, or even his fortune, is exposed to the rage of
war. In the disastrous period of the fall of the Roman empire,
which may justly be dated from the reign of Valens, the happiness
and security of each individual were personally attacked; and the
arts and labors of ages were rudely defaced by the Barbarians of
Scythia and Germany. The invasion of the Huns precipitated on the
provinces of the West the Gothic nation, which advanced, in less
than forty years, from the Danube to the Atlantic, and opened a
way, by the success of their arms, to the inroads of so many
hostile tribes, more savage than themselves. The original
principle of motion was concealed in the remote countries of the
North; and the curious observation of the pastoral life of the
Scythians,\textsuperscript{4} or Tartars,\textsuperscript{5} will illustrate the latent cause of
these destructive emigrations.

\pagenote[1]{Such is the bad taste of Ammianus, (xxvi. 10,) that
it is not easy to distinguish his facts from his metaphors. Yet
he positively affirms, that he saw the rotten carcass of a ship,
\textit{ad decundum lapidem}, at Mothone, or Modon, in Peloponnesus.}

\pagenote[2]{The earthquakes and inundations are variously
described by Libanius, (Orat. de ulciscenda Juliani nece, c. x.,
in Fabricius, Bibl. Græc. tom. vii. p. 158, with a learned note
of Olearius,) Zosimus, (l. iv. p. 221,) Sozomen, (l. vi. c. 2,)
Cedrenus, (p. 310, 314,) and Jerom, (in Chron. p. 186, and tom.
i. p. 250, in Vit. Hilarion.) Epidaurus must have been
overwhelmed, had not the prudent citizens placed St. Hilarion, an
Egyptian monk, on the beach. He made the sign of the Cross; the
mountain-wave stopped, bowed, and returned.}

\pagenote[3]{Dicæarchus, the Peripatetic, composed a formal
treatise, to prove this obvious truth; which is not the most
honorable to the human species. (Cicero, de Officiis, ii. 5.)}

\pagenote[4]{The original Scythians of Herodotus (l. iv. c.
47—57, 99—101) were confined, by the Danube and the Palus Mæotis,
within a square of 4000 stadia, (400 Roman miles.) See D’Anville
(Mém. de l’Académie, tom. xxxv. p. 573—591.) Diodorus Siculus
(tom. i. l. ii. p. 155, edit. Wesseling) has marked the gradual
progress of the \textit{name} and nation.}

\pagenote[5]{The \textit{Tatars}, or Tartars, were a primitive tribe,
the rivals, and at length the subjects, of the Moguls. In the
victorious armies of Zingis Khan, and his successors, the Tartars
formed the vanguard; and the name, which first reached the ears
of foreigners, was applied to the whole nation, (Freret, in the
Hist. de l’Académie, tom. xviii. p. 60.) In speaking of all, or
any of the northern shepherds of Europe, or Asia, I indifferently
use the appellations of \textit{Scythians} or \textit{Tartars}. * Note: The
Moguls, (Mongols,) according to M. Klaproth, are a tribe of the
Tartar nation. Tableaux Hist. de l’Asie, p. 154.—M.}

The different characters that mark the civilized nations of the
globe, may be ascribed to the use, and the abuse, of reason;
which so variously shapes, and so artificially composes, the
manners and opinions of a European, or a Chinese. But the
operation of instinct is more sure and simple than that of
reason: it is much easier to ascertain the appetites of a
quadruped than the speculations of a philosopher; and the savage
tribes of mankind, as they approach nearer to the condition of
animals, preserve a stronger resemblance to themselves and to
each other. The uniform stability of their manners is the natural
consequence of the imperfection of their faculties. Reduced to a
similar situation, their wants, their desires, their enjoyments,
still continue the same: and the influence of food or climate,
which, in a more improved state of society, is suspended, or
subdued, by so many moral causes, most powerfully contributes to
form, and to maintain, the national character of Barbarians. In
every age, the immense plains of Scythia, or Tartary, have been
inhabited by vagrant tribes of hunters and shepherds, whose
indolence refuses to cultivate the earth, and whose restless
spirit disdains the confinement of a sedentary life. In every
age, the Scythians, and Tartars, have been renowned for their
invincible courage and rapid conquests. The thrones of Asia have
been repeatedly overturned by the shepherds of the North; and
their arms have spread terror and devastation over the most
fertile and warlike countries of Europe.\textsuperscript{6} On this occasion, as
well as on many others, the sober historian is forcibly awakened
from a pleasing vision; and is compelled, with some reluctance,
to confess, that the pastoral manners, which have been adorned
with the fairest attributes of peace and innocence, are much
better adapted to the fierce and cruel habits of a military life.
To illustrate this observation, I shall now proceed to consider a
nation of shepherds and of warriors, in the three important
articles of, I. Their diet; II. Their habitations; and, III.
Their exercises. The narratives of antiquity are justified by the
experience of modern times;\textsuperscript{7} and the banks of the Borysthenes,
of the Volga, or of the Selinga, will indifferently present the
same uniform spectacle of similar and native manners.\textsuperscript{8}

\pagenote[6]{Imperium Asiæ \textit{ter} quæsivere: ipsi perpetuo ab
alieno imperio, aut intacti aut invicti, mansere. Since the time
of Justin, (ii. 2,) they have multiplied this account. Voltaire,
in a few words, (tom. x. p. 64, Hist. Generale, c. 156,) has
abridged the Tartar conquests.

Oft o’er the trembling nations from afar,
\pagenote[Has Scythia breathed the living cloud of war.
Note *: Gray.—M.}

7]{The fourth book of Herodotus affords a curious
though imperfect, portrait of the Scythians. Among the moderns,
who describe the uniform scene, the Khan of Khowaresm, Abulghazi
Bahadur, expresses his native feelings; and his genealogical
history of the Tartars has been copiously illustrated by the
French and English editors. Carpin, Ascelin, and Rubruquis (in
the Hist. des Voyages, tom. vii.) represent the Moguls of the
fourteenth century. To these guides I have added Gerbillon, and
the other Jesuits, (Description de la China par du Halde, tom.
iv.,) who accurately surveyed the Chinese Tartary; and that
honest and intelligent traveller, Bell, of Antermony, (two
volumes in 4to. Glasgow, 1763.) * Note: Of the various works
published since the time of Gibbon, which throw fight on the
nomadic population of Central Asia, may be particularly remarked
the Travels and Dissertations of Pallas; and above all, the very
curious work of Bergman, Nomadische Streifereyen. Riga, 1805.—M.}

\pagenote[8]{The Uzbecks are the most altered from their
primitive manners; 1. By the profession of the Mahometan
religion; and 2. By the possession of the cities and harvests of
the great Bucharia.}

I. The corn, or even the rice, which constitutes the ordinary and
wholesome food of a civilized people, can be obtained only by the
patient toil of the husbandman. Some of the happy savages, who
dwell between the tropics, are plentifully nourished by the
liberality of nature; but in the climates of the North, a nation
of shepherds is reduced to their flocks and herds. The skilful
practitioners of the medical art will determine (if they are able
to determine) how far the temper of the human mind may be
affected by the use of animal, or of vegetable, food; and whether
the common association of carniverous and cruel deserves to be
considered in any other light than that of an innocent, perhaps a
salutary, prejudice of humanity.\textsuperscript{9} Yet, if it be true, that the
sentiment of compassion is imperceptibly weakened by the sight
and practice of domestic cruelty, we may observe, that the horrid
objects which are disguised by the arts of European refinement,
are exhibited in their naked and most disgusting simplicity in
the tent of a Tartarian shepherd. The ox, or the sheep, are
slaughtered by the same hand from which they were accustomed to
receive their daily food; and the bleeding limbs are served, with
very little preparation, on the table of their unfeeling
murderer. In the military profession, and especially in the
conduct of a numerous army, the exclusive use of animal food
appears to be productive of the most solid advantages. Corn is a
bulky and perishable commodity; and the large magazines, which
are indispensably necessary for the subsistence of our troops,
must be slowly transported by the labor of men or horses. But the
flocks and herds, which accompany the march of the Tartars,
afford a sure and increasing supply of flesh and milk: in the far
greater part of the uncultivated waste, the vegetation of the
grass is quick and luxuriant; and there are few places so
extremely barren, that the hardy cattle of the North cannot find
some tolerable pasture.

The supply is multiplied and prolonged by the undistinguishing
appetite, and patient abstinence, of the Tartars. They
indifferently feed on the flesh of those animals that have been
killed for the table, or have died of disease. Horseflesh, which
in every age and country has been proscribed by the civilized
nations of Europe and Asia, they devour with peculiar greediness;
and this singular taste facilitates the success of their military
operations. The active cavalry of Scythia is always followed, in
their most distant and rapid incursions, by an adequate number of
spare horses, who may be occasionally used, either to redouble
the speed, or to satisfy the hunger, of the Barbarians. Many are
the resources of courage and poverty. When the forage round a
camp of Tartars is almost consumed, they slaughter the greatest
part of their cattle, and preserve the flesh, either smoked, or
dried in the sun. On the sudden emergency of a hasty march, they
provide themselves with a sufficient quantity of little balls of
cheese, or rather of hard curd, which they occasionally dissolve
in water; and this unsubstantial diet will support, for many
days, the life, and even the spirits, of the patient warrior. But
this extraordinary abstinence, which the Stoic would approve, and
the hermit might envy, is commonly succeeded by the most
voracious indulgence of appetite. The wines of a happier climate
are the most grateful present, or the most valuable commodity,
that can be offered to the Tartars; and the only example of their
industry seems to consist in the art of extracting from mare’s
milk a fermented liquor, which possesses a very strong power of
intoxication. Like the animals of prey, the savages, both of the
old and new world, experience the alternate vicissitudes of
famine and plenty; and their stomach is inured to sustain,
without much inconvenience, the opposite extremes of hunger and
of intemperance.

\pagenote[9]{Il est certain que les grands mangeurs de viande
sont en général cruels et féroces plus que les autres hommes.
Cette observation est de tous les lieux, et de tous les temps: la
barbarie Angloise est connue, \&c. Emile de Rousseau, tom. i. p.
274. Whatever we may think of the general observation, \textit{we} shall
not easily allow the truth of his example. The good-natured
complaints of Plutarch, and the pathetic lamentations of Ovid,
seduce our reason, by exciting our sensibility.}

II. In the ages of rustic and martial simplicity, a people of
soldiers and husbandmen are dispersed over the face of an
extensive and cultivated country; and some time must elapse
before the warlike youth of Greece or Italy could be assembled
under the same standard, either to defend their own confines, or
to invade the territories of the adjacent tribes. The progress of
manufactures and commerce insensibly collects a large multitude
within the walls of a city: but these citizens are no longer
soldiers; and the arts which adorn and improve the state of civil
society, corrupt the habits of the military life. The pastoral
manners of the Scythians seem to unite the different advantages
of simplicity and refinement. The individuals of the same tribe
are constantly assembled, but they are assembled in a camp; and
the native spirit of these dauntless shepherds is animated by
mutual support and emulation. The houses of the Tartars are no
more than small tents, of an oval form, which afford a cold and
dirty habitation, for the promiscuous youth of both sexes. The
palaces of the rich consist of wooden huts, of such a size that
they may be conveniently fixed on large wagons, and drawn by a
team perhaps of twenty or thirty oxen. The flocks and herds,
after grazing all day in the adjacent pastures, retire, on the
approach of night, within the protection of the camp. The
necessity of preventing the most mischievous confusion, in such a
perpetual concourse of men and animals, must gradually introduce,
in the distribution, the order, and the guard, of the encampment,
the rudiments of the military art. As soon as the forage of a
certain district is consumed, the tribe, or rather army, of
shepherds, makes a regular march to some fresh pastures; and thus
acquires, in the ordinary occupations of the pastoral life, the
practical knowledge of one of the most important and difficult
operations of war. The choice of stations is regulated by the
difference of the seasons: in the summer, the Tartars advance
towards the North, and pitch their tents on the banks of a river,
or, at least, in the neighborhood of a running stream. But in the
winter, they return to the South, and shelter their camp, behind
some convenient eminence, against the winds, which are chilled in
their passage over the bleak and icy regions of Siberia. These
manners are admirably adapted to diffuse, among the wandering
tribes, the spirit of emigration and conquest. The connection
between the people and their territory is of so frail a texture,
that it may be broken by the slightest accident. The camp, and
not the soil, is the native country of the genuine Tartar. Within
the precincts of that camp, his family, his companions, his
property, are always included; and, in the most distant marches,
he is still surrounded by the objects which are dear, or
valuable, or familiar in his eyes. The thirst of rapine, the
fear, or the resentment of injury, the impatience of servitude,
have, in every age, been sufficient causes to urge the tribes of
Scythia boldly to advance into some unknown countries, where they
might hope to find a more plentiful subsistence or a less
formidable enemy. The revolutions of the North have frequently
determined the fate of the South; and in the conflict of hostile
nations, the victor and the vanquished have alternately drove,
and been driven, from the confines of China to those of Germany.\textsuperscript{10}
These great emigrations, which have been sometimes executed
with almost incredible diligence, were rendered more easy by the
peculiar nature of the climate. It is well known that the cold of
Tartary is much more severe than in the midst of the temperate
zone might reasonably be expected; this uncommon rigor is
attributed to the height of the plains, which rise, especially
towards the East, more than half a mile above the level of the
sea; and to the quantity of saltpetre with which the soil is
deeply impregnated.\textsuperscript{11} In the winter season, the broad and rapid
rivers, that discharge their waters into the Euxine, the Caspian,
or the Icy Sea, are strongly frozen; the fields are covered with
a bed of snow; and the fugitive, or victorious, tribes may
securely traverse, with their families, their wagons, and their
cattle, the smooth and hard surface of an immense plain.

\pagenote[10]{These Tartar emigrations have been discovered by M.
de Guignes (Histoire des Huns, tom. i. ii.) a skilful and
laborious interpreter of the Chinese language; who has thus laid
open new and important scenes in the history of mankind.}

\pagenote[11]{A plain in the Chinese Tartary, only eighty leagues
from the great wall, was found by the missionaries to be three
thousand geometrical paces above the level of the sea.
Montesquieu, who has used, and abused, the relations of
travellers, deduces the revolutions of Asia from this important
circumstance, that heat and cold, weakness and strength, touch
each other without any temperate zone, (Esprit des Loix, l. xvii.
c. 3.)}

III. The pastoral life, compared with the labors of agriculture
and manufactures, is undoubtedly a life of idleness; and as the
most honorable shepherds of the Tartar race devolve on their
captives the domestic management of the cattle, their own leisure
is seldom disturbed by any servile and assiduous cares. But this
leisure, instead of being devoted to the soft enjoyments of love
and harmony, is usefully spent in the violent and sanguinary
exercise of the chase. The plains of Tartary are filled with a
strong and serviceable breed of horses, which are easily trained
for the purposes of war and hunting. The Scythians of every age
have been celebrated as bold and skilful riders; and constant
practice had seated them so firmly on horseback, that they were
supposed by strangers to perform the ordinary duties of civil
life, to eat, to drink, and even to sleep, without dismounting
from their steeds. They excel in the dexterous management of the
lance; the long Tartar bow is drawn with a nervous arm; and the
weighty arrow is directed to its object with unerring aim and
irresistible force. These arrows are often pointed against the
harmless animals of the desert, which increase and multiply in
the absence of their most formidable enemy; the hare, the goat,
the roebuck, the fallow-deer, the stag, the elk, and the
antelope. The vigor and patience, both of the men and horses, are
continually exercised by the fatigues of the chase; and the
plentiful supply of game contributes to the subsistence, and even
luxury, of a Tartar camp. But the exploits of the hunters of
Scythia are not confined to the destruction of timid or innoxious
beasts; they boldly encounter the angry wild boar, when he turns
against his pursuers, excite the sluggish courage of the bear,
and provoke the fury of the tiger, as he slumbers in the thicket.
Where there is danger, there may be glory; and the mode of
hunting, which opens the fairest field to the exertions of valor,
may justly be considered as the image, and as the school, of war.
The general hunting matches, the pride and delight of the Tartar
princes, compose an instructive exercise for their numerous
cavalry. A circle is drawn, of many miles in circumference, to
encompass the game of an extensive district; and the troops that
form the circle regularly advance towards a common centre; where
the captive animals, surrounded on every side, are abandoned to
the darts of the hunters. In this march, which frequently
continues many days, the cavalry are obliged to climb the hills,
to swim the rivers, and to wind through the valleys, without
interrupting the prescribed order of their gradual progress. They
acquire the habit of directing their eye, and their steps, to a
remote object; of preserving their intervals of suspending or
accelerating their pace, according to the motions of the troops
on their right and left; and of watching and repeating the
signals of their leaders. Their leaders study, in this practical
school, the most important lesson of the military art; the prompt
and accurate judgment of ground, of distance, and of time. To
employ against a human enemy the same patience and valor, the
same skill and discipline, is the only alteration which is
required in real war; and the amusements of the chase serve as a
prelude to the conquest of an empire.\textsuperscript{12}

\pagenote[12]{Petit de la Croix (Vie de Gengiscan, l. iii. c. 6)
represents the full glory and extent of the Mogul chase. The
Jesuits Gerbillon and Verbiest followed the emperor Khamhi when
he hunted in Tartary, (Duhalde, Déscription de la Chine, tom. iv.
p. 81, 290, \&c., folio edit.) His grandson, Kienlong, who unites
the Tartar discipline with the laws and learning of China,
describes (Eloge de Moukden, p. 273—285) as a poet the pleasures
which he had often enjoyed as a sportsman.}

The political society of the ancient Germans has the appearance
of a voluntary alliance of independent warriors. The tribes of
Scythia, distinguished by the modern appellation of \textit{Hords},
assume the form of a numerous and increasing family; which, in
the course of successive generations, has been propagated from
the same original stock. The meanest, and most ignorant, of the
Tartars, preserve, with conscious pride, the inestimable treasure
of their genealogy; and whatever distinctions of rank may have
been introduced, by the unequal distribution of pastoral wealth,
they mutually respect themselves, and each other, as the
descendants of the first founder of the tribe. The custom, which
still prevails, of adopting the bravest and most faithful of the
captives, may countenance the very probable suspicion, that this
extensive consanguinity is, in a great measure, legal and
fictitious. But the useful prejudice, which has obtained the
sanction of time and opinion, produces the effects of truth; the
haughty Barbarians yield a cheerful and voluntary obedience to
the head of their blood; and their chief, or \textit{mursa}, as the
representative of their great father, exercises the authority of
a judge in peace, and of a leader in war. In the original state
of the pastoral world, each of the \textit{mursas} (if we may continue
to use a modern appellation) acted as the independent chief of a
large and separate family; and the limits of their peculiar
territories were gradually fixed by superior force, or mutual
consent. But the constant operation of various and permanent
causes contributed to unite the vagrant Hords into national
communities, under the command of a supreme head. The weak were
desirous of support, and the strong were ambitious of dominion;
the power, which is the result of union, oppressed and collected
the divided force of the adjacent tribes; and, as the vanquished
were freely admitted to share the advantages of victory, the most
valiant chiefs hastened to range themselves and their followers
under the formidable standard of a confederate nation. The most
successful of the Tartar princes assumed the military command, to
which he was entitled by the superiority, either of merit or of
power. He was raised to the throne by the acclamations of his
equals; and the title of \textit{Khan} expresses, in the language of the
North of Asia, the full extent of the regal dignity. The right of
hereditary succession was long confined to the blood of the
founder of the monarchy; and at this moment all the Khans, who
reign from Crimea to the wall of China, are the lineal
descendants of the renowned Zingis.\textsuperscript{13} But, as it is the
indispensable duty of a Tartar sovereign to lead his warlike
subjects into the field, the claims of an infant are often
disregarded; and some royal kinsman, distinguished by his age and
valor, is intrusted with the sword and sceptre of his
predecessor. Two distinct and regular taxes are levied on the
tribes, to support the dignity of the national monarch, and of
their peculiar chief; and each of those contributions amounts to
the tithe, both of their property, and of their spoil. A Tartar
sovereign enjoys the tenth part of the wealth of his people; and
as his own domestic riches of flocks and herds increase in a much
larger proportion, he is able plentifully to maintain the rustic
splendor of his court, to reward the most deserving, or the most
favored of his followers, and to obtain, from the gentle
influence of corruption, the obedience which might be sometimes
refused to the stern mandates of authority. The manners of his
subjects, accustomed, like himself, to blood and rapine, might
excuse, in their eyes, such partial acts of tyranny, as would
excite the horror of a civilized people; but the power of a
despot has never been acknowledged in the deserts of Scythia. The
immediate jurisdiction of the khan is confined within the limits
of his own tribe; and the exercise of his royal prerogative has
been moderated by the ancient institution of a national council.
The Coroulai,\textsuperscript{14} or Diet, of the Tartars, was regularly held in
the spring and autumn, in the midst of a plain; where the princes
of the reigning family, and the mursas of the respective tribes,
may conveniently assemble on horseback, with their martial and
numerous trains; and the ambitious monarch, who reviewed the
strength, must consult the inclination of an armed people. The
rudiments of a feudal government may be discovered in the
constitution of the Scythian or Tartar nations; but the perpetual
conflict of those hostile nations has sometimes terminated in the
establishment of a powerful and despotic empire. The victor,
enriched by the tribute, and fortified by the arms of dependent
kings, has spread his conquests over Europe or Asia: the
successful shepherds of the North have submitted to the
confinement of arts, of laws, and of cities; and the introduction
of luxury, after destroying the freedom of the people, has
undermined the foundations of the throne.\textsuperscript{15}

\pagenote[13]{See the second volume of the Genealogical History
of the Tartars; and the list of the Khans, at the end of the life
of Geng’s, or Zingis. Under the reign of Timur, or Tamerlane, one
of his subjects, a descendant of Zingis, still bore the regal
appellation of Khan and the conqueror of Asia contented himself
with the title of Emir or Sultan. Abulghazi, part v. c. 4.
D’Herbelot, Bibliothèque Orien tale, p. 878.}

\pagenote[14]{See the Diets of the ancient Huns, (De Guignes,
tom. ii. p. 26,) and a curious description of those of Zingis,
(Vie de Gengiscan, l. i. c. 6, l. iv. c. 11.) Such assemblies are
frequently mentioned in the Persian history of Timur; though they
served only to countenance the resolutions of their master.}

\pagenote[15]{Montesquieu labors to explain a difference, which
has not existed, between the liberty of the Arabs, and the
\textit{perpetual} slavery of the Tartars. (Esprit des Loix, l. xvii. c.
5, l. xviii. c. 19, \&c.)}

The memory of past events cannot long be preserved in the
frequent and remote emigrations of illiterate Barbarians. The
modern Tartars are ignorant of the conquests of their ancestors;\textsuperscript{16}
and our knowledge of the history of the Scythians is derived
from their intercourse with the learned and civilized nations of
the South, the Greeks, the Persians, and the Chinese. The Greeks,
who navigated the Euxine, and planted their colonies along the
sea-coast, made the gradual and imperfect discovery of Scythia;
from the Danube, and the confines of Thrace, as far as the frozen
Mæotis, the seat of eternal winter, and Mount Caucasus, which, in
the language of poetry, was described as the utmost boundary of
the earth. They celebrated, with simple credulity, the virtues of
the pastoral life:\textsuperscript{17} they entertained a more rational
apprehension of the strength and numbers of the warlike
Barbarians,\textsuperscript{18} who contemptuously baffled the immense armament of
Darius, the son of Hystaspes.\textsuperscript{19} The Persian monarchs had
extended their western conquests to the banks of the Danube, and
the limits of European Scythia. The eastern provinces of their
empire were exposed to the Scythians of Asia; the wild
inhabitants of the plains beyond the Oxus and the Jaxartes, two
mighty rivers, which direct their course towards the Caspian Sea.
The long and memorable quarrel of Iran and Touran is still the
theme of history or romance: the famous, perhaps the fabulous,
valor of the Persian heroes, Rustan and Asfendiar, was
signalized, in the defence of their country, against the
Afrasiabs of the North;\textsuperscript{20} and the invincible spirit of the same
Barbarians resisted, on the same ground, the victorious arms of
Cyrus and Alexander.\textsuperscript{21} In the eyes of the Greeks and Persians,
the real geography of Scythia was bounded, on the East, by the
mountains of Imaus, or Caf; and their distant prospect of the
extreme and inaccessible parts of Asia was clouded by ignorance,
or perplexed by fiction. But those inaccessible regions are the
ancient residence of a powerful and civilized nation,\textsuperscript{22} which
ascends, by a probable tradition, above forty centuries;\textsuperscript{23} and
which is able to verify a series of near two thousand years, by
the perpetual testimony of accurate and contemporary historians.\textsuperscript{24}
The annals of China\textsuperscript{25} illustrate the state and revolutions of
the pastoral tribes, which may still be distinguished by the
vague appellation of Scythians, or Tartars; the vassals, the
enemies, and sometimes the conquerors, of a great empire; whose
policy has uniformly opposed the blind and impetuous valor of the
Barbarians of the North. From the mouth of the Danube to the Sea
of Japan, the whole longitude of Scythia is about one hundred and
ten degrees, which, in that parallel, are equal to more than five
thousand miles. The latitude of these extensive deserts cannot be
so easily, or so accurately, measured; but, from the fortieth
degree, which touches the wall of China, we may securely advance
above a thousand miles to the northward, till our progress is
stopped by the excessive cold of Siberia. In that dreary climate,
instead of the animated picture of a Tartar camp, the smoke that
issues from the earth, or rather from the snow, betrays the
subterraneous dwellings of the Tongouses, and the Samoides: the
want of horses and oxen is imperfectly supplied by the use of
reindeer, and of large dogs; and the conquerors of the earth
insensibly degenerate into a race of deformed and diminutive
savages, who tremble at the sound of arms.\textsuperscript{26}

\pagenote[16]{Abulghasi Khan, in the two first parts of his
Genealogical History, relates the miserable tales and traditions
of the Uzbek Tartars concerning the times which preceded the
reign of Zingis. * Note: The differences between the various
pastoral tribes and nations comprehended by the ancients under
the vague name of Scythians, and by Gibbon under inst of Tartars,
have received some, and still, perhaps, may receive more, light
from the comparisons of their dialects and languages by modern
scholars.—M}

\pagenote[17]{In the thirteenth book of the Iliad, Jupiter turns
away his eyes from the bloody fields of Troy, to the plains of
Thrace and Scythia. He would not, by changing the prospect,
behold a more peaceful or innocent scene.}

\pagenote[18]{Thucydides, l. ii. c. 97.}

\pagenote[19]{See the fourth book of Herodotus. When Darius
advanced into the Moldavian desert, between the Danube and the
Niester, the king of the Scythians sent him a mouse, a frog, a
bird, and five arrows; a tremendous allegory!}

\pagenote[20]{These wars and heroes may be found under their
respective \textit{titles}, in the Bibliothèque Orientale of D’Herbelot.
They have been celebrated in an epic poem of sixty thousand
rhymed couplets, by Ferdusi, the Homer of Persia. See the history
of Nadir Shah, p. 145, 165. The public must lament that Mr. Jones
has suspended the pursuit of Oriental learning. Note: Ferdusi is
yet imperfectly known to European readers. An abstract of the
whole poem has been published by Goerres in German, under the
title “das Heldenbuch des Iran.” In English, an abstract with
poetical translations, by Mr. Atkinson, has appeared, under the
auspices of the Oriental Fund. But to translate a poet a man must
be a poet. The best account of the poem is in an article by Von
Hammer in the Vienna Jahrbucher, 1820: or perhaps in a masterly
article in Cochrane’s Foreign Quarterly Review, No. 1, 1835. A
splendid and critical edition of the whole work has been
published by a very learned English Orientalist, Captain Macan,
at the expense of the king of Oude. As to the number of 60,000
couplets, Captain Macan (Preface, p. 39) states that he never saw
a MS. containing more than 56,685, including doubtful and
spurious passages and episodes.—M. * Note: The later studies of
Sir W. Jones were more in unison with the wishes of the public,
thus expressed by Gibbon.—M.}

\pagenote[21]{The Caspian Sea, with its rivers and adjacent
tribes, are laboriously illustrated in the Examen Critique des
Historiens d’Alexandre, which compares the true geography, and
the errors produced by the vanity or ignorance of the Greeks.}

\pagenote[22]{The original seat of the nation appears to have
been in the Northwest of China, in the provinces of Chensi and
Chansi. Under the two first dynasties, the principal town was
still a movable camp; the villages were thinly scattered; more
land was employed in pasture than in tillage; the exercise of
hunting was ordained to clear the country from wild beasts;
Petcheli (where Pekin stands) was a desert, and the Southern
provinces were peopled with Indian savages. The dynasty of the
\textit{Han} (before Christ 206) gave the empire its actual form and
extent.}

\pagenote[23]{The æra of the Chinese monarchy has been variously
fixed from 2952 to 2132 years before Christ; and the year 2637
has been chosen for the lawful epoch, by the authority of the
present emperor. The difference arises from the uncertain
duration of the two first dynasties; and the vacant space that
lies beyond them, as far as the real, or fabulous, times of Fohi,
or Hoangti. Sematsien dates his authentic chronology from the
year 841; the thirty-six eclipses of Confucius (thirty-one of
which have been verified) were observed between the years 722 and
480 before Christ. The \textit{historical} period of China does not
ascend above the Greek Olympiads.}

\pagenote[24]{After several ages of anarchy and despotism, the
dynasty of the Han (before Christ 206) was the æra of the revival
of learning. The fragments of ancient literature were restored;
the characters were improved and fixed; and the future
preservation of books was secured by the useful inventions of
ink, paper, and the art of printing. Ninety-seven years before
Christ, Sematsien published the first history of China. His
labors were illustrated, and continued, by a series of one
hundred and eighty historians. The substance of their works is
still extant; and the most considerable of them are now deposited
in the king of France’s library.}

\pagenote[25]{China has been illustrated by the labors of the
French; of the missionaries at Pekin, and Messrs. Freret and De
Guignes at Paris. The substance of the three preceding notes is
extracted from the Chou-king, with the preface and notes of M. de
Guignes, Paris, 1770. The \textit{Tong-Kien-Kang-Mou}, translated by P.
de Mailla, under the name of Hist. Génerale de la Chine, tom. i.
p. xlix.—cc.; the Mémoires sur la Chine, Paris, 1776, \&c., tom.
i. p. 1—323; tom. ii. p. 5—364; the Histoire des Huns, tom. i. p.
4—131, tom. v. p. 345—362; and the Mémoires de l’Académie des
Inscriptions, tom. x. p. 377—402; tom. xv. p. 495—564; tom.
xviii. p. 178—295; xxxvi. p. 164—238.}

\pagenote[26]{See the Histoire Generale des Voyages, tom. xviii.,
and the Genealogical History, vol. ii. p. 620—664.}

\section{Part \thesection.}

The Huns, who under the reign of Valens threatened the empire of
Rome, had been formidable, in a much earlier period, to the
empire of China.\textsuperscript{27} Their ancient, perhaps their original, seat
was an extensive, though dry and barren, tract of country,
immediately on the north side of the great wall. Their place is
at present occupied by the forty-nine Hords or Banners of the
Mongous, a pastoral nation, which consists of about two hundred
thousand families.\textsuperscript{28} But the valor of the Huns had extended the
narrow limits of their dominions; and their rustic chiefs, who
assumed the appellation of \textit{Tanjou}, gradually became the
conquerors, and the sovereigns of a formidable empire. Towards
the East, their victorious arms were stopped only by the ocean;
and the tribes, which are thinly scattered between the Amoor and
the extreme peninsula of Corea, adhered, with reluctance, to the
standard of the Huns. On the West, near the head of the Irtish,
in the valleys of Imaus, they found a more ample space, and more
numerous enemies. One of the lieutenants of the Tanjou subdued,
in a single expedition, twenty-six nations; the Igours,\textsuperscript{29}
distinguished above the Tartar race by the use of letters, were
in the number of his vassals; and, by the strange connection of
human events, the flight of one of those vagrant tribes recalled
the victorious Parthians from the invasion of Syria.\textsuperscript{30} On the
side of the North, the ocean was assigned as the limit of the
power of the Huns. Without enemies to resist their progress, or
witnesses to contradict their vanity, they might securely achieve
a real, or imaginary, conquest of the frozen regions of Siberia.
The \textit{Northern Sea} was fixed as the remote boundary of their
empire. But the name of that sea, on whose shores the patriot
Sovou embraced the life of a shepherd and an exile,\textsuperscript{31} may be
transferred, with much more probability, to the Baikal, a
capacious basin, above three hundred miles in length, which
disdains the modest appellation of a lake\textsuperscript{32} and which actually
communicates with the seas of the North, by the long course of
the Angara, the Tongusha, and the Jenissea. The submission of so
many distant nations might flatter the pride of the Tanjou; but
the valor of the Huns could be rewarded only by the enjoyment of
the wealth and luxury of the empire of the South. In the third
century\textsuperscript{3211} before the Christian æra, a wall of fifteen hundred
miles in length was constructed, to defend the frontiers of China
against the inroads of the Huns;\textsuperscript{33} but this stupendous work,
which holds a conspicuous place in the map of the world, has
never contributed to the safety of an unwarlike people. The
cavalry of the Tanjou frequently consisted of two or three
hundred thousand men, formidable by the matchless dexterity with
which they managed their bows and their horses: by their hardy
patience in supporting the inclemency of the weather; and by the
incredible speed of their march, which was seldom checked by
torrents, or precipices, by the deepest rivers, or by the most
lofty mountains. They spread themselves at once over the face of
the country; and their rapid impetuosity surprised, astonished,
and disconcerted the grave and elaborate tactics of a Chinese
army. The emperor Kaoti,\textsuperscript{34} a soldier of fortune, whose personal
merit had raised him to the throne, marched against the Huns with
those veteran troops which had been trained in the civil wars of
China. But he was soon surrounded by the Barbarians; and, after a
siege of seven days, the monarch, hopeless of relief, was reduced
to purchase his deliverance by an ignominious capitulation. The
successors of Kaoti, whose lives were dedicated to the arts of
peace, or the luxury of the palace, submitted to a more permanent
disgrace. They too hastily confessed the insufficiency of arms
and fortifications. They were too easily convinced, that while
the blazing signals announced on every side the approach of the
Huns, the Chinese troops, who slept with the helmet on their
head, and the cuirass on their back, were destroyed by the
incessant labor of ineffectual marches.\textsuperscript{35} A regular payment of
money, and silk, was stipulated as the condition of a temporary
and precarious peace; and the wretched expedient of disguising a
real tribute, under the names of a gift or subsidy, was practised
by the emperors of China as well as by those of Rome. But there
still remained a more disgraceful article of tribute, which
violated the sacred feelings of humanity and nature. The
hardships of the savage life, which destroy in their infancy the
children who are born with a less healthy and robust
constitution, introduced a remarkable disproportion between the
numbers of the two sexes. The Tartars are an ugly and even
deformed race; and while they consider their own women as the
instruments of domestic labor, their desires, or rather their
appetites, are directed to the enjoyment of more elegant beauty.
A select band of the fairest maidens of China was annually
devoted to the rude embraces of the Huns;\textsuperscript{36} and the alliance of
the haughty Tanjous was secured by their marriage with the
genuine, or adopted, daughters of the Imperial family, which
vainly attempted to escape the sacrilegious pollution. The
situation of these unhappy victims is described in the verses of
a Chinese princess, who laments that she had been condemned by
her parents to a distant exile, under a Barbarian husband; who
complains that sour milk was her only drink, raw flesh her only
food, a tent her only palace; and who expresses, in a strain of
pathetic simplicity, the natural wish, that she were transformed
into a bird, to fly back to her dear country; the object of her
tender and perpetual regret.\textsuperscript{37}

\pagenote[27]{M. de Guignes (tom. ii. p. 1—124) has given the
original history of the ancient Hiong-nou, or Huns. The Chinese
geography of their country (tom. i. part. p. lv.—lxiii.) seems to
comprise a part of their conquests. * Note: The theory of De
Guignes on the early history of the Huns is, in general, rejected
by modern writers. De Guignes advanced no valid proof of the
identity of the Hioung-nou of the Chinese writers with the Huns,
except the similarity of name. Schlozer, (Allgemeine Nordische
Geschichte, p. 252,) Klaproth, (Tableaux Historiques de l’Asie,
p. 246,) St. Martin, iv. 61, and A. Remusat, (Recherches sur les
Langues Tartares, D. P. xlvi, and p. 328; though in the latter
passage he considers the theory of De Guignes not absolutely
disproved,) concur in considering the Huns as belonging to the
Finnish stock, distinct from the Moguls the Mandscheus, and the
Turks. The Hiong-nou, according to Klaproth, were Turks. The
names of the Hunnish chiefs could not be pronounced by a Turk;
and, according to the same author, the Hioung-nou, which is
explained in Chinese as \textit{detestable slaves}, as early as the year
91 J. C., were dispersed by the Chinese, and assumed the name of
Yue-po or Yue-pan. M. St. Martin does not consider it impossible
that the appellation of Hioung-nou may have belonged to the Huns.
But all agree in considering the Madjar or Magyar of modern
Hungary the descendants of the Huns. Their language (compare
Gibbon, c. lv. n. 22) is nearly related to the Lapponian and
Vogoul. The noble forms of the modern Hungarians, so strongly
contrasted with the hideous pictures which the fears and the
hatred of the Romans give of the Huns, M. Klaproth accounts for
by the intermingling with other races, Turkish and Slavonian. The
present state of the question is thus stated in the last edition
of Malte Brun, and a new and ingenious hypothesis suggested to
resolve all the difficulties of the question.
Were the Huns Finns? This obscure question has not been
debated till very recently, and is yet very far from being
decided. We are of opinion that it will be so hereafter in
the same manner as that with regard to the Scythians. We
shall trace in the portrait of Attila a dominant tribe or
Mongols, or Kalmucks, with all the hereditary ugliness of
that race; but in the mass of the Hunnish army and nation
will be recognized the Chuni and the Ounni of the Greek
Geography. the Kuns of the Hungarians, the European Huns, and
a race in close relationship with the Flemish stock. Malte
Brun, vi. p. 94. This theory is more fully and ably
developed, p. 743. Whoever has seen the emperor of Austria’s
Hungarian guard, will not readily admit their descent from
the Huns described by Sidonius Appolinaris.—M}

\pagenote[28]{See in Duhalde (tom. iv. p. 18—65) a circumstantial
description, with a correct map, of the country of the Mongous.}

\pagenote[29]{The Igours, or Vigours, were divided into three
branches; hunters, shepherds, and husbandmen; and the last class
was despised by the two former. See Abulghazi, part ii. c. 7. *
Note: On the Ouigour or Igour characters, see the work of M. A.
Remusat, Sur les Langues Tartares. He conceives the Ouigour
alphabet of sixteen letters to have been formed from the Syriac,
and introduced by the Nestorian Christians.—Ch. ii. M.}

\pagenote[30]{Mémoires de l’Académie des Inscriptions, tom. xxv.
p. 17—33. The comprehensive view of M. de Guignes has compared
these distant events.}

\pagenote[31]{The fame of Sovou, or So-ou, his merit, and his
singular adventurers, are still celebrated in China. See the
Eloge de Moukden, p. 20, and notes, p. 241—247; and Mémoires sur
la Chine, tom. iii. p. 317—360.}

\pagenote[32]{See Isbrand Ives in Harris’s Collection, vol. ii.
p. 931; Bell’s Travels, vol. i. p. 247—254; and Gmelin, in the
Hist. Generale des Voyages, tom. xviii. 283—329. They all remark
the vulgar opinion that the \textit{holy sea} grows angry and
tempestuous if any one presumes to call it a \textit{lake}. This
grammatical nicety often excites a dispute between the absurd
superstition of the mariners and the absurd obstinacy of
travellers.}

\pagenote[3211]{224 years before Christ. It was built by
Chi-hoang-ti of the Dynasty Thsin. It is from twenty to
twenty-five feet high. Ce monument, aussi gigantesque
qu’impuissant, arreterait bien les incursions de quelques
Nomades; mais il n’a jamais empéché les invasions des Turcs, des
Mongols, et des Mandchous. Abe Remusat Rech. Asiat. 2d ser. vol.
i. p. 58—M.}

\pagenote[33]{The construction of the wall of China is mentioned
by Duhalde (tom. ii. p. 45) and De Guignes, (tom. ii. p. 59.)}

\pagenote[34]{See the life of Lieoupang, or Kaoti, in the Hist,
de la Chine, published at Paris, 1777, \&c., tom. i. p. 442—522.
This voluminous work is the translation (by the P. de Mailla) of
the \textit{Tong- Kien-Kang-Mou}, the celebrated abridgment of the great
History of Semakouang (A.D. 1084) and his continuators.}

\pagenote[35]{See a free and ample memorial, presented by a
Mandarin to the emperor Venti, (before Christ 180—157,) in
Duhalde, (tom. ii. p. 412—426,) from a collection of State papers
marked with the red pencil by Kamhi himself, (p. 354—612.)
Another memorial from the minister of war (Kang-Mou, tom. ii. p
555) supplies some curious circumstances of the manners of the
Huns.}

\pagenote[36]{A supply of women is mentioned as a customary
article of treaty and tribute, (Hist. de la Conquete de la Chine,
par les Tartares Mantcheoux, tom. i. p. 186, 187, with the note
of the editor.)}

\pagenote[37]{De Guignes, Hist. des Huns, tom. ii. p. 62.}

The conquest of China has been twice achieved by the pastoral
tribes of the North: the forces of the Huns were not inferior to
those of the Moguls, or of the Mantcheoux; and their ambition
might entertain the most sanguine hopes of success. But their
pride was humbled, and their progress was checked, by the arms
and policy of Vouti,\textsuperscript{38} the fifth emperor of the powerful dynasty
of the Han. In his long reign of fifty- four years, the
Barbarians of the southern provinces submitted to the laws and
manners of China; and the ancient limits of the monarchy were
enlarged, from the great river of Kiang, to the port of Canton.
Instead of confining himself to the timid operations of a
defensive war, his lieutenants penetrated many hundred miles into
the country of the Huns. In those boundless deserts, where it is
impossible to form magazines, and difficult to transport a
sufficient supply of provisions, the armies of Vouti were
repeatedly exposed to intolerable hardships: and, of one hundred
and forty thousand soldiers, who marched against the Barbarians,
thirty thousand only returned in safety to the feet of their
master. These losses, however, were compensated by splendid and
decisive success. The Chinese generals improved the superiority
which they derived from the temper of their arms, their chariots
of war, and the service of their Tartar auxiliaries. The camp of
the Tanjou was surprised in the midst of sleep and intemperance;
and, though the monarch of the Huns bravely cut his way through
the ranks of the enemy, he left above fifteen thousand of his
subjects on the field of battle. Yet this signal victory, which
was preceded and followed by many bloody engagements, contributed
much less to the destruction of the power of the Huns than the
effectual policy which was employed to detach the tributary
nations from their obedience. Intimidated by the arms, or allured
by the promises, of Vouti and his successors, the most
considerable tribes, both of the East and of the West, disclaimed
the authority of the Tanjou. While some acknowledged themselves
the allies or vassals of the empire, they all became the
implacable enemies of the Huns; and the numbers of that haughty
people, as soon as they were reduced to their native strength,
might, perhaps, have been contained within the walls of one of
the great and populous cities of China.\textsuperscript{39} The desertion of his
subjects, and the perplexity of a civil war, at length compelled
the Tanjou himself to renounce the dignity of an independent
sovereign, and the freedom of a warlike and high-spirited nation.
He was received at Sigan, the capital of the monarchy, by the
troops, the mandarins, and the emperor himself, with all the
honors that could adorn and disguise the triumph of Chinese
vanity.\textsuperscript{40} A magnificent palace was prepared for his reception;
his place was assigned above all the princes of the royal family;
and the patience of the Barbarian king was exhausted by the
ceremonies of a banquet, which consisted of eight courses of
meat, and of nine solemn pieces of music. But he performed, on
his knees, the duty of a respectful homage to the emperor of
China; pronounced, in his own name, and in the name of his
successors, a perpetual oath of fidelity; and gratefully accepted
a seal, which was bestowed as the emblem of his regal dependence.
After this humiliating submission, the Tanjous sometimes departed
from their allegiance and seized the favorable moments of war and
rapine; but the monarchy of the Huns gradually declined, till it
was broken, by civil dissension, into two hostile and separate
kingdoms. One of the princes of the nation was urged, by fear and
ambition, to retire towards the South with eight hords, which
composed between forty and fifty thousand families. He obtained,
with the title of Tanjou, a convenient territory on the verge of
the Chinese provinces; and his constant attachment to the service
of the empire was secured by weakness, and the desire of revenge.
From the time of this fatal schism, the Huns of the North
continued to languish about fifty years; till they were oppressed
on every side by their foreign and domestic enemies. The proud
inscription\textsuperscript{41} of a column, erected on a lofty mountain,
announced to posterity, that a Chinese army had marched seven
hundred miles into the heart of their country. The Sienpi,\textsuperscript{42} a
tribe of Oriental Tartars, retaliated the injuries which they had
formerly sustained; and the power of the Tanjous, after a reign
of thirteen hundred years, was utterly destroyed before the end
of the first century of the Christian æra.\textsuperscript{43}

\pagenote[38]{See the reign of the emperor Vouti, in the
Kang-Mou, tom. iii. p. 1—98. His various and inconsistent
character seems to be impartially drawn.}

\pagenote[39]{This expression is used in the memorial to the
emperor Venti, (Duhalde, tom. ii. p. 411.) Without adopting the
exaggerations of Marco Polo and Isaac Vossius, we may rationally
allow for Pekin two millions of inhabitants. The cities of the
South, which contain the manufactures of China, are still more
populous.}

\pagenote[40]{See the Kang-Mou, tom. iii. p. 150, and the
subsequent events under the proper years. This memorable festival
is celebrated in the Eloge de Moukden, and explained in a note by
the P. Gaubil, p. 89, 90.}

\pagenote[41]{This inscription was composed on the spot by
Parkou, President of the Tribunal of History (Kang-Mou, tom. iii.
p. 392.) Similar monuments have been discovered in many parts of
Tartary, (Histoire des Huns, tom. ii. p. 122.)}

\pagenote[42]{M. de Guignes (tom. i. p. 189) has inserted a short
account of the Sienpi.}

\pagenote[43]{The æra of the Huns is placed, by the Chinese, 1210
years before Christ. But the series of their kings does not
commence till the year 230, (Hist. des Huns, tom. ii. p. 21,
123.)}

The fate of the vanquished Huns was diversified by the various
influence of character and situation.\textsuperscript{44} Above one hundred
thousand persons, the poorest, indeed, and the most pusillanimous
of the people, were contented to remain in their native country,
to renounce their peculiar name and origin, and to mingle with
the victorious nation of the Sienpi. Fifty-eight hords, about two
hundred thousand men, ambitious of a more honorable servitude,
retired towards the South; implored the protection of the
emperors of China; and were permitted to inhabit, and to guard,
the extreme frontiers of the province of Chansi and the territory
of Ortous. But the most warlike and powerful tribes of the Huns
maintained, in their adverse fortune, the undaunted spirit of
their ancestors. The Western world was open to their valor; and
they resolved, under the conduct of their hereditary chieftains,
to conquer and subdue some remote country, which was still
inaccessible to the arms of the Sienpi, and to the laws of China.\textsuperscript{45}
The course of their emigration soon carried them beyond the
mountains of Imaus, and the limits of the Chinese geography; but
\textit{we} are able to distinguish the two great divisions of these
formidable exiles, which directed their march towards the Oxus,
and towards the Volga. The first of these colonies established
their dominion in the fruitful and extensive plains of Sogdiana,
on the eastern side of the Caspian; where they preserved the name
of Huns, with the epithet of Euthalites, or Nepthalites.\textsuperscript{4511}
Their manners were softened, and even their features were
insensibly improved, by the mildness of the climate, and their
long residence in a flourishing province,\textsuperscript{46} which might still
retain a faint impression of the arts of Greece.\textsuperscript{47} The \textit{white}
Huns, a name which they derived from the change of their
complexions, soon abandoned the pastoral life of Scythia. Gorgo,
which, under the appellation of Carizme, has since enjoyed a
temporary splendor, was the residence of the king, who exercised
a legal authority over an obedient people. Their luxury was
maintained by the labor of the Sogdians; and the only vestige of
their ancient barbarism, was the custom which obliged all the
companions, perhaps to the number of twenty, who had shared the
liberality of a wealthy lord, to be buried alive in the same
grave.\textsuperscript{48} The vicinity of the Huns to the provinces of Persia,
involved them in frequent and bloody contests with the power of
that monarchy. But they respected, in peace, the faith of
treaties; in war, the dictates of humanity; and their memorable
victory over Peroses, or Firuz, displayed the moderation, as well
as the valor, of the Barbarians. The \textit{second} division of their
countrymen, the Huns, who gradually advanced towards the
North-west, were exercised by the hardships of a colder climate,
and a more laborious march. Necessity compelled them to exchange
the silks of China for the furs of Siberia; the imperfect
rudiments of civilized life were obliterated; and the native
fierceness of the Huns was exasperated by their intercourse with
the savage tribes, who were compared, with some propriety, to the
wild beasts of the desert. Their independent spirit soon rejected
the hereditary succession of the Tanjous; and while each horde
was governed by its peculiar mursa, their tumultuary council
directed the public measures of the whole nation. As late as the
thirteenth century, their transient residence on the eastern
banks of the Volga was attested by the name of Great Hungary.\textsuperscript{49}
In the winter, they descended with their flocks and herds towards
the mouth of that mighty river; and their summer excursions
reached as high as the latitude of Saratoff, or perhaps the
conflux of the Kama. Such at least were the recent limits of the
black Calmucks,\textsuperscript{50} who remained about a century under the
protection of Russia; and who have since returned to their native
seats on the frontiers of the Chinese empire. The march, and the
return, of those wandering Tartars, whose united camp consists of
fifty thousand tents or families, illustrate the distant
emigrations of the ancient Huns.\textsuperscript{51}

\pagenote[44]{The various accidents, the downfall, and the flight
of the Huns, are related in the Kang-Mou, tom. iii. p. 88, 91,
95, 139, \&c. The small numbers of each horde may be due to their
losses and divisions.}

\pagenote[45]{M. de Guignes has skilfully traced the footsteps of
the Huns through the vast deserts of Tartary, (tom. ii. p. 123,
277, \&c., 325, \&c.)}

\pagenote[4511]{The Armenian authors often mention this people
under the name of Hepthal. St. Martin considers that the name of
Nepthalites is an error of a copyist. St. Martin, iv. 254.—M.}

\pagenote[46]{Mohammed, sultan of Carizme, reigned in Sogdiana
when it was invaded (A.D. 1218) by Zingis and his moguls. The
Oriental historians (see D’Herbelot, Petit de la Croix, \&c.,)
celebrate the populous cities which he ruined, and the fruitful
country which he desolated. In the next century, the same
provinces of Chorasmia and Nawaralnahr were described by
Abulfeda, (Hudson, Geograph. Minor. tom. iii.) Their actual
misery may be seen in the Genealogical History of the Tartars, p.
423—469.}

\pagenote[47]{Justin (xli. 6) has left a short abridgment of the
Greek kings of Bactriana. To their industry I should ascribe the
new and extraordinary trade, which transported the merchandises
of India into Europe, by the Oxus, the Caspian, the Cyrus, the
Phasis, and the Euxine. The other ways, both of the land and sea,
were possessed by the Seleucides and the Ptolemies. (See l’Esprit
des Loix, l. xxi.)}

\pagenote[48]{Procopius de Bell. Persico, l. i. c. 3, p. 9.}

\pagenote[49]{In the thirteenth century, the monk Rubruquis (who
traversed the immense plain of Kipzak, in his journey to the
court of the Great Khan) observed the remarkable name of
\textit{Hungary}, with the traces of a common language and origin,
(Hist. des Voyages, tom. vii. p. 269.)}

\pagenote[50]{Bell, (vol. i. p. 29—34,) and the editors of the
Genealogical History, (p. 539,) have described the Calmucks of
the Volga in the beginning of the present century.}

\pagenote[51]{This great transmigration of 300,000 Calmucks, or
Torgouts, happened in the year 1771. The original narrative of
Kien-long, the reigning emperor of China, which was intended for
the inscription of a column, has been translated by the
missionaries of Pekin, (Mémoires sur la Chine, tom. i. p.
401—418.) The emperor affects the smooth and specious language of
the Son of Heaven, and the Father of his People.}

It is impossible to fill the dark interval of time, which
elapsed, after the Huns of the Volga were lost in the eyes of the
Chinese, and before they showed themselves to those of the
Romans. There is some reason, however, to apprehend, that the
same force which had driven them from their native seats, still
continued to impel their march towards the frontiers of Europe.
The power of the Sienpi, their implacable enemies, which extended
above three thousand miles from East to West,\textsuperscript{52} must have
gradually oppressed them by the weight and terror of a formidable
neighborhood; and the flight of the tribes of Scythia would
inevitably tend to increase the strength or to contract the
territories, of the Huns. The harsh and obscure appellations of
those tribes would offend the ear, without informing the
understanding, of the reader; but I cannot suppress the very
natural suspicion, \textit{that} the Huns of the North derived a
considerable reenforcement from the ruin of the dynasty of the
South, which, in the course of the third century, submitted to
the dominion of China; \textit{that} the bravest warriors marched away
in search of their free and adventurous countrymen; \textit{and} that,
as they had been divided by prosperity, they were easily reunited
by the common hardships of their adverse fortune.\textsuperscript{53} The Huns,
with their flocks and herds, their wives and children, their
dependents and allies, were transported to the west of the Volga,
and they boldly advanced to invade the country of the Alani, a
pastoral people, who occupied, or wasted, an extensive tract of
the deserts of Scythia. The plains between the Volga and the
Tanais were covered with the tents of the Alani, but their name
and manners were diffused over the wide extent of their
conquests; and the painted tribes of the Agathyrsi and Geloni
were confounded among their vassals. Towards the North, they
penetrated into the frozen regions of Siberia, among the savages
who were accustomed, in their rage or hunger, to the taste of
human flesh; and their Southern inroads were pushed as far as the
confines of Persia and India. The mixture of Samartic and German
blood had contributed to improve the features of the Alani,\textsuperscript{5311}
to whiten their swarthy complexions, and to tinge their hair with
a yellowish cast, which is seldom found in the Tartar race. They
were less deformed in their persons, less brutish in their
manners, than the Huns; but they did not yield to those
formidable Barbarians in their martial and independent spirit; in
the love of freedom, which rejected even the use of domestic
slaves; and in the love of arms, which considered war and rapine
as the pleasure and the glory of mankind. A naked cimeter, fixed
in the ground, was the only object of their religious worship;
the scalps of their enemies formed the costly trappings of their
horses; and they viewed, with pity and contempt, the
pusillanimous warriors, who patiently expected the infirmities of
age, and the tortures of lingering disease.\textsuperscript{54} On the banks of
the Tanais, the military power of the Huns and the Alani
encountered each other with equal valor, but with unequal
success. The Huns prevailed in the bloody contest; the king of
the Alani was slain; and the remains of the vanquished nation
were dispersed by the ordinary alternative of flight or
submission.\textsuperscript{55} A colony of exiles found a secure refuge in the
mountains of Caucasus, between the Euxine and the Caspian, where
they still preserve their name and their independence. Another
colony advanced, with more intrepid courage, towards the shores
of the Baltic; associated themselves with the Northern tribes of
Germany; and shared the spoil of the Roman provinces of Gaul and
Spain. But the greatest part of the nation of the Alani embraced
the offers of an honorable and advantageous union; and the Huns,
who esteemed the valor of their less fortunate enemies,
proceeded, with an increase of numbers and confidence, to invade
the limits of the Gothic empire.

\pagenote[52]{The Khan-Mou (tom. iii. p. 447) ascribes to their
conquests a space of 14,000 \textit{lis}. According to the present
standard, 200 \textit{lis} (or more accurately 193) are equal to one
degree of latitude; and one English mile consequently exceeds
three miles of China. But there are strong reasons to believe
that the ancient \textit{li} scarcely equalled one half of the modern.
See the elaborate researches of M. D’Anville, a geographer who is
not a stranger in any age or climate of the globe. (Mémoires de
l’Acad. tom. ii. p. 125-502. Itineraires, p. 154-167.)}

\pagenote[53]{See Histoire des Huns, tom. ii. p. 125—144. The
subsequent history (p. 145—277) of three or four Hunnic dynasties
evidently proves that their martial spirit was not impaired by a
long residence in China.}

\pagenote[5311]{Compare M. Klaproth’s curious speculations on the
Alani. He supposes them to have been the people, known by the
Chinese, at the time of their first expeditions to the West,
under the name of Yath-sai or A-lanna, the Alanân of Persian
tradition, as preserved in Ferdusi; the same, according to
Ammianus, with the Massagetæ, and with the Albani. The remains of
the nation still exist in the Ossetæ of Mount Caucasus. Klaproth,
Tableaux Historiques de l’Asie, p. 174.—M. Compare Shafarik
Slawische alterthümer, i. p. 350.—M. 1845.}

\pagenote[54]{Utque hominibus quietis et placidis otium est
voluptabile, ita illos pericula juvent et bella. Judicatur ibi
beatus qui in prœlio profuderit animam: senescentes etiam et
fortuitis mortibus mundo digressos, ut degeneres et ignavos,
conviciis atrocibus insectantur. [Ammian. xxxi. 11.] We must
think highly of the conquerors of \textit{such} men.}

\pagenote[55]{On the subject of the Alani, see Ammianus, (xxxi.
2,) Jornandes, (de Rebus Geticis, c. 24,) M. de Guignes, (Hist.
des Huns, tom. ii. p. 279,) and the Genealogical History of the
Tartars, (tom. ii. p. 617.)}

The great Hermanric, whose dominions extended from the Baltic to
the Euxine, enjoyed, in the full maturity of age and reputation,
the fruit of his victories, when he was alarmed by the formidable
approach of a host of unknown enemies,\textsuperscript{56} on whom his barbarous
subjects might, without injustice, bestow the epithet of
Barbarians. The numbers, the strength, the rapid motions, and the
implacable cruelty of the Huns, were felt, and dreaded, and
magnified, by the astonished Goths; who beheld their fields and
villages consumed with flames, and deluged with indiscriminate
slaughter. To these real terrors they added the surprise and
abhorrence which were excited by the shrill voice, the uncouth
gestures, and the strange deformity of the Huns.\textsuperscript{5611} These
savages of Scythia were compared (and the picture had some
resemblance) to the animals who walk very awkwardly on two legs
and to the misshapen figures, the \textit{Termini}, which were often
placed on the bridges of antiquity. They were distinguished from
the rest of the human species by their broad shoulders, flat
noses, and small black eyes, deeply buried in the head; and as
they were almost destitute of beards, they never enjoyed either
the manly grace of youth, or the venerable aspect of age.\textsuperscript{57} A
fabulous origin was assigned, worthy of their form and manners;
that the witches of Scythia, who, for their foul and deadly
practices, had been driven from society, had copulated in the
desert with infernal spirits; and that the Huns were the
offspring of this execrable conjunction.\textsuperscript{58} The tale, so full of
horror and absurdity, was greedily embraced by the credulous
hatred of the Goths; but, while it gratified their hatred, it
increased their fear, since the posterity of dæmons and witches
might be supposed to inherit some share of the præternatural
powers, as well as of the malignant temper, of their parents.
Against these enemies, Hermanric prepared to exert the united
forces of the Gothic state; but he soon discovered that his
vassal tribes, provoked by oppression, were much more inclined to
second, than to repel, the invasion of the Huns. One of the
chiefs of the Roxolani\textsuperscript{59} had formerly deserted the standard of
Hermanric, and the cruel tyrant had condemned the innocent wife
of the traitor to be torn asunder by wild horses. The brothers of
that unfortunate woman seized the favorable moment of revenge.

The aged king of the Goths languished some time after the
dangerous wound which he received from their daggers; but the
conduct of the war was retarded by his infirmities; and the
public councils of the nation were distracted by a spirit of
jealousy and discord. His death, which has been imputed to his
own despair, left the reins of government in the hands of
Withimer, who, with the doubtful aid of some Scythian
mercenaries, maintained the unequal contest against the arms of
the Huns and the Alani, till he was defeated and slain in a
decisive battle. The Ostrogoths submitted to their fate; and the
royal race of the Amali will hereafter be found among the
subjects of the haughty Attila. But the person of Witheric, the
infant king, was saved by the diligence of Alatheus and Saphrax;
two warriors of approved valor and fiedlity, who, by cautious
marches, conducted the independent remains of the nation of the
Ostrogoths towards the Danastus, or Niester; a considerable
river, which now separates the Turkish dominions from the empire
of Russia. On the banks of the Niester, the prudent Athanaric,
more attentive to his own than to the general safety, had fixed
the camp of the Visigoths; with the firm resolution of opposing
the victorious Barbarians, whom he thought it less advisable to
provoke. The ordinary speed of the Huns was checked by the weight
of baggage, and the encumbrance of captives; but their military
skill deceived, and almost destroyed, the army of Athanaric.
While the Judge of the Visigoths defended the banks of the
Niester, he was encompassed and attacked by a numerous detachment
of cavalry, who, by the light of the moon, had passed the river
in a fordable place; and it was not without the utmost efforts of
courage and conduct, that he was able to effect his retreat
towards the hilly country. The undaunted general had already
formed a new and judicious plan of defensive war; and the strong
lines, which he was preparing to construct between the mountains,
the Pruth, and the Danube, would have secured the extensive and
fertile territory that bears the modern name of Walachia, from
the destructive inroads of the Huns.\textsuperscript{60} But the hopes and
measures of the Judge of the Visigoths was soon disappointed, by
the trembling impatience of his dismayed countrymen; who were
persuaded by their fears, that the interposition of the Danube
was the only barrier that could save them from the rapid pursuit,
and invincible valor, of the Barbarians of Scythia. Under the
command of Fritigern and Alavivus,\textsuperscript{61} the body of the nation
hastily advanced to the banks of the great river, and implored
the protection of the Roman emperor of the East. Athanaric
himself, still anxious to avoid the guilt of perjury, retired,
with a band of faithful followers, into the mountainous country
of Caucaland; which appears to have been guarded, and almost
concealed, by the impenetrable forests of Transylvania.\textsuperscript{62} \textsuperscript{6211}

\pagenote[56]{As we are possessed of the authentic history of the
Huns, it would be impertinent to repeat, or to refute, the fables
which misrepresent their origin and progress, their passage of
the mud or water of the Mæotis, in pursuit of an ox or stag, les
Indes qu’ils avoient découvertes, \&c., (Zosimus, l. iv. p. 224.
Sozomen, l. vi. c. 37. Procopius, Hist. Miscell. c. 5. Jornandes,
c. 24. Grandeur et Décadence, \&c., des Romains, c. 17.)}

\pagenote[5611]{Art added to their native ugliness; in fact, it
is difficult to ascribe the proper share in the features of this
hideous picture to nature, to the barbarous skill with which they
were self-disfigured, or to the terror and hatred of the Romans.
Their noses were flattened by their nurses, their cheeks were
gashed by an iron instrument, that the scars might look more
fearful, and prevent the growth of the beard. Jornandes and
Sidonius Apollinaris:—

Obtundit teneras circumdata fascia nares, Ut galeis cedant.

Yet he adds that their forms were robust and manly, their height
of a middle size, but, from the habit of riding, disproportioned.

Stant pectora vasta, Insignes humer, succincta sub ilibus alvus.
Forma quidem pediti media est, procera sed extat Si cernas
equites, sic longi sæpe putantur Si sedeant.}

\pagenote[57]{Prodigiosæ formæ, et pandi; ut bipedes existimes
bestias; vel quales in commarginandis pontibus, effigiati
stipites dolantur incompte. Ammian. xxxi. i. Jornandes (c. 24)
draws a strong caricature of a Calmuck face. Species pavenda
nigredine... quædam deformis offa, non fecies; habensque magis
puncta quam lumina. See Buffon. Hist. Naturelle, tom. iii. 380.}

\pagenote[58]{This execrable origin, which Jornandes (c. 24)
describes with the rancor of a Goth, might be originally derived
from a more pleasing fable of the Greeks. (Herodot. l. iv. c. 9,
\&c.)}

\pagenote[59]{The Roxolani may be the fathers of the the
\textit{Russians}, (D’Anville, Empire de Russie, p. 1—10,) whose
residence (A.D. 862) about Novogrod Veliki cannot be very remote
from that which the Geographer of Ravenna (i. 12, iv. 4, 46, v.
28, 30) assigns to the Roxolani, (A.D. 886.) * Note: See, on the
origin of the Russ, Schlozer, Nordische Geschichte, p. 78—M.}

\pagenote[60]{The text of Ammianus seems to be imperfect or
corrupt; but the nature of the ground explains, and almost
defines, the Gothic rampart. Mémoires de l’Académie, \&c., tom.
xxviii. p. 444—462.}

\pagenote[61]{M. de Buat (Hist. des Peuples de l’Europe, tom. vi.
p. 407) has conceived a strange idea, that Alavivus was the same
person as Ulphilas, the Gothic bishop; and that Ulphilas, the
grandson of a Cappadocian captive, became a temporal prince of
the Goths.}

\pagenote[62]{Ammianus (xxxi. 3) and Jornandes (de Rebus Geticis,
c. 24) describe the subversion of the Gothic empire by the Huns.}

\pagenote[6211]{The most probable opinion as to the position of
this land is that of M. Malte-Brun. He thinks that Caucaland is
the territory of the Cacoenses, placed by Ptolemy (l. iii. c. 8)
towards the Carpathian Mountains, on the side of the present
Transylvania, and therefore the canton of Cacava, to the south of
Hermanstadt, the capital of the principality. Caucaland it is
evident, is the Gothic form of these different names. St. Martin,
iv 103.—M.}

\section{Part \thesection.}

After Valens had terminated the Gothic war with some appearance
of glory and success, he made a progress through his dominions of
Asia, and at length fixed his residence in the capital of Syria.
The five years\textsuperscript{63} which he spent at Antioch was employed to
watch, from a secure distance, the hostile designs of the Persian
monarch; to check the depredations of the Saracens and Isaurians;\textsuperscript{64}
to enforce, by arguments more prevalent than those of reason
and eloquence, the belief of the Arian theology; and to satisfy
his anxious suspicions by the promiscuous execution of the
innocent and the guilty. But the attention of the emperor was
most seriously engaged, by the important intelligence which he
received from the civil and military officers who were intrusted
with the defence of the Danube. He was informed, that the North
was agitated by a furious tempest; that the irruption of the
Huns, an unknown and monstrous race of savages, had subverted the
power of the Goths; and that the suppliant multitudes of that
warlike nation, whose pride was now humbled in the dust, covered
a space of many miles along the banks of the river. With
outstretched arms, and pathetic lamentations, they loudly
deplored their past misfortunes and their present danger;
acknowledged that their only hope of safety was in the clemency
of the Roman government; and most solemnly protested, that if the
gracious liberality of the emperor would permit them to cultivate
the waste lands of Thrace, they should ever hold themselves
bound, by the strongest obligations of duty and gratitude, to
obey the laws, and to guard the limits, of the republic. These
assurances were confirmed by the ambassadors of the Goths,\textsuperscript{6411}
who impatiently expected from the mouth of Valens an answer that
must finally determine the fate of their unhappy countrymen. The
emperor of the East was no longer guided by the wisdom and
authority of his elder brother, whose death happened towards the
end of the preceding year; and as the distressful situation of
the Goths required an instant and peremptory decision, he was
deprived of the favorite resources of feeble and timid minds, who
consider the use of dilatory and ambiguous measures as the most
admirable efforts of consummate prudence. As long as the same
passions and interests subsist among mankind, the questions of
war and peace, of justice and policy, which were debated in the
councils of antiquity, will frequently present themselves as the
subject of modern deliberation. But the most experienced
statesman of Europe has never been summoned to consider the
propriety, or the danger, of admitting, or rejecting, an
innumerable multitude of Barbarians, who are driven by despair
and hunger to solicit a settlement on the territories of a
civilized nation. When that important proposition, so essentially
connected with the public safety, was referred to the ministers
of Valens, they were perplexed and divided; but they soon
acquiesced in the flattering sentiment which seemed the most
favorable to the pride, the indolence, and the avarice of their
sovereign. The slaves, who were decorated with the titles of
præfects and generals, dissembled or disregarded the terrors of
this national emigration; so extremely different from the partial
and accidental colonies, which had been received on the extreme
limits of the empire. But they applauded the liberality of
fortune, which had conducted, from the most distant countries of
the globe, a numerous and invincible army of strangers, to defend
the throne of Valens; who might now add to the royal treasures
the immense sums of gold supplied by the provincials to
compensate their annual proportion of recruits. The prayers of
the Goths were granted, and their service was accepted by the
Imperial court: and orders were immediately despatched to the
civil and military governors of the Thracian diocese, to make the
necessary preparations for the passage and subsistence of a great
people, till a proper and sufficient territory could be allotted
for their future residence. The liberality of the emperor was
accompanied, however, with two harsh and rigorous conditions,
which prudence might justify on the side of the Romans; but which
distress alone could extort from the indignant Goths. Before they
passed the Danube, they were required to deliver their arms: and
it was insisted, that their children should be taken from them,
and dispersed through the provinces of Asia; where they might be
civilized by the arts of education, and serve as hostages to
secure the fidelity of their parents.

\pagenote[63]{The Chronology of Ammianus is obscure and
imperfect. Tillemont has labored to clear and settle the annals
of Valens.}

\pagenote[64]{Zosimus, l. iv. p. 223. Sozomen, l. vi. c. 38. The
Isaurians, each winter, infested the roads of Asia Minor, as far
as the neighborhood of Constantinople. Basil, Epist. cel. apud
Tillemont, Hist. des Empereurs, tom. v. p. 106.}

\pagenote[6411]{Sozomen and Philostorgius say that the bishop
Ulphilas was one of these ambassadors.—M.}

During the suspense of a doubtful and distant negotiation, the
impatient Goths made some rash attempts to pass the Danube,
without the permission of the government, whose protection they
had implored. Their motions were strictly observed by the
vigilance of the troops which were stationed along the river and
their foremost detachments were defeated with considerable
slaughter; yet such were the timid councils of the reign of
Valens, that the brave officers who had served their country in
the execution of their duty, were punished by the loss of their
employments, and narrowly escaped the loss of their heads. The
Imperial mandate was at length received for transporting over the
Danube the whole body of the Gothic nation;\textsuperscript{65} but the execution
of this order was a task of labor and difficulty. The stream of
the Danube, which in those parts is above a mile broad,\textsuperscript{66} had
been swelled by incessant rains; and in this tumultuous passage,
many were swept away, and drowned, by the rapid violence of the
current. A large fleet of vessels, of boats, and of canoes, was
provided; many days and nights they passed and repassed with
indefatigable toil; and the most strenuous diligence was exerted
by the officers of Valens, that not a single Barbarian, of those
who were reserved to subvert the foundations of Rome, should be
left on the opposite shore. It was thought expedient that an
accurate account should be taken of their numbers; but the
persons who were employed soon desisted, with amazement and
dismay, from the prosecution of the endless and impracticable
task:\textsuperscript{67} and the principal historian of the age most seriously
affirms, that the prodigious armies of Darius and Xerxes, which
had so long been considered as the fables of vain and credulous
antiquity, were now justified, in the eyes of mankind, by the
evidence of fact and experience. A probable testimony has fixed
the number of the Gothic warriors at two hundred thousand men:
and if we can venture to add the just proportion of women, of
children, and of slaves, the whole mass of people which composed
this formidable emigration, must have amounted to near a million
of persons, of both sexes, and of all ages. The children of the
Goths, those at least of a distinguished rank, were separated
from the multitude. They were conducted, without delay, to the
distant seats assigned for their residence and education; and as
the numerous train of hostages or captives passed through the
cities, their gay and splendid apparel, their robust and martial
figure, excited the surprise and envy of the Provincials.\textsuperscript{6711}
But the stipulation, the most offensive to the Goths, and the
most important to the Romans, was shamefully eluded. The
Barbarians, who considered their arms as the ensigns of honor and
the pledges of safety, were disposed to offer a price, which the
lust or avarice of the Imperial officers was easily tempted to
accept. To preserve their arms, the haughty warriors consented,
with some reluctance, to prostitute their wives or their
daughters; the charms of a beauteous maid, or a comely boy,
secured the connivance of the inspectors; who sometimes cast an
eye of covetousness on the fringed carpets and linen garments of
their new allies,\textsuperscript{68} or who sacrificed their duty to the mean
consideration of filling their farms with cattle, and their
houses with slaves. The Goths, with arms in their hands, were
permitted to enter the boats; and when their strength was
collected on the other side of the river, the immense camp which
was spread over the plains and the hills of the Lower Mæsia,
assumed a threatening and even hostile aspect. The leaders of the
Ostrogoths, Alatheus and Saphrax, the guardians of their infant
king, appeared soon afterwards on the Northern banks of the
Danube; and immediately despatched their ambassadors to the court
of Antioch, to solicit, with the same professions of allegiance
and gratitude, the same favor which had been granted to the
suppliant Visigoths. The absolute refusal of Valens suspended
their progress, and discovered the repentance, the suspicions,
and the fears, of the Imperial council.

\pagenote[65]{The passage of the Danube is exposed by Ammianus,
(xxxi. 3, 4,) Zosimus, (l. iv. p. 223, 224,) Eunapius in Excerpt.
Legat. (p. 19, 20,) and Jornandes, (c. 25, 26.) Ammianus declares
(c. 5) that he means only, ispas rerum digerere \textit{summitates}. But
he often takes a false measure of their importance; and his
superfluous prolixity is disagreeably balanced by his
unseasonable brevity.}

\pagenote[66]{Chishull, a curious traveller, has remarked the
breadth of the Danube, which he passed to the south of Bucharest
near the conflux of the Argish, (p. 77.) He admires the beauty
and spontaneous plenty of Mæsia, or Bulgaria.}

\pagenote[67]{
Quem sci scire velit, Libyci velit æquoris idem Discere quam multæ
Zephyro turbentur harenæ.

Ammianus has inserted, in his prose, these lines of Virgil,
(Georgia l. ii. 105,) originally designed by the poet to express
the impossibility of numbering the different sorts of vines. See
Plin. Hist. Natur l. xiv.}

\pagenote[6711]{A very curious, but obscure, passage of Eunapius,
appears to me to have been misunderstood by M. Mai, to whom we
owe its discovery. The substance is as follows: “The Goths
transported over the river their native deities, with their
priests of both sexes; but concerning their rites they maintained
a deep and ‘\textit{adamantine} silence.’ To the Romans they pretended
to be generally Christians, and placed certain persons to
represent bishops in a conspicuous manner on their wagons. There
was even among them a sort of what are called monks, persons whom
it was not difficult to mimic; it was enough to wear black
raiment, to be wicked, and held in respect.” (Eunapius hated the
“black-robed monks,” as appears in another passage, with the
cordial detestation of a heathen philosopher.) “Thus, while they
faithfully but secretly adhered to their own religion, the Romans
were weak enough to suppose them perfect Christians.” Mai, 277.
Eunapius in Niebuhr, 82.—M}

\pagenote[68]{Eunapius and Zosimus curiously specify these
articles of Gothic wealth and luxury. Yet it must be presumed,
that they were the manufactures of the provinces; which the
Barbarians had acquired as the spoils of war; or as the gifts, or
merchandise, of peace.}

An undisciplined and unsettled nation of Barbarians required the
firmest temper, and the most dexterous management. The daily
subsistence of near a million of extraordinary subjects could be
supplied only by constant and skilful diligence, and might
continually be interrupted by mistake or accident. The insolence,
or the indignation, of the Goths, if they conceived themselves to
be the objects either of fear or of contempt, might urge them to
the most desperate extremities; and the fortune of the state
seemed to depend on the prudence, as well as the integrity, of
the generals of Valens. At this important crisis, the military
government of Thrace was exercised by Lupicinus and Maximus, in
whose venal minds the slightest hope of private emolument
outweighed every consideration of public advantage; and whose
guilt was only alleviated by their incapacity of discerning the
pernicious effects of their rash and criminal administration.

Instead of obeying the orders of their sovereign, and satisfying,
with decent liberality, the demands of the Goths, they levied an
ungenerous and oppressive tax on the wants of the hungry
Barbarians. The vilest food was sold at an extravagant price;
and, in the room of wholesome and substantial provisions, the
markets were filled with the flesh of dogs, and of unclean
animals, who had died of disease. To obtain the valuable
acquisition of a pound of bread, the Goths resigned the
possession of an expensive, though serviceable, slave; and a
small quantity of meat was greedily purchased with ten pounds of
a precious, but useless metal,\textsuperscript{69} when their property was
exhausted, they continued this necessary traffic by the sale of
their sons and daughters; and notwithstanding the love of
freedom, which animated every Gothic breast, they submitted to
the humiliating maxim, that it was better for their children to
be maintained in a servile condition, than to perish in a state
of wretched and helpless independence. The most lively resentment
is excited by the tyranny of pretended benefactors, who sternly
exact the debt of gratitude which they have cancelled by
subsequent injuries: a spirit of discontent insensibly arose in
the camp of the Barbarians, who pleaded, without success, the
merit of their patient and dutiful behavior; and loudly
complained of the inhospitable treatment which they had received
from their new allies. They beheld around them the wealth and
plenty of a fertile province, in the midst of which they suffered
the intolerable hardships of artificial famine. But the means of
relief, and even of revenge, were in their hands; since the
rapaciousness of their tyrants had left to an injured people the
possession and the use of arms. The clamors of a multitude,
untaught to disguise their sentiments, announced the first
symptoms of resistance, and alarmed the timid and guilty minds of
Lupicinus and Maximus. Those crafty ministers, who substituted
the cunning of temporary expedients to the wise and salutary
counsels of general policy, attempted to remove the Goths from
their dangerous station on the frontiers of the empire; and to
disperse them, in separate quarters of cantonment, through the
interior provinces. As they were conscious how ill they had
deserved the respect, or confidence, of the Barbarians, they
diligently collected, from every side, a military force, that
might urge the tardy and reluctant march of a people, who had not
yet renounced the title, or the duties, of Roman subjects. But
the generals of Valens, while their attention was solely directed
to the discontented Visigoths, imprudently disarmed the ships and
the fortifications which constituted the defence of the Danube.
The fatal oversight was observed, and improved, by Alatheus and
Saphrax, who anxiously watched the favorable moment of escaping
from the pursuit of the Huns. By the help of such rafts and
vessels as could be hastily procured, the leaders of the
Ostrogoths transported, without opposition, their king and their
army; and boldly fixed a hostile and independent camp on the
territories of the empire.\textsuperscript{70}

\pagenote[69]{\textit{Decem libras;} the word \textit{silver} must be
understood. Jornandes betrays the passions and prejudices of a
Goth. The servile Geeks, Eunapius and Zosimus, disguise the Roman
oppression, and execrate the perfidy of the Barbarians. Ammianus,
a patriot historian, slightly, and reluctantly, touches on the
odious subject. Jerom, who wrote almost on the spot, is fair,
though concise. Per avaritaim aximi ducis, ad rebellionem fame
\textit{coacti} sunt, (in Chron.) * Note: A new passage from the history
of Eunapius is nearer to the truth. ‘It appeared to our
commanders a legitimate source of gain to be bribed by the
Barbarians: Edit. Niebuhr, p. 82.—M.}

\pagenote[70]{Ammianus, xxxi. 4, 5.}

Under the name of Judges, Alavivus and Fritigern were the leaders
of the Visigoths in peace and war; and the authority which they
derived from their birth was ratified by the free consent of the
nation. In a season of tranquility, their power might have been
equal, as well as their rank; but, as soon as their countrymen
were exasperated by hunger and oppression, the superior abilities
of Fritigern assumed the military command, which he was qualified
to exercise for the public welfare. He restrained the impatient
spirit of the Visigoths till the injuries and the insults of
their tyrants should justify their resistance in the opinion of
mankind: but he was not disposed to sacrifice any solid
advantages for the empty praise of justice and moderation.
Sensible of the benefits which would result from the union of the
Gothic powers under the same standard, he secretly cultivated the
friendship of the Ostrogoths; and while he professed an implicit
obedience to the orders of the Roman generals, he proceeded by
slow marches towards Marcianopolis, the capital of the Lower
Mæsia, about seventy miles from the banks of the Danube. On that
fatal spot, the flames of discord and mutual hatred burst forth
into a dreadful conflagration. Lupicinus had invited the Gothic
chiefs to a splendid entertainment; and their martial train
remained under arms at the entrance of the palace. But the gates
of the city were strictly guarded, and the Barbarians were
sternly excluded from the use of a plentiful market, to which
they asserted their equal claim of subjects and allies. Their
humble prayers were rejected with insolence and derision; and as
their patience was now exhausted, the townsmen, the soldiers, and
the Goths, were soon involved in a conflict of passionate
altercation and angry reproaches. A blow was imprudently given; a
sword was hastily drawn; and the first blood that was spilt in
this accidental quarrel, became the signal of a long and
destructive war. In the midst of noise and brutal intemperance,
Lupicinus was informed, by a secret messenger, that many of his
soldiers were slain, and despoiled of their arms; and as he was
already inflamed by wine, and oppressed by sleep he issued a rash
command, that their death should be revenged by the massacre of
the guards of Fritigern and Alavivus.

The clamorous shouts and dying groans apprised Fritigern of his
extreme danger; and, as he possessed the calm and intrepid spirit
of a hero, he saw that he was lost if he allowed a moment of
deliberation to the man who had so deeply injured him. “A
trifling dispute,” said the Gothic leader, with a firm but gentle
tone of voice, “appears to have arisen between the two nations;
but it may be productive of the most dangerous consequences,
unless the tumult is immediately pacified by the assurance of our
safety, and the authority of our presence.” At these words,
Fritigern and his companions drew their swords, opened their
passage through the unresisting crowd, which filled the palace,
the streets, and the gates, of Marcianopolis, and, mounting their
horses, hastily vanished from the eyes of the astonished Romans.
The generals of the Goths were saluted by the fierce and joyful
acclamations of the camp; war was instantly resolved, and the
resolution was executed without delay: the banners of the nation
were displayed according to the custom of their ancestors; and
the air resounded with the harsh and mournful music of the
Barbarian trumpet.\textsuperscript{71} The weak and guilty Lupicinus, who had
dared to provoke, who had neglected to destroy, and who still
presumed to despise, his formidable enemy, marched against the
Goths, at the head of such a military force as could be collected
on this sudden emergency. The Barbarians expected his approach
about nine miles from Marcianopolis; and on this occasion the
talents of the general were found to be of more prevailing
efficacy than the weapons and discipline of the troops. The valor
of the Goths was so ably directed by the genius of Fritigern,
that they broke, by a close and vigorous attack, the ranks of the
Roman legions. Lupicinus left his arms and standards, his
tribunes and his bravest soldiers, on the field of battle; and
their useless courage served only to protect the ignominious
flight of their leader. “That successful day put an end to the
distress of the Barbarians, and the security of the Romans: from
that day, the Goths, renouncing the precarious condition of
strangers and exiles, assumed the character of citizens and
masters, claimed an absolute dominion over the possessors of
land, and held, in their own right, the northern provinces of the
empire, which are bounded by the Danube.” Such are the words of
the Gothic historian,\textsuperscript{72} who celebrates, with rude eloquence, the
glory of his countrymen. But the dominion of the Barbarians was
exercised only for the purposes of rapine and destruction. As
they had been deprived, by the ministers of the emperor, of the
common benefits of nature, and the fair intercourse of social
life, they retaliated the injustice on the subjects of the
empire; and the crimes of Lupicinus were expiated by the ruin of
the peaceful husbandmen of Thrace, the conflagration of their
villages, and the massacre, or captivity, of their innocent
families. The report of the Gothic victory was soon diffused over
the adjacent country; and while it filled the minds of the Romans
with terror and dismay, their own hasty imprudence contributed to
increase the forces of Fritigern, and the calamities of the
province. Some time before the great emigration, a numerous body
of Goths, under the command of Suerid and Colias, had been
received into the protection and service of the empire.\textsuperscript{73} They
were encamped under the walls of Hadrianople; but the ministers
of Valens were anxious to remove them beyond the Hellespont, at a
distance from the dangerous temptation which might so easily be
communicated by the neighborhood, and the success, of their
countrymen. The respectful submission with which they yielded to
the order of their march, might be considered as a proof of their
fidelity; and their moderate request of a sufficient allowance of
provisions, and of a delay of only two days was expressed in the
most dutiful terms. But the first magistrate of Hadrianople,
incensed by some disorders which had been committed at his
country-house, refused this indulgence; and arming against them
the inhabitants and manufacturers of a populous city, he urged,
with hostile threats, their instant departure. The Barbarians
stood silent and amazed, till they were exasperated by the
insulting clamors, and missile weapons, of the populace: but when
patience or contempt was fatigued, they crushed the undisciplined
multitude, inflicted many a shameful wound on the backs of their
flying enemies, and despoiled them of the splendid armor,\textsuperscript{74}
which they were unworthy to bear. The resemblance of their
sufferings and their actions soon united this victorious
detachment to the nation of the Visigoths; the troops of Colias
and Suerid expected the approach of the great Fritigern, ranged
themselves under his standard, and signalized their ardor in the
siege of Hadrianople. But the resistance of the garrison informed
the Barbarians, that in the attack of regular fortifications, the
efforts of unskillful courage are seldom effectual. Their general
acknowledged his error, raised the siege, declared that “he was
at peace with stone walls,”\textsuperscript{75} and revenged his disappointment on
the adjacent country. He accepted, with pleasure, the useful
reenforcement of hardy workmen, who labored in the gold mines of
Thrace,\textsuperscript{76} for the emolument, and under the lash, of an unfeeling
master:\textsuperscript{77} and these new associates conducted the Barbarians,
through the secret paths, to the most sequestered places, which
had been chosen to secure the inhabitants, the cattle, and the
magazines of corn. With the assistance of such guides, nothing
could remain impervious or inaccessible; resistance was fatal;
flight was impracticable; and the patient submission of helpless
innocence seldom found mercy from the Barbarian conqueror. In the
course of these depredations, a great number of the children of
the Goths, who had been sold into captivity, were restored to the
embraces of their afflicted parents; but these tender interviews,
which might have revived and cherished in their minds some
sentiments of humanity, tended only to stimulate their native
fierceness by the desire of revenge. They listened, with eager
attention, to the complaints of their captive children, who had
suffered the most cruel indignities from the lustful or angry
passions of their masters, and the same cruelties, the same
indignities, were severely retaliated on the sons and daughters
of the Romans.\textsuperscript{78}

\pagenote[71]{Vexillis de \textit{more} sublatis, auditisque \textit{triste
sonantibus classicis}. Ammian. xxxi. 5. These are the \textit{rauca
cornua} of Claudian, (in Rufin. ii. 57,) the large horns of the
\textit{Uri}, or wild bull; such as have been more recently used by the
Swiss Cantons of Uri and Underwald. (Simler de Republicâ Helvet,
l. ii. p. 201, edit. Fuselin. Tigur 1734.) Their military horn is
finely, though perhaps casually, introduced in an original
narrative of the battle of Nancy, (A.D. 1477.) “Attendant le
combat le dit cor fut corné par trois fois, tant que le vent du
souffler pouvoit durer: ce qui esbahit fort Monsieur de
Bourgoigne; \textit{car deja à Morat l’avoit ouy}.” (See the Pièces
Justificatives in the 4to. edition of Philippe de Comines, tom.
iii. p. 493.)}

\pagenote[72]{Jornandes de Rebus Geticis, c. 26, p. 648, edit.
Grot. These \textit{splendidi panni} (they are comparatively such) are
undoubtedly transcribed from the larger histories of Priscus,
Ablavius, or Cassiodorus.}

\pagenote[73]{Cum populis suis longe ante suscepti. We are
ignorant of the precise date and circumstances of their
transmigration.}

\pagenote[74]{An Imperial manufacture of shields, \&c., was
established at Hadrianople; and the populace were headed by the
Fabricenses, or workmen. (Vales. ad Ammian. xxxi. 6.)}

\pagenote[75]{Pacem sibi esse cum parietibus memorans. Ammian.
xxxi. 7.}

\pagenote[76]{These mines were in the country of the Bessi, in
the ridge of mountains, the Rhodope, that runs between Philippi
and Philippopolis; two Macedonian cities, which derived their
name and origin from the father of Alexander. From the mines of
Thrace he annually received the value, not the weight, of a
thousand talents, (200,000l.,) a revenue which paid the phalanx,
and corrupted the orators of Greece. See Diodor. Siculus, tom.
ii. l. xvi. p. 88, edit. Wesseling. Godefroy’s Commentary on the
Theodosian Code, tom. iii. p. 496. Cellarius, Geograph. Antiq.
tom. i. p. 676, 857. D Anville, Geographie Ancienne, tom. i. p.
336.}

\pagenote[77]{As those unhappy workmen often ran away, Valens had
enacted severe laws to drag them from their hiding-places. Cod.
Theodosian, l. x. tit xix leg. 5, 7.}

\pagenote[78]{See Ammianus, xxxi. 5, 6. The historian of the
Gothic war loses time and space, by an unseasonable
recapitulation of the ancient inroads of the Barbarians.}

The imprudence of Valens and his ministers had introduced into
the heart of the empire a nation of enemies; but the Visigoths
might even yet have been reconciled, by the manly confession of
past errors, and the sincere performance of former engagements.
These healing and temperate measures seemed to concur with the
timorous disposition of the sovereign of the East: but, on this
occasion alone, Valens was brave; and his unseasonable bravery
was fatal to himself and to his subjects. He declared his
intention of marching from Antioch to Constantinople, to subdue
this dangerous rebellion; and, as he was not ignorant of the
difficulties of the enterprise, he solicited the assistance of
his nephew, the emperor Gratian, who commanded all the forces of
the West. The veteran troops were hastily recalled from the
defence of Armenia; that important frontier was abandoned to the
discretion of Sapor; and the immediate conduct of the Gothic war
was intrusted, during the absence of Valens, to his lieutenants
Trajan and Profuturus, two generals who indulged themselves in a
very false and favorable opinion of their own abilities. On their
arrival in Thrace, they were joined by Richomer, count of the
domestics; and the auxiliaries of the West, that marched under
his banner, were composed of the Gallic legions, reduced indeed,
by a spirit of desertion, to the vain appearances of strength and
numbers. In a council of war, which was influenced by pride,
rather than by reason, it was resolved to seek, and to encounter,
the Barbarians, who lay encamped in the spacious and fertile
meadows, near the most southern of the six mouths of the Danube.\textsuperscript{79}
Their camp was surrounded by the usual fortification of
wagons;\textsuperscript{80} and the Barbarians, secure within the vast circle of
the enclosure, enjoyed the fruits of their valor, and the spoils
of the province. In the midst of riotous intemperance, the
watchful Fritigern observed the motions, and penetrated the
designs, of the Romans. He perceived, that the numbers of the
enemy were continually increasing: and, as he understood their
intention of attacking his rear, as soon as the scarcity of
forage should oblige him to remove his camp, he recalled to their
standard his predatory detachments, which covered the adjacent
country. As soon as they descried the flaming beacons,\textsuperscript{81} they
obeyed, with incredible speed, the signal of their leader: the
camp was filled with the martial crowd of Barbarians; their
impatient clamors demanded the battle, and their tumultuous zeal
was approved and animated by the spirit of their chiefs. The
evening was already far advanced; and the two armies prepared
themselves for the approaching combat, which was deferred only
till the dawn of day.

While the trumpets sounded to arms, the undaunted courage of the
Goths was confirmed by the mutual obligation of a solemn oath;
and as they advanced to meet the enemy, the rude songs, which
celebrated the glory of their forefathers, were mingled with
their fierce and dissonant outcries, and opposed to the
artificial harmony of the Roman shout. Some military skill was
displayed by Fritigern to gain the advantage of a commanding
eminence; but the bloody conflict, which began and ended with the
light, was maintained on either side, by the personal and
obstinate efforts of strength, valor, and agility. The legions of
Armenia supported their fame in arms; but they were oppressed by
the irresistible weight of the hostile multitude the left wing of
the Romans was thrown into disorder and the field was strewed
with their mangled carcasses. This partial defeat was balanced,
however, by partial success; and when the two armies, at a late
hour of the evening, retreated to their respective camps, neither
of them could claim the honors, or the effects, of a decisive
victory. The real loss was more severely felt by the Romans, in
proportion to the smallness of their numbers; but the Goths were
so deeply confounded and dismayed by this vigorous, and perhaps
unexpected, resistance, that they remained seven days within the
circle of their fortifications. Such funeral rites, as the
circumstances of time and place would admit, were piously
discharged to some officers of distinguished rank; but the
indiscriminate vulgar was left unburied on the plain. Their flesh
was greedily devoured by the birds of prey, who in that age
enjoyed very frequent and delicious feasts; and several years
afterwards the white and naked bones, which covered the wide
extent of the fields, presented to the eyes of Ammianus a
dreadful monument of the battle of Salices.\textsuperscript{82}

79]{The Itinerary of Antoninus (p. 226, 227, edit.
Wesseling) marks the situation of this place about sixty miles
north of Tomi, Ovid’s exile; and the name of \textit{Salices} (the
willows) expresses the nature of the soil.}

\pagenote[80]{This circle of wagons, the \textit{Carrago}, was the usual
fortification of the Barbarians. (Vegetius de Re Militari, l.
iii. c. 10. Valesius ad Ammian. xxxi. 7.) The practice and the
name were preserved by their descendants as late as the fifteenth
century. The \textit{Charroy}, which surrounded the \textit{Ost}, is a word
familiar to the readers of Froissard, or Comines.}

\pagenote[81]{Statim ut accensi malleoli. I have used the literal
sense of real torches or beacons; but I almost suspect, that it
is only one of those turgid metaphors, those false ornaments,
that perpetually disfigure to style of Ammianus.}

\pagenote[82]{Indicant nunc usque albentes ossibus campi. Ammian.
xxxi. 7. The historian might have viewed these plains, either as
a soldier, or as a traveller. But his modesty has suppressed the
adventures of his own life subsequent to the Persian wars of
Constantius and Julian. We are ignorant of the time when he
quitted the service, and retired to Rome, where he appears to
have composed his History of his Own Times.}

The progress of the Goths had been checked by the doubtful event
of that bloody day; and the Imperial generals, whose army would
have been consumed by the repetition of such a contest, embraced
the more rational plan of destroying the Barbarians by the wants
and pressure of their own multitudes. They prepared to confine
the Visigoths in the narrow angle of land between the Danube, the
desert of Scythia, and the mountains of Hæmus, till their
strength and spirit should be insensibly wasted by the inevitable
operation of famine. The design was prosecuted with some conduct
and success: the Barbarians had almost exhausted their own
magazines, and the harvests of the country; and the diligence of
Saturninus, the master-general of the cavalry, was employed to
improve the strength, and to contract the extent, of the Roman
fortifications. His labors were interrupted by the alarming
intelligence, that new swarms of Barbarians had passed the
unguarded Danube, either to support the cause, or to imitate the
example, of Fritigern. The just apprehension, that he himself
might be surrounded, and overwhelmed, by the arms of hostile and
unknown nations, compelled Saturninus to relinquish the siege of
the Gothic camp; and the indignant Visigoths, breaking from their
confinement, satiated their hunger and revenge by the repeated
devastation of the fruitful country, which extends above three
hundred miles from the banks of the Danube to the straits of the
Hellespont.\textsuperscript{83} The sagacious Fritigern had successfully appealed
to the passions, as well as to the interest, of his Barbarian
allies; and the love of rapine, and the hatred of Rome, seconded,
or even prevented, the eloquence of his ambassadors. He cemented
a strict and useful alliance with the great body of his
countrymen, who obeyed Alatheus and Saphrax as the guardians of
their infant king: the long animosity of rival tribes was
suspended by the sense of their common interest; the independent
part of the nation was associated under one standard; and the
chiefs of the Ostrogoths appear to have yielded to the superior
genius of the general of the Visigoths. He obtained the
formidable aid of the Taifalæ,\textsuperscript{8311} whose military renown was
disgraced and polluted by the public infamy of their domestic
manners. Every youth, on his entrance into the world, was united
by the ties of honorable friendship, and brutal love, to some
warrior of the tribe; nor could he hope to be released from this
unnatural connection, till he had approved his manhood by
slaying, in single combat, a huge bear, or a wild boar of the
forest.\textsuperscript{84} But the most powerful auxiliaries of the Goths were
drawn from the camp of those enemies who had expelled them from
their native seats. The loose subordination, and extensive
possessions, of the Huns and the Alani, delayed the conquests,
and distracted the councils, of that victorious people. Several
of the hords were allured by the liberal promises of Fritigern;
and the rapid cavalry of Scythia added weight and energy to the
steady and strenuous efforts of the Gothic infantry. The
Sarmatians, who could never forgive the successor of Valentinian,
enjoyed and increased the general confusion; and a seasonable
irruption of the Alemanni, into the provinces of Gaul, engaged
the attention, and diverted the forces, of the emperor of the
West.\textsuperscript{85}

\pagenote[83]{Ammian. xxxi. 8.}

\pagenote[8311]{The Taifalæ, who at this period inhabited the
country which now forms the principality of Wallachia, were, in
my opinion, the last remains of the great and powerful nation of
the Dacians, (Daci or Dahæ.) which has given its name to these
regions, over which they had ruled so long. The Taifalæ passed
with the Goths into the territory of the empire. A great number
of them entered the Roman service, and were quartered in
different provinces. They are mentioned in the Notitia Imperii.
There was a considerable body in the country of the Pictavi, now
Poithou. They long retained their manners and language, and
caused the name of the Theofalgicus pagus to be given to the
district they inhabited. Two places in the department of La
Vendee, Tiffanges and La Tiffardière, still preserve evident
traces of this denomination. St. Martin, iv. 118.—M.}

\pagenote[84]{Hanc Taifalorum gentem turpem, et obscenæ vitæ
flagitiis ita accipimus mersam; ut apud eos nefandi concubitûs
fœdere copulentur mares puberes, ætatis viriditatem in eorum
pollutis usibus consumpturi. Porro, siqui jam adultus aprum
exceperit solus, vel interemit ursum immanem, colluvione
liberatur incesti. Ammian. xxxi. 9. ——Among the Greeks, likewise,
more especially among the Cretans, the holy bands of friendship
were confirmed, and sullied, by unnatural love.}

\pagenote[85]{Ammian. xxxi. 8, 9. Jerom (tom. i. p. 26)
enumerates the nations and marks a calamitous period of twenty
years. This epistle to Heliodorus was composed in the year 397,
(Tillemont, Mém. Eccles tom xii. p. 645.)}

\section{Part \thesection.}

One of the most dangerous inconveniences of the introduction of
the Barbarians into the army and the palace, was sensibly felt in
their correspondence with their hostile countrymen; to whom they
imprudently, or maliciously, revealed the weakness of the Roman
empire. A soldier, of the lifeguards of Gratian, was of the
nation of the Alemanni, and of the tribe of the Lentienses, who
dwelt beyond the Lake of Constance. Some domestic business
obliged him to request a leave of absence. In a short visit to
his family and friends, he was exposed to their curious
inquiries: and the vanity of the loquacious soldier tempted him
to display his intimate acquaintance with the secrets of the
state, and the designs of his master. The intelligence, that
Gratian was preparing to lead the military force of Gaul, and of
the West, to the assistance of his uncle Valens, pointed out to
the restless spirit of the Alemanni the moment, and the mode, of
a successful invasion. The enterprise of some light detachments,
who, in the month of February, passed the Rhine upon the ice, was
the prelude of a more important war. The boldest hopes of rapine,
perhaps of conquest, outweighed the considerations of timid
prudence, or national faith. Every forest, and every village,
poured forth a band of hardy adventurers; and the great army of
the Alemanni, which, on their approach, was estimated at forty
thousand men by the fears of the people, was afterwards magnified
to the number of seventy thousand by the vain and credulous
flattery of the Imperial court. The legions, which had been
ordered to march into Pannonia, were immediately recalled, or
detained, for the defence of Gaul; the military command was
divided between Nanienus and Mellobaudes; and the youthful
emperor, though he respected the long experience and sober wisdom
of the former, was much more inclined to admire, and to follow,
the martial ardor of his colleague; who was allowed to unite the
incompatible characters of count of the domestics, and of king of
the Franks. His rival Priarius, king of the Alemanni, was guided,
or rather impelled, by the same headstrong valor; and as their
troops were animated by the spirit of their leaders, they met,
they saw, they encountered each other, near the town of
Argentaria, or Colmar,\textsuperscript{86} in the plains of Alsace. The glory of
the day was justly ascribed to the missile weapons, and
well-practised evolutions, of the Roman soldiers; the Alemanni,
who long maintained their ground, were slaughtered with
unrelenting fury; five thousand only of the Barbarians escaped to
the woods and mountains; and the glorious death of their king on
the field of battle saved him from the reproaches of the people,
who are always disposed to accuse the justice, or policy, of an
unsuccessful war. After this signal victory, which secured the
peace of Gaul, and asserted the honor of the Roman arms, the
emperor Gratian appeared to proceed without delay on his Eastern
expedition; but as he approached the confines of the Alemanni, he
suddenly inclined to the left, surprised them by his unexpected
passage of the Rhine, and boldly advanced into the heart of their
country. The Barbarians opposed to his progress the obstacles of
nature and of courage; and still continued to retreat, from one
hill to another, till they were satisfied, by repeated trials, of
the power and perseverance of their enemies. Their submission was
accepted as a proof, not indeed of their sincere repentance, but
of their actual distress; and a select number of their brave and
robust youth was exacted from the faithless nation, as the most
substantial pledge of their future moderation. The subjects of
the empire, who had so often experienced that the Alemanni could
neither be subdued by arms, nor restrained by treaties, might not
promise themselves any solid or lasting tranquillity: but they
discovered, in the virtues of their young sovereign, the prospect
of a long and auspicious reign. When the legions climbed the
mountains, and scaled the fortifications of the Barbarians, the
valor of Gratian was distinguished in the foremost ranks; and the
gilt and variegated armor of his guards was pierced and shattered
by the blows which they had received in their constant attachment
to the person of their sovereign. At the age of nineteen, the son
of Valentinian seemed to possess the talents of peace and war;
and his personal success against the Alemanni was interpreted as
a sure presage of his Gothic triumphs.\textsuperscript{87}

\pagenote[86]{The field of battle, \textit{Argentaria} or
\textit{Argentovaria}, is accurately fixed by M. D’Anville (Notice de
l’Ancienne Gaule, p. 96—99) at twenty-three Gallic leagues, or
thirty-four and a half Roman miles to the south of Strasburg.
From its ruins the adjacent town of \textit{Colmar} has arisen. Note: It
is rather Horburg, on the right bank of the River Ill, opposite
to Colmar. From Schoepflin, Alsatia Illustrata. St. Martin, iv.
121.—M.}

\pagenote[87]{The full and impartial narrative of Ammianus (xxxi.
10) may derive some additional light from the Epitome of Victor,
the Chronicle of Jerom, and the History of Orosius, (l. vii. c.
33, p. 552, edit. Havercamp.)}

While Gratian deserved and enjoyed the applause of his subjects,
the emperor Valens, who, at length, had removed his court and
army from Antioch, was received by the people of Constantinople
as the author of the public calamity. Before he had reposed
himself ten days in the capital, he was urged by the licentious
clamors of the Hippodrome to march against the Barbarians, whom
he had invited into his dominions; and the citizens, who are
always brave at a distance from any real danger, declared, with
confidence, that, if they were supplied with arms, \textit{they} alone
would undertake to deliver the province from the ravages of an
insulting foe.\textsuperscript{88} The vain reproaches of an ignorant multitude
hastened the downfall of the Roman empire; they provoked the
desperate rashness of Valens; who did not find, either in his
reputation or in his mind, any motives to support with firmness
the public contempt. He was soon persuaded, by the successful
achievements of his lieutenants, to despise the power of the
Goths, who, by the diligence of Fritigern, were now collected in
the neighborhood of Hadrianople. The march of the Taifalæ had
been intercepted by the valiant Frigeric: the king of those
licentious Barbarians was slain in battle; and the suppliant
captives were sent into distant exile to cultivate the lands of
Italy, which were assigned for their settlement in the vacant
territories of Modena and Parma.\textsuperscript{89} The exploits of Sebastian,\textsuperscript{90}
who was recently engaged in the service of Valens, and promoted
to the rank of master-general of the infantry, were still more
honorable to himself, and useful to the republic. He obtained the
permission of selecting three hundred soldiers from each of the
legions; and this separate detachment soon acquired the spirit of
discipline, and the exercise of arms, which were almost forgotten
under the reign of Valens. By the vigor and conduct of Sebastian,
a large body of the Goths were surprised in their camp; and the
immense spoil, which was recovered from their hands, filled the
city of Hadrianople, and the adjacent plain. The splendid
narratives, which the general transmitted of his own exploits,
alarmed the Imperial court by the appearance of superior merit;
and though he cautiously insisted on the difficulties of the
Gothic war, his valor was praised, his advice was rejected; and
Valens, who listened with pride and pleasure to the flattering
suggestions of the eunuchs of the palace, was impatient to seize
the glory of an easy and assured conquest. His army was
strengthened by a numerous reenforcement of veterans; and his
march from Constantinople to Hadrianople was conducted with so
much military skill, that he prevented the activity of the
Barbarians, who designed to occupy the intermediate defiles, and
to intercept either the troops themselves, or their convoys of
provisions. The camp of Valens, which he pitched under the walls
of Hadrianople, was fortified, according to the practice of the
Romans, with a ditch and rampart; and a most important council
was summoned, to decide the fate of the emperor and of the
empire. The party of reason and of delay was strenuously
maintained by Victor, who had corrected, by the lessons of
experience, the native fierceness of the Sarmatian character;
while Sebastian, with the flexible and obsequious eloquence of a
courtier, represented every precaution, and every measure, that
implied a doubt of immediate victory, as unworthy of the courage
and majesty of their invincible monarch. The ruin of Valens was
precipitated by the deceitful arts of Fritigern, and the prudent
admonitions of the emperor of the West. The advantages of
negotiating in the midst of war were perfectly understood by the
general of the Barbarians; and a Christian ecclesiastic was
despatched, as the holy minister of peace, to penetrate, and to
perplex, the councils of the enemy. The misfortunes, as well as
the provocations, of the Gothic nation, were forcibly and truly
described by their ambassador; who protested, in the name of
Fritigern, that he was still disposed to lay down his arms, or to
employ them only in the defence of the empire; if he could secure
for his wandering countrymen a tranquil settlement on the waste
lands of Thrace, and a sufficient allowance of corn and cattle.
But he added, in a whisper of confidential friendship, that the
exasperated Barbarians were averse to these reasonable
conditions; and that Fritigern was doubtful whether he could
accomplish the conclusion of the treaty, unless he found himself
supported by the presence and terrors of an Imperial army. About
the same time, Count Richomer returned from the West to announce
the defeat and submission of the Alemanni, to inform Valens that
his nephew advanced by rapid marches at the head of the veteran
and victorious legions of Gaul, and to request, in the name of
Gratian and of the republic, that every dangerous and decisive
measure might be suspended, till the junction of the two emperors
should insure the success of the Gothic war. But the feeble
sovereign of the East was actuated only by the fatal illusions of
pride and jealousy. He disdained the importunate advice; he
rejected the humiliating aid; he secretly compared the
ignominious, at least the inglorious, period of his own reign,
with the fame of a beardless youth; and Valens rushed into the
field, to erect his imaginary trophy, before the diligence of his
colleague could usurp any share of the triumphs of the day.

\pagenote[88]{Moratus paucissimos dies, seditione popularium
levium pulsus Ammian. xxxi. 11. Socrates (l. iv. c. 38) supplies
the dates and some circumstances. * Note: Compare fragment of
Eunapius. Mai, 272, in Niebuhr, p. 77.—M}

\pagenote[89]{Vivosque omnes circa Mutinam, Regiumque, et Parmam,
Italica oppida, rura culturos exterminavit. Ammianus, xxxi. 9.
Those cities and districts, about ten years after the colony of
the Taifalæ, appear in a very desolate state. See Muratori,
Dissertazioni sopra le Antichità Italiane, tom. i. Dissertat.
xxi. p. 354.}

\pagenote[90]{Ammian. xxxi. 11. Zosimus, l. iv. p. 228—230. The
latter expatiates on the desultory exploits of Sebastian, and
despatches, in a few lines, the important battle of Hadrianople.
According to the ecclesiastical critics, who hate Sebastian, the
praise of Zosimus is disgrace, (Tillemont, Hist. des Empereurs,
tom. v. p. 121.) His prejudice and ignorance undoubtedly render
him a very questionable judge of merit.}

On the ninth of August, a day which has deserved to be marked
among the most inauspicious of the Roman Calendar,\textsuperscript{91} the emperor
Valens, leaving, under a strong guard, his baggage and military
treasure, marched from Hadrianople to attack the Goths, who were
encamped about twelve miles from the city.\textsuperscript{92} By some mistake of
the orders, or some ignorance of the ground, the right wing, or
column of cavalry arrived in sight of the enemy, whilst the left
was still at a considerable distance; the soldiers were
compelled, in the sultry heat of summer, to precipitate their
pace; and the line of battle was formed with tedious confusion
and irregular delay. The Gothic cavalry had been detached to
forage in the adjacent country; and Fritigern still continued to
practise his customary arts. He despatched messengers of peace,
made proposals, required hostages, and wasted the hours, till the
Romans, exposed without shelter to the burning rays of the sun,
were exhausted by thirst, hunger, and intolerable fatigue. The
emperor was persuaded to send an ambassador to the Gothic camp;
the zeal of Richomer, who alone had courage to accept the
dangerous commission, was applauded; and the count of the
domestics, adorned with the splendid ensigns of his dignity, had
proceeded some way in the space between the two armies, when he
was suddenly recalled by the alarm of battle. The hasty and
imprudent attack was made by Bacurius the Iberian, who commanded
a body of archers and targeteers; and as they advanced with
rashness, they retreated with loss and disgrace. In the same
moment, the flying squadrons of Alatheus and Saphrax, whose
return was anxiously expected by the general of the Goths,
descended like a whirlwind from the hills, swept across the
plain, and added new terrors to the tumultuous, but irresistible
charge of the Barbarian host. The event of the battle of
Hadrianople, so fatal to Valens and to the empire, may be
described in a few words: the Roman cavalry fled; the infantry
was abandoned, surrounded, and cut in pieces. The most skilful
evolutions, the firmest courage, are scarcely sufficient to
extricate a body of foot, encompassed, on an open plain, by
superior numbers of horse; but the troops of Valens, oppressed by
the weight of the enemy and their own fears, were crowded into a
narrow space, where it was impossible for them to extend their
ranks, or even to use, with effect, their swords and javelins. In
the midst of tumult, of slaughter, and of dismay, the emperor,
deserted by his guards and wounded, as it was supposed, with an
arrow, sought protection among the Lancearii and the Mattiarii,
who still maintained their ground with some appearance of order
and firmness. His faithful generals, Trajan and Victor, who
perceived his danger, loudly exclaimed that all was lost, unless
the person of the emperor could be saved. Some troops, animated
by their exhortation, advanced to his relief: they found only a
bloody spot, covered with a heap of broken arms and mangled
bodies, without being able to discover their unfortunate prince,
either among the living or the dead. Their search could not
indeed be successful, if there is any truth in the circumstances
with which some historians have related the death of the emperor.

By the care of his attendants, Valens was removed from the field
of battle to a neighboring cottage, where they attempted to dress
his wound, and to provide for his future safety. But this humble
retreat was instantly surrounded by the enemy: they tried to
force the door, they were provoked by a discharge of arrows from
the roof, till at length, impatient of delay, they set fire to a
pile of dry magots, and consumed the cottage with the Roman
emperor and his train. Valens perished in the flames; and a
youth, who dropped from the window, alone escaped, to attest the
melancholy tale, and to inform the Goths of the inestimable prize
which they had lost by their own rashness. A great number of
brave and distinguished officers perished in the battle of
Hadrianople, which equalled in the actual loss, and far surpassed
in the fatal consequences, the misfortune which Rome had formerly
sustained in the fields of Cannæ.\textsuperscript{93} Two master-generals of the
cavalry and infantry, two great officers of the palace, and
thirty-five tribunes, were found among the slain; and the death
of Sebastian might satisfy the world, that he was the victim, as
well as the author, of the public calamity. Above two thirds of
the Roman army were destroyed: and the darkness of the night was
esteemed a very favorable circumstance, as it served to conceal
the flight of the multitude, and to protect the more orderly
retreat of Victor and Richomer, who alone, amidst the general
consternation, maintained the advantage of calm courage and
regular discipline.\textsuperscript{94}

\pagenote[91]{Ammianus (xxxi. 12, 13) almost alone describes the
councils and actions which were terminated by the fatal battle of
Hadrianople. We might censure the vices of his style, the
disorder and perplexity of his narrative: but we must now take
leave of this impartial historian; and reproach is silenced by
our regret for such an irreparable loss.}

\pagenote[92]{The difference of the eight miles of Ammianus, and
the twelve of Idatius, can only embarrass those critics (Valesius
ad loc.,) who suppose a great army to be a mathematical point,
without space or dimensions.}

\pagenote[93]{Nec ulla annalibus, præter Cannensem pugnam, ita ad
internecionem res legitur gesta. Ammian. xxxi. 13. According to
the grave Polybius, no more than 370 horse, and 3,000 foot,
escaped from the field of Cannæ: 10,000 were made prisoners; and
the number of the slain amounted to 5,630 horse, and 70,000 foot,
(Polyb. l. iii. p 371, edit. Casaubon, 8vo.) Livy (xxii. 49) is
somewhat less bloody: he slaughters only 2,700 horse, and 40,000
foot. The Roman army was supposed to consist of 87,200 effective
men, (xxii. 36.)}

\pagenote[94]{We have gained some faint light from Jerom, (tom.
i. p. 26 and in Chron. p. 188,) Victor, (in Epitome,) Orosius,
(l. vii. c. 33, p. 554,) Jornandes, (c. 27,) Zosimus, (l. iv. p.
230,) Socrates, (l. iv. c. 38,) Sozomen, (l. vi. c. 40,) Idatius,
(in Chron.) But their united evidence, if weighed against
Ammianus alone, is light and unsubstantial.}

While the impressions of grief and terror were still recent in
the minds of men, the most celebrated rhetorician of the age
composed the funeral oration of a vanquished army, and of an
unpopular prince, whose throne was already occupied by a
stranger. “There are not wanting,” says the candid Libanius,
“those who arraign the prudence of the emperor, or who impute the
public misfortune to the want of courage and discipline in the
troops. For my own part, I reverence the memory of their former
exploits: I reverence the glorious death, which they bravely
received, standing, and fighting in their ranks: I reverence the
field of battle, stained with \textit{their} blood, and the blood of the
Barbarians. Those honorable marks have been already washed away
by the rains; but the lofty monuments of their bones, the bones
of generals, of centurions, and of valiant warriors, claim a
longer period of duration. The king himself fought and fell in
the foremost ranks of the battle. His attendants presented him
with the fleetest horses of the Imperial stable, that would soon
have carried him beyond the pursuit of the enemy. They vainly
pressed him to reserve his important life for the future service
of the republic. He still declared that he was unworthy to
survive so many of the bravest and most faithful of his subjects;
and the monarch was nobly buried under a mountain of the slain.
Let none, therefore, presume to ascribe the victory of the
Barbarians to the fear, the weakness, or the imprudence, of the
Roman troops. The chiefs and the soldiers were animated by the
virtue of their ancestors, whom they equalled in discipline and
the arts of war. Their generous emulation was supported by the
love of glory, which prompted them to contend at the same time
with heat and thirst, with fire and the sword; and cheerfully to
embrace an honorable death, as their refuge against flight and
infamy. The indignation of the gods has been the only cause of
the success of our enemies.” The truth of history may disclaim
some parts of this panegyric, which cannot strictly be reconciled
with the character of Valens, or the circumstances of the battle:
but the fairest commendation is due to the eloquence, and still
more to the generosity, of the sophist of Antioch.\textsuperscript{95}

\pagenote[95]{Libanius de ulciscend. Julian. nece, c. 3, in
Fabricius, Bibliot Græc. tom. vii. p. 146—148.}

The pride of the Goths was elated by this memorable victory; but
their avarice was disappointed by the mortifying discovery, that
the richest part of the Imperial spoil had been within the walls
of Hadrianople. They hastened to possess the reward of their
valor; but they were encountered by the remains of a vanquished
army, with an intrepid resolution, which was the effect of their
despair, and the only hope of their safety. The walls of the
city, and the ramparts of the adjacent camp, were lined with
military engines, that threw stones of an enormous weight; and
astonished the ignorant Barbarians by the noise, and velocity,
still more than by the real effects, of the discharge. The
soldiers, the citizens, the provincials, the domestics of the
palace, were united in the danger, and in the defence: the
furious assault of the Goths was repulsed; their secret arts of
treachery and treason were discovered; and, after an obstinate
conflict of many hours, they retired to their tents; convinced,
by experience, that it would be far more advisable to observe the
treaty, which their sagacious leader had tacitly stipulated with
the fortifications of great and populous cities. After the hasty
and impolitic massacre of three hundred deserters, an act of
justice extremely useful to the discipline of the Roman armies,
the Goths indignantly raised the siege of Hadrianople. The scene
of war and tumult was instantly converted into a silent solitude:
the multitude suddenly disappeared; the secret paths of the woods
and mountains were marked with the footsteps of the trembling
fugitives, who sought a refuge in the distant cities of Illyricum
and Macedonia; and the faithful officers of the household, and
the treasury, cautiously proceeded in search of the emperor, of
whose death they were still ignorant. The tide of the Gothic
inundation rolled from the walls of Hadrianople to the suburbs of
Constantinople. The Barbarians were surprised with the splendid
appearance of the capital of the East, the height and extent of
the walls, the myriads of wealthy and affrighted citizens who
crowded the ramparts, and the various prospect of the sea and
land. While they gazed with hopeless desire on the inaccessible
beauties of Constantinople, a sally was made from one of the
gates by a party of Saracens,\textsuperscript{96} who had been fortunately engaged
in the service of Valens. The cavalry of Scythia was forced to
yield to the admirable swiftness and spirit of the Arabian
horses: their riders were skilled in the evolutions of irregular
war; and the Northern Barbarians were astonished and dismayed, by
the inhuman ferocity of the Barbarians of the South.

A Gothic soldier was slain by the dagger of an Arab; and the
hairy, naked savage, applying his lips to the wound, expressed a
horrid delight, while he sucked the blood of his vanquished
enemy.\textsuperscript{97} The army of the Goths, laden with the spoils of the
wealthy suburbs and the adjacent territory, slowly moved, from
the Bosphorus, to the mountains which form the western boundary
of Thrace. The important pass of Succi was betrayed by the fear,
or the misconduct, of Maurus; and the Barbarians, who no longer
had any resistance to apprehend from the scattered and vanquished
troops of the East, spread themselves over the face of a fertile
and cultivated country, as far as the confines of Italy and the
Hadriatic Sea.\textsuperscript{98}

\pagenote[96]{Valens had gained, or rather purchased, the
friendship of the Saracens, whose vexatious inroads were felt on
the borders of Phœnicia, Palestine, and Egypt. The Christian
faith had been lately introduced among a people, reserved, in a
future age, to propagate another religion, (Tillemont, Hist. des
Empereurs, tom. v. p. 104, 106, 141. Mém. Eccles. tom. vii. p.
593.)}

\pagenote[97]{Crinitus quidam, nudus omnia præter pubem,
subraunum et ugubre strepens. Ammian. xxxi. 16, and Vales. ad
loc. The Arabs often fought naked; a custom which may be ascribed
to their sultry climate, and ostentatious bravery. The
description of this unknown savage is the lively portrait of
Derar, a name so dreadful to the Christians of Syria. See
Ockley’s Hist. of the Saracens, vol. i. p. 72, 84, 87.}

\pagenote[98]{The series of events may still be traced in the
last pages of Ammianus, (xxxi. 15, 16.) Zosimus, (l. iv. p. 227,
231,) whom we are now reduced to cherish, misplaces the sally of
the Arabs before the death of Valens. Eunapius (in Excerpt.
Legat. p. 20) praises the fertility of Thrace, Macedonia, \&c.}

The Romans, who so coolly, and so concisely, mention the acts of
\textit{justice} which were exercised by the legions,\textsuperscript{99} reserve their
compassion, and their eloquence, for their own sufferings, when
the provinces were invaded, and desolated, by the arms of the
successful Barbarians. The simple circumstantial narrative (did
such a narrative exist) of the ruin of a single town, of the
misfortunes of a single family,\textsuperscript{100} might exhibit an interesting
and instructive picture of human manners: but the tedious
repetition of vague and declamatory complaints would fatigue the
attention of the most patient reader. The same censure may be
applied, though not perhaps in an equal degree, to the profane,
and the ecclesiastical, writers of this unhappy period; that
their minds were inflamed by popular and religious animosity; and
that the true size and color of every object is falsified by the
exaggerations of their corrupt eloquence. The vehement Jerom\textsuperscript{101}
might justly deplore the calamities inflicted by the Goths, and
their barbarous allies, on his native country of Pannonia, and
the wide extent of the provinces, from the walls of
Constantinople to the foot of the Julian Alps; the rapes, the
massacres, the conflagrations; and, above all, the profanation of
the churches, that were turned into stables, and the contemptuous
treatment of the relics of holy martyrs. But the Saint is surely
transported beyond the limits of nature and history, when he
affirms, “that, in those desert countries, nothing was left
except the sky and the earth; that, after the destruction of the
cities, and the extirpation of the human race, the land was
overgrown with thick forests and inextricable brambles; and that
the universal desolation, announced by the prophet Zephaniah, was
accomplished, in the scarcity of the beasts, the birds, and even
of the fish.” These complaints were pronounced about twenty years
after the death of Valens; and the Illyrian provinces, which were
constantly exposed to the invasion and passage of the Barbarians,
still continued, after a calamitous period of ten centuries, to
supply new materials for rapine and destruction. Could it even be
supposed, that a large tract of country had been left without
cultivation and without inhabitants, the consequences might not
have been so fatal to the inferior productions of animated
nature. The useful and feeble animals, which are nourished by the
hand of man, might suffer and perish, if they were deprived of
his protection; but the beasts of the forest, his enemies or his
victims, would multiply in the free and undisturbed possession of
their solitary domain. The various tribes that people the air, or
the waters, are still less connected with the fate of the human
species; and it is highly probable that the fish of the Danube
would have felt more terror and distress, from the approach of a
voracious pike, than from the hostile inroad of a Gothic army.

\pagenote[99]{Observe with how much indifference Cæsar relates,
in the Commentaries of the Gallic war, \textit{that} he put to death the
whole senate of the Veneti, who had yielded to his mercy, (iii.
16;) \textit{that} he labored to extirpate the whole nation of the
Eburones, (vi. 31;) \textit{that} forty thousand persons were massacred
at Bourges by the just revenge of his soldiers, who spared
neither age nor sex, (vii. 27,) \&c.}

\pagenote[100]{Such are the accounts of the sack of Magdeburgh,
by the ecclesiastic and the fisherman, which Mr. Harte has
transcribed, (Hist. of Gustavus Adolphus, vol. i. p. 313—320,)
with some apprehension of violating the \textit{dignity} of history.}

\pagenote[101]{Et vastatis urbibus, hominibusque interfectis,
solitudinem et \textit{raritatem bestiarum} quoque fieri, \textit{et
volatilium, pisciumque:} testis Illyricum est, testis Thracia,
testis in quo ortus sum solum, (Pannonia;) ubi præter cœlum et
terram, et crescentes vepres, et condensa sylvarum \textit{cuncta
perierunt}. Tom. vii. p. 250, l, Cap. Sophonias and tom. i. p.
26.}

\section{Part \thesection.}

Whatever may have been the just measure of the calamities of
Europe, there was reason to fear that the same calamities would
soon extend to the peaceful countries of Asia. The sons of the
Goths had been judiciously distributed through the cities of the
East; and the arts of education were employed to polish, and
subdue, the native fierceness of their temper. In the space of
about twelve years, their numbers had continually increased; and
the children, who, in the first emigration, were sent over the
Hellespont, had attained, with rapid growth, the strength and
spirit of perfect manhood.\textsuperscript{102} It was impossible to conceal from
their knowledge the events of the Gothic war; and, as those
daring youths had not studied the language of dissimulation, they
betrayed their wish, their desire, perhaps their intention, to
emulate the glorious example of their fathers. The danger of the
times seemed to justify the jealous suspicions of the
provincials; and these suspicions were admitted as unquestionable
evidence, that the Goths of Asia had formed a secret and
dangerous conspiracy against the public safety. The death of
Valens had left the East without a sovereign; and Julius, who
filled the important station of master-general of the troops,
with a high reputation of diligence and ability, thought it his
duty to consult the senate of Constantinople; which he
considered, during the vacancy of the throne, as the
representative council of the nation. As soon as he had obtained
the discretionary power of acting as he should judge most
expedient for the good of the republic, he assembled the
principal officers, and privately concerted effectual measures
for the execution of his bloody design. An order was immediately
promulgated, that, on a stated day, the Gothic youth should
assemble in the capital cities of their respective provinces;
and, as a report was industriously circulated, that they were
summoned to receive a liberal gift of lands and money, the
pleasing hope allayed the fury of their resentment, and, perhaps,
suspended the motions of the conspiracy. On the appointed day,
the unarmed crowd of the Gothic youth was carefully collected in
the square or Forum; the streets and avenues were occupied by the
Roman troops, and the roofs of the houses were covered with
archers and slingers. At the same hour, in all the cities of the
East, the signal was given of indiscriminate slaughter; and the
provinces of Asia were delivered by the cruel prudence of Julius,
from a domestic enemy, who, in a few months, might have carried
fire and sword from the Hellespont to the Euphrates.\textsuperscript{103} The
urgent consideration of the public safety may undoubtedly
authorize the violation of every positive law. How far that, or
any other, consideration may operate to dissolve the natural
obligations of humanity and justice, is a doctrine of which I
still desire to remain ignorant.

\pagenote[102]{Eunapius (in Excerpt. Legat. p. 20) foolishly
supposes a præternatural growth of the young Goths, that he may
introduce Cadmus’s armed men, who sprang from the dragon’s teeth,
\&c. Such was the Greek eloquence of the times.}

\pagenote[103]{Ammianus evidently approves this execution,
efficacia velox et salutaris, which concludes his work, (xxxi.
16.) Zosimus, who is curious and copious, (l. iv. p. 233—236,)
mistakes the date, and labors to find the reason, why Julius did
not consult the emperor Theodosius who had not yet ascended the
throne of the East.}

The emperor Gratian was far advanced on his march towards the
plains of Hadrianople, when he was informed, at first by the
confused voice of fame, and afterwards by the more accurate
reports of Victor and Richomer, that his impatient colleague had
been slain in battle, and that two thirds of the Roman army were
exterminated by the sword of the victorious Goths. Whatever
resentment the rash and jealous vanity of his uncle might
deserve, the resentment of a generous mind is easily subdued by
the softer emotions of grief and compassion; and even the sense
of pity was soon lost in the serious and alarming consideration
of the state of the republic. Gratian was too late to assist, he
was too weak to revenge, his unfortunate colleague; and the
valiant and modest youth felt himself unequal to the support of a
sinking world. A formidable tempest of the Barbarians of Germany
seemed ready to burst over the provinces of Gaul; and the mind of
Gratian was oppressed and distracted by the administration of the
Western empire. In this important crisis, the government of the
East, and the conduct of the Gothic war, required the undivided
attention of a hero and a statesman. A subject invested with such
ample command would not long have preserved his fidelity to a
distant benefactor; and the Imperial council embraced the wise
and manly resolution of conferring an obligation, rather than of
yielding to an insult. It was the wish of Gratian to bestow the
purple as the reward of virtue; but, at the age of nineteen, it
is not easy for a prince, educated in the supreme rank, to
understand the true characters of his ministers and generals. He
attempted to weigh, with an impartial hand, their various merits
and defects; and, whilst he checked the rash confidence of
ambition, he distrusted the cautious wisdom which despaired of
the republic. As each moment of delay diminished something of the
power and resources of the future sovereign of the East, the
situation of the times would not allow a tedious debate. The
choice of Gratian was soon declared in favor of an exile, whose
father, only three years before, had suffered, under the sanction
of \textit{his} authority, an unjust and ignominious death. The great
Theodosius, a name celebrated in history, and dear to the
Catholic church,\textsuperscript{104} was summoned to the Imperial court, which
had gradually retreated from the confines of Thrace to the more
secure station of Sirmium. Five months after the death of Valens,
the emperor Gratian produced before the assembled troops \textit{his}
colleague and \textit{their} master; who, after a modest, perhaps a
sincere, resistance, was compelled to accept, amidst the general
acclamations, the diadem, the purple, and the equal title of
Augustus.\textsuperscript{105} The provinces of Thrace, Asia, and Egypt, over
which Valens had reigned, were resigned to the administration of
the new emperor; but, as he was specially intrusted with the
conduct of the Gothic war, the Illyrian præfecture was
dismembered; and the two great dioceses of Dacia and Macedonia
were added to the dominions of the Eastern empire.\textsuperscript{106}

\pagenote[104]{A life of Theodosius the Great was composed in the
last century, (Paris, 1679, in 4to-1680, 12mo.,) to inflame the
mind of the young Dauphin with Catholic zeal. The author,
Flechier, afterwards bishop of Nismes, was a celebrated preacher;
and his history is adorned, or tainted, with pulpit eloquence;
but he takes his learning from Baronius, and his principles from
St. Ambrose and St Augustin.}

\pagenote[105]{The birth, character, and elevation of Theodosius
are marked in Pacatus, (in Panegyr. Vet. xii. 10, 11, 12,)
Themistius, (Orat. xiv. p. 182,) (Zosimus, l. iv. p. 231,)
Augustin. (de Civitat. Dei. v. 25,) Orosius, (l. vii. c. 34,)
Sozomen, (l. vii. c. 2,) Socrates, (l. v. c. 2,) Theodoret, (l.
v. c. 5,) Philostorgius, (l. ix. c. 17, with Godefroy, p. 393,)
the Epitome of Victor, and the Chronicles of Prosper, Idatius,
and Marcellinus, in the Thesaurus Temporum of Scaliger. * Note:
Add a hostile fragment of Eunapius. Mai, p. 273, in Niebuhr, p
178—M.}

\pagenote[106]{Tillemont, Hist. des Empereurs, tom. v. p. 716,
\&c.}

The same province, and perhaps the same city,\textsuperscript{107} which had given
to the throne the virtues of Trajan, and the talents of Hadrian,
was the orignal seat of another family of Spaniards, who, in a
less fortunate age, possessed, near fourscore years, the
declining empire of Rome.\textsuperscript{108} They emerged from the obscurity of
municipal honors by the active spirit of the elder Theodosius, a
general whose exploits in Britain and Africa have formed one of
the most splendid parts of the annals of Valentinian. The son of
that general, who likewise bore the name of Theodosius, was
educated, by skilful preceptors, in the liberal studies of youth;
but he was instructed in the art of war by the tender care and
severe discipline of his father.\textsuperscript{109} Under the standard of such a
leader, young Theodosius sought glory and knowledge, in the most
distant scenes of military action; inured his constitution to the
difference of seasons and climates; distinguished his valor by
sea and land; and observed the various warfare of the Scots, the
Saxons, and the Moors. His own merit, and the recommendation of
the conqueror of Africa, soon raised him to a separate command;
and, in the station of Duke of Mæsia, he vanquished an army of
Sarmatians; saved the province; deserved the love of the
soldiers; and provoked the envy of the court.\textsuperscript{110} His rising
fortunes were soon blasted by the disgrace and execution of his
illustrious father; and Theodosius obtained, as a favor, the
permission of retiring to a private life in his native province
of Spain. He displayed a firm and temperate character in the ease
with which he adapted himself to this new situation. His time was
almost equally divided between the town and country; the spirit,
which had animated his public conduct, was shown in the active
and affectionate performance of every social duty; and the
diligence of the soldier was profitably converted to the
improvement of his ample patrimony,\textsuperscript{111} which lay between
Valladolid and Segovia, in the midst of a fruitful district,
still famous for a most exquisite breed of sheep.\textsuperscript{112} From the
innocent, but humble labors of his farm, Theodosius was
transported, in less than four months, to the throne of the
Eastern empire; and the whole period of the history of the world
will not perhaps afford a similar example, of an elevation at the
same time so pure and so honorable. The princes who peaceably
inherit the sceptre of their fathers, claim and enjoy a legal
right, the more secure as it is absolutely distinct from the
merits of their personal characters. The subjects, who, in a
monarchy, or a popular state, acquire the possession of supreme
power, may have raised themselves, by the superiority either of
genius or virtue, above the heads of their equals; but their
virtue is seldom exempt from ambition; and the cause of the
successful candidate is frequently stained by the guilt of
conspiracy, or civil war. Even in those governments which allow
the reigning monarch to declare a colleague or a successor, his
partial choice, which may be influenced by the blindest passions,
is often directed to an unworthy object But the most suspicious
malignity cannot ascribe to Theodosius, in his obscure solitude
of Caucha, the arts, the desires, or even the hopes, of an
ambitious statesman; and the name of the Exile would long since
have been forgotten, if his genuine and distinguished virtues had
not left a deep impression in the Imperial court. During the
season of prosperity, he had been neglected; but, in the public
distress, his superior merit was universally felt and
acknowledged. What confidence must have been reposed in his
integrity, since Gratian could trust, that a pious son would
forgive, for the sake of the republic, the murder of his father!
What expectations must have been formed of his abilities to
encourage the hope, that a single man could save, and restore,
the empire of the East! Theodosius was invested with the purple
in the thirty-third year of his age. The vulgar gazed with
admiration on the manly beauty of his face, and the graceful
majesty of his person, which they were pleased to compare with
the pictures and medals of the emperor Trajan; whilst intelligent
observers discovered, in the qualities of his heart and
understanding, a more important resemblance to the best and
greatest of the Roman princes.

\pagenote[107]{\textit{Italica}, founded by Scipio Africanus for his
wounded veterans of \textit{Italy}. The ruins still appear, about a
league above Seville, but on the opposite bank of the river. See
the Hispania Illustrata of Nonius, a short though valuable
treatise, c. xvii. p. 64—67.}

\pagenote[108]{I agree with Tillemont (Hist. des Empereurs, tom.
v. p. 726) in suspecting the royal pedigree, which remained a
secret till the promotion of Theodosius. Even after that event,
the silence of Pacatus outweighs the venal evidence of
Themistius, Victor, and Claudian, who connect the family of
Theodosius with the blood of Trajan and Hadrian.}

\pagenote[109]{Pacatas compares, and consequently prefers, the
youth of Theodosius to the military education of Alexander,
Hannibal, and the second Africanus; who, like him, had served
under their fathers, (xii. 8.)}

\pagenote[110]{Ammianus (xxix. 6) mentions this victory of
Theodosius Junior Dux Mæsiæ, prima etiam tum lanugine juvenis,
princeps postea perspectissimus. The same fact is attested by
Themistius and Zosimus but Theodoret, (l. v. c. 5,) who adds some
curious circumstances, strangely applies it to the time of the
interregnum.}

\pagenote[111]{Pacatus (in Panegyr. Vet. xii. 9) prefers the
rustic life of Theodosius to that of Cincinnatus; the one was the
effect of choice, the other of poverty.}

\pagenote[112]{M. D’Anville (Geographie Ancienne, tom. i. p. 25)
has fixed the situation of Caucha, or Coca, in the old province
of Gallicia, where Zosimus and Idatius have placed the birth, or
patrimony, of Theodosius.}

It is not without the most sincere regret, that I must now take
leave of an accurate and faithful guide, who has composed the
history of his own times, without indulging the prejudices and
passions, which usually affect the mind of a contemporary.
Ammianus Marcellinus, who terminates his useful work with the
defeat and death of Valens, recommends the more glorious subject
of the ensuing reign to the youthful vigor and eloquence of the
rising generation.\textsuperscript{113} The rising generation was not disposed to
accept his advice or to imitate his example;\textsuperscript{114} and, in the
study of the reign of Theodosius, we are reduced to illustrate
the partial narrative of Zosimus, by the obscure hints of
fragments and chronicles, by the figurative style of poetry or
panegyric, and by the precarious assistance of the ecclesiastical
writers, who, in the heat of religious faction, are apt to
despise the profane virtues of sincerity and moderation.
Conscious of these disadvantages, which will continue to involve
a considerable portion of the decline and fall of the Roman
empire, I shall proceed with doubtful and timorous steps. Yet I
may boldly pronounce, that the battle of Hadrianople was never
revenged by any signal or decisive victory of Theodosius over the
Barbarians: and the expressive silence of his venal orators may
be confirmed by the observation of the condition and
circumstances of the times. The fabric of a mighty state, which
has been reared by the labors of successive ages, could not be
overturned by the misfortune of a single day, if the fatal power
of the imagination did not exaggerate the real measure of the
calamity. The loss of forty thousand Romans, who fell in the
plains of Hadrianople, might have been soon recruited in the
populous provinces of the East, which contained so many millions
of inhabitants. The courage of a soldier is found to be the
cheapest, and most common, quality of human nature; and
sufficient skill to encounter an undisciplined foe might have
been speedily taught by the care of the surviving centurions. If
the Barbarians were mounted on the horses, and equipped with the
armor, of their vanquished enemies, the numerous studs of
Cappadocia and Spain would have supplied new squadrons of
cavalry; the thirty-four arsenals of the empire were plentifully
stored with magazines of offensive and defensive arms: and the
wealth of Asia might still have yielded an ample fund for the
expenses of the war. But the effects which were produced by the
battle of Hadrianople on the minds of the Barbarians and of the
Romans, extended the victory of the former, and the defeat of the
latter, far beyond the limits of a single day. A Gothic chief was
heard to declare, with insolent moderation, that, for his own
part, he was fatigued with slaughter: but that he was astonished
how a people, who fled before him like a flock of sheep, could
still presume to dispute the possession of their treasures and
provinces.\textsuperscript{115} The same terrors which the name of the Huns had
spread among the Gothic tribes, were inspired, by the formidable
name of the Goths, among the subjects and soldiers of the Roman
empire.\textsuperscript{116} If Theodosius, hastily collecting his scattered
forces, had led them into the field to encounter a victorious
enemy, his army would have been vanquished by their own fears;
and his rashness could not have been excused by the chance of
success. But the \textit{great} Theodosius, an epithet which he
honorably deserved on this momentous occasion, conducted himself
as the firm and faithful guardian of the republic. He fixed his
head-quarters at Thessalonica, the capital of the Macedonian
diocese;\textsuperscript{117} from whence he could watch the irregular motions of
the Barbarians, and direct the operations of his lieutenants,
from the gates of Constantinople to the shores of the Hadriatic.
The fortifications and garrisons of the cities were strengthened;
and the troops, among whom a sense of order and discipline was
revived, were insensibly emboldened by the confidence of their
own safety. From these secure stations, they were encouraged to
make frequent sallies on the Barbarians, who infested the
adjacent country; and, as they were seldom allowed to engage,
without some decisive superiority, either of ground or of
numbers, their enterprises were, for the most part, successful;
and they were soon convinced, by their own experience, of the
possibility of vanquishing their \textit{invincible} enemies. The
detachments of these separate garrisons were generally united
into small armies; the same cautious measures were pursued,
according to an extensive and well-concerted plan of operations;
the events of each day added strength and spirit to the Roman
arms; and the artful diligence of the emperor, who circulated the
most favorable reports of the success of the war, contributed to
subdue the pride of the Barbarians, and to animate the hopes and
courage of his subjects. If, instead of this faint and imperfect
outline, we could accurately represent the counsels and actions
of Theodosius, in four successive campaigns, there is reason to
believe, that his consummate skill would deserve the applause of
every military reader. The republic had formerly been saved by
the delays of Fabius; and, while the splendid trophies of Scipio,
in the field of Zama, attract the eyes of posterity, the camps
and marches of the dictator among the hills of the Campania, may
claim a juster proportion of the solid and independent fame,
which the general is not compelled to share, either with fortune
or with his troops. Such was likewise the merit of Theodosius;
and the infirmities of his body, which most unseasonably
languished under a long and dangerous disease, could not oppress
the vigor of his mind, or divert his attention from the public
service.\textsuperscript{118}

\pagenote[113]{Let us hear Ammianus himself. Hæc, ut miles
quondam et Græcus, a principatu Cæsaris Nervæ exorsus, adusque
Valentis inter, pro virium explicavi mensurâ: opus veritatem
professum nun quam, ut arbitror, sciens, silentio ausus
corrumpere vel mendacio. Scribant reliqua potiores ætate,
doctrinisque florentes. Quos id, si libuerit, aggressuros,
procudere linguas ad majores moneo stilos. Ammian. xxxi. 16. The
first thirteen books, a superficial epitome of two hundred and
fifty-seven years, are now lost: the last eighteen, which contain
no more than twenty-five years, still preserve the copious and
authentic history of his own times.}

\pagenote[114]{Ammianus was the last subject of Rome who composed
a profane history in the Latin language. The East, in the next
century, produced some rhetorical historians, Zosimus,
Olympiedorus, Malchus, Candidus \&c. See Vossius de Historicis
Græcis, l. ii. c. 18, de Historicis Latinis l. ii. c. 10, \&c.}

\pagenote[115]{Chrysostom, tom. i. p. 344, edit. Montfaucon. I
have verified and examined this passage: but I should never,
without the aid of Tillemont, (Hist. des Emp. tom. v. p. 152,)
have detected an historical anecdote, in a strange medley of
moral and mystic exhortations, addressed, by the preacher of
Antioch, to a young widow.}

\pagenote[116]{Eunapius, in Excerpt. Legation. p. 21.}

\pagenote[117]{See Godefroy’s Chronology of the Laws. Codex
Theodos tom. l. Prolegomen. p. xcix.—civ.}

\pagenote[118]{Most writers insist on the illness, and long
repose, of Theodosius, at Thessalonica: Zosimus, to diminish his
glory; Jornandes, to favor the Goths; and the ecclesiastical
writers, to introduce his baptism.}

The deliverance and peace of the Roman provinces\textsuperscript{119} was the work
of prudence, rather than of valor: the prudence of Theodosius was
seconded by fortune: and the emperor never failed to seize, and
to improve, every favorable circumstance. As long as the superior
genius of Fritigern preserved the union, and directed the motions
of the Barbarians, their power was not inadequate to the conquest
of a great empire. The death of that hero, the predecessor and
master of the renowned Alaric, relieved an impatient multitude
from the intolerable yoke of discipline and discretion. The
Barbarians, who had been restrained by his authority, abandoned
themselves to the dictates of their passions; and their passions
were seldom uniform or consistent. An army of conquerors was
broken into many disorderly bands of savage robbers; and their
blind and irregular fury was not less pernicious to themselves,
than to their enemies. Their mischievous disposition was shown in
the destruction of every object which they wanted strength to
remove, or taste to enjoy; and they often consumed, with
improvident rage, the harvests, or the granaries, which soon
afterwards became necessary for their own subsistence. A spirit
of discord arose among the independent tribes and nations, which
had been united only by the bands of a loose and voluntary
alliance. The troops of the Huns and the Alani would naturally
upbraid the flight of the Goths; who were not disposed to use
with moderation the advantages of their fortune; the ancient
jealousy of the Ostrogoths and the Visigoths could not long be
suspended; and the haughty chiefs still remembered the insults
and injuries, which they had reciprocally offered, or sustained,
while the nation was seated in the countries beyond the Danube.
The progress of domestic faction abated the more diffusive
sentiment of national animosity; and the officers of Theodosius
were instructed to purchase, with liberal gifts and promises, the
retreat or service of the discontented party. The acquisition of
Modar, a prince of the royal blood of the Amali, gave a bold and
faithful champion to the cause of Rome. The illustrious deserter
soon obtained the rank of master-general, with an important
command; surprised an army of his countrymen, who were immersed
in wine and sleep; and, after a cruel slaughter of the astonished
Goths, returned with an immense spoil, and four thousand wagons,
to the Imperial camp.\textsuperscript{120} In the hands of a skilful politician,
the most different means may be successfully applied to the same
ends; and the peace of the empire, which had been forwarded by
the divisions, was accomplished by the reunion, of the Gothic
nation. Athanaric, who had been a patient spectator of these
extraordinary events, was at length driven, by the chance of
arms, from the dark recesses of the woods of Caucaland. He no
longer hesitated to pass the Danube; and a very considerable part
of the subjects of Fritigern, who already felt the inconveniences
of anarchy, were easily persuaded to acknowledge for their king a
Gothic Judge, whose birth they respected, and whose abilities
they had frequently experienced. But age had chilled the daring
spirit of Athanaric; and, instead of leading his people to the
field of battle and victory, he wisely listened to the fair
proposal of an honorable and advantageous treaty. Theodosius, who
was acquainted with the merit and power of his new ally,
condescended to meet him at the distance of several miles from
Constantinople; and entertained him in the Imperial city, with
the confidence of a friend, and the magnificence of a monarch.
“The Barbarian prince observed, with curious attention, the
variety of objects which attracted his notice, and at last broke
out into a sincere and passionate exclamation of wonder. I now
behold (said he) what I never could believe, the glories of this
stupendous capital! And as he cast his eyes around, he viewed,
and he admired, the commanding situation of the city, the
strength and beauty of the walls and public edifices, the
capacious harbor, crowded with innumerable vessels, the perpetual
concourse of distant nations, and the arms and discipline of the
troops. Indeed, (continued Athanaric,) the emperor of the Romans
is a god upon earth; and the presumptuous man, who dares to lift
his hand against him, is guilty of his own blood.”\textsuperscript{121} The Gothic
king did not long enjoy this splendid and honorable reception;
and, as temperance was not the virtue of his nation, it may
justly be suspected, that his mortal disease was contracted
amidst the pleasures of the Imperial banquets. But the policy of
Theodosius derived more solid benefit from the death, than he
could have expected from the most faithful services, of his ally.
The funeral of Athanaric was performed with solemn rites in the
capital of the East; a stately monument was erected to his
memory; and his whole army, won by the liberal courtesy, and
decent grief, of Theodosius, enlisted under the standard of the
Roman empire.\textsuperscript{122} The submission of so great a body of the
Visigoths was productive of the most salutary consequences; and
the mixed influence of force, of reason, and of corruption,
became every day more powerful, and more extensive. Each
independent chieftain hastened to obtain a separate treaty, from
the apprehension that an obstinate delay might expose \textit{him},
alone and unprotected, to the revenge, or justice, of the
conqueror. The general, or rather the final, capitulation of the
Goths, may be dated four years, one month, and twenty-five days,
after the defeat and death of the emperor Valens.\textsuperscript{123}

\pagenote[119]{Compare Themistius (Orat, xiv. p. 181) with
Zosimus (l. iv. p. 232,) Jornandes, (c. xxvii. p. 649,) and the
prolix Commentary of M. de Buat, (Hist. de Peuples, \&c., tom. vi.
p. 477—552.) The Chronicles of Idatius and Marcellinus allude, in
general terms, to magna certamina, \textit{magna multaque} prælia. The
two epithets are not easily reconciled.}

\pagenote[120]{Zosimus (l. iv. p. 232) styles him a Scythian, a
name which the more recent Greeks seem to have appropriated to
the Goths.}

\pagenote[121]{The reader will not be displeased to see the
original words of Jornandes, or the author whom he transcribed.
Regiam urbem ingressus est, miransque, En, inquit, cerno quod
sæpe incredulus audiebam, famam videlicet tantæ urbis. Et huc
illuc oculos volvens, nunc situm urbis, commeatumque navium, nunc
mœnia clara pro spectans, miratur; populosque diversarum gentium,
quasi fonte in uno e diversis partibus scaturiente unda, sic
quoque militem ordinatum aspiciens; Deus, inquit, sine dubio est
terrenus Imperator, et quisquis adversus eum manum moverit, ipse
sui sanguinis reus existit Jornandes (c. xxviii. p. 650) proceeds
to mention his death and funeral.}

\pagenote[122]{Jornandes, c. xxviii. p. 650. Even Zosimus (l. v.
p. 246) is compelled to approve the generosity of Theodosius, so
honorable to himself, and so beneficial to the public.}

\pagenote[123]{The short, but authentic, hints in the \textit{Fasti} of
Idatius (Chron. Scaliger. p. 52) are stained with contemporary
passion. The fourteenth oration of Themistius is a compliment to
Peace, and the consul Saturninus, (A.D. 383.)}

The provinces of the Danube had been already relieved from the
oppressive weight of the Gruthungi, or Ostrogoths, by the
voluntary retreat of Alatheus and Saphrax, whose restless spirit
had prompted them to seek new scenes of rapine and glory. Their
destructive course was pointed towards the West; but we must be
satisfied with a very obscure and imperfect knowledge of their
various adventures. The Ostrogoths impelled several of the German
tribes on the provinces of Gaul; concluded, and soon violated, a
treaty with the emperor Gratian; advanced into the unknown
countries of the North; and, after an interval of more than four
years, returned, with accumulated force, to the banks of the
Lower Danube. Their troops were recruited with the fiercest
warriors of Germany and Scythia; and the soldiers, or at least
the historians, of the empire, no longer recognized the name and
countenances of their former enemies.\textsuperscript{124} The general who
commanded the military and naval powers of the Thracian frontier,
soon perceived that his superiority would be disadvantageous to
the public service; and that the Barbarians, awed by the presence
of his fleet and legions, would probably defer the passage of the
river till the approaching winter. The dexterity of the spies,
whom he sent into the Gothic camp, allured the Barbarians into a
fatal snare. They were persuaded that, by a bold attempt, they
might surprise, in the silence and darkness of the night, the
sleeping army of the Romans; and the whole multitude was hastily
embarked in a fleet of three thousand canoes.\textsuperscript{125} The bravest of
the Ostrogoths led the van; the main body consisted of the
remainder of their subjects and soldiers; and the women and
children securely followed in the rear. One of the nights without
a moon had been selected for the execution of their design; and
they had almost reached the southern bank of the Danube, in the
firm confidence that they should find an easy landing and an
unguarded camp. But the progress of the Barbarians was suddenly
stopped by an unexpected obstacle a triple line of vessels,
strongly connected with each other, and which formed an
impenetrable chain of two miles and a half along the river. While
they struggled to force their way in the unequal conflict, their
right flank was overwhelmed by the irresistible attack of a fleet
of galleys, which were urged down the stream by the united
impulse of oars and of the tide. The weight and velocity of those
ships of war broke, and sunk, and dispersed, the rude and feeble
canoes of the Barbarians; their valor was ineffectual; and
Alatheus, the king, or general, of the Ostrogoths, perished with
his bravest troops, either by the sword of the Romans, or in the
waves of the Danube. The last division of this unfortunate fleet
might regain the opposite shore; but the distress and disorder of
the multitude rendered them alike incapable, either of action or
counsel; and they soon implored the clemency of the victorious
enemy. On this occasion, as well as on many others, it is a
difficult task to reconcile the passions and prejudices of the
writers of the age of Theodosius. The partial and malignant
historian, who misrepresents every action of his reign, affirms,
that the emperor did not appear in the field of battle till the
Barbarians had been vanquished by the valor and conduct of his
lieutenant Promotus.\textsuperscript{126} The flattering poet, who celebrated, in
the court of Honorius, the glory of the father and of the son,
ascribes the victory to the personal prowess of Theodosius; and
almost insinuates, that the king of the Ostrogoths was slain by
the hand of the emperor.\textsuperscript{127} The truth of history might perhaps
be found in a just medium between these extreme and contradictory
assertions.

\pagenote[124]{Zosimus, l. iv. p. 252.}

\pagenote[125]{I am justified, by reason and example, in applying
this Indian name to the the Barbarians, the single trees hollowed
into the shape of a boat. Zosimus, l. iv. p. 253. Ausi Danubium
quondam tranare Gruthungi In lintres fregere nemus: ter mille
ruebant Per fluvium plenæ cuneis immanibus alni. Claudian, in iv.
Cols. Hon. 623.}

\pagenote[126]{Zosimus, l. iv. p. 252—255. He too frequently
betrays his poverty of judgment by disgracing the most serious
narratives with trifling and incredible circumstances.}

\pagenote[127]{—Odothæi Regis \textit{opima} Retulit—Ver. 632. The
\textit{opima} were the spoils which a Roman general could only win from
the king, or general, of the enemy, whom he had slain with his
own hands: and no more than three such examples are celebrated in
the victorious ages of Rome.}

The original treaty which fixed the settlement of the Goths,
ascertained their privileges, and stipulated their obligations,
would illustrate the history of Theodosius and his successors.
The series of their history has imperfectly preserved the spirit
and substance of this single agreement.\textsuperscript{128} The ravages of war
and tyranny had provided many large tracts of fertile but
uncultivated land for the use of those Barbarians who might not
disdain the practice of agriculture. A numerous colony of the
Visigoths was seated in Thrace; the remains of the Ostrogoths
were planted in Phrygia and Lydia; their immediate wants were
supplied by a distribution of corn and cattle; and their future
industry was encouraged by an exemption from tribute, during a
certain term of years. The Barbarians would have deserved to feel
the cruel and perfidious policy of the Imperial court, if they
had suffered themselves to be dispersed through the provinces.
They required, and they obtained, the sole possession of the
villages and districts assigned for their residence; they still
cherished and propagated their native manners and language;
asserted, in the bosom of despotism, the freedom of their
domestic government; and acknowledged the sovereignty of the
emperor, without submitting to the inferior jurisdiction of the
laws and magistrates of Rome. The hereditary chiefs of the tribes
and families were still permitted to command their followers in
peace and war; but the royal dignity was abolished; and the
generals of the Goths were appointed and removed at the pleasure
of the emperor. An army of forty thousand Goths was maintained
for the perpetual service of the empire of the East; and those
haughty troops, who assumed the title of \textit{Fæderati}, or allies,
were distinguished by their gold collars, liberal pay, and
licentious privileges. Their native courage was improved by the
use of arms and the knowledge of discipline; and, while the
republic was guarded, or threatened, by the doubtful sword of the
Barbarians, the last sparks of the military flame were finally
extinguished in the minds of the Romans.\textsuperscript{129} Theodosius had the
address to persuade his allies, that the conditions of peace,
which had been extorted from him by prudence and necessity, were
the voluntary expressions of his sincere friendship for the
Gothic nation.\textsuperscript{130} A different mode of vindication or apology was
opposed to the complaints of the people; who loudly censured
these shameful and dangerous concessions.\textsuperscript{131} The calamities of
the war were painted in the most lively colors; and the first
symptoms of the return of order, of plenty, and security, were
diligently exaggerated. The advocates of Theodosius could affirm,
with some appearance of truth and reason, that it was impossible
to extirpate so many warlike tribes, who were rendered desperate
by the loss of their native country; and that the exhausted
provinces would be revived by a fresh supply of soldiers and
husbandmen. The Barbarians still wore an angry and hostile
aspect; but the experience of past times might encourage the
hope, that they would acquire the habits of industry and
obedience; that their manners would be polished by time,
education, and the influence of Christianity; and that their
posterity would insensibly blend with the great body of the Roman
people.\textsuperscript{132}

\pagenote[128]{See Themistius, Orat. xvi. p. 211. Claudian (in
Eutrop. l. ii. 112) mentions the Phrygian colony:——Ostrogothis
colitur mistisque Gruthungis Phyrx ager——and then proceeds to
name the rivers of Lydia, the Pactolus, and Herreus.}

\pagenote[129]{Compare Jornandes, (c. xx. 27,) who marks the
condition and number of the Gothic \textit{Fæderati}, with Zosimus, (l.
iv. p. 258,) who mentions their golden collars; and Pacatus, (in
Panegyr. Vet. xii. 37,) who applauds, with false or foolish joy,
their bravery and discipline.}

\pagenote[130]{Amator pacis generisque Gothorum, is the praise
bestowed by the Gothic historian, (c. xxix.,) who represents his
nation as innocent, peaceable men, slow to anger, and patient of
injuries. According to Livy, the Romans conquered the world in
their own defence.}

\pagenote[131]{Besides the partial invectives of Zosimus, (always
discontented with the Christian reigns,) see the grave
representations which Synesius addresses to the emperor Arcadius,
(de Regno, p. 25, 26, edit. Petav.) The philosophic bishop of
Cyrene was near enough to judge; and he was sufficiently removed
from the temptation of fear or flattery.}

\pagenote[132]{Themistius (Orat. xvi. p. 211, 212) composes an
elaborate and rational apology, which is not, however, exempt
from the puerilities of Greek rhetoric. Orpheus could \textit{only}
charm the wild beasts of Thrace; but Theodosius enchanted the men
and women, whose predecessors in the same country had torn
Orpheus in pieces, \&c.}

Notwithstanding these specious arguments, and these sanguine
expectations, it was apparent to every discerning eye, that the
Goths would long remain the enemies, and might soon become the
conquerors of the Roman empire. Their rude and insolent behavior
expressed their contempt of the citizens and provincials, whom
they insulted with impunity.\textsuperscript{133} To the zeal and valor of the
Barbarians Theodosius was indebted for the success of his arms:
but their assistance was precarious; and they were sometimes
seduced, by a treacherous and inconstant disposition, to abandon
his standard, at the moment when their service was the most
essential. During the civil war against Maximus, a great number
of Gothic deserters retired into the morasses of Macedonia,
wasted the adjacent provinces, and obliged the intrepid monarch
to expose his person, and exert his power, to suppress the rising
flame of rebellion.\textsuperscript{134} The public apprehensions were fortified
by the strong suspicion, that these tumults were not the effect
of accidental passion, but the result of deep and premeditated
design. It was generally believed, that the Goths had signed the
treaty of peace with a hostile and insidious spirit; and that
their chiefs had previously bound themselves, by a solemn and
secret oath, never to keep faith with the Romans; to maintain the
fairest show of loyalty and friendship, and to watch the
favorable moment of rapine, of conquest, and of revenge. But as
the minds of the Barbarians were not insensible to the power of
gratitude, several of the Gothic leaders sincerely devoted
themselves to the service of the empire, or, at least, of the
emperor; the whole nation was insensibly divided into two
opposite factions, and much sophistry was employed in
conversation and dispute, to compare the obligations of their
first, and second, engagements. The Goths, who considered
themselves as the friends of peace, of justice, and of Rome, were
directed by the authority of Fravitta, a valiant and honorable
youth, distinguished above the rest of his countrymen by the
politeness of his manners, the liberality of his sentiments, and
the mild virtues of social life. But the more numerous faction
adhered to the fierce and faithless Priulf,\textsuperscript{13411} who inflamed
the passions, and asserted the independence, of his warlike
followers. On one of the solemn festivals, when the chiefs of
both parties were invited to the Imperial table, they were
insensibly heated by wine, till they forgot the usual restraints
of discretion and respect, and betrayed, in the presence of
Theodosius, the fatal secret of their domestic disputes. The
emperor, who had been the reluctant witness of this extraordinary
controversy, dissembled his fears and resentment, and soon
dismissed the tumultuous assembly. Fravitta, alarmed and
exasperated by the insolence of his rival, whose departure from
the palace might have been the signal of a civil war, boldly
followed him; and, drawing his sword, laid Priulf dead at his
feet. Their companions flew to arms; and the faithful champion of
Rome would have been oppressed by superior numbers, if he had not
been protected by the seasonable interposition of the Imperial
guards.\textsuperscript{135} Such were the scenes of Barbaric rage, which
disgraced the palace and table of the Roman emperor; and, as the
impatient Goths could only be restrained by the firm and
temperate character of Theodosius, the public safety seemed to
depend on the life and abilities of a single man.\textsuperscript{136}

\pagenote[133]{Constantinople was deprived half a day of the
public allowance of bread, to expiate the murder of a Gothic
soldier: was the guilt of the people. Libanius, Orat. xii. p.
394, edit. Morel.}

\pagenote[134]{Zosimus, l. iv. p. 267-271. He tells a long and
ridiculous story of the adventurous prince, who roved the country
with only five horsemen, of a spy whom they detected, whipped,
and killed in an old woman’s cottage, \&c.}

\pagenote[13411]{Eunapius.—M.}

\pagenote[135]{Compare Eunapius (in Excerpt. Legat. p. 21, 22)
with Zosimus, (l. iv. p. 279.) The difference of circumstances
and names must undoubtedly be applied to the same story.
Fravitta, or Travitta, was afterwards consul, (A.D. 401.) and
still continued his faithful services to the eldest son of
Theodosius. (Tillemont, Hist. des Empereurs, tom. v. p. 467.)}

\pagenote[136]{Les Goths ravagerent tout depuis le Danube
jusqu’au Bosphore; exterminerent Valens et son armée; et ne
repasserent le Danube, que pour abandonner l’affreuse solitude
qu’ils avoient faite, (Œuvres de Montesquieu, tom. iii. p. 479.
Considerations sur les \textit{Causes} de la Grandeur et de la Décadence
des Romains, c. xvii.) The president Montesquieu seems ignorant
that the Goths, after the defeat of Valens, \textit{never} abandoned the
Roman territory. It is now thirty years, says Claudian, (de Bello
Getico, 166, \&c., A.D. 404,) Ex quo jam patrios gens hæc oblita
Triones, Atque Istrum transvecta semel, vestigia fixit Threicio
funesta solo—the error is inexcusable; since it disguises the
principal and immediate cause of the fall of the Western empire
of Rome.}

