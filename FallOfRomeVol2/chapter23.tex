\chapter{Reign Of Julian.}
\section{Part \thesection.}

\textit{The Religion Of Julian. — Universal Toleration. — He Attempts To
Restore And Reform The Pagan Worship — To Rebuild The Temple Of
Jerusalem — His Artful Persecution Of The Christians. — Mutual Zeal
And Injustice.}
\vspace{\onelineskip}

The character of Apostate has injured the reputation of Julian;
and the enthusiasm which clouded his virtues has exaggerated the
real and apparent magnitude of his faults. Our partial ignorance
may represent him as a philosophic monarch, who studied to
protect, with an equal hand, the religious factions of the
empire; and to allay the theological fever which had inflamed the
minds of the people, from the edicts of Diocletian to the exile
of Athanasius. A more accurate view of the character and conduct
of Julian will remove this favorable prepossession for a prince
who did not escape the general contagion of the times. We enjoy
the singular advantage of comparing the pictures which have been
delineated by his fondest admirers and his implacable enemies.
The actions of Julian are faithfully related by a judicious and
candid historian, the impartial spectator of his life and death.
The unanimous evidence of his contemporaries is confirmed by the
public and private declarations of the emperor himself; and his
various writings express the uniform tenor of his religious
sentiments, which policy would have prompted him to dissemble
rather than to affect. A devout and sincere attachment for the
gods of Athens and Rome constituted the ruling passion of Julian;\textsuperscript{1}
the powers of an enlightened understanding were betrayed and
corrupted by the influence of superstitious prejudice; and the
phantoms which existed only in the mind of the emperor had a real
and pernicious effect on the government of the empire. The
vehement zeal of the Christians, who despised the worship, and
overturned the altars of those fabulous deities, engaged their
votary in a state of irreconcilable hostility with a very
numerous party of his subjects; and he was sometimes tempted by
the desire of victory, or the shame of a repulse, to violate the
laws of prudence, and even of justice. The triumph of the party,
which he deserted and opposed, has fixed a stain of infamy on the
name of Julian; and the unsuccessful apostate has been
overwhelmed with a torrent of pious invectives, of which the
signal was given by the sonorous trumpet\textsuperscript{2} of Gregory Nazianzen.\textsuperscript{3}
The interesting nature of the events which were crowded into
the short reign of this active emperor, deserve a just and
circumstantial narrative. His motives, his counsels, and his
actions, as far as they are connected with the history of
religion, will be the subject of the present chapter.

\pagenote[1]{I shall transcribe some of his own expressions from
a short religious discourse which the Imperial pontiff composed
to censure the bold impiety of a Cynic. Orat. vii. p. 212. The
variety and copiousness of the Greek tongue seem inadequate to
the fervor of his devotion.}

\pagenote[2]{The orator, with some eloquence, much enthusiasm,
and more vanity, addresses his discourse to heaven and earth, to
men and angels, to the living and the dead; and above all, to the
great Constantius, an odd Pagan expression. He concludes with a
bold assurance, that he has erected a monument not less durable,
and much more portable, than the columns of Hercules. See Greg.
Nazianzen, Orat. iii. p. 50, iv. p. 134.}

\pagenote[3]{See this long invective, which has been
injudiciously divided into two orations in Gregory’s works, tom.
i. p. 49-134, Paris, 1630. It was published by Gregory and his
friend Basil, (iv. p. 133,) about six months after the death of
Julian, when his remains had been carried to Tarsus, (iv. p.
120;) but while Jovian was still on the throne, (iii. p. 54, iv.
p. 117) I have derived much assistance from a French version and
remarks, printed at Lyons, 1735.}

The cause of his strange and fatal apostasy may be derived from
the early period of his life, when he was left an orphan in the
hands of the murderers of his family. The names of Christ and of
Constantius, the ideas of slavery and of religion, were soon
associated in a youthful imagination, which was susceptible of
the most lively impressions. The care of his infancy was
intrusted to Eusebius, bishop of Nicomedia,\textsuperscript{4} who was related to
him on the side of his mother; and till Julian reached the
twentieth year of his age, he received from his Christian
preceptors the education, not of a hero, but of a saint. The
emperor, less jealous of a heavenly than of an earthly crown,
contented himself with the imperfect character of a catechumen,
while he bestowed the advantages of baptism\textsuperscript{5} on the nephews of
Constantine.\textsuperscript{6} They were even admitted to the inferior offices of
the ecclesiastical order; and Julian publicly read the Holy
Scriptures in the church of Nicomedia. The study of religion,
which they assiduously cultivated, appeared to produce the
fairest fruits of faith and devotion.\textsuperscript{7} They prayed, they fasted,
they distributed alms to the poor, gifts to the clergy, and
oblations to the tombs of the martyrs; and the splendid monument
of St. Mamas, at Cæsarea, was erected, or at least was
undertaken, by the joint labor of Gallus and Julian.\textsuperscript{8} They
respectfully conversed with the bishops, who were eminent for
superior sanctity, and solicited the benediction of the monks and
hermits, who had introduced into Cappadocia the voluntary
hardships of the ascetic life.\textsuperscript{9} As the two princes advanced
towards the years of manhood, they discovered, in their religious
sentiments, the difference of their characters. The dull and
obstinate understanding of Gallus embraced, with implicit zeal,
the doctrines of Christianity; which never influenced his
conduct, or moderated his passions. The mild disposition of the
younger brother was less repugnant to the precepts of the gospel;
and his active curiosity might have been gratified by a
theological system, which explains the mysterious essence of the
Deity, and opens the boundless prospect of invisible and future
worlds. But the independent spirit of Julian refused to yield the
passive and unresisting obedience which was required, in the name
of religion, by the haughty ministers of the church. Their
speculative opinions were imposed as positive laws, and guarded
by the terrors of eternal punishments; but while they prescribed
the rigid formulary of the thoughts, the words, and the actions
of the young prince; whilst they silenced his objections, and
severely checked the freedom of his inquiries, they secretly
provoked his impatient genius to disclaim the authority of his
ecclesiastical guides. He was educated in the Lesser Asia, amidst
the scandals of the Arian controversy.\textsuperscript{10} The fierce contests of
the Eastern bishops, the incessant alterations of their creeds,
and the profane motives which appeared to actuate their conduct,
insensibly strengthened the prejudice of Julian, that they
neither understood nor believed the religion for which they so
fiercely contended. Instead of listening to the proofs of
Christianity with that favorable attention which adds weight to
the most respectable evidence, he heard with suspicion, and
disputed with obstinacy and acuteness, the doctrines for which he
already entertained an invincible aversion. Whenever the young
princes were directed to compose declamations on the subject of
the prevailing controversies, Julian always declared himself the
advocate of Paganism; under the specious excuse that, in the
defence of the weaker cause, his learning and ingenuity might be
more advantageously exercised and displayed.

\pagenote[4]{Nicomediæ ab Eusebio educatus Episcopo, quem genere
longius contingebat, (Ammian. xxii. 9.) Julian never expresses
any gratitude towards that Arian prelate; but he celebrates his
preceptor, the eunuch Mardonius, and describes his mode of
education, which inspired his pupil with a passionate admiration
for the genius, and perhaps the religion of Homer. Misopogon, p.
351, 352.}

\pagenote[5]{Greg. Naz. iii. p. 70. He labored to effect that
holy mark in the blood, perhaps of a Taurobolium. Baron. Annal.
Eccles. A. D. 361, No. 3, 4.}

\pagenote[6]{Julian himself (Epist. li. p. 454) assures the
Alexandrians that he had been a Christian (he must mean a sincere
one) till the twentieth year of his age.}

\pagenote[7]{See his Christian, and even ecclesiastical
education, in Gregory, (iii. p. 58,) Socrates, (l. iii. c. 1,)
and Sozomen, (l. v. c. 2.) He escaped very narrowly from being a
bishop, and perhaps a saint.}

\pagenote[8]{The share of the work which had been allotted to
Gallus, was prosecuted with vigor and success; but the earth
obstinately rejected and subverted the structures which were
imposed by the sacrilegious hand of Julian. Greg. iii. p. 59, 60,
61. Such a partial earthquake, attested by many living
spectators, would form one of the clearest miracles in
ecclesiastical story.}

\pagenote[9]{The \textit{philosopher} (Fragment, p. 288,) ridicules the
iron chains, \&c, of these solitary fanatics, (see Tillemont, Mém.
Eccles. tom. ix. p. 661, 632,) who had forgot that man is by
nature a gentle and social animal. The \textit{Pagan} supposes, that
because they had renounced the gods, they were possessed and
tormented by evil dæmons.}

\pagenote[10]{See Julian apud Cyril, l. vi. p. 206, l. viii. p.
253, 262. “You persecute,” says he, “those heretics who do not
mourn the dead man precisely in the way which you approve.” He
shows himself a tolerable theologian; but he maintains that the
Christian Trinity is not derived from the doctrine of Paul, of
Jesus, or of Moses.}

As soon as Gallus was invested with the honors of the purple,
Julian was permitted to breathe the air of freedom, of
literature, and of Paganism.\textsuperscript{11} The crowd of sophists, who were
attracted by the taste and liberality of their royal pupil, had
formed a strict alliance between the learning and the religion of
Greece; and the poems of Homer, instead of being admired as the
original productions of human genius, were seriously ascribed to
the heavenly inspiration of Apollo and the muses. The deities of
Olympus, as they are painted by the immortal bard, imprint
themselves on the minds which are the least addicted to
superstitious credulity. Our familiar knowledge of their names
and characters, their forms and attributes, \textit{seems} to bestow on
those airy beings a real and substantial existence; and the
pleasing enchantment produces an imperfect and momentary assent
of the imagination to those fables, which are the most repugnant
to our reason and experience. In the age of Julian, every
circumstance contributed to prolong and fortify the illusion; the
magnificent temples of Greece and Asia; the works of those
artists who had expressed, in painting or in sculpture, the
divine conceptions of the poet; the pomp of festivals and
sacrifices; the successful arts of divination; the popular
traditions of oracles and prodigies; and the ancient practice of
two thousand years. The weakness of polytheism was, in some
measure, excused by the moderation of its claims; and the
devotion of the Pagans was not incompatible with the most
licentious scepticism.\textsuperscript{12} Instead of an indivisible and regular
system, which occupies the whole extent of the believing mind,
the mythology of the Greeks was composed of a thousand loose and
flexible parts, and the servant of the gods was at liberty to
define the degree and measure of his religious faith. The creed
which Julian adopted for his own use was of the largest
dimensions; and, by strange contradiction, he disdained the
salutary yoke of the gospel, whilst he made a voluntary offering
of his reason on the altars of Jupiter and Apollo. One of the
orations of Julian is consecrated to the honor of Cybele, the
mother of the gods, who required from her effeminate priests the
bloody sacrifice, so rashly performed by the madness of the
Phrygian boy. The pious emperor condescends to relate, without a
blush, and without a smile, the voyage of the goddess from the
shores of Pergamus to the mouth of the Tyber, and the stupendous
miracle, which convinced the senate and people of Rome that the
lump of clay, which their ambassadors had transported over the
seas, was endowed with life, and sentiment, and divine power.\textsuperscript{13}
For the truth of this prodigy he appeals to the public monuments
of the city; and censures, with some acrimony, the sickly and
affected taste of those men, who impertinently derided the sacred
traditions of their ancestors.\textsuperscript{14}

\pagenote[11]{Libanius, Orat. Parentalis, c. 9, 10, p. 232, \&c.
Greg. Nazianzen. Orat. iii. p 61. Eunap. Vit. Sophist. in Maximo,
p. 68, 69, 70, edit Commelin.}

\pagenote[12]{A modern philosopher has ingeniously compared the
different operation of theism and polytheism, with regard to the
doubt or conviction which they produce in the human mind. See
Hume’s Essays vol. ii. p. 444- 457, in 8vo. edit. 1777.}

\pagenote[13]{The Idæan mother landed in Italy about the end of
the second Punic war. The miracle of Claudia, either virgin or
matron, who cleared her fame by disgracing the graver modesty of
the Roman Indies, is attested by a cloud of witnesses. Their
evidence is collected by Drakenborch, (ad Silium Italicum, xvii.
33;) but we may observe that Livy (xxix. 14) slides over the
transaction with discreet ambiguity.}

\pagenote[14]{I cannot refrain from transcribing the emphatical
words of Julian: Orat. v. p. 161. Julian likewise declares his
firm belief in the ancilia, the holy shields, which dropped from
heaven on the Quirinal hill; and pities the strange blindness of
the Christians, who preferred the cross to these celestial
trophies. Apud Cyril. l. vi. p. 194.}

But the devout philosopher, who sincerely embraced, and warmly
encouraged, the superstition of the people, reserved for himself
the privilege of a liberal interpretation; and silently withdrew
from the foot of the altars into the sanctuary of the temple. The
extravagance of the Grecian mythology proclaimed, with a clear
and audible voice, that the pious inquirer, instead of being
scandalized or satisfied with the literal sense, should
diligently explore the occult wisdom, which had been disguised,
by the prudence of antiquity, under the mask of folly and of
fable.\textsuperscript{15} The philosophers of the Platonic school,\textsuperscript{16} Plotinus,
Porphyry, and the divine Iamblichus, were admired as the most
skilful masters of this allegorical science, which labored to
soften and harmonize the deformed features of Paganism. Julian
himself, who was directed in the mysterious pursuit by Ædesius,
the venerable successor of Iamblichus, aspired to the possession
of a treasure, which he esteemed, if we may credit his solemn
asseverations, far above the empire of the world.\textsuperscript{17} It was
indeed a treasure, which derived its value only from opinion; and
every artist who flattered himself that he had extracted the
precious ore from the surrounding dross, claimed an equal right
of stamping the name and figure the most agreeable to his
peculiar fancy. The fable of Atys and Cybele had been already
explained by Porphyry; but his labors served only to animate the
pious industry of Julian, who invented and published his own
allegory of that ancient and mystic tale. This freedom of
interpretation, which might gratify the pride of the Platonists,
exposed the vanity of their art. Without a tedious detail, the
modern reader could not form a just idea of the strange
allusions, the forced etymologies, the solemn trifling, and the
impenetrable obscurity of these sages, who professed to reveal
the system of the universe. As the traditions of Pagan mythology
were variously related, the sacred interpreters were at liberty
to select the most convenient circumstances; and as they
translated an arbitrary cipher, they could extract from \textit{any}
fable \textit{any} sense which was adapted to their favorite system of
religion and philosophy. The lascivious form of a naked Venus was
tortured into the discovery of some moral precept, or some
physical truth; and the castration of Atys explained the
revolution of the sun between the tropics, or the separation of
the human soul from vice and error.\textsuperscript{18}

\pagenote[15]{See the principles of allegory, in Julian, (Orat.
vii. p. 216, 222.) His reasoning is less absurd than that of some
modern theologians, who assert that an extravagant or
contradictory doctrine must be divine; since no man alive could
have thought of inventing it.}

\pagenote[16]{Eunapius has made these sophists the subject of a
partial and fanatical history; and the learned Brucker (Hist.
Philosoph. tom. ii. p. 217-303) has employed much labor to
illustrate their obscure lives and incomprehensible doctrines.}

\pagenote[17]{Julian, Orat. vii p 222. He swears with the most
fervent and enthusiastic devotion; and trembles, lest he should
betray too much of these holy mysteries, which the profane might
deride with an impious Sardonic laugh.}

\pagenote[18]{See the fifth oration of Julian. But all the
allegories which ever issued from the Platonic school are not
worth the short poem of Catullus on the same extraordinary
subject. The transition of Atys, from the wildest enthusiasm to
sober, pathetic complaint, for his irretrievable loss, must
inspire a man with pity, a eunuch with despair.}

The theological system of Julian appears to have contained the
sublime and important principles of natural religion. But as the
faith, which is not founded on revelation, must remain destitute
of any firm assurance, the disciple of Plato imprudently relapsed
into the habits of vulgar superstition; and the popular and
philosophic notion of the Deity seems to have been confounded in
the practice, the writings, and even in the mind of Julian.\textsuperscript{19}
The pious emperor acknowledged and adored the Eternal Cause of
the universe, to whom he ascribed all the perfections of an
infinite nature, invisible to the eyes and inaccessible to the
understanding, of feeble mortals. The Supreme God had created, or
rather, in the Platonic language, had generated, the gradual
succession of dependent spirits, of gods, of dæmons, of heroes,
and of men; and every being which derived its existence
immediately from the First Cause, received the inherent gift of
immortality. That so precious an advantage might not be lavished
upon unworthy objects, the Creator had intrusted to the skill and
power of the inferior gods the office of forming the human body,
and of arranging the beautiful harmony of the animal, the
vegetable, and the mineral kingdoms. To the conduct of these
divine ministers he delegated the temporal government of this
lower world; but their imperfect administration is not exempt
from discord or error. The earth and its inhabitants are divided
among them, and the characters of Mars or Minerva, of Mercury or
Venus, may be distinctly traced in the laws and manners of their
peculiar votaries. As long as our immortal souls are confined in
a mortal prison, it is our interest, as well as our duty, to
solicit the favor, and to deprecate the wrath, of the powers of
heaven; whose pride is gratified by the devotion of mankind; and
whose grosser parts may be supposed to derive some nourishment
from the fumes of sacrifice.\textsuperscript{20} The inferior gods might sometimes
condescend to animate the statues, and to inhabit the temples,
which were dedicated to their honor. They might occasionally
visit the earth, but the heavens were the proper throne and
symbol of their glory. The invariable order of the sun, moon, and
stars, was hastily admitted by Julian, as a proof of their
\textit{eternal} duration; and their eternity was a sufficient evidence
that they were the workmanship, not of an inferior deity, but of
the Omnipotent King. In the system of Platonists, the visible was
a type of the invisible world. The celestial bodies, as they were
informed by a divine spirit, might be considered as the objects
the most worthy of religious worship. The Sun, whose genial
influence pervades and sustains the universe, justly claimed the
adoration of mankind, as the bright representative of the Logos,
the lively, the rational, the beneficent image of the
intellectual Father.\textsuperscript{21}

\pagenote[19]{The true religion of Julian may be deduced from the
Cæsars, p. 308, with Spanheim’s notes and illustrations, from the
fragments in Cyril, l. ii. p. 57, 58, and especially from the
theological oration in Solem Regem, p. 130-158, addressed in the
confidence of friendship, to the præfect Sallust.}

\pagenote[20]{Julian adopts this gross conception by ascribing to
his favorite Marcus Antoninus, (Cæsares, p. 333.) The Stoics and
Platonists hesitated between the analogy of bodies and the purity
of spirits; yet the gravest philosophers inclined to the
whimsical fancy of Aristophanes and Lucian, that an unbelieving
age might starve the immortal gods. See Observations de Spanheim,
p. 284, 444, \&c.}

\pagenote[21]{Julian. Epist. li. In another place, (apud Cyril.
l. ii. p. 69,) he calls the Sun God, and the throne of God.
Julian believed the Platonician Trinity; and only blames the
Christians for preferring a mortal to an immortal \textit{Logos}.}

In every age, the absence of genuine inspiration is supplied by
the strong illusions of enthusiasm, and the mimic arts of
imposture. If, in the time of Julian, these arts had been
practised only by the pagan priests, for the support of an
expiring cause, some indulgence might perhaps be allowed to the
interest and habits of the sacerdotal character. But it may
appear a subject of surprise and scandal, that the philosophers
themselves should have contributed to abuse the superstitious
credulity of mankind,\textsuperscript{22} and that the Grecian mysteries should
have been supported by the magic or theurgy of the modern
Platonists. They arrogantly pretended to control the order of
nature, to explore the secrets of futurity, to command the
service of the inferior dæmons, to enjoy the view and
conversation of the superior gods, and by disengaging the soul
from her material bands, to reunite that immortal particle with
the Infinite and Divine Spirit.

\pagenote[22]{The sophists of Eunapias perform as many miracles
as the saints of the desert; and the only circumstance in their
favor is, that they are of a less gloomy complexion. Instead of
devils with horns and tails, Iamblichus evoked the genii of love,
Eros and Anteros, from two adjacent fountains. Two beautiful boys
issued from the water, fondly embraced him as their father, and
retired at his command, p. 26, 27.}

The devout and fearless curiosity of Julian tempted the
philosophers with the hopes of an easy conquest; which, from the
situation of their young proselyte, might be productive of the
most important consequences.\textsuperscript{23} Julian imbibed the first
rudiments of the Platonic doctrines from the mouth of Ædesius,
who had fixed at Pergamus his wandering and persecuted school.
But as the declining strength of that venerable sage was unequal
to the ardor, the diligence, the rapid conception of his pupil,
two of his most learned disciples, Chrysanthes and Eusebius,
supplied, at his own desire, the place of their aged master.
These philosophers seem to have prepared and distributed their
respective parts; and they artfully contrived, by dark hints and
affected disputes, to excite the impatient hopes of the
\textit{aspirant}, till they delivered him into the hands of their
associate, Maximus, the boldest and most skilful master of the
Theurgic science. By his hands, Julian was secretly initiated at
Ephesus, in the twentieth year of his age. His residence at
Athens confirmed this unnatural alliance of philosophy and
superstition.

He obtained the privilege of a solemn initiation into the
mysteries of Eleusis, which, amidst the general decay of the
Grecian worship, still retained some vestiges of their primæval
sanctity; and such was the zeal of Julian, that he afterwards
invited the Eleusinian pontiff to the court of Gaul, for the sole
purpose of consummating, by mystic rites and sacrifices, the
great work of his sanctification. As these ceremonies were
performed in the depth of caverns, and in the silence of the
night, and as the inviolable secret of the mysteries was
preserved by the discretion of the initiated, I shall not presume
to describe the horrid sounds, and fiery apparitions, which were
presented to the senses, or the imagination, of the credulous
aspirant,\textsuperscript{24} till the visions of comfort and knowledge broke upon
him in a blaze of celestial light.\textsuperscript{25} In the caverns of Ephesus
and Eleusis, the mind of Julian was penetrated with sincere,
deep, and unalterable enthusiasm; though he might sometimes
exhibit the vicissitudes of pious fraud and hypocrisy, which may
be observed, or at least suspected, in the characters of the most
conscientious fanatics. From that moment he consecrated his life
to the service of the gods; and while the occupations of war, of
government, and of study, seemed to claim the whole measure of
his time, a stated portion of the hours of the night was
invariably reserved for the exercise of private devotion. The
temperance which adorned the severe manners of the soldier and
the philosopher was connected with some strict and frivolous
rules of religious abstinence; and it was in honor of Pan or
Mercury, of Hecate or Isis, that Julian, on particular days,
denied himself the use of some particular food, which might have
been offensive to his tutelar deities. By these voluntary fasts,
he prepared his senses and his understanding for the frequent and
familiar visits with which he was honored by the celestial
powers. Notwithstanding the modest silence of Julian himself, we
may learn from his faithful friend, the orator Libanius, that he
lived in a perpetual intercourse with the gods and goddesses;
that they descended upon earth to enjoy the conversation of their
favorite hero; that they gently interrupted his slumbers by
touching his hand or his hair; that they warned him of every
impending danger, and conducted him, by their infallible wisdom,
in every action of his life; and that he had acquired such an
intimate knowledge of his heavenly guests, as readily to
distinguish the voice of Jupiter from that of Minerva, and the
form of Apollo from the figure of Hercules.\textsuperscript{26} These sleeping or
waking visions, the ordinary effects of abstinence and
fanaticism, would almost degrade the emperor to the level of an
Egyptian monk. But the useless lives of Antony or Pachomius were
consumed in these vain occupations. Julian could break from the
dream of superstition to arm himself for battle; and after
vanquishing in the field the enemies of Rome, he calmly retired
into his tent, to dictate the wise and salutary laws of an
empire, or to indulge his genius in the elegant pursuits of
literature and philosophy.

\pagenote[23]{The dexterous management of these sophists, who
played their credulous pupil into each other’s hands, is fairly
told by Eunapius (p. 69- 79) with unsuspecting simplicity. The
Abbé de la Bleterie understands, and neatly describes, the whole
comedy, (Vie de Julian, p. 61-67.)}

\pagenote[24]{When Julian, in a momentary panic, made the sign of
the cross the dæmons instantly disappeared, (Greg. Naz. Orat.
iii. p. 71.) Gregory supposes that they were frightened, but the
priests declared that they were indignant. The reader, according
to the measure of his faith, will determine this profound
question.}

\pagenote[25]{A dark and distant view of the terrors and joys of
initiation is shown by Dion Chrysostom, Themistius, Proclus, and
Stobæus. The learned author of the Divine Legation has exhibited
their words, (vol. i. p. 239, 247, 248, 280, edit. 1765,) which
he dexterously or forcibly applies to his own hypothesis.}

\pagenote[26]{Julian’s modesty confined him to obscure and
occasional hints: but Libanius expiates with pleasure on the
facts and visions of the religious hero. (Legat. ad Julian. p.
157, and Orat. Parental. c. lxxxiii. p. 309, 310.)}

The important secret of the apostasy of Julian was intrusted to
the fidelity of the \textit{initiated}, with whom he was united by the
sacred ties of friendship and religion.\textsuperscript{27} The pleasing rumor was
cautiously circulated among the adherents of the ancient worship;
and his future greatness became the object of the hopes, the
prayers, and the predictions of the Pagans, in every province of
the empire. From the zeal and virtues of their royal proselyte,
they fondly expected the cure of every evil, and the restoration
of every blessing; and instead of disapproving of the ardor of
their pious wishes, Julian ingenuously confessed, that he was
ambitious to attain a situation in which he might be useful to
his country and to his religion. But this religion was viewed
with a hostile eye by the successor of Constantine, whose
capricious passions altercately saved and threatened the life of
Julian. The arts of magic and divination were strictly prohibited
under a despotic government, which condescended to fear them; and
if the Pagans were reluctantly indulged in the exercise of their
superstition, the rank of Julian would have excepted him from the
general toleration. The apostate soon became the presumptive heir
of the monarchy, and his death could alone have appeased the just
apprehensions of the Christians.\textsuperscript{28} But the young prince, who
aspired to the glory of a hero rather than of a martyr, consulted
his safety by dissembling his religion; and the easy temper of
polytheism permitted him to join in the public worship of a sect
which he inwardly despised. Libanius has considered the hypocrisy
of his friend as a subject, not of censure, but of praise. “As
the statues of the gods,” says that orator, “which have been
defiled with filth, are again placed in a magnificent temple, so
the beauty of truth was seated in the mind of Julian, after it
had been purified from the errors and follies of his education.
His sentiments were changed; but as it would have been dangerous
to have avowed his sentiments, his conduct still continued the
same. Very different from the ass in Æsop, who disguised himself
with a lion’s hide, our lion was obliged to conceal himself under
the skin of an ass; and, while he embraced the dictates of
reason, to obey the laws of prudence and necessity.”\textsuperscript{29} The
dissimulation of Julian lasted about ten years, from his secret
initiation at Ephesus to the beginning of the civil war; when he
declared himself at once the implacable enemy of Christ and of
Constantius. This state of constraint might contribute to
strengthen his devotion; and as soon as he had satisfied the
obligation of assisting, on solemn festivals, at the assemblies
of the Christians, Julian returned, with the impatience of a
lover, to burn his free and voluntary incense on the domestic
chapels of Jupiter and Mercury. But as every act of dissimulation
must be painful to an ingenuous spirit, the profession of
Christianity increased the aversion of Julian for a religion
which oppressed the freedom of his mind, and compelled him to
hold a conduct repugnant to the noblest attributes of human
nature, sincerity and courage.

\pagenote[27]{Libanius, Orat. Parent. c. x. p. 233, 234. Gallus
had some reason to suspect the secret apostasy of his brother;
and in a letter, which may be received as genuine, he exhorts
Julian to adhere to the religion of their \textit{ancestors;} an
argument which, as it should seem, was not yet perfectly ripe.
See Julian, Op. p. 454, and Hist. de Jovien tom ii. p. 141.}

\pagenote[28]{Gregory, (iii. p. 50,) with inhuman zeal, censures
Constantius for paring the infant apostate. His French translator
(p. 265) cautiously observes, that such expressions must not be
prises à la lettre.}

\pagenote[29]{Libanius, Orat. Parental. c ix. p. 233.}

\section{Part \thesection.}

The inclination of Julian might prefer the gods of Homer, and of
the Scipios, to the new faith, which his uncle had established in
the Roman empire; and in which he himself had been sanctified by
the sacrament of baptism. But, as a philosopher, it was incumbent
on him to justify his dissent from Christianity, which was
supported by the number of its converts, by the chain of
prophecy, the splendor of miracles, and the weight of evidence.
The elaborate work,\textsuperscript{30} which he composed amidst the preparations
of the Persian war, contained the substance of those arguments
which he had long revolved in his mind. Some fragments have been
transcribed and preserved, by his adversary, the vehement Cyril
of Alexandria;\textsuperscript{31} and they exhibit a very singular mixture of wit
and learning, of sophistry and fanaticism. The elegance of the
style and the rank of the author, recommended his writings to the
public attention;\textsuperscript{32} and in the impious list of the enemies of
Christianity, the celebrated name of Porphyry was effaced by the
superior merit or reputation of Julian. The minds of the faithful
were either seduced, or scandalized, or alarmed; and the pagans,
who sometimes presumed to engage in the unequal dispute, derived,
from the popular work of their Imperial missionary, an
inexhaustible supply of fallacious objections. But in the
assiduous prosecution of these theological studies, the emperor
of the Romans imbibed the illiberal prejudices and passions of a
polemic divine. He contracted an irrevocable obligation to
maintain and propagate his religious opinions; and whilst he
secretly applauded the strength and dexterity with which he
wielded the weapons of controversy, he was tempted to distrust
the sincerity, or to despise the understandings, of his
antagonists, who could obstinately resist the force of reason and
eloquence.

\pagenote[30]{Fabricius (Biblioth. Græc. l. v. c. viii, p. 88-90)
and Lardner (Heathen Testimonies, vol. iv. p. 44-47) have
accurately compiled all that can now be discovered of Julian’s
work against the Christians.}

\pagenote[31]{About seventy years after the death of Julian, he
executed a task which had been feebly attempted by Philip of
Side, a prolix and contemptible writer. Even the work of Cyril
has not entirely satisfied the most favorable judges; and the
Abbé de la Bleterie (Preface a l’Hist. de Jovien, p. 30, 32)
wishes that some \textit{theologien philosophe} (a strange centaur)
would undertake the refutation of Julian.}

\pagenote[32]{Libanius, (Orat. Parental. c. lxxxvii. p. 313,) who
has been suspected of assisting his friend, prefers this divine
vindication (Orat. ix in necem Julian. p. 255, edit. Morel.) to
the writings of Porphyry. His judgment may be arraigned,
(Socrates, l. iii. c. 23,) but Libanius cannot be accused of
flattery to a dead prince.}

The Christians, who beheld with horror and indignation the
apostasy of Julian, had much more to fear from his power than
from his arguments. The pagans, who were conscious of his fervent
zeal, expected, perhaps with impatience, that the flames of
persecution should be immediately kindled against the enemies of
the gods; and that the ingenious malice of Julian would invent
some cruel refinements of death and torture which had been
unknown to the rude and inexperienced fury of his predecessors.
But the hopes, as well as the fears, of the religious factions
were apparently disappointed, by the prudent humanity of a
prince,\textsuperscript{33} who was careful of his own fame, of the public peace,
and of the rights of mankind. Instructed by history and
reflection, Julian was persuaded, that if the diseases of the
body may sometimes be cured by salutary violence, neither steel
nor fire can eradicate the erroneous opinions of the mind. The
reluctant victim may be dragged to the foot of the altar; but the
heart still abhors and disclaims the sacrilegious act of the
hand. Religious obstinacy is hardened and exasperated by
oppression; and, as soon as the persecution subsides, those who
have yielded are restored as penitents, and those who have
resisted are honored as saints and martyrs. If Julian adopted the
unsuccessful cruelty of Diocletian and his colleagues, he was
sensible that he should stain his memory with the name of a
tyrant, and add new glories to the Catholic church, which had
derived strength and increase from the severity of the pagan
magistrates. Actuated by these motives, and apprehensive of
disturbing the repose of an unsettled reign, Julian surprised the
world by an edict, which was not unworthy of a statesman, or a
philosopher. He extended to all the inhabitants of the Roman
world the benefits of a free and equal toleration; and the only
hardship which he inflicted on the Christians, was to deprive
them of the power of tormenting their fellow-subjects, whom they
stigmatized with the odious titles of idolaters and heretics. The
pagans received a gracious permission, or rather an express
order, to open All their temples;\textsuperscript{34} and they were at once
delivered from the oppressive laws, and arbitrary vexations,
which they had sustained under the reign of Constantine, and of
his sons. At the same time the bishops and clergy, who had been
banished by the Arian monarch, were recalled from exile, and
restored to their respective churches; the Donatists, the
Novatians, the Macedonians, the Eunomians, and those who, with a
more prosperous fortune, adhered to the doctrine of the Council
of Nice. Julian, who understood and derided their theological
disputes, invited to the palace the leaders of the hostile sects,
that he might enjoy the agreeable spectacle of their furious
encounters. The clamor of controversy sometimes provoked the
emperor to exclaim, “Hear me! the Franks have heard me, and the
Alemanni;” but he soon discovered that he was now engaged with
more obstinate and implacable enemies; and though he exerted the
powers of oratory to persuade them to live in concord, or at
least in peace, he was perfectly satisfied, before he dismissed
them from his presence, that he had nothing to dread from the
union of the Christians. The impartial Ammianus has ascribed this
affected clemency to the desire of fomenting the intestine
divisions of the church, and the insidious design of undermining
the foundations of Christianity, was inseparably connected with
the zeal which Julian professed, to restore the ancient religion
of the empire.\textsuperscript{35}

\pagenote[33]{Libanius (Orat. Parent. c. lviii. p. 283, 284) has
eloquently explained the tolerating principles and conduct of his
Imperial friend. In a very remarkable epistle to the people of
Bostra, Julian himself (Epist. lii.) professes his moderation,
and betrays his zeal, which is acknowledged by Ammianus, and
exposed by Gregory (Orat. iii. p.72)}

\pagenote[34]{In Greece the temples of Minerva were opened by his
express command, before the death of Constantius, (Liban. Orat.
Parent. c. 55, p. 280;) and Julian declares himself a Pagan in
his public manifesto to the Athenians. This unquestionable
evidence may correct the hasty assertion of Ammianus, who seems
to suppose Constantinople to be the place where he discovered his
attachment to the gods}

\pagenote[35]{Ammianus, xxii. 5. Sozomen, l. v. c. 5. Bestia
moritur, tranquillitas redit.... omnes episcopi qui de propriis
sedibus fuerant exterminati per indulgentiam novi principis ad
acclesias redeunt. Jerom. adversus Luciferianos, tom. ii. p. 143.
Optatus accuses the Donatists for owing their safety to an
apostate, (l. ii. c. 16, p. 36, 37, edit. Dupin.)}

As soon as he ascended the throne, he assumed, according to the
custom of his predecessors, the character of supreme pontiff; not
only as the most honorable title of Imperial greatness, but as a
sacred and important office; the duties of which he was resolved
to execute with pious diligence. As the business of the state
prevented the emperor from joining every day in the public
devotion of his subjects, he dedicated a domestic chapel to his
tutelar deity the Sun; his gardens were filled with statues and
altars of the gods; and each apartment of the palace displaced
the appearance of a magnificent temple. Every morning he saluted
the parent of light with a sacrifice; the blood of another victim
was shed at the moment when the Sun sunk below the horizon; and
the Moon, the Stars, and the Genii of the night received their
respective and seasonable honors from the indefatigable devotion
of Julian. On solemn festivals, he regularly visited the temple
of the god or goddess to whom the day was peculiarly consecrated,
and endeavored to excite the religion of the magistrates and
people by the example of his own zeal. Instead of maintaining the
lofty state of a monarch, distinguished by the splendor of his
purple, and encompassed by the golden shields of his guards,
Julian solicited, with respectful eagerness, the meanest offices
which contributed to the worship of the gods. Amidst the sacred
but licentious crowd of priests, of inferior ministers, and of
female dancers, who were dedicated to the service of the temple,
it was the business of the emperor to bring the wood, to blow the
fire, to handle the knife, to slaughter the victim, and,
thrusting his bloody hands into the bowels of the expiring
animal, to draw forth the heart or liver, and to read, with the
consummate skill of an haruspex, imaginary signs of future
events. The wisest of the Pagans censured this extravagant
superstition, which affected to despise the restraints of
prudence and decency. Under the reign of a prince, who practised
the rigid maxims of economy, the expense of religious worship
consumed a very large portion of the revenue; a constant supply
of the scarcest and most beautiful birds was transported from
distant climates, to bleed on the altars of the gods; a hundred
oxen were frequently sacrificed by Julian on one and the same
day; and it soon became a popular jest, that if he should return
with conquest from the Persian war, the breed of horned cattle
must infallibly be extinguished. Yet this expense may appear
inconsiderable, when it is compared with the splendid presents
which were offered either by the hand, or by order, of the
emperor, to all the celebrated places of devotion in the Roman
world; and with the sums allotted to repair and decorate the
ancient temples, which had suffered the silent decay of time, or
the recent injuries of Christian rapine. Encouraged by the
example, the exhortations, the liberality, of their pious
sovereign, the cities and families resumed the practice of their
neglected ceremonies. “Every part of the world,” exclaims
Libanius, with devout transport, “displayed the triumph of
religion; and the grateful prospect of flaming altars, bleeding
victims, the smoke of incense, and a solemn train of priests and
prophets, without fear and without danger. The sound of prayer
and of music was heard on the tops of the highest mountains; and
the same ox afforded a sacrifice for the gods, and a supper for
their joyous votaries.”\textsuperscript{36}

\pagenote[36]{The restoration of the Pagan worship is described
by Julian, (Misopogon, p. 346,) Libanius, (Orat. Parent. c. 60,
p. 286, 287, and Orat. Consular. ad Julian. p. 245, 246, edit.
Morel.,) Ammianus, (xxii. 12,) and Gregory Nazianzen, (Orat. iv.
p. 121.) These writers agree in the essential, and even minute,
facts; but the different lights in which they view the extreme
devotion of Julian, are expressive of the gradations of
self-applause, passionate admiration, mild reproof, and partial
invective.}

But the genius and power of Julian were unequal to the enterprise
of restoring a religion which was destitute of theological
principles, of moral precepts, and of ecclesiastical discipline;
which rapidly hastened to decay and dissolution, and was not
susceptible of any solid or consistent reformation. The
jurisdiction of the supreme pontiff, more especially after that
office had been united with the Imperial dignity, comprehended
the whole extent of the Roman empire. Julian named for his
vicars, in the several provinces, the priests and philosophers
whom he esteemed the best qualified to cooperate in the execution
of his great design; and his pastoral letters,\textsuperscript{37} if we may use
that name, still represent a very curious sketch of his wishes
and intentions. He directs, that in every city the sacerdotal
order should be composed, without any distinction of birth and
fortune, of those persons who were the most conspicuous for the
love of the gods, and of men. “If they are guilty,” continues he,
“of any scandalous offence, they should be censured or degraded
by the superior pontiff; but as long as they retain their rank,
they are entitled to the respect of the magistrates and people.
Their humility may be shown in the plainness of their domestic
garb; their dignity, in the pomp of holy vestments. When they are
summoned in their turn to officiate before the altar, they ought
not, during the appointed number of days, to depart from the
precincts of the temple; nor should a single day be suffered to
elapse, without the prayers and the sacrifice, which they are
obliged to offer for the prosperity of the state, and of
individuals. The exercise of their sacred functions requires an
immaculate purity, both of mind and body; and even when they are
dismissed from the temple to the occupations of common life, it
is incumbent on them to excel in decency and virtue the rest of
their fellow-citizens. The priest of the gods should never be
seen in theatres or taverns. His conversation should be chaste,
his diet temperate, his friends of honorable reputation; and if
he sometimes visits the Forum or the Palace, he should appear
only as the advocate of those who have vainly solicited either
justice or mercy. His studies should be suited to the sanctity of
his profession. Licentious tales, or comedies, or satires, must
be banished from his library, which ought solely to consist of
historical or philosophical writings; of history, which is
founded in truth, and of philosophy, which is connected with
religion. The impious opinions of the Epicureans and sceptics
deserve his abhorrence and contempt;\textsuperscript{38} but he should diligently
study the systems of Pythagoras, of Plato, and of the Stoics,
which unanimously teach that there \textit{are} gods; that the world is
governed by their providence; that their goodness is the source
of every temporal blessing; and that they have prepared for the
human soul a future state of reward or punishment.” The Imperial
pontiff inculcates, in the most persuasive language, the duties
of benevolence and hospitality; exhorts his inferior clergy to
recommend the universal practice of those virtues; promises to
assist their indigence from the public treasury; and declares his
resolution of establishing hospitals in every city, where the
poor should be received without any invidious distinction of
country or of religion. Julian beheld with envy the wise and
humane regulations of the church; and he very frankly confesses
his intention to deprive the Christians of the applause, as well
as advantage, which they had acquired by the exclusive practice
of charity and beneficence.\textsuperscript{39} The same spirit of imitation might
dispose the emperor to adopt several ecclesiastical institutions,
the use and importance of which were approved by the success of
his enemies. But if these imaginary plans of reformation had been
realized, the forced and imperfect copy would have been less
beneficial to Paganism, than honorable to Christianity.\textsuperscript{40} The
Gentiles, who peaceably followed the customs of their ancestors,
were rather surprised than pleased with the introduction of
foreign manners; and in the short period of his reign, Julian had
frequent occasions to complain of the want of fervor of his own
party.\textsuperscript{41}

\pagenote[37]{See Julian. Epistol. xlix. lxii. lxiii., and a long
and curious fragment, without beginning or end, (p. 288-305.) The
supreme pontiff derides the Mosaic history and the Christian
discipline, prefers the Greek poets to the Hebrew prophets, and
palliates, with the skill of a Jesuit the \textit{relative} worship of
images.}

\pagenote[38]{The exultation of Julian (p. 301) that these
impious sects and even their writings, are extinguished, may be
consistent enough with the sacerdotal character; but it is
unworthy of a philosopher to wish that any opinions and arguments
the most repugnant to his own should be concealed from the
knowledge of mankind.}

\pagenote[39]{Yet he insinuates, that the Christians, under the
pretence of charity, inveigled children from their religion and
parents, conveyed them on shipboard, and devoted those victims to
a life of poverty or pervitude in a remote country, (p. 305.) Had
the charge been proved it was his duty, not to complain, but to
punish.}

\pagenote[40]{Gregory Nazianzen is facetious, ingenious, and
argumentative, (Orat. iii. p. 101, 102, \&c.) He ridicules the
folly of such vain imitation; and amuses himself with inquiring,
what lessons, moral or theological, could be extracted from the
Grecian fables.}

\pagenote[41]{He accuses one of his pontiffs of a secret
confederacy with the Christian bishops and presbyters, (Epist.
lxii.) \&c. Epist. lxiii.}

The enthusiasm of Julian prompted him to embrace the friends of
Jupiter as his personal friends and brethren; and though he
partially overlooked the merit of Christian constancy, he admired
and rewarded the noble perseverance of those Gentiles who had
preferred the favor of the gods to that of the emperor.\textsuperscript{42} If
they cultivated the literature, as well as the religion, of the
Greeks, they acquired an additional claim to the friendship of
Julian, who ranked the Muses in the number of his tutelar
deities. In the religion which he had adopted, piety and learning
were almost synonymous;\textsuperscript{43} and a crowd of poets, of rhetoricians,
and of philosophers, hastened to the Imperial court, to occupy
the vacant places of the bishops, who had seduced the credulity
of Constantius. His successor esteemed the ties of common
initiation as far more sacred than those of consanguinity; he
chose his favorites among the sages, who were deeply skilled in
the occult sciences of magic and divination; and every impostor,
who pretended to reveal the secrets of futurity, was assured of
enjoying the present hour in honor and affluence.\textsuperscript{44} Among the
philosophers, Maximus obtained the most eminent rank in the
friendship of his royal disciple, who communicated, with
unreserved confidence, his actions, his sentiments, and his
religious designs, during the anxious suspense of the civil war.\textsuperscript{45}
As soon as Julian had taken possession of the palace of
Constantinople, he despatched an honorable and pressing
invitation to Maximus, who then resided at Sardes in Lydia, with
Chrysanthius, the associate of his art and studies. The prudent
and superstitious Chrysanthius refused to undertake a journey
which showed itself, according to the rules of divination, with
the most threatening and malignant aspect: but his companion,
whose fanaticism was of a bolder cast, persisted in his
interrogations, till he had extorted from the gods a seeming
consent to his own wishes, and those of the emperor. The journey
of Maximus through the cities of Asia displayed the triumph of
philosophic vanity; and the magistrates vied with each other in
the honorable reception which they prepared for the friend of
their sovereign. Julian was pronouncing an oration before the
senate, when he was informed of the arrival of Maximus. The
emperor immediately interrupted his discourse, advanced to meet
him, and after a tender embrace, conducted him by the hand into
the midst of the assembly; where he publicly acknowledged the
benefits which he had derived from the instructions of the
philosopher. Maximus,\textsuperscript{46} who soon acquired the confidence, and
influenced the councils of Julian, was insensibly corrupted by
the temptations of a court. His dress became more splendid, his
demeanor more lofty, and he was exposed, under a succeeding
reign, to a disgraceful inquiry into the means by which the
disciple of Plato had accumulated, in the short duration of his
favor, a very scandalous proportion of wealth. Of the other
philosophers and sophists, who were invited to the Imperial
residence by the choice of Julian, or by the success of Maximus,
few were able to preserve their innocence or their reputation.
The liberal gifts of money, lands, and houses, were insufficient
to satiate their rapacious avarice; and the indignation of the
people was justly excited by the remembrance of their abject
poverty and disinterested professions. The penetration of Julian
could not always be deceived: but he was unwilling to despise the
characters of those men whose talents deserved his esteem: he
desired to escape the double reproach of imprudence and
inconstancy; and he was apprehensive of degrading, in the eyes of
the profane, the honor of letters and of religion.\textsuperscript{47} \textsuperscript{48}

\pagenote[42]{He praises the fidelity of Callixene, priestess of
Ceres, who had been twice as constant as Penelope, and rewards
her with the priesthood of the Phrygian goddess at Pessinus,
(Julian. Epist. xxi.) He applauds the firmness of Sopater of
Hierapolis, who had been repeatedly pressed by Constantius and
Gallus to \textit{apostatize}, (Epist. xxvii p. 401.)}

\pagenote[43]{Orat. Parent. c. 77, p. 202. The same sentiment is
frequently inculcated by Julian, Libanius, and the rest of their
party.}

\pagenote[44]{The curiosity and credulity of the emperor, who
tried every mode of divination, are fairly exposed by Ammianus,
xxii. 12.}

\pagenote[45]{Julian. Epist. xxxviii. Three other epistles, (xv.
xvi. xxxix.,) in the same style of friendship and confidence, are
addressed to the philosopher Maximus.}

\pagenote[46]{Eunapius (in Maximo, p. 77, 78, 79, and in
Chrysanthio, p. 147, 148) has minutely related these anecdotes,
which he conceives to be the most important events of the age.
Yet he fairly confesses the frailty of Maximus. His reception at
Constantinople is described by Libanius (Orat. Parent. c. 86, p.
301) and Ammianus, (xxii. 7.) * Note: Eunapius wrote a
continuation of the History of Dexippus. Some valuable fragments
of this work have been recovered by M. Mai, and reprinted in
Niebuhr’s edition of the Byzantine Historians.—M.}

\pagenote[47]{Chrysanthius, who had refused to quit Lydia, was
created high priest of the province. His cautious and temperate
use of power secured him after the revolution; and he lived in
peace, while Maximus, Priscus, \&c., were persecuted by the
Christian ministers. See the adventures of those fanatic
sophists, collected by Brucker, tom ii. p. 281-293.}

\pagenote[48]{Sec Libanius (Orat. Parent. c. 101, 102, p. 324,
325, 326) and Eunapius, (Vit. Sophist. in Proæresio, p. 126.)
Some students, whose expectations perhaps were groundless, or
extravagant, retired in disgust, (Greg. Naz. Orat. iv. p. 120.)
It is strange that we should not be able to contradict the title
of one of Tillemont’s chapters, (Hist. des Empereurs, tom. iv. p.
960,) “La Cour de Julien est pleine de philosphes et de gens
perdus.”}

The favor of Julian was almost equally divided between the
Pagans, who had firmly adhered to the worship of their ancestors,
and the Christians, who prudently embraced the religion of their
sovereign. The acquisition of new proselytes\textsuperscript{49} gratified the
ruling passions of his soul, superstition and vanity; and he was
heard to declare, with the enthusiasm of a missionary, that if he
could render each individual richer than Midas, and every city
greater than Babylon, he should not esteem himself the benefactor
of mankind, unless, at the same time, he could reclaim his
subjects from their impious revolt against the immortal gods.\textsuperscript{50}
A prince who had studied human nature, and who possessed the
treasures of the Roman empire, could adapt his arguments, his
promises, and his rewards, to every order of Christians;\textsuperscript{51} and
the merit of a seasonable conversion was allowed to supply the
defects of a candidate, or even to expiate the guilt of a
criminal. As the army is the most forcible engine of absolute
power, Julian applied himself, with peculiar diligence, to
corrupt the religion of his troops, without whose hearty
concurrence every measure must be dangerous and unsuccessful; and
the natural temper of soldiers made this conquest as easy as it
was important. The legions of Gaul devoted themselves to the
faith, as well as to the fortunes, of their victorious leader;
and even before the death of Constantius, he had the satisfaction
of announcing to his friends, that they assisted with fervent
devotion, and voracious appetite, at the sacrifices, which were
repeatedly offered in his camp, of whole hecatombs of fat oxen.\textsuperscript{52}
The armies of the East, which had been trained under the
standard of the cross, and of Constantius, required a more artful
and expensive mode of persuasion. On the days of solemn and
public festivals, the emperor received the homage, and rewarded
the merit, of the troops. His throne of state was encircled with
the military ensigns of Rome and the republic; the holy name of
Christ was erased from the \textit{Labarum;} and the symbols of war, of
majesty, and of pagan superstition, were so dexterously blended,
that the faithful subject incurred the guilt of idolatry, when he
respectfully saluted the person or image of his sovereign. The
soldiers passed successively in review; and each of them, before
he received from the hand of Julian a liberal donative,
proportioned to his rank and services, was required to cast a few
grains of incense into the flame which burnt upon the altar. Some
Christian confessors might resist, and others might repent; but
the far greater number, allured by the prospect of gold, and awed
by the presence of the emperor, contracted the criminal
engagement; and their future perseverance in the worship of the
gods was enforced by every consideration of duty and of interest.

By the frequent repetition of these arts, and at the expense of
sums which would have purchased the service of half the nations
of Scythia, Julian gradually acquired for his troops the
imaginary protection of the gods, and for himself the firm and
effectual support of the Roman legions.\textsuperscript{53} It is indeed more than
probable, that the restoration and encouragement of Paganism
revealed a multitude of pretended Christians, who, from motives
of temporal advantage, had acquiesced in the religion of the
former reign; and who afterwards returned, with the same
flexibility of conscience, to the faith which was professed by
the successors of Julian.

\pagenote[49]{Under the reign of Lewis XIV. his subjects of every
rank aspired to the glorious title of \textit{Convertisseur}, expressive
of their zea and success in making proselytes. The word and the
idea are growing obsolete in France may they never be introduced
into England.}

\pagenote[50]{See the strong expressions of Libanius, which were
probably those of Julian himself, (Orat. Parent. c. 59, p. 285.)}

\pagenote[51]{When Gregory Nazianzen (Orat. x. p. 167) is
desirous to magnify the Christian firmness of his brother
Cæsarius, physician to the Imperial court, he owns that Cæsarius
disputed with a formidable adversary. In his invectives he
scarcely allows any share of wit or courage to the apostate.}

\pagenote[52]{Julian, Epist. xxxviii. Ammianus, xxii. 12. Adeo ut
in dies pæne singulos milites carnis distentiore sagina
victitantes incultius, potusque aviditate correpti, humeris
impositi transeuntium per plateas, ex publicis ædibus..... ad sua
diversoria portarentur. The devout prince and the indignant
historian describe the same scene; and in Illyricum or Antioch,
similar causes must have produced similar effects.}

\pagenote[53]{Gregory (Orat. iii. p. 74, 75, 83-86) and Libanius,
(Orat. Parent. c. lxxxi. lxxxii. p. 307, 308,). The sophist owns
and justifies the expense of these military conversions.}

While the devout monarch incessantly labored to restore and
propagate the religion of his ancestors, he embraced the
extraordinary design of rebuilding the temple of Jerusalem. In a
public epistle\textsuperscript{54} to the nation or community of the Jews,
dispersed through the provinces, he pities their misfortunes,
condemns their oppressors, praises their constancy, declares
himself their gracious protector, and expresses a pious hope,
that after his return from the Persian war, he may be permitted
to pay his grateful vows to the Almighty in his holy city of
Jerusalem. The blind superstition, and abject slavery, of those
unfortunate exiles, must excite the contempt of a philosophic
emperor; but they deserved the friendship of Julian, by their
implacable hatred of the Christian name. The barren synagogue
abhorred and envied the fecundity of the rebellious church; the
power of the Jews was not equal to their malice; but their
gravest rabbis approved the private murder of an apostate;\textsuperscript{55} and
their seditious clamors had often awakened the indolence of the
Pagan magistrates. Under the reign of Constantine, the Jews
became the subjects of their revolted children nor was it long
before they experienced the bitterness of domestic tyranny. The
civil immunities which had been granted, or confirmed, by
Severus, were gradually repealed by the Christian princes; and a
rash tumult, excited by the Jews of Palestine,\textsuperscript{56} seemed to
justify the lucrative modes of oppression which were invented by
the bishops and eunuchs of the court of Constantius. The Jewish
patriarch, who was still permitted to exercise a precarious
jurisdiction, held his residence at Tiberias;\textsuperscript{57} and the
neighboring cities of Palestine were filled with the remains of a
people who fondly adhered to the promised land. But the edict of
Hadrian was renewed and enforced; and they viewed from afar the
walls of the holy city, which were profaned in their eyes by the
triumph of the cross and the devotion of the Christians.\textsuperscript{58}

\pagenote[54]{Julian’s epistle (xxv.) is addressed to the
community of the Jews. Aldus (Venet. 1499) has branded it with
an; but this stigma is justly removed by the subsequent editors,
Petavius and Spanheim. This epistle is mentioned by Sozomen, (l.
v. c. 22,) and the purport of it is confirmed by Gregory, (Orat.
iv. p. 111.) and by Julian himself (Fragment. p. 295.)}

\pagenote[55]{The Misnah denounced death against those who
abandoned the foundation. The judgment of zeal is explained by
Marsham (Canon. Chron. p. 161, 162, edit. fol. London, 1672) and
Basnage, (Hist. des Juifs, tom. viii. p. 120.) Constantine made a
law to protect Christian converts from Judaism. Cod. Theod. l.
xvi. tit. viii. leg. 1. Godefroy, tom. vi. p. 215.}

\pagenote[56]{Et interea (during the civil war of Magnentius)
Judæorum seditio, qui Patricium, nefarie in regni speciem
sustulerunt, oppressa. Aurelius Victor, in Constantio, c. xlii.
See Tillemont, Hist. des Empereurs, tom. iv. p. 379, in 4to.}

\pagenote[57]{The city and synagogue of Tiberias are curiously
described by Reland. Palestin. tom. ii. p. 1036-1042.}

\pagenote[58]{Basnage has fully illustrated the state of the Jews
under Constantine and his successors, (tom. viii. c. iv. p.
111-153.)}

\section{Part \thesection.}

In the midst of a rocky and barren country, the walls of
Jerusalem\textsuperscript{59} enclosed the two mountains of Sion and Acra, within
an oval figure of about three English miles.\textsuperscript{60} Towards the
south, the upper town, and the fortress of David, were erected on
the lofty ascent of Mount Sion: on the north side, the buildings
of the lower town covered the spacious summit of Mount Acra; and
a part of the hill, distinguished by the name of Moriah, and
levelled by human industry, was crowned with the stately temple
of the Jewish nation. After the final destruction of the temple
by the arms of Titus and Hadrian, a ploughshare was drawn over
the consecrated ground, as a sign of perpetual interdiction. Sion
was deserted; and the vacant space of the lower city was filled
with the public and private edifices of the Ælian colony, which
spread themselves over the adjacent hill of Calvary. The holy
places were polluted with mountains of idolatry; and, either from
design or accident, a chapel was dedicated to Venus, on the spot
which had been sanctified by the death and resurrection of
Christ.\textsuperscript{61} \textsuperscript{6111} Almost three hundred years after those stupendous
events, the profane chapel of Venus was demolished by the order
of Constantine; and the removal of the earth and stones revealed
the holy sepulchre to the eyes of mankind. A magnificent church
was erected on that mystic ground, by the first Christian
emperor; and the effects of his pious munificence were extended
to every spot which had been consecrated by the footstep of
patriarchs, of prophets, and of the Son of God.\textsuperscript{62}

\pagenote[59]{Reland (Palestin. l. i. p. 309, 390, l. iii. p.
838) describes, with learning and perspicuity, Jerusalem, and the
face of the adjacent country.}

\pagenote[60]{I have consulted a rare and curious treatise of M.
D’Anville, (sur l’Ancienne Jerusalem, Paris, 1747, p. 75.) The
circumference of the ancient city (Euseb. Preparat. Evangel. l.
ix. c. 36) was 27 stadia, or 2550 \textit{toises}. A plan, taken on the
spot, assigns no more than 1980 for the modern town. The circuit
is defined by natural landmarks, which cannot be mistaken or
removed.}

\pagenote[61]{See two curious passages in Jerom, (tom. i. p. 102,
tom. vi. p. 315,) and the ample details of Tillemont, (Hist, des
Empereurs, tom. i. p. 569. tom. ii. p. 289, 294, 4to edition.)}

\pagenote[6111]{On the site of the Holy Sepulchre, compare the
chapter in Professor Robinson’s Travels in Palestine, which has
renewed the old controversy with great vigor. To me, this temple
of Venus, said to have been erected by Hadrian to insult the
Christians, is not the least suspicious part of the whole
legend.-M. 1845.}

\pagenote[62]{Eusebius in Vit. Constantin. l. iii. c. 25-47,
51-53. The emperor likewise built churches at Bethlem, the Mount
of Olives, and the oa of Mambre. The holy sepulchre is described
by Sandys, (Travels, p. 125-133,) and curiously delineated by Le
Bruyn, (Voyage au Levant, p. 28-296.)}

The passionate desire of contemplating the original monuments of
their redemption attracted to Jerusalem a successive crowd of
pilgrims, from the shores of the Atlantic Ocean, and the most
distant countries of the East;\textsuperscript{63} and their piety was authorized
by the example of the empress Helena, who appears to have united
the credulity of age with the warm feelings of a recent
conversion. Sages and heroes, who have visited the memorable
scenes of ancient wisdom or glory, have confessed the inspiration
of the genius of the place;\textsuperscript{64} and the Christian who knelt before
the holy sepulchre, ascribed his lively faith, and his fervent
devotion, to the more immediate influence of the Divine Spirit.
The zeal, perhaps the avarice, of the clergy of Jerusalem,
cherished and multiplied these beneficial visits. They fixed, by
unquestionable tradition, the scene of each memorable event. They
exhibited the instruments which had been used in the passion of
Christ; the nails and the lance that had pierced his hands, his
feet, and his side; the crown of thorns that was planted on his
head; the pillar at which he was scourged; and, above all, they
showed the cross on which he suffered, and which was dug out of
the earth in the reign of those princes, who inserted the symbol
of Christianity in the banners of the Roman legions.\textsuperscript{65} Such
miracles as seemed necessary to account for its extraordinary
preservation, and seasonable discovery, were gradually propagated
without opposition. The custody of the \textit{true cross}, which on
Easter Sunday was solemnly exposed to the people, was intrusted
to the bishop of Jerusalem; and he alone might gratify the
curious devotion of the pilgrims, by the gift of small pieces,
which they encased in gold or gems, and carried away in triumph
to their respective countries. But as this gainful branch of
commerce must soon have been annihilated, it was found convenient
to suppose, that the marvelous wood possessed a secret power of
vegetation; and that its substance, though continually
diminished, still remained entire and unimpaired.\textsuperscript{66} It might
perhaps have been expected, that the influence of the place and
the belief of a perpetual miracle, should have produced some
salutary effects on the morals, as well as on the faith, of the
people. Yet the most respectable of the ecclesiastical writers
have been obliged to confess, not only that the streets of
Jerusalem were filled with the incessant tumult of business and
pleasure,\textsuperscript{67} but that every species of vice—adultery, theft,
idolatry, poisoning, murder—was familiar to the inhabitants of
the holy city.\textsuperscript{68} The wealth and preëminence of the church of
Jerusalem excited the ambition of Arian, as well as orthodox,
candidates; and the virtues of Cyril, who, since his death, has
been honored with the title of Saint, were displayed in the
exercise, rather than in the acquisition, of his episcopal
dignity.\textsuperscript{69}

\pagenote[63]{The Itinerary from Bourdeaux to Jerusalem was
composed in the year 333, for the use of pilgrims; among whom
Jerom (tom. i. p. 126) mentions the Britons and the Indians. The
causes of this superstitious fashion are discussed in the learned
and judicious preface of Wesseling. (Itinarar. p. 537-545.)
——Much curious information on this subject is collected in the
first chapter of Wilken, Geschichte der Kreuzzüge.—M.}

\pagenote[64]{Cicero (de Finibus, v. 1) has beautifully expressed
the common sense of mankind.}

\pagenote[65]{Baronius (Annal. Eccles. A. D. 326, No. 42-50) and
Tillemont (Mém. Eccles. tom. xii. p. 8-16) are the historians and
champions of the miraculous \textit{invention} of the cross, under the
reign of Constantine. Their oldest witnesses are Paulinus,
Sulpicius Severus, Rufinus, Ambrose, and perhaps Cyril of
Jerusalem. The silence of Eusebius, and the Bourdeaux pilgrim,
which satisfies those who think perplexes those who believe. See
Jortin’s sensible remarks, vol. ii. p 238-248.}

\pagenote[66]{This multiplication is asserted by Paulinus,
(Epist. xxxvi. See Dupin. Bibliot. Eccles. tom. iii. p. 149,) who
seems to have improved a rhetorical flourish of Cyril into a real
fact. The same supernatural privilege must have been communicated
to the Virgin’s milk, (Erasmi Opera, tom. i. p. 778, Lugd. Batav.
1703, in Colloq. de Peregrinat. Religionis ergo,) saints’ heads,
\&c. and other relics, which are repeated in so many different
churches. * Note: Lord Mahon, in a memoir read before the Society
of Antiquaries, (Feb. 1831,) has traced in a brief but
interesting manner, the singular adventures of the “true” cross.
It is curious to inquire, what authority we have, except of
\textit{late} tradition, for the \textit{Hill} of Calvary. There is none in the
sacred writings; the uniform use of the common word, instead of
any word expressing assent or acclivity, is against the
notion.—M.}

\pagenote[67]{Jerom, (tom. i. p. 103,) who resided in the
neighboring village of Bethlem, describes the vices of Jerusalem
from his personal experience.}

\pagenote[68]{Gregor. Nyssen, apud Wesseling, p. 539. The whole
epistle, which condemns either the use or the abuse of religious
pilgrimage, is painful to the Catholic divines, while it is dear
and familiar to our Protestant polemics.}

\pagenote[69]{He renounced his orthodox ordination, officiated as
a deacon, and was re-ordained by the hands of the Arians. But
Cyril afterwards changed with the times, and prudently conformed
to the Nicene faith. Tillemont, (Mém. Eccles. tom. viii.,) who
treats his memory with tenderness and respect, has thrown his
virtues into the text, and his faults into the notes, in decent
obscurity, at the end of the volume.}

The vain and ambitious mind of Julian might aspire to restore the
ancient glory of the temple of Jerusalem.\textsuperscript{70} As the Christians
were firmly persuaded that a sentence of everlasting destruction
had been pronounced against the whole fabric of the Mosaic law,
the Imperial sophist would have converted the success of his
undertaking into a specious argument against the faith of
prophecy, and the truth of revelation.\textsuperscript{71} He was displeased with
the spiritual worship of the synagogue; but he approved the
institutions of Moses, who had not disdained to adopt many of the
rites and ceremonies of Egypt.\textsuperscript{72} The local and national deity of
the Jews was sincerely adored by a polytheist, who desired only
to multiply the number of the gods;\textsuperscript{73} and such was the appetite
of Julian for bloody sacrifice, that his emulation might be
excited by the piety of Solomon, who had offered, at the feast of
the dedication, twenty-two thousand oxen, and one hundred and
twenty thousand sheep.\textsuperscript{74} These considerations might influence
his designs; but the prospect of an immediate and important
advantage would not suffer the impatient monarch to expect the
remote and uncertain event of the Persian war. He resolved to
erect, without delay, on the commanding eminence of Moriah, a
stately temple, which might eclipse the splendor of the church of
the resurrection on the adjacent hill of Calvary; to establish an
order of priests, whose interested zeal would detect the arts,
and resist the ambition, of their Christian rivals; and to invite
a numerous colony of Jews, whose stern fanaticism would be always
prepared to second, and even to anticipate, the hostile measures
of the Pagan government. Among the friends of the emperor (if the
names of emperor, and of friend, are not incompatible) the first
place was assigned, by Julian himself, to the virtuous and
learned Alypius.\textsuperscript{75} The humanity of Alypius was tempered by
severe justice and manly fortitude; and while he exercised his
abilities in the civil administration of Britain, he imitated, in
his poetical compositions, the harmony and softness of the odes
of Sappho. This minister, to whom Julian communicated, without
reserve, his most careless levities, and his most serious
counsels, received an extraordinary commission to restore, in its
pristine beauty, the temple of Jerusalem; and the diligence of
Alypius required and obtained the strenuous support of the
governor of Palestine. At the call of their great deliverer, the
Jews, from all the provinces of the empire, assembled on the holy
mountain of their fathers; and their insolent triumph alarmed and
exasperated the Christian inhabitants of Jerusalem. The desire of
rebuilding the temple has in every age been the ruling passion of
the children of Israel. In this propitious moment the men forgot
their avarice, and the women their delicacy; spades and pickaxes
of silver were provided by the vanity of the rich, and the
rubbish was transported in mantles of silk and purple. Every
purse was opened in liberal contributions, every hand claimed a
share in the pious labor, and the commands of a great monarch
were executed by the enthusiasm of a whole people.\textsuperscript{76}

\pagenote[70]{Imperii sui memoriam magnitudine operum gestiens
propagare Ammian. xxiii. 1. The temple of Jerusalem had been
famous even among the Gentiles. \textit{They} had many temples in each
city, (at Sichem five, at Gaza eight, at Rome four hundred and
twenty-four;) but the wealth and religion of the Jewish nation
was centred in one spot.}

\pagenote[71]{The secret intentions of Julian are revealed by the
late bishop of Gloucester, the learned and dogmatic Warburton;
who, with the authority of a theologian, prescribes the motives
and conduct of the Supreme Being. The discourse entitled \textit{Julian}
(2d edition, London, 1751) is strongly marked with all the
peculiarities which are imputed to the Warburtonian school.}

\pagenote[72]{I shelter myself behind Maimonides, Marsham,
Spencer, Le Clerc, Warburton, \&c., who have fairly derided the
fears, the folly, and the falsehood of some superstitious
divines. See Divine Legation, vol. iv. p. 25, \&c.}

\pagenote[73]{Julian (Fragment. p. 295) respectfully styles him,
and mentions him elsewhere (Epist. lxiii.) with still higher
reverence. He doubly condemns the Christians for believing, and
for renouncing, the religion of the Jews. Their Deity was a
\textit{true}, but not the \textit{only}, God Apul Cyril. l. ix. p. 305, 306.}

\pagenote[74]{1 Kings, viii. 63. 2 Chronicles, vii. 5. Joseph.
Antiquitat. Judaic. l. viii. c. 4, p. 431, edit. Havercamp. As
the blood and smoke of so many hecatombs might be inconvenient,
Lightfoot, the Christian Rabbi, removes them by a miracle. Le
Clerc (ad loca) is bold enough to suspect to fidelity of the
numbers. * Note: According to the historian Kotobeddym, quoted by
Burckhardt, (Travels in Arabia, p. 276,) the Khalif Mokteder
sacrificed, during his pilgrimage to Mecca, in the year of the
Hejira 350, forty thousand camels and cows, and fifty thousand
sheep. Barthema describes thirty thousand oxen slain, and their
carcasses given to the poor. Quarterly Review, xiii.p.39—M.}

\pagenote[75]{Julian, epist. xxix. xxx. La Bleterie has neglected
to translate the second of these epistles.}

\pagenote[76]{See the zeal and impatience of the Jews in Gregory
Nazianzen (Orat. iv. p. 111) and Theodoret. (l. iii. c. 20.)}

Yet, on this occasion, the joint efforts of power and enthusiasm
were unsuccessful; and the ground of the Jewish temple, which is
now covered by a Mahometan mosque,\textsuperscript{77} still continued to exhibit
the same edifying spectacle of ruin and desolation. Perhaps the
absence and death of the emperor, and the new maxims of a
Christian reign, might explain the interruption of an arduous
work, which was attempted only in the last six months of the life
of Julian.\textsuperscript{78} But the Christians entertained a natural and pious
expectation, that, in this memorable contest, the honor of
religion would be vindicated by some signal miracle. An
earthquake, a whirlwind, and a fiery eruption, which overturned
and scattered the new foundations of the temple, are attested,
with some variations, by contemporary and respectable evidence.\textsuperscript{79}
This public event is described by Ambrose,\textsuperscript{80} bishop of Milan,
in an epistle to the emperor Theodosius, which must provoke the
severe animadversion of the Jews; by the eloquent Chrysostom,\textsuperscript{81}
who might appeal to the memory of the elder part of his
congregation at Antioch; and by Gregory Nazianzen,\textsuperscript{82} who
published his account of the miracle before the expiration of the
same year. The last of these writers has boldly declared, that
this preternatural event was not disputed by the infidels; and
his assertion, strange as it may seem is confirmed by the
unexceptionable testimony of Ammianus Marcellinus.\textsuperscript{83} The
philosophic soldier, who loved the virtues, without adopting the
prejudices, of his master, has recorded, in his judicious and
candid history of his own times, the extraordinary obstacles
which interrupted the restoration of the temple of Jerusalem.
“Whilst Alypius, assisted by the governor of the province, urged,
with vigor and diligence, the execution of the work, horrible
balls of fire breaking out near the foundations, with frequent
and reiterated attacks, rendered the place, from time to time,
inaccessible to the scorched and blasted workmen; and the
victorious element continuing in this manner obstinately and
resolutely bent, as it were, to drive them to a distance, the
undertaking was abandoned.”\textsuperscript{8311} Such authority should satisfy a
believing, and must astonish an incredulous, mind. Yet a
philosopher may still require the original evidence of impartial
and intelligent spectators. At this important crisis, any
singular accident of nature would assume the appearance, and
produce the effects of a real prodigy. This glorious deliverance
would be speedily improved and magnified by the pious art of the
clergy of Jerusalem, and the active credulity of the Christian
world and, at the distance of twenty years, a Roman historian,
careless of theological disputes, might adorn his work with the
specious and splendid miracle.\textsuperscript{84}

\pagenote[77]{Built by Omar, the second Khalif, who died A. D.
644. This great mosque covers the whole consecrated ground of the
Jewish temple, and constitutes almost a square of 760 \textit{toises},
or one Roman mile in circumference. See D’Anville, Jerusalem, p.
45.}

\pagenote[78]{Ammianus records the consults of the year 363,
before he proceeds to mention the \textit{thoughts} of Julian. Templum.
... instaurare sumptibus \textit{cogitabat} immodicis. Warburton has a
secret wish to anticipate the design; but he must have
understood, from former examples, that the execution of such a
work would have demanded many years.}

\pagenote[79]{The subsequent witnesses, Socrates, Sozomen,
Theodoret, Philostorgius, \&c., add contradictions rather than
authority. Compare the objections of Basnage (Hist. des Juifs,
tom. viii. p. 156-168) with Warburton’s answers, (Julian, p.
174-258.) The bishop has ingeniously explained the miraculous
crosses which appeared on the garments of the spectators by a
similar instance, and the natural effects of lightning.}

\pagenote[80]{Ambros. tom. ii. epist. xl. p. 946, edit.
Benedictin. He composed this fanatic epistle (A. D. 388) to
justify a bishop who had been condemned by the civil magistrate
for burning a synagogue.}

\pagenote[81]{Chrysostom, tom. i. p. 580, advers. Judæos et
Gentes, tom. ii. p. 574, de Sto Babyla, edit. Montfaucon. I have
followed the common and natural supposition; but the learned
Benedictine, who dates the composition of these sermons in the
year 383, is confident they were never pronounced from the
pulpit.}

\pagenote[82]{Greg. Nazianzen, Orat. iv. p. 110-113.}

\pagenote[83]{Ammian. xxiii. 1. Cum itaque rei fortiter instaret
Alypius, juvaretque provinciæ rector, metuendi globi flammarum
prope fundamenta crebris assultibus erumpentes fecere locum
exustis aliquoties operantibus inaccessum; hocque modo elemento
destinatius repellente, cessavit inceptum. Warburton labors (p.
60-90) to extort a confession of the miracle from the mouths of
Julian and Libanius, and to employ the evidence of a rabbi who
lived in the fifteenth century. Such witnesses can only be
received by a very favorable judge.}

\pagenote[8311]{Michaelis has given an ingenious and sufficiently
probable explanation of this remarkable incident, which the
positive testimony of Ammianus, a contemporary and a pagan, will
not permit us to call in question. It was suggested by a passage
in Tacitus. That historian, speaking of Jerusalem, says, [I omit
the first part of the quotation adduced by M. Guizot, which only
by a most extraordinary mistranslation of muri introrsus sinuati
by “\textit{enfoncemens}” could be made to bear on the question.—M.]

“The Temple itself was a kind of citadel, which had its own
walls, superior in their workmanship and construction to those of
the city. The porticos themselves, which surrounded the temple,
were an excellent fortification. There was a fountain of
constantly running water; \textit{subterranean excavations under the
mountain; reservoirs and cisterns to collect the rain-water}.”
Tac. Hist. v. ii. 12. These excavations and reservoirs must have
been very considerable. The latter furnished water during the
whole siege of Jerusalem to 1,100,000 inhabitants, for whom the
fountain of Siloe could not have sufficed, and who had no fresh
rain-water, the siege having taken place from the month of April
to the month of August, a period of the year during which it
rarely rains in Jerusalem. As to the excavations, they served
after, and even before, the return of the Jews from Babylon, to
contain not only magazines of oil, wine, and corn, but also the
treasures which were laid up in the Temple. Josephus has related
several incidents which show their extent. When Jerusalem was on
the point of being taken by Titus, the rebel chiefs, placing
their last hopes in these vast subterranean cavities, formed a
design of concealing themselves there, and remaining during the
conflagration of the city, and until the Romans had retired to a
distance. The greater part had not time to execute their design;
but one of them, Simon, the Son of Gioras, having provided
himself with food, and tools to excavate the earth descended into
this retreat with some companions: he remained there till Titus
had set out for Rome: under the pressure of famine he issued
forth on a sudden in the very place where the Temple had stood,
and appeared in the midst of the Roman guard. He was seized and
carried to Rome for the triumph. His appearance made it be
suspected that other Jews might have chosen the same asylum;
search was made, and a great number discovered. Joseph. de Bell.
Jud. l. vii. c. 2. It is probable that the greater part of these
excavations were the remains of the time of Solomon, when it was
the custom to work to a great extent under ground: no other date
can be assigned to them. The Jews, on their return from the
captivity, were too poor to undertake such works; and, although
Herod, on rebuilding the Temple, made some excavations, (Joseph.
Ant. Jud. xv. 11, vii.,) the haste with which that building was
completed will not allow us to suppose that they belonged to that
period. Some were used for sewers and drains, others served to
conceal the immense treasures of which Crassus, a hundred and
twenty years before, plundered the Jews, and which doubtless had
been since replaced. The Temple was destroyed A. C. 70; the
attempt of Julian to rebuild it, and the fact related by
Ammianus, coincide with the year 363. There had then elapsed
between these two epochs an interval of near 300 years, during
which the excavations, choked up with ruins, must have become
full of inflammable air. The workmen employed by Julian as they
were digging, arrived at the excavations of the Temple; they
would take torches to explore them; sudden flames repelled those
who approached; explosions were heard, and these phenomena were
renewed every time that they penetrated into new subterranean
passages. This explanation is confirmed by the relation of an
event nearly similar, by Josephus. King Herod having heard that
immense treasures had been concealed in the sepulchre of David,
he descended into it with a few confidential persons; he found in
the first subterranean chamber only jewels and precious stuffs:
but having wished to penetrate into a second chamber, which had
been long closed, he was repelled, when he opened it, by flames
which killed those who accompanied him. (Ant. Jud. xvi. 7, i.) As
here there is no room for miracle, this fact may be considered as
a new proof of the veracity of that related by Ammianus and the
contemporary writers.—G. ——To the illustrations of the extent of
the subterranean chambers adduced by Michaelis, may be added,
that when John of Gischala, during the siege, surprised the
Temple, the party of Eleazar took refuge within them. Bell. Jud.
vi. 3, i. The sudden sinking of the hill of Sion when Jerusalem
was occupied by Barchocab, may have been connected with similar
excavations. Hist. of Jews, vol. iii. 122 and 186.—M. ——It is a
fact now popularly known, that when mines which have been long
closed are opened, one of two things takes place; either the
torches are extinguished and the men fall first into a swoor and
soon die; or, if the air is inflammable, a little flame is seen
to flicker round the lamp, which spreads and multiplies till the
conflagration becomes general, is followed by an explosion, and
kill all who are in the way.—G.}

\pagenote[84]{Dr. Lardner, perhaps alone of the Christian
critics, presumes to doubt the truth of this famous miracle.
(Jewish and Heathen Testimonies, vol. iv. p. 47-71.)}

The silence of Jerom would lead to a suspicion that the same
story which was celebrated at a distance, might be despised on
the spot. * Note: Gibbon has forgotten Basnage, to whom Warburton
replied.—M.

\section{Part \thesection.}

The restoration of the Jewish temple was secretly connected with
the ruin of the Christian church. Julian still continued to
maintain the freedom of religious worship, without distinguishing
whether this universal toleration proceeded from his justice or
his clemency. He affected to pity the unhappy Christians, who
were mistaken in the most important object of their lives; but
his pity was degraded by contempt, his contempt was embittered by
hatred; and the sentiments of Julian were expressed in a style of
sarcastic wit, which inflicts a deep and deadly wound, whenever
it issues from the mouth of a sovereign. As he was sensible that
the Christians gloried in the name of their Redeemer, he
countenanced, and perhaps enjoined, the use of the less honorable
appellation of Galilæans.\textsuperscript{85} He declared, that by the folly of
the Galilæans, whom he describes as a sect of fanatics,
contemptible to men, and odious to the gods, the empire had been
reduced to the brink of destruction; and he insinuates in a
public edict, that a frantic patient might sometimes be cured by
salutary violence.\textsuperscript{86} An ungenerous distinction was admitted into
the mind and counsels of Julian, that, according to the
difference of their religious sentiments, one part of his
subjects deserved his favor and friendship, while the other was
entitled only to the common benefits that his justice could not
refuse to an obedient people. According to a principle, pregnant
with mischief and oppression, the emperor transferred to the
pontiffs of his own religion the management of the liberal
allowances from the public revenue, which had been granted to the
church by the piety of Constantine and his sons. The proud system
of clerical honors and immunities, which had been constructed
with so much art and labor, was levelled to the ground; the hopes
of testamentary donations were intercepted by the rigor of the
laws; and the priests of the Christian sect were confounded with
the last and most ignominious class of the people. Such of these
regulations as appeared necessary to check the ambition and
avarice of the ecclesiastics, were soon afterwards imitated by
the wisdom of an orthodox prince. The peculiar distinctions which
policy has bestowed, or superstition has lavished, on the
sacerdotal order, \textit{must} be confined to those priests who profess
the religion of the state. But the will of the legislator was not
exempt from prejudice and passion; and it was the object of the
insidious policy of Julian, to deprive the Christians of all the
temporal honors and advantages which rendered them respectable in
the eyes of the world.\textsuperscript{88}

\pagenote[85]{Greg. Naz. Orat. iii. p. 81. And this law was
confirmed by the invariable practice of Julian himself. Warburton
has justly observed (p. 35,) that the Platonists believed in the
mysterious virtue of words and Julian’s dislike for the name of
Christ might proceed from superstition, as well as from
contempt.}

\pagenote[86]{Fragment. Julian. p. 288. He derides the (Epist.
vii.,) and so far loses sight of the principles of toleration, as
to wish (Epist. xlii.).}

\pagenote[88]{These laws, which affected the clergy, may be found
in the slight hints of Julian himself, (Epist. lii.) in the vague
declamations of Gregory, (Orat. iii. p. 86, 87,) and in the
positive assertions of Sozomen, (l. v. c. 5.)}

A just and severe censure has been inflicted on the law which
prohibited the Christians from teaching the arts of grammar and
rhetoric.\textsuperscript{89} The motives alleged by the emperor to justify this
partial and oppressive measure, might command, during his
lifetime, the silence of slaves and the applause of Gatterers.
Julian abuses the ambiguous meaning of a word which might be
indifferently applied to the language and the religion of the
Greeks: he contemptuously observes, that the men who exalt the
merit of implicit faith are unfit to claim or to enjoy the
advantages of science; and he vainly contends, that if they
refuse to adore the gods of Homer and Demosthenes, they ought to
content themselves with expounding Luke and Matthew in the church
of the Galilæans.\textsuperscript{90} In all the cities of the Roman world, the
education of the youth was intrusted to masters of grammar and
rhetoric; who were elected by the magistrates, maintained at the
public expense, and distinguished by many lucrative and honorable
privileges. The edict of Julian appears to have included the
physicians, and professors of all the liberal arts; and the
emperor, who reserved to himself the approbation of the
candidates, was authorized by the laws to corrupt, or to punish,
the religious constancy of the most learned of the Christians.\textsuperscript{91}
As soon as the resignation of the more obstinate\textsuperscript{92} teachers had
established the unrivalled dominion of the Pagan sophists, Julian
invited the rising generation to resort with freedom to the
public schools, in a just confidence, that their tender minds
would receive the impressions of literature and idolatry. If the
greatest part of the Christian youth should be deterred by their
own scruples, or by those of their parents, from accepting this
dangerous mode of instruction, they must, at the same time,
relinquish the benefits of a liberal education. Julian had reason
to expect that, in the space of a few years, the church would
relapse into its primæval simplicity, and that the theologians,
who possessed an adequate share of the learning and eloquence of
the age, would be succeeded by a generation of blind and ignorant
fanatics, incapable of defending the truth of their own
principles, or of exposing the various follies of Polytheism.\textsuperscript{93}

\pagenote[89]{Inclemens.... perenni obruendum silentio. Ammian.
xxii. 10, ixv. 5.}

\pagenote[90]{The edict itself, which is still extant among the
epistles of Julian, (xlii.,) may be compared with the loose
invectives of Gregory (Orat. iii. p. 96.) Tillemont (Mém. Eccles.
tom. vii. p. 1291-1294) has collected the seeming differences of
ancients and moderns. They may be easily reconciled. The
Christians were \textit{directly} forbid to teach, they were
\textit{indirectly} forbid to learn; since they would not frequent the
schools of the Pagans.}

\pagenote[91]{Codex Theodos. l. xiii. tit. iii. de medicis et
professoribus, leg. 5, (published the 17th of June, received, at
Spoleto in Italy, the 29th of July, A. D. 363,) with Godefroy’s
Illustrations, tom. v. p. 31.}

\pagenote[92]{Orosius celebrates their disinterested resolution,
Sicut a majori bus nostris compertum habemus, omnes ubique
propemodum... officium quam fidem deserere maluerunt, vii. 30.
Proæresius, a Christian sophist, refused to accept the partial
favor of the emperor Hieronym. in Chron. p. 185, edit. Scaliger.
Eunapius in Proæresio p. 126.}

\pagenote[93]{They had recourse to the expedient of composing
books for their own schools. Within a few months Apollinaris
produced his Christian imitations of Homer, (a sacred history in
twenty-four books,) Pindar, Euripides, and Menander; and Sozomen
is satisfied, that they equalled, or excelled, the originals. *
Note: Socrates, however, implies that, on the death of Julian,
they were contemptuously thrown aside by the Christians. Socr.
Hist. iii.16.—M.}

It was undoubtedly the wish and design of Julian to deprive the
Christians of the advantages of wealth, of knowledge, and of
power; but the injustice of excluding them from all offices of
trust and profit seems to have been the result of his general
policy, rather than the immediate consequence of any positive
law.\textsuperscript{94} Superior merit might deserve and obtain, some
extraordinary exceptions; but the greater part of the Christian
officers were gradually removed from their employments in the
state, the army, and the provinces. The hopes of future
candidates were extinguished by the declared partiality of a
prince, who maliciously reminded them, that it was unlawful for a
Christian to use the sword, either of justice, or of war; and who
studiously guarded the camp and the tribunals with the ensigns of
idolatry. The powers of government were intrusted to the pagans,
who professed an ardent zeal for the religion of their ancestors;
and as the choice of the emperor was often directed by the rules
of divination, the favorites whom he preferred as the most
agreeable to the gods, did not always obtain the approbation of
mankind.\textsuperscript{95} Under the administration of their enemies, the
Christians had much to suffer, and more to apprehend. The temper
of Julian was averse to cruelty; and the care of his reputation,
which was exposed to the eyes of the universe, restrained the
philosophic monarch from violating the laws of justice and
toleration, which he himself had so recently established. But the
provincial ministers of his authority were placed in a less
conspicuous station. In the exercise of arbitrary power, they
consulted the wishes, rather than the commands, of their
sovereign; and ventured to exercise a secret and vexatious
tyranny against the sectaries, on whom they were not permitted to
confer the honors of martyrdom. The emperor, who dissembled as
long as possible his knowledge of the injustice that was
exercised in his name, expressed his real sense of the conduct of
his officers, by gentle reproofs and substantial rewards.\textsuperscript{96}

\pagenote[94]{It was the instruction of Julian to his
magistrates, (Epist. vii.,). Sozomen (l. v. c. 18) and Socrates
(l. iii. c. 13) must be reduced to the standard of Gregory,
(Orat. iii. p. 95,) not less prone to exaggeration, but more
restrained by the actual knowledge of his contemporary readers.}

\pagenote[95]{Libanius, Orat. Parent. 88, p. 814.}

\pagenote[96]{Greg. Naz. Orat. iii. p. 74, 91, 92. Socrates, l.
iii. c. 14. The doret, l. iii. c. 6. Some drawback may, however,
be allowed for the violence of \textit{their} zeal, not less partial
than the zeal of Julian}

The most effectual instrument of oppression, with which they were
armed, was the law that obliged the Christians to make full and
ample satisfaction for the temples which they had destroyed under
the preceding reign. The zeal of the triumphant church had not
always expected the sanction of the public authority; and the
bishops, who were secure of impunity, had often marched at the
head of their congregation, to attack and demolish the fortresses
of the prince of darkness. The consecrated lands, which had
increased the patrimony of the sovereign or of the clergy, were
clearly defined, and easily restored. But on these lands, and on
the ruins of Pagan superstition, the Christians had frequently
erected their own religious edifices: and as it was necessary to
remove the church before the temple could be rebuilt, the justice
and piety of the emperor were applauded by one party, while the
other deplored and execrated his sacrilegious violence.\textsuperscript{97} After
the ground was cleared, the restitution of those stately
structures which had been levelled with the dust, and of the
precious ornaments which had been converted to Christian uses,
swelled into a very large account of damages and debt. The
authors of the injury had neither the ability nor the inclination
to discharge this accumulated demand: and the impartial wisdom of
a legislator would have been displayed in balancing the adverse
claims and complaints, by an equitable and temperate arbitration.

But the whole empire, and particularly the East, was thrown into
confusion by the rash edicts of Julian; and the Pagan
magistrates, inflamed by zeal and revenge, abused the rigorous
privilege of the Roman law, which substitutes, in the place of
his inadequate property, the person of the insolvent debtor.
Under the preceding reign, Mark, bishop of Arethusa,\textsuperscript{98} had
labored in the conversion of his people with arms more effectual
than those of persuasion.\textsuperscript{99} The magistrates required the full
value of a temple which had been destroyed by his intolerant
zeal: but as they were satisfied of his poverty, they desired
only to bend his inflexible spirit to the promise of the
slightest compensation. They apprehended the aged prelate, they
inhumanly scourged him, they tore his beard; and his naked body,
annointed with honey, was suspended, in a net, between heaven and
earth, and exposed to the stings of insects and the rays of a
Syrian sun.\textsuperscript{100} From this lofty station, Mark still persisted to
glory in his crime, and to insult the impotent rage of his
persecutors. He was at length rescued from their hands, and
dismissed to enjoy the honor of his divine triumph. The Arians
celebrated the virtue of their pious confessor; the Catholics
ambitiously claimed his alliance;\textsuperscript{101} and the Pagans, who might
be susceptible of shame or remorse, were deterred from the
repetition of such unavailing cruelty.\textsuperscript{102} Julian spared his
life: but if the bishop of Arethusa had saved the infancy of
Julian,\textsuperscript{103} posterity will condemn the ingratitude, instead of
praising the clemency, of the emperor.

\pagenote[97]{If we compare the gentle language of Libanius
(Orat. Parent c. 60. p. 286) with the passionate exclamations of
Gregory, (Orat. iii. p. 86, 87,) we may find it difficult to
persuade ourselves that the two orators are really describing the
same events.}

\pagenote[98]{Restan, or Arethusa, at the equal distance of
sixteen miles between Emesa (\textit{Hems}) and Epiphania, (\textit{Hamath},)
was founded, or at least named, by Seleucus Nicator. Its peculiar
æra dates from the year of Rome 685, according to the medals of
the city. In the decline of the Seleucides, Emesa and Arethusa
were usurped by the Arab Sampsiceramus, whose posterity, the
vassals of Rome, were not extinguished in the reign of
Vespasian.——See D’Anville’s Maps and Geographie Ancienne, tom.
ii. p. 134. Wesseling, Itineraria, p. 188, and Noris. Epoch
Syro-Macedon, p. 80, 481, 482.}

\pagenote[99]{Sozomen, l. v. c. 10. It is surprising, that
Gregory and Theodoret should suppress a circumstance, which, in
their eyes, must have enhanced the religious merit of the
confessor.}

\pagenote[100]{The sufferings and constancy of Mark, which
Gregory has so tragically painted, (Orat. iii. p. 88-91,) are
confirmed by the unexceptionable and reluctant evidence of
Libanius. Epist. 730, p. 350, 351. Edit. Wolf. Amstel. 1738.}

\pagenote[101]{Certatim eum sibi (Christiani) vindicant. It is
thus that La Croze and Wolfius (ad loc.) have explained a Greek
word, whose true signification had been mistaken by former
interpreters, and even by Le Clerc, (Bibliothèque Ancienne et
Moderne, tom. iii. p. 371.) Yet Tillemont is strangely puzzled to
understand (Mém. Eccles. tom. vii. p. 1309) \textit{how} Gregory and
Theodoret could mistake a Semi-Arian bishop for a saint.}

\pagenote[102]{See the probable advice of Sallust, (Greg.
Nazianzen, Orat. iii. p. 90, 91.) Libanius intercedes for a
similar offender, lest they should find many \textit{Marks;} yet he
allows, that if Orion had secreted the consecrated wealth, he
deserved to suffer the punishment of Marsyas; to be flayed alive,
(Epist. 730, p. 349-351.)}

\pagenote[103]{Gregory (Orat. iii. p. 90) is satisfied that, by
saving the apostate, Mark had deserved still more than he had
suffered.}

At the distance of five miles from Antioch, the Macedonian kings
of Syria had consecrated to Apollo one of the most elegant places
of devotion in the Pagan world.\textsuperscript{104} A magnificent temple rose in
honor of the god of light; and his colossal figure\textsuperscript{105} almost
filled the capacious sanctuary, which was enriched with gold and
gems, and adorned by the skill of the Grecian artists. The deity
was represented in a bending attitude, with a golden cup in his
hand, pouring out a libation on the earth; as if he supplicated
the venerable mother to give to his arms the cold and beauteous
Daphne: for the spot was ennobled by fiction; and the fancy of
the Syrian poets had transported the amorous tale from the banks
of the Peneus to those of the Orontes. The ancient rites of
Greece were imitated by the royal colony of Antioch. A stream of
prophecy, which rivalled the truth and reputation of the Delphic
oracle, flowed from the \textit{Castalian} fountain of Daphne.\textsuperscript{106} In
the adjacent fields a stadium was built by a special privilege,\textsuperscript{107}
which had been purchased from Elis; the Olympic games were
celebrated at the expense of the city; and a revenue of thirty
thousand pounds sterling was annually applied to the public
pleasures.\textsuperscript{108} The perpetual resort of pilgrims and spectators
insensibly formed, in the neighborhood of the temple, the stately
and populous village of Daphne, which emulated the splendor,
without acquiring the title, of a provincial city. The temple and
the village were deeply bosomed in a thick grove of laurels and
cypresses, which reached as far as a circumference of ten miles,
and formed in the most sultry summers a cool and impenetrable
shade. A thousand streams of the purest water, issuing from every
hill, preserved the verdure of the earth, and the temperature of
the air; the senses were gratified with harmonious sounds and
aromatic odors; and the peaceful grove was consecrated to health
and joy, to luxury and love. The vigorous youth pursued, like
Apollo, the object of his desires; and the blushing maid was
warned, by the fate of Daphne, to shun the folly of unseasonable
coyness. The soldier and the philosopher wisely avoided the
temptation of this sensual paradise:\textsuperscript{109} where pleasure, assuming
the character of religion, imperceptibly dissolved the firmness
of manly virtue. But the groves of Daphne continued for many ages
to enjoy the veneration of natives and strangers; the privileges
of the holy ground were enlarged by the munificence of succeeding
emperors; and every generation added new ornaments to the
splendor of the temple.\textsuperscript{110}

\pagenote[104]{The grove and temple of Daphne are described by
Strabo, (l. xvi. p. 1089, 1090, edit. Amstel. 1707,) Libanius,
(Nænia, p. 185-188. Antiochic. Orat. xi. p. 380, 381,) and
Sozomen, (l. v. c. 19.) Wesseling (Itinerar. p. 581) and Casaubon
(ad Hist. August. p. 64) illustrate this curious subject.}

\pagenote[105]{Simulacrum in eo Olympiaci Jovis imitamenti
æquiparans magnitudinem. Ammian. xxii. 13. The Olympic Jupiter
was sixty feet high, and his bulk was consequently equal to that
of a thousand men. See a curious \textit{Mémoire} of the Abbé Gedoyn,
(Académie des Inscriptions, tom. ix. p. 198.)}

\pagenote[106]{Hadrian read the history of his future fortunes on
a leaf dipped in the Castalian stream; a trick which, according
to the physician Vandale, (de Oraculis, p. 281, 282,) might be
easily performed by chemical preparations. The emperor stopped
the source of such dangerous knowledge; which was again opened by
the devout curiosity of Julian.}

\pagenote[107]{It was purchased, A. D. 44, in the year 92 of the
æra of Antioch, (Noris. Epoch. Syro-Maced. p. 139-174,) for the
term of ninety Olympiads. But the Olympic games of Antioch were
not regularly celebrated till the reign of Commodus. See the
curious details in the Chronicle of John Malala, (tom. i. p. 290,
320, 372-381,) a writer whose merit and authority are confined
within the limits of his native city.}

\pagenote[108]{Fifteen talents of gold, bequeathed by Sosibius,
who died in the reign of Augustus. The theatrical merits of the
Syrian cities in the reign of Constantine, are computed in the
Expositio totius Murd, p. 8, (Hudson, Geograph. Minor tom. iii.)}

\pagenote[109]{Avidio Cassio Syriacas legiones dedi luxuria
diffluentes et \textit{Daphnicis} moribus. These are the words of the
emperor Marcus Antoninus in an original letter preserved by his
biographer in Hist. August. p. 41. Cassius dismissed or punished
every soldier who was seen at Daphne.}

\pagenote[110]{Aliquantum agrorum Daphnensibus dedit, (\textit{Pompey},)
quo lucus ibi spatiosior fieret; delectatus amœnitate loci et
aquarum abundantiz, Eutropius, vi. 14. Sextus Rufus, de
Provinciis, c. 16.}

When Julian, on the day of the annual festival, hastened to adore
the Apollo of Daphne, his devotion was raised to the highest
pitch of eagerness and impatience. His lively imagination
anticipated the grateful pomp of victims, of libations and of
incense; a long procession of youths and virgins, clothed in
white robes, the symbol of their innocence; and the tumultuous
concourse of an innumerable people. But the zeal of Antioch was
diverted, since the reign of Christianity, into a different
channel. Instead of hecatombs of fat oxen sacrificed by the
tribes of a wealthy city to their tutelar deity the emperor
complains that he found only a single goose, provided at the
expense of a priest, the pale and solitary inhabitant of this
decayed temple.\textsuperscript{111} The altar was deserted, the oracle had been
reduced to silence, and the holy ground was profaned by the
introduction of Christian and funereal rites. After Babylas\textsuperscript{112}
(a bishop of Antioch, who died in prison in the persecution of
Decius) had rested near a century in his grave, his body, by the
order of Cæsar Gallus, was transported into the midst of the
grove of Daphne. A magnificent church was erected over his
remains; a portion of the sacred lands was usurped for the
maintenance of the clergy, and for the burial of the Christians
at Antioch, who were ambitious of lying at the feet of their
bishop; and the priests of Apollo retired, with their affrighted
and indignant votaries. As soon as another revolution seemed to
restore the fortune of Paganism, the church of St. Babylas was
demolished, and new buildings were added to the mouldering
edifice which had been raised by the piety of Syrian kings. But
the first and most serious care of Julian was to deliver his
oppressed deity from the odious presence of the dead and living
Christians, who had so effectually suppressed the voice of fraud
or enthusiasm.\textsuperscript{113} The scene of infection was purified, according
to the forms of ancient rituals; the bodies were decently
removed; and the ministers of the church were permitted to convey
the remains of St. Babylas to their former habitation within the
walls of Antioch. The modest behavior which might have assuaged
the jealousy of a hostile government was neglected, on this
occasion, by the zeal of the Christians. The lofty car, that
transported the relics of Babylas, was followed, and accompanied,
and received, by an innumerable multitude; who chanted, with
thundering acclamations, the Psalms of David the most expressive
of their contempt for idols and idolaters. The return of the
saint was a triumph; and the triumph was an insult on the
religion of the emperor, who exerted his pride to dissemble his
resentment. During the night which terminated this indiscreet
procession, the temple of Daphne was in flames; the statue of
Apollo was consumed; and the walls of the edifice were left a
naked and awful monument of ruin. The Christians of Antioch
asserted, with religious confidence, that the powerful
intercession of St. Babylas had pointed the lightnings of heaven
against the devoted roof: but as Julian was reduced to the
alternative of believing either a crime or a miracle, he chose,
without hesitation, without evidence, but with some color of
probability, to impute the fire of Daphne to the revenge of the
Galilæans.\textsuperscript{114} Their offence, had it been sufficiently proved,
might have justified the retaliation, which was immediately
executed by the order of Julian, of shutting the doors, and
confiscating the wealth, of the cathedral of Antioch. To discover
the criminals who were guilty of the tumult, of the fire, or of
secreting the riches of the church, several of the ecclesiastics
were tortured;\textsuperscript{115} and a Presbyter, of the name of Theodoret, was
beheaded by the sentence of the Count of the East. But this hasty
act was blamed by the emperor; who lamented, with real or
affected concern, that the imprudent zeal of his ministers would
tarnish his reign with the disgrace of persecution.\textsuperscript{116}

\pagenote[111]{Julian (Misopogon, p. 367, 362) discovers his own
character with \textit{naïveté}, that unconscious simplicity which
always constitutes genuine humor.}

\pagenote[112]{Babylas is named by Eusebius in the succession of
the bishops of Antioch, (Hist. Eccles. l. vi. c. 29, 39.) His
triumph over two emperors (the first fabulous, the second
historical) is diffusely celebrated by Chrysostom, (tom. ii. p.
536-579, edit. Montfaucon.) Tillemont (Mém. Eccles. tom. iii.
part ii. p. 287-302, 459-465) becomes almost a sceptic.}

\pagenote[113]{Ecclesiastical critics, particularly those who
love relics, exult in the confession of Julian (Misopogon, p.
361) and Libanius, (Lænia, p. 185,) that Apollo was disturbed by
the vicinity of \textit{one} dead man. Yet Ammianus (xxii. 12) clears
and purifies the whole ground, according to the rites which the
Athenians formerly practised in the Isle of Delos.}

\pagenote[114]{Julian (in Misopogon, p. 361) rather insinuates,
than affirms, their guilt. Ammianus (xxii. 13) treats the
imputation as \textit{levissimus rumor}, and relates the story with
extraordinary candor.}

\pagenote[115]{Quo tam atroci casu repente consumpto, ad id usque
e imperatoris ira provexit, ut quæstiones agitare juberet solito
acriores, (yet Julian blames the lenity of the magistrates of
Antioch,) et majorem ecclesiam Antiochiæ claudi. This
interdiction was performed with some circumstances of indignity
and profanation; and the seasonable death of the principal actor,
Julian’s uncle, is related with much superstitious complacency by
the Abbé de la Bleterie. Vie de Julien, p. 362-369.}

\pagenote[116]{Besides the ecclesiastical historians, who are
more or less to be suspected, we may allege the passion of St.
Theodore, in the Acta Sincera of Ruinart, p. 591. The complaint
of Julian gives it an original and authentic air.}

\section{Part \thesection.}

The zeal of the ministers of Julian was instantly checked by the
frown of their sovereign; but when the father of his country
declares himself the leader of a faction, the license of popular
fury cannot easily be restrained, nor consistently punished.
Julian, in a public composition, applauds the devotion and
loyalty of the holy cities of Syria, whose pious inhabitants had
destroyed, at the first signal, the sepulchres of the Galilæans;
and faintly complains, that they had revenged the injuries of the
gods with less moderation than he should have recommended.\textsuperscript{117}
This imperfect and reluctant confession may appear to confirm the
ecclesiastical narratives; that in the cities of Gaza, Ascalon,
Cæsarea, Heliopolis, \&c., the Pagans abused, without prudence or
remorse, the moment of their prosperity. That the unhappy objects
of their cruelty were released from torture only by death; and as
their mangled bodies were dragged through the streets, they were
pierced (such was the universal rage) by the spits of cooks, and
the distaffs of enraged women; and that the entrails of Christian
priests and virgins, after they had been tasted by those bloody
fanatics, were mixed with barley, and contemptuously thrown to
the unclean animals of the city.\textsuperscript{118} Such scenes of religious
madness exhibit the most contemptible and odious picture of human
nature; but the massacre of Alexandria attracts still more
attention, from the certainty of the fact, the rank of the
victims, and the splendor of the capital of Egypt.

\pagenote[117]{Julian. Misopogon, p. 361.}

\pagenote[118]{See Gregory Nazianzen, (Orat. iii. p. 87.) Sozomen
(l. v. c. 9) may be considered as an original, though not
impartial, witness. He was a native of Gaza, and had conversed
with the confessor Zeno, who, as bishop of Maiuma, lived to the
age of a hundred, (l. vii. c. 28.) Philostorgius (l. vii. c. 4,
with Godefroy’s Dissertations, p. 284) adds some tragic
circumstances, of Christians who were \textit{literally} sacrificed at
the altars of the gods, \&c.}

George,\textsuperscript{119} from his parents or his education, surnamed the
Cappadocian, was born at Epiphania in Cilicia, in a fuller’s
shop. From this obscure and servile origin he raised himself by
the talents of a parasite; and the patrons, whom he assiduously
flattered, procured for their worthless dependent a lucrative
commission, or contract, to supply the army with bacon. His
employment was mean; he rendered it infamous. He accumulated
wealth by the basest arts of fraud and corruption; but his
malversations were so notorious, that George was compelled to
escape from the pursuits of justice. After this disgrace, in
which he appears to have saved his fortune at the expense of his
honor, he embraced, with real or affected zeal, the profession of
Arianism. From the love, or the ostentation, of learning, he
collected a valuable library of history rhetoric, philosophy, and
theology,\textsuperscript{120} and the choice of the prevailing faction promoted
George of Cappadocia to the throne of Athanasius. The entrance of
the new archbishop was that of a Barbarian conqueror; and each
moment of his reign was polluted by cruelty and avarice. The
Catholics of Alexandria and Egypt were abandoned to a tyrant,
qualified, by nature and education, to exercise the office of
persecution; but he oppressed with an impartial hand the various
inhabitants of his extensive diocese. The primate of Egypt
assumed the pomp and insolence of his lofty station; but he still
betrayed the vices of his base and servile extraction. The
merchants of Alexandria were impoverished by the unjust, and
almost universal, monopoly, which he acquired, of nitre, salt,
paper, funerals, \&c.: and the spiritual father of a great people
condescended to practise the vile and pernicious arts of an
informer. The Alexandrians could never forget, nor forgive, the
tax, which he suggested, on all the houses of the city; under an
obsolete claim, that the royal founder had conveyed to his
successors, the Ptolemies and the Cæsars, the perpetual property
of the soil. The Pagans, who had been flattered with the hopes of
freedom and toleration, excited his devout avarice; and the rich
temples of Alexandria were either pillaged or insulted by the
haughty prince, who exclaimed, in a loud and threatening tone,
“How long will these sepulchres be permitted to stand?” Under the
reign of Constantius, he was expelled by the fury, or rather by
the justice, of the people; and it was not without a violent
struggle, that the civil and military powers of the state could
restore his authority, and gratify his revenge. The messenger who
proclaimed at Alexandria the accession of Julian, announced the
downfall of the archbishop. George, with two of his obsequious
ministers, Count Diodorus, and Dracontius, master of the mint
were ignominiously dragged in chains to the public prison. At the
end of twenty-four days, the prison was forced open by the rage
of a superstitious multitude, impatient of the tedious forms of
judicial proceedings. The enemies of gods and men expired under
their cruel insults; the lifeless bodies of the archbishop and
his associates were carried in triumph through the streets on the
back of a camel;\textsuperscript{12011} and the inactivity of the Athanasian party\textsuperscript{121}
was esteemed a shining example of evangelical patience. The
remains of these guilty wretches were thrown into the sea; and
the popular leaders of the tumult declared their resolution to
disappoint the devotion of the Christians, and to intercept the
future honors of these \textit{martyrs}, who had been punished, like
their predecessors, by the enemies of their religion.\textsuperscript{122} The
fears of the Pagans were just, and their precautions ineffectual.
The meritorious death of the archbishop obliterated the memory of
his life. The rival of Athanasius was dear and sacred to the
Arians, and the seeming conversion of those sectaries introduced
his worship into the bosom of the Catholic church.\textsuperscript{123} The odious
stranger, disguising every circumstance of time and place,
assumed the mask of a martyr, a saint, and a Christian hero;\textsuperscript{124}
and the infamous George of Cappadocia has been transformed\textsuperscript{125}
into the renowned St. George of England, the patron of arms, of
chivalry, and of the garter.\textsuperscript{126}

\pagenote[119]{The life and death of George of Cappadocia are
described by Ammianus, (xxii. 11,) Gregory of Nazianzen, (Orat.
xxi. p. 382, 385, 389, 390,) and Epiphanius, (Hæres. lxxvi.) The
invectives of the two saints might not deserve much credit,
unless they were confirmed by the testimony of the cool and
impartial infidel.}

\pagenote[120]{After the massacre of George, the emperor Julian
repeatedly sent orders to preserve the library for his own use,
and to torture the slaves who might be suspected of secreting any
books. He praises the merit of the collection, from whence he had
borrowed and transcribed several manuscripts while he pursued his
studies in Cappadocia. He could wish, indeed, that the works of
the Galiæans might perish but he requires an exact account even
of those theological volumes lest other treatises more valuable
should be confounded in their less Julian. Epist. ix. xxxvi.}

\pagenote[12011]{Julian himself says, that they tore him to
pieces like dogs, Epist. x.—M.}

\pagenote[121]{Philostorgius, with cautious malice, insinuates
their guilt, l. vii. c. ii. Godefroy p. 267.}

\pagenote[122]{Cineres projecit in mare, id metuens ut clamabat,
ne, collectis supremis, ædes illis exstruerentur ut reliquis, qui
deviare a religione compulsi, pertulere, cruciabiles pœnas,
adusque gloriosam mortem intemeratâ fide progressi, et nunc
Martyres appellantur. Ammian. xxii. 11. Epiphanius proves to the
Arians, that George was not a martyr.}

\pagenote[123]{Some Donatists (Optatus Milev. p. 60, 303, edit.
Dupin; and Tillemont, Mém. Eccles. tom. vi. p. 713, in 4to.) and
Priscillianists (Tillemont, Mém. Eccles. tom. viii. p. 517, in
4to.) have in like manner usurped the honors of the Catholic
saints and martyrs.}

\pagenote[124]{The saints of Cappadocia, Basil, and the
Gregories, were ignorant of their holy companion. Pope Gelasius,
(A. D. 494,) the first Catholic who acknowledges St. George,
places him among the martyrs “qui Deo magis quam hominibus noti
sunt.” He rejects his Acts as the composition of heretics. Some,
perhaps, not the oldest, of the spurious Acts, are still extant;
and, through a cloud of fiction, we may yet distinguish the
combat which St. George of Cappadocia sustained, in the presence
of Queen \textit{Alexandria}, against the \textit{magician Athanasius}.}

\pagenote[125]{This transformation is not given as absolutely
certain, but as \textit{extremely} probable. See the Longueruana, tom.
i. p. 194. ——Note: The late Dr. Milner (the Roman Catholic
bishop) wrote a tract to vindicate the existence and the
orthodoxy of the tutelar saint of England. He succeeds, I think,
in tracing the worship of St. George up to a period which makes
it improbable that so notorious an Arian could be palmed upon the
Catholic church as a saint and a martyr. The Acts rejected by
Gelasius may have been of Arian origin, and designed to ingraft
the story of their hero on the obscure adventures of some earlier
saint. See an Historical and Critical Inquiry into the Existence
and Character of Saint George, in a letter to the Earl of
Leicester, by the Rev. J. Milner. F. S. A. London 1792.—M.}

\pagenote[126]{A curious history of the worship of St. George,
from the sixth century, (when he was already revered in
Palestine, in Armenia at Rome, and at Treves in Gaul,) might be
extracted from Dr. Heylin (History of St. George, 2d edition,
London, 1633, in 4to. p. 429) and the Bollandists, (Act. Ss.
Mens. April. tom. iii. p. 100-163.) His fame and popularity in
Europe, and especially in England, proceeded from the Crusades.}

About the same time that Julian was informed of the tumult of
Alexandria, he received intelligence from Edessa, that the proud
and wealthy faction of the Arians had insulted the weakness of
the Valentinians, and committed such disorders as ought not to be
suffered with impunity in a well-regulated state. Without
expecting the slow forms of justice, the exasperated prince
directed his mandate to the magistrates of Edessa,\textsuperscript{127} by which
he confiscated the whole property of the church: the money was
distributed among the soldiers; the lands were added to the
domain; and this act of oppression was aggravated by the most
ungenerous irony. “I show myself,” says Julian, “the true friend
of the Galilæans. Their \textit{admirable} law has promised the kingdom
of heaven to the poor; and they will advance with more diligence
in the paths of virtue and salvation, when they are relieved by
my assistance from the load of temporal possessions. Take care,”
pursued the monarch, in a more serious tone, “take care how you
provoke my patience and humanity. If these disorders continue, I
will revenge on the magistrates the crimes of the people; and you
will have reason to dread, not only confiscation and exile, but
fire and the sword.” The tumults of Alexandria were doubtless of
a more bloody and dangerous nature: but a Christian bishop had
fallen by the hands of the Pagans; and the public epistle of
Julian affords a very lively proof of the partial spirit of his
administration. His reproaches to the citizens of Alexandria are
mingled with expressions of esteem and tenderness; and he
laments, that, on this occasion, they should have departed from
the gentle and generous manners which attested their Grecian
extraction. He gravely censures the offence which they had
committed against the laws of justice and humanity; but he
recapitulates, with visible complacency, the intolerable
provocations which they had so long endured from the impious
tyranny of George of Cappadocia. Julian admits the principle,
that a wise and vigorous government should chastise the insolence
of the people; yet, in consideration of their founder Alexander,
and of Serapis their tutelar deity, he grants a free and gracious
pardon to the guilty city, for which he again feels the affection
of a brother.\textsuperscript{128}

\pagenote[127]{Julian. Epist. xliii.}

\pagenote[128]{Julian. Epist. x. He allowed his friends to
assuage his anger Ammian. xxii. 11.}

After the tumult of Alexandria had subsided, Athanasius, amidst
the public acclamations, seated himself on the throne from whence
his unworthy competitor had been precipitated: and as the zeal of
the archbishop was tempered with discretion, the exercise of his
authority tended not to inflame, but to reconcile, the minds of
the people. His pastoral labors were not confined to the narrow
limits of Egypt. The state of the Christian world was present to
his active and capacious mind; and the age, the merit, the
reputation of Athanasius, enabled him to assume, in a moment of
danger, the office of Ecclesiastical Dictator.\textsuperscript{129} Three years
were not yet elapsed since the majority of the bishops of the
West had ignorantly, or reluctantly, subscribed the Confession of
Rimini. They repented, they believed, but they dreaded the
unseasonable rigor of their orthodox brethren; and if their pride
was stronger than their faith, they might throw themselves into
the arms of the Arians, to escape the indignity of a public
penance, which must degrade them to the condition of obscure
laymen. At the same time the domestic differences concerning the
union and distinction of the divine persons, were agitated with
some heat among the Catholic doctors; and the progress of this
metaphysical controversy seemed to threaten a public and lasting
division of the Greek and Latin churches. By the wisdom of a
select synod, to which the name and presence of Athanasius gave
the authority of a general council, the bishops, who had unwarily
deviated into error, were admitted to the communion of the
church, on the easy condition of subscribing the Nicene Creed;
without any formal acknowledgment of their past fault, or any
minute definition of their scholastic opinions. The advice of the
primate of Egypt had already prepared the clergy of Gaul and
Spain, of Italy and Greece, for the reception of this salutary
measure; and, notwithstanding the opposition of some ardent
spirits,\textsuperscript{130} the fear of the common enemy promoted the peace and
harmony of the Christians.\textsuperscript{131}

\pagenote[129]{See Athanas. ad Rufin. tom. ii. p. 40, 41, and
Greg. Nazianzen Orat. iii. p. 395, 396; who justly states the
temperate zeal of the primate, as much more meritorious than his
prayers, his fasts, his persecutions, \&c.}

\pagenote[130]{I have not leisure to follow the blind obstinacy
of Lucifer of Cagliari. See his adventures in Tillemont, (Mém.
Eccles. tom. vii. p. 900-926;) and observe how the color of the
narrative insensibly changes, as the confessor becomes a
schismatic.}

\pagenote[131]{Assensus est huic sententiæ Occidens, et, per tam
necessarium conilium, Satanæ faucibus mundus ereptus. The lively
and artful dialogue of Jerom against the Luciferians (tom. ii. p.
135-155) exhibits an original picture of the ecclesiastical
policy of the times.}

The skill and diligence of the primate of Egypt had improved the
season of tranquillity, before it was interrupted by the hostile
edicts of the emperor.\textsuperscript{132} Julian, who despised the Christians,
honored Athanasius with his sincere and peculiar hatred. For his
sake alone, he introduced an arbitrary distinction, repugnant at
least to the spirit of his former declarations. He maintained,
that the Galilæans, whom he had recalled from exile, were not
restored, by that general indulgence, to the possession of their
respective churches; and he expressed his astonishment, that a
criminal, who had been repeatedly condemned by the judgment of
the emperors, should dare to insult the majesty of the laws, and
insolently usurp the archiepiscopal throne of Alexandria, without
expecting the orders of his sovereign. As a punishment for the
imaginary offence, he again banished Athanasius from the city;
and he was pleased to suppose, that this act of justice would be
highly agreeable to his pious subjects. The pressing
solicitations of the people soon convinced him, that the majority
of the Alexandrians were Christians; and that the greatest part
of the Christians were firmly attached to the cause of their
oppressed primate. But the knowledge of their sentiments, instead
of persuading him to recall his decree, provoked him to extend to
all Egypt the term of the exile of Athanasius. The zeal of the
multitude rendered Julian still more inexorable: he was alarmed
by the danger of leaving at the head of a tumultuous city, a
daring and popular leader; and the language of his resentment
discovers the opinion which he entertained of the courage and
abilities of Athanasius. The execution of the sentence was still
delayed, by the caution or negligence of Ecdicius, præfect of
Egypt, who was at length awakened from his lethargy by a severe
reprimand. “Though you neglect,” says Julian, “to write to me on
any other subject, at least it is your duty to inform me of your
conduct towards Athanasius, the enemy of the gods. My intentions
have been long since communicated to you. I swear by the great
Serapis, that unless, on the calends of December, Athanasius has
departed from Alexandria, nay, from Egypt, the officers of your
government shall pay a fine of one hundred pounds of gold. You
know my temper: I am slow to condemn, but I am still slower to
forgive.” This epistle was enforced by a short postscript,
written with the emperor’s own hand. “The contempt that is shown
for all the gods fills me with grief and indignation. There is
nothing that I should see, nothing that I should hear, with more
pleasure, than the expulsion of Athanasius from all Egypt. The
abominable wretch! Under my reign, the baptism of several Grecian
ladies of the highest rank has been the effect of his
persecutions.”\textsuperscript{133} The death of Athanasius was not \textit{expressly}
commanded; but the præfect of Egypt understood that it was safer
for him to exceed, than to neglect, the orders of an irritated
master. The archbishop prudently retired to the monasteries of
the Desert; eluded, with his usual dexterity, the snares of the
enemy; and lived to triumph over the ashes of a prince, who, in
words of formidable import, had declared his wish that the whole
venom of the Galilæan school were contained in the single person
of Athanasius.\textsuperscript{134} \textsuperscript{13411}

\pagenote[132]{Tillemont, who supposes that George was massacred
in August crowds the actions of Athanasius into a narrow space,
(Mém. Eccles. tom. viii. p. 360.) An original fragment, published
by the Marquis Maffei, from the old Chapter library of Verona,
(Osservazioni Letterarie, tom. iii. p. 60-92,) affords many
important dates, which are authenticated by the computation of
Egyptian months.}

\pagenote[133]{I have preserved the ambiguous sense of the last
word, the ambiguity of a tyrant who wished to find, or to create,
guilt.}

\pagenote[134]{The three epistles of Julian, which explain his
intentions and conduct with regard to Athanasius, should be
disposed in the following chronological order, xxvi. x. vi. * See
likewise, Greg. Nazianzen xxi. p. 393. Sozomen, l. v. c. 15.
Socrates, l. iii. c. 14. Theodoret, l iii. c. 9, and Tillemont,
Mém. Eccles. tom. viii. p. 361-368, who has used some materials
prepared by the Bollandists.}

\pagenote[13411]{The sentence in the text is from Epist. li.
addressed to the people of Alexandria.—M.}

I have endeavored faithfully to represent the artful system by
which Julian proposed to obtain the effects, without incurring
the guilt, or reproach, of persecution. But if the deadly spirit
of fanaticism perverted the heart and understanding of a virtuous
prince, it must, at the same time, be confessed that the \textit{real}
sufferings of the Christians were inflamed and magnified by human
passions and religious enthusiasm. The meekness and resignation
which had distinguished the primitive disciples of the gospel,
was the object of the applause, rather than of the imitation of
their successors. The Christians, who had now possessed above
forty years the civil and ecclesiastical government of the
empire, had contracted the insolent vices of prosperity,\textsuperscript{135} and
the habit of believing that the saints alone were entitled to
reign over the earth. As soon as the enmity of Julian deprived
the clergy of the privileges which had been conferred by the
favor of Constantine, they complained of the most cruel
oppression; and the free toleration of idolaters and heretics was
a subject of grief and scandal to the orthodox party.\textsuperscript{136} The
acts of violence, which were no longer countenanced by the
magistrates, were still committed by the zeal of the people. At
Pessinus, the altar of Cybele was overturned almost in the
presence of the emperor; and in the city of Cæsarea in
Cappadocia, the temple of Fortune, the sole place of worship
which had been left to the Pagans, was destroyed by the rage of a
popular tumult. On these occasions, a prince, who felt for the
honor of the gods, was not disposed to interrupt the course of
justice; and his mind was still more deeply exasperated, when he
found that the fanatics, who had deserved and suffered the
punishment of incendiaries, were rewarded with the honors of
martyrdom.\textsuperscript{137} The Christian subjects of Julian were assured of
the hostile designs of their sovereign; and, to their jealous
apprehension, every circumstance of his government might afford
some grounds of discontent and suspicion. In the ordinary
administration of the laws, the Christians, who formed so large a
part of the people, must frequently be condemned: but their
indulgent brethren, without examining the merits of the cause,
presumed their innocence, allowed their claims, and imputed the
severity of their judge to the partial malice of religious
persecution.\textsuperscript{138} These present hardships, intolerable as they
might appear, were represented as a slight prelude of the
impending calamities. The Christians considered Julian as a cruel
and crafty tyrant; who suspended the execution of his revenge
till he should return victorious from the Persian war. They
expected, that as soon as he had triumphed over the foreign
enemies of Rome, he would lay aside the irksome mask of
dissimulation; that the amphitheatre would stream with the blood
of hermits and bishops; and that the Christians who still
persevered in the profession of the faith, would be deprived of
the common benefits of nature and society.\textsuperscript{139} Every calumny\textsuperscript{140}
that could wound the reputation of the Apostate, was credulously
embraced by the fears and hatred of his adversaries; and their
indiscreet clamors provoked the temper of a sovereign, whom it
was their duty to respect, and their interest to flatter.

They still protested, that prayers and tears were their only
weapons against the impious tyrant, whose head they devoted to
the justice of offended Heaven. But they insinuated, with sullen
resolution, that their submission was no longer the effect of
weakness; and that, in the imperfect state of human virtue, the
patience, which is founded on principle, may be exhausted by
persecution. It is impossible to determine how far the zeal of
Julian would have prevailed over his good sense and humanity; but
if we seriously reflect on the strength and spirit of the church,
we shall be convinced, that before the emperor could have
extinguished the religion of Christ, he must have involved his
country in the horrors of a civil war.\textsuperscript{141}

\pagenote[135]{See the fair confession of Gregory, (Orat. iii. p.
61, 62.)}

\pagenote[136]{Hear the furious and absurd complaint of Optatus,
(de Schismat Denatist. l. ii. c. 16, 17.)}

\pagenote[137]{Greg. Nazianzen, Orat. iii. p. 91, iv. p. 133. He
praises the rioters of Cæsarea. See Sozomen, l. v. 4, 11.
Tillemont (Mém. Eccles. tom. vii. p. 649, 650) owns, that their
behavior was not dans l’ordre commun: but he is perfectly
satisfied, as the great St. Basil always celebrated the festival
of these blessed martyrs.}

\pagenote[138]{Julian determined a lawsuit against the new
Christian city at Maiuma, the port of Gaza; and his sentence,
though it might be imputed to bigotry, was never reversed by his
successors. Sozomen, l. v. c. 3. Reland, Palestin. tom. ii. p.
791.}

\pagenote[139]{Gregory (Orat. iii. p. 93, 94, 95. Orat. iv. p.
114) pretends to speak from the information of Julian’s
confidants, whom Orosius (vii. 30) could not have seen.}

\pagenote[140]{Gregory (Orat. iii. p. 91) charges the Apostate
with secret sacrifices of boys and girls; and positively affirms,
that the dead bodies were thrown into the Orontes. See Theodoret,
l. iii. c. 26, 27; and the equivocal candor of the Abbé de la
Bleterie, Vie de Julien, p. 351, 352. Yet \textit{contemporary} malice
could not impute to Julian the troops of martyrs, more especially
in the West, which Baronius so greedily swallows, and Tillemont
so faintly rejects, (Mém. Eccles. tom. vii. p. 1295-1315.)}

\pagenote[141]{The resignation of Gregory is truly edifying,
(Orat. iv. p. 123, 124.) Yet, when an officer of Julian attempted
to seize the church of Nazianzus, he would have lost his life, if
he had not yielded to the zeal of the bishop and people, (Orat.
xix. p. 308.) See the reflections of Chrysostom, as they are
alleged by Tillemont, (Mém. Eccles. tom. vii. p. 575.)}

