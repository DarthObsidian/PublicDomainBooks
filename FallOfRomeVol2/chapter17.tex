\chapter{Foundation Of Constantinople.}
\section{Part \thesection.}

\textit{Foundation Of Constantinople. — Political System Constantine, And
His Successors. — Military Discipline. — The Palace. — The Finances.}
\vspace{\onelineskip}

The unfortunate Licinius was the last rival who opposed the
greatness, and the last captive who adorned the triumph, of
Constantine. After a tranquil and prosperous reign, the conquerer
bequeathed to his family the inheritance of the Roman empire; a
new capital, a new policy, and a new religion; and the
innovations which he established have been embraced and
consecrated by succeeding generations. The age of the great
Constantine and his sons is filled with important events; but the
historian must be oppressed by their number and variety, unless
he diligently separates from each other the scenes which are
connected only by the order of time. He will describe the
political institutions that gave strength and stability to the
empire, before he proceeds to relate the wars and revolutions
which hastened its decline. He will adopt the division unknown to
the ancients of civil and ecclesiastical affairs: the victory of
the Christians, and their intestine discord, will supply copious
and distinct materials both for edification and for scandal.

After the defeat and abdication of Licinius, his victorious rival
proceeded to lay the foundations of a city destined to reign in
future times, the mistress of the East, and to survive the empire
and religion of Constantine. The motives, whether of pride or of
policy, which first induced Diocletian to withdraw himself from
the ancient seat of government, had acquired additional weight by
the example of his successors, and the habits of forty years.
Rome was insensibly confounded with the dependent kingdoms which
had once acknowledged her supremacy; and the country of the
Cæsars was viewed with cold indifference by a martial prince,
born in the neighborhood of the Danube, educated in the courts
and armies of Asia, and invested with the purple by the legions
of Britain. The Italians, who had received Constantine as their
deliverer, submissively obeyed the edicts which he sometimes
condescended to address to the senate and people of Rome; but
they were seldom honored with the presence of their new
sovereign. During the vigor of his age, Constantine, according to
the various exigencies of peace and war, moved with slow dignity,
or with active diligence, along the frontiers of his extensive
dominions; and was always prepared to take the field either
against a foreign or a domestic enemy. But as he gradually
reached the summit of prosperity and the decline of life, he
began to meditate the design of fixing in a more permanent
station the strength as well as majesty of the throne. In the
choice of an advantageous situation, he preferred the confines of
Europe and Asia; to curb with a powerful arm the barbarians who
dwelt between the Danube and the Tanais; to watch with an eye of
jealousy the conduct of the Persian monarch, who indignantly
supported the yoke of an ignominious treaty. With these views,
Diocletian had selected and embellished the residence of
Nicomedia: but the memory of Diocletian was justly abhorred by
the protector of the church: and Constantine was not insensible
to the ambition of founding a city which might perpetuate the
glory of his own name. During the late operations of the war
against Licinius, he had sufficient opportunity to contemplate,
both as a soldier and as a statesman, the incomparable position
of Byzantium; and to observe how strongly it was guarded by
nature against a hostile attack, whilst it was accessible on
every side to the benefits of commercial intercourse. Many ages
before Constantine, one of the most judicious historians of
antiquity\textsuperscript{1} had described the advantages of a situation, from
whence a feeble colony of Greeks derived the command of the sea,
and the honors of a flourishing and independent republic.\textsuperscript{2}

\pagenote[1]{Polybius, l. iv. p. 423, edit. Casaubon. He observes
that the peace of the Byzantines was frequently disturbed, and
the extent of their territory contracted, by the inroads of the
wild Thracians.}

\pagenote[2]{The navigator Byzas, who was styled the son of
Neptune, founded the city 656 years before the Christian æra. His
followers were drawn from Argos and Megara. Byzantium was
afterwards rebuild and fortified by the Spartan general
Pausanias. See Scaliger Animadvers. ad Euseb. p. 81. Ducange,
Constantinopolis, l. i part i. cap 15, 16. With regard to the
wars of the Byzantines against Philip, the Gauls, and the kings
of Bithynia, we should trust none but the ancient writers who
lived before the greatness of the Imperial city had excited a
spirit of flattery and fiction.}

If we survey Byzantium in the extent which it acquired with the
august name of Constantinople, the figure of the Imperial city
may be represented under that of an unequal triangle. The obtuse
point, which advances towards the east and the shores of Asia,
meets and repels the waves of the Thracian Bosphorus. The
northern side of the city is bounded by the harbor; and the
southern is washed by the Propontis, or Sea of Marmara. The basis
of the triangle is opposed to the west, and terminates the
continent of Europe. But the admirable form and division of the
circumjacent land and water cannot, without a more ample
explanation, be clearly or sufficiently understood. The winding
channel through which the waters of the Euxine flow with a rapid
and incessant course towards the Mediterranean, received the
appellation of Bosphorus, a name not less celebrated in the
history, than in the fables, of antiquity.\textsuperscript{3} A crowd of temples
and of votive altars, profusely scattered along its steep and
woody banks, attested the unskilfulness, the terrors, and the
devotion of the Grecian navigators, who, after the example of the
Argonauts, explored the dangers of the inhospitable Euxine. On
these banks tradition long preserved the memory of the palace of
Phineus, infested by the obscene harpies;\textsuperscript{4} and of the sylvan
reign of Amycus, who defied the son of Leda to the combat of the
cestus.\textsuperscript{5} The straits of the Bosphorus are terminated by the
Cyanean rocks, which, according to the description of the poets,
had once floated on the face of the waters; and were destined by
the gods to protect the entrance of the Euxine against the eye of
profane curiosity.\textsuperscript{6} From the Cyanean rocks to the point and
harbor of Byzantium, the winding length of the Bosphorus extends
about sixteen miles,\textsuperscript{7} and its most ordinary breadth may be
computed at about one mile and a half. The new castles of Europe
and Asia are constructed, on either continent, upon the
foundations of two celebrated temples, of Serapis and of Jupiter
Urius. The \textit{old} castles, a work of the Greek emperors, command
the narrowest part of the channel in a place where the opposite
banks advance within five hundred paces of each other. These
fortresses were destroyed and strengthened by Mahomet the Second,
when he meditated the siege of Constantinople:\textsuperscript{8} but the Turkish
conqueror was most probably ignorant, that near two thousand
years before his reign, Darius had chosen the same situation to
connect the two continents by a bridge of boats.\textsuperscript{9} At a small
distance from the old castles we discover the little town of
Chrysopolis, or Scutari, which may almost be considered as the
Asiatic suburb of Constantinople. The Bosphorus, as it begins to
open into the Propontis, passes between Byzantium and Chalcedon.
The latter of those cities was built by the Greeks, a few years
before the former; and the blindness of its founders, who
overlooked the superior advantages of the opposite coast, has
been stigmatized by a proverbial expression of contempt.\textsuperscript{10}

\pagenote[3]{The Bosphorus has been very minutely described by
Dionysius of Byzantium, who lived in the time of Domitian,
(Hudson, Geograph Minor, tom. iii.,) and by Gilles or Gyllius, a
French traveller of the XVIth century. Tournefort (Lettre XV.)
seems to have used his own eyes, and the learning of Gyllius. Add
Von Hammer, Constantinopolis und der Bosphoros, 8vo.—M.}

\pagenote[4]{There are very few conjectures so happy as that of
Le Clere, (Bibliotehque Universelle, tom. i. p. 148,) who
supposes that the harpies were only locusts. The Syriac or
Phœnician name of those insects, their noisy flight, the stench
and devastation which they occasion, and the north wind which
drives them into the sea, all contribute to form the striking
resemblance.}

\pagenote[5]{The residence of Amycus was in Asia, between the old
and the new castles, at a place called Laurus Insana. That of
Phineus was in Europe, near the village of Mauromole and the
Black Sea. See Gyllius de Bosph. l. ii. c. 23. Tournefort, Lettre
XV.}

\pagenote[6]{The deception was occasioned by several pointed
rocks, alternately sovered and abandoned by the waves. At present
there are two small islands, one towards either shore; that of
Europe is distinguished by the column of Pompey.}

\pagenote[7]{The ancients computed one hundred and twenty stadia,
or fifteen Roman miles. They measured only from the new castles,
but they carried the straits as far as the town of Chalcedon.}

\pagenote[8]{Ducas. Hist. c. 34. Leunclavius Hist. Turcica
Mussulmanica, l. xv. p. 577. Under the Greek empire these castles
were used as state prisons, under the tremendous name of Lethe,
or towers of oblivion.}

\pagenote[9]{Darius engraved in Greek and Assyrian letters, on
two marble columns, the names of his subject nations, and the
amazing numbers of his land and sea forces. The Byzantines
afterwards transported these columns into the city, and used them
for the altars of their tutelar deities. Herodotus, l. iv. c.
87.}

\pagenote[10]{Namque arctissimo inter Europam Asiamque divortio
Byzantium in extremâ Europâ posuere Greci, quibus, Pythium
Apollinem consulentibus ubi conderent urbem, redditum oraculum
est, quærerent sedem \textit{cæcerum} terris adversam. Ea ambage
Chalcedonii monstrabantur quod priores illuc advecti, prævisâ
locorum utilitate pejora legissent Tacit. Annal. xii. 63.}

The harbor of Constantinople, which may be considered as an arm
of the Bosphorus, obtained, in a very remote period, the
denomination of the \textit{Golden Horn}. The curve which it describes
might be compared to the horn of a stag, or as it should seem,
with more propriety, to that of an ox.\textsuperscript{11} The epithet of \textit{golden}
was expressive of the riches which every wind wafted from the
most distant countries into the secure and capacious port of
Constantinople. The River Lycus, formed by the conflux of two
little streams, pours into the harbor a perpetual supply of fresh
water, which serves to cleanse the bottom, and to invite the
periodical shoals of fish to seek their retreat in that
convenient recess. As the vicissitudes of tides are scarcely felt
in those seas, the constant depth of the harbor allows goods to
be landed on the quays without the assistance of boats; and it
has been observed, that in many places the largest vessels may
rest their prows against the houses, while their sterns are
floating in the water.\textsuperscript{12} From the mouth of the Lycus to that of
the harbor, this arm of the Bosphorus is more than seven miles in
length. The entrance is about five hundred yards broad, and a
strong chain could be occasionally drawn across it, to guard the
port and city from the attack of a hostile navy.\textsuperscript{13}

\pagenote[11]{Strabo, l. vii. p. 492, [edit. Casaub.] Most of the
antlers are now broken off; or, to speak less figuratively, most
of the recesses of the harbor are filled up. See Gill. de
Bosphoro Thracio, l. i. c. 5.}

\pagenote[12]{Procopius de Ædificiis, l. i. c. 5. His description
is confirmed by modern travellers. See Thevenot, part i. l. i. c.
15. Tournefort, Lettre XII. Niebuhr, Voyage d’Arabie, p. 22.}

\pagenote[13]{See Ducange, C. P. l. i. part i. c. 16, and his
Observations sur Villehardouin, p. 289. The chain was drawn from
the Acropolis near the modern Kiosk, to the tower of Galata; and
was supported at convenient distances by large wooden piles.}

Between the Bosphorus and the Hellespont, the shores of Europe
and Asia, receding on either side, enclose the sea of Marmara,
which was known to the ancients by the denomination of Propontis.
The navigation from the issue of the Bosphorus to the entrance of
the Hellespont is about one hundred and twenty miles.

Those who steer their westward course through the middle of the
Propontis, may at once descry the high lands of Thrace and
Bithynia, and never lose sight of the lofty summit of Mount
Olympus, covered with eternal snows.\textsuperscript{14} They leave on the left a
deep gulf, at the bottom of which Nicomedia was seated, the
Imperial residence of Diocletian; and they pass the small islands
of Cyzicus and Proconnesus before they cast anchor at Gallipoli;
where the sea, which separates Asia from Europe, is again
contracted into a narrow channel.

\pagenote[14]{Thevenot (Voyages au Levant, part i. l. i. c. 14)
contracts the measure to 125 small Greek miles. Belon
(Observations, l. ii. c. 1.) gives a good description of the
Propontis, but contents himself with the vague expression of one
day and one night’s sail. When Sandy’s (Travels, p. 21) talks of
150 furlongs in length, as well as breadth we can only suppose
some mistake of the press in the text of that judicious
traveller.}

The geographers who, with the most skilful accuracy, have
surveyed the form and extent of the Hellespont, assign about
sixty miles for the winding course, and about three miles for the
ordinary breadth of those celebrated straits.\textsuperscript{15} But the
narrowest part of the channel is found to the northward of the
old Turkish castles between the cities of Sestus and Abydus. It
was here that the adventurous Leander braved the passage of the
flood for the possession of his mistress.\textsuperscript{16} It was here
likewise, in a place where the distance between the opposite
banks cannot exceed five hundred paces, that Xerxes imposed a
stupendous bridge of boats, for the purpose of transporting into
Europe a hundred and seventy myriads of barbarians.\textsuperscript{17} A sea
contracted within such narrow limits may seem but ill to deserve
the singular epithet of \textit{broad}, which Homer, as well as Orpheus,
has frequently bestowed on the Hellespont.\textsuperscript{1711} But our ideas of
greatness are of a relative nature: the traveller, and especially
the poet, who sailed along the Hellespont, who pursued the
windings of the stream, and contemplated the rural scenery, which
appeared on every side to terminate the prospect, insensibly lost
the remembrance of the sea; and his fancy painted those
celebrated straits, with all the attributes of a mighty river
flowing with a swift current, in the midst of a woody and inland
country, and at length, through a wide mouth, discharging itself
into the Ægean or Archipelago.\textsuperscript{18} Ancient Troy,\textsuperscript{19} seated on a an
eminence at the foot of Mount Ida, overlooked the mouth of the
Hellespont, which scarcely received an accession of waters from
the tribute of those immortal rivulets the Simois and Scamander.
The Grecian camp had stretched twelve miles along the shore from
the Sigæan to the Rhætean promontory; and the flanks of the army
were guarded by the bravest chiefs who fought under the banners
of Agamemnon. The first of those promontories was occupied by
Achilles with his invincible myrmidons, and the dauntless Ajax
pitched his tents on the other. After Ajax had fallen a sacrifice
to his disappointed pride, and to the ingratitude of the Greeks,
his sepulchre was erected on the ground where he had defended the
navy against the rage of Jove and of Hector; and the citizens of
the rising town of Rhæteum celebrated his memory with divine
honors.\textsuperscript{20} Before Constantine gave a just preference to the
situation of Byzantium, he had conceived the design of erecting
the seat of empire on this celebrated spot, from whence the
Romans derived their fabulous origin. The extensive plain which
lies below ancient Troy, towards the Rhætean promontory and the
tomb of Ajax, was first chosen for his new capital; and though
the undertaking was soon relinquished the stately remains of
unfinished walls and towers attracted the notice of all who
sailed through the straits of the Hellespont.\textsuperscript{21}

\pagenote[15]{See an admirable dissertation of M. d’Anville upon
the Hellespont or Dardanelles, in the Mémoires tom. xxviii. p.
318—346. Yet even that ingenious geographer is too fond of
supposing new, and perhaps imaginary \textit{measures}, for the purpose
of rendering ancient writers as accurate as himself. The stadia
employed by Herodotus in the description of the Euxine, the
Bosphorus, \&c., (l. iv. c. 85,) must undoubtedly be all of the
same species; but it seems impossible to reconcile them either
with truth or with each other.}

\pagenote[16]{The oblique distance between Sestus and Abydus was
thirty stadia. The improbable tale of Hero and Leander is exposed
by M. Mahudel, but is defended on the authority of poets and
medals by M. de la Nauze. See the Académie des Inscriptions, tom.
vii. Hist. p. 74. elem. p. 240. Note: The practical illustration
of the possibility of Leander’s feat by Lord Byron and other
English swimmers is too well known to need particularly
reference—M.}

\pagenote[17]{See the seventh book of Herodotus, who has erected
an elegant trophy to his own fame and to that of his country. The
review appears to have been made with tolerable accuracy; but the
vanity, first of the Persians, and afterwards of the Greeks, was
interested to magnify the armament and the victory. I should much
doubt whether the \textit{invaders} have ever outnumbered the \textit{men} of
any country which they attacked.}

\pagenote[1711]{Gibbon does not allow greater width between the
two nearest points of the shores of the Hellespont than between
those of the Bosphorus; yet all the ancient writers speak of the
Hellespontic strait as broader than the other: they agree in
giving it seven stadia in its narrowest width, (Herod. in Melp.
c. 85. Polym. c. 34. Strabo, p. 591. Plin. iv. c. 12.) which make
875 paces. It is singular that Gibbon, who in the fifteenth note
of this chapter reproaches d’Anville with being fond of supposing
new and perhaps imaginary measures, has here adopted the peculiar
measurement which d’Anville has assigned to the stadium. This
great geographer believes that the ancients had a stadium of
fifty-one toises, and it is that which he applies to the walls of
Babylon. Now, seven of these stadia are equal to about 500 paces,
7 stadia = 2142 feet: 500 paces = 2135 feet 5 inches.—G. See
Rennell, Geog. of Herod. p. 121. Add Ukert, Geographie der
Griechen und Romer, v. i. p. 2, 71.—M.}

\pagenote[18]{See Wood’s Observations on Homer, p. 320. I have,
with pleasure, selected this remark from an author who in general
seems to have disappointed the expectation of the public as a
critic, and still more as a traveller. He had visited the banks
of the Hellespont; and had read Strabo; he ought to have
consulted the Roman itineraries. How was it possible for him to
confound Ilium and Alexandria Troas, (Observations, p. 340, 341,)
two cities which were sixteen miles distant from each other? *
Note: Compare Walpole’s Memoirs on Turkey, v. i. p. 101. Dr.
Clarke adopted Mr. Walpole’s interpretation of the salt
Hellespont. But the old interpretation is more graphic and
Homeric. Clarke’s Travels, ii. 70.—M.}

\pagenote[19]{Demetrius of Scepsis wrote sixty books on thirty
lines of Homer’s catalogue. The XIIIth Book of Strabo is
sufficient for \textit{our} curiosity.}

\pagenote[20]{Strabo, l. xiii. p. 595, [890, edit. Casaub.] The
disposition of the ships, which were drawn upon dry land, and the
posts of Ajax and Achilles, are very clearly described by Homer.
See Iliad, ix. 220.}

\pagenote[21]{Zosim. l. ii. [c. 30,] p. 105. Sozomen, l. ii. c.
3. Theophanes, p. 18. Nicephorus Callistus, l. vii. p. 48.
Zonaras, tom. ii. l. xiii. p. 6. Zosimus places the new city
between Ilium and Alexandria, but this apparent difference may be
reconciled by the large extent of its circumference. Before the
foundation of Constantinople, Thessalonica is mentioned by
Cedrenus, (p. 283,) and Sardica by Zonaras, as the intended
capital. They both suppose with very little probability, that the
emperor, if he had not been prevented by a prodigy, would have
repeated the mistake of the \textit{blind} Chalcedonians.}

We are at present qualified to view the advantageous position of
Constantinople; which appears to have been formed by nature for
the centre and capital of a great monarchy. Situated in the
forty-first degree of latitude, the Imperial city commanded, from
her seven hills,\textsuperscript{22} the opposite shores of Europe and Asia; the
climate was healthy and temperate, the soil fertile, the harbor
secure and capacious; and the approach on the side of the
continent was of small extent and easy defence. The Bosphorus and
the Hellespont may be considered as the two gates of
Constantinople; and the prince who possessed those important
passages could always shut them against a naval enemy, and open
them to the fleets of commerce. The preservation of the eastern
provinces may, in some degree, be ascribed to the policy of
Constantine, as the barbarians of the Euxine, who in the
preceding age had poured their armaments into the heart of the
Mediterranean, soon desisted from the exercise of piracy, and
despaired of forcing this insurmountable barrier. When the gates
of the Hellespont and Bosphorus were shut, the capital still
enjoyed within their spacious enclosure every production which
could supply the wants, or gratify the luxury, of its numerous
inhabitants. The sea-coasts of Thrace and Bithynia, which
languish under the weight of Turkish oppression, still exhibit a
rich prospect of vineyards, of gardens, and of plentiful
harvests; and the Propontis has ever been renowned for an
inexhaustible store of the most exquisite fish, that are taken in
their stated seasons, without skill, and almost without labor.\textsuperscript{23}
But when the passages of the straits were thrown open for trade,
they alternately admitted the natural and artificial riches of
the north and south, of the Euxine, and of the Mediterranean.
Whatever rude commodities were collected in the forests of
Germany and Scythia, and far as the sources of the Tanais and the
Borysthenes; whatsoever was manufactured by the skill of Europe
or Asia; the corn of Egypt, and the gems and spices of the
farthest India, were brought by the varying winds into the port
of Constantinople, which for many ages attracted the commerce of
the ancient world.\textsuperscript{24}

[See Basilica Of Constantinople]

\pagenote[22]{Pocock’s Description of the East, vol. ii. part ii.
p. 127. His plan of the seven hills is clear and accurate. That
traveller is seldom unsatisfactory.}

\pagenote[23]{See Belon, Observations, c. 72—76. Among a variety
of different species, the Pelamides, a sort of Thunnies, were the
most celebrated. We may learn from Polybius, Strabo, and Tacitus,
that the profits of the fishery constituted the principal revenue
of Byzantium.}

\pagenote[24]{See the eloquent description of Busbequius,
epistol. i. p. 64. Est in Europa; habet in conspectu Asiam,
Egyptum. Africamque a dextrâ: quæ tametsi contiguæ non sunt,
maris tamen navigandique commoditate veluti junguntur. A sinistra
vero Pontus est Euxinus, \&c.}

The prospect of beauty, of safety, and of wealth, united in a
single spot, was sufficient to justify the choice of Constantine.
But as some decent mixture of prodigy and fable has, in every
age, been supposed to reflect a becoming majesty on the origin of
great cities,\textsuperscript{25} the emperor was desirous of ascribing his
resolution, not so much to the uncertain counsels of human
policy, as to the infallible and eternal decrees of divine
wisdom. In one of his laws he has been careful to instruct
posterity, that in obedience to the commands of God, he laid the
everlasting foundations of Constantinople:\textsuperscript{26} and though he has
not condescended to relate in what manner the celestial
inspiration was communicated to his mind, the defect of his
modest silence has been liberally supplied by the ingenuity of
succeeding writers; who describe the nocturnal vision which
appeared to the fancy of Constantine, as he slept within the
walls of Byzantium. The tutelar genius of the city, a venerable
matron sinking under the weight of years and infirmities, was
suddenly transformed into a blooming maid, whom his own hands
adorned with all the symbols of Imperial greatness.\textsuperscript{27} The
monarch awoke, interpreted the auspicious omen, and obeyed,
without hesitation, the will of Heaven. The day which gave birth
to a city or colony was celebrated by the Romans with such
ceremonies as had been ordained by a generous superstition;\textsuperscript{28}
and though Constantine might omit some rites which savored too
strongly of their Pagan origin, yet he was anxious to leave a
deep impression of hope and respect on the minds of the
spectators. On foot, with a lance in his hand, the emperor
himself led the solemn procession; and directed the line, which
was traced as the boundary of the destined capital: till the
growing circumference was observed with astonishment by the
assistants, who, at length, ventured to observe, that he had
already exceeded the most ample measure of a great city. “I shall
still advance,” replied Constantine, “till He, the invisible
guide who marches before me, thinks proper to stop.”\textsuperscript{29} Without
presuming to investigate the nature or motives of this
extraordinary conductor, we shall content ourselves with the more
humble task of describing the extent and limits of
Constantinople.\textsuperscript{30}

\pagenote[25]{Datur hæc venia antiquitati, ut miscendo humana
divinis, primordia urbium augustiora faciat. T. Liv. in proœm.}

\pagenote[26]{He says in one of his laws, pro commoditate urbis
quam æterno nomine, jubente Deo, donavimus. Cod. Theodos. l.
xiii. tit. v. leg. 7.}

\pagenote[27]{The Greeks, Theophanes, Cedrenus, and the author of
the Alexandrian Chronicle, confine themselves to vague and
general expressions. For a more particular account of the vision,
we are obliged to have recourse to such Latin writers as William
of Malmesbury. See Ducange, C. P. l. i. p. 24, 25.}

\pagenote[28]{See Plutarch in Romul. tom. i. p. 49, edit. Bryan.
Among other ceremonies, a large hole, which had been dug for that
purpose, was filled up with handfuls of earth, which each of the
settlers brought from the place of his birth, and thus adopted
his new country.}

\pagenote[29]{Philostorgius, l. ii. c. 9. This incident, though
borrowed from a suspected writer, is characteristic and
probable.}

\pagenote[30]{See in the Mémoires de l’Académie, tom. xxxv p.
747-758, a dissertation of M. d’Anville on the extent of
Constantinople. He takes the plan inserted in the Imperium
Orientale of Banduri as the most complete; but, by a series of
very nice observations, he reduced the extravagant proportion of
the scale, and instead of 9500, determines the circumference of
the city as consisting of about 7800 French \textit{toises}.}

In the actual state of the city, the palace and gardens of the
Seraglio occupy the eastern promontory, the first of the seven
hills, and cover about one hundred and fifty acres of our own
measure. The seat of Turkish jealousy and despotism is erected on
the foundations of a Grecian republic; but it may be supposed
that the Byzantines were tempted by the conveniency of the harbor
to extend their habitations on that side beyond the modern limits
of the Seraglio. The new walls of Constantine stretched from the
port to the Propontis across the enlarged breadth of the
triangle, at the distance of fifteen stadia from the ancient
fortification; and with the city of Byzantium they enclosed five
of the seven hills, which, to the eyes of those who approach
Constantinople, appear to rise above each other in beautiful
order.\textsuperscript{31} About a century after the death of the founder, the new
buildings, extending on one side up the harbor, and on the other
along the Propontis, already covered the narrow ridge of the
sixth, and the broad summit of the seventh hill. The necessity of
protecting those suburbs from the incessant inroads of the
barbarians engaged the younger Theodosius to surround his capital
with an adequate and permanent enclosure of walls.\textsuperscript{32} From the
eastern promontory to the golden gate, the extreme length of
Constantinople was about three Roman miles;\textsuperscript{33} the circumference
measured between ten and eleven; and the surface might be
computed as equal to about two thousand English acres. It is
impossible to justify the vain and credulous exaggerations of
modern travellers, who have sometimes stretched the limits of
Constantinople over the adjacent villages of the European, and
even of the Asiatic coast.\textsuperscript{34} But the suburbs of Pera and Galata,
though situate beyond the harbor, may deserve to be considered as
a part of the city;\textsuperscript{35} and this addition may perhaps authorize
the measure of a Byzantine historian, who assigns sixteen Greek
(about fourteen Roman) miles for the circumference of his native
city.\textsuperscript{36} Such an extent may not seem unworthy of an Imperial
residence. Yet Constantinople must yield to Babylon and Thebes,\textsuperscript{37}
to ancient Rome, to London, and even to Paris.\textsuperscript{38}

\pagenote[31]{Codinus, Antiquitat. Const. p. 12. He assigns the
church of St. Anthony as the boundary on the side of the harbor.
It is mentioned in Ducange, l. iv. c. 6; but I have tried,
without success, to discover the exact place where it was
situated.}

\pagenote[32]{The new wall of Theodosius was constructed in the
year 413. In 447 it was thrown down by an earthquake, and rebuilt
in three months by the diligence of the præfect Cyrus. The suburb
of the Blanchernæ was first taken into the city in the reign of
Heraclius Ducange, Const. l. i. c. 10, 11.}

\pagenote[33]{The measurement is expressed in the Notitia by
14,075 feet. It is reasonable to suppose that these were Greek
feet, the proportion of which has been ingeniously determined by
M. d’Anville. He compares the 180 feet with 78 Hashemite cubits,
which in different writers are assigned for the heights of St.
Sophia. Each of these cubits was equal to 27 French inches.}

\pagenote[34]{The accurate Thevenot (l. i. c. 15) walked in one
hour and three quarters round two of the sides of the triangle,
from the Kiosk of the Seraglio to the seven towers. D’Anville
examines with care, and receives with confidence, this decisive
testimony, which gives a circumference of ten or twelve miles.
The extravagant computation of Tournefort (Lettre XI) of
thirty-tour or thirty miles, without including Scutari, is a
strange departure from his usual character.}

\pagenote[35]{The sycæ, or fig-trees, formed the thirteenth
region, and were very much embellished by Justinian. It has since
borne the names of Pera and Galata. The etymology of the former
is obvious; that of the latter is unknown. See Ducange, Const. l.
i. c. 22, and Gyllius de Byzant. l. iv. c. 10.}

\pagenote[36]{One hundred and eleven stadia, which may be
translated into modern Greek miles each of seven stadia, or 660,
sometimes only 600 French toises. See D’Anville, Mesures
Itineraires, p. 53.}

\pagenote[37]{When the ancient texts, which describe the size of
Babylon and Thebes, are settled, the exaggerations reduced, and
the measures ascertained, we find that those famous cities filled
the great but not incredible circumference of about twenty-five
or thirty miles. Compare D’Anville, Mém. de l’Académie, tom.
xxviii. p. 235, with his Description de l’Egypte, p. 201, 202.}

\pagenote[38]{If we divide Constantinople and Paris into equal
squares of 50 French \textit{toises}, the former contains 850, and the
latter 1160, of those divisions.}

\section{Part \thesection.}

The master of the Roman world, who aspired to erect an eternal
monument of the glories of his reign could employ in the
prosecution of that great work, the wealth, the labor, and all
that yet remained of the genius of obedient millions. Some
estimate may be formed of the expense bestowed with Imperial
liberality on the foundation of Constantinople, by the allowance
of about two millions five hundred thousand pounds for the
construction of the walls, the porticos, and the aqueducts.\textsuperscript{39}
The forests that overshadowed the shores of the Euxine, and the
celebrated quarries of white marble in the little island of
Proconnesus, supplied an inexhaustible stock of materials, ready
to be conveyed, by the convenience of a short water carriage, to
the harbor of Byzantium.\textsuperscript{40} A multitude of laborers and
artificers urged the conclusion of the work with incessant toil:
but the impatience of Constantine soon discovered, that, in the
decline of the arts, the skill as well as numbers of his
architects bore a very unequal proportion to the greatness of his
designs. The magistrates of the most distant provinces were
therefore directed to institute schools, to appoint professors,
and by the hopes of rewards and privileges, to engage in the
study and practice of architecture a sufficient number of
ingenious youths, who had received a liberal education.\textsuperscript{41} The
buildings of the new city were executed by such artificers as the
reign of Constantine could afford; but they were decorated by the
hands of the most celebrated masters of the age of Pericles and
Alexander. To revive the genius of Phidias and Lysippus,
surpassed indeed the power of a Roman emperor; but the immortal
productions which they had bequeathed to posterity were exposed
without defence to the rapacious vanity of a despot. By his
commands the cities of Greece and Asia were despoiled of their
most valuable ornaments.\textsuperscript{42} The trophies of memorable wars, the
objects of religious veneration, the most finished statues of the
gods and heroes, of the sages and poets, of ancient times,
contributed to the splendid triumph of Constantinople; and gave
occasion to the remark of the historian Cedrenus,\textsuperscript{43} who
observes, with some enthusiasm, that nothing seemed wanting
except the souls of the illustrious men whom these admirable
monuments were intended to represent. But it is not in the city
of Constantine, nor in the declining period of an empire, when
the human mind was depressed by civil and religious slavery, that
we should seek for the souls of Homer and of Demosthenes.

\pagenote[39]{Six hundred centenaries, or sixty thousand pounds’
weight of gold. This sum is taken from Codinus, Antiquit. Const.
p. 11; but unless that contemptible author had derived his
information from some purer sources, he would probably have been
unacquainted with so obsolete a mode of reckoning.}

\pagenote[40]{For the forests of the Black Sea, consult
Tournefort, Lettre XVI. for the marble quarries of Proconnesus,
see Strabo, l. xiii. p. 588, (881, edit. Casaub.) The latter had
already furnished the materials of the stately buildings of
Cyzicus.}

\pagenote[41]{See the Codex Theodos. l. xiii. tit. iv. leg. 1.
This law is dated in the year 334, and was addressed to the
præfect of Italy, whose jurisdiction extended over Africa. The
commentary of Godefroy on the whole title well deserves to be
consulted.}

\pagenote[42]{Constantinopolis dedicatur pœne omnium urbium
nuditate. Hieronym. Chron. p. 181. See Codinus, p. 8, 9. The
author of the Antiquitat. Const. l. iii. (apud Banduri Imp.
Orient. tom. i. p. 41) enumerates Rome, Sicily, Antioch, Athens,
and a long list of other cities. The provinces of Greece and Asia
Minor may be supposed to have yielded the richest booty.}

\pagenote[43]{Hist. Compend. p. 369. He describes the statue, or
rather bust, of Homer with a degree of taste which plainly
indicates that Cadrenus copied the style of a more fortunate
age.}

During the siege of Byzantium, the conqueror had pitched his tent
on the commanding eminence of the second hill. To perpetuate the
memory of his success, he chose the same advantageous position
for the principal Forum;\textsuperscript{44} which appears to have been of a
circular, or rather elliptical form. The two opposite entrances
formed triumphal arches; the porticos, which enclosed it on every
side, were filled with statues; and the centre of the Forum was
occupied by a lofty column, of which a mutilated fragment is now
degraded by the appellation of the \textit{burnt pillar}. This column
was erected on a pedestal of white marble twenty feet high; and
was composed of ten pieces of porphyry, each of which measured
about ten feet in height, and about thirty-three in
circumference.\textsuperscript{45} On the summit of the pillar, above one hundred
and twenty feet from the ground, stood the colossal statue of
Apollo. It was a bronze, had been transported either from Athens
or from a town of Phrygia, and was supposed to be the work of
Phidias. The artist had represented the god of day, or, as it was
afterwards interpreted, the emperor Constantine himself, with a
sceptre in his right hand, the globe of the world in his left,
and a crown of rays glittering on his head.\textsuperscript{46} The Circus, or
Hippodrome, was a stately building about four hundred paces in
length, and one hundred in breadth.\textsuperscript{47} The space between the two
\textit{metæ} or goals were filled with statues and obelisks; and we may
still remark a very singular fragment of antiquity; the bodies of
three serpents, twisted into one pillar of brass. Their triple
heads had once supported the golden tripod which, after the
defeat of Xerxes, was consecrated in the temple of Delphi by the
victorious Greeks.\textsuperscript{48} The beauty of the Hippodrome has been long
since defaced by the rude hands of the Turkish conquerors;\textsuperscript{4811}
but, under the similar appellation of Atmeidan, it still serves
as a place of exercise for their horses. From the throne, whence
the emperor viewed the Circensian games, a winding staircase\textsuperscript{49}
descended to the palace; a magnificent edifice, which scarcely
yielded to the residence of Rome itself, and which, together with
the dependent courts, gardens, and porticos, covered a
considerable extent of ground upon the banks of the Propontis
between the Hippodrome and the church of St. Sophia.\textsuperscript{50} We might
likewise celebrate the baths, which still retained the name of
Zeuxippus, after they had been enriched, by the munificence of
Constantine, with lofty columns, various marbles, and above
threescore statues of bronze.\textsuperscript{51} But we should deviate from the
design of this history, if we attempted minutely to describe the
different buildings or quarters of the city. It may be sufficient
to observe, that whatever could adorn the dignity of a great
capital, or contribute to the benefit or pleasure of its numerous
inhabitants, was contained within the walls of Constantinople. A
particular description, composed about a century after its
foundation, enumerates a capitol or school of learning, a circus,
two theatres, eight public, and one hundred and fifty-three
private baths, fifty-two porticos, five granaries, eight
aqueducts or reservoirs of water, four spacious halls for the
meetings of the senate or courts of justice, fourteen churches,
fourteen palaces, and four thousand three hundred and
eighty-eight houses, which, for their size or beauty, deserved to
be distinguished from the multitude of plebeian inhabitants.\textsuperscript{52}

\pagenote[44]{Zosim. l. ii. p. 106. Chron. Alexandrin. vel
Paschal. p. 284, Ducange, Const. l. i. c. 24. Even the last of
those writers seems to confound the Forum of Constantine with the
Augusteum, or court of the palace. I am not satisfied whether I
have properly distinguished what belongs to the one and the
other.}

\pagenote[45]{The most tolerable account of this column is given
by Pocock. Description of the East, vol. ii. part ii. p. 131. But
it is still in many instances perplexed and unsatisfactory.}

\pagenote[46]{Ducange, Const. l. i. c. 24, p. 76, and his notes
ad Alexiad. p. 382. The statue of Constantine or Apollo was
thrown down under the reign of Alexius Comnenus. * Note: On this
column (says M. von Hammer) Constantine, with singular
shamelessness, placed his own statue with the attributes of
Apollo and Christ. He substituted the nails of the Passion for
the rays of the sun. Such is the direct testimony of the author
of the Antiquit. Constantinop. apud Banduri. Constantine was
replaced by the “great and religious” Julian, Julian, by
Theodosius. A. D. 1412, the key stone was loosened by an
earthquake. The statue fell in the reign of Alexius Comnenus, and
was replaced by the cross. The Palladium was said to be buried
under the pillar. Von Hammer, Constantinopolis und der Bosporos,
i. 162.—M.}

\pagenote[47]{Tournefort (Lettre XII.) computes the Atmeidan at
four hundred paces. If he means geometrical paces of five feet
each, it was three hundred \textit{toises} in length, about forty more
than the great circus of Rome. See D’Anville, Mesures
Itineraires, p. 73.}

\pagenote[48]{The guardians of the most holy relics would rejoice
if they were able to produce such a chain of evidence as may be
alleged on this occasion. See Banduri ad Antiquitat. Const. p.
668. Gyllius de Byzant. l. ii. c. 13. 1. The original
consecration of the tripod and pillar in the temple of Delphi may
be proved from Herodotus and Pausanias. 2. The Pagan Zosimus
agrees with the three ecclesiastical historians, Eusebius,
Socrates, and Sozomen, that the sacred ornaments of the temple of
Delphi were removed to Constantinople by the order of
Constantine; and among these the serpentine pillar of the
Hippodrome is particularly mentioned. 3. All the European
travellers who have visited Constantinople, from Buondelmonte to
Pocock, describe it in the same place, and almost in the same
manner; the differences between them are occasioned only by the
injuries which it has sustained from the Turks. Mahomet the
Second broke the under jaw of one of the serpents with a stroke
of his battle axe Thevenot, l. i. c. 17. * Note: See note 75, ch.
lxviii. for Dr. Clarke’s rejection of Thevenot’s authority. Von
Hammer, however, repeats the story of Thevenot without
questioning its authenticity.—M.}

\pagenote[4811]{In 1808 the Janizaries revolted against the
vizier Mustapha Baisactar, who wished to introduce a new system
of military organization, besieged the quarter of the Hippodrome,
in which stood the palace of the viziers, and the Hippodrome was
consumed in the conflagration.—G.}

\pagenote[49]{The Latin name \textit{Cochlea} was adopted by the Greeks,
and very frequently occurs in the Byzantine history. Ducange,
Const. i. c. l, p. 104.}

\pagenote[50]{There are three topographical points which indicate
the situation of the palace. 1. The staircase which connected it
with the Hippodrome or Atmeidan. 2. A small artificial port on
the Propontis, from whence there was an easy ascent, by a flight
of marble steps, to the gardens of the palace. 3. The Augusteum
was a spacious court, one side of which was occupied by the front
of the palace, and another by the church of St. Sophia.}

\pagenote[51]{Zeuxippus was an epithet of Jupiter, and the baths
were a part of old Byzantium. The difficulty of assigning their
true situation has not been felt by Ducange. History seems to
connect them with St. Sophia and the palace; but the original
plan inserted in Banduri places them on the other side of the
city, near the harbor. For their beauties, see Chron. Paschal. p.
285, and Gyllius de Byzant. l. ii. c. 7. Christodorus (see
Antiquitat. Const. l. vii.) composed inscriptions in verse for
each of the statues. He was a Theban poet in genius as well as in
birth:—Bæotum in crasso jurares aëre natum. * Note: Yet, for his
age, the description of the statues of Hecuba and of Homer are by
no means without merit. See Antholog. Palat. (edit. Jacobs) i.
37—M.}

\pagenote[52]{See the Notitia. Rome only reckoned 1780 large
houses, \textit{domus;} but the word must have had a more dignified
signification. No \textit{insulæ} are mentioned at Constantinople. The
old capital consisted of 42 streets, the new of 322.}

The populousness of his favored city was the next and most
serious object of the attention of its founder. In the dark ages
which succeeded the translation of the empire, the remote and the
immediate consequences of that memorable event were strangely
confounded by the vanity of the Greeks and the credulity of the
Latins.\textsuperscript{53} It was asserted, and believed, that all the noble
families of Rome, the senate, and the equestrian order, with
their innumerable attendants, had followed their emperor to the
banks of the Propontis; that a spurious race of strangers and
plebeians was left to possess the solitude of the ancient
capital; and that the lands of Italy, long since converted into
gardens, were at once deprived of cultivation and inhabitants.\textsuperscript{54}
In the course of this history, such exaggerations will be reduced
to their just value: yet, since the growth of Constantinople
cannot be ascribed to the general increase of mankind and of
industry, it must be admitted that this artificial colony was
raised at the expense of the ancient cities of the empire. Many
opulent senators of Rome, and of the eastern provinces, were
probably invited by Constantine to adopt for their country the
fortunate spot, which he had chosen for his own residence. The
invitations of a master are scarcely to be distinguished from
commands; and the liberality of the emperor obtained a ready and
cheerful obedience. He bestowed on his favorites the palaces
which he had built in the several quarters of the city, assigned
them lands and pensions for the support of their dignity,\textsuperscript{55} and
alienated the demesnes of Pontus and Asia to grant hereditary
estates by the easy tenure of maintaining a house in the capital.\textsuperscript{56}
But these encouragements and obligations soon became
superfluous, and were gradually abolished. Wherever the seat of
government is fixed, a considerable part of the public revenue
will be expended by the prince himself, by his ministers, by the
officers of justice, and by the domestics of the palace. The most
wealthy of the provincials will be attracted by the powerful
motives of interest and duty, of amusement and curiosity. A third
and more numerous class of inhabitants will insensibly be formed,
of servants, of artificers, and of merchants, who derive their
subsistence from their own labor, and from the wants or luxury of
the superior ranks. In less than a century, Constantinople
disputed with Rome itself the preëminence of riches and numbers.
New piles of buildings, crowded together with too little regard
to health or convenience, scarcely allowed the intervals of
narrow streets for the perpetual throng of men, of horses, and of
carriages. The allotted space of ground was insufficient to
contain the increasing people; and the additional foundations,
which, on either side, were advanced into the sea, might alone
have composed a very considerable city.\textsuperscript{57}

\pagenote[53]{Liutprand, Legatio ad Imp. Nicephornm, p. 153. The
modern Greeks have strangely disfigured the antiquities of
Constantinople. We might excuse the errors of the Turkish or
Arabian writers; but it is somewhat astonishing, that the Greeks,
who had access to the authentic materials preserved in their own
language, should prefer fiction to truth, and loose tradition to
genuine history. In a single page of Codinus we may detect twelve
unpardonable mistakes; the reconciliation of Severus and Niger,
the marriage of their son and daughter, the siege of Byzantium by
the Macedonians, the invasion of the Gauls, which recalled
Severus to Rome, the \textit{sixty} years which elapsed from his death
to the foundation of Constantinople, \&c.}

\pagenote[54]{Montesquieu, Grandeur et Décadence des Romains, c.
17.}

\pagenote[55]{Themist. Orat. iii. p. 48, edit. Hardouin. Sozomen,
l. ii. c. 3. Zosim. l. ii. p. 107. Anonym. Valesian. p. 715. If
we could credit Codinus, (p. 10,) Constantine built houses for
the senators on the exact model of their Roman palaces, and
gratified them, as well as himself, with the pleasure of an
agreeable surprise; but the whole story is full of fictions and
inconsistencies.}

\pagenote[56]{The law by which the younger Theodosius, in the
year 438, abolished this tenure, may be found among the Novellæ
of that emperor at the end of the Theodosian Code, tom. vi. nov.
12. M. de Tillemont (Hist. des Empereurs, tom. iv. p. 371) has
evidently mistaken the nature of these estates. With a grant from
the Imperial demesnes, the same condition was accepted as a
favor, which would justly have been deemed a hardship, if it had
been imposed upon private property.}

\pagenote[57]{The passages of Zosimus, of Eunapius, of Sozomen,
and of Agathias, which relate to the increase of buildings and
inhabitants at Constantinople, are collected and connected by
Gyllius de Byzant. l. i. c. 3. Sidonius Apollinaris (in Panegyr.
Anthem. 56, p. 279, edit. Sirmond) describes the moles that were
pushed forwards into the sea, they consisted of the famous
Puzzolan sand, which hardens in the water.}

The frequent and regular distributions of wine and oil, of corn
or bread, of money or provisions, had almost exempted the poorest
citizens of Rome from the necessity of labor. The magnificence of
the first Cæsars was in some measure imitated by the founder of
Constantinople:\textsuperscript{58} but his liberality, however it might excite
the applause of the people, has incurred the censure of
posterity. A nation of legislators and conquerors might assert
their claim to the harvests of Africa, which had been purchased
with their blood; and it was artfully contrived by Augustus,
that, in the enjoyment of plenty, the Romans should lose the
memory of freedom. But the prodigality of Constantine could not
be excused by any consideration either of public or private
interest; and the annual tribute of corn imposed upon Egypt for
the benefit of his new capital, was applied to feed a lazy and
insolent populace, at the expense of the husbandmen of an
industrious province.\textsuperscript{59} \textsuperscript{5911} Some other regulations of this
emperor are less liable to blame, but they are less deserving of
notice. He divided Constantinople into fourteen regions or
quarters,\textsuperscript{60} dignified the public council with the appellation of
senate,\textsuperscript{61} communicated to the citizens the privileges of Italy,\textsuperscript{62}
and bestowed on the rising city the title of Colony, the first
and most favored daughter of ancient Rome. The venerable parent
still maintained the legal and acknowledged supremacy, which was
due to her age, her dignity, and to the remembrance of her former
greatness.\textsuperscript{63}

\pagenote[58]{Sozomen, l. ii. c. 3. Philostorg. l. ii. c. 9.
Codin. Antiquitat. Const. p. 8. It appears by Socrates, l. ii. c.
13, that the daily allowance of the city consisted of eight
myriads of σίτου, which we may either translate, with Valesius, by the
words modii of corn, or consider us expressive of the number of
loaves of bread. * Note: At Rome the poorer citizens who received
these gratuities were inscribed in a register; they had only a
personal right. Constantine attached the right to the houses in
his new capital, to engage the lower classes of the people to
build their houses with expedition. Codex Therodos. l. xiv.—G.}

\pagenote[59]{See Cod. Theodos. l. xiii. and xiv., and Cod.
Justinian. Edict. xii. tom. ii. p. 648, edit. Genev. See the
beautiful complaint of Rome in the poem of Claudian de Bell.
Gildonico, ver. 46-64.——Cum subiit par Roma mihi, divisaque
sumsit Æquales aurora togas; Ægyptia rura In partem cessere
novam.}

\pagenote[5911]{This was also at the expense of Rome. The emperor
ordered that the fleet of Alexandria should transport to
Constantinople the grain of Egypt which it carried before to
Rome: this grain supplied Rome during four months of the year.
Claudian has described with force the famine occasioned by this
measure:—

Hæc nobis, hæc ante dabas; nunc pabula tantum Roma precor:
miserere tuæ; pater optime, gentis: Extremam defende famem. Claud.
de Bell. Gildon. v. 34.—G.

It was scarcely this measure. Gildo had cut off the African as
well as the Egyptian supplies.—M.}

\pagenote[60]{The regions of Constantinople are mentioned in the
code of Justinian, and particularly described in the Notitia of
the younger Theodosius; but as the four last of them are not
included within the wall of Constantine, it may be doubted
whether this division of the city should be referred to the
founder.}

\pagenote[61]{Senatum constituit secundi ordinis; \textit{Claros}
vocavit. Anonym Valesian. p. 715. The senators of old Rome were
styled \textit{Clarissimi}. See a curious note of Valesius ad Ammian.
Marcellin. xxii. 9. From the eleventh epistle of Julian, it
should seem that the place of senator was considered as a burden,
rather than as an honor; but the Abbé de la Bleterie (Vie de
Jovien, tom. ii. p. 371) has shown that this epistle could not
relate to Constantinople. Might we not read, instead of the
celebrated name of the obscure but more probable word Bisanthe or
Rhœdestus, now Rhodosto, was a small maritime city of Thrace. See
Stephan. Byz. de Urbibus, p. 225, and Cellar. Geograph. tom. i.
p. 849.}

\pagenote[62]{Cod. Theodos. l. xiv. 13. The commentary of
Godefroy (tom. v. p. 220) is long, but perplexed; nor indeed is
it easy to ascertain in what the Jus Italicum could consist,
after the freedom of the city had been communicated to the whole
empire. * Note: “This right, (the Jus Italicum,) which by most
writers is referred with out foundation to the personal condition
of the citizens, properly related to the city as a whole, and
contained two parts. First, the Roman or quiritarian property in
the soil, (commercium,) and its capability of mancipation,
usucaption, and vindication; moreover, as an inseparable
consequence of this, exemption from land-tax. Then, secondly, a
free constitution in the Italian form, with Duumvirs,
Quinquennales. and Ædiles, and especially with Jurisdiction.”
Savigny, Geschichte des Rom. Rechts i. p. 51—M.}

\pagenote[63]{Julian (Orat. i. p. 8) celebrates Constantinople as
not less superior to all other cities than she was inferior to
Rome itself. His learned commentator (Spanheim, p. 75, 76)
justifies this language by several parallel and contemporary
instances. Zosimus, as well as Socrates and Sozomen, flourished
after the division of the empire between the two sons of
Theodosius, which established a perfect \textit{equality} between the
old and the new capital.}

As Constantine urged the progress of the work with the impatience
of a lover, the walls, the porticos, and the principal edifices
were completed in a few years, or, according to another account,
in a few months;\textsuperscript{64} but this extraordinary diligence should
excite the less admiration, since many of the buildings were
finished in so hasty and imperfect a manner, that under the
succeeding reign, they were preserved with difficulty from
impending ruin.\textsuperscript{65} But while they displayed the vigor and
freshness of youth, the founder prepared to celebrate the
dedication of his city.\textsuperscript{66} The games and largesses which crowned
the pomp of this memorable festival may easily be supposed; but
there is one circumstance of a more singular and permanent
nature, which ought not entirely to be overlooked. As often as
the birthday of the city returned, the statue of Constantine,
framed by his order, of gilt wood, and bearing in his right hand
a small image of the genius of the place, was erected on a
triumphal car. The guards, carrying white tapers, and clothed in
their richest apparel, accompanied the solemn procession as it
moved through the Hippodrome. When it was opposite to the throne
of the reigning emperor, he rose from his seat, and with grateful
reverence adored the memory of his predecessor.\textsuperscript{67} At the
festival of the dedication, an edict, engraved on a column of
marble, bestowed the title of Second or New Rome on the city of
Constantine.\textsuperscript{68} But the name of Constantinople\textsuperscript{69} has prevailed
over that honorable epithet; and after the revolution of fourteen
centuries, still perpetuates the fame of its author.\textsuperscript{70}

\pagenote[64]{Codinus (Antiquitat. p. 8) affirms, that the
foundations of Constantinople were laid in the year of the world
5837, (A. D. 329,) on the 26th of September, and that the city
was dedicated the 11th of May, 5838, (A. D. 330.) He connects
those dates with several characteristic epochs, but they
contradict each other; the authority of Codinus is of little
weight, and the space which he assigns must appear insufficient.
The term of ten years is given us by Julian, (Orat. i. p. 8;) and
Spanheim labors to establish the truth of it, (p. 69-75,) by the
help of two passages from Themistius, (Orat. iv. p. 58,) and of
Philostorgius, (l. ii. c. 9,) which form a period from the year
324 to the year 334. Modern critics are divided concerning this
point of chronology and their different sentiments are very
accurately described by Tillemont, Hist. des Empereurs, tom. iv.
p. 619-625.}

\pagenote[65]{Themistius. Orat. iii. p. 47. Zosim. l. ii. p. 108.
Constantine himself, in one of his laws, (Cod. Theod. l. xv. tit.
i.,) betrays his impatience.}

\pagenote[66]{Cedrenus and Zonaras, faithful to the mode of
superstition which prevailed in their own times, assure us that
Constantinople was consecrated to the virgin Mother of God.}

\pagenote[67]{The earliest and most complete account of this
extraordinary ceremony may be found in the Alexandrian Chronicle,
p. 285. Tillemont, and the other friends of Constantine, who are
offended with the air of Paganism which seems unworthy of a
Christian prince, had a right to consider it as doubtful, but
they were not authorized to omit the mention of it.}

\pagenote[68]{Sozomen, l. ii. c. 2. Ducange C. P. l. i. c. 6.
Velut ipsius Romæ filiam, is the expression of Augustin. de
Civitat. Dei, l. v. c. 25.}

\pagenote[69]{Eutropius, l. x. c. 8. Julian. Orat. i. p. 8.
Ducange C. P. l. i. c. 5. The name of Constantinople is extant on
the medals of Constantine.}

\pagenote[70]{The lively Fontenelle (Dialogues des Morts, xii.)
affects to deride the vanity of human ambition, and seems to
triumph in the disappointment of Constantine, whose immortal name
is now lost in the vulgar appellation of Istambol, a Turkish
corruption of είς τήν πόλιω. Yet the original name is still preserved, 1. By
the nations of Europe. 2. By the modern Greeks. 3. By the Arabs,
whose writings are diffused over the wide extent of their
conquests in Asia and Africa. See D’Herbelot, Bibliothèque
Orientale, p. 275. 4. By the more learned Turks, and by the
emperor himself in his public mandates Cantemir’s History of the
Othman Empire, p. 51.}

The foundation of a new capital is naturally connected with the
establishment of a new form of civil and military administration.
The distinct view of the complicated system of policy, introduced
by Diocletian, improved by Constantine, and completed by his
immediate successors, may not only amuse the fancy by the
singular picture of a great empire, but will tend to illustrate
the secret and internal causes of its rapid decay. In the pursuit
of any remarkable institution, we may be frequently led into the
more early or the more recent times of the Roman history; but the
proper limits of this inquiry will be included within a period of
about one hundred and thirty years, from the accession of
Constantine to the publication of the Theodosian code;\textsuperscript{71} from
which, as well as from the \textit{Notitia}\textsuperscript{7111} of the East and West,\textsuperscript{72}
we derive the most copious and authentic information of the
state of the empire. This variety of objects will suspend, for
some time, the course of the narrative; but the interruption will
be censured only by those readers who are insensible to the
importance of laws and manners, while they peruse, with eager
curiosity, the transient intrigues of a court, or the accidental
event of a battle.

\pagenote[71]{The Theodosian code was promulgated A. D. 438. See
the Prolegomena of Godefroy, c. i. p. 185.}

\pagenote[7111]{The Notitia Dignitatum Imperii is a description
of all the offices in the court and the state, of the legions,
\&c. It resembles our court almanacs, (Red Books,) with this
single difference, that our almanacs name the persons in office,
the Notitia only the offices. It is of the time of the emperor
Theodosius II., that is to say, of the fifth century, when the
empire was divided into the Eastern and Western. It is probable
that it was not made for the first time, and that descriptions of
the same kind existed before.—G.}

\pagenote[72]{Pancirolus, in his elaborate Commentary, assigns to
the Notitia a date almost similar to that of the Theodosian Code;
but his proofs, or rather conjectures, are extremely feeble. I
should be rather inclined to place this useful work between the
final division of the empire (A. D. 395) and the successful
invasion of Gaul by the barbarians, (A. D. 407.) See Histoire des
Anciens Peuples de l’Europe, tom. vii. p. 40.}

\section{Part \thesection.}

The manly pride of the Romans, content with substantial power,
had left to the vanity of the East the forms and ceremonies of
ostentatious greatness.\textsuperscript{73} But when they lost even the semblance
of those virtues which were derived from their ancient freedom,
the simplicity of Roman manners was insensibly corrupted by the
stately affectation of the courts of Asia. The distinctions of
personal merit and influence, so conspicuous in a republic, so
feeble and obscure under a monarchy, were abolished by the
despotism of the emperors; who substituted in their room a severe
subordination of rank and office from the titled slaves who were
seated on the steps of the throne, to the meanest instruments of
arbitrary power. This multitude of abject dependants was
interested in the support of the actual government from the dread
of a revolution, which might at once confound their hopes and
intercept the reward of their services. In this divine hierarchy
(for such it is frequently styled) every rank was marked with the
most scrupulous exactness, and its dignity was displayed in a
variety of trifling and solemn ceremonies, which it was a study
to learn, and a sacrilege to neglect.\textsuperscript{74} The purity of the Latin
language was debased, by adopting, in the intercourse of pride
and flattery, a profusion of epithets, which Tully would scarcely
have understood, and which Augustus would have rejected with
indignation. The principal officers of the empire were saluted,
even by the sovereign himself, with the deceitful titles of your
\textit{Sincerity}, your \textit{Gravity}, your \textit{Excellency}, your \textit{Eminence},
your \textit{sublime and wonderful Magnitude}, your \textit{illustrious and
magnificent Highness}.\textsuperscript{75} The codicils or patents of their office
were curiously emblazoned with such emblems as were best adapted
to explain its nature and high dignity; the image or portrait of
the reigning emperors; a triumphal car; the book of mandates
placed on a table, covered with a rich carpet, and illuminated by
four tapers; the allegorical figures of the provinces which they
governed; or the appellations and standards of the troops whom
they commanded. Some of these official ensigns were really
exhibited in their hall of audience; others preceded their
pompous march whenever they appeared in public; and every
circumstance of their demeanor, their dress, their ornaments, and
their train, was calculated to inspire a deep reverence for the
representatives of supreme majesty. By a philosophic observer,
the system of the Roman government might have been mistaken for a
splendid theatre, filled with players of every character and
degree, who repeated the language, and imitated the passions, of
their original model.\textsuperscript{76}

\pagenote[73]{Scilicet externæ superbiæ sueto, non inerat notitia
nostri, (perhaps \textit{nostræ;}) apud quos vis Imperii valet, inania
transmittuntur. Tacit. Annal. xv. 31. The gradation from the
style of freedom and simplicity, to that of form and servitude,
may be traced in the Epistles of Cicero, of Pliny, and of
Symmachus.}

\pagenote[74]{The emperor Gratian, after confirming a law of
precedency published by Valentinian, the father of his
\textit{Divinity}, thus continues: Siquis igitur indebitum sibi locum
usurpaverit, nulla se ignoratione defendat; sitque plane
\textit{sacrilegii} reus, qui \textit{divina} præcepta neglexerit. Cod. Theod.
l. vi. tit. v. leg. 2.}

\pagenote[75]{Consult the \textit{Notitia Dignitatum} at the end of the
Theodosian code, tom. vi. p. 316. * Note: Constantin, qui
remplaca le grand Patriciat par une noblesse titree et qui
changea avec d’autres institutions la nature de la societe
Latine, est le veritable fondateur de la royaute moderne, dans ce
quelle conserva de Romain. Chateaubriand, Etud. Histor. Preface,
i. 151. Manso, (Leben Constantins des Grossen,) p. 153, \&c., has
given a lucid view of the dignities and duties of the officers in
the Imperial court.—M.}

\pagenote[76]{Pancirolus ad Notitiam utriusque Imperii, p. 39.
But his explanations are obscure, and he does not sufficiently
distinguish the painted emblems from the effective ensigns of
office.}

All the magistrates of sufficient importance to find a place in
the general state of the empire, were accurately divided into
three classes. 1. The \textit{Illustrious}. 2. The \textit{Spectabiles}, or
\textit{Respectable}. And, 3. the \textit{Clarissimi;} whom we may translate by
the word \textit{Honorable}. In the times of Roman simplicity, the
last-mentioned epithet was used only as a vague expression of
deference, till it became at length the peculiar and appropriated
title of all who were members of the senate,\textsuperscript{77} and consequently
of all who, from that venerable body, were selected to govern the
provinces. The vanity of those who, from their rank and office,
might claim a superior distinction above the rest of the
senatorial order, was long afterwards indulged with the new
appellation of \textit{Respectable;} but the title of \textit{Illustrious} was
always reserved to some eminent personages who were obeyed or
reverenced by the two subordinate classes. It was communicated
only, I. To the consuls and patricians; II. To the Prætorian
præfects, with the præfects of Rome and Constantinople; III. To
the masters-general of the cavalry and the infantry; and IV. To
the seven ministers of the palace, who exercised their \textit{sacred}
functions about the person of the emperor.\textsuperscript{78} Among those
illustrious magistrates who were esteemed coordinate with each
other, the seniority of appointment gave place to the union of
dignities.\textsuperscript{79} By the expedient of honorary codicils, the
emperors, who were fond of multiplying their favors, might
sometimes gratify the vanity, though not the ambition, of
impatient courtiers.\textsuperscript{80}

\pagenote[77]{In the Pandects, which may be referred to the
reigns of the Antonines, Clarissimus is the ordinary and legal
title of a senator.}

\pagenote[78]{Pancirol. p. 12-17. I have not taken any notice of
the two inferior ranks, \textit{Prefectissimus} and \textit{Egregius}, which
were given to many persons who were not raised to the senatorial
dignity.}

\pagenote[79]{Cod. Theodos. l. vi. tit. vi. The rules of
precedency are ascertained with the most minute accuracy by the
emperors, and illustrated with equal prolixity by their learned
interpreter.}

\pagenote[80]{Cod. Theodos. l. vi. tit. xxii.}

I. As long as the Roman consuls were the first magistrates of a
free state, they derived their right to power from the choice of
the people. As long as the emperors condescended to disguise the
servitude which they imposed, the consuls were still elected by
the real or apparent suffrage of the senate. From the reign of
Diocletian, even these vestiges of liberty were abolished, and
the successful candidates who were invested with the annual
honors of the consulship, affected to deplore the humiliating
condition of their predecessors. The Scipios and the Catos had
been reduced to solicit the votes of plebeians, to pass through
the tedious and expensive forms of a popular election, and to
expose their dignity to the shame of a public refusal; while
their own happier fate had reserved them for an age and
government in which the rewards of virtue were assigned by the
unerring wisdom of a gracious sovereign.\textsuperscript{81} In the epistles which
the emperor addressed to the two consuls elect, it was declared,
that they were created by his sole authority.\textsuperscript{82} Their names and
portraits, engraved on gilt tables of ivory, were dispersed over
the empire as presents to the provinces, the cities, the
magistrates, the senate, and the people.\textsuperscript{83} Their solemn
inauguration was performed at the place of the Imperial
residence; and during a period of one hundred and twenty years,
Rome was constantly deprived of the presence of her ancient
magistrates.\textsuperscript{84}

\pagenote[81]{Ausonius (in Gratiarum Actione) basely expatiates
on this unworthy topic, which is managed by Mamertinus (Panegyr.
Vet. xi. [x.] 16, 19) with somewhat more freedom and ingenuity.}

\pagenote[82]{Cum de Consulibus in annum creandis, solus mecum
volutarem.... te Consulem et designavi, et declaravi, et priorem
nuncupavi; are some of the expressions employed by the emperor
Gratian to his preceptor, the poet Ausonius.}

\pagenote[83]{Immanesque... dentes Qui secti ferro in tabulas
auroque micantes, Inscripti rutilum cœlato Consule nomen Per
proceres et vulgus eant. —Claud. in ii. Cons. Stilichon. 456.

Montfaucon has represented some of these tablets or dypticks see
Supplement à l’Antiquité expliquée, tom. iii. p. 220.}

\pagenote[84]{Consule lætatur post plurima seculo viso Pallanteus apex:
agnoscunt rostra curules Auditas quondam proavis: desuetaque
cingit Regius auratis Fora fascibus Ulpia lictor. —Claud. in vi.
Cons. Honorii, 643.

From the reign of Carus to the sixth consulship of Honorius,
there was an interval of one hundred and twenty years, during
which the emperors were always absent from Rome on the first day
of January. See the Chronologie de Tillemonte, tom. iii. iv. and
v.}

On the morning of the first of January, the consuls assumed the
ensigns of their dignity. Their dress was a robe of purple,
embroidered in silk and gold, and sometimes ornamented with
costly gems.\textsuperscript{85} On this solemn occasion they were attended by the
most eminent officers of the state and army, in the habit of
senators; and the useless fasces, armed with the once formidable
axes, were borne before them by the lictors. The procession moved
from the palace\textsuperscript{87} to the Forum or principal square of the city;
where the consuls ascended their tribunal, and seated themselves
in the curule chairs, which were framed after the fashion of
ancient times. They immediately exercised an act of jurisdiction,
by the manumission of a slave, who was brought before them for
that purpose; and the ceremony was intended to represent the
celebrated action of the elder Brutus, the author of liberty and
of the consulship, when he admitted among his fellow-citizens the
faithful Vindex, who had revealed the conspiracy of the Tarquins.\textsuperscript{88}
The public festival was continued during several days in all
the principal cities in Rome, from custom; in Constantinople,
from imitation in Carthage, Antioch, and Alexandria, from the
love of pleasure, and the superfluity of wealth.\textsuperscript{89} In the two
capitals of the empire the annual games of the theatre, the
circus, and the amphitheatre,\textsuperscript{90} cost four thousand pounds of
gold, (about) one hundred and sixty thousand pounds sterling: and
if so heavy an expense surpassed the faculties or the
inclinations of the magistrates themselves, the sum was supplied
from the Imperial treasury.\textsuperscript{91} As soon as the consuls had
discharged these customary duties, they were at liberty to retire
into the shade of private life, and to enjoy, during the
remainder of the year, the undisturbed contemplation of their own
greatness. They no longer presided in the national councils; they
no longer executed the resolutions of peace or war. Their
abilities (unless they were employed in more effective offices)
were of little moment; and their names served only as the legal
date of the year in which they had filled the chair of Marius and
of Cicero. Yet it was still felt and acknowledged, in the last
period of Roman servitude, that this empty name might be
compared, and even preferred, to the possession of substantial
power. The title of consul was still the most splendid object of
ambition, the noblest reward of virtue and loyalty. The emperors
themselves, who disdained the faint shadow of the republic, were
conscious that they acquired an additional splendor and majesty
as often as they assumed the annual honors of the consular
dignity.\textsuperscript{92}

\pagenote[85]{See Claudian in Cons. Prob. et Olybrii, 178, \&c.;
and in iv. Cons. Honorii, 585, \&c.; though in the latter it is
not easy to separate the ornaments of the emperor from those of
the consul. Ausonius received from the liberality of Gratian a
\textit{vestis palmata}, or robe of state, in which the figure of the
emperor Constantius was embroidered. Cernis et armorum proceres
legumque potentes: Patricios sumunt habitus; et more Gabino
Discolor incedit legio, positisque parumper Bellorum signis,
sequitur vexilla Quirini. Lictori cedunt aquilæ, ridetque togatus
Miles, et in mediis effulget curia castris. —Claud. in iv. Cons.
Honorii, 5. —\textit{strictaque} procul radiare \textit{secures}. —In Cons.
Prob. 229}

\pagenote[87]{See Valesius ad Ammian. Marcellin. l. xxii. c. 7.}

\pagenote[88]{Auspice mox læto sonuit clamore tribunal; Te fastos
ineunte quater; solemnia ludit Omina libertas; deductum Vindice
morem Lex servat, famulusque jugo laxatus herili Ducitur, et
grato remeat securior ictu. —Claud. in iv Cons. Honorii, 611}

\pagenote[89]{Celebrant quidem solemnes istos dies omnes ubique
urbes quæ sub legibus agunt; et Roma de more, et Constantinopolis
de imitatione, et Antiochia pro luxu, et discincta Carthago, et
domus fluminis Alexandria, sed Treviri Principis beneficio.
Ausonius in Grat. Actione.}

\pagenote[90]{Claudian (in Cons. Mall. Theodori, 279-331)
describes, in a lively and fanciful manner, the various games of
the circus, the theatre, and the amphitheatre, exhibited by the
new consul. The sanguinary combats of gladiators had already been
prohibited.}

\pagenote[91]{Procopius in Hist. Arcana, c. 26.}

\pagenote[92]{In Consulatu honos sine labore suscipitur.
(Mamertin. in Panegyr. Vet. xi. [x.] 2.) This exalted idea of the
consulship is borrowed from an oration (iii. p. 107) pronounced
by Julian in the servile court of Constantius. See the Abbé de la
Bleterie, (Mémoires de l’Académie, tom. xxiv. p. 289,) who
delights to pursue the vestiges of the old constitution, and who
sometimes finds them in his copious fancy}

The proudest and most perfect separation which can be found in
any age or country, between the nobles and the people, is perhaps
that of the Patricians and the Plebeians, as it was established
in the first age of the Roman republic. Wealth and honors, the
offices of the state, and the ceremonies of religion, were almost
exclusively possessed by the former who, preserving the purity of
their blood with the most insulting jealousy,\textsuperscript{93} held their
clients in a condition of specious vassalage. But these
distinctions, so incompatible with the spirit of a free people,
were removed, after a long struggle, by the persevering efforts
of the Tribunes. The most active and successful of the Plebeians
accumulated wealth, aspired to honors, deserved triumphs,
contracted alliances, and, after some generations, assumed the
pride of ancient nobility.\textsuperscript{94} The Patrician families, on the
other hand, whose original number was never recruited till the
end of the commonwealth, either failed in the ordinary course of
nature, or were extinguished in so many foreign and domestic
wars, or, through a want of merit or fortune, insensibly mingled
with the mass of the people.\textsuperscript{95} Very few remained who could
derive their pure and genuine origin from the infancy of the
city, or even from that of the republic, when Cæsar and Augustus,
Claudius and Vespasian, created from the body of the senate a
competent number of new Patrician families, in the hope of
perpetuating an order, which was still considered as honorable
and sacred.\textsuperscript{96} But these artificial supplies (in which the
reigning house was always included) were rapidly swept away by
the rage of tyrants, by frequent revolutions, by the change of
manners, and by the intermixture of nations.\textsuperscript{97} Little more was
left when Constantine ascended the throne, than a vague and
imperfect tradition, that the Patricians had once been the first
of the Romans. To form a body of nobles, whose influence may
restrain, while it secures the authority of the monarch, would
have been very inconsistent with the character and policy of
Constantine; but had he seriously entertained such a design, it
might have exceeded the measure of his power to ratify, by an
arbitrary edict, an institution which must expect the sanction of
time and of opinion. He revived, indeed, the title of Patricians,
but he revived it as a personal, not as an hereditary
distinction. They yielded only to the transient superiority of
the annual consuls; but they enjoyed the pre-eminence over all
the great officers of state, with the most familiar access to the
person of the prince. This honorable rank was bestowed on them
for life; and as they were usually favorites, and ministers who
had grown old in the Imperial court, the true etymology of the
word was perverted by ignorance and flattery; and the Patricians
of Constantine were reverenced as the adopted \textit{Fathers} of the
emperor and the republic.\textsuperscript{98}

\pagenote[93]{Intermarriages between the Patricians and Plebeians
were prohibited by the laws of the XII Tables; and the uniform
operations of human nature may attest that the custom survived
the law. See in Livy (iv. 1-6) the pride of family urged by the
consul, and the rights of mankind asserted by the tribune
Canuleius.}

\pagenote[94]{See the animated picture drawn by Sallust, in the
Jugurthine war, of the pride of the nobles, and even of the
virtuous Metellus, who was unable to brook the idea that the
honor of the consulship should be bestowed on the obscure merit
of his lieutenant Marius. (c. 64.) Two hundred years before, the
race of the Metelli themselves were confounded among the
Plebeians of Rome; and from the etymology of their name of
\textit{Cæcilius}, there is reason to believe that those haughty nobles
derived their origin from a sutler.}

\pagenote[95]{In the year of Rome 800, very few remained, not
only of the old Patrician families, but even of those which had
been created by Cæsar and Augustus. (Tacit. Annal. xi. 25.) The
family of Scaurus (a branch of the Patrician Æmilii) was degraded
so low that his father, who exercised the trade of a charcoal
merchant, left him only teu slaves, and somewhat less than three
hundred pounds sterling. (Valerius Maximus, l. iv. c. 4, n. 11.
Aurel. Victor in Scauro.) The family was saved from oblivion by
the merit of the son.}

\pagenote[96]{Tacit. Annal. xi. 25. Dion Cassius, l. iii. p. 698.
The virtues of Agricola, who was created a Patrician by the
emperor Vespasian, reflected honor on that ancient order; but his
ancestors had not any claim beyond an Equestrian nobility.}

\pagenote[97]{This failure would have been almost impossible if
it were true, as Casaubon compels Aurelius Victor to affirm (ad
Sueton, in Cæsar v. 24. See Hist. August p. 203 and Casaubon
Comment., p. 220) that Vespasian created at once a thousand
Patrician families. But this extravagant number is too much even
for the whole Senatorial order. unless we should include all the
Roman knights who were distinguished by the permission of wearing
the laticlave.}

\pagenote[98]{Zosimus, l. ii. p. 118; and Godefroy ad Cod.
Theodos. l. vi. tit. vi.}

II. The fortunes of the Prætorian præfects were essentially
different from those of the consuls and Patricians. The latter
saw their ancient greatness evaporate in a vain title.

The former, rising by degrees from the most humble condition,
were invested with the civil and military administration of the
Roman world. From the reign of Severus to that of Diocletian, the
guards and the palace, the laws and the finances, the armies and
the provinces, were intrusted to their superintending care; and,
like the Viziers of the East, they held with one hand the seal,
and with the other the standard, of the empire. The ambition of
the præfects, always formidable, and sometimes fatal to the
masters whom they served, was supported by the strength of the
Prætorian bands; but after those haughty troops had been weakened
by Diocletian, and finally suppressed by Constantine, the
præfects, who survived their fall, were reduced without
difficulty to the station of useful and obedient ministers. When
they were no longer responsible for the safety of the emperor’s
person, they resigned the jurisdiction which they had hitherto
claimed and exercised over all the departments of the palace.
They were deprived by Constantine of all military command, as
soon as they had ceased to lead into the field, under their
immediate orders, the flower of the Roman troops; and at length,
by a singular revolution, the captains of the guards were
transformed into the civil magistrates of the provinces.
According to the plan of government instituted by Diocletian, the
four princes had each their Prætorian præfect; and after the
monarchy was once more united in the person of Constantine, he
still continued to create the same number of Four Præfects, and
intrusted to their care the same provinces which they already
administered. 1. The præfect of the East stretched his ample
jurisdiction into the three parts of the globe which were subject
to the Romans, from the cataracts of the Nile to the banks of the
Phasis, and from the mountains of Thrace to the frontiers of
Persia. 2. The important provinces of Pannonia, Dacia, Macedonia,
and Greece, once acknowledged the authority of the præfect of
Illyricum. 3. The power of the præfect of Italy was not confined
to the country from whence he derived his title; it extended over
the additional territory of Rhætia as far as the banks of the
Danube, over the dependent islands of the Mediterranean, and over
that part of the continent of Africa which lies between the
confines of Cyrene and those of Tingitania. 4. The præfect of the
Gauls comprehended under that plural denomination the kindred
provinces of Britain and Spain, and his authority was obeyed from
the wall of Antoninus to the foot of Mount Atlas.\textsuperscript{99}

\pagenote[99]{Zosimus, l. ii. p. 109, 110. If we had not
fortunately possessed this satisfactory account of the division
of the power and provinces of the Prætorian præfects, we should
frequently have been perplexed amidst the copious details of the
Code, and the circumstantial minuteness of the Notitia.}

After the Prætorian præfects had been dismissed from all military
command, the civil functions which they were ordained to exercise
over so many subject nations, were adequate to the ambition and
abilities of the most consummate ministers. To their wisdom was
committed the supreme administration of justice and of the
finances, the two objects which, in a state of peace, comprehend
almost all the respective duties of the sovereign and of the
people; of the former, to protect the citizens who are obedient
to the laws; of the latter, to contribute the share of their
property which is required for the expenses of the state. The
coin, the highways, the posts, the granaries, the manufactures,
whatever could interest the public prosperity, was moderated by
the authority of the Prætorian præfects. As the immediate
representatives of the Imperial majesty, they were empowered to
explain, to enforce, and on some occasions to modify, the general
edicts by their discretionary proclamations. They watched over
the conduct of the provincial governors, removed the negligent,
and inflicted punishments on the guilty. From all the inferior
jurisdictions, an appeal in every matter of importance, either
civil or criminal, might be brought before the tribunal of the
præfect; but \textit{his} sentence was final and absolute; and the
emperors themselves refused to admit any complaints against the
judgment or the integrity of a magistrate whom they honored with
such unbounded confidence.\textsuperscript{100} His appointments were suitable to
his dignity;\textsuperscript{101} and if avarice was his ruling passion, he
enjoyed frequent opportunities of collecting a rich harvest of
fees, of presents, and of perquisites. Though the emperors no
longer dreaded the ambition of their præfects, they were
attentive to counterbalance the power of this great office by the
uncertainty and shortness of its duration.\textsuperscript{102}

\pagenote[100]{See a law of Constantine himself. A præfectis
autem prætorio provocare, non sinimus. Cod. Justinian. l. vii.
tit. lxii. leg. 19. Charisius, a lawyer of the time of
Constantine, (Heinec. Hist. Romani, p. 349,) who admits this law
as a fundamental principle of jurisprudence, compares the
Prætorian præfects to the masters of the horse of the ancient
dictators. Pandect. l. i. tit. xi.}

\pagenote[101]{When Justinian, in the exhausted condition of the
empire, instituted a Prætorian præfect for Africa, he allowed him
a salary of one hundred pounds of gold. Cod. Justinian. l. i.
tit. xxvii. leg. i.}

\pagenote[102]{For this, and the other dignities of the empire,
it may be sufficient to refer to the ample commentaries of
Pancirolus and Godefroy, who have diligently collected and
accurately digested in their proper order all the legal and
historical materials. From those authors, Dr. Howell (History of
the World, vol. ii. p. 24-77) has deduced a very distinct
abridgment of the state of the Roman empire}

From their superior importance and dignity, Rome and
Constantinople were alone excepted from the jurisdiction of the
Prætorian præfects. The immense size of the city, and the
experience of the tardy, ineffectual operation of the laws, had
furnished the policy of Augustus with a specious pretence for
introducing a new magistrate, who alone could restrain a servile
and turbulent populace by the strong arm of arbitrary power.\textsuperscript{103}
Valerius Messalla was appointed the first præfect of Rome, that
his reputation might countenance so invidious a measure; but, at
the end of a few days, that accomplished citizen\textsuperscript{104} resigned his
office, declaring, with a spirit worthy of the friend of Brutus,
that he found himself incapable of exercising a power
incompatible with public freedom.\textsuperscript{105} As the sense of liberty
became less exquisite, the advantages of order were more clearly
understood; and the præfect, who seemed to have been designed as
a terror only to slaves and vagrants, was permitted to extend his
civil and criminal jurisdiction over the equestrian and noble
families of Rome. The prætors, annually created as the judges of
law and equity, could not long dispute the possession of the
Forum with a vigorous and permanent magistrate, who was usually
admitted into the confidence of the prince. Their courts were
deserted, their number, which had once fluctuated between twelve
and eighteen,\textsuperscript{106} was gradually reduced to two or three, and
their important functions were confined to the expensive
obligation\textsuperscript{107} of exhibiting games for the amusement of the
people. After the office of the Roman consuls had been changed
into a vain pageant, which was rarely displayed in the capital,
the præfects assumed their vacant place in the senate, and were
soon acknowledged as the ordinary presidents of that venerable
assembly. They received appeals from the distance of one hundred
miles; and it was allowed as a principle of jurisprudence, that
all municipal authority was derived from them alone.\textsuperscript{108} In the
discharge of his laborious employment, the governor of Rome was
assisted by fifteen officers, some of whom had been originally
his equals, or even his superiors. The principal departments were
relative to the command of a numerous watch, established as a
safeguard against fires, robberies, and nocturnal disorders; the
custody and distribution of the public allowance of corn and
provisions; the care of the port, of the aqueducts, of the common
sewers, and of the navigation and bed of the Tyber; the
inspection of the markets, the theatres, and of the private as
well as the public works. Their vigilance insured the three
principal objects of a regular police, safety, plenty, and
cleanliness; and as a proof of the attention of government to
preserve the splendor and ornaments of the capital, a particular
inspector was appointed for the statues; the guardian, as it
were, of that inanimate people, which, according to the
extravagant computation of an old writer, was scarcely inferior
in number to the living inhabitants of Rome. About thirty years
after the foundation of Constantinople, a similar magistrate was
created in that rising metropolis, for the same uses and with the
same powers. A perfect equality was established between the
dignity of the \textit{two} municipal, and that of the \textit{four} Prætorian
præfects.\textsuperscript{109}

\pagenote[103]{Tacit. Annal. vi. 11. Euseb. in Chron. p. 155.
Dion Cassius, in the oration of Mæcenas, (l. lvii. p. 675,)
describes the prerogatives of the præfect of the city as they
were established in his own time.}

\pagenote[104]{The fame of Messalla has been scarcely equal to
his merit. In the earliest youth he was recommended by Cicero to
the friendship of Brutus. He followed the standard of the
republic till it was broken in the fields of Philippi; he then
accepted and deserved the favor of the most moderate of the
conquerors; and uniformly asserted his freedom and dignity in the
court of Augustus. The triumph of Messalla was justified by the
conquest of Aquitain. As an orator, he disputed the palm of
eloquence with Cicero himself. Messalla cultivated every muse,
and was the patron of every man of genius. He spent his evenings
in philosophic conversation with Horace; assumed his place at
table between Delia and Tibullus; and amused his leisure by
encouraging the poetical talents of young Ovid.}

\pagenote[105]{Incivilem esse potestatem contestans, says the
translator of Eusebius. Tacitus expresses the same idea in other
words; quasi nescius exercendi.}

\pagenote[106]{See Lipsius, Excursus D. ad 1 lib. Tacit. Annal.}

\pagenote[107]{Heineccii. Element. Juris Civilis secund ordinem
Pandect i. p. 70. See, likewise, Spanheim de Usu. Numismatum,
tom. ii. dissertat. x. p. 119. In the year 450, Marcian published
a law, that \textit{three} citizens should be annually created Prætors
of Constantinople by the choice of the senate, but with their own
consent. Cod. Justinian. li. i. tit. xxxix. leg. 2.}

\pagenote[108]{Quidquid igitur intra urbem admittitur, ad P. U.
videtur pertinere; sed et siquid intra contesimum milliarium.
Ulpian in Pandect l. i. tit. xiii. n. 1. He proceeds to enumerate
the various offices of the præfect, who, in the code of
Justinian, (l. i. tit. xxxix. leg. 3,) is declared to precede and
command all city magistrates sine injuria ac detrimento honoris
alieni.}

\pagenote[109]{Besides our usual guides, we may observe that
Felix Cantelorius has written a separate treatise, De Præfecto
Urbis; and that many curious details concerning the police of
Rome and Constantinople are contained in the fourteenth book of
the Theodosian Code.}

\section{Part \thesection.}

Those who, in the imperial hierarchy, were distinguished by the
title of \textit{Respectable}, formed an intermediate class between the
\textit{illustrious} præfects, and the \textit{honorable} magistrates of the
provinces. In this class the proconsuls of Asia, Achaia, and
Africa, claimed a preëminence, which was yielded to the
remembrance of their ancient dignity; and the appeal from their
tribunal to that of the præfects was almost the only mark of
their dependence.\textsuperscript{110} But the civil government of the empire was
distributed into thirteen great Dioceses, each of which equalled
the just measure of a powerful kingdom. The first of these
dioceses was subject to the jurisdiction of the \textit{count} of the
east; and we may convey some idea of the importance and variety
of his functions, by observing, that six hundred apparitors, who
would be styled at present either secretaries, or clerks, or
ushers, or messengers, were employed in his immediate office.\textsuperscript{111}
The place of \textit{Augustal præfect} of Egypt was no longer filled by
a Roman knight; but the name was retained; and the extraordinary
powers which the situation of the country, and the temper of the
inhabitants, had once made indispensable, were still continued to
the governor. The eleven remaining dioceses, of Asiana, Pontica,
and Thrace; of Macedonia, Dacia, and Pannonia, or Western
Illyricum; of Italy and Africa; of Gaul, Spain, and Britain; were
governed by twelve \textit{vicars} or \textit{vice-præfects},\textsuperscript{112} whose name
sufficiently explains the nature and dependence of their office.
It may be added, that the lieutenant-generals of the Roman
armies, the military counts and dukes, who will be hereafter
mentioned, were allowed the rank and title of \textit{Respectable}.

\pagenote[110]{Eunapius affirms, that the proconsul of Asia was
independent of the præfect; which must, however, be understood
with some allowance. the jurisdiction of the vice-præfect he most
assuredly disclaimed. Pancirolus, p. 161.}

\pagenote[111]{The proconsul of Africa had four hundred
apparitors; and they all received large salaries, either from the
treasury or the province See Pancirol. p. 26, and Cod. Justinian.
l. xii. tit. lvi. lvii.}

\pagenote[112]{In Italy there was likewise the \textit{Vicar of Rome}.
It has been much disputed whether his jurisdiction measured one
hundred miles from the city, or whether it stretched over the ten
thousand provinces of Italy.}

As the spirit of jealousy and ostentation prevailed in the
councils of the emperors, they proceeded with anxious diligence
to divide the substance and to multiply the titles of power. The
vast countries which the Roman conquerors had united under the
same simple form of administration, were imperceptibly crumbled
into minute fragments; till at length the whole empire was
distributed into one hundred and sixteen provinces, each of which
supported an expensive and splendid establishment. Of these,
three were governed by \textit{proconsuls}, thirty-seven by \textit{consulars},
five by \textit{correctors}, and seventy-one by \textit{presidents}. The
appellations of these magistrates were different; they ranked in
successive order, and the ensigns of and their situation, from
accidental circumstances, might be more or less agreeable or
advantageous. But they were all (excepting only the pro-consuls)
alike included in the class of \textit{honorable} persons; and they were
alike intrusted, during the pleasure of the prince, and under the
authority of the præfects or their deputies, with the
administration of justice and the finances in their respective
districts. The ponderous volumes of the Codes and Pandects\textsuperscript{113}
would furnish ample materials for a minute inquiry into the
system of provincial government, as in the space of six centuries
it was approved by the wisdom of the Roman statesmen and lawyers.

It may be sufficient for the historian to select two singular and
salutary provisions, intended to restrain the abuse of authority.

1. For the preservation of peace and order, the governors of the
provinces were armed with the sword of justice. They inflicted
corporal punishments, and they exercised, in capital offences,
the power of life and death. But they were not authorized to
indulge the condemned criminal with the choice of his own
execution, or to pronounce a sentence of the mildest and most
honorable kind of exile. These prerogatives were reserved to the
præfects, who alone could impose the heavy fine of fifty pounds
of gold: their vicegerents were confined to the trifling weight
of a few ounces.\textsuperscript{114} This distinction, which seems to grant the
larger, while it denies the smaller degree of authority, was
founded on a very rational motive. The smaller degree was
infinitely more liable to abuse. The passions of a provincial
magistrate might frequently provoke him into acts of oppression,
which affected only the freedom or the fortunes of the subject;
though, from a principle of prudence, perhaps of humanity, he
might still be terrified by the guilt of innocent blood. It may
likewise be considered, that exile, considerable fines, or the
choice of an easy death, relate more particularly to the rich and
the noble; and the persons the most exposed to the avarice or
resentment of a provincial magistrate, were thus removed from his
obscure persecution to the more august and impartial tribunal of
the Prætorian præfect. 2. As it was reasonably apprehended that
the integrity of the judge might be biased, if his interest was
concerned, or his affections were engaged, the strictest
regulations were established, to exclude any person, without the
special dispensation of the emperor, from the government of the
province where he was born;\textsuperscript{115} and to prohibit the governor or
his son from contracting marriage with a native, or an
inhabitant;\textsuperscript{116} or from purchasing slaves, lands, or houses,
within the extent of his jurisdiction.\textsuperscript{117} Notwithstanding these
rigorous precautions, the emperor Constantine, after a reign of
twenty-five years, still deplores the venal and oppressive
administration of justice, and expresses the warmest indignation
that the audience of the judge, his despatch of business, his
seasonable delays, and his final sentence, were publicly sold,
either by himself or by the officers of his court. The
continuance, and perhaps the impunity, of these crimes, is
attested by the repetition of impotent laws and ineffectual
menaces.\textsuperscript{118}

\pagenote[113]{Among the works of the celebrated Ulpian, there
was one in ten books, concerning the office of a proconsul, whose
duties in the most essential articles were the same as those of
an ordinary governor of a province.}

\pagenote[114]{The presidents, or consulars, could impose only
two ounces; the vice-præfects, three; the proconsuls, count of
the east, and præfect of Egypt, six. See Heineccii Jur. Civil.
tom. i. p. 75. Pandect. l. xlviii. tit. xix. n. 8. Cod.
Justinian. l. i. tit. liv. leg. 4, 6.}

\pagenote[115]{Ut nulli patriæ suæ administratio sine speciali
principis permissu permittatur. Cod. Justinian. l. i. tit. xli.
This law was first enacted by the emperor Marcus, after the
rebellion of Cassius. (Dion. l. lxxi.) The same regulation is
observed in China, with equal strictness, and with equal effect.}

\pagenote[116]{Pandect. l. xxiii. tit. ii. n. 38, 57, 63.}

\pagenote[117]{In jure continetur, ne quis in administratione
constitutus aliquid compararet. Cod. Theod. l. viii. tit. xv.
leg. l. This maxim of common law was enforced by a series of
edicts (see the remainder of the title) from Constantine to
Justin. From this prohibition, which is extended to the meanest
officers of the governor, they except only clothes and
provisions. The purchase within five years may be recovered;
after which on information, it devolves to the treasury.}

\pagenote[118]{Cessent rapaces jam nunc officialium manus;
cessent, inquam nam si moniti non cessaverint, gladiis
præcidentur, \&c. Cod. Theod. l. i. tit. vii. leg. l. Zeno enacted
that all governors should remain in the province, to answer any
accusations, fifty days after the expiration of their power. Cod
Justinian. l. ii. tit. xlix. leg. l.}

All the civil magistrates were drawn from the profession of the
law. The celebrated Institutes of Justinian are addressed to the
youth of his dominions, who had devoted themselves to the study
of Roman jurisprudence; and the sovereign condescends to animate
their diligence, by the assurance that their skill and ability
would in time be rewarded by an adequate share in the government
of the republic.\textsuperscript{119} The rudiments of this lucrative science were
taught in all the considerable cities of the east and west; but
the most famous school was that of Berytus,\textsuperscript{120} on the coast of
Phœnicia; which flourished above three centuries from the time of
Alexander Severus, the author perhaps of an institution so
advantageous to his native country. After a regular course of
education, which lasted five years, the students dispersed
themselves through the provinces, in search of fortune and
honors; nor could they want an inexhaustible supply of business
in a great empire already corrupted by the multiplicity of laws,
of arts, and of vices. The court of the Prætorian præfect of the
east could alone furnish employment for one hundred and fifty
advocates, sixty-four of whom were distinguished by peculiar
privileges, and two were annually chosen, with a salary of sixty
pounds of gold, to defend the causes of the treasury. The first
experiment was made of their judicial talents, by appointing them
to act occasionally as assessors to the magistrates; from thence
they were often raised to preside in the tribunals before which
they had pleaded. They obtained the government of a province;
and, by the aid of merit, of reputation, or of favor, they
ascended, by successive steps, to the \textit{illustrious} dignities of
the state.\textsuperscript{121} In the practice of the bar, these men had
considered reason as the instrument of dispute; they interpreted
the laws according to the dictates of private interest and the
same pernicious habits might still adhere to their characters in
the public administration of the state. The honor of a liberal
profession has indeed been vindicated by ancient and modern
advocates, who have filled the most important stations, with pure
integrity and consummate wisdom: but in the decline of Roman
jurisprudence, the ordinary promotion of lawyers was pregnant
with mischief and disgrace. The noble art, which had once been
preserved as the sacred inheritance of the patricians, was fallen
into the hands of freedmen and plebeians,\textsuperscript{122} who, with cunning
rather than with skill, exercised a sordid and pernicious trade.
Some of them procured admittance into families for the purpose of
fomenting differences, of encouraging suits, and of preparing a
harvest of gain for themselves or their brethren. Others, recluse
in their chambers, maintained the dignity of legal professors, by
furnishing a rich client with subtleties to confound the plainest
truths, and with arguments to color the most unjustifiable
pretensions. The splendid and popular class was composed of the
advocates, who filled the Forum with the sound of their turgid
and loquacious rhetoric. Careless of fame and of justice, they
are described, for the most part, as ignorant and rapacious
guides, who conducted their clients through a maze of expense, of
delay, and of disappointment; from whence, after a tedious series
of years, they were at length dismissed, when their patience and
fortune were almost exhausted.\textsuperscript{123}

\pagenote[119]{Summâ igitur ope, et alacri studio has leges
nostras accipite; et vosmetipsos sic eruditos ostendite, ut spes
vos pulcherrima foveat; toto legitimo opere perfecto, posse etiam
nostram rempublicam in par tibus ejus vobis credendis gubernari.
Justinian in proem. Institutionum.}

\pagenote[120]{The splendor of the school of Berytus, which
preserved in the east the language and jurisprudence of the
Romans, may be computed to have lasted from the third to the
middle of the sixth century Heinecc. Jur. Rom. Hist. p. 351-356.}

\pagenote[121]{As in a former period I have traced the civil and
military promotion of Pertinax, I shall here insert the civil
honors of Mallius Theodorus. 1. He was distinguished by his
eloquence, while he pleaded as an advocate in the court of the
Prætorian præfect. 2. He governed one of the provinces of Africa,
either as president or consular, and deserved, by his
administration, the honor of a brass statue. 3. He was appointed
vicar, or vice-præfect, of Macedonia. 4. Quæstor. 5. Count of the
sacred largesses. 6. Prætorian præfect of the Gauls; whilst he
might yet be represented as a young man. 7. After a retreat,
perhaps a disgrace of many years, which Mallius (confounded by
some critics with the poet Manilius; see Fabricius Bibliothec.
Latin. Edit. Ernest. tom. i.c. 18, p. 501) employed in the study
of the Grecian philosophy he was named Prætorian præfect of
Italy, in the year 397. 8. While he still exercised that great
office, he was created, it the year 399, consul for the West; and
his name, on account of the infamy of his colleague, the eunuch
Eutropius, often stands alone in the Fasti. 9. In the year 408,
Mallius was appointed a second time Prætorian præfect of Italy.
Even in the venal panegyric of Claudian, we may discover the
merit of Mallius Theodorus, who, by a rare felicity, was the
intimate friend, both of Symmachus and of St. Augustin. See
Tillemont, Hist. des Emp. tom. v. p. 1110-1114.}

\pagenote[122]{Mamertinus in Panegyr. Vet. xi. [x.] 20. Asterius
apud Photium, p. 1500.}

\pagenote[123]{The curious passage of Ammianus, (l. xxx. c. 4,)
in which he paints the manners of contemporary lawyers, affords a
strange mixture of sound sense, false rhetoric, and extravagant
satire. Godefroy (Prolegom. ad. Cod. Theod. c. i. p. 185)
supports the historian by similar complaints and authentic facts.
In the fourth century, many camels might have been laden with
law-books. Eunapius in Vit. Ædesii, p. 72.}

III. In the system of policy introduced by Augustus, the
governors, those at least of the Imperial provinces, were
invested with the full powers of the sovereign himself. Ministers
of peace and war, the distribution of rewards and punishments
depended on them alone, and they successively appeared on their
tribunal in the robes of civil magistracy, and in complete armor
at the head of the Roman legions.\textsuperscript{124} The influence of the
revenue, the authority of law, and the command of a military
force, concurred to render their power supreme and absolute; and
whenever they were tempted to violate their allegiance, the loyal
province which they involved in their rebellion was scarcely
sensible of any change in its political state. From the time of
Commodus to the reign of Constantine, near one hundred governors
might be enumerated, who, with various success, erected the
standard of revolt; and though the innocent were too often
sacrificed, the guilty might be sometimes prevented, by the
suspicious cruelty of their master.\textsuperscript{125} To secure his throne and
the public tranquillity from these formidable servants,
Constantine resolved to divide the military from the civil
administration, and to establish, as a permanent and professional
distinction, a practice which had been adopted only as an
occasional expedient. The supreme jurisdiction exercised by the
Prætorian præfects over the armies of the empire, was transferred
to the two \textit{masters-general} whom he instituted, the one for the
\textit{cavalry}, the other for the \textit{infantry;} and though each of these
\textit{illustrious} officers was more peculiarly responsible for the
discipline of those troops which were under his immediate
inspection, they both indifferently commanded in the field the
several bodies, whether of horse or foot, which were united in
the same army.\textsuperscript{126} Their number was soon doubled by the division
of the east and west; and as separate generals of the same rank
and title were appointed on the four important frontiers of the
Rhine, of the Upper and the Lower Danube, and of the Euphrates,
the defence of the Roman empire was at length committed to eight
masters-general of the cavalry and infantry. Under their orders,
thirty-five military commanders were stationed in the provinces:
three in Britain, six in Gaul, one in Spain, one in Italy, five
on the Upper, and four on the Lower Danube; in Asia, eight, three
in Egypt, and four in Africa. The titles of \textit{counts}, and
\textit{dukes},\textsuperscript{127} by which they were properly distinguished, have
obtained in modern languages so very different a sense, that the
use of them may occasion some surprise. But it should be
recollected, that the second of those appellations is only a
corruption of the Latin word, which was indiscriminately applied
to any military chief. All these provincial generals were
therefore \textit{dukes;} but no more than ten among them were dignified
with the rank of \textit{counts} or companions, a title of honor, or
rather of favor, which had been recently invented in the court of
Constantine. A gold belt was the ensign which distinguished the
office of the counts and dukes; and besides their pay, they
received a liberal allowance sufficient to maintain one hundred
and ninety servants, and one hundred and fifty-eight horses. They
were strictly prohibited from interfering in any matter which
related to the administration of justice or the revenue; but the
command which they exercised over the troops of their department,
was independent of the authority of the magistrates. About the
same time that Constantine gave a legal sanction to the
ecclesiastical order, he instituted in the Roman empire the nice
balance of the civil and the military powers. The emulation, and
sometimes the discord, which reigned between two professions of
opposite interests and incompatible manners, was productive of
beneficial and of pernicious consequences. It was seldom to be
expected that the general and the civil governor of a province
should either conspire for the disturbance, or should unite for
the service, of their country. While the one delayed to offer the
assistance which the other disdained to solicit, the troops very
frequently remained without orders or without supplies; the
public safety was betrayed, and the defenceless subjects were
left exposed to the fury of the Barbarians. The divided
administration which had been formed by Constantine, relaxed the
vigor of the state, while it secured the tranquillity of the
monarch.

\pagenote[124]{See a very splendid example in the life of
Agricola, particularly c. 20, 21. The lieutenant of Britain was
intrusted with the same powers which Cicero, proconsul of
Cilicia, had exercised in the name of the senate and people.}

\pagenote[125]{The Abbé Dubos, who has examined with accuracy
(see Hist. de la Monarchie Françoise, tom. i. p. 41-100, edit.
1742) the institutions of Augustus and of Constantine, observes,
that if Otho had been put to death the day before he executed his
conspiracy, Otho would now appear in history as innocent as
Corbulo.}

\pagenote[126]{Zosimus, l. ii. p. 110. Before the end of the
reign of Constantius, the \textit{magistri militum} were already
increased to four. See Velesius ad Ammian. l. xvi. c. 7.}

\pagenote[127]{Though the military counts and dukes are
frequently mentioned, both in history and the codes, we must have
recourse to the Notitia for the exact knowledge of their number
and stations. For the institution, rank, privileges, \&c., of the
counts in general see Cod. Theod. l. vi. tit. xii.—xx., with the
commentary of Godefroy.}

The memory of Constantine has been deservedly censured for
another innovation, which corrupted military discipline and
prepared the ruin of the empire. The nineteen years which
preceded his final victory over Licinius, had been a period of
license and intestine war. The rivals who contended for the
possession of the Roman world, had withdrawn the greatest part of
their forces from the guard of the general frontier; and the
principal cities which formed the boundary of their respective
dominions were filled with soldiers, who considered their
countrymen as their most implacable enemies. After the use of
these internal garrisons had ceased with the civil war, the
conqueror wanted either wisdom or firmness to revive the severe
discipline of Diocletian, and to suppress a fatal indulgence,
which habit had endeared and almost confirmed to the military
order. From the reign of Constantine, a popular and even legal
distinction was admitted between the \textit{Palatines}\textsuperscript{128} and the
\textit{Borderers;} the troops of the court, as they were improperly
styled, and the troops of the frontier. The former, elevated by
the superiority of their pay and privileges, were permitted,
except in the extraordinary emergencies of war, to occupy their
tranquil stations in the heart of the provinces. The most
flourishing cities were oppressed by the intolerable weight of
quarters. The soldiers insensibly forgot the virtues of their
profession, and contracted only the vices of civil life. They
were either degraded by the industry of mechanic trades, or
enervated by the luxury of baths and theatres. They soon became
careless of their martial exercises, curious in their diet and
apparel; and while they inspired terror to the subjects of the
empire, they trembled at the hostile approach of the Barbarians.\textsuperscript{129}
The chain of fortifications which Diocletian and his
colleagues had extended along the banks of the great rivers, was
no longer maintained with the same care, or defended with the
same vigilance. The numbers which still remained under the name
of the troops of the frontier, might be sufficient for the
ordinary defence; but their spirit was degraded by the
humiliating reflection, that \textit{they} who were exposed to the
hardships and dangers of a perpetual warfare, were rewarded only
with about two thirds of the pay and emoluments which were
lavished on the troops of the court. Even the bands or legions
that were raised the nearest to the level of those unworthy
favorites, were in some measure disgraced by the title of honor
which they were allowed to assume. It was in vain that
Constantine repeated the most dreadful menaces of fire and sword
against the Borderers who should dare desert their colors, to
connive at the inroads of the Barbarians, or to participate in
the spoil.\textsuperscript{130} The mischiefs which flow from injudicious counsels
are seldom removed by the application of partial severities; and
though succeeding princes labored to restore the strength and
numbers of the frontier garrisons, the empire, till the last
moment of its dissolution, continued to languish under the mortal
wound which had been so rashly or so weakly inflicted by the hand
of Constantine.

\pagenote[128]{Zosimus, l ii. p. 111. The distinction between the
two classes of Roman troops, is very darkly expressed in the
historians, the laws, and the Notitia. Consult, however, the
copious \textit{paratitlon}, or abstract, which Godefroy has drawn up of
the seventh book, de Re Militari, of the Theodosian Code, l. vii.
tit. i. leg. 18, l. viii. tit. i. leg. 10.}

\pagenote[129]{Ferox erat in suos miles et rapax, ignavus vero in
hostes et fractus. Ammian. l. xxii. c. 4. He observes, that they
loved downy beds and houses of marble; and that their cups were
heavier than their swords.}

\pagenote[130]{Cod. Theod. l. vii. tit. i. leg. 1, tit. xii. leg.
i. See Howell’s Hist. of the World, vol. ii. p. 19. That learned
historian, who is not sufficiently known, labors to justify the
character and policy of Constantine.}

The same timid policy, of dividing whatever is united, of
reducing whatever is eminent, of dreading every active power, and
of expecting that the most feeble will prove the most obedient,
seems to pervade the institutions of several princes, and
particularly those of Constantine. The martial pride of the
legions, whose victorious camps had so often been the scene of
rebellion, was nourished by the memory of their past exploits,
and the consciousness of their actual strength. As long as they
maintained their ancient establishment of six thousand men, they
subsisted, under the reign of Diocletian, each of them singly, a
visible and important object in the military history of the Roman
empire. A few years afterwards, these gigantic bodies were shrunk
to a very diminutive size; and when \textit{seven} legions, with some
auxiliaries, defended the city of Amida against the Persians, the
total garrison, with the inhabitants of both sexes, and the
peasants of the deserted country, did not exceed the number of
twenty thousand persons.\textsuperscript{131} From this fact, and from similar
examples, there is reason to believe, that the constitution of
the legionary troops, to which they partly owed their valor and
discipline, was dissolved by Constantine; and that the bands of
Roman infantry, which still assumed the same names and the same
honors, consisted only of one thousand or fifteen hundred men.\textsuperscript{132}
The conspiracy of so many separate detachments, each of which
was awed by the sense of its own weakness, could easily be
checked; and the successors of Constantine might indulge their
love of ostentation, by issuing their orders to one hundred and
thirty-two legions, inscribed on the muster-roll of their
numerous armies. The remainder of their troops was distributed
into several hundred cohorts of infantry, and squadrons of
cavalry. Their arms, and titles, and ensigns, were calculated to
inspire terror, and to display the variety of nations who marched
under the Imperial standard. And not a vestige was left of that
severe simplicity, which, in the ages of freedom and victory, had
distinguished the line of battle of a Roman army from the
confused host of an Asiatic monarch.\textsuperscript{133} A more particular
enumeration, drawn from the \textit{Notitia}, might exercise the
diligence of an antiquary; but the historian will content himself
with observing, that the number of permanent stations or
garrisons established on the frontiers of the empire, amounted to
five hundred and eighty-three; and that, under the successors of
Constantine, the complete force of the military establishment was
computed at six hundred and forty-five thousand soldiers.\textsuperscript{134} An
effort so prodigious surpassed the wants of a more ancient, and
the faculties of a later, period.

\pagenote[131]{Ammian. l. xix. c. 2. He observes, (c. 5,) that
the desperate sallies of two Gallic legions were like a handful
of water thrown on a great conflagration.}

\pagenote[132]{Pancirolus ad Notitiam, p. 96. Mémoires de
l’Académie des Inscriptions, tom. xxv. p. 491.}

\pagenote[133]{Romana acies unius prope formæ erat et hominum et
armorum genere.—Regia acies varia magis multis gentibus
dissimilitudine armorum auxiliorumque erat. T. Liv. l. xxxvii. c.
39, 40. Flaminius, even before the event, had compared the army
of Antiochus to a supper in which the flesh of one vile animal
was diversified by the skill of the cooks. See the Life of
Flaminius in Plutarch.}

\pagenote[134]{Agathias, l. v. p. 157, edit. Louvre.}

In the various states of society, armies are recruited from very
different motives. Barbarians are urged by the love of war; the
citizens of a free republic may be prompted by a principle of
duty; the subjects, or at least the nobles, of a monarchy, are
animated by a sentiment of honor; but the timid and luxurious
inhabitants of a declining empire must be allured into the
service by the hopes of profit, or compelled by the dread of
punishment. The resources of the Roman treasury were exhausted by
the increase of pay, by the repetition of donatives, and by the
invention of new emolument and indulgences, which, in the opinion
of the provincial youth might compensate the hardships and
dangers of a military life. Yet, although the stature was
lowered,\textsuperscript{135} although slaves, least by a tacit connivance, were
indiscriminately received into the ranks, the insurmountable
difficulty of procuring a regular and adequate supply of
volunteers, obliged the emperors to adopt more effectual and
coercive methods. The lands bestowed on the veterans, as the free
reward of their valor were henceforward granted under a condition
which contain the first rudiments of the feudal tenures; that
their sons, who succeeded to the inheritance, should devote
themselves to the profession of arms, as soon as they attained
the age of manhood; and their cowardly refusal was punished by
the loss of honor, of fortune, or even of life.\textsuperscript{136} But as the
annual growth of the sons of the veterans bore a very small
proportion to the demands of the service, levies of men were
frequently required from the provinces, and every proprietor was
obliged either to take up arms, or to procure a substitute, or to
purchase his exemption by the payment of a heavy fine. The sum of
forty-two pieces of gold, to which it was \textit{reduced} ascertains
the exorbitant price of volunteers, and the reluctance with which
the government admitted of this alternative.\textsuperscript{137} Such was the
horror for the profession of a soldier, which had affected the
minds of the degenerate Romans, that many of the youth of Italy
and the provinces chose to cut off the fingers of their right
hand, to escape from being pressed into the service; and this
strange expedient was so commonly practised, as to deserve the
severe animadversion of the laws,\textsuperscript{138} and a peculiar name in the
Latin language.\textsuperscript{139}

\pagenote[135]{Valentinian (Cod. Theodos. l. vii. tit. xiii. leg.
3) fixes the standard at five feet seven inches, about five feet
four inches and a half, English measure. It had formerly been
five feet ten inches, and in the best corps, six Roman feet. Sed
tunc erat amplior multitude se et plures sequebantur militiam
armatam. Vegetius de Re Militari l. i. c. v.}

\pagenote[136]{See the two titles, De Veteranis and De Filiis
Veteranorum, in the seventh book of the Theodosian Code. The age
at which their military service was required, varied from
twenty-five to sixteen. If the sons of the veterans appeared with
a horse, they had a right to serve in the cavalry; two horses
gave them some valuable privileges}

\pagenote[137]{Cod. Theod. l. vii. tit. xiii. leg. 7. According
to the historian Socrates, (see Godefroy ad loc.,) the same
emperor Valens sometimes required eighty pieces of gold for a
recruit. In the following law it is faintly expressed, that
slaves shall not be admitted inter optimas lectissimorum militum
turmas.}

\pagenote[138]{The person and property of a Roman knight, who had
mutilated his two sons, were sold at public auction by order of
Augustus. (Sueton. in August. c. 27.) The moderation of that
artful usurper proves, that this example of severity was
justified by the spirit of the times. Ammianus makes a
distinction between the effeminate Italians and the hardy Gauls.
(L. xv. c. 12.) Yet only 15 years afterwards, Valentinian, in a
law addressed to the præfect of Gaul, is obliged to enact that
these cowardly deserters shall be burnt alive. (Cod. Theod. l.
vii. tit. xiii. leg. 5.) Their numbers in Illyricum were so
considerable, that the province complained of a scarcity of
recruits. (Id. leg. 10.)}

\pagenote[139]{They were called \textit{Murci. Murcidus} is found in
Plautus and Festus, to denote a lazy and cowardly person, who,
according to Arnobius and Augustin, was under the immediate
protection of the goddess \textit{Murcia}. From this particular instance
of cowardice, \textit{murcare} is used as synonymous to \textit{mutilare}, by
the writers of the middle Latinity. See Linder brogius and
Valesius ad Ammian. Marcellin, l. xv. c. 12}

\section{Part \thesection.}

The introduction of Barbarians into the Roman armies became every
day more universal, more necessary, and more fatal. The most
daring of the Scythians, of the Goths, and of the Germans, who
delighted in war, and who found it more profitable to defend than
to ravage the provinces, were enrolled, not only in the
auxiliaries of their respective nations, but in the legions
themselves, and among the most distinguished of the Palatine
troops. As they freely mingled with the subjects of the empire,
they gradually learned to despise their manners, and to imitate
their arts. They abjured the implicit reverence which the pride
of Rome had exacted from their ignorance, while they acquired the
knowledge and possession of those advantages by which alone she
supported her declining greatness. The Barbarian soldiers, who
displayed any military talents, were advanced, without exception,
to the most important commands; and the names of the tribunes, of
the counts and dukes, and of the generals themselves, betray a
foreign origin, which they no longer condescended to disguise.
They were often intrusted with the conduct of a war against their
countrymen; and though most of them preferred the ties of
allegiance to those of blood, they did not always avoid the
guilt, or at least the suspicion, of holding a treasonable
correspondence with the enemy, of inviting his invasion, or of
sparing his retreat. The camps and the palace of the son of
Constantine were governed by the powerful faction of the Franks,
who preserved the strictest connection with each other, and with
their country, and who resented every personal affront as a
national indignity.\textsuperscript{140} When the tyrant Caligula was suspected of
an intention to invest a very extraordinary candidate with the
consular robes, the sacrilegious profanation would have scarcely
excited less astonishment, if, instead of a horse, the noblest
chieftain of Germany or Britain had been the object of his
choice. The revolution of three centuries had produced so
remarkable a change in the prejudices of the people, that, with
the public approbation, Constantine showed his successors the
example of bestowing the honors of the consulship on the
Barbarians, who, by their merit and services, had deserved to be
ranked among the first of the Romans.\textsuperscript{141} But as these hardy
veterans, who had been educated in the ignorance or contempt of
the laws, were incapable of exercising any civil offices, the
powers of the human mind were contracted by the irreconcilable
separation of talents as well as of professions. The accomplished
citizens of the Greek and Roman republics, whose characters could
adapt themselves to the bar, the senate, the camp, or the
schools, had learned to write, to speak, and to act with the same
spirit, and with equal abilities.

\pagenote[140]{Malarichus—adhibitis Francis quorum ea tempestate
in palatio multitudo florebat, erectius jam loquebatur
tumultuabaturque. Ammian. l. xv. c. 5.}

\pagenote[141]{Barbaros omnium primus, ad usque fasces auxerat et
trabeas consulares. Ammian. l. xx. c. 10. Eusebius (in Vit.
Constantin. l. iv c.7) and Aurelius Victor seem to confirm the
truth of this assertion yet in the thirty-two consular Fasti of
the reign of Constantine cannot discover the name of a single
Barbarian. I should therefore interpret the liberality of that
prince as relative to the ornaments rather than to the office, of
the consulship.}

IV. Besides the magistrates and generals, who at a distance from
the court diffused their delegated authority over the provinces
and armies, the emperor conferred the rank of \textit{Illustrious} on
seven of his more immediate servants, to whose fidelity he
intrusted his safety, or his counsels, or his treasures. 1. The
private apartments of the palace were governed by a favorite
eunuch, who, in the language of that age, was styled the
\textit{præpositus}, or præfect of the sacred bed-chamber. His duty was
to attend the emperor in his hours of state, or in those of
amusement, and to perform about his person all those menial
services, which can only derive their splendor from the influence
of royalty. Under a prince who deserved to reign, the great
chamberlain (for such we may call him) was a useful and humble
domestic; but an artful domestic, who improves every occasion of
unguarded confidence, will insensibly acquire over a feeble mind
that ascendant which harsh wisdom and uncomplying virtue can
seldom obtain. The degenerate grandsons of Theodosius, who were
invisible to their subjects, and contemptible to their enemies,
exalted the præfects of their bed-chamber above the heads of all
the ministers of the palace;\textsuperscript{142} and even his deputy, the first
of the splendid train of slaves who waited in the presence, was
thought worthy to rank before the \textit{respectable} proconsuls of
Greece or Asia. The jurisdiction of the chamberlain was
acknowledged by the \textit{counts}, or superintendents, who regulated
the two important provinces of the magnificence of the wardrobe,
and of the luxury of the Imperial table.\textsuperscript{143} 2. The principal
administration of public affairs was committed to the diligence
and abilities of the \textit{master of the offices}.\textsuperscript{144} He was the
supreme magistrate of the palace, inspected the discipline of the
civil and military \textit{schools}, and received appeals from all parts
of the empire, in the causes which related to that numerous army
of privileged persons, who, as the servants of the court, had
obtained for themselves and families a right to decline the
authority of the ordinary judges. The correspondence between the
prince and his subjects was managed by the four \textit{scrinia}, or
offices of this minister of state. The first was appropriated to
memorials, the second to epistles, the third to petitions, and
the fourth to papers and orders of a miscellaneous kind. Each of
these was directed by an \textit{inferior} master of \textit{respectable}
dignity, and the whole business was despatched by a hundred and
forty-eight secretaries, chosen for the most part from the
profession of the law, on account of the variety of abstracts of
reports and references which frequently occurred in the exercise
of their several functions. From a condescension, which in former
ages would have been esteemed unworthy the Roman majesty, a
particular secretary was allowed for the Greek language; and
interpreters were appointed to receive the ambassadors of the
Barbarians; but the department of foreign affairs, which
constitutes so essential a part of modern policy, seldom diverted
the attention of the master of the offices. His mind was more
seriously engaged by the general direction of the posts and
arsenals of the empire. There were thirty-four cities, fifteen in
the East, and nineteen in the West, in which regular companies of
workmen were perpetually employed in fabricating defensive armor,
offensive weapons of all sorts, and military engines, which were
deposited in the arsenals, and occasionally delivered for the
service of the troops. 3. In the course of nine centuries, the
office of \textit{quæstor} had experienced a very singular revolution.
In the infancy of Rome, two inferior magistrates were annually
elected by the people, to relieve the consuls from the invidious
management of the public treasure;\textsuperscript{145} a similar assistant was
granted to every proconsul, and to every prætor, who exercised a
military or provincial command; with the extent of conquest, the
two quæstors were gradually multiplied to the number of four, of
eight, of twenty, and, for a short time, perhaps, of forty;\textsuperscript{146}
and the noblest citizens ambitiously solicited an office which
gave them a seat in the senate, and a just hope of obtaining the
honors of the republic. Whilst Augustus affected to maintain the
freedom of election, he consented to accept the annual privilege
of recommending, or rather indeed of nominating, a certain
proportion of candidates; and it was his custom to select one of
these distinguished youths, to read his orations or epistles in
the assemblies of the senate.\textsuperscript{147} The practice of Augustus was
imitated by succeeding princes; the occasional commission was
established as a permanent office; and the favored quæstor,
assuming a new and more illustrious character, alone survived the
suppression of his ancient and useless colleagues.\textsuperscript{148} As the
orations which he composed in the name of the emperor,\textsuperscript{149}
acquired the force, and, at length, the form, of absolute edicts,
he was considered as the representative of the legislative power,
the oracle of the council, and the original source of the civil
jurisprudence. He was sometimes invited to take his seat in the
supreme judicature of the Imperial consistory, with the Prætorian
præfects, and the master of the offices; and he was frequently
requested to resolve the doubts of inferior judges: but as he was
not oppressed with a variety of subordinate business, his leisure
and talents were employed to cultivate that dignified style of
eloquence, which, in the corruption of taste and language, still
preserves the majesty of the Roman laws.\textsuperscript{150} In some respects,
the office of the Imperial quæstor may be compared with that of a
modern chancellor; but the use of a great seal, which seems to
have been adopted by the illiterate barbarians, was never
introduced to attest the public acts of the emperors. 4. The
extraordinary title of \textit{count of the sacred largesses} was
bestowed on the treasurer-general of the revenue, with the
intention perhaps of inculcating, that every payment flowed from
the voluntary bounty of the monarch. To conceive the almost
infinite detail of the annual and daily expense of the civil and
military administration in every part of a great empire, would
exceed the powers of the most vigorous imagination.

The actual account employed several hundred persons, distributed
into eleven different offices, which were artfully contrived to
examine and control their respective operations. The multitude of
these agents had a natural tendency to increase; and it was more
than once thought expedient to dismiss to their native homes the
useless supernumeraries, who, deserting their honest labors, had
pressed with too much eagerness into the lucrative profession of
the finances.\textsuperscript{151} Twenty-nine provincial receivers, of whom
eighteen were honored with the title of count, corresponded with
the treasurer; and he extended his jurisdiction over the mines
from whence the precious metals were extracted, over the mints,
in which they were converted into the current coin, and over the
public treasuries of the most important cities, where they were
deposited for the service of the state. The foreign trade of the
empire was regulated by this minister, who directed likewise all
the linen and woollen manufactures, in which the successive
operations of spinning, weaving, and dyeing were executed,
chiefly by women of a servile condition, for the use of the
palace and army. Twenty-six of these institutions are enumerated
in the West, where the arts had been more recently introduced,
and a still larger proportion may be allowed for the industrious
provinces of the East.\textsuperscript{152} 5. Besides the public revenue, which
an absolute monarch might levy and expend according to his
pleasure, the emperors, in the capacity of opulent citizens,
possessed a very extensive property, which was administered by
the \textit{count} or treasurer of \textit{the private estate}. Some part had
perhaps been the ancient demesnes of kings and republics; some
accessions might be derived from the families which were
successively invested with the purple; but the most considerable
portion flowed from the impure source of confiscations and
forfeitures. The Imperial estates were scattered through the
provinces, from Mauritania to Britain; but the rich and fertile
soil of Cappadocia tempted the monarch to acquire in that country
his fairest possessions,\textsuperscript{153} and either Constantine or his
successors embraced the occasion of justifying avarice by
religious zeal. They suppressed the rich temple of Comana, where
the high priest of the goddess of war supported the dignity of a
sovereign prince; and they applied to their private use the
consecrated lands, which were inhabited by six thousand subjects
or slaves of the deity and her ministers.\textsuperscript{154} But these were not
the valuable inhabitants: the plains that stretch from the foot
of Mount Argæus to the banks of the Sarus, bred a generous race
of horses, renowned above all others in the ancient world for
their majestic shape and incomparable swiftness. These \textit{sacred}
animals, destined for the service of the palace and the Imperial
games, were protected by the laws from the profanation of a
vulgar master.\textsuperscript{155} The demesnes of Cappadocia were important
enough to require the inspection of a count;\textsuperscript{156} officers of an
inferior rank were stationed in the other parts of the empire;
and the deputies of the private, as well as those of the public,
treasurer were maintained in the exercise of their independent
functions, and encouraged to control the authority of the
provincial magistrates.\textsuperscript{157} 6, 7. The chosen bands of cavalry and
infantry, which guarded the person of the emperor, were under the
immediate command of the \textit{two counts of the domestics}. The whole
number consisted of three thousand five hundred men, divided into
seven \textit{schools}, or troops, of five hundred each; and in the
East, this honorable service was almost entirely appropriated to
the Armenians. Whenever, on public ceremonies, they were drawn up
in the courts and porticos of the palace, their lofty stature,
silent order, and splendid arms of silver and gold, displayed a
martial pomp not unworthy of the Roman majesty.\textsuperscript{158} From the
seven schools two companies of horse and foot were selected, of
the \textit{protectors}, whose advantageous station was the hope and
reward of the most deserving soldiers. They mounted guard in the
interior apartments, and were occasionally despatched into the
provinces, to execute with celerity and vigor the orders of their
master.\textsuperscript{159} The counts of the domestics had succeeded to the
office of the Prætorian præfects; like the præfects, they aspired
from the service of the palace to the command of armies.

\pagenote[142]{Cod. Theod. l. vi. tit. 8.}

\pagenote[143]{By a very singular metaphor, borrowed from the
military character of the first emperors, the steward of their
household was styled the count of their camp, (comes castrensis.)
Cassiodorus very seriously represents to him, that his own fame,
and that of the empire, must depend on the opinion which foreign
ambassadors may conceive of the plenty and magnificence of the
royal table. (Variar. l. vi. epistol. 9.)}

\pagenote[144]{Gutherius (de Officiis Domûs Augustæ, l. ii. c.
20, l. iii.) has very accurately explained the functions of the
master of the offices, and the constitution of the subordinate
\textit{scrinia}. But he vainly attempts, on the most doubtful
authority, to deduce from the time of the Antonines, or even of
Nero, the origin of a magistrate who cannot be found in history
before the reign of Constantine.}

\pagenote[145]{Tacitus (Annal. xi. 22) says, that the first
quæstors were elected by the people, sixty-four years after the
foundation of the republic; but he is of opinion, that they had,
long before that period, been annually appointed by the consuls,
and even by the kings. But this obscure point of antiquity is
contested by other writers.}

\pagenote[146]{Tacitus (Annal. xi. 22) seems to consider twenty
as the highest number of quæstors; and Dion (l. xliii. p 374)
insinuates, that if the dictator Cæsar once created forty, it was
only to facilitate the payment of an immense debt of gratitude.
Yet the augmentation which he made of prætors subsisted under the
succeeding reigns.}

\pagenote[147]{Sueton. in August. c. 65, and Torrent. ad loc.
Dion. Cas. p. 755.}

\pagenote[148]{The youth and inexperience of the quæstors, who
entered on that important office in their twenty-fifth year,
(Lips. Excurs. ad Tacit. l. iii. D.,) engaged Augustus to remove
them from the management of the treasury; and though they were
restored by Claudius, they seem to have been finally dismissed by
Nero. (Tacit Annal. xiii. 29. Sueton. in Aug. c. 36, in Claud. c.
24. Dion, p. 696, 961, \&c. Plin. Epistol. x. 20, et alibi.) In
the provinces of the Imperial division, the place of the quæstors
was more ably supplied by the \textit{procurators}, (Dion Cas. p. 707.
Tacit. in Vit. Agricol. c. 15;) or, as they were afterwards
called, \textit{rationales}. (Hist. August. p. 130.) But in the
provinces of the senate we may still discover a series of
quæstors till the reign of Marcus Antoninus. (See the
Inscriptions of Gruter, the Epistles of Pliny, and a decisive
fact in the Augustan History, p. 64.) From Ulpian we may learn,
(Pandect. l. i. tit. 13,) that under the government of the house
of Severus, their provincial administration was abolished; and in
the subsequent troubles, the annual or triennial elections of
quæstors must have naturally ceased.}

\pagenote[149]{Cum patris nomine et epistolas ipse dictaret, et
edicta conscrib eret, orationesque in senatu recitaret, etiam
quæstoris vice. Sueton, in Tit. c. 6. The office must have
acquired new dignity, which was occasionally executed by the heir
apparent of the empire. Trajan intrusted the same care to
Hadrian, his quæstor and cousin. See Dodwell, Prælection.
Cambden, x. xi. p. 362-394.}

\pagenote[150]{Terris edicta daturus; Supplicibus
responsa.—Oracula regis Eloquio crevere tuo; nec dignius unquam
Majestas meminit sese Romana locutam.——Claudian in Consulat.
Mall. Theodor. 33. See likewise Symmachus (Epistol. i. 17) and
Cassiodorus. (Variar. iv. 5.)}

\pagenote[151]{Cod. Theod. l. vi. tit. 30. Cod. Justinian. l.
xii. tit. 24.}

\pagenote[152]{In the departments of the two counts of the
treasury, the eastern part of the \textit{Notitia} happens to be very
defective. It may be observed, that we had a treasury chest in
London, and a gyneceum or manufacture at Winchester. But Britain
was not thought worthy either of a mint or of an arsenal. Gaul
alone possessed three of the former, and eight of the latter.}

\pagenote[153]{Cod. Theod. l. vi. tit. xxx. leg. 2, and Godefroy
ad loc.}

\pagenote[154]{Strabon. Geograph. l. xxii. p. 809, [edit.
Casaub.] The other temple of Comana, in Pontus, was a colony from
that of Cappadocia, l. xii. p. 835. The President Des Brosses
(see his Saluste, tom. ii. p. 21, [edit. Causub.]) conjectures
that the deity adored in both Comanas was Beltis, the Venus of
the east, the goddess of generation; a very different being
indeed from the goddess of war.}

\pagenote[155]{Cod. Theod. l. x. tit. vi. de Grege Dominico.
Godefroy has collected every circumstance of antiquity relative
to the Cappadocian horses. One of the finest breeds, the
Palmatian, was the forfeiture of a rebel, whose estate lay about
sixteen miles from Tyana, near the great road between
Constantinople and Antioch.}

\pagenote[156]{Justinian (Novell. 30) subjected the province of
the count of Cappadocia to the immediate authority of the
favorite eunuch, who presided over the sacred bed-chamber.}

\pagenote[157]{Cod. Theod. l. vi. tit. xxx. leg. 4, \&c.}

\pagenote[158]{Pancirolus, p. 102, 136. The appearance of these
military domestics is described in the Latin poem of Corippus, de
Laudibus Justin. l. iii. 157-179. p. 419, 420 of the Appendix
Hist. Byzantin. Rom. 177.}

\pagenote[159]{Ammianus Marcellinus, who served so many years,
obtained only the rank of a protector. The first ten among these
honorable soldiers were \textit{Clarissimi}.}

The perpetual intercourse between the court and the provinces was
facilitated by the construction of roads and the institution of
posts. But these beneficial establishments were accidentally
connected with a pernicious and intolerable abuse. Two or three
hundred \textit{agents} or messengers were employed, under the
jurisdiction of the master of the offices, to announce the names
of the annual consuls, and the edicts or victories of the
emperors. They insensibly assumed the license of reporting
whatever they could observe of the conduct either of magistrates
or of private citizens; and were soon considered as the eyes of
the monarch,\textsuperscript{160} and the scourge of the people. Under the warm
influence of a feeble reign, they multiplied to the incredible
number of ten thousand, disdained the mild though frequent
admonitions of the laws, and exercised in the profitable
management of the posts a rapacious and insolent oppression.
These official spies, who regularly corresponded with the palace,
were encouraged by favor and reward, anxiously to watch the
progress of every treasonable design, from the faint and latent
symptoms of disaffection, to the actual preparation of an open
revolt. Their careless or criminal violation of truth and justice
was covered by the consecrated mask of zeal; and they might
securely aim their poisoned arrows at the breast either of the
guilty or the innocent, who had provoked their resentment, or
refused to purchase their silence. A faithful subject, of Syria
perhaps, or of Britain, was exposed to the danger, or at least to
the dread, of being dragged in chains to the court of Milan or
Constantinople, to defend his life and fortune against the
malicious charge of these privileged informers. The ordinary
administration was conducted by those methods which extreme
necessity can alone palliate; and the defects of evidence were
diligently supplied by the use of torture.\textsuperscript{161}

\pagenote[160]{Xenophon, Cyropæd. l. viii. Brisson, de Regno
Persico, l. i No 190, p. 264. The emperors adopted with pleasure
this Persian metaphor.}

\pagenote[161]{For the \textit{Agentes in Rebus}, see Ammian. l. xv. c.
3, l. xvi. c. 5, l. xxii. c. 7, with the curious annotations of
Valesius. Cod. Theod. l. vi. tit. xxvii. xxviii. xxix. Among the
passages collected in the Commentary of Godefroy, the most
remarkable is one from Libanius, in his discourse concerning the
death of Julian.}

The deceitful and dangerous experiment of the criminal
\textit{quæstion}, as it is emphatically styled, was admitted, rather
than approved, in the jurisprudence of the Romans. They applied
this sanguinary mode of examination only to servile bodies, whose
sufferings were seldom weighed by those haughty republicans in
the scale of justice or humanity; but they would never consent to
violate the sacred person of a citizen, till they possessed the
clearest evidence of his guilt.\textsuperscript{162} The annals of tyranny, from
the reign of Tiberius to that of Domitian, circumstantially
relate the executions of many innocent victims; but, as long as
the faintest remembrance was kept alive of the national freedom
and honor, the last hours of a Roman were secured from the danger
of ignominions torture.\textsuperscript{163} The conduct of the provincial
magistrates was not, however, regulated by the practice of the
city, or the strict maxims of the civilians. They found the use
of torture established not only among the slaves of oriental
despotism, but among the Macedonians, who obeyed a limited
monarch; among the Rhodians, who flourished by the liberty of
commerce; and even among the sage Athenians, who had asserted and
adorned the dignity of human kind.\textsuperscript{164} The acquiescence of the
provincials encouraged their governors to acquire, or perhaps to
usurp, a discretionary power of employing the rack, to extort
from vagrants or plebeian criminals the confession of their
guilt, till they insensibly proceeded to confound the distinction
of rank, and to disregard the privileges of Roman citizens. The
apprehensions of the subjects urged them to solicit, and the
interest of the sovereign engaged him to grant, a variety of
special exemptions, which tacitly allowed, and even authorized,
the general use of torture. They protected all persons of
illustrious or honorable rank, bishops and their presbyters,
professors of the liberal arts, soldiers and their families,
municipal officers, and their posterity to the third generation,
and all children under the age of puberty.\textsuperscript{165} But a fatal maxim
was introduced into the new jurisprudence of the empire, that in
the case of treason, which included every offence that the
subtlety of lawyers could derive from a \textit{hostile intention}
towards the prince or republic,\textsuperscript{166} all privileges were
suspended, and all conditions were reduced to the same
ignominious level. As the safety of the emperor was avowedly
preferred to every consideration of justice or humanity, the
dignity of age and the tenderness of youth were alike exposed to
the most cruel tortures; and the terrors of a malicious
information, which might select them as the accomplices, or even
as the witnesses, perhaps, of an imaginary crime, perpetually
hung over the heads of the principal citizens of the Roman world.\textsuperscript{167}

\pagenote[162]{The Pandects (l. xlviii. tit. xviii.) contain the
sentiments of the most celebrated civilians on the subject of
torture. They strictly confine it to slaves; and Ulpian himself
is ready to acknowledge that Res est fragilis, et periculosa, et
quæ veritatem fallat.}

\pagenote[163]{In the conspiracy of Piso against Nero, Epicharis
(libertina mulier) was the only person tortured; the rest were
\textit{intacti tormentis}. It would be superfluous to add a weaker, and
it would be difficult to find a stronger, example. Tacit. Annal.
xv. 57.}

\pagenote[164]{Dicendum... de Institutis Atheniensium, Rhodiorum,
doctissimorum hominum, apud quos etiam (id quod acerbissimum est)
liberi, civesque torquentur. Cicero, Partit. Orat. c. 34. We may
learn from the trial of Philotas the practice of the Macedonians.
(Diodor. Sicul. l. xvii. p. 604. Q. Curt. l. vi. c. 11.)}

\pagenote[165]{Heineccius (Element. Jur. Civil. part vii. p. 81)
has collected these exemptions into one view.}

\pagenote[166]{This definition of the sage Ulpian (Pandect. l.
xlviii. tit. iv.) seems to have been adapted to the court of
Caracalla, rather than to that of Alexander Severus. See the
Codes of Theodosius and ad leg. Juliam majestatis.}

\pagenote[167]{Arcadius Charisius is the oldest lawyer quoted to
justify the universal practice of torture in all cases of
treason; but this maxim of tyranny, which is admitted by Ammianus
with the most respectful terror, is enforced by several laws of
the successors of Constantine. See Cod. Theod. l. ix. tit. xxxv.
majestatis crimine omnibus æqua est conditio.}

These evils, however terrible they may appear, were confined to
the smaller number of Roman subjects, whose dangerous situation
was in some degree compensated by the enjoyment of those
advantages, either of nature or of fortune, which exposed them to
the jealousy of the monarch. The obscure millions of a great
empire have much less to dread from the cruelty than from the
avarice of their masters, and \textit{their} humble happiness is
principally affected by the grievance of excessive taxes, which,
gently pressing on the wealthy, descend with accelerated weight
on the meaner and more indigent classes of society. An ingenious
philosopher\textsuperscript{168} has calculated the universal measure of the
public impositions by the degrees of freedom and servitude; and
ventures to assert, that, according to an invariable law of
nature, it must always increase with the former, and diminish in
a just proportion to the latter. But this reflection, which would
tend to alleviate the miseries of despotism, is contradicted at
least by the history of the Roman empire; which accuses the same
princes of despoiling the senate of its authority, and the
provinces of their wealth. Without abolishing all the various
customs and duties on merchandises, which are imperceptibly
discharged by the apparent choice of the purchaser, the policy of
Constantine and his successors preferred a simple and direct mode
of taxation, more congenial to the spirit of an arbitrary
government.\textsuperscript{169}

\pagenote[168]{Montesquieu, Esprit des Loix, l. xii. c. 13.}

\pagenote[169]{Mr. Hume (Essays, vol. i. p. 389) has seen this
importance with some degree of perplexity.}

\section{Part \thesection.}

The name and use of the \textit{indictions},\textsuperscript{170} which serve to
ascertain the chronology of the middle ages, were derived from
the regular practice of the Roman tributes.\textsuperscript{171} The emperor
subscribed with his own hand, and in purple ink, the solemn
edict, or indiction, which was fixed up in the principal city of
each diocese, during two months previous to the first day of
September. And by a very easy connection of ideas, the word
\textit{indiction} was transferred to the measure of tribute which it
prescribed, and to the annual term which it allowed for the
payment. This general estimate of the supplies was proportioned
to the real and imaginary wants of the state; but as often as the
expense exceeded the revenue, or the revenue fell short of the
computation, an additional tax, under the name of
\textit{superindiction}, was imposed on the people, and the most
valuable attribute of sovereignty was communicated to the
Prætorian præfects, who, on some occasions, were permitted to
provide for the unforeseen and extraordinary exigencies of the
public service. The execution of these laws (which it would be
tedious to pursue in their minute and intricate detail) consisted
of two distinct operations: the resolving the general imposition
into its constituent parts, which were assessed on the provinces,
the cities, and the individuals of the Roman world; and the
collecting the separate contributions of the individuals, the
cities, and the provinces, till the accumulated sums were poured
into the Imperial treasuries. But as the account between the
monarch and the subject was perpetually open, and as the renewal
of the demand anticipated the perfect discharge of the preceding
obligation, the weighty machine of the finances was moved by the
same hands round the circle of its yearly revolution. Whatever
was honorable or important in the administration of the revenue,
was committed to the wisdom of the præfects, and their provincia.
representatives; the lucrative functions were claimed by a crowd
of subordinate officers, some of whom depended on the treasurer,
others on the governor of the province; and who, in the
inevitable conflicts of a perplexed jurisdiction, had frequent
opportunities of disputing with each other the spoils of the
people. The laborious offices, which could be productive only of
envy and reproach, of expense and danger, were imposed on the
\textit{Decurions}, who formed the corporations of the cities, and whom
the severity of the Imperial laws had condemned to sustain the
burdens of civil society.\textsuperscript{172} The whole landed property of the
empire (without excepting the patrimonial estates of the monarch)
was the object of ordinary taxation; and every new purchaser
contracted the obligations of the former proprietor. An accurate
\textit{census},\textsuperscript{173} or survey, was the only equitable mode of
ascertaining the proportion which every citizen should be obliged
to contribute for the public service; and from the well-known
period of the indictions, there is reason to believe that this
difficult and expensive operation was repeated at the regular
distance of fifteen years. The lands were measured by surveyors,
who were sent into the provinces; their nature, whether arable or
pasture, or vineyards or woods, was distinctly reported; and an
estimate was made of their common value from the average produce
of five years. The numbers of slaves and of cattle constituted an
essential part of the report; an oath was administered to the
proprietors, which bound them to disclose the true state of their
affairs; and their attempts to prevaricate, or elude the
intention of the legislator, were severely watched, and punished
as a capital crime, which included the double guilt of treason
and sacrilege.\textsuperscript{174} A large portion of the tribute was paid in
money; and of the current coin of the empire, gold alone could be
legally accepted.\textsuperscript{175} The remainder of the taxes, according to
the proportions determined by the annual indiction, was furnished
in a manner still more direct, and still more oppressive.
According to the different nature of lands, their real produce in
the various articles of wine or oil, corn or barley, wood or
iron, was transported by the labor or at the expense of the
provincials\textsuperscript{17511} to the Imperial magazines, from whence they
were occasionally distributed for the use of the court, of the
army, and of two capitals, Rome and Constantinople. The
commissioners of the revenue were so frequently obliged to make
considerable purchases, that they were strictly prohibited from
allowing any compensation, or from receiving in money the value
of those supplies which were exacted in kind. In the primitive
simplicity of small communities, this method may be well adapted
to collect the almost voluntary offerings of the people; but it
is at once susceptible of the utmost latitude, and of the utmost
strictness, which in a corrupt and absolute monarchy must
introduce a perpetual contest between the power of oppression and
the arts of fraud.\textsuperscript{176} The agriculture of the Roman provinces was
insensibly ruined, and, in the progress of despotism which tends
to disappoint its own purpose, the emperors were obliged to
derive some merit from the forgiveness of debts, or the remission
of tributes, which their subjects were utterly incapable of
paying. According to the new division of Italy, the fertile and
happy province of Campania, the scene of the early victories and
of the delicious retirements of the citizens of Rome, extended
between the sea and the Apennine, from the Tiber to the Silarus.
Within sixty years after the death of Constantine, and on the
evidence of an actual survey, an exemption was granted in favor
of three hundred and thirty thousand English acres of desert and
uncultivated land; which amounted to one eighth of the whole
surface of the province. As the footsteps of the Barbarians had
not yet been seen in Italy, the cause of this amazing desolation,
which is recorded in the laws, can be ascribed only to the
administration of the Roman emperors.\textsuperscript{177}

\pagenote[170]{The cycle of indictions, which may be traced as
high as the reign of Constantius, or perhaps of his father,
Constantine, is still employed by the Papal court; but the
commencement of the year has been very reasonably altered to the
first of January. See l’Art de Verifier les Dates, p. xi.; and
Dictionnaire Raison. de la Diplomatique, tom. ii. p. 25; two
accurate treatises, which come from the workshop of the
Benedictines. —— It does not appear that the establishment of the
indiction is to be at tributed to Constantine: it existed before
he had been created \textit{Augustus} at Rome, and the remission granted
by him to the city of Autun is the proof. He would not have
ventured while only \textit{Cæsar}, and under the necessity of courting
popular favor, to establish such an odious impost. Aurelius
Victor and Lactantius agree in designating Diocletian as the
author of this despotic institution. Aur. Vict. de Cæs. c. 39.
Lactant. de Mort. Pers. c. 7—G.}

\pagenote[171]{The first twenty-eight titles of the eleventh book
of the Theodosian Code are filled with the circumstantial
regulations on the important subject of tributes; but they
suppose a clearer knowledge of fundamental principles than it is
at present in our power to attain.}

\pagenote[172]{The title concerning the Decurions (l. xii. tit.
i.) is the most ample in the whole Theodosian Code; since it
contains not less than one hundred and ninety-two distinct laws
to ascertain the duties and privileges of that useful order of
citizens. * Note: The Decurions were charged with assessing,
according to the census of property prepared by the tabularii,
the payment due from each proprietor. This odious office was
authoritatively imposed on the richest citizens of each town;
they had no salary, and all their compensation was, to be exempt
from certain corporal punishments, in case they should have
incurred them. The Decurionate was the ruin of all the rich.
Hence they tried every way of avoiding this dangerous honor; they
concealed themselves, they entered into military service; but
their efforts were unavailing; they were seized, they were
compelled to become Decurions, and the dread inspired by this
title was termed \textit{Impiety}.—G. ——The Decurions were mutually
responsible; they were obliged to undertake for pieces of ground
abandoned by their owners on account of the pressure of the
taxes, and, finally, to make up all deficiencies. Savigny chichte
des Rom. Rechts, i. 25.—M.}

\pagenote[173]{Habemus enim et hominum numerum qui delati sunt,
et agrun modum. Eumenius in Panegyr. Vet. viii. 6. See Cod.
Theod. l. xiii. tit. x. xi., with Godefroy’s Commentary.}

\pagenote[174]{Siquis sacrilegâ vitem falce succiderit, aut
feracium ramorum fœtus hebetaverit, quo delinet fidem Censuum, et
mentiatur callide paupertatis ingenium, mox detectus capitale
subibit exitium, et bona ejus in Fisci jura migrabunt. Cod.
Theod. l. xiii. tit. xi. leg. 1. Although this law is not without
its studied obscurity, it is, however clear enough to prove the
minuteness of the inquisition, and the disproportion of the
penalty.}

\pagenote[175]{The astonishment of Pliny would have ceased.
Equidem miror P. R. victis gentibus argentum semper imperitasse
non aurum. Hist Natur. xxxiii. 15.}

\pagenote[17511]{The proprietors were not charged with the
expense of this transport in the provinces situated on the
sea-shore or near the great rivers, there were companies of
boatmen, and of masters of vessels, who had this commission, and
furnished the means of transport at their own expense. In return,
they were themselves exempt, altogether, or in part, from the
indiction and other imposts. They had certain privileges;
particular regulations determined their rights and obligations.
(Cod. Theod. l. xiii. tit. v. ix.) The transports by land were
made in the same manner, by the intervention of a privileged
company called Bastaga; the members were called Bastagarii Cod.
Theod. l. viii. tit. v.—G.}

\pagenote[176]{Some precautions were taken (see Cod. Theod. l.
xi. tit. ii. and Cod. Justinian. l. x. tit. xxvii. leg. 1, 2, 3)
to restrain the magistrates from the abuse of their authority,
either in the exaction or in the purchase of corn: but those who
had learning enough to read the orations of Cicero against
Verres, (iii. de Frumento,) might instruct themselves in all the
various arts of oppression, with regard to the weight, the price,
the quality, and the carriage. The avarice of an unlettered
governor would supply the ignorance of precept or precedent.}

\pagenote[177]{Cod. Theod. l. xi. tit. xxviii. leg. 2, published
the 24th of March, A. D. 395, by the emperor Honorius, only two
months after the death of his father, Theodosius. He speaks of
528,042 Roman jugera, which I have reduced to the English
measure. The jugerum contained 28,800 square Roman feet.}

Either from design or from accident, the mode of assessment
seemed to unite the substance of a land tax with the forms of a
capitation.\textsuperscript{178} The returns which were sent of every province or
district, expressed the number of tributary subjects, and the
amount of the public impositions. The latter of these sums was
divided by the former; and the estimate, that such a province
contained so many \textit{capita}, or heads of tribute; and that each
\textit{head} was rated at such a price, was universally received, not
only in the popular, but even in the legal computation. The value
of a tributary head must have varied, according to many
accidental, or at least fluctuating circumstances; but some
knowledge has been preserved of a very curious fact, the more
important, since it relates to one of the richest provinces of
the Roman empire, and which now flourishes as the most splendid
of the European kingdoms. The rapacious ministers of Constantius
had exhausted the wealth of Gaul, by exacting twenty-five pieces
of gold for the annual tribute of every head. The humane policy
of his successor reduced the capitation to seven pieces.\textsuperscript{179} A
moderate proportion between these opposite extremes of
extraordinary oppression and of transient indulgence, may
therefore be fixed at sixteen pieces of gold, or about nine
pounds sterling, the common standard, perhaps, of the impositions
of Gaul.\textsuperscript{180} But this calculation, or rather, indeed, the facts
from whence it is deduced, cannot fail of suggesting two
difficulties to a thinking mind, who will be at once surprised by
the \textit{equality}, and by the \textit{enormity}, of the capitation. An
attempt to explain them may perhaps reflect some light on the
interesting subject of the finances of the declining empire.

\pagenote[178]{Godefroy (Cod. Theod. tom. vi. p. 116) argues with
weight and learning on the subject of the capitation; but while
he explains the \textit{caput}, as a share or measure of property, he
too absolutely excludes the idea of a personal assessment.}

\pagenote[179]{Quid profuerit (\textit{Julianus}) anhelantibus extremâ
penuriâ Gallis, hinc maxime claret, quod primitus partes eas
ingressus, pro \textit{capitibus} singulis tributi nomine vicenos quinos
aureos reperit flagitari; discedens vero septenos tantum numera
universa complentes. Ammian. l. xvi. c. 5.}

\pagenote[180]{In the calculation of any sum of money under
Constantine and his successors, we need only refer to the
excellent discourse of Mr. Greaves on the Denarius, for the proof
of the following principles; 1. That the ancient and modern Roman
pound, containing 5256 grains of Troy weight, is about one
twelfth lighter than the English pound, which is composed of 5760
of the same grains. 2. That the pound of gold, which had once
been divided into forty-eight \textit{aurei}, was at this time coined
into seventy-two smaller pieces of the same denomination. 3. That
five of these aurei were the legal tender for a pound of silver,
and that consequently the pound of gold was exchanged for
fourteen pounds eight ounces of silver, according to the Roman,
or about thirteen pounds according to the English weight. 4. That
the English pound of silver is coined into sixty-two shillings.
From these elements we may compute the Roman pound of gold, the
usual method of reckoning large sums, at forty pounds sterling,
and we may fix the currency of the \textit{aureus} at somewhat more than
eleven shillings. * Note: See, likewise, a Dissertation of M.
Letronne, “Considerations Génerales sur l’Evaluation des Monnaies
Grecques et Romaines” Paris, 1817—M.}

I. It is obvious, that, as long as the immutable constitution of
human nature produces and maintains so unequal a division of
property, the most numerous part of the community would be
deprived of their subsistence, by the equal assessment of a tax
from which the sovereign would derive a very trifling revenue.
Such indeed might be the theory of the Roman capitation; but in
the practice, this unjust equality was no longer felt, as the
tribute was collected on the principle of a \textit{real}, not of a
\textit{personal} imposition.\textsuperscript{18011} Several indigent citizens
contributed to compose a single \textit{head}, or share of taxation;
while the wealthy provincial, in proportion to his fortune, alone
represented several of those imaginary beings. In a poetical
request, addressed to one of the last and most deserving of the
Roman princes who reigned in Gaul, Sidonius Apollinaris
personifies his tribute under the figure of a triple monster, the
Geryon of the Grecian fables, and entreats the new Hercules that
he would most graciously be pleased to save his life by cutting
off three of his heads.\textsuperscript{181} The fortune of Sidonius far exceeded
the customary wealth of a poet; but if he had pursued the
allusion, he might have painted many of the Gallic nobles with
the hundred heads of the deadly Hydra, spreading over the face of
the country, and devouring the substance of a hundred families.
II. The difficulty of allowing an annual sum of about nine pounds
sterling, even for the average of the capitation of Gaul, may be
rendered more evident by the comparison of the present state of
the same country, as it is now governed by the absolute monarch
of an industrious, wealthy, and affectionate people. The taxes of
France cannot be magnified, either by fear or by flattery, beyond
the annual amount of eighteen millions sterling, which ought
perhaps to be shared among four and twenty millions of
inhabitants.\textsuperscript{182} Seven millions of these, in the capacity of
fathers, or brothers, or husbands, may discharge the obligations
of the remaining multitude of women and children; yet the equal
proportion of each tributary subject will scarcely rise above
fifty shillings of our money, instead of a proportion almost four
times as considerable, which was regularly imposed on their
Gallic ancestors. The reason of this difference may be found, not
so much in the relative scarcity or plenty of gold and silver, as
in the different state of society, in ancient Gaul and in modern
France. In a country where personal freedom is the privilege of
every subject, the whole mass of taxes, whether they are levied
on property or on consumption, may be fairly divided among the
whole body of the nation. But the far greater part of the lands
of ancient Gaul, as well as of the other provinces of the Roman
world, were cultivated by slaves, or by peasants, whose dependent
condition was a less rigid servitude.\textsuperscript{183} In such a state the
poor were maintained at the expense of the masters who enjoyed
the fruits of their labor; and as the rolls of tribute were
filled only with the names of those citizens who possessed the
means of an honorable, or at least of a decent subsistence, the
comparative smallness of their numbers explains and justifies the
high rate of their capitation. The truth of this assertion may be
illustrated by the following example: The Ædui, one of the most
powerful and civilized tribes or \textit{cities} of Gaul, occupied an
extent of territory, which now contains about five hundred
thousand inhabitants, in the two ecclesiastical dioceses of Autun
and Nevers;\textsuperscript{184} and with the probable accession of those of
Châlons and Maçon,\textsuperscript{185} the population would amount to eight
hundred thousand souls. In the time of Constantine, the territory
of the Ædui afforded no more than twenty-five thousand \textit{heads} of
capitation, of whom seven thousand were discharged by that prince
from the intolerable weight of tribute.\textsuperscript{186} A just analogy would
seem to countenance the opinion of an ingenious historian,\textsuperscript{187}
that the free and tributary citizens did not surpass the number
of half a million; and if, in the ordinary administration of
government, their annual payments may be computed at about four
millions and a half of our money, it would appear, that although
the share of each individual was four times as considerable, a
fourth part only of the modern taxes of France was levied on the
Imperial province of Gaul. The exactions of Constantius may be
calculated at seven millions sterling, which were reduced to two
millions by the humanity or the wisdom of Julian.

\pagenote[18011]{Two masterly dissertations of M. Savigny, in the
Mem. of the Berlin Academy (1822 and 1823) have thrown new light
on the taxation system of the Empire. Gibbon, according to M.
Savigny, is mistaken in supposing that there was but one kind of
capitation tax; there was a land tax, and a capitation tax,
strictly so called. The land tax was, in its operation, a
proprietor’s or landlord’s tax. But, besides this, there was a
direct capitation tax on all who were not possessed of landed
property. This tax dates from the time of the Roman conquests;
its amount is not clearly known. Gradual exemptions released
different persons and classes from this tax. One edict exempts
painters. In Syria, all under twelve or fourteen, or above
sixty-five, were exempted; at a later period, all under twenty,
and all unmarried females; still later, all under twenty-five,
widows and nuns, soldiers, veterani and clerici—whole dioceses,
that of Thrace and Illyricum. Under Galerius and Licinius, the
plebs urbana became exempt; though this, perhaps, was only an
ordinance for the East. By degrees, however, the exemption was
extended to all the inhabitants of towns; and as it was strictly
capitatio plebeia, from which all possessors were exempted it
fell at length altogether on the coloni and agricultural slaves.
These were registered in the same cataster (capitastrum) with the
land tax. It was paid by the proprietor, who raised it again from
his coloni and laborers.—M.}

\pagenote[181]{Geryones nos esse puta, monstrumque tributum, Hîc \textit{capita} ut
vivam, tu mihi tolle \textit{tria}. Sidon. Apollinar. Carm. xiii.

The reputation of Father Sirmond led me to expect more
satisfaction than I have found in his note (p. 144) on this
remarkable passage. The words, suo vel \textit{suorum} nomine, betray
the perplexity of the commentator.}

\pagenote[182]{This assertion, however formidable it may seem, is
founded on the original registers of births, deaths, and
marriages, collected by public authority, and now deposited in
the \textit{Contrôlee General} at Paris. The annual average of births
throughout the whole kingdom, taken in five years, (from 1770 to
1774, both inclusive,) is 479,649 boys, and 449,269 girls, in all
928,918 children. The province of French Hainault alone furnishes
9906 births; and we are assured, by an actual enumeration of the
people, annually repeated from the year 1773 to the year 1776,
that upon an average, Hainault contains 257,097 inhabitants. By
the rules of fair analogy, we might infer, that the ordinary
proportion of annual births to the whole people, is about 1 to
26; and that the kingdom of France contains 24,151,868 persons of
both sexes and of every age. If we content ourselves with the
more moderate proportion of 1 to 25, the whole population will
amount to 23,222,950. From the diligent researches of the French
Government, (which are not unworthy of our own imitation,) we may
hope to obtain a still greater degree of certainty on this
important subject * Note: On no subject has so much valuable
information been collected since the time of Gibbon, as the
statistics of the different countries of Europe but much is still
wanting as to our own—M.}

\pagenote[183]{Cod. Theod. l. v. tit. ix. x. xi. Cod. Justinian.
l. xi. tit. lxiii. Coloni appellantur qui conditionem debent
genitali solo, propter agriculturum sub dominio possessorum.
Augustin. de Civitate Dei, l. x. c. i.}

\pagenote[184]{The ancient jurisdiction of (\textit{Augustodunum}) Autun
in Burgundy, the capital of the Ædui, comprehended the adjacent
territory of (\textit{Noviodunum}) Nevers. See D’Anville, Notice de
l’Ancienne Gaule, p. 491. The two dioceses of Autun and Nevers
are now composed, the former of 610, and the latter of 160
parishes. The registers of births, taken during eleven years, in
476 parishes of the same province of Burgundy, and multiplied by
the moderate proportion of 25, (see Messance Recherches sur la
Population, p. 142,) may authorizes us to assign an average
number of 656 persons for each parish, which being again
multiplied by the 770 parishes of the dioceses of Nevers and
Autun, will produce the sum of 505,120 persons for the extent of
country which was once possessed by the Ædui.}

\pagenote[185]{We might derive an additional supply of 301,750
inhabitants from the dioceses of Châlons (\textit{Cabillonum}) and of
Maçon, (\textit{Matisco},) since they contain, the one 200, and the
other 260 parishes. This accession of territory might be
justified by very specious reasons. 1. Châlons and Maçon were
undoubtedly within the original jurisdiction of the Ædui. (See
D’Anville, Notice, p. 187, 443.) 2. In the Notitia of Gaul, they
are enumerated not as \textit{Civitates}, but merely as \textit{Castra}. 3.
They do not appear to have been episcopal seats before the fifth
and sixth centuries. Yet there is a passage in Eumenius (Panegyr.
Vet. viii. 7) which very forcibly deters me from extending the
territory of the Ædui, in the reign of Constantine, along the
beautiful banks of the navigable Saône. * Note: In this passage
of Eumenius, Savigny supposes the original number to have been
32,000: 7000 being discharged, there remained 25,000 liable to
the tribute. See Mem. quoted above.—M.}

\pagenote[186]{Eumenius in Panegyr Vet. viii. 11.}

\pagenote[187]{L’Abbé du Bos, Hist. Critique de la M. F. tom. i.
p. 121}

But this tax, or capitation, on the proprietors of land, would
have suffered a rich and numerous class of free citizens to
escape. With the view of sharing that species of wealth which is
derived from art or labor, and which exists in money or in
merchandise, the emperors imposed a distinct and personal tribute
on the trading part of their subjects.\textsuperscript{188} Some exemptions, very
strictly confined both in time and place, were allowed to the
proprietors who disposed of the produce of their own estates.
Some indulgence was granted to the profession of the liberal
arts: but every other branch of commercial industry was affected
by the severity of the law. The honorable merchant of Alexandria,
who imported the gems and spices of India for the use of the
western world; the usurer, who derived from the interest of money
a silent and ignominious profit; the ingenious manufacturer, the
diligent mechanic, and even the most obscure retailer of a
sequestered village, were obliged to admit the officers of the
revenue into the partnership of their gain; and the sovereign of
the Roman empire, who tolerated the profession, consented to
share the infamous salary, of public prostitutes.\textsuperscript{18811} As this
general tax upon industry was collected every fourth year, it was
styled the \textit{Lustral Contribution:} and the historian Zosimus\textsuperscript{189}
laments that the approach of the fatal period was announced by
the tears and terrors of the citizens, who were often compelled
by the impending scourge to embrace the most abhorred and
unnatural methods of procuring the sum at which their property
had been assessed. The testimony of Zosimus cannot indeed be
justified from the charge of passion and prejudice; but, from the
nature of this tribute it seems reasonable to conclude, that it
was arbitrary in the distribution, and extremely rigorous in the
mode of collecting. The secret wealth of commerce, and the
precarious profits of art or labor, are susceptible only of a
discretionary valuation, which is seldom disadvantageous to the
interest of the treasury; and as the person of the trader
supplies the want of a visible and permanent security, the
payment of the imposition, which, in the case of a land tax, may
be obtained by the seizure of property, can rarely be extorted by
any other means than those of corporal punishments. The cruel
treatment of the insolvent debtors of the state, is attested, and
was perhaps mitigated by a very humane edict of Constantine, who,
disclaiming the use of racks and of scourges, allots a spacious
and airy prison for the place of their confinement.\textsuperscript{190}

\pagenote[188]{See Cod. Theod. l. xiii. tit. i. and iv.}

\pagenote[18811]{The emperor Theodosius put an end, by a law. to
this disgraceful source of revenue. (Godef. ad Cod. Theod. xiii.
tit. i. c. 1.) But before he deprived himself of it, he made sure
of some way of replacing this deficit. A rich patrician,
Florentius, indignant at this legalized licentiousness, had made
representations on the subject to the emperor. To induce him to
tolerate it no longer, he offered his own property to supply the
diminution of the revenue. The emperor had the baseness to accept
his offer—G.}

\pagenote[189]{Zosimus, l. ii. p. 115. There is probably as much
passion and prejudice in the attack of Zosimus, as in the
elaborate defence of the memory of Constantine by the zealous Dr.
Howell. Hist. of the World, vol. ii. p. 20.}

\pagenote[190]{Cod. Theod. l. xi. tit vii. leg. 3.}

These general taxes were imposed and levied by the absolute
authority of the monarch; but the occasional offerings of the
\textit{coronary gold} still retained the name and semblance of popular
consent. It was an ancient custom that the allies of the
republic, who ascribed their safety or deliverance to the success
of the Roman arms, and even the cities of Italy, who admired the
virtues of their victorious general, adorned the pomp of his
triumph by their voluntary gifts of crowns of gold, which after
the ceremony were consecrated in the temple of Jupiter, to remain
a lasting monument of his glory to future ages. The progress of
zeal and flattery soon multiplied the number, and increased the
size, of these popular donations; and the triumph of Cæsar was
enriched with two thousand eight hundred and twenty-two massy
crowns, whose weight amounted to twenty thousand four hundred and
fourteen pounds of gold. This treasure was immediately melted
down by the prudent dictator, who was satisfied that it would be
more serviceable to his soldiers than to the gods: his example
was imitated by his successors; and the custom was introduced of
exchanging these splendid ornaments for the more acceptable
present of the current gold coin of the empire.\textsuperscript{191} The
spontaneous offering was at length exacted as the debt of duty;
and instead of being confined to the occasion of a triumph, it
was supposed to be granted by the several cities and provinces of
the monarchy, as often as the emperor condescended to announce
his accession, his consulship, the birth of a son, the creation
of a Cæsar, a victory over the Barbarians, or any other real or
imaginary event which graced the annals of his reign. The
peculiar free gift of the senate of Rome was fixed by custom at
sixteen hundred pounds of gold, or about sixty-four thousand
pounds sterling. The oppressed subjects celebrated their own
felicity, that their sovereign should graciously consent to
accept this feeble but voluntary testimony of their loyalty and
gratitude.\textsuperscript{192}

\pagenote[191]{See Lipsius de Magnitud. Romana, l. ii. c. 9. The
Tarragonese Spain presented the emperor Claudius with a crown of
gold of seven, and Gaul with another of nine, \textit{hundred} pounds
weight. I have followed the rational emendation of Lipsius. *
Note: This custom is of still earlier date, the Romans had
borrowed it from Greece. Who is not acquainted with the famous
oration of Demosthenes for the golden crown, which his citizens
wished to bestow, and Æschines to deprive him of?—G.}

\pagenote[192]{Cod. Theod. l. xii. tit. xiii. The senators were
supposed to be exempt from the \textit{Aurum Coronarium;} but the \textit{Auri
Oblatio}, which was required at their hands, was precisely of the
same nature.}

A people elated by pride, or soured by discontent, are seldom
qualified to form a just estimate of their actual situation. The
subjects of Constantine were incapable of discerning the decline
of genius and manly virtue, which so far degraded them below the
dignity of their ancestors; but they could feel and lament the
rage of tyranny, the relaxation of discipline, and the increase
of taxes. The impartial historian, who acknowledges the justice
of their complaints, will observe some favorable circumstances
which tended to alleviate the misery of their condition. The
threatening tempest of Barbarians, which so soon subverted the
foundations of Roman greatness, was still repelled, or suspended,
on the frontiers. The arts of luxury and literature were
cultivated, and the elegant pleasures of society were enjoyed, by
the inhabitants of a considerable portion of the globe. The
forms, the pomp, and the expense of the civil administration
contributed to restrain the irregular license of the soldiers;
and although the laws were violated by power, or perverted by
subtlety, the sage principles of the Roman jurisprudence
preserved a sense of order and equity, unknown to the despotic
governments of the East. The rights of mankind might derive some
protection from religion and philosophy; and the name of freedom,
which could no longer alarm, might sometimes admonish, the
successors of Augustus, that they did not reign over a nation of
Slaves or Barbarians.\textsuperscript{193}

\pagenote[193]{The great Theodosius, in his judicious advice to
his son, (Claudian in iv. Consulat. Honorii, 214, \&c.,)
distinguishes the station of a Roman prince from that of a
Parthian monarch. Virtue was necessary for the one; birth might
suffice for the other.}

