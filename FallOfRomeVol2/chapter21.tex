\chapter{Persecution Of Heresy, State Of The Church.}
\section{Part \thesection.}

\textit{Persecution Of Heresy. — The Schism Of The Donatists. — The Arian
Controversy. — Athanasius. — Distracted State Of The Church And Empire
Under Constantine And His Sons. — Toleration Of Paganism.}
\vspace{\onelineskip}

The grateful applause of the clergy has consecrated the memory of
a prince who indulged their passions and promoted their interest.
Constantine gave them security, wealth, honors, and revenge; and
the support of the orthodox faith was considered as the most
sacred and important duty of the civil magistrate. The edict of
Milan, the great charter of toleration, had confirmed to each
individual of the Roman world the privilege of choosing and
professing his own religion. But this inestimable privilege was
soon violated; with the knowledge of truth, the emperor imbibed
the maxims of persecution; and the sects which dissented from the
Catholic church were afflicted and oppressed by the triumph of
Christianity. Constantine easily believed that the Heretics, who
presumed to dispute \textit{his} opinions, or to oppose \textit{his} commands,
were guilty of the most absurd and criminal obstinacy; and that a
seasonable application of moderate severities might save those
unhappy men from the danger of an everlasting condemnation. Not a
moment was lost in excluding the ministers and teachers of the
separated congregations from any share of the rewards and
immunities which the emperor had so liberally bestowed on the
orthodox clergy. But as the sectaries might still exist under the
cloud of royal disgrace, the conquest of the East was immediately
followed by an edict which announced their total destruction.\textsuperscript{1}
After a preamble filled with passion and reproach, Constantine
absolutely prohibits the assemblies of the Heretics, and
confiscates their public property to the use either of the
revenue or of the Catholic church. The sects against whom the
Imperial severity was directed, appear to have been the adherents
of Paul of Samosata; the Montanists of Phrygia, who maintained an
enthusiastic succession of prophecy; the Novatians, who sternly
rejected the temporal efficacy of repentance; the Marcionites and
Valentinians, under whose leading banners the various Gnostics of
Asia and Egypt had insensibly rallied; and perhaps the
Manichæans, who had recently imported from Persia a more artful
composition of Oriental and Christian theology.\textsuperscript{2} The design of
extirpating the name, or at least of restraining the progress, of
these odious Heretics, was prosecuted with vigor and effect. Some
of the penal regulations were copied from the edicts of
Diocletian; and this method of conversion was applauded by the
same bishops who had felt the hand of oppression, and pleaded for
the rights of humanity. Two immaterial circumstances may serve,
however, to prove that the mind of Constantine was not entirely
corrupted by the spirit of zeal and bigotry. Before he condemned
the Manichæans and their kindred sects, he resolved to make an
accurate inquiry into the nature of their religious principles.
As if he distrusted the impartiality of his ecclesiastical
counsellors, this delicate commission was intrusted to a civil
magistrate, whose learning and moderation he justly esteemed, and
of whose venal character he was probably ignorant.\textsuperscript{3} The emperor
was soon convinced, that he had too hastily proscribed the
orthodox faith and the exemplary morals of the Novatians, who had
dissented from the church in some articles of discipline which
were not perhaps essential to salvation. By a particular edict,
he exempted them from the general penalties of the law;\textsuperscript{4} allowed
them to build a church at Constantinople, respected the miracles
of their saints, invited their bishop Acesius to the council of
Nice; and gently ridiculed the narrow tenets of his sect by a
familiar jest; which, from the mouth of a sovereign, must have
been received with applause and gratitude.\textsuperscript{5}

\pagenote[1]{Eusebius in Vit. Constantin. l. iii. c. 63, 64, 65, 66.}

\pagenote[2]{After some examination of the various opinions of
Tillemont, Beausobre, Lardner, \&c., I am convinced that Manes did
not propagate his sect, even in Persia, before the year 270. It
is strange, that a philosophic and foreign heresy should have
penetrated so rapidly into the African provinces; yet I cannot
easily reject the edict of Diocletian against the Manichæans,
which may be found in Baronius. (Annal Eccl. A. D. 287.)}

\pagenote[3]{Constantinus enim, cum limatius superstitionum
quæroret sectas, Manichæorum et similium, \&c. Ammian. xv. 15.
Strategius, who from this commission obtained the surname of
\textit{Musonianus}, was a Christian of the Arian sect. He acted as one
of the counts at the council of Sardica. Libanius praises his
mildness and prudence. Vales. ad locum Ammian.}

\pagenote[4]{Cod. Theod. l. xvi. tit. 5, leg. 2. As the general
law is not inserted in the Theodosian Code, it probable that, in
the year 438, the sects which it had condemned were already
extinct.}

\pagenote[5]{Sozomen, l. i. c. 22. Socrates, l. i. c. 10. These
historians have been suspected, but I think without reason, of an
attachment to the Novatian doctrine. The emperor said to the
bishop, “Acesius, take a ladder, and get up to heaven by
yourself.” Most of the Christian sects have, by turns, borrowed
the ladder of Acesius.}

The complaints and mutual accusations which assailed the throne
of Constantine, as soon as the death of Maxentius had submitted
Africa to his victorious arms, were ill adapted to edify an
imperfect proselyte. He learned, with surprise, that the
provinces of that great country, from the confines of Cyrene to
the columns of Hercules, were distracted with religious discord.\textsuperscript{6}
The source of the division was derived from a double election
in the church of Carthage; the second, in rank and opulence, of
the ecclesiastical thrones of the West. Cæcilian and Majorinus
were the two rival prelates of Africa; and the death of the
latter soon made room for Donatus, who, by his superior abilities
and apparent virtues, was the firmest support of his party. The
advantage which Cæcilian might claim from the priority of his
ordination, was destroyed by the illegal, or at least indecent,
haste, with which it had been performed, without expecting the
arrival of the bishops of Numidia. The authority of these
bishops, who, to the number of seventy, condemned Cæcilian, and
consecrated Majorinus, is again weakened by the infamy of some of
their personal characters; and by the female intrigues,
sacrilegious bargains, and tumultuous proceedings, which are
imputed to this Numidian council.\textsuperscript{7} The bishops of the contending
factions maintained, with equal ardor and obstinacy, that their
adversaries were degraded, or at least dishonored, by the odious
crime of delivering the Holy Scriptures to the officers of
Diocletian. From their mutual reproaches, as well as from the
story of this dark transaction, it may justly be inferred, that
the late persecution had imbittered the zeal, without reforming
the manners, of the African Christians. That divided church was
incapable of affording an impartial judicature; the controversy
was solemnly tried in five successive tribunals, which were
appointed by the emperor; and the whole proceeding, from the
first appeal to the final sentence, lasted above three years. A
severe inquisition, which was taken by the Prætorian vicar, and
the proconsul of Africa, the report of two episcopal visitors who
had been sent to Carthage, the decrees of the councils of Rome
and of Arles, and the supreme judgment of Constantine himself in
his sacred consistory, were all favorable to the cause of
Cæcilian; and he was unanimously acknowledged by the civil and
ecclesiastical powers, as the true and lawful primate of Africa.
The honors and estates of the church were attributed to \textit{his}
suffragan bishops, and it was not without difficulty, that
Constantine was satisfied with inflicting the punishment of exile
on the principal leaders of the Donatist faction. As their cause
was examined with attention, perhaps it was determined with
justice. Perhaps their complaint was not without foundation, that
the credulity of the emperor had been abused by the insidious
arts of his favorite Osius. The influence of falsehood and
corruption might procure the condemnation of the innocent, or
aggravate the sentence of the guilty. Such an act, however, of
injustice, if it concluded an importunate dispute, might be
numbered among the transient evils of a despotic administration,
which are neither felt nor remembered by posterity.

\pagenote[6]{The best materials for this part of ecclesiastical
history may be found in the edition of Optatus Milevitanus,
published (Paris, 1700) by M. Dupin, who has enriched it with
critical notes, geographical discussions, original records, and
an accurate abridgment of the whole controversy. M. de Tillemont
has bestowed on the Donatists the greatest part of a volume,
(tom. vi. part i.;) and I am indebted to him for an ample
collection of all the passages of his favorite St. Augustin,
which relate to those heretics.}

\pagenote[7]{Schisma igitur illo tempore confusæ mulieris
iracundia peperit; ambitus nutrivit; avaritia roboravit. Optatus,
l. i. c. 19. The language of Purpurius is that of a furious
madman. Dicitur te necasse lilios sororis tuæ duos. Purpurius
respondit: Putas me terreri a te.. occidi; et occido eos qui
contra me faciunt. Acta Concil. Cirtenais, ad calc. Optat. p.
274. When Cæcilian was invited to an assembly of bishops,
Purpurius said to his brethren, or rather to his accomplices,
“Let him come hither to receive our imposition of hands, and we
will break his head by way of penance.” Optat. l. i. c. 19.}

But this incident, so inconsiderable that it scarcely deserves a
place in history, was productive of a memorable schism which
afflicted the provinces of Africa above three hundred years, and
was extinguished only with Christianity itself. The inflexible
zeal of freedom and fanaticism animated the Donatists to refuse
obedience to the usurpers, whose election they disputed, and
whose spiritual powers they denied. Excluded from the civil and
religious communion of mankind, they boldly excommunicated the
rest of mankind, who had embraced the impious party of Cæcilian,
and of the Traditors, from which he derived his pretended
ordination. They asserted with confidence, and almost with
exultation, that the Apostolical succession was interrupted; that
\textit{all} the bishops of Europe and Asia were infected by the
contagion of guilt and schism; and that the prerogatives of the
Catholic church were confined to the chosen portion of the
African believers, who alone had preserved inviolate the
integrity of their faith and discipline. This rigid theory was
supported by the most uncharitable conduct. Whenever they
acquired a proselyte, even from the distant provinces of the
East, they carefully repeated the sacred rites of baptism\textsuperscript{8} and
ordination; as they rejected the validity of those which he had
already received from the hands of heretics or schismatics.
Bishops, virgins, and even spotless infants, were subjected to
the disgrace of a public penance, before they could be admitted
to the communion of the Donatists. If they obtained possession of
a church which had been used by their Catholic adversaries, they
purified the unhallowed building with the same zealous care which
a temple of idols might have required. They washed the pavement,
scraped the walls, burnt the altar, which was commonly of wood,
melted the consecrated plate, and cast the Holy Eucharist to the
dogs, with every circumstance of ignominy which could provoke and
perpetuate the animosity of religious factions.\textsuperscript{9} Notwithstanding
this irreconcilable aversion, the two parties, who were mixed and
separated in all the cities of Africa, had the same language and
manners, the same zeal and learning, the same faith and worship.
Proscribed by the civil and ecclesiastical powers of the empire,
the Donatists still maintained in some provinces, particularly in
Numidia, their superior numbers; and four hundred bishops
acknowledged the jurisdiction of their primate. But the
invincible spirit of the sect sometimes preyed on its own vitals:
and the bosom of their schismatical church was torn by intestine
divisions. A fourth part of the Donatist bishops followed the
independent standard of the Maximianists. The narrow and solitary
path which their first leaders had marked out, continued to
deviate from the great society of mankind. Even the imperceptible
sect of the Rogatians could affirm, without a blush, that when
Christ should descend to judge the earth, he would find his true
religion preserved only in a few nameless villages of the
Cæsarean Mauritania.\textsuperscript{10}

\pagenote[8]{The councils of Arles, of Nice, and of Trent,
confirmed the wise and moderate practice of the church of Rome.
The Donatists, however, had the advantage of maintaining the
sentiment of Cyprian, and of a considerable part of the primitive
church. Vincentius Lirinesis (p. 532, ap. Tillemont, Mém. Eccles.
tom. vi. p. 138) has explained why the Donatists are eternally
burning with the Devil, while St. Cyprian reigns in heaven with
Jesus Christ.}

\pagenote[9]{See the sixth book of Optatus Milevitanus, p. 91-100.}

\pagenote[10]{Tillemont, Mém. Ecclésiastiques, tom. vi. part i.
p. 253. He laughs at their partial credulity. He revered
Augustin, the great doctor of the system of predestination.}

The schism of the Donatists was confined to Africa: the more
diffusive mischief of the Trinitarian controversy successively
penetrated into every part of the Christian world. The former was
an accidental quarrel, occasioned by the abuse of freedom; the
latter was a high and mysterious argument, derived from the abuse
of philosophy. From the age of Constantine to that of Clovis and
Theodoric, the temporal interests both of the Romans and
Barbarians were deeply involved in the theological disputes of
Arianism. The historian may therefore be permitted respectfully
to withdraw the veil of the sanctuary; and to deduce the progress
of reason and faith, of error and passion from the school of
Plato, to the decline and fall of the empire.

The genius of Plato, informed by his own meditation, or by the
traditional knowledge of the priests of Egypt,\textsuperscript{11} had ventured to
explore the mysterious nature of the Deity. When he had elevated
his mind to the sublime contemplation of the first self-existent,
necessary cause of the universe, the Athenian sage was incapable
of conceiving \textit{how} the simple unity of his essence could admit
the infinite variety of distinct and successive ideas which
compose the model of the intellectual world; \textit{how} a Being purely
incorporeal could execute that perfect model, and mould with a
plastic hand the rude and independent chaos. The vain hope of
extricating himself from these difficulties, which must ever
oppress the feeble powers of the human mind, might induce Plato
to consider the divine nature under the threefold modification—of
the first cause, the reason, or \textit{Logos}, and the soul or spirit
of the universe. His poetical imagination sometimes fixed and
animated these metaphysical abstractions; the three \textit{archical} on
original principles were represented in the Platonic system as
three Gods, united with each other by a mysterious and ineffable
generation; and the Logos was particularly considered under the
more accessible character of the Son of an Eternal Father, and
the Creator and Governor of the world. Such appear to have been
the secret doctrines which were cautiously whispered in the
gardens of the academy; and which, according to the more recent
disciples of Plato,\textsuperscript{1111} could not be perfectly understood, till
after an assiduous study of thirty years.\textsuperscript{12}

\pagenote[11]{Plato Ægyptum peragravit ut a sacerdotibus Barbaris
numeros et \textit{cælestia} acciperet. Cicero de Finibus, v. 25. The
Egyptians might still preserve the traditional creed of the
Patriarchs. Josephus has persuaded many of the Christian fathers,
that Plato derived a part of his knowledge from the Jews; but
this vain opinion cannot be reconciled with the obscure state and
unsocial manners of the Jewish people, whose scriptures were not
accessible to Greek curiosity till more than one hundred years
after the death of Plato. See Marsham Canon. Chron. p. 144 Le
Clerc, Epistol. Critic. vii. p. 177-194.}

\pagenote[1111]{This exposition of the doctrine of Plato appears
to me contrary to the true sense of that philosopher’s writings.
The brilliant imagination which he carried into metaphysical
inquiries, his style, full of allegories and figures, have misled
those interpreters who did not seek, from the whole tenor of his
works and beyond the images which the writer employs, the system
of this philosopher. In my opinion, there is no Trinity in Plato;
he has established no mysterious generation between the three
pretended principles which he is made to distinguish. Finally, he
conceives only as \textit{attributes} of the Deity, or of matter, those
ideas, of which it is supposed that he made \textit{substances}, real
beings.
According to Plato, God and matter existed from all eternity.
Before the creation of the world, matter had in itself a
principle of motion, but without end or laws: it is this
principle which Plato calls the irrational soul of the world,
because, according to his doctrine, every spontaneous and
original principle of motion is called soul. God wished to
impress \textit{form} upon matter, that is to say, 1. To mould
matter, and make it into a body; 2. To regulate its motion,
and subject it to some end and to certain laws. The Deity, in
this operation, could not act but according to the ideas
existing in his intelligence: their union filled this, and
formed the ideal type of the world. It is this ideal world,
this divine intelligence, existing with God from all
eternity, and called by Plato which he is supposed to
personify, to substantialize; while an attentive examination
is sufficient to convince us that he has never assigned it an
existence external to the Deity, (hors de la Divinité,) and
that he considered the as the aggregate of the ideas of God,
the divine understanding in its relation to the world. The
contrary opinion is irreconcilable with all his philosophy:
thus he says (Timæus, p. 348, edit. Bip.) that to the idea of
the Deity is essentially united that of intelligence, of a
\textit{logos}. He would thus have admitted a double \textit{logos;} one
inherent in the Deity as an attribute, the other
independently existing as a substance. He affirms that the
intelligence, the principle of order cannot exist but as an
attribute of a soul, the principle of motion and of life, of
which the nature is unknown to us. How, then, according to
this, could he consider the \textit{logos} as a substance endowed
with an independent existence? In other places, he explains
it by these two words, knowledge, science, and intelligence
which signify the attributes of the Deity. When Plato
separates God, the ideal archetype of the world and matter,
it is to explain how, according to his system, God has
proceeded, at the creation, to unite the principle of order
which he had within himself, his proper intelligence, the
principle of motion, to the principle of motion, the
irrational soul which was in matter. When he speaks of the
place occupied by the ideal world, it is to designate the
divine intelligence, which is its cause. Finally, in no part
of his writings do we find a true personification of the
pretended beings of which he is said to have formed a
trinity: and if this personification existed, it would
equally apply to many other notions, of which might be formed
many different trinities.
This error, into which many ancient as well as modern
interpreters of Plato have fallen, was very natural. Besides
the snares which were concealed in his figurative style;
besides the necessity of comprehending as a whole the system
of his ideas, and not to explain isolated passages, the
nature of his doctrine itself would conduce to this error.
When Plato appeared, the uncertainty of human knowledge, and
the continual illusions of the senses, were acknowledged, and
had given rise to a general scepticism. Socrates had aimed at
raising morality above the influence of this scepticism:
Plato endeavored to save metaphysics, by seeking in the human
intellect a source of certainty which the senses could not
furnish. He invented the system of innate ideas, of which the
aggregate formed, according to him, the ideal world, and
affirmed that these ideas were real attributes, not only
attached to our conceptions of objects, but to the nature of
the objects themselves; a nature of which from them we might
obtain a knowledge. He gave, then, to these ideas a positive
existence as attributes; his commentators could easily give
them a real existence as substances; especially as the terms
which he used to designate them, essential beauty, essential
goodness, lent themselves to this substantialization,
(hypostasis.)—G.
We have retained this view of the original philosophy of
Plato, in which there is probably much truth. The genius of
Plato was rather metaphysical than impersonative: his poetry
was in his language, rather than, like that of the Orientals,
in his conceptions.—M.}

\pagenote[12]{The modern guides who lead me to the knowledge of
the Platonic system are Cudworth, Basnage, Le Clerc, and Brucker.
As the learning of these writers was equal, and their intention
different, an inquisitive observer may derive instruction from
their disputes, and certainty from their agreement.}

The arms of the Macedonians diffused over Asia and Egypt the
language and learning of Greece; and the theological system of
Plato was taught, with less reserve, and perhaps with some
improvements, in the celebrated school of Alexandria.\textsuperscript{13} A
numerous colony of Jews had been invited, by the favor of the
Ptolemies, to settle in their new capital.\textsuperscript{14} While the bulk of
the nation practised the legal ceremonies, and pursued the
lucrative occupations of commerce, a few Hebrews, of a more
liberal spirit, devoted their lives to religious and
philosophical contemplation.\textsuperscript{15} They cultivated with diligence,
and embraced with ardor, the theological system of the Athenian
sage. But their national pride would have been mortified by a
fair confession of their former poverty: and they boldly marked,
as the sacred inheritance of their ancestors, the gold and jewels
which they had so lately stolen from their Egyptian masters. One
hundred years before the birth of Christ, a philosophical
treatise, which manifestly betrays the style and sentiments of
the school of Plato, was produced by the Alexandrian Jews, and
unanimously received as a genuine and valuable relic of the
inspired Wisdom of Solomon.\textsuperscript{16} A similar union of the Mosaic
faith and the Grecian philosophy, distinguishes the works of
Philo, which were composed, for the most part, under the reign of
Augustus.\textsuperscript{17} The material soul of the universe\textsuperscript{18} might offend
the piety of the Hebrews: but they applied the character of the
Logos to the Jehovah of Moses and the patriarchs; and the Son of
God was introduced upon earth under a visible, and even human
appearance, to perform those familiar offices which seem
incompatible with the nature and attributes of the Universal
Cause.\textsuperscript{19}

\pagenote[13]{Brucker, Hist. Philosoph. tom. i. p. 1349-1357. The
Alexandrian school is celebrated by Strabo (l. xvii.) and
Ammianus, (xxii. 6.) Note: The philosophy of Plato was not the
only source of that professed in the school of Alexandria. That
city, in which Greek, Jewish, and Egyptian men of letters were
assembled, was the scene of a strange fusion of the system of
these three people. The Greeks brought a Platonism, already much
changed; the Jews, who had acquired at Babylon a great number of
Oriental notions, and whose theological opinions had undergone
great changes by this intercourse, endeavored to reconcile
Platonism with their new doctrine, and disfigured it entirely:
lastly, the Egyptians, who were not willing to abandon notions
for which the Greeks themselves entertained respect, endeavored
on their side to reconcile their own with those of their
neighbors. It is in Ecclesiasticus and the Wisdom of Solomon that
we trace the influence of Oriental philosophy rather than that of
Platonism. We find in these books, and in those of the later
prophets, as in Ezekiel, notions unknown to the Jews before the
Babylonian captivity, of which we do not discover the germ in
Plato, but which are manifestly derived from the Orientals. Thus
God represented under the image of light, and the principle of
evil under that of darkness; the history of the good and bad
angels; paradise and hell, \&c., are doctrines of which the
origin, or at least the positive determination, can only be
referred to the Oriental philosophy. Plato supposed matter
eternal; the Orientals and the Jews considered it as a creation
of God, who alone was eternal. It is impossible to explain the
philosophy of the Alexandrian school solely by the blending of
the Jewish theology with the Greek philosophy. The Oriental
philosophy, however little it may be known, is recognized at
every instant. Thus, according to the Zend Avesta, it is by the
Word (honover) more ancient than the world, that Ormuzd created
the universe. This word is the logos of Philo, consequently very
different from that of Plato. I have shown that Plato never
personified the logos as the ideal archetype of the world: Philo
ventured this personification. The Deity, according to him, has a
double logos; the first is the ideal archetype of the world, the
ideal world, the \textit{first-born} of the Deity; the second is the
word itself of God, personified under the image of a being acting
to create the sensible world, and to make it like to the ideal
world: it is the second-born of God. Following out his
imaginations, Philo went so far as to personify anew the ideal
world, under the image of a celestial man, the primitive type of
man, and the sensible world under the image of another man less
perfect than the celestial man. Certain notions of the Oriental
philosophy may have given rise to this strange abuse of allegory,
which it is sufficient to relate, to show what alterations
Platonism had already undergone, and what was their source.
Philo, moreover, of all the Jews of Alexandria, is the one whose
Platonism is the most pure. It is from this mixture of
Orientalism, Platonism, and Judaism, that Gnosticism arose, which
had produced so many theological and philosophical
extravagancies, and in which Oriental notions evidently
predominate.—G.}

\pagenote[14]{Joseph. Antiquitat, l. xii. c. 1, 3. Basnage, Hist.
des Juifs, l. vii. c. 7.}

\pagenote[15]{For the origin of the Jewish philosophy, see
Eusebius, Præparat. Evangel. viii. 9, 10. According to Philo, the
Therapeutæ studied philosophy; and Brucker has proved (Hist.
Philosoph. tom. ii. p. 787) that they gave the preference to that
of Plato.}

\pagenote[16]{See Calmet, Dissertations sur la Bible, tom. ii. p.
277. The book of the Wisdom of Solomon was received by many of
the fathers as the work of that monarch: and although rejected by
the Protestants for want of a Hebrew original, it has obtained,
with the rest of the Vulgate, the sanction of the council of
Trent.}

\pagenote[17]{The Platonism of Philo, which was famous to a
proverb, is proved beyond a doubt by Le Clerc, (Epist. Crit.
viii. p. 211-228.) Basnage (Hist. des Juifs, l. iv. c. 5) has
clearly ascertained, that the theological works of Philo were
composed before the death, and most probably before the birth, of
Christ. In such a time of darkness, the knowledge of Philo is
more astonishing than his errors. Bull, Defens. Fid. Nicen. s. i.
c. i. p. 12.}

\pagenote[18]{Mens agitat molem, et magno se corpore \textit{miscet}.
Besides this material soul, Cudworth has discovered (p. 562) in
Amelius, Porphyry, Plotinus, and, as he thinks, in Plato himself,
a superior, spiritual \textit{upercosmian} soul of the universe. But
this double soul is exploded by Brucker, Basnage, and Le Clerc,
as an idle fancy of the latter Platonists.}

\pagenote[19]{Petav. Dogmata Theologica, tom. ii. l. viii. c. 2,
p. 791. Bull, Defens. Fid. Nicen. s. i. c. l. p. 8, 13. This
notion, till it was abused by the Arians, was freely adopted in
the Christian theology. Tertullian (adv. Praxeam, c. 16) has a
remarkable and dangerous passage. After contrasting, with
indiscreet wit, the nature of God, and the actions of Jehovah, he
concludes: Scilicet ut hæc de filio Dei non credenda fuisse, si
non scripta essent; fortasse non credenda de l’atre licet
scripta. * Note: Tertullian is here arguing against the
Patripassians; those who asserted that the Father was born of the
Virgin, died and was buried.—M.}

\section{Part \thesection.}

The eloquence of Plato, the name of Solomon, the authority of the
school of Alexandria, and the consent of the Jews and Greeks,
were insufficient to establish the truth of a mysterious
doctrine, which might please, but could not satisfy, a rational
mind. A prophet, or apostle, inspired by the Deity, can alone
exercise a lawful dominion over the faith of mankind: and the
theology of Plato might have been forever confounded with the
philosophical visions of the Academy, the Porch, and the Lycæum,
if the name and divine attributes of the \textit{Logos} had not been
confirmed by the celestial pen of the last and most sublime of
the Evangelists.\textsuperscript{20} The Christian Revelation, which was
consummated under the reign of Nerva, disclosed to the world the
amazing secret, that the Logos, who was with God from the
beginning, and was God, who had made all things, and for whom all
things had been made, was incarnate in the person of Jesus of
Nazareth; who had been born of a virgin, and suffered death on
the cross. Besides the general design of fixing on a perpetual
basis the divine honors of Christ, the most ancient and
respectable of the ecclesiastical writers have ascribed to the
evangelic theologian a particular intention to confute two
opposite heresies, which disturbed the peace of the primitive
church.\textsuperscript{21} I. The faith of the Ebionites,\textsuperscript{22} perhaps of the
Nazarenes,\textsuperscript{23} was gross and imperfect. They revered Jesus as the
greatest of the prophets, endowed with supernatural virtue and
power. They ascribed to his person and to his future reign all
the predictions of the Hebrew oracles which relate to the
spiritual and everlasting kingdom of the promised Messiah.\textsuperscript{24}
Some of them might confess that he was born of a virgin; but they
obstinately rejected the preceding existence and divine
perfections of the \textit{Logos}, or Son of God, which are so clearly
defined in the Gospel of St. John. About fifty years afterwards,
the Ebionites, whose errors are mentioned by Justin Martyr with
less severity than they seem to deserve,\textsuperscript{25} formed a very
inconsiderable portion of the Christian name. II. The Gnostics,
who were distinguished by the epithet of \textit{Docetes}, deviated into
the contrary extreme; and betrayed the human, while they asserted
the divine, nature of Christ. Educated in the school of Plato,
accustomed to the sublime idea of the Logos, they readily
conceived that the brightest \textit{Æon}, or \textit{Emanation} of the Deity,
might assume the outward shape and visible appearances of a
mortal;\textsuperscript{26} but they vainly pretended, that the imperfections of
matter are incompatible with the purity of a celestial substance.

While the blood of Christ yet smoked on Mount Calvary, the
Docetes invented the impious and extravagant hypothesis, that,
instead of issuing from the womb of the Virgin,\textsuperscript{27} he had
descended on the banks of the Jordan in the form of perfect
manhood; that he had imposed on the senses of his enemies, and of
his disciples; and that the ministers of Pilate had wasted their
impotent rage on an ury phantom, who \textit{seemed} to expire on the
cross, and, after three days, to rise from the dead.\textsuperscript{28}

\pagenote[20]{The Platonists admired the beginning of the Gospel
of St. John as containing an exact transcript of their own
principles. Augustin de Civitat. Dei, x. 29. Amelius apud Cyril.
advers. Julian. l. viii. p. 283. But in the third and fourth
centuries, the Platonists of Alexandria might improve their
Trinity by the secret study of the Christian theology. Note: A
short discussion on the sense in which St. John has used the word
Logos, will prove that he has not borrowed it from the philosophy
of Plato. The evangelist adopts this word without previous
explanation, as a term with which his contemporaries were already
familiar, and which they could at once comprehend. To know the
sense which he gave to it, we must inquire that which it
generally bore in his time. We find two: the one attached to the
word \textit{logos} by the Jews of Palestine, the other by the school of
Alexandria, particularly by Philo. The Jews had feared at all
times to pronounce the name of Jehovah; they had formed a habit
of designating God by one of his attributes; they called him
sometimes Wisdom, sometimes the Word. \textit{By the word of the Lord
were the heavens made}. (Psalm xxxiii. 6.) Accustomed to
allegories, they often addressed themselves to this attribute of
the Deity as a real being. Solomon makes Wisdom say “The Lord
possessed me in the beginning of his way, before his works of
old. I was set up from everlasting, from the beginning, or ever
the earth was.” (Prov. viii. 22, 23.) Their residence in Persia
only increased this inclination to sustained allegories. In the
Ecclesiasticus of the son of Sirach, and the Book of Wisdom, we
find allegorical descriptions of Wisdom like the following: “I
came out of the mouth of the Most High; I covered the earth as a
cloud;... I alone compassed the circuit of heaven, and walked in
the bottom of the deep... The Creator created me from the
beginning, before the world, and I shall never fail.” (Eccles.
xxiv. 35- 39.) See also the Wisdom of Solomon, c. vii. v. 9. [The
latter book is clearly Alexandrian.—M.] We see from this that the
Jews understood from the Hebrew and Chaldaic words which signify
Wisdom, the Word, and which were translated into Greek, a simple
attribute of the Deity, allegorically personified, but of which
they did not make a real particular being separate from the
Deity.
The school of Alexandria, on the contrary, and Philo among
the rest, mingling Greek with Jewish and Oriental notions,
and abandoning himself to his inclination to mysticism,
personified the logos, and represented it a distinct being,
created by God, and intermediate between God and man. This is
the second \textit{logos} of Philo, that which acts from the
beginning of the world, alone in its kind, creator of the
sensible world, formed by God according to the ideal world
which he had in himself, and which was the first logos, the
first- born of the Deity. The logos taken in this sense,
then, was a created being, but, anterior to the creation of
the world, near to God, and charged with his revelations to
mankind.
Which of these two senses is that which St. John intended to
assign to the word logos in the first chapter of his Gospel,
and in all his writings? St. John was a Jew, born and
educated in Palestine; he had no knowledge, at least very
little, of the philosophy of the Greeks, and that of the
Grecizing Jews: he would naturally, then, attach to the word
\textit{logos} the sense attached to it by the Jews of Palestine.
If, in fact, we compare the attributes which he assigns to
the \textit{logos} with those which are assigned to it in Proverbs,
in the Wisdom of Solomon, in Ecclesiasticus, we shall see
that they are the same. The Word was in the world, and the
world was made by him; in him was life, and the life was the
light of men, (c. i. v. 10-14.) It is impossible not to trace
in this chapter the ideas which the Jews had formed of the
allegorized logos. The evangelist afterwards really
personifies that which his predecessors have personified only
poetically; for he affirms “\textit{that the Word became flesh},”
(v. 14.) It was to prove this that he wrote. Closely
examined, the ideas which he gives of the logos cannot agree
with those of Philo and the school of Alexandria; they
correspond, on the contrary, with those of the Jews of
Palestine. Perhaps St. John, employing a well-known term to
explain a doctrine which was yet unknown, has slightly
altered the sense; it is this alteration which we appear to
discover on comparing different passages of his writings.
It is worthy of remark, that the Jews of Palestine, who did
not perceive this alteration, could find nothing
extraordinary in what St. John said of the Logos; at least
they comprehended it without difficulty, while the Greeks and
Grecizing Jews, on their part, brought to it prejudices and
preconceptions easily reconciled with those of the
evangelist, who did not expressly contradict them. This
circumstance must have much favored the progress of
Christianity. Thus the fathers of the church in the two first
centuries and later, formed almost all in the school of
Alexandria, gave to the Logos of St. John a sense nearly
similar to that which it received from Philo. Their doctrine
approached very near to that which in the fourth century the
council of Nice condemned in the person of Arius.—G.
M. Guizot has forgotten the long residence of St. John at
Ephesus, the centre of the mingling opinions of the East and
West, which were gradually growing up into Gnosticism. (See
Matter. Hist. du Gnosticisme, vol. i. p. 154.) St. John’s
sense of the Logos seems as far removed from the simple
allegory ascribed to the Palestinian Jews as from the
Oriental impersonation of the Alexandrian. The simple truth
may be that St. John took the familiar term, and, as it were
infused into it the peculiar and Christian sense in which it
is used in his writings.—M.}

\pagenote[21]{See Beausobre, Hist. Critique du Manicheisme, tom.
i. p. 377. The Gospel according to St. John is supposed to have
been published about seventy years after the death of Christ.}

\pagenote[22]{The sentiments of the Ebionites are fairly stated
by Mosheim (p. 331) and Le Clerc, (Hist. Eccles. p. 535.) The
Clementines, published among the apostolical fathers, are
attributed by the critics to one of these sectaries.}

\pagenote[23]{Stanch polemics, like a Bull, (Judicium Eccles.
Cathol. c. 2,) insist on the orthodoxy of the Nazarenes; which
appears less pure and certain in the eyes of Mosheim, (p. 330.)}

\pagenote[24]{The humble condition and sufferings of Jesus have
always been a stumbling-block to the Jews. “Deus... contrariis
coloribus Messiam depinxerat: futurus erat Rex, Judex, Pastor,”
\&c. See Limborch et Orobio Amica Collat. p. 8, 19, 53-76,
192-234. But this objection has obliged the believing Christians
to lift up their eyes to a spiritual and everlasting kingdom.}

\pagenote[25]{Justin Martyr, Dialog. cum Tryphonte, p. 143, 144.
See Le Clerc, Hist. Eccles. p. 615. Bull and his editor Grabe
(Judicium Eccles. Cathol. c. 7, and Appendix) attempt to distort
either the sentiments or the words of Justin; but their violent
correction of the text is rejected even by the Benedictine
editors.}

\pagenote[26]{The Arians reproached the orthodox party with
borrowing their Trinity from the Valentinians and Marcionites.
See Beausobre, Hist. de Manicheisme, l. iii. c. 5, 7.}

\pagenote[27]{Non dignum est ex utero credere Deum, et Deum
Christum.... non dignum est ut tanta majestas per sordes et
squalores muli eris transire credatur. The Gnostics asserted the
impurity of matter, and of marriage; and they were scandalized by
the gross interpretations of the fathers, and even of Augustin
himself. See Beausobre, tom. ii. p. 523, * Note: The greater part
of the Docetæ rejected the true divinity of Jesus Christ, as well
as his human nature. They belonged to the Gnostics, whom some
philosophers, in whose party Gibbon has enlisted, make to derive
their opinions from those of Plato. These philosophers did not
consider that Platonism had undergone continual alterations, and
that those who gave it some analogy with the notions of the
Gnostics were later in their origin than most of the sects
comprehended under this name Mosheim has proved (in his Instit.
Histor. Eccles. Major. s. i. p. 136, sqq and p. 339, sqq.) that
the Oriental philosophy, combined with the cabalistical
philosophy of the Jews, had given birth to Gnosticism. The
relations which exist between this doctrine and the records which
remain to us of that of the Orientals, the Chaldean and Persian,
have been the source of the errors of the Gnostic Christians, who
wished to reconcile their ancient notions with their new belief.
It is on this account that, denying the human nature of Christ,
they also denied his intimate union with God, and took him for
one of the substances (æons) created by God. As they believed in
the eternity of matter, and considered it to be the principle of
evil, in opposition to the Deity, the first cause and principle
of good, they were unwilling to admit that one of the pure
substances, one of the æons which came forth from God, had, by
partaking in the material nature, allied himself to the principle
of evil; and this was their motive for rejecting the real
humanity of Jesus Christ. See Ch. G. F. Walch, Hist. of Heresies
in Germ. t. i. p. 217, sqq. Brucker, Hist. Crit. Phil. ii. p
639.—G.}

\pagenote[28]{Apostolis adhuc in sæculo superstitibus apud Judæam
Christi sanguine recente, et \textit{phantasma} corpus Domini
asserebatur. Cotelerius thinks (Patres Apostol. tom. ii. p. 24)
that those who will not allow the \textit{Docetes} to have arisen in the
time of the Apostles, may with equal reason deny that the sun
shines at noonday. These \textit{Docetes}, who formed the most
considerable party among the Gnostics, were so called, because
they granted only a \textit{seeming} body to Christ. * Note: The name of
Docetæ was given to these sectaries only in the course of the
second century: this name did not designate a sect, properly so
called; it applied to all the sects who taught the non- reality
of the material body of Christ; of this number were the
Valentinians, the Basilidians, the Ophites, the Marcionites,
(against whom Tertullian wrote his book, De Carne Christi,) and
other Gnostics. In truth, Clement of Alexandria (l. iii. Strom.
c. 13, p. 552) makes express mention of a sect of Docetæ, and
even names as one of its heads a certain Cassianus; but every
thing leads us to believe that it was not a distinct sect.
Philastrius (de Hæres, c. 31) reproaches Saturninus with being a
Docete. Irenæus (adv. Hær. c. 23) makes the same reproach against
Basilides. Epiphanius and Philastrius, who have treated in detail
on each particular heresy, do not specially name that of the
Docetæ. Serapion, bishop of Antioch, (Euseb. Hist. Eccles. l. vi.
c. 12,) and Clement of Alexandria, (l. vii. Strom. p. 900,)
appear to be the first who have used the generic name. It is not
found in any earlier record, though the error which it points out
existed even in the time of the Apostles. See Ch. G. F. Walch,
Hist. of Her. v. i. p. 283. Tillemont, Mempour servir a la Hist
Eccles. ii. p. 50. Buddæus de Eccles. Apost. c. 5 \& 7—G.}

The divine sanction, which the Apostle had bestowed on the
fundamental principle of the theology of Plato, encouraged the
learned proselytes of the second and third centuries to admire
and study the writings of the Athenian sage, who had thus
marvellously anticipated one of the most surprising discoveries
of the Christian revelation. The respectable name of Plato was
used by the orthodox,\textsuperscript{29} and abused by the heretics,\textsuperscript{30} as the
common support of truth and error: the authority of his skilful
commentators, and the science of dialectics, were employed to
justify the remote consequences of his opinions and to supply the
discreet silence of the inspired writers. The same subtle and
profound questions concerning the nature, the generation, the
distinction, and the equality of the three divine persons of the
mysterious \textit{Triad}, or \textit{Trinity},\textsuperscript{31} were agitated in the
philosophical and in the Christian schools of Alexandria. An
eager spirit of curiosity urged them to explore the secrets of
the abyss; and the pride of the professors, and of their
disciples, was satisfied with the sciences of words. But the most
sagacious of the Christian theologians, the great Athanasius
himself, has candidly confessed,\textsuperscript{32} that whenever he forced his
understanding to meditate on the divinity of the \textit{Logos}, his
toilsome and unavailing efforts recoiled on themselves; that the
more he thought, the less he comprehended; and the more he wrote,
the less capable was he of expressing his thoughts. In every step
of the inquiry, we are compelled to feel and acknowledge the
immeasurable disproportion between the size of the object and the
capacity of the human mind. We may strive to abstract the notions
of time, of space, and of matter, which so closely adhere to all
the perceptions of our experimental knowledge. But as soon as we
presume to reason of infinite substance, of spiritual generation;
as often as we deduce any positive conclusions from a negative
idea, we are involved in darkness, perplexity, and inevitable
contradiction. As these difficulties arise from the nature of the
subject, they oppress, with the same insuperable weight, the
philosophic and the theological disputant; but we may observe two
essential and peculiar circumstances, which discriminated the
doctrines of the Catholic church from the opinions of the
Platonic school.

\pagenote[29]{Some proofs of the respect which the Christians
entertained for the person and doctrine of Plato may be found in
De la Mothe le Vayer, tom. v. p. 135, \&c., edit. 1757; and
Basnage, Hist. des Juifs tom. iv. p. 29, 79, \&c.}

\pagenote[30]{Doleo bona fide, Platonem omnium heræticorum
condimentarium factum. Tertullian. de Anima, c. 23. Petavius
(Dogm. Theolog. tom. iii. proleg. 2) shows that this was a
general complaint. Beausobre (tom. i. l. iii. c. 9, 10) has
deduced the Gnostic errors from Platonic principles; and as, in
the school of Alexandria, those principles were blended with the
Oriental philosophy, (Brucker, tom. i. p. 1356,) the sentiment of
Beausobre may be reconciled with the opinion of Mosheim, (General
History of the Church, vol. i. p. 37.)}

\pagenote[31]{If Theophilus, bishop of Antioch, (see Dupin,
Bibliothèque Ecclesiastique, tom. i. p. 66,) was the first who
employed the word \textit{Triad}, \textit{Trinity}, that abstract term, which
was already familiar to the schools of philosophy, must have been
introduced into the theology of the Christians after the middle
of the second century.}

\pagenote[32]{Athanasius, tom. i. p. 808. His expressions have an
uncommon energy; and as he was writing to monks, there could not
be any occasion for him to \textit{affect} a rational language.}

I. A chosen society of philosophers, men of a liberal education
and curious disposition, might silently meditate, and temperately
discuss in the gardens of Athens or the library of Alexandria,
the abstruse questions of metaphysical science. The lofty
speculations, which neither convinced the understanding, nor
agitated the passions, of the Platonists themselves, were
carelessly overlooked by the idle, the busy, and even the
studious part of mankind.\textsuperscript{33} But after the \textit{Logos} had been
revealed as the sacred object of the faith, the hope, and the
religious worship of the Christians, the mysterious system was
embraced by a numerous and increasing multitude in every province
of the Roman world. Those persons who, from their age, or sex, or
occupations, were the least qualified to judge, who were the
least exercised in the habits of abstract reasoning, aspired to
contemplate the economy of the Divine Nature: and it is the boast
of Tertullian,\textsuperscript{34} that a Christian mechanic could readily answer
such questions as had perplexed the wisest of the Grecian sages.
Where the subject lies so far beyond our reach, the difference
between the highest and the lowest of human understandings may
indeed be calculated as infinitely small; yet the degree of
weakness may perhaps be measured by the degree of obstinacy and
dogmatic confidence. These speculations, instead of being treated
as the amusement of a vacant hour, became the most serious
business of the present, and the most useful preparation for a
future, life. A theology, which it was incumbent to believe,
which it was impious to doubt, and which it might be dangerous,
and even fatal, to mistake, became the familiar topic of private
meditation and popular discourse. The cold indifference of
philosophy was inflamed by the fervent spirit of devotion; and
even the metaphors of common language suggested the fallacious
prejudices of sense and experience. The Christians, who abhorred
the gross and impure generation of the Greek mythology,\textsuperscript{35} were
tempted to argue from the familiar analogy of the filial and
paternal relations. The character of \textit{Son} seemed to imply a
perpetual subordination to the voluntary author of his existence;\textsuperscript{36}
but as the act of generation, in the most spiritual and
abstracted sense, must be supposed to transmit the properties of
a common nature,\textsuperscript{37} they durst not presume to circumscribe the
powers or the duration of the Son of an eternal and omnipotent
Father. Fourscore years after the death of Christ, the Christians
of Bithynia, declared before the tribunal of Pliny, that they
invoked him as a god: and his divine honors have been perpetuated
in every age and country, by the various sects who assume the
name of his disciples.\textsuperscript{38} Their tender reverence for the memory
of Christ, and their horror for the profane worship of any
created being, would have engaged them to assert the equal and
absolute divinity of the \textit{Logos}, if their rapid ascent towards
the throne of heaven had not been imperceptibly checked by the
apprehension of violating the unity and sole supremacy of the
great Father of Christ and of the Universe. The suspense and
fluctuation produced in the minds of the Christians by these
opposite tendencies, may be observed in the writings of the
theologians who flourished after the end of the apostolic age,
and before the origin of the Arian controversy. Their suffrage is
claimed, with equal confidence, by the orthodox and by the
heretical parties; and the most inquisitive critics have fairly
allowed, that if they had the good fortune of possessing the
Catholic verity, they have delivered their conceptions in loose,
inaccurate, and sometimes contradictory language.\textsuperscript{39}

\pagenote[33]{In a treatise, which professed to explain the
opinions of the ancient philosophers concerning the nature of the
gods we might expect to discover the theological Trinity of
Plato. But Cicero very honestly confessed, that although he had
translated the Timæus, he could never understand that mysterious
dialogue. See Hieronym. præf. ad l. xii. in Isaiam, tom. v. p.
154.}

\pagenote[34]{Tertullian. in Apolog. c. 46. See Bayle,
Dictionnaire, au mot \textit{Simonide}. His remarks on the presumption
of Tertullian are profound and interesting.}

\pagenote[35]{Lactantius, iv. 8. Yet the \textit{Probole}, or
\textit{Prolatio}, which the most orthodox divines borrowed without
scruple from the Valentinians, and illustrated by the comparisons
of a fountain and stream, the sun and its rays, \&c., either meant
nothing, or favored a material idea of the divine generation. See
Beausobre, tom. i. l. iii. c. 7, p. 548.}

\pagenote[36]{Many of the primitive writers have frankly
confessed, that the Son owed his being to the \textit{will} of the
Father.——See Clarke’s Scripture Trinity, p. 280-287. On the other
hand, Athanasius and his followers seem unwilling to grant what
they are afraid to deny. The schoolmen extricate themselves from
this difficulty by the distinction of a \textit{preceding} and a
\textit{concomitant} will. Petav. Dogm. Theolog. tom. ii. l. vi. c. 8,
p. 587-603.}

\pagenote[37]{See Petav. Dogm. Theolog. tom. ii. l. ii. c. 10, p. 159.}

\pagenote[38]{Carmenque Christo quasi Deo dicere secum invicem.
Plin. Epist. x. 97. The sense of \textit{Deus, Elohim}, in the ancient
languages, is critically examined by Le Clerc, (Ars Critica, p.
150-156,) and the propriety of worshipping a very excellent
creature is ably defended by the Socinian Emlyn, (Tracts, p.
29-36, 51-145.)}

\pagenote[39]{See Daille de Usu Patrum, and Le Clerc,
Bibliothèque Universelle, tom. x. p. 409. To arraign the faith of
the Ante-Nicene fathers, was the object, or at least has been the
effect, of the stupendous work of Petavius on the Trinity, (Dogm.
Theolog. tom. ii.;) nor has the deep impression been erased by
the learned defence of Bishop Bull. Note: Dr. Burton’s work on
the doctrine of the Ante-Nicene fathers must be consulted by
those who wish to obtain clear notions on this subject.—M.}

\section{Part \thesection.}

II. The devotion of individuals was the first circumstance which
distinguished the Christians from the Platonists: the second was
the authority of the church. The disciples of philosophy asserted
the rights of intellectual freedom, and their respect for the
sentiments of their teachers was a liberal and voluntary tribute,
which they offered to superior reason. But the Christians formed
a numerous and disciplined society; and the jurisdiction of their
laws and magistrates was strictly exercised over the minds of the
faithful. The loose wanderings of the imagination were gradually
confined by creeds and confessions;\textsuperscript{40} the freedom of private
judgment submitted to the public wisdom of synods; the authority
of a theologian was determined by his ecclesiastical rank; and
the episcopal successors of the apostles inflicted the censures
of the church on those who deviated from the orthodox belief. But
in an age of religious controversy, every act of oppression adds
new force to the elastic vigor of the mind; and the zeal or
obstinacy of a spiritual rebel was sometimes stimulated by secret
motives of ambition or avarice. A metaphysical argument became
the cause or pretence of political contests; the subtleties of
the Platonic school were used as the badges of popular factions,
and the distance which separated their respective tenets were
enlarged or magnified by the acrimony of dispute. As long as the
dark heresies of Praxeas and Sabellius labored to confound the
\textit{Father} with the \textit{Son},\textsuperscript{41} the orthodox party might be excused
if they adhered more strictly and more earnestly to the
\textit{distinction}, than to the \textit{equality}, of the divine persons. But
as soon as the heat of controversy had subsided, and the progress
of the Sabellians was no longer an object of terror to the
churches of Rome, of Africa, or of Egypt, the tide of theological
opinion began to flow with a gentle but steady motion towards the
contrary extreme; and the most orthodox doctors allowed
themselves the use of the terms and definitions which had been
censured in the mouth of the sectaries.\textsuperscript{42} After the edict of
toleration had restored peace and leisure to the Christians, the
Trinitarian controversy was revived in the ancient seat of
Platonism, the learned, the opulent, the tumultuous city of
Alexandria; and the flame of religious discord was rapidly
communicated from the schools to the clergy, the people, the
province, and the East. The abstruse question of the eternity of
the \textit{Logos} was agitated in ecclesiastic conferences and popular
sermons; and the heterodox opinions of Arius\textsuperscript{43} were soon made
public by his own zeal, and by that of his adversaries. His most
implacable adversaries have acknowledged the learning and
blameless life of that eminent presbyter, who, in a former
election, had declared, and perhaps generously declined, his
pretensions to the episcopal throne.\textsuperscript{44} His competitor Alexander
assumed the office of his judge. The important cause was argued
before him; and if at first he seemed to hesitate, he at length
pronounced his final sentence, as an absolute rule of faith.\textsuperscript{45}
The undaunted presbyter, who presumed to resist the authority of
his angry bishop, was separated from the community of the church.
But the pride of Arius was supported by the applause of a
numerous party. He reckoned among his immediate followers two
bishops of Egypt, seven presbyters, twelve deacons, and (what may
appear almost incredible) seven hundred virgins. A large majority
of the bishops of Asia appeared to support or favor his cause;
and their measures were conducted by Eusebius of Cæsarea, the
most learned of the Christian prelates; and by Eusebius of
Nicomedia, who had acquired the reputation of a statesman without
forfeiting that of a saint. Synods in Palestine and Bithynia were
opposed to the synods of Egypt. The attention of the prince and
people was attracted by this theological dispute; and the
decision, at the end of six years,\textsuperscript{46} was referred to the supreme
authority of the general council of Nice.

\pagenote[40]{The most ancient creeds were drawn up with the
greatest latitude. See Bull, (Judicium Eccles. Cathol.,) who
tries to prevent Episcopius from deriving any advantage from this
observation.}

\pagenote[41]{The heresies of Praxeas, Sabellius, \&c., are
accurately explained by Mosheim (p. 425, 680-714.) Praxeas, who
came to Rome about the end of the second century, deceived, for
some time, the simplicity of the bishop, and was confuted by the
pen of the angry Tertullian.}

\pagenote[42]{Socrates acknowledges, that the heresy of Arius
proceeded from his strong desire to embrace an opinion the most
diametrically opposite to that of Sabellius.}

\pagenote[43]{The figure and manners of Arius, the character and
numbers of his first proselytes, are painted in very lively
colors by Epiphanius, (tom. i. Hæres. lxix. 3, p. 729,) and we
cannot but regret that he should soon forget the historian, to
assume the task of controversy.}

\pagenote[44]{See Philostorgius (l. i. c. 3,) and Godefroy’s
ample Commentary. Yet the credibility of Philostorgius is
lessened, in the eyes of the orthodox, by his Arianism; and in
those of rational critics, by his passion, his prejudice, and his
ignorance.}

\pagenote[45]{Sozomen (l. i. c. 15) represents Alexander as
indifferent, and even ignorant, in the beginning of the
controversy; while Socrates (l. i. c. 5) ascribes the origin of
the dispute to the vain curiosity of his theological
speculations. Dr. Jortin (Remarks on Ecclesiastical History, vol.
ii. p. 178) has censured, with his usual freedom, the conduct of
Alexander.}

\pagenote[46]{The flames of Arianism might burn for some time in
secret; but there is reason to believe that they burst out with
violence as early as the year 319. Tillemont, Mém. Eccles. tom.
vi. p. 774-780.}

When the mysteries of the Christian faith were dangerously
exposed to public debate, it might be observed, that the human
understanding was capable of forming three district, though
imperfect systems, concerning the nature of the Divine Trinity;
and it was pronounced, that none of these systems, in a pure and
absolute sense, were exempt from heresy and error.\textsuperscript{47} I.
According to the first hypothesis, which was maintained by Arius
and his disciples, the \textit{Logos} was a dependent and spontaneous
production, created from nothing by the will of the father. The
Son, by whom all things were made,\textsuperscript{48} had been begotten before
all worlds, and the longest of the astronomical periods could be
compared only as a fleeting moment to the extent of his duration;
yet this duration was not infinite,\textsuperscript{49} and there \textit{had} been a
time which preceded the ineffable generation of the \textit{Logos}. On
this only-begotten Son, the Almighty Father had transfused his
ample spirit, and impressed the effulgence of his glory. Visible
image of invisible perfection, he saw, at an immeasurable
distance beneath his feet, the thrones of the brightest
archangels; yet he shone only with a reflected light, and, like
the sons of the Romans emperors, who were invested with the
titles of Cæsar or Augustus,\textsuperscript{50} he governed the universe in
obedience to the will of his Father and Monarch. II. In the
second hypothesis, the \textit{Logos} possessed all the inherent,
incommunicable perfections, which religion and philosophy
appropriate to the Supreme God. Three distinct and infinite minds
or substances, three coëqual and coëternal beings, composed the
Divine Essence;\textsuperscript{51} and it would have implied contradiction, that
any of them should not have existed, or that they should ever
cease to exist.\textsuperscript{52} The advocates of a system which seemed to
establish three independent Deities, attempted to preserve the
unity of the First Cause, so conspicuous in the design and order
of the world, by the perpetual concord of their administration,
and the essential agreement of their will. A faint resemblance of
this unity of action may be discovered in the societies of men,
and even of animals. The causes which disturb their harmony,
proceed only from the imperfection and inequality of their
faculties; but the omnipotence which is guided by infinite wisdom
and goodness, cannot fail of choosing the same means for the
accomplishment of the same ends. III. Three beings, who, by the
self-derived necessity of their existence, possess all the divine
attributes in the most perfect degree; who are eternal in
duration, infinite in space, and intimately present to each
other, and to the whole universe; irresistibly force themselves
on the astonished mind, as one and the same being,\textsuperscript{53} who, in the
economy of grace, as well as in that of nature, may manifest
himself under different forms, and be considered under different
aspects. By this hypothesis, a real substantial trinity is
refined into a trinity of names, and abstract modifications, that
subsist only in the mind which conceives them. The \textit{Logos} is no
longer a person, but an attribute; and it is only in a figurative
sense that the epithet of Son can be applied to the eternal
reason, which was with God from the beginning, and by \textit{which},
not by \textit{whom}, all things were made. The incarnation of the
\textit{Logos} is reduced to a mere inspiration of the Divine Wisdom,
which filled the soul, and directed all the actions, of the man
Jesus. Thus, after revolving around the theological circle, we
are surprised to find that the Sabellian ends where the Ebionite
had begun; and that the incomprehensible mystery which excites
our adoration, eludes our inquiry.\textsuperscript{54}

\pagenote[47]{Quid credidit? Certe, \textit{aut} tria nomina audiens
tres Deos esse credidit, et idololatra effectus est; \textit{aut} in
tribus vocabulis trinominem credens Deum, in Sabellii hæresim
incurrit; \textit{aut} edoctus ab Arianis unum esse verum Deum Patrem,
filium et spiritum sanctum credidit creaturas. Aut extra hæc quid
credere potuerit nescio. Hieronym adv. Luciferianos. Jerom
reserves for the last the orthodox system, which is more
complicated and difficult.}

\pagenote[48]{As the doctrine of absolute creation from nothing
was gradually introduced among the Christians, (Beausobre, tom.
ii. p. 165- 215,) the dignity of the \textit{workman} very naturally
rose with that of the \textit{work}.}

\pagenote[49]{The metaphysics of Dr. Clarke (Scripture Trinity,
p. 276-280) could digest an eternal generation from an infinite
cause.}

\pagenote[50]{This profane and absurd simile is employed by
several of the primitive fathers, particularly by Athenagoras, in
his Apology to the emperor Marcus and his son; and it is alleged,
without censure, by Bull himself. See Defens. Fid. Nicen. sect.
iii. c. 5, No. 4.}

\pagenote[51]{See Cudworth’s Intellectual System, p. 559, 579.
This dangerous hypothesis was countenanced by the two Gregories,
of Nyssa and Nazianzen, by Cyril of Alexandria, John of Damascus,
\&c. See Cudworth, p. 603. Le Clerc, Bibliothèque Universelle, tom
xviii. p. 97-105.}

\pagenote[52]{Augustin seems to envy the freedom of the
Philosophers. Liberis verbis loquuntur philosophi.... Nos autem
non dicimus duo vel tria principia, duos vel tres Deos. De
Civitat. Dei, x. 23.}

\pagenote[53]{Boetius, who was deeply versed in the philosophy of
Plato and Aristotle, explains the unity of the Trinity by the
\textit{indifference} of the three persons. See the judicious remarks of
Le Clerc, Bibliothèque Choisie, tom. xvi. p. 225, \&c.}

\pagenote[54]{If the Sabellians were startled at this conclusion,
they were driven another precipice into the confession, that the
Father was born of a virgin, that \textit{he} had suffered on the cross;
and thus deserved the epithet of \textit{Patripassians}, with which they
were branded by their adversaries. See the invectives of
Tertullian against Praxeas, and the temperate reflections of
Mosheim, (p. 423, 681;) and Beausobre, tom. i. l. iii. c. 6, p. 533.}

If the bishops of the council of Nice\textsuperscript{55} had been permitted to
follow the unbiased dictates of their conscience, Arius and his
associates could scarcely have flattered themselves with the
hopes of obtaining a majority of votes, in favor of an hypothesis
so directly averse to the two most popular opinions of the
Catholic world. The Arians soon perceived the danger of their
situation, and prudently assumed those modest virtues, which, in
the fury of civil and religious dissensions, are seldom
practised, or even praised, except by the weaker party. They
recommended the exercise of Christian charity and moderation;
urged the incomprehensible nature of the controversy, disclaimed
the use of any terms or definitions which could not be found in
the Scriptures; and offered, by very liberal concessions, to
satisfy their adversaries without renouncing the integrity of
their own principles. The victorious faction received all their
proposals with haughty suspicion; and anxiously sought for some
irreconcilable mark of distinction, the rejection of which might
involve the Arians in the guilt and consequences of heresy. A
letter was publicly read, and ignominiously torn, in which their
patron, Eusebius of Nicomedia, ingenuously confessed, that the
admission of the Homoousion, or Consubstantial, a word already
familiar to the Platonists, was incompatible with the principles
of their theological system. The fortunate opportunity was
eagerly embraced by the bishops, who governed the resolutions of
the synod; and, according to the lively expression of Ambrose,\textsuperscript{56}
they used the sword, which heresy itself had drawn from the
scabbard, to cut off the head of the hated monster. The
consubstantiality of the Father and the Son was established by
the council of Nice, and has been unanimously received as a
fundamental article of the Christian faith, by the consent of the
Greek, the Latin, the Oriental, and the Protestant churches. But
if the same word had not served to stigmatize the heretics, and
to unite the Catholics, it would have been inadequate to the
purpose of the majority, by whom it was introduced into the
orthodox creed. This majority was divided into two parties,
distinguished by a contrary tendency to the sentiments of the
Tritheists and of the Sabellians. But as those opposite extremes
seemed to overthrow the foundations either of natural or revealed
religion, they mutually agreed to qualify the rigor of their
principles; and to disavow the just, but invidious, consequences,
which might be urged by their antagonists. The interest of the
common cause inclined them to join their numbers, and to conceal
their differences; their animosity was softened by the healing
counsels of toleration, and their disputes were suspended by the
use of the mysterious \textit{Homoousion}, which either party was free
to interpret according to their peculiar tenets. The Sabellian
sense, which, about fifty years before, had obliged the council
of Antioch\textsuperscript{57} to prohibit this celebrated term, had endeared it
to those theologians who entertained a secret but partial
affection for a nominal Trinity. But the more fashionable saints
of the Arian times, the intrepid Athanasius, the learned Gregory
Nazianzen, and the other pillars of the church, who supported
with ability and success the Nicene doctrine, appeared to
consider the expression of \textit{substance} as if it had been
synonymous with that of \textit{nature;} and they ventured to illustrate
their meaning, by affirming that three men, as they belong to the
same common species, are consubstantial, or homoousian to each
other.\textsuperscript{58} This pure and distinct equality was tempered, on the
one hand, by the internal connection, and spiritual penetration
which indissolubly unites the divine persons;\textsuperscript{59} and, on the
other, by the preëminence of the Father, which was acknowledged
as far as it is compatible with the independence of the Son.\textsuperscript{60}
Within these limits, the almost invisible and tremulous ball of
orthodoxy was allowed securely to vibrate. On either side, beyond
this consecrated ground, the heretics and the dæmons lurked in
ambush to surprise and devour the unhappy wanderer. But as the
degrees of theological hatred depend on the spirit of the war,
rather than on the importance of the controversy, the heretics
who degraded, were treated with more severity than those who
annihilated, the person of the Son. The life of Athanasius was
consumed in irreconcilable opposition to the impious \textit{madness} of
the Arians;\textsuperscript{61} but he defended above twenty years the
Sabellianism of Marcellus of Ancyra; and when at last he was
compelled to withdraw himself from his communion, he continued to
mention, with an ambiguous smile, the venial errors of his
respectable friend.\textsuperscript{62}

\pagenote[55]{The transactions of the council of Nice are related
by the ancients, not only in a partial, but in a very imperfect
manner. Such a picture as Fra Paolo would have drawn, can never
be recovered; but such rude sketches as have been traced by the
pencil of bigotry, and that of reason, may be seen in Tillemont,
(Mém. Eccles. tom. v. p. 669-759,) and in Le Clerc, (Bibliothèque
Universelle, tom. x p. 435-454.)}

\pagenote[56]{We are indebted to Ambrose (De Fide, l. iii.)
knowledge of this curious anecdote. Hoc verbum quod viderunt
adversariis esse formidini; ut ipsis gladio, ipsum nefandæ caput
hæreseos.}

\pagenote[57]{See Bull, Defens. Fid. Nicen. sect. ii. c. i. p.
25-36. He thinks it his duty to reconcile two orthodox synods.}

\pagenote[58]{According to Aristotle, the stars were homoousian
to each other. “That \textit{Homoousios} means of one substance in
\textit{kind}, hath been shown by Petavius, Curcellæus, Cudworth, Le
Clerc, \&c., and to prove it would be \textit{actum agere}.” This is the
just remark of Dr. Jortin, (vol. ii p. 212,) who examines the
Arian controversy with learning, candor, and ingenuity.}

\pagenote[59]{See Petavius, (Dogm. Theolog. tom. ii. l. iv. c.
16, p. 453, \&c.,) Cudworth, (p. 559,) Bull, (sect. iv. p.
285-290, edit. Grab.) The \textit{circumincessio}, is perhaps the
deepest and darkest he whole theological abyss.}

\pagenote[60]{The third section of Bull’s Defence of the Nicene
Faith, which some of his antagonists have called nonsense, and
others heresy, is consecrated to the supremacy of the Father.}

\pagenote[61]{The ordinary appellation with which Athanasius and
his followers chose to compliment the Arians, was that of
\textit{Ariomanites}.}

\pagenote[62]{Epiphanius, tom i. Hæres. lxxii. 4, p. 837. See the
adventures of Marcellus, in Tillemont, (Mém. Eccles. tom. v. i.
p. 880- 899.) His work, in \textit{one} book, of the unity of God, was
answered in the \textit{three} books, which are still extant, of
Eusebius.——After a long and careful examination, Petavius (tom.
ii. l. i. c. 14, p. 78) has reluctantly pronounced the
condemnation of Marcellus.}

The authority of a general council, to which the Arians
themselves had been compelled to submit, inscribed on the banners
of the orthodox party the mysterious characters of the word
\textit{Homoousion}, which essentially contributed, notwithstanding some
obscure disputes, some nocturnal combats, to maintain and
perpetuate the uniformity of faith, or at least of language. The
consubstantialists, who by their success have deserved and
obtained the title of Catholics, gloried in the simplicity and
steadiness of their own creed, and insulted the repeated
variations of their adversaries, who were destitute of any
certain rule of faith. The sincerity or the cunning of the Arian
chiefs, the fear of the laws or of the people, their reverence
for Christ, their hatred of Athanasius, all the causes, human and
divine, that influence and disturb the counsels of a theological
faction, introduced among the sectaries a spirit of discord and
inconstancy, which, in the course of a few years, erected
eighteen different models of religion,\textsuperscript{63} and avenged the
violated dignity of the church. The zealous Hilary,\textsuperscript{64} who, from
the peculiar hardships of his situation, was inclined to
extenuate rather than to aggravate the errors of the Oriental
clergy, declares, that in the wide extent of the ten provinces of
Asia, to which he had been banished, there could be found very
few prelates who had preserved the knowledge of the true God.\textsuperscript{65}
The oppression which he had felt, the disorders of which he was
the spectator and the victim, appeased, during a short interval,
the angry passions of his soul; and in the following passage, of
which I shall transcribe a few lines, the bishop of Poitiers
unwarily deviates into the style of a Christian philosopher. “It
is a thing,” says Hilary, “equally deplorable and dangerous, that
there are as many creeds as opinions among men, as many doctrines
as inclinations, and as many sources of blasphemy as there are
faults among us; because we make creeds arbitrarily, and explain
them as arbitrarily. The Homoousion is rejected, and received,
and explained away by successive synods. The partial or total
resemblance of the Father and of the Son is a subject of dispute
for these unhappy times. Every year, nay, every moon, we make new
creeds to describe invisible mysteries. We repent of what we have
done, we defend those who repent, we anathematize those whom we
defended. We condemn either the doctrine of others in ourselves,
or our own in that of others; and reciprocally tearing one
another to pieces, we have been the cause of each other’s ruin.”\textsuperscript{66}

\pagenote[63]{Athanasius, in his epistle concerning the Synods of
Seleucia and Rimini, (tom. i. p. 886-905,) has given an ample
list of Arian creeds, which has been enlarged and improved by the
labors of the indefatigable Tillemont, (Mém. Eccles. tom. vi. p.
477.)}

\pagenote[64]{Erasmus, with admirable sense and freedom, has
delineated the just character of Hilary. To revise his text, to
compose the annals of his life, and to justify his sentiments and
conduct, is the province of the Benedictine editors.}

\pagenote[65]{Absque episcopo Eleusio et paucis cum eo, ex majore
parte Asianæ decem provinciæ, inter quas consisto, vere Deum
nesciunt. Atque utinam penitus nescirent! cum procliviore enim
venia ignorarent quam obtrectarent. Hilar. de Synodis, sive de
Fide Orientalium, c. 63, p. 1186, edit. Benedict. In the
celebrated parallel between atheism and superstition, the bishop
of Poitiers would have been surprised in the philosophic society
of Bayle and Plutarch.}

\pagenote[66]{Hilarius ad Constantium, l. i. c. 4, 5, p. 1227,
1228. This remarkable passage deserved the attention of Mr.
Locke, who has transcribed it (vol. iii. p. 470) into the model
of his new common-place book.}

It will not be expected, it would not perhaps be endured, that I
should swell this theological digression, by a minute examination
of the eighteen creeds, the authors of which, for the most part,
disclaimed the odious name of their parent Arius. It is amusing
enough to delineate the form, and to trace the vegetation, of a
singular plant; but the tedious detail of leaves without flowers,
and of branches without fruit, would soon exhaust the patience,
and disappoint the curiosity, of the laborious student. One
question, which gradually arose from the Arian controversy, may,
however, be noticed, as it served to produce and discriminate the
three sects, who were united only by their common aversion to the
Homoousion of the Nicene synod. 1. If they were asked whether the
Son was \textit{like} unto the Father, the question was resolutely
answered in the negative, by the heretics who adhered to the
principles of Arius, or indeed to those of philosophy; which seem
to establish an infinite difference between the Creator and the
most excellent of his creatures. This obvious consequence was
maintained by Ætius,\textsuperscript{67} on whom the zeal of his adversaries
bestowed the surname of the Atheist. His restless and aspiring
spirit urged him to try almost every profession of human life. He
was successively a slave, or at least a husbandman, a travelling
tinker, a goldsmith, a physician, a schoolmaster, a theologian,
and at last the apostle of a new church, which was propagated by
the abilities of his disciple Eunomius.\textsuperscript{68} Armed with texts of
Scripture, and with captious syllogisms from the logic of
Aristotle, the subtle Ætius had acquired the fame of an
invincible disputant, whom it was impossible either to silence or
to convince. Such talents engaged the friendship of the Arian
bishops, till they were forced to renounce, and even to
persecute, a dangerous ally, who, by the accuracy of his
reasoning, had prejudiced their cause in the popular opinion, and
offended the piety of their most devoted followers. 2. The
omnipotence of the Creator suggested a specious and respectful
solution of the \textit{likeness} of the Father and the Son; and faith
might humbly receive what reason could not presume to deny, that
the Supreme God might communicate his infinite perfections, and
create a being similar only to himself.\textsuperscript{69} These Arians were
powerfully supported by the weight and abilities of their
leaders, who had succeeded to the management of the Eusebian
interest, and who occupied the principal thrones of the East.
They detested, perhaps with some affectation, the impiety of
Ætius; they professed to believe, either without reserve, or
according to the Scriptures, that the Son was different from all
\textit{other} creatures, and similar only to the Father. But they
denied, the he was either of the same, or of a similar substance;
sometimes boldly justifying their dissent, and sometimes
objecting to the use of the word substance, which seems to imply
an adequate, or at least, a distinct, notion of the nature of the
Deity. 3. The sect which deserted the doctrine of a similar
substance, was the most numerous, at least in the provinces of
Asia; and when the leaders of both parties were assembled in the
council of Seleucia,\textsuperscript{70} \textit{their} opinion would have prevailed by a
majority of one hundred and five to forty-three bishops. The
Greek word, which was chosen to express this mysterious
resemblance, bears so close an affinity to the orthodox symbol,
that the profane of every age have derided the furious contests
which the difference of a single diphthong excited between the
Homoousians and the Homoiousians. As it frequently happens, that
the sounds and characters which approach the nearest to each
other accidentally represent the most opposite ideas, the
observation would be itself ridiculous, if it were possible to
mark any real and sensible distinction between the doctrine of
the Semi-Arians, as they were improperly styled, and that of the
Catholics themselves. The bishop of Poitiers, who in his Phrygian
exile very wisely aimed at a coalition of parties, endeavors to
prove that by a pious and faithful interpretation,\textsuperscript{71} the
\textit{Homoiousion} may be reduced to a consubstantial sense. Yet he
confesses that the word has a dark and suspicious aspect; and, as
if darkness were congenial to theological disputes, the
Semi-Arians, who advanced to the doors of the church, assailed
them with the most unrelenting fury.

\pagenote[67]{In Philostorgius (l. iii. c. 15) the character and
adventures of Ætius appear singular enough, though they are
carefully softened by the hand of a friend. The editor, Godefroy,
(p. 153,) who was more attached to his principles than to his
author, has collected the odious circumstances which his various
adversaries have preserved or invented.}

\pagenote[68]{According to the judgment of a man who respected
both these sectaries, Ætius had been endowed with a stronger
understanding and Eunomius had acquired more art and learning.
(Philostorgius l. viii. c. 18.) The confession and apology of
Eunomius (Fabricius, Bibliot. Græc. tom. viii. p. 258-305) is one
of the few heretical pieces which have escaped.}

\pagenote[69]{Yet, according to the opinion of Estius and Bull,
(p. 297,) there is one power—that of creation—which God \textit{cannot}
communicate to a creature. Estius, who so accurately defined the
limits of Omnipotence was a Dutchman by birth, and by trade a
scholastic divine. Dupin Bibliot. Eccles. tom. xvii. p. 45.}

\pagenote[70]{Sabinus ap. Socrat. (l. ii. c. 39) had copied the
acts: Athanasius and Hilary have explained the divisions of this
Arian synod; the other circumstances which are relative to it are
carefully collected by Baro and Tillemont}

\pagenote[71]{Fideli et piâ intelligentiâ... De Synod. c. 77, p.
1193. In his his short apologetical notes (first published by the
Benedictines from a MS. of Chartres) he observes, that he used
this cautious expression, qui intelligerum et impiam, p. 1206.
See p. 1146. Philostorgius, who saw those objects through a
different medium, is inclined to forget the difference of the
important diphthong. See in particular viii. 17, and Godefroy, p. 352.}

The provinces of Egypt and Asia, which cultivated the language
and manners of the Greeks, had deeply imbibed the venom of the
Arian controversy. The familiar study of the Platonic system, a
vain and argumentative disposition, a copious and flexible idiom,
supplied the clergy and people of the East with an inexhaustible
flow of words and distinctions; and, in the midst of their fierce
contentions, they easily forgot the doubt which is recommended by
philosophy, and the submission which is enjoined by religion. The
inhabitants of the West were of a less inquisitive spirit; their
passions were not so forcibly moved by invisible objects, their
minds were less frequently exercised by the habits of dispute;
and such was the happy ignorance of the Gallican church, that
Hilary himself, above thirty years after the first general
council, was still a stranger to the Nicene creed.\textsuperscript{72} The Latins
had received the rays of divine knowledge through the dark and
doubtful medium of a translation. The poverty and stubbornness of
their native tongue was not always capable of affording just
equivalents for the Greek terms, for the technical words of the
Platonic philosophy,\textsuperscript{73} which had been consecrated, by the gospel
or by the church, to express the mysteries of the Christian
faith; and a verbal defect might introduce into the Latin
theology a long train of error or perplexity.\textsuperscript{74} But as the
western provincials had the good fortune of deriving their
religion from an orthodox source, they preserved with steadiness
the doctrine which they had accepted with docility; and when the
Arian pestilence approached their frontiers, they were supplied
with the seasonable preservative of the Homoousion, by the
paternal care of the Roman pontiff. Their sentiments and their
temper were displayed in the memorable synod of Rimini, which
surpassed in numbers the council of Nice, since it was composed
of above four hundred bishops of Italy, Africa, Spain, Gaul,
Britain, and Illyricum. From the first debates it appeared, that
only fourscore prelates adhered to the party, though \textit{they}
affected to anathematize the name and memory, of Arius. But this
inferiority was compensated by the advantages of skill, of
experience, and of discipline; and the minority was conducted by
Valens and Ursacius, two bishops of Illyricum, who had spent
their lives in the intrigues of courts and councils, and who had
been trained under the Eusebian banner in the religious wars of
the East. By their arguments and negotiations, they embarrassed,
they confounded, they at last deceived, the honest simplicity of
the Latin bishops; who suffered the palladium of the faith to be
extorted from their hand by fraud and importunity, rather than by
open violence. The council of Rimini was not allowed to separate,
till the members had imprudently subscribed a captious creed, in
which some expressions, susceptible of an heretical sense, were
inserted in the room of the Homoousion. It was on this occasion,
that, according to Jerom, the world was surprised to find itself
Arian.\textsuperscript{75} But the bishops of the Latin provinces had no sooner
reached their respective dioceses, than they discovered their
mistake, and repented of their weakness. The ignominious
capitulation was rejected with disdain and abhorrence; and the
Homoousian standard, which had been shaken but not overthrown,
was more firmly replanted in all the churches of the West.\textsuperscript{76}

\pagenote[72]{Testor Deum cœli atque terræ me cum neutrum
audissem, semper tamen utrumque sensisse.... Regeneratus pridem
et in episcopatu aliquantisper manens fidem Nicenam nunquam nisi
exsulaturus audivi. Hilar. de Synodis, c. xci. p. 1205. The
Benedictines are persuaded that he governed the diocese of
Poitiers several years before his exile.}

\pagenote[73]{Seneca (Epist. lviii.) complains that even the of
the Platonists (the \textit{ens} of the bolder schoolmen) could not be
expressed by a Latin noun.}

\pagenote[74]{The preference which the fourth council of the
Lateran at length gave to a \textit{numerical} rather than a \textit{generical}
unity (See Petav. tom. ii. l. v. c. 13, p. 424) was favored by
the Latin language: seems to excite the idea of substance,
\textit{trinitas} of qualities.}

\pagenote[75]{Ingemuit totus orbis, et Arianum se esse miratus
est. Hieronym. adv. Lucifer. tom. i. p. 145.}

\pagenote[76]{The story of the council of Rimini is very
elegantly told by Sulpicius Severus, (Hist. Sacra, l. ii. p.
419-430, edit. Lugd. Bat. 1647,) and by Jerom, in his dialogue
against the Luciferians. The design of the latter is to apologize
for the conduct of the Latin bishops, who were deceived, and who
repented.}

\section{Part \thesection.}

Such was the rise and progress, and such were the natural
revolutions of those theological disputes, which disturbed the
peace of Christianity under the reigns of Constantine and of his
sons. But as those princes presumed to extend their despotism
over the faith, as well as over the lives and fortunes, of their
subjects, the weight of their suffrage sometimes inclined the
ecclesiastical balance: and the prerogatives of the King of
Heaven were settled, or changed, or modified, in the cabinet of
an earthly monarch. The unhappy spirit of discord which pervaded
the provinces of the East, interrupted the triumph of
Constantine; but the emperor continued for some time to view,
with cool and careless indifference, the object of the dispute.
As he was yet ignorant of the difficulty of appeasing the
quarrels of theologians, he addressed to the contending parties,
to Alexander and to Arius, a moderating epistle;\textsuperscript{77} which may be
ascribed, with far greater reason, to the untutored sense of a
soldier and statesman, than to the dictates of any of his
episcopal counsellors. He attributes the origin of the whole
controversy to a trifling and subtle question, concerning an
incomprehensible point of law, which was foolishly asked by the
bishop, and imprudently resolved by the presbyter. He laments
that the Christian people, who had the same God, the same
religion, and the same worship, should be divided by such
inconsiderable distinctions; and he seriously recommends to the
clergy of Alexandria the example of the Greek philosophers; who
could maintain their arguments without losing their temper, and
assert their freedom without violating their friendship. The
indifference and contempt of the sovereign would have been,
perhaps, the most effectual method of silencing the dispute, if
the popular current had been less rapid and impetuous, and if
Constantine himself, in the midst of faction and fanaticism,
could have preserved the calm possession of his own mind. But his
ecclesiastical ministers soon contrived to seduce the
impartiality of the magistrate, and to awaken the zeal of the
proselyte. He was provoked by the insults which had been offered
to his statues; he was alarmed by the real, as well as the
imaginary magnitude of the spreading mischief; and he
extinguished the hope of peace and toleration, from the moment
that he assembled three hundred bishops within the walls of the
same palace. The presence of the monarch swelled the importance
of the debate; his attention multiplied the arguments; and he
exposed his person with a patient intrepidity, which animated the
valor of the combatants. Notwithstanding the applause which has
been bestowed on the eloquence and sagacity of Constantine,\textsuperscript{78} a
Roman general, whose religion might be still a subject of doubt,
and whose mind had not been enlightened either by study or by
inspiration, was indifferently qualified to discuss, in the Greek
language, a metaphysical question, or an article of faith. But
the credit of his favorite Osius, who appears to have presided in
the council of Nice, might dispose the emperor in favor of the
orthodox party; and a well-timed insinuation, that the same
Eusebius of Nicomedia, who now protected the heretic, had lately
assisted the tyrant,\textsuperscript{79} might exasperate him against their
adversaries. The Nicene creed was ratified by Constantine; and
his firm declaration, that those who resisted the divine judgment
of the synod, must prepare themselves for an immediate exile,
annihilated the murmurs of a feeble opposition; which, from
seventeen, was almost instantly reduced to two, protesting
bishops. Eusebius of Cæsarea yielded a reluctant and ambiguous
consent to the Homoousion;\textsuperscript{80} and the wavering conduct of the
Nicomedian Eusebius served only to delay, about three months, his
disgrace and exile.\textsuperscript{81} The impious Arius was banished into one of
the remote provinces of Illyricum; his person and disciples were
branded by law with the odious name of Porphyrians; his writings
were condemned to the flames, and a capital punishment was
denounced against those in whose possession they should be found.
The emperor had now imbibed the spirit of controversy, and the
angry, sarcastic style of his edicts was designed to inspire his
subjects with the hatred which he had conceived against the
enemies of Christ.\textsuperscript{82}

\pagenote[77]{Eusebius, in Vit. Constant. l. ii. c. 64-72. The
principles of toleration and religious indifference, contained in
this epistle, have given great offence to Baronius, Tillemont,
\&c., who suppose that the emperor had some evil counsellor,
either Satan or Eusebius, at his elbow. See Cortin’s Remarks,
tom. ii. p. 183. * Note: Heinichen (Excursus xi.) quotes with
approbation the term “golden words,” applied by Ziegler to this
moderate and tolerant letter of Constantine. May an English
clergyman venture to express his regret that “the fine gold soon
became dim” in the Christian church?—M.}

\pagenote[78]{Eusebius in Vit. Constantin. l. iii. c. 13.}

\pagenote[79]{Theodoret has preserved (l. i. c. 20) an epistle
from Constantine to the people of Nicomedia, in which the monarch
declares himself the public accuser of one of his subjects; he
styles Eusebius and complains of his hostile behavior during the
civil war.}

\pagenote[80]{See in Socrates, (l. i. c. 8,) or rather in
Theodoret, (l. i. c. 12,) an original letter of Eusebius of
Cæsarea, in which he attempts to justify his subscribing the
Homoousion. The character of Eusebius has always been a problem;
but those who have read the second critical epistle of Le Clerc,
(Ars Critica, tom. iii. p. 30-69,) must entertain a very
unfavorable opinion of the orthodoxy and sincerity of the bishop
of Cæsarea.}

\pagenote[81]{Athanasius, tom. i. p. 727. Philostorgius, l. i. c.
10, and Godefroy’s Commentary, p. 41.}

\pagenote[82]{Socrates, l. i. c. 9. In his circular letters,
which were addressed to the several cities, Constantine employed
against the heretics the arms of ridicule and \textit{comic} raillery.}

But, as if the conduct of the emperor had been guided by passion
instead of principle, three years from the council of Nice were
scarcely elapsed before he discovered some symptoms of mercy, and
even of indulgence, towards the proscribed sect, which was
secretly protected by his favorite sister. The exiles were
recalled, and Eusebius, who gradually resumed his influence over
the mind of Constantine, was restored to the episcopal throne,
from which he had been ignominiously degraded. Arius himself was
treated by the whole court with the respect which would have been
due to an innocent and oppressed man. His faith was approved by
the synod of Jerusalem; and the emperor seemed impatient to
repair his injustice, by issuing an absolute command, that he
should be solemnly admitted to the communion in the cathedral of
Constantinople. On the same day, which had been fixed for the
triumph of Arius, he expired; and the strange and horrid
circumstances of his death might excite a suspicion, that the
orthodox saints had contributed more efficaciously than by their
prayers, to deliver the church from the most formidable of her
enemies.\textsuperscript{83} The three principal leaders of the Catholics,
Athanasius of Alexandria, Eustathius of Antioch, and Paul of
Constantinople were deposed on various f accusations, by the
sentence of numerous councils; and were afterwards banished into
distant provinces by the first of the Christian emperors, who, in
the last moments of his life, received the rites of baptism from
the Arian bishop of Nicomedia. The ecclesiastical government of
Constantine cannot be justified from the reproach of levity and
weakness. But the credulous monarch, unskilled in the stratagems
of theological warfare, might be deceived by the modest and
specious professions of the heretics, whose sentiments he never
perfectly understood; and while he protected Arius, and
persecuted Athanasius, he still considered the council of Nice as
the bulwark of the Christian faith, and the peculiar glory of his
own reign.\textsuperscript{84}

\pagenote[83]{We derive the original story from Athanasius, (tom.
i. p. 670,) who expresses some reluctance to stigmatize the
memory of the dead. He might exaggerate; but the perpetual
commerce of Alexandria and Constantinople would have rendered it
dangerous to invent. Those who press the literal narrative of the
death of Arius (his bowels suddenly burst out in a privy) must
make their option between \textit{poison} and \textit{miracle}.}

\pagenote[84]{The change in the sentiments, or at least in the
conduct, of Constantine, may be traced in Eusebius, (in Vit.
Constant. l. iii. c. 23, l. iv. c. 41,) Socrates, (l. i. c.
23-39,) Sozomen, (l. ii. c. 16-34,) Theodoret, (l. i. c. 14-34,)
and Philostorgius, (l. ii. c. 1-17.) But the first of these
writers was too near the scene of action, and the others were too
remote from it. It is singular enough, that the important task of
continuing the history of the church should have been left for
two laymen and a heretic.}

The sons of Constantine must have been admitted from their
childhood into the rank of catechumens; but they imitated, in the
delay of their baptism, the example of their father. Like him
they presumed to pronounce their judgment on mysteries into which
they had never been regularly initiated;\textsuperscript{85} and the fate of the
Trinitarian controversy depended, in a great measure, on the
sentiments of Constantius; who inherited the provinces of the
East, and acquired the possession of the whole empire. The Arian
presbyter or bishop, who had secreted for his use the testament
of the deceased emperor, improved the fortunate occasion which
had introduced him to the familiarity of a prince, whose public
counsels were always swayed by his domestic favorites. The
eunuchs and slaves diffused the spiritual poison through the
palace, and the dangerous infection was communicated by the
female attendants to the guards, and by the empress to her
unsuspicious husband.\textsuperscript{86} The partiality which Constantius always
expressed towards the Eusebian faction, was insensibly fortified
by the dexterous management of their leaders; and his victory
over the tyrant Magnentius increased his inclination, as well as
ability, to employ the arms of power in the cause of Arianism.
While the two armies were engaged in the plains of Mursa, and the
fate of the two rivals depended on the chance of war, the son of
Constantine passed the anxious moments in a church of the martyrs
under the walls of the city. His spiritual comforter, Valens, the
Arian bishop of the diocese, employed the most artful precautions
to obtain such early intelligence as might secure either his
favor or his escape. A secret chain of swift and trusty
messengers informed him of the vicissitudes of the battle; and
while the courtiers stood trembling round their affrighted
master, Valens assured him that the Gallic legions gave way; and
insinuated with some presence of mind, that the glorious event
had been revealed to him by an angel. The grateful emperor
ascribed his success to the merits and intercession of the bishop
of Mursa, whose faith had deserved the public and miraculous
approbation of Heaven.\textsuperscript{87} The Arians, who considered as their own
the victory of Constantius, preferred his glory to that of his
father.\textsuperscript{88} Cyril, bishop of Jerusalem, immediately composed the
description of a celestial cross, encircled with a splendid
rainbow; which during the festival of Pentecost, about the third
hour of the day, had appeared over the Mount of Olives, to the
edification of the devout pilgrims, and the people of the holy
city.\textsuperscript{89} The size of the meteor was gradually magnified; and the
Arian historian has ventured to affirm, that it was conspicuous
to the two armies in the plains of Pannonia; and that the tyrant,
who is purposely represented as an idolater, fled before the
auspicious sign of orthodox Christianity.\textsuperscript{90}

\pagenote[85]{Quia etiam tum catechumenus sacramentum fidei
merito videretiu potuisse nescire. Sulp. Sever. Hist. Sacra, l.
ii. p. 410.}

\pagenote[86]{Socrates, l. ii. c. 2. Sozomen, l. iii. c. 18.
Athanas. tom. i. p. 813, 834. He observes that the eunuchs are
the natural enemies of the \textit{Son}. Compare Dr. Jortin’s Remarks on
Ecclesiastical History, vol. iv. p. 3 with a certain genealogy in
\textit{Candide}, (ch. iv.,) which ends with one of the first companions
of Christopher Columbus.}

\pagenote[87]{Sulpicius Severus in Hist. Sacra, l. ii. p. 405, 406.}

\pagenote[88]{Cyril (apud Baron. A. D. 353, No. 26) expressly
observes that in the reign of Constantine, the cross had been
found in the bowels of the earth; but that it had appeared, in
the reign of Constantius, in the midst of the heavens. This
opposition evidently proves, that Cyril was ignorant of the
stupendous miracle to which the conversion of Constantine is
attributed; and this ignorance is the more surprising, since it
was no more than twelve years after his death that Cyril was
consecrated bishop of Jerusalem, by the immediate successor of
Eusebius of Cæsarea. See Tillemont, Mém. Eccles. tom. viii. p. 715.}

\pagenote[89]{It is not easy to determine how far the ingenuity
of Cyril might be assisted by some natural appearances of a solar halo.}

\pagenote[90]{Philostorgius, l. iii. c. 26. He is followed by the
author of the Alexandrian Chronicle, by Cedrenus, and by
Nicephorus. (See Gothofred. Dissert. p. 188.) They could not
refuse a miracle, even from the hand of an enemy.}

The sentiments of a judicious stranger, who has impartially
considered the progress of civil or ecclesiastical discord, are
always entitled to our notice; and a short passage of Ammianus,
who served in the armies, and studied the character of
Constantius, is perhaps of more value than many pages of
theological invectives. “The Christian religion, which, in
itself,” says that moderate historian, “is plain and simple, \textit{he}
confounded by the dotage of superstition. Instead of reconciling
the parties by the weight of his authority, he cherished and
promulgated, by verbal disputes, the differences which his vain
curiosity had excited. The highways were covered with troops of
bishops galloping from every side to the assemblies, which they
call synods; and while they labored to reduce the whole sect to
their own particular opinions, the public establishment of the
posts was almost ruined by their hasty and repeated journeys.”\textsuperscript{91}
Our more intimate knowledge of the ecclesiastical transactions of
the reign of Constantius would furnish an ample commentary on
this remarkable passage, which justifies the rational
apprehensions of Athanasius, that the restless activity of the
clergy, who wandered round the empire in search of the true
faith, would excite the contempt and laughter of the unbelieving
world.\textsuperscript{92} As soon as the emperor was relieved from the terrors of
the civil war, he devoted the leisure of his winter quarters at
Arles, Milan, Sirmium, and Constantinople, to the amusement or
toils of controversy: the sword of the magistrate, and even of
the tyrant, was unsheathed, to enforce the reasons of the
theologian; and as he opposed the orthodox faith of Nice, it is
readily confessed that his incapacity and ignorance were equal to
his presumption.\textsuperscript{93} The eunuchs, the women, and the bishops, who
governed the vain and feeble mind of the emperor, had inspired
him with an insuperable dislike to the Homoousion; but his timid
conscience was alarmed by the impiety of Ætius. The guilt of that
atheist was aggravated by the suspicious favor of the unfortunate
Gallus; and even the death of the Imperial ministers, who had
been massacred at Antioch, were imputed to the suggestions of
that dangerous sophist. The mind of Constantius, which could
neither be moderated by reason, nor fixed by faith, was blindly
impelled to either side of the dark and empty abyss, by his
horror of the opposite extreme; he alternately embraced and
condemned the sentiments, he successively banished and recalled
the leaders, of the Arian and Semi-Arian factions.\textsuperscript{94} During the
season of public business or festivity, he employed whole days,
and even nights, in selecting the words, and weighing the
syllables, which composed his fluctuating creeds. The subject of
his meditations still pursued and occupied his slumbers: the
incoherent dreams of the emperor were received as celestial
visions, and he accepted with complacency the lofty title of
bishop of bishops, from those ecclesiastics who forgot the
interest of their order for the gratification of their passions.
The design of establishing a uniformity of doctrine, which had
engaged him to convene so many synods in Gaul, Italy, Illyricum,
and Asia, was repeatedly baffled by his own levity, by the
divisions of the Arians, and by the resistance of the Catholics;
and he resolved, as the last and decisive effort, imperiously to
dictate the decrees of a general council. The destructive
earthquake of Nicomedia, the difficulty of finding a convenient
place, and perhaps some secret motives of policy, produced an
alteration in the summons. The bishops of the East were directed
to meet at Seleucia, in Isauria; while those of the West held
their deliberations at Rimini, on the coast of the Hadriatic; and
instead of two or three deputies from each province, the whole
episcopal body was ordered to march. The Eastern council, after
consuming four days in fierce and unavailing debate, separated
without any definitive conclusion. The council of the West was
protracted till the seventh month. Taurus, the Prætorian præfect
was instructed not to dismiss the prelates till they should all
be united in the same opinion; and his efforts were supported by
the power of banishing fifteen of the most refractory, and a
promise of the consulship if he achieved so difficult an
adventure. His prayers and threats, the authority of the
sovereign, the sophistry of Valens and Ursacius, the distress of
cold and hunger, and the tedious melancholy of a hopeless exile,
at length extorted the reluctant consent of the bishops of
Rimini. The deputies of the East and of the West attended the
emperor in the palace of Constantinople, and he enjoyed the
satisfaction of imposing on the world a profession of faith which
established the \textit{likeness}, without expressing the
\textit{consubstantiality}, of the Son of God.\textsuperscript{95} But the triumph of
Arianism had been preceded by the removal of the orthodox clergy,
whom it was impossible either to intimidate or to corrupt; and
the reign of Constantius was disgraced by the unjust and
ineffectual persecution of the great Athanasius.

\pagenote[91]{So curious a passage well deserves to be
transcribed. Christianam religionem absolutam et simplicem, anili
superstitione confundens; in qua scrutanda perplexius, quam
componenda gravius excitaret discidia plurima; quæ progressa
fusius aluit concertatione verborum, ut catervis antistium
jumentis publicis ultro citroque discarrentibus, per synodos
(quas appellant) dum ritum omnem ad suum sahere conantur
(Valesius reads \textit{conatur}) rei vehiculariæ concideret servos.
Ammianus, xxi. 16.}

\pagenote[92]{Athanas. tom. i. p. 870.}

\pagenote[93]{Socrates, l. ii. c. 35-47. Sozomen, l. iv. c.
12-30. Theodore li. c. 18-32. Philostorg. l. iv. c. 4—12, l. v.
c. 1-4, l. vi. c. 1-5}

\pagenote[94]{Sozomen, l. iv. c. 23. Athanas. tom. i. p. 831.
Tillemont (Mem Eccles. tom. vii. p. 947) has collected several
instances of the haughty fanaticism of Constantius from the
detached treatises of Lucifer of Cagliari. The very titles of
these treaties inspire zeal and terror; “Moriendum pro Dei
Filio.” “De Regibus Apostaticis.” “De non conveniendo cum
Hæretico.” “De non parcendo in Deum delinquentibus.”}

\pagenote[95]{Sulp. Sever. Hist. Sacra, l. ii. p. 418-430. The
Greek historians were very ignorant of the affairs of the West.}

We have seldom an opportunity of observing, either in active or
speculative life, what effect may be produced, or what obstacles
may be surmounted, by the force of a single mind, when it is
inflexibly applied to the pursuit of a single object. The
immortal name of Athanasius\textsuperscript{96} will never be separated from the
Catholic doctrine of the Trinity, to whose defence he consecrated
every moment and every faculty of his being. Educated in the
family of Alexander, he had vigorously opposed the early progress
of the Arian heresy: he exercised the important functions of
secretary under the aged prelate; and the fathers of the Nicene
council beheld with surprise and respect the rising virtues of
the young deacon. In a time of public danger, the dull claims of
age and of rank are sometimes superseded; and within five months
after his return from Nice, the deacon Athanasius was seated on
the archiepiscopal throne of Egypt. He filled that eminent
station above forty-six years, and his long administration was
spent in a perpetual combat against the powers of Arianism. Five
times was Athanasius expelled from his throne; twenty years he
passed as an exile or a fugitive: and almost every province of
the Roman empire was successively witness to his merit, and his
sufferings in the cause of the Homoousion, which he considered as
the sole pleasure and business, as the duty, and as the glory of
his life. Amidst the storms of persecution, the archbishop of
Alexandria was patient of labor, jealous of fame, careless of
safety; and although his mind was tainted by the contagion of
fanaticism, Athanasius displayed a superiority of character and
abilities, which would have qualified him, far better than the
degenerate sons of Constantine, for the government of a great
monarchy. His learning was much less profound and extensive than
that of Eusebius of Cæsarea, and his rude eloquence could not be
compared with the polished oratory of Gregory of Basil; but
whenever the primate of Egypt was called upon to justify his
sentiments, or his conduct, his unpremeditated style, either of
speaking or writing, was clear, forcible, and persuasive. He has
always been revered, in the orthodox school, as one of the most
accurate masters of the Christian theology; and he was supposed
to possess two profane sciences, less adapted to the episcopal
character, the knowledge of jurisprudence,\textsuperscript{97} and that of
divination.\textsuperscript{98} Some fortunate conjectures of future events, which
impartial reasoners might ascribe to the experience and judgment
of Athanasius, were attributed by his friends to heavenly
inspiration, and imputed by his enemies to infernal magic.

\pagenote[96]{We may regret that Gregory Nazianzen composed a
panegyric instead of a life of Athanasius; but we should enjoy
and improve the advantage of drawing our most authentic materials
from the rich fund of his own epistles and apologies, (tom. i. p.
670-951.) I shall not imitate the example of Socrates, (l. ii. c.
l.) who published the first edition of the history, without
giving himself the trouble to consult the writings of Athanasius.
Yet even Socrates, the more curious Sozomen, and the learned
Theodoret, connect the life of Athanasius with the series of
ecclesiastical history. The diligence of Tillemont, (tom. viii,)
and of the Benedictine editors, has collected every fact, and
examined every difficulty}

\pagenote[97]{Sulpicius Severus (Hist. Sacra, l. ii. p. 396)
calls him a lawyer, a jurisconsult. This character cannot now be
discovered either in the life or writings of Athanasius.}

\pagenote[98]{Dicebatur enim fatidicarum sortium fidem, quæve
augurales portenderent alites scientissime callens aliquoties
prædixisse futura. Ammianus, xv. 7. A prophecy, or rather a joke,
is related by Sozomen, (l. iv c. 10,) which evidently proves (if
the crows speak Latin) that Athanasius understood the language of
the crows.}

But as Athanasius was continually engaged with the prejudices and
passions of every order of men, from the monk to the emperor, the
knowledge of human nature was his first and most important
science. He preserved a distinct and unbroken view of a scene
which was incessantly shifting; and never failed to improve those
decisive moments which are irrecoverably past before they are
perceived by a common eye. The archbishop of Alexandria was
capable of distinguishing how far he might boldly command, and
where he must dexterously insinuate; how long he might contend
with power, and when he must withdraw from persecution; and while
he directed the thunders of the church against heresy and
rebellion, he could assume, in the bosom of his own party, the
flexible and indulgent temper of a prudent leader. The election
of Athanasius has not escaped the reproach of irregularity and
precipitation;\textsuperscript{99} but the propriety of his behavior conciliated
the affections both of the clergy and of the people. The
Alexandrians were impatient to rise in arms for the defence of an
eloquent and liberal pastor. In his distress he always derived
support, or at least consolation, from the faithful attachment of
his parochial clergy; and the hundred bishops of Egypt adhered,
with unshaken zeal, to the cause of Athanasius. In the modest
equipage which pride and policy would affect, he frequently
performed the episcopal visitation of his provinces, from the
mouth of the Nile to the confines of Æthiopia; familiarly
conversing with the meanest of the populace, and humbly saluting
the saints and hermits of the desert.\textsuperscript{100} Nor was it only in
ecclesiastical assemblies, among men whose education and manners
were similar to his own, that Athanasius displayed the ascendancy
of his genius. He appeared with easy and respectful firmness in
the courts of princes; and in the various turns of his prosperous
and adverse fortune he never lost the confidence of his friends,
or the esteem of his enemies.

\pagenote[99]{The irregular ordination of Athanasius was slightly
mentioned in the councils which were held against him. See
Philostorg. l. ii. c. 11, and Godefroy, p. 71; but it can
scarcely be supposed that the assembly of the bishops of Egypt
would solemnly attest a \textit{public} falsehood. Athanas. tom. i. p.
726.}

\pagenote[100]{See the history of the Fathers of the Desert,
published by Rosweide; and Tillemont, Mém. Eccles. tom. vii., in
the lives of Antony, Pachomius, \&c. Athanasius himself, who did
not disdain to compose the life of his friend Antony, has
carefully observed how often the holy monk deplored and
prophesied the mischiefs of the Arian heresy Athanas. tom. ii. p.
492, 498, \&c.}

In his youth, the primate of Egypt resisted the great
Constantine, who had repeatedly signified his will, that Arius
should be restored to the Catholic communion.\textsuperscript{101} The emperor
respected, and might forgive, this inflexible resolution; and the
faction who considered Athanasius as their most formidable enemy,
was constrained to dissemble their hatred, and silently to
prepare an indirect and distant assault. They scattered rumors
and suspicions, represented the archbishop as a proud and
oppressive tyrant, and boldly accused him of violating the treaty
which had been ratified in the Nicene council, with the
schismatic followers of Meletius.\textsuperscript{102} Athanasius had openly
disapproved that ignominious peace, and the emperor was disposed
to believe that he had abused his ecclesiastical and civil power,
to prosecute those odious sectaries: that he had sacrilegiously
broken a chalice in one of their churches of Mareotis; that he
had whipped or imprisoned six of their bishops; and that
Arsenius, a seventh bishop of the same party, had been murdered,
or at least mutilated, by the cruel hand of the primate.\textsuperscript{103}
These charges, which affected his honor and his life, were
referred by Constantine to his brother Dalmatius the censor, who
resided at Antioch; the synods of Cæsarea and Tyre were
successively convened; and the bishops of the East were
instructed to judge the cause of Athanasius, before they
proceeded to consecrate the new church of the Resurrection at
Jerusalem. The primate might be conscious of his innocence; but
he was sensible that the same implacable spirit which had
dictated the accusation, would direct the proceeding, and
pronounce the sentence. He prudently declined the tribunal of his
enemies; despised the summons of the synod of Cæsarea; and, after
a long and artful delay, submitted to the peremptory commands of
the emperor, who threatened to punish his criminal disobedience
if he refused to appear in the council of Tyre.\textsuperscript{104} Before
Athanasius, at the head of fifty Egyptian prelates, sailed from
Alexandria, he had wisely secured the alliance of the Meletians;
and Arsenius himself, his imaginary victim, and his secret
friend, was privately concealed in his train. The synod of Tyre
was conducted by Eusebius of Cæsarea, with more passion, and with
less art, than his learning and experience might promise; his
numerous faction repeated the names of homicide and tyrant; and
their clamors were encouraged by the seeming patience of
Athanasius, who expected the decisive moment to produce Arsenius
alive and unhurt in the midst of the assembly. The nature of the
other charges did not admit of such clear and satisfactory
replies; yet the archbishop was able to prove, that in the
village, where he was accused of breaking a consecrated chalice,
neither church nor altar nor chalice could really exist.

The Arians, who had secretly determined the guilt and
condemnation of their enemy, attempted, however, to disguise
their injustice by the imitation of judicial forms: the synod
appointed an episcopal commission of six delegates to collect
evidence on the spot; and this measure which was vigorously
opposed by the Egyptian bishops, opened new scenes of violence
and perjury.\textsuperscript{105} After the return of the deputies from
Alexandria, the majority of the council pronounced the final
sentence of degradation and exile against the primate of Egypt.
The decree, expressed in the fiercest language of malice and
revenge, was communicated to the emperor and the Catholic church;
and the bishops immediately resumed a mild and devout aspect,
such as became their holy pilgrimage to the Sepulchre of Christ.\textsuperscript{106}

\pagenote[101]{At first Constantine threatened in \textit{speaking}, but
requested in \textit{writing}. His letters gradually assumed a menacing
tone; by while he required that the entrance of the church should
be open to \textit{all}, he avoided the odious name of Arius.
Athanasius, like a skilful politician, has accurately marked
these distinctions, (tom. i. p. 788.) which allowed him some
scope for excuse and delay}

\pagenote[102]{The Meletians in Egypt, like the Donatists in
Africa, were produced by an episcopal quarrel which arose from
the persecution. I have not leisure to pursue the obscure
controversy, which seems to have been misrepresented by the
partiality of Athanasius and the ignorance of Epiphanius. See
Mosheim’s General History of the Church, vol. i. p. 201.}

\pagenote[103]{The treatment of the six bishops is specified by
Sozomen, (l. ii. c. 25;) but Athanasius himself, so copious on
the subject of Arsenius and the chalice, leaves this grave
accusation without a reply. Note: This grave charge, if made,
(and it rests entirely on the authority of Soz omen,) seems to
have been silently dropped by the parties themselves: it is never
alluded to in the subsequent investigations. From Sozomen
himself, who gives the unfavorable report of the commission of
inquiry sent to Egypt concerning the cup. it does not appear that
they noticed this accusation of personal violence.—M}

\pagenote[104]{Athanas, tom. i. p. 788. Socrates, l. i.c. 28.
Sozomen, l. ii. c 25. The emperor, in his Epistle of Convocation,
(Euseb. in Vit. Constant. l. iv. c. 42,) seems to prejudge some
members of the clergy and it was more than probable that the
synod would apply those reproaches to Athanasius.}

\pagenote[105]{See, in particular, the second Apology of
Athanasius, (tom. i. p. 763-808,) and his Epistles to the Monks,
(p. 808-866.) They are justified by original and authentic
documents; but they would inspire more confidence if he appeared
less innocent, and his enemies less absurd.}

\pagenote[106]{Eusebius in Vit. Constantin. l. iv. c. 41-47.}

\section{Part \thesection.}

But the injustice of these ecclesiastical judges had not been
countenanced by the submission, or even by the presence, of
Athanasius. He resolved to make a bold and dangerous experiment,
whether the throne was inaccessible to the voice of truth; and
before the final sentence could be pronounced at Tyre, the
intrepid primate threw himself into a bark which was ready to
hoist sail for the Imperial city. The request of a formal
audience might have been opposed or eluded; but Athanasius
concealed his arrival, watched the moment of Constantine’s return
from an adjacent villa, and boldly encountered his angry
sovereign as he passed on horseback through the principal street
of Constantinople. So strange an apparition excited his surprise
and indignation; and the guards were ordered to remove the
importunate suitor; but his resentment was subdued by involuntary
respect; and the haughty spirit of the emperor was awed by the
courage and eloquence of a bishop, who implored his justice and
awakened his conscience.\textsuperscript{107} Constantine listened to the
complaints of Athanasius with impartial and even gracious
attention; the members of the synod of Tyre were summoned to
justify their proceedings; and the arts of the Eusebian faction
would have been confounded, if they had not aggravated the guilt
of the primate, by the dexterous supposition of an unpardonable
offence; a criminal design to intercept and detain the corn-fleet
of Alexandria, which supplied the subsistence of the new capital.\textsuperscript{108}
The emperor was satisfied that the peace of Egypt would be
secured by the absence of a popular leader; but he refused to
fill the vacancy of the archiepiscopal throne; and the sentence,
which, after long hesitation, he pronounced, was that of a
jealous ostracism, rather than of an ignominious exile. In the
remote province of Gaul, but in the hospitable court of Treves,
Athanasius passed about twenty eight months. The death of the
emperor changed the face of public affairs and, amidst the
general indulgence of a young reign, the primate was restored to
his country by an honorable edict of the younger Constantine, who
expressed a deep sense of the innocence and merit of his
venerable guest.\textsuperscript{109}

\pagenote[107]{Athanas. tom. i. p. 804. In a church dedicated to
St. Athanasius this situation would afford a better subject for a
picture, than most of the stories of miracles and martyrdoms.}

\pagenote[108]{Athanas. tom. i. p. 729. Eunapius has related (in
Vit. Sophist. p. 36, 37, edit. Commelin) a strange example of the
cruelty and credulity of Constantine on a similar occasion. The
eloquent Sopater, a Syrian philosopher, enjoyed his friendship,
and provoked the resentment of Ablavius, his Prætorian præfect.
The corn-fleet was detained for want of a south wind; the people
of Constantinople were discontented; and Sopater was beheaded, on
a charge that he had \textit{bound} the winds by the power of magic.
Suidas adds, that Constantine wished to prove, by this execution,
that he had absolutely renounced the superstition of the
Gentiles.}

\pagenote[109]{In his return he saw Constantius twice, at
Viminiacum, and at Cæsarea in Cappadocia, (Athanas. tom. i. p.
676.) Tillemont supposes that Constantine introduced him to the
meeting of the three royal brothers in Pannonia, (Mémoires
Eccles. tom. viii. p. 69.)}

The death of that prince exposed Athanasius to a second
persecution; and the feeble Constantius, the sovereign of the
East, soon became the secret accomplice of the Eusebians. Ninety
bishops of that sect or faction assembled at Antioch, under the
specious pretence of dedicating the cathedral. They composed an
ambiguous creed, which is faintly tinged with the colors of
Semi-Arianism, and twenty-five canons, which still regulate the
discipline of the orthodox Greeks.\textsuperscript{110} It was decided, with some
appearance of equity, that a bishop, deprived by a synod, should
not resume his episcopal functions till he had been absolved by
the judgment of an equal synod; the law was immediately applied
to the case of Athanasius; the council of Antioch pronounced, or
rather confirmed, his degradation: a stranger, named Gregory, was
seated on his throne; and Philagrius,\textsuperscript{111} the præfect of Egypt,
was instructed to support the new primate with the civil and
military powers of the province. Oppressed by the conspiracy of
the Asiatic prelates, Athanasius withdrew from Alexandria, and
passed three years\textsuperscript{112} as an exile and a suppliant on the holy
threshold of the Vatican.\textsuperscript{113} By the assiduous study of the Latin
language, he soon qualified himself to negotiate with the western
clergy; his decent flattery swayed and directed the haughty
Julius; the Roman pontiff was persuaded to consider his appeal as
the peculiar interest of the Apostolic see: and his innocence was
unanimously declared in a council of fifty bishops of Italy. At
the end of three years, the primate was summoned to the court of
Milan by the emperor Constans, who, in the indulgence of unlawful
pleasures, still professed a lively regard for the orthodox
faith. The cause of truth and justice was promoted by the
influence of gold,\textsuperscript{114} and the ministers of Constans advised
their sovereign to require the convocation of an ecclesiastical
assembly, which might act as the representatives of the Catholic
church. Ninety-four bishops of the West, seventy-six bishops of
the East, encountered each other at Sardica, on the verge of the
two empires, but in the dominions of the protector of Athanasius.
Their debates soon degenerated into hostile altercations; the
Asiatics, apprehensive for their personal safety, retired to
Philippopolis in Thrace; and the rival synods reciprocally hurled
their spiritual thunders against their enemies, whom they piously
condemned as the enemies of the true God. Their decrees were
published and ratified in their respective provinces: and
Athanasius, who in the West was revered as a saint, was exposed
as a criminal to the abhorrence of the East.\textsuperscript{115} The council of
Sardica reveals the first symptoms of discord and schism between
the Greek and Latin churches which were separated by the
accidental difference of faith, and the permanent distinction of
language.

\pagenote[110]{See Beveridge, Pandect. tom. i. p. 429-452, and
tom. ii. Annotation. p. 182. Tillemont, Mém. Eccles. tom. vi. p.
310-324. St. Hilary of Poitiers has mentioned this synod of
Antioch with too much favor and respect. He reckons ninety-seven
bishops.}

\pagenote[111]{This magistrate, so odious to Athanasius, is
praised by Gregory Nazianzen, tom. i. Orat. xxi. p. 390, 391.

Sæpe premente Deo fert Deus alter opem.

For the credit of human nature, I am always pleased to discover
some good qualities in those men whom party has represented as
tyrants and monsters.}

\pagenote[112]{The chronological difficulties which perplex the
residence of Athanasius at Rome, are strenuously agitated by
Valesius (Observat ad Calcem, tom. ii. Hist. Eccles. l. i. c.
1-5) and Tillemont, (Men: Eccles. tom. viii. p. 674, \&c.) I have
followed the simple hypothesis of Valesius, who allows only one
journey, after the intrusion Gregory.}

\pagenote[113]{I cannot forbear transcribing a judicious
observation of Wetstein, (Prolegomen. N.S. p. 19: ) Si tamen
Historiam Ecclesiasticam velimus consulere, patebit jam inde a
seculo quarto, cum, ortis controversiis, ecclesiæ Græciæ doctores
in duas partes scinderentur, ingenio, eloquentia, numero, tantum
non æquales, eam partem quæ vincere cupiebat Romam confugisse,
majestatemque pontificis comiter coluisse, eoque pacto oppressis
per pontificem et episcopos Latinos adversariis prævaluisse,
atque orthodoxiam in conciliis stabilivisse. Eam ob causam
Athanasius, non sine comitatu, Roman petiit, pluresque annos ibi hæsit.}

\pagenote[114]{Philostorgius, l. iii. c. 12. If any corruption
was used to promote the interest of religion, an advocate of
Athanasius might justify or excuse this questionable conduct, by
the example of Cato and Sidney; the former of whom is \textit{said} to
have given, and the latter to have received, a bribe in the cause
of liberty.}

\pagenote[115]{The canon which allows appeals to the Roman
pontiffs, has almost raised the council of Sardica to the dignity
of a general council; and its acts have been ignorantly or
artfully confounded with those of the Nicene synod. See
Tillemont, tom. vii. p. 689, and Geddos’s Tracts, vol. ii. p. 419-460.}

During his second exile in the West, Athanasius was frequently
admitted to the Imperial presence; at Capua, Lodi, Milan, Verona,
Padua, Aquileia, and Treves. The bishop of the diocese usually
assisted at these interviews; the master of the offices stood
before the veil or curtain of the sacred apartment; and the
uniform moderation of the primate might be attested by these
respectable witnesses, to whose evidence he solemnly appeals.\textsuperscript{116}
Prudence would undoubtedly suggest the mild and respectful tone
that became a subject and a bishop. In these familiar conferences
with the sovereign of the West, Athanasius might lament the error
of Constantius, but he boldly arraigned the guilt of his eunuchs
and his Arian prelates; deplored the distress and danger of the
Catholic church; and excited Constans to emulate the zeal and
glory of his father. The emperor declared his resolution of
employing the troops and treasures of Europe in the orthodox
cause; and signified, by a concise and peremptory epistle to his
brother Constantius, that unless he consented to the immediate
restoration of Athanasius, he himself, with a fleet and army,
would seat the archbishop on the throne of Alexandria.\textsuperscript{117} But
this religious war, so horrible to nature, was prevented by the
timely compliance of Constantius; and the emperor of the East
condescended to solicit a reconciliation with a subject whom he
had injured. Athanasius waited with decent pride, till he had
received three successive epistles full of the strongest
assurances of the protection, the favor, and the esteem of his
sovereign; who invited him to resume his episcopal seat, and who
added the humiliating precaution of engaging his principal
ministers to attest the sincerity of his intentions. They were
manifested in a still more public manner, by the strict orders
which were despatched into Egypt to recall the adherents of
Athanasius, to restore their privileges, to proclaim their
innocence, and to erase from the public registers the illegal
proceedings which had been obtained during the prevalence of the
Eusebian faction. After every satisfaction and security had been
given, which justice or even delicacy could require, the primate
proceeded, by slow journeys, through the provinces of Thrace,
Asia, and Syria; and his progress was marked by the abject homage
of the Oriental bishops, who excited his contempt without
deceiving his penetration.\textsuperscript{118} At Antioch he saw the emperor
Constantius; sustained, with modest firmness, the embraces and
protestations of his master, and eluded the proposal of allowing
the Arians a single church at Alexandria, by claiming, in the
other cities of the empire, a similar toleration for his own
party; a reply which might have appeared just and moderate in the
mouth of an independent prince. The entrance of the archbishop
into his capital was a triumphal procession; absence and
persecution had endeared him to the Alexandrians; his authority,
which he exercised with rigor, was more firmly established; and
his fame was diffused from Æthiopia to Britain, over the whole
extent of the Christian world.\textsuperscript{119}

\pagenote[116]{As Athanasius dispersed secret invectives against
Constantius, (see the Epistle to the Monks,) at the same time
that he assured him of his profound respect, we might distrust
the professions of the archbishop. Tom. i. p. 677.}

\pagenote[117]{Notwithstanding the discreet silence of
Athanasius, and the manifest forgery of a letter inserted by
Socrates, these menaces are proved by the unquestionable evidence
of Lucifer of Cagliari, and even of Constantius himself. See
Tillemont, tom. viii. p. 693}

\pagenote[118]{I have always entertained some doubts concerning
the retraction of Ursacius and Valens, (Athanas. tom. i. p. 776.)
Their epistles to Julius, bishop of Rome, and to Athanasius
himself, are of so different a cast from each other, that they
cannot both be genuine. The one speaks the language of criminals
who confess their guilt and infamy; the other of enemies, who
solicit on equal terms an honorable reconciliation. * Note: I
cannot quite comprehend the ground of Gibbon’s doubts. Athanasius
distinctly asserts the fact of their retractation. (Athan. Op. i.
p. 124, edit. Benedict.) The epistles are apparently translations
from the Latin, if, in fact, more than the substance of the
epistles. That to Athanasius is brief, almost abrupt. Their
retractation is likewise mentioned in the address of the orthodox
bishops of Rimini to Constantius. Athan. de Synodis, Op t. i. p
723-M.}

\pagenote[119]{The circumstances of his second return may be
collected from Athanasius himself, tom. i. p. 769, and 822, 843.
Socrates, l. ii. c. 18, Sozomen, l. iii. c. 19. Theodoret, l. ii.
c. 11, 12. Philostorgius, l. iii. c. 12.}

But the subject who has reduced his prince to the necessity of
dissembling, can never expect a sincere and lasting forgiveness;
and the tragic fate of Constans soon deprived Athanasius of a
powerful and generous protector. The civil war between the
assassin and the only surviving brother of Constans, which
afflicted the empire above three years, secured an interval of
repose to the Catholic church; and the two contending parties
were desirous to conciliate the friendship of a bishop, who, by
the weight of his personal authority, might determine the
fluctuating resolutions of an important province. He gave
audience to the ambassadors of the tyrant, with whom he was
afterwards accused of holding a secret correspondence;\textsuperscript{120} and
the emperor Constantius repeatedly assured his dearest father,
the most reverend Athanasius, that, notwithstanding the malicious
rumors which were circulated by their common enemies, he had
inherited the sentiments, as well as the throne, of his deceased
brother.\textsuperscript{121} Gratitude and humanity would have disposed the
primate of Egypt to deplore the untimely fate of Constans, and to
abhor the guilt of Magnentius; but as he clearly understood that
the apprehensions of Constantius were his only safeguard, the
fervor of his prayers for the success of the righteous cause
might perhaps be somewhat abated. The ruin of Athanasius was no
longer contrived by the obscure malice of a few bigoted or angry
bishops, who abused the authority of a credulous monarch. The
monarch himself avowed the resolution, which he had so long
suppressed, of avenging his private injuries;\textsuperscript{122} and the first
winter after his victory, which he passed at Arles, was employed
against an enemy more odious to him than the vanquished tyrant of
Gaul.

\pagenote[120]{Athanasius (tom. i. p. 677, 678) defends his
innocence by pathetic complaints, solemn assertions, and specious
arguments. He admits that letters had been forged in his name,
but he requests that his own secretaries and those of the tyrant
might be examined, whether those letters had been written by the
former, or received by the latter.}

\pagenote[121]{Athanas. tom. i. p. 825-844.}

\pagenote[122]{Athanas. tom. i. p. 861. Theodoret, l. ii. c. 16.
The emperor declared that he was more desirous to subdue
Athanasius, than he had been to vanquish Magnentius or Sylvanus.}

If the emperor had capriciously decreed the death of the most
eminent and virtuous citizen of the republic, the cruel order
would have been executed without hesitation, by the ministers of
open violence or of specious injustice. The caution, the delay,
the difficulty with which he proceeded in the condemnation and
punishment of a popular bishop, discovered to the world that the
privileges of the church had already revived a sense of order and
freedom in the Roman government. The sentence which was
pronounced in the synod of Tyre, and subscribed by a large
majority of the Eastern bishops, had never been expressly
repealed; and as Athanasius had been once degraded from his
episcopal dignity by the judgment of his brethren, every
subsequent act might be considered as irregular, and even
criminal. But the memory of the firm and effectual support which
the primate of Egypt had derived from the attachment of the
Western church, engaged Constantius to suspend the execution of
the sentence till he had obtained the concurrence of the Latin
bishops. Two years were consumed in ecclesiastical negotiations;
and the important cause between the emperor and one of his
subjects was solemnly debated, first in the synod of Arles, and
afterwards in the great council of Milan,\textsuperscript{123} which consisted of
above three hundred bishops. Their integrity was gradually
undermined by the arguments of the Arians, the dexterity of the
eunuchs, and the pressing solicitations of a prince who gratified
his revenge at the expense of his dignity, and exposed his own
passions, whilst he influenced those of the clergy. Corruption,
the most infallible symptom of constitutional liberty, was
successfully practised; honors, gifts, and immunities were
offered and accepted as the price of an episcopal vote;\textsuperscript{124} and
the condemnation of the Alexandrian primate was artfully
represented as the only measure which could restore the peace and
union of the Catholic church. The friends of Athanasius were not,
however, wanting to their leader, or to their cause. With a manly
spirit, which the sanctity of their character rendered less
dangerous, they maintained, in public debate, and in private
conference with the emperor, the eternal obligation of religion
and justice. They declared, that neither the hope of his favor,
nor the fear of his displeasure, should prevail on them to join
in the condemnation of an absent, an innocent, a respectable
brother.\textsuperscript{125} They affirmed, with apparent reason, that the
illegal and obsolete decrees of the council of Tyre had long
since been tacitly abolished by the Imperial edicts, the
honorable reestablishment of the archbishop of Alexandria, and
the silence or recantation of his most clamorous adversaries.
They alleged, that his innocence had been attested by the
unanimous bishops of Egypt, and had been acknowledged in the
councils of Rome and Sardica,\textsuperscript{126} by the impartial judgment of
the Latin church. They deplored the hard condition of Athanasius,
who, after enjoying so many years his seat, his reputation, and
the seeming confidence of his sovereign, was again called upon to
confute the most groundless and extravagant accusations. Their
language was specious; their conduct was honorable: but in this
long and obstinate contest, which fixed the eyes of the whole
empire on a single bishop, the ecclesiastical factions were
prepared to sacrifice truth and justice to the more interesting
object of defending or removing the intrepid champion of the
Nicene faith. The Arians still thought it prudent to disguise, in
ambiguous language, their real sentiments and designs; but the
orthodox bishops, armed with the favor of the people, and the
decrees of a general council, insisted on every occasion, and
particularly at Milan, that their adversaries should purge
themselves from the suspicion of heresy, before they presumed to
arraign the conduct of the great Athanasius.\textsuperscript{127}

\pagenote[123]{The affairs of the council of Milan are so
imperfectly and erroneously related by the Greek writers, that we
must rejoice in the supply of some letters of Eusebius, extracted
by Baronius from the archives of the church of Vercellæ, and of
an old life of Dionysius of Milan, published by Bollandus. See
Baronius, A.D. 355, and Tillemont, tom. vii. p. 1415.}

\pagenote[124]{The honors, presents, feasts, which seduced so
many bishops, are mentioned with indignation by those who were
too pure or too proud to accept them. “We combat (says Hilary of
Poitiers) against Constantius the Antichrist; who strokes the
belly instead of scourging the back;” qui non dorsa cædit; sed
ventrem palpat. Hilarius contra Constant c. 5, p. 1240.}

\pagenote[125]{Something of this opposition is mentioned by
Ammianus (x. 7,) who had a very dark and superficial knowledge of
ecclesiastical history. Liberius... perseveranter renitebatur,
nec visum hominem, nec auditum damnare, nefas ultimum sæpe
exclamans; aperte scilicet recalcitrans Imperatoris arbitrio. Id
enim ille Athanasio semper infestus, \&c.}

\pagenote[126]{More properly by the orthodox part of the council
of Sardica. If the bishops of both parties had fairly voted, the
division would have been 94 to 76. M. de Tillemont (see tom.
viii. p. 1147-1158) is justly surprised that so small a majority
should have proceeded as vigorously against their adversaries,
the principal of whom they immediately deposed.}

\pagenote[127]{Sulp. Severus in Hist. Sacra, l. ii. p. 412.}

But the voice of reason (if reason was indeed on the side of
Athanasius) was silenced by the clamors of a factious or venal
majority; and the councils of Arles and Milan were not dissolved,
till the archbishop of Alexandria had been solemnly condemned and
deposed by the judgment of the Western, as well as of the
Eastern, church. The bishops who had opposed, were required to
subscribe, the sentence, and to unite in religious communion with
the suspected leaders of the adverse party. A formulary of
consent was transmitted by the messengers of state to the absent
bishops: and all those who refused to submit their private
opinion to the public and inspired wisdom of the councils of
Arles and Milan, were immediately banished by the emperor, who
affected to execute the decrees of the Catholic church. Among
those prelates who led the honorable band of confessors and
exiles, Liberius of Rome, Osius of Cordova, Paulinus of Treves,
Dionysius of Milan, Eusebius of Vercellæ, Lucifer of Cagliari and
Hilary of Poitiers, may deserve to be particularly distinguished.
The eminent station of Liberius, who governed the capital of the
empire; the personal merit and long experience of the venerable
Osius, who was revered as the favorite of the great Constantine,
and the father of the Nicene faith, placed those prelates at the
head of the Latin church: and their example, either of submission
or resistance, would probable be imitated by the episcopal crowd.
But the repeated attempts of the emperor to seduce or to
intimidate the bishops of Rome and Cordova, were for some time
ineffectual. The Spaniard declared himself ready to suffer under
Constantius, as he had suffered threescore years before under his
grandfather Maximian. The Roman, in the presence of his
sovereign, asserted the innocence of Athanasius and his own
freedom. When he was banished to Beræa in Thrace, he sent back a
large sum which had been offered for the accommodation of his
journey; and insulted the court of Milan by the haughty remark,
that the emperor and his eunuchs might want that gold to pay
their soldiers and their bishops.\textsuperscript{128} The resolution of Liberius
and Osius was at length subdued by the hardships of exile and
confinement. The Roman pontiff purchased his return by some
criminal compliances; and afterwards expiated his guilt by a
seasonable repentance. Persuasion and violence were employed to
extort the reluctant signature of the decrepit bishop of Cordova,
whose strength was broken, and whose faculties were perhaps
impaired by the weight of a hundred years; and the insolent
triumph of the Arians provoked some of the orthodox party to
treat with inhuman severity the character, or rather the memory,
of an unfortunate old man, to whose former services Christianity
itself was so deeply indebted.\textsuperscript{129}

\pagenote[128]{The exile of Liberius is mentioned by Ammianus,
xv. 7. See Theodoret, l. ii. c. 16. Athanas. tom. i. p. 834-837.
Hilar. Fragment l.}

\pagenote[129]{The life of Osius is collected by Tillemont, (tom.
vii. p. 524-561,) who in the most extravagant terms first
admires, and then reprobates, the bishop of Cordova. In the midst
of their lamentations on his fall, the prudence of Athanasius may
be distinguished from the blind and intemperate zeal of Hilary.}

The fall of Liberius and Osius reflected a brighter lustre on the
firmness of those bishops who still adhered, with unshaken
fidelity, to the cause of Athanasius and religious truth. The
ingenious malice of their enemies had deprived them of the
benefit of mutual comfort and advice, separated those illustrious
exiles into distant provinces, and carefully selected the most
inhospitable spots of a great empire.\textsuperscript{130} Yet they soon
experienced that the deserts of Libya, and the most barbarous
tracts of Cappadocia, were less inhospitable than the residence
of those cities in which an Arian bishop could satiate, without
restraint, the exquisite rancor of theological hatred.\textsuperscript{131} Their
consolation was derived from the consciousness of rectitude and
independence, from the applause, the visits, the letters, and the
liberal alms of their adherents,\textsuperscript{132} and from the satisfaction
which they soon enjoyed of observing the intestine divisions of
the adversaries of the Nicene faith. Such was the nice and
capricious taste of the emperor Constantius; and so easily was he
offended by the slightest deviation from his imaginary standard
of Christian truth, that he persecuted, with equal zeal, those
who defended the \textit{consubstantiality}, those who asserted the
\textit{similar substance}, and those who denied the \textit{likeness} of the
Son of God. Three bishops, degraded and banished for those
adverse opinions, might possibly meet in the same place of exile;
and, according to the difference of their temper, might either
pity or insult the blind enthusiasm of their antagonists, whose
present sufferings would never be compensated by future
happiness.

\pagenote[130]{The confessors of the West were successively
banished to the deserts of Arabia or Thebais, the lonely places
of Mount Taurus, the wildest parts of Phrygia, which were in the
possession of the impious Montanists, \&c. When the heretic Ætius
was too favorably entertained at Mopsuestia in Cilicia, the place
of his exile was changed, by the advice of Acacius, to Amblada, a
district inhabited by savages and infested by war and pestilence.
Philostorg. l. v. c. 2.}

\pagenote[131]{See the cruel treatment and strange obstinacy of
Eusebius, in his own letters, published by Baronius, A.D. 356,
No. 92-102.}

\pagenote[132]{Cæterum exules satis constat, totius orbis studiis
celebratos pecuniasque eis in sumptum affatim congestas,
legationibus quoque plebis Catholicæ ex omnibus fere provinciis
frequentatos. Sulp. Sever Hist. Sacra, p. 414. Athanas. tom. i.
p. 836, 840.}

The disgrace and exile of the orthodox bishops of the West were
designed as so many preparatory steps to the ruin of Athanasius
himself.\textsuperscript{133} Six-and-twenty months had elapsed, during which the
Imperial court secretly labored, by the most insidious arts, to
remove him from Alexandria, and to withdraw the allowance which
supplied his popular liberality. But when the primate of Egypt,
deserted and proscribed by the Latin church, was left destitute
of any foreign support, Constantius despatched two of his
secretaries with a verbal commission to announce and execute the
order of his banishment. As the justice of the sentence was
publicly avowed by the whole party, the only motive which could
restrain Constantius from giving his messengers the sanction of a
written mandate, must be imputed to his doubt of the event; and
to a sense of the danger to which he might expose the second
city, and the most fertile province, of the empire, if the people
should persist in the resolution of defending, by force of arms,
the innocence of their spiritual father. Such extreme caution
afforded Athanasius a specious pretence respectfully to dispute
the truth of an order, which he could not reconcile, either with
the equity, or with the former declarations, of his gracious
master. The civil powers of Egypt found themselves inadequate to
the task of persuading or compelling the primate to abdicate his
episcopal throne; and they were obliged to conclude a treaty with
the popular leaders of Alexandria, by which it was stipulated,
that all proceedings and all hostilities should be suspended till
the emperor’s pleasure had been more distinctly ascertained. By
this seeming moderation, the Catholics were deceived into a false
and fatal security; while the legions of the Upper Egypt, and of
Libya, advanced, by secret orders and hasty marches, to besiege,
or rather to surprise, a capital habituated to sedition, and
inflamed by religious zeal.\textsuperscript{134} The position of Alexandria,
between the sea and the Lake Mareotis, facilitated the approach
and landing of the troops; who were introduced into the heart of
the city, before any effectual measures could be taken either to
shut the gates or to occupy the important posts of defence. At
the hour of midnight, twenty-three days after the signature of
the treaty, Syrianus, duke of Egypt, at the head of five thousand
soldiers, armed and prepared for an assault, unexpectedly
invested the church of St. Theonas, where the archbishop, with a
part of his clergy and people, performed their nocturnal
devotions. The doors of the sacred edifice yielded to the
impetuosity of the attack, which was accompanied with every
horrid circumstance of tumult and bloodshed; but, as the bodies
of the slain, and the fragments of military weapons, remained the
next day an unexceptionable evidence in the possession of the
Catholics, the enterprise of Syrianus may be considered as a
successful irruption rather than as an absolute conquest. The
other churches of the city were profaned by similar outrages;
and, during at least four months, Alexandria was exposed to the
insults of a licentious army, stimulated by the ecclesiastics of
a hostile faction. Many of the faithful were killed; who may
deserve the name of martyrs, if their deaths were neither
provoked nor revenged; bishops and presbyters were treated with
cruel ignominy; consecrated virgins were stripped naked, scourged
and violated; the houses of wealthy citizens were plundered; and,
under the mask of religious zeal, lust, avarice, and private
resentment were gratified with impunity, and even with applause.
The Pagans of Alexandria, who still formed a numerous and
discontented party, were easily persuaded to desert a bishop whom
they feared and esteemed. The hopes of some peculiar favors, and
the apprehension of being involved in the general penalties of
rebellion, engaged them to promise their support to the destined
successor of Athanasius, the famous George of Cappadocia. The
usurper, after receiving the consecration of an Arian synod, was
placed on the episcopal throne by the arms of Sebastian, who had
been appointed Count of Egypt for the execution of that important
design. In the use, as well as in the acquisition, of power, the
tyrant, George disregarded the laws of religion, of justice, and
of humanity; and the same scenes of violence and scandal which
had been exhibited in the capital, were repeated in more than
ninety episcopal cities of Egypt. Encouraged by success,
Constantius ventured to approve the conduct of his minister. By a
public and passionate epistle, the emperor congratulates the
deliverance of Alexandria from a popular tyrant, who deluded his
blind votaries by the magic of his eloquence; expatiates on the
virtues and piety of the most reverend George, the elected
bishop; and aspires, as the patron and benefactor of the city to
surpass the fame of Alexander himself. But he solemnly declares
his unalterable resolution to pursue with fire and sword the
seditious adherents of the wicked Athanasius, who, by flying from
justice, has confessed his guilt, and escaped the ignominious
death which he had so often deserved.\textsuperscript{135}

\pagenote[133]{Ample materials for the history of this third
persecution of Athanasius may be found in his own works. See
particularly his very able Apology to Constantius, (tom. i. p.
673,) his first Apology for his flight (p. 701,) his prolix
Epistle to the Solitaries, (p. 808,) and the original protest of
the people of Alexandria against the violences committed by
Syrianus, (p. 866.) Sozomen (l. iv. c. 9) has thrown into the
narrative two or three luminous and important circumstances.}

\pagenote[134]{Athanasius had lately sent for Antony, and some of
his chosen monks. They descended from their mountains, announced
to the Alexandrians the sanctity of Athanasius, and were
honorably conducted by the archbishop as far as the gates of the
city. Athanas tom. ii. p. 491, 492. See likewise Rufinus, iii.
164, in Vit. Patr. p. 524.}

\pagenote[135]{Athanas. tom. i. p. 694. The emperor, or his Arian
secretaries while they express their resentment, betray their
fears and esteem of Athanasius.}

\section{Part \thesection.}

Athanasius had indeed escaped from the most imminent dangers; and
the adventures of that extraordinary man deserve and fix our
attention. On the memorable night when the church of St. Theonas
was invested by the troops of Syrianus, the archbishop, seated on
his throne, expected, with calm and intrepid dignity, the
approach of death. While the public devotion was interrupted by
shouts of rage and cries of terror, he animated his trembling
congregation to express their religious confidence, by chanting
one of the psalms of David which celebrates the triumph of the
God of Israel over the haughty and impious tyrant of Egypt. The
doors were at length burst open: a cloud of arrows was discharged
among the people; the soldiers, with drawn swords, rushed
forwards into the sanctuary; and the dreadful gleam of their arms
was reflected by the holy luminaries which burnt round the altar.\textsuperscript{136}
Athanasius still rejected the pious importunity of the monks
and presbyters, who were attached to his person; and nobly
refused to desert his episcopal station, till he had dismissed in
safety the last of the congregation. The darkness and tumult of
the night favored the retreat of the archbishop; and though he
was oppressed by the waves of an agitated multitude, though he
was thrown to the ground, and left without sense or motion, he
still recovered his undaunted courage, and eluded the eager
search of the soldiers, who were instructed by their Arian
guides, that the head of Athanasius would be the most acceptable
present to the emperor. From that moment the primate of Egypt
disappeared from the eyes of his enemies, and remained above six
years concealed in impenetrable obscurity.\textsuperscript{137}

\pagenote[136]{These minute circumstances are curious, as they
are literally transcribed from the protest, which was publicly
presented three days afterwards by the Catholics of Alexandria.
See Athanas. tom. l. n. 867}

\pagenote[137]{The Jansenists have often compared Athanasius and
Arnauld, and have expatiated with pleasure on the faith and zeal,
the merit and exile, of those celebrated doctors. This concealed
parallel is very dexterously managed by the Abbé de la Bleterie,
Vie de Jovien, tom. i. p. 130.}

The despotic power of his implacable enemy filled the whole
extent of the Roman world; and the exasperated monarch had
endeavored, by a very pressing epistle to the Christian princes
of Ethiopia,\textsuperscript{13711} to exclude Athanasius from the most remote and
sequestered regions of the earth. Counts, præfects, tribunes,
whole armies, were successively employed to pursue a bishop and a
fugitive; the vigilance of the civil and military powers was
excited by the Imperial edicts; liberal rewards were promised to
the man who should produce Athanasius, either alive or dead; and
the most severe penalties were denounced against those who should
dare to protect the public enemy.\textsuperscript{138} But the deserts of Thebais
were now peopled by a race of wild, yet submissive fanatics, who
preferred the commands of their abbot to the laws of their
sovereign. The numerous disciples of Antony and Pachonnus
received the fugitive primate as their father, admired the
patience and humility with which he conformed to their strictest
institutions, collected every word which dropped from his lips as
the genuine effusions of inspired wisdom; and persuaded
themselves that their prayers, their fasts, and their vigils,
were less meritorious than the zeal which they expressed, and the
dangers which they braved, in the defence of truth and innocence.\textsuperscript{139}
The monasteries of Egypt were seated in lonely and desolate
places, on the summit of mountains, or in the islands of the
Nile; and the sacred horn or trumpet of Tabenne was the
well-known signal which assembled several thousand robust and
determined monks, who, for the most part, had been the peasants
of the adjacent country. When their dark retreats were invaded by
a military force, which it was impossible to resist, they
silently stretched out their necks to the executioner; and
supported their national character, that tortures could never
wrest from an Egyptian the confession of a secret which he was
resolved not to disclose.\textsuperscript{140} The archbishop of Alexandria, for
whose safety they eagerly devoted their lives, was lost among a
uniform and well-disciplined multitude; and on the nearer
approach of danger, he was swiftly removed, by their officious
hands, from one place of concealment to another, till he reached
the formidable deserts, which the gloomy and credulous temper of
superstition had peopled with dæmons and savage monsters. The
retirement of Athanasius, which ended only with the life of
Constantius, was spent, for the most part, in the society of the
monks, who faithfully served him as guards, as secretaries, and
as messengers; but the importance of maintaining a more intimate
connection with the Catholic party tempted him, whenever the
diligence of the pursuit was abated, to emerge from the desert,
to introduce himself into Alexandria, and to trust his person to
the discretion of his friends and adherents. His various
adventures might have furnished the subject of a very
entertaining romance. He was once secreted in a dry cistern,
which he had scarcely left before he was betrayed by the
treachery of a female slave;\textsuperscript{141} and he was once concealed in a
still more extraordinary asylum, the house of a virgin, only
twenty years of age, and who was celebrated in the whole city for
her exquisite beauty. At the hour of midnight, as she related the
story many years afterwards, she was surprised by the appearance
of the archbishop in a loose undress, who, advancing with hasty
steps, conjured her to afford him the protection which he had
been directed by a celestial vision to seek under her hospitable
roof. The pious maid accepted and preserved the sacred pledge
which was intrusted to her prudence and courage. Without
imparting the secret to any one, she instantly conducted
Athanasius into her most secret chamber, and watched over his
safety with the tenderness of a friend and the assiduity of a
servant. As long as the danger continued, she regularly supplied
him with books and provisions, washed his feet, managed his
correspondence, and dexterously concealed from the eye of
suspicion this familiar and solitary intercourse between a saint
whose character required the most unblemished chastity, and a
female whose charms might excite the most dangerous emotions.\textsuperscript{142}
During the six years of persecution and exile, Athanasius
repeated his visits to his fair and faithful companion; and the
formal declaration, that he \textit{saw} the councils of Rimini and
Seleucia,\textsuperscript{143} forces us to believe that he was secretly present
at the time and place of their convocation. The advantage of
personally negotiating with his friends, and of observing and
improving the divisions of his enemies, might justify, in a
prudent statesman, so bold and dangerous an enterprise: and
Alexandria was connected by trade and navigation with every
seaport of the Mediterranean. From the depth of his inaccessible
retreat the intrepid primate waged an incessant and offensive war
against the protector of the Arians; and his seasonable writings,
which were diligently circulated and eagerly perused, contributed
to unite and animate the orthodox party. In his public apologies,
which he addressed to the emperor himself, he sometimes affected
the praise of moderation; whilst at the same time, in secret and
vehement invectives, he exposed Constantius as a weak and wicked
prince, the executioner of his family, the tyrant of the
republic, and the Antichrist of the church. In the height of his
prosperity, the victorious monarch, who had chastised the
rashness of Gallus, and suppressed the revolt of Sylvanus, who
had taken the diadem from the head of Vetranio, and vanquished in
the field the legions of Magnentius, received from an invisible
hand a wound, which he could neither heal nor revenge; and the
son of Constantine was the first of the Christian princes who
experienced the strength of those principles, which, in the cause
of religion, could resist the most violent exertions\textsuperscript{144} of the
civil power.

\pagenote[13711]{These princes were called Aeizanas and
Saiazanas. Athanasius calls them the kings of Axum. In the
superscription of his letter, Constantius gives them no title.
Mr. Salt, during his first journey in Ethiopia, (in 1806,)
discovered, in the ruins of Axum, a long and very interesting
inscription relating to these princes. It was erected to
commemorate the victory of Aeizanas over the Bougaitæ, (St.
Martin considers them the Blemmyes, whose true name is Bedjah or
Bodjah.) Aeizanas is styled king of the Axumites, the Homerites,
of Raeidan, of the Ethiopians, of the Sabsuites, of Silea, of
Tiamo, of the Bougaites, and of Kaei. It appears that at this
time the king of the Ethiopians ruled over the Homerites, the
inhabitants of Yemen. He was not yet a Christian, as he calls
himself son of the invincible Mars. Another brother besides
Saiazanas, named Adephas, is mentioned, though Aeizanas seems to
have been sole king. See St. Martin, note on Le Beau, ii. 151.
Salt’s Travels. De Sacy, note in Annales des Voyages, xii. p.
53.—M.}

\pagenote[138]{Hinc jam toto orbe profugus Athanasius, nec ullus
ci tutus ad latendum supererat locus. Tribuni, Præfecti, Comites,
exercitus quoque ad pervestigandum cum moventur edictis
Imperialibus; præmia dela toribus proponuntur, si quis eum vivum,
si id minus, caput certe Atha casii detulisset. Rufin. l. i. c.
16.}

\pagenote[139]{Gregor. Nazianzen. tom. i. Orat. xxi. p. 384, 385.
See Tillemont Mém. Eccles. tom. vii. p. 176-410, 820-830.}

\pagenote[140]{Et nulla tormentorum vis inveneri, adhuc potuit,
quæ obdurato illius tractus latroni invito elicere potuit, ut
nomen proprium dicat Ammian. xxii. 16, and Valesius ad locum.}

\pagenote[141]{Rufin. l. i. c. 18. Sozomen, l. iv. c. 10. This
and the following story will be rendered impossible, if we
suppose that Athanasius always inhabited the asylum which he
accidentally or occasionally had used.}

\pagenote[142]{Paladius, (Hist. Lausiac. c. 136, in Vit. Patrum,
p. 776,) the original author of this anecdote, had conversed with
the damsel, who in her old age still remembered with pleasure so
pious and honorable a connection. I cannot indulge the delicacy
of Baronius, Valesius, Tillemont, \&c., who almost reject a story
so unworthy, as they deem it, of the gravity of ecclesiastical
history.}

\pagenote[143]{Athanas. tom. i. p. 869. I agree with Tillemont,
(tom. iii. p. 1197,) that his expressions imply a personal,
though perhaps secret visit to the synods.}

\pagenote[144]{The epistle of Athanasius to the monks is filled
with reproaches, which the public must feel to be true, (vol. i.
p. 834, 856;) and, in compliment to his readers, he has
introduced the comparisons of Pharaoh, Ahab, Belshazzar, \&c. The
boldness of Hilary was attended with less danger, if he published
his invective in Gaul after the revolt of Julian; but Lucifer
sent his libels to Constantius, and almost challenged the reward
of martyrdom. See Tillemont, tom. vii. p. 905.}

The persecution of Athanasius, and of so many respectable
bishops, who suffered for the truth of their opinions, or at
least for the integrity of their conscience, was a just subject
of indignation and discontent to all Christians, except those who
were blindly devoted to the Arian faction. The people regretted
the loss of their faithful pastors, whose banishment was usually
followed by the intrusion of a stranger\textsuperscript{145} into the episcopal
chair; and loudly complained, that the right of election was
violated, and that they were condemned to obey a mercenary
usurper, whose person was unknown, and whose principles were
suspected. The Catholics might prove to the world, that they were
not involved in the guilt and heresy of their ecclesiastical
governor, by publicly testifying their dissent, or by totally
separating themselves from his communion. The first of these
methods was invented at Antioch, and practised with such success,
that it was soon diffused over the Christian world. The doxology
or sacred hymn, which celebrates the \textit{glory} of the Trinity, is
susceptible of very nice, but material, inflections; and the
substance of an orthodox, or an heretical, creed, may be
expressed by the difference of a disjunctive, or a copulative,
particle. Alternate responses, and a more regular psalmody,\textsuperscript{146}
were introduced into the public service by Flavianus and
Diodorus, two devout and active laymen, who were attached to the
Nicene faith. Under their conduct a swarm of monks issued from
the adjacent desert, bands of well-disciplined singers were
stationed in the cathedral of Antioch, the Glory to the Father,
And the Son, And the Holy Ghost,\textsuperscript{147} was triumphantly chanted by
a full chorus of voices; and the Catholics insulted, by the
purity of their doctrine, the Arian prelate, who had usurped the
throne of the venerable Eustathius. The same zeal which inspired
their songs prompted the more scrupulous members of the orthodox
party to form separate assemblies, which were governed by the
presbyters, till the death of their exiled bishop allowed the
election and consecration of a new episcopal pastor.\textsuperscript{148} The
revolutions of the court multiplied the number of pretenders; and
the same city was often disputed, under the reign of Constantius,
by two, or three, or even four, bishops, who exercised their
spiritual jurisdiction over their respective followers, and
alternately lost and regained the temporal possessions of the
church. The abuse of Christianity introduced into the Roman
government new causes of tyranny and sedition; the bands of civil
society were torn asunder by the fury of religious factions; and
the obscure citizen, who might calmly have surveyed the elevation
and fall of successive emperors, imagined and experienced, that
his own life and fortune were connected with the interests of a
popular ecclesiastic. The example of the two capitals, Rome and
Constantinople, may serve to represent the state of the empire,
and the temper of mankind, under the reign of the sons of
Constantine.

\pagenote[145]{Athanasius (tom. i. p. 811) complains in general
of this practice, which he afterwards exemplifies (p. 861) in the
pretended election of Fælix. Three eunuchs represented the Roman
people, and three prelates, who followed the court, assumed the
functions of the bishops of the Suburbicarian provinces.}

\pagenote[146]{Thomassin (Discipline de l’Eglise, tom. i. l. ii.
c. 72, 73, p. 966-984) has collected many curious facts
concerning the origin and progress of church singing, both in the
East and West. * Note: Arius appears to have been the first who
availed himself of this means of impressing his doctrines on the
popular ear: he composed songs for sailors, millers, and
travellers, and set them to common airs; “beguiling the ignorant,
by the sweetness of his music, into the impiety of his
doctrines.” Philostorgius, ii. 2. Arian singers used to parade
the streets of Constantinople by night, till Chrysostom arrayed
against them a band of orthodox choristers. Sozomen, viii. 8.—M.}

\pagenote[147]{Philostorgius, l. iii. c. 13. Godefroy has
examined this subject with singular accuracy, (p. 147, \&c.) There
were three heterodox forms: “To the Father \textit{by} the Son, \textit{and} in
the Holy Ghost.” “To the Father, \textit{and} the Son \textit{in} the Holy
Ghost;” and “To the Father \textit{in} the Son \textit{and} the Holy Ghost.”}

\pagenote[148]{After the exile of Eustathius, under the reign of
Constantine, the rigid party of the orthodox formed a separation
which afterwards degenerated into a schism, and lasted about
fourscore years. See Tillemont, Mém. Eccles. tom. vii. p. 35-54,
1137-1158, tom. viii. p. 537-632, 1314-1332. In many churches,
the Arians and Homoousians, who had renounced each other’s
\textit{communion}, continued for some time to join in prayer.
Philostorgius, l. iii. c. 14.}

I. The Roman pontiff, as long as he maintained his station and
his principles, was guarded by the warm attachment of a great
people; and could reject with scorn the prayers, the menaces, and
the oblations of an heretical prince. When the eunuchs had
secretly pronounced the exile of Liberius, the well-grounded
apprehension of a tumult engaged them to use the utmost
precautions in the execution of the sentence. The capital was
invested on every side, and the præfect was commanded to seize
the person of the bishop, either by stratagem or by open force.
The order was obeyed, and Liberius, with the greatest difficulty,
at the hour of midnight, was swiftly conveyed beyond the reach of
the Roman people, before their consternation was turned into
rage. As soon as they were informed of his banishment into
Thrace, a general assembly was convened, and the clergy of Rome
bound themselves, by a public and solemn oath, never to desert
their bishop, never to acknowledge the usurper Fælix; who, by the
influence of the eunuchs, had been irregularly chosen and
consecrated within the walls of a profane palace. At the end of
two years, their pious obstinacy subsisted entire and unshaken;
and when Constantius visited Rome, he was assailed by the
importunate solicitations of a people, who had preserved, as the
last remnant of their ancient freedom, the right of treating
their sovereign with familiar insolence. The wives of many of the
senators and most honorable citizens, after pressing their
husbands to intercede in favor of Liberius, were advised to
undertake a commission, which in their hands would be less
dangerous, and might prove more successful. The emperor received
with politeness these female deputies, whose wealth and dignity
were displayed in the magnificence of their dress and ornaments:
he admired their inflexible resolution of following their beloved
pastor to the most distant regions of the earth; and consented
that the two bishops, Liberius and Fælix, should govern in peace
their respective congregations. But the ideas of toleration were
so repugnant to the practice, and even to the sentiments, of
those times, that when the answer of Constantius was publicly
read in the Circus of Rome, so reasonable a project of
accommodation was rejected with contempt and ridicule. The eager
vehemence which animated the spectators in the decisive moment of
a horse-race, was now directed towards a different object; and
the Circus resounded with the shout of thousands, who repeatedly
exclaimed, “One God, One Christ, One Bishop!” The zeal of the
Roman people in the cause of Liberius was not confined to words
alone; and the dangerous and bloody sedition which they excited
soon after the departure of Constantius determined that prince to
accept the submission of the exiled prelate, and to restore him
to the undivided dominion of the capital. After some ineffectual
resistance, his rival was expelled from the city by the
permission of the emperor and the power of the opposite faction;
the adherents of Fælix were inhumanly murdered in the streets, in
the public places, in the baths, and even in the churches; and
the face of Rome, upon the return of a Christian bishop, renewed
the horrid image of the massacres of Marius, and the
proscriptions of Sylla.\textsuperscript{149}

\pagenote[149]{See, on this ecclesiastical revolution of Rome,
Ammianus, xv. 7 Athanas. tom. i. p. 834, 861. Sozomen, l. iv. c.
15. Theodoret, l. ii c. 17. Sulp. Sever. Hist. Sacra, l. ii. p.
413. Hieronym. Chron. Marcellin. et Faustin. Libell. p. 3, 4.
Tillemont, Mém. Eccles. tom. vi. p.}

II. Notwithstanding the rapid increase of Christians under the
reign of the Flavian family, Rome, Alexandria, and the other
great cities of the empire, still contained a strong and powerful
faction of Infidels, who envied the prosperity, and who
ridiculed, even in their theatres, the theological disputes of
the church. Constantinople alone enjoyed the advantage of being
born and educated in the bosom of the faith. The capital of the
East had never been polluted by the worship of idols; and the
whole body of the people had deeply imbibed the opinions, the
virtues, and the passions, which distinguished the Christians of
that age from the rest of mankind. After the death of Alexander,
the episcopal throne was disputed by Paul and Macedonius. By
their zeal and abilities they both deserved the eminent station
to which they aspired; and if the moral character of Macedonius
was less exceptionable, his competitor had the advantage of a
prior election and a more orthodox doctrine. His firm attachment
to the Nicene creed, which has given Paul a place in the calendar
among saints and martyrs, exposed him to the resentment of the
Arians. In the space of fourteen years he was five times driven
from his throne; to which he was more frequently restored by the
violence of the people, than by the permission of the prince; and
the power of Macedonius could be secured only by the death of his
rival. The unfortunate Paul was dragged in chains from the sandy
deserts of Mesopotamia to the most desolate places of Mount
Taurus,\textsuperscript{150} confined in a dark and narrow dungeon, left six days
without food, and at length strangled, by the order of Philip,
one of the principal ministers of the emperor Constantius.\textsuperscript{151}
The first blood which stained the new capital was spilt in this
ecclesiastical contest; and many persons were slain on both
sides, in the furious and obstinate seditions of the people. The
commission of enforcing a sentence of banishment against Paul had
been intrusted to Hermogenes, the master-general of the cavalry;
but the execution of it was fatal to himself. The Catholics rose
in the defence of their bishop; the palace of Hermogenes was
consumed; the first military officer of the empire was dragged by
the heels through the streets of Constantinople, and, after he
expired, his lifeless corpse was exposed to their wanton insults.\textsuperscript{152}
The fate of Hermogenes instructed Philip, the Prætorian
præfect, to act with more precaution on a similar occasion. In
the most gentle and honorable terms, he required the attendance
of Paul in the baths of Xeuxippus, which had a private
communication with the palace and the sea. A vessel, which lay
ready at the garden stairs, immediately hoisted sail; and, while
the people were still ignorant of the meditated sacrilege, their
bishop was already embarked on his voyage to Thessalonica. They
soon beheld, with surprise and indignation, the gates of the
palace thrown open, and the usurper Macedonius seated by the side
of the præfect on a lofty chariot, which was surrounded by troops
of guards with drawn swords. The military procession advanced
towards the cathedral; the Arians and the Catholics eagerly
rushed to occupy that important post; and three thousand one
hundred and fifty persons lost their lives in the confusion of
the tumult. Macedonius, who was supported by a regular force,
obtained a decisive victory; but his reign was disturbed by
clamor and sedition; and the causes which appeared the least
connected with the subject of dispute, were sufficient to nourish
and to kindle the flame of civil discord. As the chapel in which
the body of the great Constantine had been deposited was in a
ruinous condition, the bishop transported those venerable remains
into the church of St. Acacius. This prudent and even pious
measure was represented as a wicked profanation by the whole
party which adhered to the Homoousian doctrine. The factions
immediately flew to arms, the consecrated ground was used as
their field of battle; and one of the ecclesiastical historians
has observed, as a real fact, not as a figure of rhetoric, that
the well before the church overflowed with a stream of blood,
which filled the porticos and the adjacent courts. The writer who
should impute these tumults solely to a religious principle,
would betray a very imperfect knowledge of human nature; yet it
must be confessed that the motive which misled the sincerity of
zeal, and the pretence which disguised the licentiousness of
passion, suppressed the remorse which, in another cause, would
have succeeded to the rage of the Christians at Constantinople.\textsuperscript{153}

\pagenote[150]{Cucusus was the last stage of his life and
sufferings. The situation of that lonely town, on the confines of
Cappadocia, Cilicia, and the Lesser Armenia, has occasioned some
geographical perplexity; but we are directed to the true spot by
the course of the Roman road from Cæsarea to Anazarbus. See
Cellarii Geograph. tom. ii. p. 213. Wesseling ad Itinerar. p.
179, 703.}

\pagenote[151]{Athanasius (tom. i. p. 703, 813, 814) affirms, in
the most positive terms, that Paul was murdered; and appeals, not
only to common fame, but even to the unsuspicious testimony of
Philagrius, one of the Arian persecutors. Yet he acknowledges
that the heretics attributed to disease the death of the bishop
of Constantinople. Athanasius is servilely copied by Socrates,
(l. ii. c. 26;) but Sozomen, who discovers a more liberal temper.
presumes (l. iv. c. 2) to insinuate a prudent doubt.}

\pagenote[152]{Ammianus (xiv. 10) refers to his own account of
this tragic event. But we no longer possess that part of his
history. Note: The murder of Hermogenes took place at the first
expulsion of Paul from the see of Constantinople.—M.}

\pagenote[153]{See Socrates, l. ii. c. 6, 7, 12, 13, 15, 16, 26,
27, 38, and Sozomen, l. iii. 3, 4, 7, 9, l. iv. c. ii. 21. The
acts of St. Paul of Constantinople, of which Photius has made an
abstract, (Phot. Bibliot. p. 1419-1430,) are an indifferent copy
of these historians; but a modern Greek, who could write the life
of a saint without adding fables and miracles, is entitled to
some commendation.}

\section{Part \thesection.}

The cruel and arbitrary disposition of Constantius, which did not
always require the provocations of guilt and resistance, was
justly exasperated by the tumults of his capital, and the
criminal behavior of a faction, which opposed the authority and
religion of their sovereign. The ordinary punishments of death,
exile, and confiscation, were inflicted with partial vigor; and
the Greeks still revere the holy memory of two clerks, a reader,
and a sub-deacon, who were accused of the murder of Hermogenes,
and beheaded at the gates of Constantinople. By an edict of
Constantius against the Catholics which has not been judged
worthy of a place in the Theodosian code, those who refused to
communicate with the Arian bishops, and particularly with
Macedonius, were deprived of the immunities of ecclesiastics, and
of the rights of Christians; they were compelled to relinquish
the possession of the churches; and were strictly prohibited from
holding their assemblies within the walls of the city. The
execution of this unjust law, in the provinces of Thrace and Asia
Minor, was committed to the zeal of Macedonius; the civil and
military powers were directed to obey his commands; and the
cruelties exercised by this Semi- Arian tyrant in the support of
the \textit{Homoiousion}, exceeded the commission, and disgraced the
reign, of Constantius. The sacraments of the church were
administered to the reluctant victims, who denied the vocation,
and abhorred the principles, of Macedonius. The rites of baptism
were conferred on women and children, who, for that purpose, had
been torn from the arms of their friends and parents; the mouths
of the communicants were held open by a wooden engine, while the
consecrated bread was forced down their throat; the breasts of
tender virgins were either burnt with red-hot egg-shells, or
inhumanly compressed betweens harp and heavy boards.\textsuperscript{154} The
Novatians of Constantinople and the adjacent country, by their
firm attachment to the Homoousian standard, deserved to be
confounded with the Catholics themselves. Macedonius was
informed, that a large district of Paphlagonia\textsuperscript{155} was almost
entirely inhabited by those sectaries. He resolved either to
convert or to extirpate them; and as he distrusted, on this
occasion, the efficacy of an ecclesiastical mission, he commanded
a body of four thousand legionaries to march against the rebels,
and to reduce the territory of Mantinium under his spiritual
dominion. The Novatian peasants, animated by despair and
religious fury, boldly encountered the invaders of their country;
and though many of the Paphlagonians were slain, the Roman
legions were vanquished by an irregular multitude, armed only
with scythes and axes; and, except a few who escaped by an
ignominious flight, four thousand soldiers were left dead on the
field of battle. The successor of Constantius has expressed, in a
concise but lively manner, some of the theological calamities
which afflicted the empire, and more especially the East, in the
reign of a prince who was the slave of his own passions, and of
those of his eunuchs: “Many were imprisoned, and persecuted, and
driven into exile. Whole troops of those who are styled heretics,
were massacred, particularly at Cyzicus, and at Samosata. In
Paphlagonia, Bithynia, Galatia, and in many other provinces,
towns and villages were laid waste, and utterly destroyed.”\textsuperscript{156}

\pagenote[154]{Socrates, l. ii. c. 27, 38. Sozomen, l. iv. c. 21.
The principal assistants of Macedonius, in the work of
persecution, were the two bishops of Nicomedia and Cyzicus, who
were esteemed for their virtues, and especially for their
charity. I cannot forbear reminding the reader, that the
difference between the \textit{Homoousion} and \textit{Homoiousion}, is almost
invisible to the nicest theological eye.}

\pagenote[155]{We are ignorant of the precise situation of
Mantinium. In speaking of these four bands of legionaries,
Socrates, Sozomen, and the author of the acts of St. Paul, use
the indefinite terms of, which Nicephorus very properly
translates thousands. Vales. ad Socrat. l. ii. c. 38.}

\pagenote[156]{Julian. Epist. lii. p. 436, edit. Spanheim.}

While the flames of the Arian controversy consumed the vitals of
the empire, the African provinces were infested by their peculiar
enemies, the savage fanatics, who, under the name of
\textit{Circumcellions}, formed the strength and scandal of the Donatist
party.\textsuperscript{157} The severe execution of the laws of Constantine had
excited a spirit of discontent and resistance, the strenuous
efforts of his son Constans, to restore the unity of the church,
exasperated the sentiments of mutual hatred, which had first
occasioned the separation; and the methods of force and
corruption employed by the two Imperial commissioners, Paul and
Macarius, furnished the schismatics with a specious contrast
between the maxims of the apostles and the conduct of their
pretended successors.\textsuperscript{158} The peasants who inhabited the villages
of Numidia and Mauritania, were a ferocious race, who had been
imperfectly reduced under the authority of the Roman laws; who
were imperfectly converted to the Christian faith; but who were
actuated by a blind and furious enthusiasm in the cause of their
Donatist teachers. They indignantly supported the exile of their
bishops, the demolition of their churches, and the interruption
of their secret assemblies. The violence of the officers of
justice, who were usually sustained by a military guard, was
sometimes repelled with equal violence; and the blood of some
popular ecclesiastics, which had been shed in the quarrel,
inflamed their rude followers with an eager desire of revenging
the death of these holy martyrs. By their own cruelty and
rashness, the ministers of persecution sometimes provoked their
fate; and the guilt of an accidental tumult precipitated the
criminals into despair and rebellion. Driven from their native
villages, the Donatist peasants assembled in formidable gangs on
the edge of the Getulian desert; and readily exchanged the habits
of labor for a life of idleness and rapine, which was consecrated
by the name of religion, and faintly condemned by the doctors of
the sect. The leaders of the Circumcellions assumed the title of
captains of the saints; their principal weapon, as they were
indifferently provided with swords and spears, was a huge and
weighty club, which they termed an \textit{Israelite;} and the
well-known sound of “Praise be to God,” which they used as their
cry of war, diffused consternation over the unarmed provinces of
Africa. At first their depredations were colored by the plea of
necessity; but they soon exceeded the measure of subsistence,
indulged without control their intemperance and avarice, burnt
the villages which they had pillaged, and reigned the licentious
tyrants of the open country. The occupations of husbandry, and
the administration of justice, were interrupted; and as the
Circumcellions pretended to restore the primitive equality of
mankind, and to reform the abuses of civil society, they opened a
secure asylum for the slaves and debtors, who flocked in crowds
to their holy standard. When they were not resisted, they usually
contented themselves with plunder, but the slightest opposition
provoked them to acts of violence and murder; and some Catholic
priests, who had imprudently signalized their zeal, were tortured
by the fanatics with the most refined and wanton barbarity. The
spirit of the Circumcellions was not always exerted against their
defenceless enemies; they engaged, and sometimes defeated, the
troops of the province; and in the bloody action of Bagai, they
attacked in the open field, but with unsuccessful valor, an
advanced guard of the Imperial cavalry. The Donatists who were
taken in arms, received, and they soon deserved, the same
treatment which might have been shown to the wild beasts of the
desert. The captives died, without a murmur, either by the sword,
the axe, or the fire; and the measures of retaliation were
multiplied in a rapid proportion, which aggravated the horrors of
rebellion, and excluded the hope of mutual forgiveness. In the
beginning of the present century, the example of the
Circumcellions has been renewed in the persecution, the boldness,
the crimes, and the enthusiasm of the Camisards; and if the
fanatics of Languedoc surpassed those of Numidia, by their
military achievements, the Africans maintained their fierce
independence with more resolution and perseverance.\textsuperscript{159}

\pagenote[157]{See Optatus Milevitanus, (particularly iii. 4,)
with the Donatis history, by M. Dupin, and the original pieces at
the end of his edition. The numerous circumstances which Augustin
has mentioned, of the fury of the Circumcellions against others,
and against themselves, have been laboriously collected by
Tillemont, Mém. Eccles. tom. vi. p. 147-165; and he has often,
though without design, exposed injuries which had provoked those
fanatics.}

\pagenote[158]{It is amusing enough to observe the language of
opposite parties, when they speak of the same men and things.
Gratus, bishop of Carthage, begins the acclamations of an
orthodox synod, “Gratias Deo omnipotenti et Christu Jesu... qui
imperavit religiosissimo Constanti Imperatori, ut votum gereret
unitatis, et mitteret ministros sancti operis \textit{famulos Dei}
Paulum et Macarium.” Monument. Vet. ad Calcem Optati, p. 313.
“Ecce subito,” (says the Donatist author of the Passion of
Marculus), “de Constantis regif tyrannica domo.. pollutum
Macarianæ persecutionis murmur increpuit, et \textit{duabus bestiis} ad
Africam missis, eodem scilicet Macario et Paulo, execrandum
prorsus ac dirum ecclesiæ certamen indictum est; ut populus
Christianus ad unionem cum traditoribus faciendam, nudatis
militum gladiis et draconum præsentibus signis, et tubarum
vocibus cogeretur.” Monument. p. 304.}

\pagenote[159]{The Histoire des Camisards, in 3 vols. 12mo.
Villefranche, 1760 may be recommended as accurate and impartial.
It requires some attention to discover the religion of the author.}

Such disorders are the natural effects of religious tyranny, but
the rage of the Donatists was inflamed by a frenzy of a very
extraordinary kind; and which, if it really prevailed among them
in so extravagant a degree, cannot surely be paralleled in any
country or in any age. Many of these fanatics were possessed with
the horror of life, and the desire of martyrdom; and they deemed
it of little moment by what means, or by what hands, they
perished, if their conduct was sanctified by the intention of
devoting themselves to the glory of the true faith, and the hope
of eternal happiness.\textsuperscript{160} Sometimes they rudely disturbed the
festivals, and profaned the temples of Paganism, with the design
of exciting the most zealous of the idolaters to revenge the
insulted honor of their gods. They sometimes forced their way
into the courts of justice, and compelled the affrighted judge to
give orders for their immediate execution. They frequently
stopped travellers on the public highways, and obliged them to
inflict the stroke of martyrdom, by the promise of a reward, if
they consented, and by the threat of instant death, if they
refused to grant so very singular a favor. When they were
disappointed of every other resource, they announced the day on
which, in the presence of their friends and brethren, they should
cast themselves headlong from some lofty rock; and many
precipices were shown, which had acquired fame by the number of
religious suicides. In the actions of these desperate
enthusiasts, who were admired by one party as the martyrs of God,
and abhorred by the other as the victims of Satan, an impartial
philosopher may discover the influence and the last abuse of that
inflexible spirit which was originally derived from the character
and principles of the Jewish nation.

\pagenote[160]{The Donatist suicides alleged in their
justification the example of Razias, which is related in the 14th
chapter of the second book of the Maccabees.}

The simple narrative of the intestine divisions, which distracted
the peace, and dishonored the triumph, of the church, will
confirm the remark of a Pagan historian, and justify the
complaint of a venerable bishop. The experience of Ammianus had
convinced him, that the enmity of the Christians towards each
other, surpassed the fury of savage beasts against man;\textsuperscript{161} and
Gregory Nazianzen most pathetically laments, that the kingdom of
heaven was converted, by discord, into the image of chaos, of a
nocturnal tempest, and of hell itself.\textsuperscript{162} The fierce and partial
writers of the times, ascribing \textit{all} virtue to themselves, and
imputing \textit{all} guilt to their adversaries, have painted the
battle of the angels and dæmons. Our calmer reason will reject
such pure and perfect monsters of vice or sanctity, and will
impute an equal, or at least an indiscriminate, measure of good
and evil to the hostile sectaries, who assumed and bestowed the
appellations of orthodox and heretics. They had been educated in
the same religion and the same civil society. Their hopes and
fears in the present, or in a future life, were balanced in the
same proportion. On either side, the error might be innocent, the
faith sincere, the practice meritorious or corrupt. Their
passions were excited by similar objects; and they might
alternately abuse the favor of the court, or of the people. The
metaphysical opinions of the Athanasians and the Arians could not
influence their moral character; and they were alike actuated by
the intolerant spirit which has been extracted from the pure and
simple maxims of the gospel.

\pagenote[161]{Nullus infestas hominibus bestias, ut sunt sibi
ferales plerique Christianorum, expertus. Ammian. xxii. 5.}

\pagenote[162]{Gregor, Nazianzen, Orav. i. p. 33. See Tillemont,
tom vi. p. 501, qua to edit.}

A modern writer, who, with a just confidence, has prefixed to his
own history the honorable epithets of political and
philosophical,\textsuperscript{163} accuses the timid prudence of Montesquieu, for
neglecting to enumerate, among the causes of the decline of the
empire, a law of Constantine, by which the exercise of the Pagan
worship was absolutely suppressed, and a considerable part of his
subjects was left destitute of priests, of temples, and of any
public religion. The zeal of the philosophic historian for the
rights of mankind, has induced him to acquiesce in the ambiguous
testimony of those ecclesiastics, who have too lightly ascribed
to their favorite hero the \textit{merit} of a general persecution.\textsuperscript{164}
Instead of alleging this imaginary law, which would have blazed
in the front of the Imperial codes, we may safely appeal to the
original epistle, which Constantine addressed to the followers of
the ancient religion; at a time when he no longer disguised his
conversion, nor dreaded the rivals of his throne. He invites and
exhorts, in the most pressing terms, the subjects of the Roman
empire to imitate the example of their master; but he declares,
that those who still refuse to open their eyes to the celestial
light, may freely enjoy their temples and their fancied gods. A
report, that the ceremonies of paganism were suppressed, is
formally contradicted by the emperor himself, who wisely assigns,
as the principle of his moderation, the invincible force of
habit, of prejudice, and of superstition.\textsuperscript{165} Without violating
the sanctity of his promise, without alarming the fears of the
Pagans, the artful monarch advanced, by slow and cautious steps,
to undermine the irregular and decayed fabric of polytheism. The
partial acts of severity which he occasionally exercised, though
they were secretly promoted by a Christian zeal, were colored by
the fairest pretences of justice and the public good; and while
Constantine designed to ruin the foundations, he seemed to reform
the abuses, of the ancient religion. After the example of the
wisest of his predecessors, he condemned, under the most rigorous
penalties, the occult and impious arts of divination; which
excited the vain hopes, and sometimes the criminal attempts, of
those who were discontented with their present condition. An
ignominious silence was imposed on the oracles, which had been
publicly convicted of fraud and falsehood; the effeminate priests
of the Nile were abolished; and Constantine discharged the duties
of a Roman censor, when he gave orders for the demolition of
several temples of Phœnicia; in which every mode of prostitution
was devoutly practised in the face of day, and to the honor of
Venus.\textsuperscript{166} The Imperial city of Constantinople was, in some
measure, raised at the expense, and was adorned with the spoils,
of the opulent temples of Greece and Asia; the sacred property
was confiscated; the statues of gods and heroes were transported,
with rude familiarity, among a people who considered them as
objects, not of adoration, but of curiosity; the gold and silver
were restored to circulation; and the magistrates, the bishops,
and the eunuchs, improved the fortunate occasion of gratifying,
at once, their zeal, their avarice, and their resentment. But
these depredations were confined to a small part of the Roman
world; and the provinces had been long since accustomed to endure
the same sacrilegious rapine, from the tyranny of princes and
proconsuls, who could not be suspected of any design to subvert
the established religion.\textsuperscript{167}

\pagenote[163]{Histoire Politique et Philosophique des
Etablissemens des Europeens dans les deux Indes, tom. i. p. 9.}

\pagenote[164]{According to Eusebius, (in Vit. Constantin. l. ii.
c. 45,) the emperor prohibited, both in cities and in the
country, the abominable acts or parts of idolatry. l Socrates (l.
i. c. 17) and Sozomen (l. ii. c. 4, 5) have represented the
conduct of Constantine with a just regard to truth and history;
which has been neglected by Theodoret (l. v. c. 21) and Orosius,
(vii. 28.) Tum deinde (says the latter) primus Constantinus
\textit{justo} ordine et \textit{pio} vicem vertit edicto; siquidem statuit
citra ullam hominum cædem, paganorum templa claudi.}

\pagenote[165]{See Eusebius in Vit. Constantin. l. ii. c. 56, 60.
In the sermon to the assembly of saints, which the emperor
pronounced when he was mature in years and piety, he declares to
the idolaters (c. xii.) that they are permitted to offer
sacrifices, and to exercise every part of their religious
worship.}

\pagenote[166]{See Eusebius, in Vit. Constantin. l. iii. c.
54-58, and l. iv. c. 23, 25. These acts of authority may be
compared with the suppression of the Bacchanals, and the
demolition of the temple of Isis, by the magistrates of Pagan Rome.}

\pagenote[167]{Eusebius (in Vit. Constan. l. iii. c. 54-58) and
Libanius (Orat. pro Templis, p. 9, 10, edit. Gothofred) both
mention the pious sacrilege of Constantine, which they viewed in
very different lights. The latter expressly declares, that “he
made use of the sacred money, but made no alteration in the legal
worship; the temples indeed were impoverished, but the sacred
rites were performed there.” Lardner’s Jewish and Heathen
Testimonies, vol. iv. p. 140.}

The sons of Constantine trod in the footsteps of their father,
with more zeal, and with less discretion. The pretences of rapine
and oppression were insensibly multiplied;\textsuperscript{168} every indulgence
was shown to the illegal behavior of the Christians; every doubt
was explained to the disadvantage of Paganism; and the demolition
of the temples was celebrated as one of the auspicious events of
the reign of Constans and Constantius.\textsuperscript{169} The name of
Constantius is prefixed to a concise law, which might have
superseded the necessity of any future prohibitions. “It is our
pleasure, that in all places, and in all cities, the temples be
immediately shut, and carefully guarded, that none may have the
power of offending. It is likewise our pleasure, that all our
subjects should abstain from sacrifices. If any one should be
guilty of such an act, let him feel the sword of vengeance, and
after his execution, let his property be confiscated to the
public use. We denounce the same penalties against the governors
of the provinces, if they neglect to punish the criminals.”\textsuperscript{170}
But there is the strongest reason to believe, that this
formidable edict was either composed without being published, or
was published without being executed. The evidence of facts, and
the monuments which are still extant of brass and marble,
continue to prove the public exercise of the Pagan worship during
the whole reign of the sons of Constantine. In the East, as well
as in the West, in cities, as well as in the country, a great
number of temples were respected, or at least were spared; and
the devout multitude still enjoyed the luxury of sacrifices, of
festivals, and of processions, by the permission, or by the
connivance, of the civil government. About four years after the
supposed date of this bloody edict, Constantius visited the
temples of Rome; and the decency of his behavior is recommended
by a pagan orator as an example worthy of the imitation of
succeeding princes. “That emperor,” says Symmachus, “suffered the
privileges of the vestal virgins to remain inviolate; he bestowed
the sacerdotal dignities on the nobles of Rome, granted the
customary allowance to defray the expenses of the public rites
and sacrifices; and, though he had embraced a different religion,
he never attempted to deprive the empire of the sacred worship of
antiquity.”\textsuperscript{171} The senate still presumed to consecrate, by
solemn decrees, the \textit{divine} memory of their sovereigns; and
Constantine himself was associated, after his death, to those
gods whom he had renounced and insulted during his life. The
title, the ensigns, the prerogatives, of sovereign pontiff, which
had been instituted by Numa, and assumed by Augustus, were
accepted, without hesitation, by seven Christian emperors; who
were invested with a more absolute authority over the religion
which they had deserted, than over that which they professed.\textsuperscript{172}

\pagenote[168]{Ammianus (xxii. 4) speaks of some court eunuchs
who were spoliis templorum pasti. Libanius says (Orat. pro Templ.
p. 23) that the emperor often gave away a temple, like a dog, or
a horse, or a slave, or a gold cup; but the devout philosopher
takes care to observe that these sacrilegious favorites very
seldom prospered.}

\pagenote[169]{See Gothofred. Cod. Theodos. tom. vi. p. 262.
Liban. Orat. Parental c. x. in Fabric. Bibl. Græc. tom. vii. p. 235.}

\pagenote[170]{Placuit omnibus locis atque urbibus universis
claudi protinus empla, et accessu vetitis omnibus licentiam
delinquendi perditis abnegari. Volumus etiam cunctos a
sacrificiis abstinere. Quod siquis aliquid forte hujusmodi
perpetraverit, gladio sternatur: facultates etiam perempti fisco
decernimus vindicari: et similiter adfligi rectores provinciarum
si facinora vindicare neglexerint. Cod. Theodos. l. xvi. tit. x.
leg. 4. Chronology has discovered some contradiction in the date
of this extravagant law; the only one, perhaps, by which the
negligence of magistrates is punished by death and confiscation.
M. de la Bastie (Mém. de l’Académie, tom. xv. p. 98) conjectures,
with a show of reason, that this was no more than the minutes of
a law, the heads of an intended bill, which were found in
Scriniis Memoriæ among the papers of Constantius, and afterwards
inserted, as a worthy model, in the Theodosian Code.}

\pagenote[171]{Symmach. Epistol. x. 54.}

\pagenote[172]{The fourth Dissertation of M. de la Bastie, sur le
Souverain Pontificat des Empereurs Romains, (in the Mém. de
l’Acad. tom. xv. p. 75- 144,) is a very learned and judicious
performance, which explains the state, and prove the toleration,
of Paganism from Constantino to Gratian. The assertion of
Zosimus, that Gratian was the first who refused the pontifical
robe, is confirmed beyond a doubt; and the murmurs of bigotry on
that subject are almost silenced.}

The divisions of Christianity suspended the ruin of \textit{Paganism;}\textsuperscript{173}
and the holy war against the infidels was less vigorously
prosecuted by princes and bishops, who were more immediately
alarmed by the guilt and danger of domestic rebellion. The
extirpation of \textit{idolatry}\textsuperscript{174} might have been justified by the
established principles of intolerance: but the hostile sects,
which alternately reigned in the Imperial court were mutually
apprehensive of alienating, and perhaps exasperating, the minds
of a powerful, though declining faction. Every motive of
authority and fashion, of interest and reason, now militated on
the side of Christianity; but two or three generations elapsed,
before their victorious influence was universally felt. The
religion which had so long and so lately been established in the
Roman empire was still revered by a numerous people, less
attached indeed to speculative opinion, than to ancient custom.
The honors of the state and army were indifferently bestowed on
all the subjects of Constantine and Constantius; and a
considerable portion of knowledge and wealth and valor was still
engaged in the service of polytheism. The superstition of the
senator and of the peasant, of the poet and the philosopher, was
derived from very different causes, but they met with equal
devotion in the temples of the gods. Their zeal was insensibly
provoked by the insulting triumph of a proscribed sect; and their
hopes were revived by the well-grounded confidence, that the
presumptive heir of the empire, a young and valiant hero, who had
delivered Gaul from the arms of the Barbarians, had secretly
embraced the religion of his ancestors.

\pagenote[173]{As I have freely anticipated the use of \textit{pagans}
and \textit{paganism}, I shall now trace the singular revolutions of
those celebrated words. 1. in the Doric dialect, so familiar to
the Italians, signifies a fountain; and the rural neighborhood,
which frequented the same fountain, derived the common
appellation of \textit{pagus} and \textit{pagans}. (Festus sub voce, and
Servius ad Virgil. Georgic. ii. 382.) 2. By an easy extension of
the word, pagan and rural became almost synonymous, (Plin. Hist.
Natur. xxviii. 5;) and the meaner rustics acquired that name,
which has been corrupted into \textit{peasants} in the modern languages
of Europe. 3. The amazing increase of the military order
introduced the necessity of a correlative term, (Hume’s Essays,
vol. i. p. 555;) and all the \textit{people} who were not enlisted in
the service of the prince were branded with the contemptuous
epithets of pagans. (Tacit. Hist. iii. 24, 43, 77. Juvenal.
Satir. 16. Tertullian de Pallio, c. 4.) 4. The Christians were
the soldiers of Christ; their adversaries, who refused his
\textit{sacrament}, or military oath of baptism might deserve the
metaphorical name of pagans; and this popular reproach was
introduced as early as the reign of Valentinian (A. D. 365) into
Imperial laws (Cod. Theodos. l. xvi. tit. ii. leg. 18) and
theological writings. 5. Christianity gradually filled the cities
of the empire: the old religion, in the time of Prudentius
(advers. Symmachum, l. i. ad fin.) and Orosius, (in Præfat.
Hist.,) retired and languished in obscure villages; and the word
\textit{pagans}, with its new signification, reverted to its primitive
origin. 6. Since the worship of Jupiter and his family has
expired, the vacant title of pagans has been successively applied
to all the idolaters and polytheists of the old and new world. 7.
The Latin Christians bestowed it, without scruple, on their
mortal enemies, the Mahometans; and the purest \textit{Unitarians} were
branded with the unjust reproach of idolatry and paganism. See
Gerard Vossius, Etymologicon Linguæ Latinæ, in his works, tom. i.
p. 420; Godefroy’s Commentary on the Theodosian Code, tom. vi. p.
250; and Ducange, Mediæ et Infimæ Latinitat. Glossar.}

\pagenote[174]{In the pure language of Ionia and Athens were
ancient and familiar words. The former expressed a likeness, an
apparition (Homer. Odys. xi. 601,) a representation, an \textit{image},
created either by fancy or art. The latter denoted any sort of
\textit{service} or slavery. The Jews of Egypt, who translated the
Hebrew Scriptures, restrained the use of these words (Exod. xx.
4, 5) to the religious worship of an image. The peculiar idiom of
the Hellenists, or Grecian Jews, has been adopted by the sacred
and ecclesiastical writers and the reproach of \textit{idolatry} has
stigmatized that visible and abject mode of superstition, which
some sects of Christianity should not hastily impute to the
polytheists of Greece and Rome.}

