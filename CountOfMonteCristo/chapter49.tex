\chapter{Haydée}

It will be recollected that the new, or rather old, acquaintances of
the Count of Monte Cristo, residing in the Rue Meslay, were no other
than Maximilian, Julie, and Emmanuel.

The very anticipations of delight to be enjoyed in his forthcoming
visits—the bright, pure gleam of heavenly happiness it diffused over
the almost deadly warfare in which he had voluntarily engaged,
illumined his whole countenance with a look of ineffable joy and
calmness, as, immediately after Villefort’s departure, his thoughts
flew back to the cheering prospect before him, of tasting, at least, a
brief respite from the fierce and stormy passions of his mind. Even
Ali, who had hastened to obey the Count’s summons, went forth from his
master’s presence in charmed amazement at the unusual animation and
pleasure depicted on features ordinarily so stern and cold; while, as
though dreading to put to flight the agreeable ideas hovering over his
patron’s meditations, whatever they were, the faithful Nubian walked on
tiptoe towards the door, holding his breath, lest its faintest sound
should dissipate his master’s happy reverie.

It was noon, and Monte Cristo had set apart one hour to be passed in
the apartments of Haydée, as though his oppressed spirit could not all
at once admit the feeling of pure and unmixed joy, but required a
gradual succession of calm and gentle emotions to prepare his mind to
receive full and perfect happiness, in the same manner as ordinary
natures demand to be inured by degrees to the reception of strong or
violent sensations.

The young Greek, as we have already said, occupied apartments wholly
unconnected with those of the count. The rooms had been fitted up in
strict accordance with Oriental ideas; the floors were covered with the
richest carpets Turkey could produce; the walls hung with brocaded silk
of the most magnificent designs and texture; while around each chamber
luxurious divans were placed, with piles of soft and yielding cushions,
that needed only to be arranged at the pleasure or convenience of such
as sought repose.

Haydée had three French maids, and one who was a Greek. The first three
remained constantly in a small waiting-room, ready to obey the summons
of a small golden bell, or to receive the orders of the Romaic slave,
who knew just enough French to be able to transmit her mistress’s
wishes to the three other waiting-women; the latter had received most
peremptory instructions from Monte Cristo to treat Haydée with all the
deference they would observe to a queen.

The young girl herself generally passed her time in the chamber at the
farther end of her apartments. This was a sort of boudoir, circular,
and lighted only from the roof, which consisted of rose-colored glass.
Haydée was reclining upon soft downy cushions, covered with blue satin
spotted with silver; her head, supported by one of her exquisitely
moulded arms, rested on the divan immediately behind her, while the
other was employed in adjusting to her lips the coral tube of a rich
narghile, through whose flexible pipe she drew the smoke fragrant by
its passage through perfumed water. Her attitude, though perfectly
natural for an Eastern woman would, in a European, have been deemed too
full of coquettish straining after effect.

Her dress, which was that of the women of Epirus, consisted of a pair
of white satin trousers, embroidered with pink roses, displaying feet
so exquisitely formed and so delicately fair, that they might well have
been taken for Parian marble, had not the eye been undeceived by their
movements as they constantly shifted in and out of a pair of little
slippers with upturned toes, beautifully ornamented with gold and
pearls. She wore a blue and white-striped vest, with long open sleeves,
trimmed with silver loops and buttons of pearls, and a sort of bodice,
which, closing only from the centre to the waist, exhibited the whole
of the ivory throat and upper part of the bosom; it was fastened with
three magnificent diamond clasps. The junction of the bodice and
drawers was entirely concealed by one of the many-colored scarves,
whose brilliant hues and rich silken fringe have rendered them so
precious in the eyes of Parisian belles.

Tilted on one side of her head she had a small cap of gold-colored
silk, embroidered with pearls; while on the other a purple rose mingled
its glowing colors with the luxuriant masses of her hair, of which the
blackness was so intense that it was tinged with blue.

The extreme beauty of the countenance, that shone forth in loveliness
that mocked the vain attempts of dress to augment it, was peculiarly
and purely Grecian; there were the large, dark, melting eyes, the
finely formed nose, the coral lips, and pearly teeth, that belonged to
her race and country.

And, to complete the whole, Haydée was in the very springtide and
fulness of youthful charms—she had not yet numbered more than nineteen
or twenty summers.

Monte Cristo summoned the Greek attendant, and bade her inquire whether
it would be agreeable to her mistress to receive his visit. Haydée’s
only reply was to direct her servant by a sign to withdraw the
tapestried curtain that hung before the door of her boudoir, the
framework of the opening thus made serving as a sort of border to the
graceful tableau presented by the young girl’s picturesque attitude and
appearance.

As Monte Cristo approached, she leaned upon the elbow of the arm that
held the narghile, and extending to him her other hand, said, with a
smile of captivating sweetness, in the sonorous language spoken by the
women of Athens and Sparta:

“Why demand permission ere you enter? Are you no longer my master, or
have I ceased to be your slave?”

Monte Cristo returned her smile.

“Haydée,” said he, “you well know.”

“Why do you address me so coldly—so distantly?” asked the young Greek.
“Have I by any means displeased you? Oh, if so, punish me as you will;
but do not—do not speak to me in tones and manner so formal and
constrained.”

“Haydée,” replied the count, “you know that you are now in France, and
are free.”

“Free to do what?” asked the young girl.

“Free to leave me.”

“Leave you? Why should I leave you?”

“That is not for me to say; but we are now about to mix in society—to
visit and be visited.”

“I don’t wish to see anybody but you.”

“And should you see one whom you could prefer, I would not be so
unjust——”

“I have never seen anyone I preferred to you, and I have never loved
anyone but you and my father.”

“My poor child,” replied Monte Cristo, “that is merely because your
father and myself are the only men who have ever talked to you.”

“I don’t want anybody else to talk to me. My father said I was his
‘joy’—you style me your ‘love,’—and both of you have called me ‘my
child.’”

“Do you remember your father, Haydée?”

The young Greek smiled.

“He is here, and here,” said she, touching her eyes and her heart.

“And where am I?” inquired Monte Cristo laughingly.

“You?” cried she, with tones of thrilling tenderness, “you are
everywhere!” Monte Cristo took the delicate hand of the young girl in
his, and was about to raise it to his lips, when the simple child of
nature hastily withdrew it, and presented her cheek.

“You now understand, Haydée,” said the count, “that from this moment
you are absolutely free; that here you exercise unlimited sway, and are
at liberty to lay aside or continue the costume of your country, as it
may suit your inclination. Within this mansion you are absolute
mistress of your actions, and may go abroad or remain in your
apartments as may seem most agreeable to you. A carriage waits your
orders, and Ali and Myrtho will accompany you whithersoever you desire
to go. There is but one favor I would entreat of you.”

“Speak.”

“Guard carefully the secret of your birth. Make no allusion to the
past; nor upon any occasion be induced to pronounce the names of your
illustrious father or ill-fated mother.”

“I have already told you, my lord, that I shall see no one.”

“It is possible, Haydée, that so perfect a seclusion, though
conformable with the habits and customs of the East, may not be
practicable in Paris. Endeavor, then, to accustom yourself to our
manner of living in these northern climes as you did to those of Rome,
Florence, Milan, and Madrid; it may be useful to you one of these days,
whether you remain here or return to the East.”

The young girl raised her tearful eyes towards Monte Cristo as she said
with touching earnestness, “Whether \textit{we} return to the East, you mean
to say, my lord, do you not?”

“My child,” returned Monte Cristo “you know full well that whenever we
part, it will be no fault or wish of mine; the tree forsakes not the
flower—the flower falls from the tree.”

“My lord,” replied Haydée, “I never will leave you, for I am sure I
could not exist without you.”

“My poor girl, in ten years I shall be old, and you will be still
young.”

“My father had a long white beard, but I loved him; he was sixty years
old, but to me he was handsomer than all the fine youths I saw.”

“Then tell me, Haydée, do you believe you shall be able to accustom
yourself to our present mode of life?”

“Shall I see you?”

“Every day.”

“Then what do you fear, my lord?”

“You might find it dull.”

“No, my lord. In the morning, I shall rejoice in the prospect of your
coming, and in the evening dwell with delight on the happiness I have
enjoyed in your presence; then too, when alone, I can call forth mighty
pictures of the past, see vast horizons bounded only by the towering
mountains of Pindus and Olympus. Oh, believe me, that when three great
passions, such as sorrow, love, and gratitude fill the heart, \textit{ennui}
can find no place.”

“You are a worthy daughter of Epirus, Haydée, and your charming and
poetical ideas prove well your descent from that race of goddesses who
claim your country as their birthplace. Depend on my care to see that
your youth is not blighted, or suffered to pass away in ungenial
solitude; and of this be well assured, that if you love me as a father,
I love you as a child.”

“You are wrong, my lord. The love I have for you is very different from
the love I had for my father. My father died, but I did not die. If you
were to die, I should die too.”

The count, with a smile of profound tenderness, extended his hand, and
she carried it to her lips.

Monte Cristo, thus attuned to the interview he proposed to hold with
Morrel and his family, departed, murmuring as he went these lines of
Pindar, “Youth is a flower of which love is the fruit; happy is he who,
after having watched its silent growth, is permitted to gather and call
it his own.” The carriage was prepared according to orders, and
stepping lightly into it, the count drove off at his usual rapid pace.
