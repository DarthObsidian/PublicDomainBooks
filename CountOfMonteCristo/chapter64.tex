\chapter{The Beggar}

The evening passed on; Madame de Villefort expressed a desire to return
to Paris, which Madame Danglars had not dared to do, notwithstanding
the uneasiness she experienced. On his wife’s request, M. de Villefort
was the first to give the signal of departure. He offered a seat in his
landau to Madame Danglars, that she might be under the care of his
wife. As for M. Danglars, absorbed in an interesting conversation with
M. Cavalcanti, he paid no attention to anything that was passing. While
Monte Cristo had begged the smelling-bottle of Madame de Villefort, he
had noticed the approach of Villefort to Madame Danglars, and he soon
guessed all that had passed between them, though the words had been
uttered in so low a voice as hardly to be heard by Madame Danglars.
Without opposing their arrangements, he allowed Morrel, Château-Renaud,
and Debray to leave on horseback, and the ladies in M. de Villefort’s
carriage. Danglars, more and more delighted with Major Cavalcanti, had
offered him a seat in his carriage. Andrea Cavalcanti found his tilbury
waiting at the door; the groom, in every respect a caricature of the
English fashion, was standing on tiptoe to hold a large iron-gray
horse.

Andrea had spoken very little during dinner; he was an intelligent lad,
and he feared to utter some absurdity before so many grand people,
amongst whom, with dilating eyes, he saw the king’s attorney. Then he
had been seized upon by Danglars, who, with a rapid glance at the
stiff-necked old major and his modest son, and taking into
consideration the hospitality of the count, made up his mind that he
was in the society of some nabob come to Paris to finish the worldly
education of his heir. He contemplated with unspeakable delight the
large diamond which shone on the major’s little finger; for the major,
like a prudent man, in case of any accident happening to his
bank-notes, had immediately converted them into an available asset.
Then, after dinner, on the pretext of business, he questioned the
father and son upon their mode of living; and the father and son,
previously informed that it was through Danglars the one was to receive
his 48,000 francs and the other 50,000 livres annually, were so full of
affability that they would have shaken hands even with the banker’s
servants, so much did their gratitude need an object to expend itself
upon.

One thing above all the rest heightened the respect, nay almost the
veneration, of Danglars for Cavalcanti. The latter, faithful to the
principle of Horace, \textit{nil admirari}, had contented himself with showing
his knowledge by declaring in what lake the best lampreys were caught.
Then he had eaten some without saying a word more; Danglars, therefore,
concluded that such luxuries were common at the table of the
illustrious descendant of the Cavalcanti, who most likely in Lucca fed
upon trout brought from Switzerland, and lobsters sent from England, by
the same means used by the count to bring the lampreys from Lake
Fusaro, and the sterlet from the Volga. Thus it was with much
politeness of manner that he heard Cavalcanti pronounce these words:

“Tomorrow, sir, I shall have the honor of waiting upon you on
business.”

“And I, sir,” said Danglars, “shall be most happy to receive you.”

Upon which he offered to take Cavalcanti in his carriage to the Hôtel
des Princes, if it would not be depriving him of the company of his
son. To this Cavalcanti replied by saying that for some time past his
son had lived independently of him, that he had his own horses and
carriages, and that not having come together, it would not be difficult
for them to leave separately. The major seated himself, therefore, by
the side of Danglars, who was more and more charmed with the ideas of
order and economy which ruled this man, and yet who, being able to
allow his son 60,000 francs a year, might be supposed to possess a
fortune of 500,000 or 600,000 livres.

As for Andrea, he began, by way of showing off, to scold his groom,
who, instead of bringing the tilbury to the steps of the house, had
taken it to the outer door, thus giving him the trouble of walking
thirty steps to reach it. The groom heard him with humility, took the
bit of the impatient animal with his left hand, and with the right held
out the reins to Andrea, who, taking them from him, rested his polished
boot lightly on the step.

At that moment a hand touched his shoulder. The young man turned round,
thinking that Danglars or Monte Cristo had forgotten something they
wished to tell him, and had returned just as they were starting. But
instead of either of these, he saw nothing but a strange face,
sunburnt, and encircled by a beard, with eyes brilliant as carbuncles,
and a smile upon the mouth which displayed a perfect set of white
teeth, pointed and sharp as the wolf’s or jackal’s. A red handkerchief
encircled his gray head; torn and filthy garments covered his large
bony limbs, which seemed as though, like those of a skeleton, they
would rattle as he walked; and the hand with which he leaned upon the
young man’s shoulder, and which was the first thing Andrea saw, seemed
of gigantic size.

30229m



Did the young man recognize that face by the light of the lantern in
his tilbury, or was he merely struck with the horrible appearance of
his interrogator? We cannot say; but only relate the fact that he
shuddered and stepped back suddenly.

“What do you want of me?” he asked.

“Pardon me, my friend, if I disturb you,” said the man with the red
handkerchief, “but I want to speak to you.”

“You have no right to beg at night,” said the groom, endeavoring to rid
his master of the troublesome intruder.

“I am not begging, my fine fellow,” said the unknown to the servant,
with so ironical an expression of the eye, and so frightful a smile,
that he withdrew; “I only wish to say two or three words to your
master, who gave me a commission to execute about a fortnight ago.”

“Come,” said Andrea, with sufficient nerve for his servant not to
perceive his agitation, “what do you want? Speak quickly, friend.”

The man said, in a low voice: “I wish—I wish you to spare me the walk
back to Paris. I am very tired, and as I have not eaten so good a
dinner as you, I can scarcely stand.”

The young man shuddered at this strange familiarity.

“Tell me,” he said—“tell me what you want?”

“Well, then, I want you to take me up in your fine carriage, and carry
me back.” Andrea turned pale, but said nothing.

“Yes,” said the man, thrusting his hands into his pockets, and looking
impudently at the youth; “I have taken the whim into my head; do you
understand, Master Benedetto?”

At this name, no doubt, the young man reflected a little, for he went
towards his groom, saying:

“This man is right; I did indeed charge him with a commission, the
result of which he must tell me; walk to the barrier, there take a cab,
that you may not be too late.”

The surprised groom retired.

“Let me at least reach a shady spot,” said Andrea.

“Oh, as for that, I’ll take you to a splendid place,” said the man with
the handkerchief; and taking the horse’s bit he led the tilbury where
it was certainly impossible for anyone to witness the honor that Andrea
conferred upon him.

“Don’t think I want the glory of riding in your fine carriage,” said
he; “oh, no, it’s only because I am tired, and also because I have a
little business to talk over with you.”

“Come, step in,” said the young man. It was a pity this scene had not
occurred in daylight, for it was curious to see this rascal throwing
himself heavily down on the cushion beside the young and elegant driver
of the tilbury. Andrea drove past the last house in the village without
saying a word to his companion, who smiled complacently, as though
well-pleased to find himself travelling in so comfortable a vehicle.
Once out of Auteuil, Andrea looked around, in order to assure himself
that he could neither be seen nor heard, and then, stopping the horse
and crossing his arms before the man, he asked:

“Now, tell me why you come to disturb my tranquillity?”

“Let me ask you why you deceived me?”

“How have I deceived you?”

“‘How,’ do you ask? When we parted at the Pont du Var, you told me you
were going to travel through Piedmont and Tuscany; but instead of that,
you come to Paris.”

“How does that annoy you?”

“It does not; on the contrary, I think it will answer my purpose.”

“So,” said Andrea, “you are speculating upon me?”

“What fine words he uses!”

“I warn you, Master Caderousse, that you are mistaken.”

“Well, well, don’t be angry, my boy; you know well enough what it is to
be unfortunate; and misfortunes make us jealous. I thought you were
earning a living in Tuscany or Piedmont by acting as \textit{facchino} or
\textit{cicerone}, and I pitied you sincerely, as I would a child of my own.
You know I always did call you my child.”

“Come, come, what then?”

“Patience—patience!”

“I am patient, but go on.”

“All at once I see you pass through the barrier with a groom, a
tilbury, and fine new clothes. You must have discovered a mine, or else
become a stockbroker.”

“So that, as you confess, you are jealous?”

“No, I am pleased—so pleased that I wished to congratulate you; but as
I am not quite properly dressed, I chose my opportunity, that I might
not compromise you.”

“Yes, and a fine opportunity you have chosen!” exclaimed Andrea; “you
speak to me before my servant.”

“How can I help that, my boy? I speak to you when I can catch you. You
have a quick horse, a light tilbury, you are naturally as slippery as
an eel; if I had missed you tonight, I might not have had another
chance.”

“You see, I do not conceal myself.”

“You are lucky; I wish I could say as much, for I do conceal myself;
and then I was afraid you would not recognize me, but you did,” added
Caderousse with his unpleasant smile. “It was very polite of you.”

“Come,” said Andrea, “what do you want?”

“You do not speak affectionately to me, Benedetto, my old friend, that
is not right—take care, or I may become troublesome.”

This menace smothered the young man’s passion. He urged the horse again
into a trot.

“You should not speak so to an old friend like me, Caderousse, as you
said just now; you are a native of Marseilles, I am——”

“Do you know then now what you are?”

“No, but I was brought up in Corsica; you are old and obstinate, I am
young and wilful. Between people like us threats are out of place,
everything should be amicably arranged. Is it my fault if fortune,
which has frowned on you, has been kind to me?”

“Fortune has been kind to you, then? Your tilbury, your groom, your
clothes, are not then hired? Good, so much the better,” said
Caderousse, his eyes sparkling with avarice.

“Oh, you knew that well enough before speaking to me,” said Andrea,
becoming more and more excited. “If I had been wearing a handkerchief
like yours on my head, rags on my back, and worn-out shoes on my feet,
you would not have known me.”

“You wrong me, my boy; now I have found you, nothing prevents my being
as well-dressed as anyone, knowing, as I do, the goodness of your
heart. If you have two coats you will give me one of them. I used to
divide my soup and beans with you when you were hungry.”

“True,” said Andrea.

“What an appetite you used to have! Is it as good now?”

“Oh, yes,” replied Andrea, laughing.

“How did you come to be dining with that prince whose house you have
just left?”

“He is not a prince; simply a count.”

“A count, and a rich one too, eh?”

“Yes; but you had better not have anything to say to him, for he is not
a very good-tempered gentleman.”

“Oh, be easy! I have no design upon your count, and you shall have him
all to yourself. But,” said Caderousse, again smiling with the
disagreeable expression he had before assumed, “you must pay for it—you
understand?”

“Well, what do you want?”

“I think that with a hundred francs a month——”

“Well?”

“I could live——”

“Upon a hundred francs!”

“Come—you understand me; but that with——”

“With?”

“With a hundred and fifty francs I should be quite happy.”

“Here are two hundred,” said Andrea; and he placed ten gold louis in
the hand of Caderousse.

30233m



“Good!” said Caderousse.

“Apply to the steward on the first day of every month, and you will
receive the same sum.”

“There now, again you degrade me.”

“How so?”

“By making me apply to the servants, when I want to transact business
with you alone.”

“Well, be it so, then. Take it from me then, and so long at least as I
receive my income, you shall be paid yours.”

“Come, come; I always said you were a fine fellow, and it is a blessing
when good fortune happens to such as you. But tell me all about it?”

“Why do you wish to know?” asked Cavalcanti.

“What? do you again defy me?”

“No; the fact is, I have found my father.”

“What? a real father?”

“Yes, so long as he pays me——”

“You’ll honor and believe him—that’s right. What is his name?”

“Major Cavalcanti.”

“Is he pleased with you?”

“So far I have appeared to answer his purpose.”

“And who found this father for you?”

“The Count of Monte Cristo.”

“The man whose house you have just left?”

“Yes.”

“I wish you would try and find me a situation with him as grandfather,
since he holds the money-chest!”

“Well, I will mention you to him. Meanwhile, what are you going to do?”

“I?”

“Yes, you.”

“It is very kind of you to trouble yourself about me.”

“Since you interest yourself in my affairs, I think it is now my turn
to ask you some questions.”

“Ah, true. Well; I shall rent a room in some respectable house, wear a
decent coat, shave every day, and go and read the papers in a café.
Then, in the evening, I shall go to the theatre; I shall look like some
retired baker. That is what I want.”

“Come, if you will only put this scheme into execution, and be steady,
nothing could be better.”

“Do you think so, M. Bossuet? And you—what will you become? A peer of
France?”

“Ah,” said Andrea, “who knows?”

“Major Cavalcanti is already one, perhaps; but then, hereditary rank is
abolished.”

“No politics, Caderousse. And now that you have all you want, and that
we understand each other, jump down from the tilbury and disappear.”

“Not at all, my good friend.”

“How? Not at all?”

“Why, just think for a moment; with this red handkerchief on my head,
with scarcely any shoes, no papers, and ten gold napoleons in my
pocket, without reckoning what was there before—making in all about two
hundred francs,—why, I should certainly be arrested at the barriers.
Then, to justify myself, I should say that you gave me the money; this
would cause inquiries, it would be found that I left Toulon without
giving due notice, and I should then be escorted back to the shores of
the Mediterranean. Then I should become simply No. 106, and good-bye to
my dream of resembling the retired baker! No, no, my boy; I prefer
remaining honorably in the capital.”

Andrea scowled. Certainly, as he had himself owned, the reputed son of
Major Cavalcanti was a wilful fellow. He drew up for a minute, threw a
rapid glance around him, and then his hand fell instantly into his
pocket, where it began playing with a pistol. But, meanwhile,
Caderousse, who had never taken his eyes off his companion, passed his
hand behind his back, and opened a long Spanish knife, which he always
carried with him, to be ready in case of need. The two friends, as we
see, were worthy of and understood one another. Andrea’s hand left his
pocket inoffensively, and was carried up to the red moustache, which it
played with for some time.

“Good Caderousse,” he said, “how happy you will be.”

“I will do my best,” said the innkeeper of the Pont du Gard, shutting
up his knife.

“Well, then, we will go into Paris. But how will you pass through the
barrier without exciting suspicion? It seems to me that you are in more
danger riding than on foot.”

“Wait,” said Caderousse, “we shall see.” He then took the greatcoat
with the large collar, which the groom had left behind in the tilbury,
and put it on his back; then he took off Cavalcanti’s hat, which he
placed upon his own head, and finally he assumed the careless attitude
of a servant whose master drives himself.

“But, tell me,” said Andrea, “am I to remain bareheaded?”

“Pooh,” said Caderousse; “it is so windy that your hat can easily
appear to have blown off.”

“Come, come; enough of this,” said Cavalcanti.

“What are you waiting for?” said Caderousse. “I hope I am not the
cause.”

“Hush,” said Andrea. They passed the barrier without accident. At the
first cross street Andrea stopped his horse, and Caderousse leaped out.

“Well!” said Andrea,—“my servant’s coat and my hat?”

“Ah,” said Caderousse, “you would not like me to risk taking cold?”

“But what am I to do?”

“You? Oh, you are young while I am beginning to get old. \textit{Au revoir},
Benedetto;” and running into a court, he disappeared.

“Alas,” said Andrea, sighing, “one cannot be completely happy in this
world!”
