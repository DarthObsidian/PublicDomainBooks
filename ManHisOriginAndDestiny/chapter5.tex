\chapter{THE DOCTRINE OF GOD—2}

HAVING presented the evidence from the scriptures that God revealed himself to his
prophets and talked to them face to face 1 proving that he is an anthropomorphic being, and
that man was created in the image of his person 2, let us turn to the evidence of later times. It
is very evident that in the days of the apostles of old the true knowledge of God was fully
understood. These apostles never confounded the separate entities of the Father and the Son.
Jesus on numerous occasions taught them the true character of his Father. He taught them to
pray to his Father, and if they forgave not others their trespasses the Father would not forgive
theirs. 3 He taught them that those who deny him he would deny before his Father. 4 On
various occasions in the presence of his disciples he prayed to his Father. 5 He taught men
not to despise little children, for their angels (spirits) when they died "do always behold the
face of my Father which is in heaven." 6 He taught that "whosoever therefore shall be
ashamed of me and of my words in this adulterous and sinful generation; of him also shall
the Son of man be ashamed, when he cometh in the glory of his Father with the holy angels."
7 He also declared that no man "knoweth who the Son is, but the Father; and who the Father
is, but the Son, and he to whom the Son will reveal him," 8 and that "as the Father hath life in
himself; so hath he given to the Son to have life in himself; And hath given him authority to
execute judgment also, because he is the Son of man," 9 and also he taught, "I have greater
witness than that of John: for the works which the Father hath given me to finish, the same
works that I do, bear witness of me, that the Father hath sent me. And the Father himself,
which hath sent me, hath borne witness of me. Ye have neither heard his voice at any time,
nor seen his shape. And ye have not his word abiding in you: for whom he hath sent, him ye
believe not." 10 "Ye have heard how I said unto you, I go away, and come again unto you. If
ye loved me, ye would rejoice, because I said, I go unto the Father: for my Father is greater
than I." 11

These are only a few out of the many declarations in which he honored his Father and taught
his disciples and the Jews that he and his Father are separate personages, and that he is in
very deed the Only Begotten Son of the Father. This truth his apostles fully understood and
in the epistles which have come down to us invariably they have given testimony of this
truth. At this point let us consider some of these testimonies; first we refer to Peter. In the
introduction to his first epistle he says:

Peter, an apostle of Jesus Christ, to the strangers scattered throughout Pontus, Galatia,
Cappadocia, Asia, and Bithynia.

Elect according to the foreknowledge of God the Father, through sanctification of the Spirit,
unto obedience and sprinkling of the blood of Jesus Christ: Grace unto you, and peace, be
multiplied.

Blessed be the God and Father of our Lord Jesus Christ, which according to his abundant
mercy hath begotten us again unto a lively hope by the resurrection of Jesus Christ from the
dead. 12

So, likewise he testifies in the introduction to his second epistle and in the body of it he
proclaims the separate entities of the Father and the Son in his testimony of the great vision
of transfiguration on the Mount as follows:

For we have not followed cunningly devised fables, when we made known unto you the
power and coming of our Lord Jesus Christ, but were eyewitnesses of his majesty.

For he received from God the Father honour and glory, when there came such a voice to him
from the excellent glory, This is my beloved Son, in whom I am well pleased.

And the voice which came from heaven we heard, when we were with him in the holy
mount. 13

The testimony of John:

That which we have seen and heard declare we unto you, that ye also may have fellowship
with us: and truly our fellowship is with the Father, and with his Son Jesus Christ.

And these things write we unto you, that your joy may be full.

This then is the message which we have heard of him, and declare unto you, that God is light,
and in him is no darkness at all.

If we say that we have fellowship with him, and walk in darkness, we lie, and do not the
truth:

But if we walk in the light, as he is in the light, we have fellowship one with another, and the
blood of Jesus Christ his Son cleanseth us from all sin. 14

And hath made us kings and priests unto God and his Father; to him be glory and dominion
for ever and ever. Amen. 15

Similar testimonies are given by James, by Jude, and in each of the epistles of Paul. In the
epistle to the Hebrews is this excellent testimony that Jesus Christ is in the "express image"
of his Father, it reads as follows:

God, who at sundry times and in divers manners spake in times past unto the fathers by the
prophets;

Hath in these last days spoken unto us by his Son, whom he hath appointed heir of all things,
by whom also he made the worlds;

Who being the brightness of his glory, and \textit{the express image of his person}, and upholding all
things by the word of his power, when he had by himself purged our sins, sat down on the
right hand of the Majesty on high;

Being made so much better than the angels, as he hath by inheritance obtained a more
excellent name than they. 16

As all of these testimonies are clearly stated, and the writings of these apostles are so plain in
relation to the nature and individual entities of the Father, the Son and the Holy Ghost, the
question arises: how is it that the Christian world of the present day has gone so far astray
notwithstanding this plainness of the scriptures? This has come through apostasy and the
mingling of pagan philosophy with Christian doctrines. There was no doubt in the minds of
the primitive members of the Church of Jesus Christ. These changes in the doctrines, the
order of the Priesthood and the knowledge of God, came gradually. The teachers of religion
had closed their eyes and their ears and declared that there was to be no more revelation or
communication with the heavens. Thus men were left to grope and stumble in their search for
truth. The teachers in the ministry had become corrupt and blindness of heart was universal.
To add to all of this there came the amalgamation of Christian doctrine and practices with
pagan worship and procedure, and this resulted in the changes in ordinances and government
and a loss of divine guidance and authority.

Then came also the influence of a pagan emperor, who, while not a member of the Church,
by his imperial influence, dominated the ecclesiastical officials who bowed to his decrees.
For a number of years at the beginning of the fourth century a controversy arose in relation to
the character and nature of God. This was not the only doctrine over which there were
contentions, but this controversy raged and it appeared that it would divide the Church,
which already had traveled the road to apostasy by the changing of ordinances and church
government so that in that day it had very little resemblance to the Church in the days of the
apostles. One of these contending factions was championed by Arius of Alexandria and the
other by Athanasius. Arius believed, as nearly as we are able to discover his views, for his
writings were destroyed and an anathema pronounced upon all those who believed them, that
the Son was younger than the Father; that there was a time when the Son did not exist. He
maintained that the Father must be older than the Son and that under such circumstances the
Son must be subordinate to the Father. This doctrine implied that there are two Gods separate
from each other. The other faction, championed by Athanasius, contended that there were not
three separate Gods in the Godhead, but only one, and that in some mysterious manner God
appeared as the Father, as the Son and as the Holy Ghost, but not three separate Gods. The
controversy spread throughout Christendom and waxed so hot that the emperor, Constantine,
took action to bring these quarrels to an end. Eventually in the year 325 A.D., he called a
council to be held at Nice with the object in view of having this controversy and other
contentions settled. The details of this controversy need not be considered here. 17 It is
sufficient to state that the Arians were defeated and the Athanasians were sustained. This,
however, did not end the contention which continued for several years, but eventually the
Athanasian doctrine prevailed and has been the doctrine of the Catholic Church and in
substance of many Protestant churches to this present day. As nearly as we can give it, the
Nicean creed as first given was as follows:

We believe in one God, the Father, Almighty, the maker of all things visible and invisible;
and in one Lord, Jesus Christ, the Son of God, begotten of the Father, only begotten (that is)
of the substance of the Father; God of God, Light of Light, Very God of Very God, begotten
not made; of the same substance with the Father, by whom all things were made, that are in
heaven and that are on earth; who for us men, and for our salvation, descended and was
incarnate, and became man; suffered and rose again the third day, ascended into the heavens
and will come to judge the living and the dead; and in the Holy Spirit. But those who say
there was a time when he (the Son) was not, and that he was not before he was begotten, and
that he was made out of nothing, or affirm that he is of any other substance or essence, or that
the Son of God was created, and mutuable, or changeable, the Catholic Church pronounces
accursed. 18

ATHANASIAN CREED, The one of the symbols of the Faith approved by the Church and
given a place in her liturgy, is a short, clear explanation of the doctrines of the Trinity and the
Incarnation, with a passing reference to several other dogmas. Unlike most of the other
creeds, or symbols, it deals almost exclusively with these two fundamental truths, which it
states and restates in terse and varied forms so as to bring out unanimously the trinity of
Persons in God, and the two-fold nature in the one Divine Person of Jesus Christ. At various
points the author calls attention to the penalty incurred, by those who refuse to accept any of
the articles therein set down. The following is the Marquess of Bute's English translation of
the text of the Creed:

Whosoever will be saved, before all things it is necessary that he hold the Catholic Faith.
Which Faith except every one do keep whole and undefiled, without doubt he shall perish
everlastingly. And the Catholic Faith is this, that we worship one God in Trinity and Trinity
in Unity, neither confounding the Persons, nor dividing the Substance. For there is one
Person of the Father, another of the Son, and another of the Holy Ghost. But the Godhead of
the Father, of the Son and of the Holy Ghost is all One, the Glory Equal, the Majesty Co-
Eternal. Such as the Father, such is the Son, and such is the Holy Ghost. The Father
Uncreate, the Son Uncreate, and the Holy Ghost Uncreate. The Father Incomprehensible, the
Son Incomprehensible, and the Holy Ghost Incomprehensible. The Father Eternal, the Son
Eternal, and the Holy Ghost Eternal and yet they are not Three Eternals but One Eternal. As
also there are not Three Uncreated, nor Three Incomprehensibles, but One Uncreated, and
One Incomprehensible. So likewise the Father is Almighty, the Son Almighty, and the Holy
Ghost Almighty. And yet there are not Three Almighties but One Almighty.

So the Father is God, the Son is God, and the Holy Ghost is God. And yet they are not Three
Gods, but One God. So likewise the Father is Lord, the Son Lord, and the Holy Ghost Lord.
And yet not Three Lords but One Lord. For, like as we are compelled by the Christian verity
to acknowledge every Person by Himself to be God and Lord, so are we forbidden by the
Catholic Religion to say, there be Three Gods or Three Lords. The Father is made of none,
neither created, nor begotten. The Son is of the Father alone; not made, nor created, but
begotten. The Holy Ghost is of the Father, and of the Son, neither made, nor created nor
begotten but proceeding.

So there is One Father, not Three Fathers; One Son, not Three Sons; One Holy Ghost, not
Three Holy Ghosts. And in this Trinity none is afore or after Other, None is greater or less
than Another, but the whole Three Persons are Co-Eternal together, and Co-Equal. So that in
all things, as is aforesaid, the Unity in Trinity, and the Trinity in Unity is to be worshipped.
He therefore that will be saved, must thus think of the Trinity.

Furthermore, it is necessary to everlasting Salvation, that he also believe rightly in
Incarnation of our Lord Jesus Christ. For the right Faith is, that we believe and confess, that
our Lord Jesus Christ, the Son of God, is God and Man. God of the substance of the Father,
begotten before the worlds; and Man, of the substance of His mother, born into the world.
Perfect God and Perfect Man, of a reasonable Soul and human Flesh subsisting. Equal to the
Father as touching His Godhead, and inferior to the Father as touching His Manhood. Who,
although He be God and Man, yet He is not two, but One Christ, One, not by conversion ofthe Godhead into Flesh, but by taking of the Manhood into God. One altogether, not by
confusion of substance, but by unity of person, For as the reasonable soul and flesh is one
Man, so God and Man is one Christ. Who suffered for our salvation, descended into Hell,
rose again the third day from the dead. He ascended into Heaven, He sitteth on the right hand
of the Father, God Almighty, from whence he shall come to judge the quick and the dead. At
whose coming all men shall rise again with their bodies, and shall give account for their own
works. And they that have done good shall go into life everlasting, and they that have done
evil into everlasting fire. This is the Catholic Faith, which except a man believe faithfully and
firmly, he cannot be saved. (\textit{Catholic Encyclopedia}, Vol. 2, pp. 33-34.) 19

The following statement is taken verbatim from the \textit{Catholic Encyclopedia} under the
heading, \textit{Nicene and Niceno-Constantinople Creed}.

The origin and history of the Nicene Creed are set forth in the articles: Nicea, Councils of;
Arius, Arianism; Eusebius of Caeserea; Filioque. As approved in amplified form at the
Council of Constantinople (381) q.v., it is the profession of the Christian Faith common to
the Catholic Church, to all Eastern Churches separated from Rome, and to most of the
Protestant denominations. Soon after the Council of Nicea new formulas of faith were
composed, most of them variations of the Nicene Symbol, however, continued to be the only
one in use among the defenders of the Faith. Gradually it came to be recognized as the proper
profession of faith, for candidates for baptism. Its alteration into the Nicene-Constantinople
formula, the one now in use, is usually ascribed to the Council of Constantinople, since the
Council of Chalcedon (451) which designated the symbol as "The Creed of the Council of
Constantinople of 381," had it twice read and inserted in its Acts. The historians Socrates,
Sozomen, and Theodoret do not mention this, although they do record that the Bishops who
remained at the council after the departure of the Macedonian confirmed the Nicene faith.
Hefele (11:9) admits the possibility of our present creed being a condensation of the "Tome,"
the exposition of the doctrine concerning the Trinity made by the Council of Constantinople;
but he prefers the opinion of Remi Ceillier and Tillemont tracing the new formula to the
"Ancoratus" of Epiphanius written in 374. Hort, Caspari, Harnack, and others are of the
opinion that the Constantinopolitan form did not originate at the Council of Constantinople,
because it is not in the Acts of the council of 381, but was inserted there at a later date;
because Gregory Nazianzen who was at the council mentions only the Nicene formula
adverting to its incompleteness about the Holy Ghost, showing that he did not know of the
Constantinopolian form which supplies this deficiency; and because the Latin Fathers
apparently know nothing of it before the middle of the fifth century.

The following is a literal translation of the Greek text of the Constantinople form, the
brackets indicating the words altered or added in the Western liturgical form in present use.
20

\textit{The Nicene-Constantinople Creed:}

We believe [I believe] in one God, the Father Almighty, maker of heaven and earth, and of
all things visible and invisible. And in one Lord Jesus Christ, the only begotten Son of God,
and born of the Father before all ages. [God of God] light of light, true God of true God.
Begotten not made, consubstantial to the Father, by whom all things were made. Who for us
men and for our salvation came down from heaven. And was incarnate of the Holy Ghost and
of the Virgin Mary and was made man; was crucified also for us under Pontius Pilate,suffered and was buried; and the third day he rose again according to the Scriptures. And
ascended into heaven, sitteth at the right hand of the Father, and shall come again with glory
to judge the living and the dead, of whose Kingdom there shall be no end. And [I believe] in
the Holy Ghost, the Lord and giver of life, who proceedeth from the Father [and the Son,]
who together with the Father and the Son is to be adored and glorified, who spake by the
Prophets. And one holy Catholic and apostolic Church. We confess [I confess] one baptism
for the remission of sins. And we look for [I look for] the resurrection of the dead and the life
of the world to come. Amen. 21

In the year 1870, at a council convoked by Pope Pius IX the following doctrine in relation to
the Godhead, was presented and approved:

OF GOD, THE CREATOR OF ALL THINGS:—The Holy Catholic Apostolic Roman
Church believes that there is one true and living God, Creator and Lord of Heaven and Earth,
Almighty Eternal, Immense, Incomprehensible, Infinite in understanding and will, and in all
perfection. He is distinct from the world. Of his own most free counsel he made alike out of
nothing two created creatures, a spiritual and a temporal, angelic and earthly. Afterwards he
made the human nature, composed of both. Moreover, God by his providences protects and
governs all things reaching from end to end mightily, and ordering all things harmoniously.
Every thing is open to his eyes, even things that come to pass by the free action of his
creatures.

OF REVELATION:—The Holy Mother Church holds that God can be known with certainty
by the natural light of human reason, but that it has also pleased him to reveal himself and the
eternal decrees of his will in a supernatural way. This supernatural revelation, as declared by
the Holy Council of Trent, is contained in the books of the Old and New Testament, as
enumerated in the decrees of that Council, and as are to be had in the old Vulgate Latin
edition. These are sacred because they were written under the inspiration of the Holy Ghost.
They have God for their author, and as such have been delivered to the Church.

And, in order to restrain restless spirits, who may give erroneous explanations, it is
decreed—renewing the decision of the Council of Trent—that no one may interpret the
sacred Scriptures contrary to the sense in which they are interpreted by Holy Mother Church,
to whom such interpretation belongs. 22

After one has given careful attention to these creeds one is convinced that they are
incomprehensible. On this point we all can be agreed. Moreover, any person who is taught
this doctrine would have to agree, if he accepts it, that God \textit{is} a mystery that cannot be
understood, and he would be left in hopeless bewilderment when he reads—if he is permitted
to read—the many evidences in the scriptures in relation to the personal and physical identity
of both the Father and the Son. The fact that it is a doctrine that cannot be understood, and
was never intended to be understood, should cause any reasoning mind to doubt the truth of
it. To all who are caught in the meshes of such a confusing doctrine the glorious words of the
Son of God, who, in the tenderness and humility of his soul shortly before he was taken to
the sacrifice to have his blood shed for the redemption of mankind, prayed to his Father:

Father, the hour is come: glorify thy Son, that thy Son also may glorify thee:

As thou hast given him power over all flesh, that he should give eternal life to as many as
thou hast given him.

\textit{And this is life eternal, that they might know thee the only true God, and Jesus Christ, whom
thou hast sent.} 23

The wonder of it all is how men can become so confused and blinded that they cannot
understand the simple truth, although they have lost the guiding influence of the Spirit of the
Lord, for the truth is so clearly and plainly taught that "wayfaring men, though fools" should
not err therein. 24 There is no doctrine in the Bible more plainly taught than the doctrines of
the anthropomorphic nature of both the Father and the Son and their separate personages. It is
written also that God "created man, male and female, after his own image and in his own
likeness, created he them." 25 It is an astounding thing that from the days of Athanasius and
the great council at Nice, the true doctrine concerning the Father and the Son and the Holy
Ghost, has been lost to the Christian world, both Catholic and Protestant. Therefore, before
the work of the "restitution of all things, which God hath spoken by the mouth of all his holy
prophets since the world began," 26 there had to come the opening of the heavens and the
restoration of communication between God and man. In this long age of spiritual darkness
from 325 A.D., to the year 1820, the world walked blindly not knowing how to worship, nor
what to worship, 27 without spiritual guidance and the power of the holy Priesthood. All of
this is implied in the oft repeated declaration universally taught that the heavens were sealed
and "We have a Bible, and there cannot be any more Bible," for the Lord has finished his
work. 28

But there were false prophets also among the people, even as there shall be false teachers
among you, who privily shall bring in damnable heresies, even denying the Lord that bought
them, and bring upon themselves swift destruction. (2 Peter 2:1.)

We now come to the evidence concerning the nature and majesty of God as made known in
the dispensation of the Fulness of Times. Although the story of the visitation of the Father
and the Son to Joseph Smith has been told numerous times, yet the nature of this work calls
for the repeating of that story, for it is the one story of the greatest worth to all the world and
has proved to be an eternal blessing to the thousands who believe it and have, through their
faith, obtained the testimony of the Spirit of the Lord that it is verily true.

\textit{Joseph Smith's Vision of the Father and the Son:—}

Owing to the many reports which have been put in circulation by evil-disposed and designing
persons, in relation to the rise and progress of the Church of Jesus Christ of Latter-day Saints,
all of which have been designated by the authors thereof to militate against its character as a
Church and its progress in the world—I have been induced to write this history, to disabuse
the public mind, and put all inquirers after truth in possession of the facts, as they have
transpired, in relation both to myself and the Church, so far as I have such facts in my
possession.

In this history I shall present the various events in relation to this Church, in truth and
righteousness, as they have transpired, or as they at present exist, being now the eighth year
since the organization of the said Church.

I was born in the year of our Lord one thousand eight hundred and five, on the twenty-third
day of December, in the town of Sharon, Windsor county, State of Vermont. . . . My father,
Joseph Smith, Sen., left the State of Vermont, and moved to Palmyra, Ontario (now Wayne)
county, in the State of New York, when I was in my tenth year, or thereabouts. In about four
years after my father's arrival in Palmyra, he moved with his family into Manchester, in the
same county of Ontario. . . .

Some time in the second year after our removal to Manchester, there was in the place where
we lived an unusual excitement on the subject of religion. It commenced with the Methodists,
but soon became general among all the sects in that region of country. Indeed, the whole
district of country seemed affected by it, and great multitudes united themselves to the
different religious parties, which created no small stir and division amongst the people, some
crying, "Lo, here!" and others "Lo, there!" Some were contending for the Methodist faith,
some for the Presbyterian, and some for the Baptists.

For, notwithstanding the great love which the converts to these different faiths expressed at
the time of their conversion, and the great zeal manifested by the respective clergy, who were
active in getting up and promoting this extraordinary scene of religious feeling, in order to
have everybody converted, as they were pleased to call it, let them join what sect they
pleased; yet when the converts began to file off, some to one party and some to another, it
was seen that the seemingly good feelings of both the priests and the converts were more
pretended than real; for a scene of great confusion and bad feeling ensued—priest contending
against priest, and convert against convert; so that all their good feelings one for another, if
they ever had any, were entirely lost in a strife of words and a contest about opinions.

I was at this time in my fifteenth year. My father's family was proselyted to the Presbyterian
faith, and four of them joined that church, namely, my mother, Lucy; my brothers Hyrum and
Samuel Harrison; and my sister Sophronia.

During this time of great excitement my mind was called up to serious reflection and great
uneasiness; but though my feelings were deep and often poignant, still I kept myself aloof
from all these parties, though I attended their several meetings as often as occasion would
permit. In process of time my mind became somewhat partial to the Methodist sect, and I felt
some desire to be united with them; but so great were the confusion and strife among the
different denominations, that it was impossible for a person young as I was, and so
unacquainted with men and things, to come to any certain conclusion who was right and who
was wrong.

My mind at times was greatly excited, the cry and tumult were so great and incessant. The
Presbyterians were most decided against the Baptists and Methodists, and used all the powers
of both reason and sophistry to prove their errors, or, at least, to make the people think they
were in error. On the other hand, the Baptists and Methodists in their turn were equally
zealous in endeavoring to establish their own tenets and disprove all others.

In the midst of this war of words and tumult of opinions, I often said to myself: What is to be
done? Who of all these parties are right; or, are they all wrong together? If any one of them
be right, which is it, and how shall I know it?

While I was laboring under the extreme difficulties caused by the contests of these parties of
religionists, I was one day reading the Epistle of James, first chapter and fifth verse, which
reads: \textit{If any of you lack wisdom, let him ask of God, that giveth to all men liberally, and
upbraideth not; and it shall be given him.}

Never did any passage of scripture come with more power to the heart of man than this did at
this time to mine. It seemed to enter with great force into every feeling of my heart. I
reflected on it again and again, knowing that if any person needed wisdom from God, I did;
for how to act I did not know, and unless I could get more wisdom than I then had, I would
never know; for the teachers of religion of the different sects understood the same passages
of scripture so differently as to destroy all confidence in settling the question by an appeal to
the Bible.

At length I came to the conclusion that I must either remain in darkness and confusion, or
else I must do as James directs, that is, ask of God. I at length came to the determination to
"ask of God," concluding that if he gave wisdom to them that lacked wisdom, and would give
liberally, and not upbraid, I might venture.

So, in accordance with this, my determination to ask of God, I retired to the woods to make
the attempt. It was on the morning of a beautiful, clear day, early in the spring of eighteen
hundred and twenty. It was the first time in my life that I had made such an attempt, for
amidst all my anxieties I had never as yet made the attempt to pray vocally.

After I had returned to the place where I had previously designed to go, having looked
around me, and finding myself alone, I kneeled down and began to offer up the desires of my
heart to God. I had scarcely done so, when immediately I was seized upon by some power
which entirely overcame me, and had such an astonishing influence over me as to bind my
tongue so that I could not speak. Thick darkness gathered around me, and it seemed to me for
a time as if I were doomed to sudden destruction.

But, exerting all my powers to call upon God to deliver me out of the power of this enemy
which had seized upon me, and at the very moment when I was ready to sink into despair and
abandon myself to destruction—not to an imaginary ruin, but to the power of some actual
being from the unseen world, who had such marvelous power as I had never before felt in
any being—just at this moment of great alarm, I saw a pillar of light exactly over my head,
above the brightness of the sun, which descended gradually until it fell upon me.

It no sooner appeared than I found myself delivered from the enemy which held me bound.
When the light rested upon me I saw two Personages, whose brightness and glory defy all
description, standing above me in the air. One of them spake unto me, calling me by name,
and said, pointing to the other—\textit{This is My Beloved Son. Hear Him!}

My object in going to inquire of the Lord was to know which of all the sects was right, that I
might know which to join. No sooner, therefore, did I get possession of myself, so as to be
able to speak, than I asked the Personages who stood above me in the light, which of all the
sects was right—and which I should join.

I was answered that I must join none of them, for they were all wrong; and the Personage
who addressed me said that all their creeds were an abomination in his sight; that thoseprofessors were all corrupt; that: "They draw near to me with their lips, but their hearts are
far from me; they teach for doctrines the commandments of men, having a form of godliness,
but they deny the power thereof."

He again forbade me to join with any of them; and many other things did he say unto me,
which I cannot write at this time. When I came to myself again, I found myself lying on my
back, looking up into heaven. When the light had departed, I had no strength; but soon
recovering in some degree, I went home. And as I leaned up to the fireplace, mother inquired
what the matter was. I replied, "Never mind, all is well—I am well enough off." I then said to
my mother, "I have learned for myself that Presbyterianism is not true." It seems as though
the adversary was aware, at a very early period of my life, that I was destined to prove a
disturber and an annoyer of his kingdom; else why should the powers of darkness combine
against me? Why the opposition and persecution that arose against me, almost in my
infancy?

Some few days after I had this vision, I happened to be in company with one of the
Methodist preachers, who was very active in the before mentioned religious excitement; and,
conversing with him on the subject of religion, I took occasion to give him an account of the
vision which I had had. I was greatly surprised at his behavior; he treated my communication
not only lightly, but with great contempt, saying it was all of the devil, that there were no
such things as visions or revelations in these days; that all such things had ceased with the
apostles, and that there would never be any more of them.

I soon found, however, that my telling the story had excited a great deal of prejudice against
me among professors of religion, and was the cause of great persecution, which continued to
increase; and though I was an obscure boy, only between fourteen and fifteen years of age,
and my circumstances in life such as to make a boy of no consequence in the world, yet men
of high standing would take notice sufficient to excite the public mind against me, and create
a bitter persecution; and this was common among all the sects—all united to persecute me.

It caused me serious reflection then, and often has since, how very strange it was that an
obscure boy, of a little over fourteen years of age, and one, too, who was doomed to the
necessity of obtaining a scanty maintenance by his daily labor, should be thought a character
of sufficient importance to attract the attention of the great ones of the most popular sects of
the day, and in a manner to create in them a spirit of the most bitter persecution and reviling.
But strange or not, so it was, and it was often the cause of great sorrow to myself.

However, it was nevertheless a fact that I had beheld a vision. I have thought since, that I felt
much like Paul, when he made his defense before King Agrippa, and related the account of
the vision he had when he saw a light, and heard a voice; but still there were but few who
believed him; some said he was dishonest, others said he was mad; and he was ridiculed and
reviled. But all this did not destroy the reality of his vision. He had seen a vision, he knew he
had, and all the persecution under heaven could not make it otherwise; and though they
should persecute him unto death, yet he knew, and would know to the latest breath, that he
had both seen a light and heard a voice speaking unto him, and all the world could not make
him think or believe otherwise.

So it was with me. I had actually seen a light, and in the midst of that light I saw two
Personages, and they did in reality speak to me; and though I was hated and persecuted forsaying that I had seen a vision, yet it was true; and while they were persecuting me, reviling
me, and speaking all manner of evil against me falsely for so saying, I was led to say in my
heart: Why persecute me for telling the truth? I have actually seen a vision; and who am I
that I can withstand God, or why does the world think to make me deny what I have actually
seen? For I had seen a vision; I knew it, and I knew that God knew it, and I could not deny it,
neither dared I do it; at least I knew that by so doing I would offend God, and come under
condemnation.

I had now got my mind satisfied so far as the sectarian world was concerned—that it was not
my duty to join with any of them, but to continue as I was until further directed. I had found
the testimony of James to be true—that a man who lacked wisdom might ask of God, and
obtain, and not be upbraided. 29

So we have the direct testimony of Joseph Smith that he had seen both the Father and the
Son, on a beautiful day, when the sun was shining brightly. The brightness of these
Personages outshone the brightness of the sun. In later interviews with heavenly messengers
the Prophet was informed that his name "should be had for good and evil among nations,
kindreds and tongues, or that it should be both good and evil spoken of among all people." 30
This prophecy was literally fulfilled. President George Q. Cannon, in his \textit{Life of Joseph
Smith}, deals with this prophecy, from which I quote:

The perusal of the history of the Church during the life of Joseph the Prophet suggests many
reflections and to many minds prompts many inquiries. One cannot fail to be struck with the
unceasing opposition with which he had to contend. From the day that he received the first
communication from heaven up to the day of his martyrdom his pathway was beset with
difficulties, his liberty and life were constantly menaced. Had he been an ordinary man he
would have been crushed in spirit and sunk in despair under the relentless attacks which were
made upon him. To find a parallel to his case we must go back to the days of our Savior and
his apostles, and the prophets who preceded them. Joseph's life was sought for with satanic
hate. The thirst for his blood was unappeasable. Had there not been a special providence
exercised in his behalf to preserve him until his mission should be fulfilled, he would have
been slain by murderous hands long before the dreadful day at Carthage.

To the inexperienced reader it seems unaccountable that any generation of men could have
been so blind to everything god-like, so dead to every human sentiment, so utterly cruel and
barbarous, as not to recognize in the teachings, works and life of God's beloved Son the
divinity with which he was clothed and to nail him upon a cross between two thieves. Also
that his chosen apostles, filled with angelic power, preaching so pure a doctrine and laboring
with such self-denial and unselfish zeal for the salvation of mankind, should have been slain
by the very people whose benefactors they sought to be.

But in our own age the same scenes are reenacted. Joseph Smith, a Prophet of God, called by
the Almighty to receive the everlasting Priesthood, to lay the foundation of the Church of
Christ and to preach the ancient pure gospel, performs the mission to which he was divinely
appointed, and is pursued with vindictive hate through his life, and is finally barbarously
slain. The explanation of all of this is given by the Lord himself in his words to his disciples:
"If ye were of the world, the world would love its own; but because ye are not of the world,
but I have chosen you out of the world, therefore the world hateth you."

According to the predictions, this is the dispensation of the fulness of times—the crowning
dispensation of all. To leave the world without excuse and to prepare the way for the second
coming of the Lord, the holy Priesthood, the pure gospel and the true Church of Christ are
restored to earth through the ministration of angels. Satan, fully conscious that if these
prevail his dominion will be overthrown, arrays all his forces against the servants and work
of God. He resorts to his old tactics to accomplish his purpose. He was a liar and a murderer
from the beginning. Lies and murder are the agencies he depends upon. Many being free
agents, and having the power to choose whom they will serve, become his instruments of
hate, and the earth is drenched with the blood of innocence. The Prophet Joseph, while he
lived, was the conspicuous object of his vengeance. Like Paul, he could have recounted a
long list of perils which he had to encounter, not the least of which, as in the case of Paul,
were "perils among false brethren." Of all the evils with which this great Prophet had to
contend none were so grievous or so hard to be borne as the deflection and treason of "false
brethren." The most deadly wounds he ever received were from those who, Judas-like, had
been his companions. When through their transgressions, they lost the Spirit of God and
turned away from the truth, the spirit of murder took possession of them, they became fit
instruments for Satan's service, and to this class, more than to any others, can the foul
murders of the 27th of June, 1844, be charged.

The great bulk of those who composed the mobs which attacked the Saints in Missouri and
Illinois were ignorant men. Their passions were easily aroused. A few cunning and
unscrupulous leaders were able to use them to accomplish their ends. Seeing the increase of
the Saints, they were easily persuaded that, if left to themselves, they would soon outnumber
the old settlers, they would out-vote them, take possession of the offices, and drive them out
of the country. By such representations and artifices as these, appealing to the lowest and
basest of motives, they were able to inflame the minds of the ignorant, unprincipled men.
Envious of the prosperity of the Saints, coveting their possessions, they thought to profit in
driving them from their homes. Apostates had personal vengeance and hates to gratify;
politicians saw a growing power which they could not control, and whose union made it
formidable in county and state affairs; the clergy saw a system of religion which they could
not controvert, and the rabble had their cupidity excited at the prospect of plunder, which
might fall to them through the abandonment of lands and improvements and stock by the
people whom they were driving away. 31

There are some interesting developments resulting from the frank and honest story told by
the Prophet Joseph Smith that cannot be overlooked and when considered cannot honestly be
ignored. We have discovered that the scriptures are perfectly clear, both the Old and the New
Testament, in the doctrine that God appeared in person to some of his servants and that he
talked with them "face to face," 32 and to others in open vision, 33 to others by the sound of
his voice. 34 The Savior prayed to his Father on numerous occasions and received answers in
an audible voice from the heavens. 35 He testified that he was in very deed the Son of God.
36 He lived some 33 years in mortal flesh. He was reviled, persecuted, and put to death on
the cross. He rose the third day and appeared to his disciples who were invited to thrust their
fingers into the wounds in his hands and side. He was with his disciples for forty days after
his resurrection and then took his departure in a cloud of glory and ascended into heaven to
sit on the right side of his Father. All of these things are recorded by the eyewitnesses who
followed him in his ministry and whose stories are true beyond all righteous dispute.

Yet we discover, in the year 1830 the Christian world, both Catholic and Protestant, was
confused about his personality. They taught that the Christ, after his ascension, in some
mysterious manner "shed" his body of flesh and bones and became an integral and
inseparable part of God, who is defined as being without "body, parts and passions:" an
ethereal, or, more correctly speaking, an incorporeal, immaterial, essence, composed of
Father, Son and Holy Ghost, "invisible," "immense," filling the boundless distances of space.
This doctrine had come down since the great council of Nice, and is accepted by the
Christian world.

It is rather strange that the learned men who were leaders in the so-called Reformation,
Martin Luther, Ulrich Zwingle, John Knox, John Calvin, and later the Wesleys and others
who formed sects and churches, failed in their criticisms of the Catholic Church to discover
this stupendous error in relation to the nature and identity of God the Father and his Son
Jesus Christ. Notwithstanding the clear, precise teachings of the scriptures, they accepted the
error of the Nicene Creed, so it remained for an obscure farmer boy in a small village in
western New York, a boy in his "teens," who had received no scholastic learning, to discover
the truth and present it again to the unbelieving world. For his pains he was ridiculed,
mocked, driven from place to place and lied about and finally put to death.

If he were wrong, how easy it should have been for these learned ministers of religion to
challenge his statement and invite him to go to the scriptures and endeavor to show him that
he was in error about his vision and that it, according to the Bible, could not be true! Some of
them tried it, but to their great consternation and discomfiture they invariably found that they
were wrong and he was right. When they discovered their plight instead of humbly
acknowledging their error and accepting the truth, they became all the more bitter towards
him and sought more intensely his destruction. You, who may be in doubt, go to your
scriptures and see for yourselves.

Other testimony concerning God is found abundantly in the revelations given to the Prophet
and his companions. It is sufficient to present two outstanding visions on this point. The first
is the vision given to Joseph Smith and Sidney Rigdon who, like Stephen the martyr, saw
both the Father and the Son, and bore testimony in these words:

And while we meditated upon these things, the Lord touched the eyes of our understandings
and they were opened, and the glory of the Lord shone around about.

And we beheld the glory of the Son, on the right hand of the Father, and received of his
fulness;

And saw the holy angels, and them who are sanctified before his throne, worshipping God,
and the Lamb, who worship him forever and ever.

And now, after the many testimonies which have been given of him, this is the testimony,
last of all, which we give of him: That he lives!

For we saw him, even on the right hand of God; and we heard the voice bearing record that
he is the Only Begotten of the Father—

That by him, and through him, and of him, the worlds are and were created, and the
inhabitants thereof are begotten sons and daughters unto God. 37

The other testimony is that of Joseph Smith the Prophet and Oliver Cowdery given in the
Kirtland Temple, April 3, 1836:

The veil was taken from our minds, and the eyes of our understanding were opened.

We saw the Lord standing upon the breast-work of the pulpit, before us; and under his feet
was a paved work of pure gold, in color like amber.

His eyes were as a flame of fire; the hair of his head was white like the pure snow; his
countenance shone above the brightness of the sun; and his voice was as the sound of the
rushing of great waters, even the voice of Jehovah, saying:

I am the first and the last; I am he who liveth, I am he who was slain; I am your advocate
with the Father. 38

In these testimonies we discover that Jesus Christ, the Great I AM, the Redeemer of the
world, did not lose his identity. His body was not dissolved. It did not expand to fill the
immensity of space, but he was in the form of man, for it was the same body which hung
upon the cross. Moreover, John on the Isle of Patmos when in exile, many years after the
ascension of our Lord, also saw him and described his glorious body and appearance in like
manner:

And I turned to see the voice that spake with me. And being turned, I saw seven golden
candlesticks;

And in the midst of the seven candlesticks one like unto the Son of man, clothed with a
garment down to the foot, and girt about the paps with a golden girdle.

His head and his hairs were white like wool, as white as snow; and his eyes were as a flame
of fire;

And his feet like unto fine brass, as if they burned in a furnace; and his voice as the sound of
many waters.

And he had in his right hand seven stars: and out of his mouth went a sharp two edged sword:
and his countenance was as the sun shineth in his strength.

And when I saw him, I fell at his feet as dead. And he laid his right hand upon me, saying
unto me, Fear not; I am the first and the last:

I am he that liveth, and was dead; and, behold, I am alive for evermore, Amen; and have the
keys of hell and of death. 39

Is it not astonishingly strange that the Christian denominations even to this day refuse to
accept the truth and return to the doctrines of our Redeemer, which again have been declaredto them by the voice of the Almighty through the testimonies of his servants the prophets, but
would rather hug to their bosoms the out-worn creed of Athanasius?

Moreover, is it not just as strange that the scientist who is an organic evolutionist, holds so
tenaciously to the "God of Nature," or the "God of Science"? who likewise fills the
immensity of space, and neither speaks, sees or hears? Why is it that men will not humble
themselves and in faith and prayer seek the truth from the divine source of Truth?

\newpage
REFERENCES—CHAPTER FIVE

Footnotes

1. Exodus 33:11; Num. 12:6-8; Deut. 34:10;Ether Chapter 3.

2. Gen. 1:26-27; 5:1.

3. Matt. 6:6-16.

4. Matt. 10:32-33.

5. Matt. 11:25-26; Luke 10:21; John 12:27-30.

6. Matt. 18:10.

7. Mark 8:38.

8. Luke 10:22.

9. John 5:26-27.

10. John 5:36-38.

11. John 14:28.

12. 1 Peter 1:1-2.

13. 2 Peter 1:16-18.

14. 1 John 1:3-7.

15. Rev. 1:6.

16. Heb. 1:1-4.

17. Roberts, B. H., \textit{Outlines of Ecclesiastical History}, p. 189.

18. \textit{Catholic Encyclopedia}, Vol. 2, p. 33.

19. \textit{Ibid.}, Vol. 11, p. 49.

20. \textit{Ibid.}, Vol. 11, p. 49.

21. Draper, Dr. J. W., \textit{Conflict Between Religion and Science}, pp. 345-346 1877 ed.

22. John 17:1-5.

23. Isaiah 25:8.24. D. \& C. 20:18.

25. Acts 3:21.

26. D. \& C. 93:19.

27. 2 Nephi 29:3.

28. P. of G. P., pp. 46-50.

29. D. H. C., Vol. 1:11-12; P. of G. P., p. 51.

30. Cannon, G. Q., \textit{Life of Joseph Smith}, pp. 484-486.

31. Gen. 32:30; Exodus 24:10; 33:11; Num. 12:5-8.

32. Ezek. 1:1; Isaiah 6:1-3; Acts 7:55.

33. Deut. 4:33; 5:24-25; Matt. 3:17; Luke 9:35; John 12:28-30.

34. Luke 3:22.

35. Matt. 26:63-64; John 4:25-26; 9:35-38.

36. D. \& C. 76:19-24.

37. \textit{Ibid.}, 110:1-4.

38. Rev. 1:12-18.

39. For a discussion of this controversy the reader is referred to \textit{The Divine Church}, the
manuals by Elder James L. Barker, written for the Priesthood quorums of the Church for the
years 1952-1954.

40. In most of its substance, or declarations, this creed was repeated at various times at
conclaves or councils. In the year 381 A.D., such a council was held in Constantinople, and
this creed with some modifications was again considered. In relation to this council the
action taken is known as the "Nicene and Niceno-Constantinople Creed."

