\chapter{THE DOCTRINE OF GOD—1}

IN the wonderful soul-touching prayer of our Lord shortly before he was taken as a prisoner
before the Sanhedrin to be condemned and delivered to Pilate to be crucified for the sins of
the world, he said:

Father, the hour is come; glorify thy Son, that thy Son may also glorify thee:

As thou hast given him power over all flesh, that he should give eternal life to as many as
thou hast given him.

\textit{And this is life eternal, that they might know thee the only true God, and Jesus Christ, whom
thou hast sent.}

I have glorified thee on the earth: I have finished the work which thou gavest me to do.

And now, O Father, glorify thou me with thine own self with the glory which I had with thee
before the world was. 1

John, the beloved disciple, who knew the Savior perhaps better than any other of the apostles,
and who was appointed to write of his ministry and of the great Apocalypse 2 had this to say
of Christ as translated in the King James version of the Bible:

In the beginning was the Word, and the Word was with God, and the Word was God.

The same was in the beginning with God.

All things were made by him; and without him was not anything made that was made.

In him was life; and the life was the light of men.

And the light shineth in darkness; and the darkness comprehended it not.

In the account revealed by the Lord to the Prophet Joseph Smith, it reads:

In the beginning was the gospel preached through the Son. And the gospel was the word, and
the word was with the Son, and the Son was with God, and the Son was God.

The same was in the beginning with God.

All things were made by him; and without him was not anything made which was made.

In him was the gospel, and the gospel was the life, and the life was the light of men;

And the light shineth in the world, and the world perceiveth it not. 3

In substance these interpretations mean practically the same. They teach that Jesus Christ is
the Son of God and that he was in the beginning with the Father, and by him the world was
created, and all things were made (organized) by him.

Our learned enemies of the Gospel and the divine mission of Jesus Christ, in their
overwhelming conceit, will not have it so. They have rejected Jesus Christ as the Son of God.
They have rejected him as the Redeemer of the world and the Savior of men. They have
denied his Godhood and his resurrection, while they condescendingly permit him to be a
merciful and moral teacher of mankind. In fact they think they have learned through their
study of science that God could not possibly be an exalted man. He could not be the Father of
Jesus Christ in the literal sense as being the Father of his body, for to them God is a force, an
influence, a mystery which no man can solve. They have universally ridiculed the idea that
there can be an Eternal Father, who could create man in his own image, in the likeness of his
body, for their god has no body, he is too great, too mighty, to be confined in a body limited
to the dimensions of a man. They glibly speak of the "God of nature," 4 the "God of
Science," 5 but he is a mystery that cannot be understood. The fact that he cannot be
understood is, to them, his great glory. He must be some force or controlling principle that
keeps the universe in order, for we have an "orderly universe." Yet their god is constantly at
work tearing down and re-creating worlds. In some respects, their doctrine is very closely
related to that of the sectarian world, which they so roundly criticize. But that the Almighty,
the Maker of the Universe, could be an anthropomorphic being, is too abusrd. They say this
in spite of all that has been revealed. Let us present some of these very profound criticisms:

Indeed, I am convinced that nothing in our time is so dangerous to the belief in God and his
kingdom, at least in scientific circles, as the attempt to foist anthropomorphic theism upon
the understanding, as a scientifically necessary theory of the universe, by means of
antiquated arguments that conflict with natural scientific investigation. (It will be seen later
on in what sense anthropomorphic theism has been and always will be impossible.)
Ignorance was ever a weak support; to attempt to cling to it looks like a tendency to
obscurantism, which makes ignorance the basis of clerical domination: \textit{nam sciunt, quod
sublata ignorantia stupor, h.e. unicum argumentandi twendaeque autoritatis medium, quod
habent, tollitur} (Spinoza, Eth., I, Appendix). 6

Jesus struck the most mortal blow that has ever been struck at all childish literalisms, at all
the ideas which underlie modern so-called fundamentalism, when he changed the literalistic
interpretation of the Jewish scriptures, the anthropomorphic conception of God prevalent up
to his time, and saw in God no longer merely a powerful human being, but a being whose
qualities transcend all human qualities; when he cried, "It hath been written . . . \textit{but} I say unto
you"; when he saw a great benevolence behind the universe; when he taught "God is a spirit";
when he said, "The kingdom of heaven is within you." . . .

Jesus had gone a long way toward destroying or refining man's primitive childish conception
of a capricious anthropomorphic God. . . .

Here was another divine event, the third stage in the evolution of man's conception of God
and, as an inevitable consequence, of his conception of duty. 7

Above all, I abstain from commenting on the Patristic conception of the Almighty; they are
too anthropomorphic, and wanting in sublimity. 8

Nestor rejected the base popular anthropomorphism, looking upon it as little better than
blasphemous, and pictured to himself an awful eternal Divinity, who pervaded the universe,
and had none of the aspects or attributes of man. . . . 9

It is enough now to remark that their heaven [i.e. Mohammedism] was arranged in seven
stories, and was only a place of Oriental carnal delight. It was filled with black-eyed
concubines and servants. The form of God was, perhaps, more awful than that of paganized
Christianity. Anthropomorphism will, however, never be obliterated from the ideas of the
unintellectual. Their God, at best, will never be anything more than the gigantic shadow of a
man—a vast phantom of humanity—like one of those Alpine spectres seen in the midst of
the clouds, by him who turns his back on the sun. 10

In the two accounts imperfectly fused together in Genesis, and also in the account of which
we have indications in the book of Job and in the Proverbs, there is presented, often with the
greatest sublimity, the same early conception of the Creator and of the creation—the
conception, so natural in the childhood of civilization, of a Creator who is an enlarged human
being working literally with his own hands, and of a creation which is "the work of his
fingers." To supplement this view there was developed the belief in this Creator as one, who
having \textit{"From his ample palm}

Launched forth the rolling planets into space,"

sits on high, enthroned "upon the circle of the heavens" perpetually controlling and directing
them.

From this idea of creation was evolved in time a somewhat nobler view. Ancient thinkers and
especially, as it is now found in Egypt, suggested that the main agency in creation was not
the hands and fingers of the Creator, but his \textit{voice}. 11 . . .

We have seen, thus far, how there came into the thinking of mankind upon the visible
universe and its inhabitants the idea of a creation virtually instantaneous and complete, and
of a Creator in human form with human attributes, who spoke matter into existence literally
by the exercise of his throat and lips, or shape and placed it with his hands and fingers.

We have seen that this view came from far; that it existed in the Chaldaeo-Babylonian and
Egyptian civilizations, and probably in others of the earliest date known to us; that its main
features passed thence into the sacred books of the Hebrews and then into the early Christian
Church, by whose theologians it was developed through the Middle Ages and maintained
during the modern period. 12

He who visits the tomb of Linnaeus today, entering the beautiful cathedral of Upsala by its
southern porch, sees above it, wrought in stone, the Hebrew legend of creation. In a series of
medallions, the Almighty—in human form—accomplished the work of each creative day. In
due order he puts in place the solid firmament with the waters above it, the sun, moon, and
stars within it, the beasts, birds, and plants below it, and finishes his task by taking man out
of a little hillock of "the earth beneath," and woman out of man's side. Doubtless Linnaeus,
as he went to his devotions, often smiled at the child-like portrayal. Yet he was never able to
break away from the idea it embodied. 13

So we might go on with innumerable quotations from these self-righteous learned men,
wherein they ridicule the thought that an Exalted Man, in whose form all men are created, is
in very deed the Almighty Ruler of the universe. They ridicule the Holy Scriptures in the
numerous declarations to this effect. They poke fun at the prophets who proclaim it, and who
testify that they have seen him, conversed with him and have received his commandments.
No! all of this is too childish for these great men. All such notions were borrowed from the
Chaldeans or Babylonians. Moses copied such things from what he learned in Egypt and
from what was gathered from the mythologies of the other ancient nations. And so, wittingly
or unwittingly, in their ignorance, they do all in their power to lead mankind astray. It is our
duty, therefore, to show by evidence, that cannot be gainsaid, that God the Father lives; that
Jesus Christ is his only begotten Son, and was born into this world in the express image of
his Father. It has been only two or three days from the time I am writing, that I heard a man
on the radio, declaring his faith and he ridiculed the idea of a personal God. He said he could
not believe that God is like a "man six feet tall and weighing two hundred pounds." He
thought, no doubt, that such an expression was very convincing.

First, let us present a few reflections. The Lord in keeping with his promise given to his
servants the prophets, has in these latter days poured out his Spirit upon all flesh. We must
not be confused by this prediction by the Prophet Joel. It is not the Holy Ghost who guides
into all truth that the world has received; but an awaking through the Spirit of Christ, or Light
of Truth, which has been given to all men and of which the Lord has said:

He that ascended up on high, as also he descended below all things, in that he comprehended
all things, that he might be in all and through all things, the light of truth;

Which truth shineth. This is the light of Christ. As also he is in the sun, and the light of the
sun, and the power thereof by which it was made.

As also he is in the moon, and is the light of the moon, and the power thereof by which it was
made:

As also the light of the stars, and the power thereof by which they were made;

And the earth also, and the power thereof, even the earth upon which you stand.

And the light which shineth, which giveth you light, is through him who enlighteneth your
eyes, which is the same light that quickeneth your understanding;

Which light proceedeth forth from the presence of God to fill the immensity of space—

The light which is in all things, which giveth life to all things, which is the law by which all
things are governed, even the power of God who sitteth upon his throne, who is in the bosom
of eternity, who is in the midst of all things. 14.

This is the light spoken of by Joel, which has been poured out in these latter days and
through which men are inspired to invent and discover the great truths which, until now, the
Lord has seen fit to keep hid from the inhabitants of the world. We see the marvelous things
which the Lord has revealed to man in this dispensation of the Fulness of Times, all in
preparation for the restoration of all things.

Man possesses the faculty to increase in knowledge, wisdom and power.

He has subdued the earth and rides upon the air and on the seas.

He has harnessed the lightnings and the cataracts and made them serve him.

His inventive genius has brought the forces of nature to obey him.

He has discovered hidden secrets of the universe.

He builds great structures reaching into the heavens, and improves and beautifies his
surroundings.

He has taken advantage of the knowledge of past ages and by his observation and adaptation
has increased his knowledge and power.

He has developed to a high degree the gift of reason.

He has inherited the gift of speech and conveys his thoughts to his fellow man, in a complex
language, both written and oral.

He has learned to send his thoughts out upon the ethereal waves almost instantly to all parts
of the earth.

Neither land nor sea stands in the way of his communications.

In his higher civilization he possesses an esthetic nature. He loves the beautiful and
appreciates things lovely and artistic. By these qualities he is able to touch the hearts of his
fellow men and sway them in their emotions. All of these powers are increased as he draws
nearer to his Creator and Father. When he forgets the source of all these qualifications and
turns from his God, then are these blessings impaired and he sinks in ignorance and sin.
Without the guidance of the Divine Presence, whence he comes, he becomes a slave to
savagery and debased ignorance, for it is the Spirit of Truth which enlightens and sustains. 15

When we see the wonderful, precise instruments that have been invented, by which the
wonders of nature are made known, the wonders of radar and other discoveries by which
elements are controlled and many of the secrets, previously hidden from the world, now
brought to light for the benefit and use of man; things too numerous to mention, even if we
had the power to mention them; when we think of the great discoveries by which disease in
many of its deadly forms has been controlled, and then think of the possibilities that still lie
ahead of us, do we not marvel and wonder at the possibilities that are open even to mortal
man? Yet the life span of man is limited to a few score years. If he can accomplish so much
in so short a time, why should we deny to an exalted Eternal Man, who has the countless ages
behind him, the power to create worlds and direct them in their courses? Why should we
doubt his power to grant to his offspring, through his eternal blessings, the powers and
privileges of becoming like him? Why should we think that all such wonderful things as the
directing and creating of a universe must be done by some mysterious, unknowable, ethereal
power that cannot be understood? Moreover, why should any man, because he lacks the
inspiration which comes from the Divine Eternal Father and his Son Jesus Christ, spend histime denying that any one else can know and feel the divine inspiration? Because he, through
his stubborn and hard-hearted indifference to spiritual things, feels it his bounden duty to
deny that a personal God exists, and hold in contempt any who say they know these things!
The great discoveries revealed to the scientist and inventor should make them humble. So
close have we come in these modern days to the hidden secrets of nature, through the radio,
television and other instruments that the prayerful god-fearing man feels that the hidden veil
separating the physical and the spiritual is becoming thinner and thinner all the time. And we
who know realize that the time will come when it will disappear entirely, for the promise is
made that when he comes, the Lord will reveal all things. 16

Now let us present the evidence of God from scripture and from personal manifestations. In
the first chapter of Genesis we read that God created man in his own image. 17 Our scientific
evolutionists say dogmatically that this and all else in Genesis regarding the creation was
taken from the Assyrian-Babylonian civilization. This is not true. The writings in Genesis go
back to the beginning long before there were Assyrians and Babylonians and Egyptians. We
Latter-day Saints, members of the Church of Jesus Christ, should be extremely thankful for
the revelations the Lord has given us. We have learned, contrary to the prevailing notion of
the scholars, that Adam had a perfect language, his children were taught, "And a book of
remembrance was kept, in the which was recorded, in the language of Adam, for it was given
unto as many as called upon God to write by the spirit of inspiration; And by them their
children were taught to read and write, having a language which was pure and undefiled." 18
Enoch kept a record. The Lord revealed to him in the language of Adam, the history of this
world from its beginning to its end. 19 Moses also had a similar revelation. 20 These records
kept in the days of Adam and Enoch, were preserved and were given into the hands of
Abraham, and were preserved by him, so that he had a knowledge of the creation as it was
kept, without question, in the language of Adam. 21 By Abraham they were preserved, so it
is a ridiculous fabrication, cherished as a choice morsel by our critics, that the knowledge
given in Genesis is a poorly constructed account of Babylonian mythology. The Lord has
revealed to us that these things were known by Moses, but many of the precious things which
he recorded were taken out of his record, but they are to be revealed, and in large measure
have been revealed. 22

In the Scriptures there are numerous accounts of the Lord revealing himself in person to his
servants the prophets. He appeared to Enoch. 23 He appeared to Abraham. 24 He appeared to
Jacob. 25 He appeared to Moses face to face. 26 He spoke to Isaiah, 27 and many others.
But, says, Dr. Millikan:

The Bible story says, "God spoke to Abraham." How did he speak? Through some Arab
sheik who just then passed that way? Then it was the sheik rather than Abraham to whom
God spoke. Through a voice that would have left a record on a phonograph concealed in the
bushes? Who wants such a childish interpretation? Or was it through the still small voice of
reflection? But even so, where did that idea come from that got into Abraham's mind? I do
not know. The most amazing thing in all life, the greatest miracle there is, is the fact that a
mind has got here at all, "created out of the dust of the earth." . . .

God spoke to Abraham. I do not know any better way in which the modern man can put it,
and certainly primitive man with his animistic and anthropomorphic conceptions literally had
no other way in which he could have stated it. God spoke to Lycurgus, too, when the Spartanlawgiver ordered human sacrifices stopped in Sparta, and at a time not many centuries after
that at which Abraham had them stopped in Palestine! 28

Then the learned doctor goes on to say that Abraham and Lycurgus were much alike in that
neither could do much more than a little faltering first step in getting away from the
"anthropomorphic conception" of God. Abraham \textit{thought} God might not be pleased with the
sacrifice of Isaac, but might be pleased if Isaac was replaced by a goat or a sheep, and out of
this a "whole religion grew up," around the notion that God, or the gods, "could be
propitiated with the sacrifice of animals." Here we have a very learned man, ridiculing the
fact that the Lord did in very deed speak with Abraham. It was merely Abraham's
imagination, a thought which struck him from that mysterious place from whence thoughts
come. Moreover the inference is laid that Abraham \textit{thought} the Lord stopped him from
offering his son Isaac as a human sacrifice and instructed him to use a goat or a sheep, and
from this developed the practice of animal sacrifice. How little these wise men know of the
law of sacrifice, and in Abraham's day they had to express themselves, "with little, faltering
footsteps in getting away from this manlike or anthropomorphic conception of the deity." So,
we see, these learned men, so profuse in their criticism of Bible teachings and divine
revelation, prefer to believe that God is a great mystery, a force, an influence, a controlling
power that cannot be defined or comprehended, notwithstanding the Savior's edict, that it is
life eternal to know both God and Jesus Christ whom the Father sent.

Well has William Jennings Bryan spoken of these higher critics of the Bible:

Besides open enemies, the Bible has enemies who are less frank—enemies who, while
claiming to be friends of Christianity, spend their time undermining faith in God, faith in the
Bible, and faith in Christ. These professed friends call themselves higher critics—a title
which—though explained by them as purely technical—smacks of an insufferable egotism.
They assume an air of superior intelligence and look down with mingled pity and contempt
upon what they regard as poor, credulous humanity. The higher critic is more dangerous than
the open enemy. The atheist approaches you boldly and tries to blow out your light, but, as
you know who he is, what he is trying to do and why, you can protect yourself. The higher
critic, however, comes to you in the guise of a friend and politely inquires: "Isn't the light too
near your eyes? I fear it will injure your sight." Then he moves the light away, a little at a
time, until it is only a speck and then—invisible.

Some who have used the title "higher critic" have approached their subject in a reverent spirit
and labored earnestly in the vain hope of satisfying intellectual doubts, when the real trouble
has been with the hearts of objectors rather than with their heads. Religion is a matter of the
heart, and the impulses of the heart often seem foolish to the mind. Faith is different from,
and superior to, reason. Faith is a spiritual extension of the vision—a moral sense that
reaches out toward the throne of God and takes hold of verities that the mind cannot grasp. It
is like "the blind leading the blind" for a higher critic, however honest, to rely on purely
intellectual methods to convey truths that are "spiritually discerned."

As a rule, however, the so-called higher critic is a man without spiritual vision, without zeal
for souls and without any deep interest in the coming of God's Kingdom. He toils not in the
Master's vineyard and yet, "Solomon in all his glory" never laid claim to such wisdom as he
boasts. He does not accept the Bible nor defend it; he mutilates it. He puts the Bible on theoperating table and cuts out the parts that he thinks are "diseased." When he has finished his
work the Bible is no longer the Book of books: it is simply a "scrap of paper." 29

We will now examine the evidence in the Bible and the testimony of ancient witnesses to
show that Jesus Christ is in very deed the begotten Son of God, and that the Father, the Son
and the Holy Ghost are three separate personages. Matthew, one of the apostles of our Lord,
speaking of the baptism of our Savior records this as translated in the King James, or
"Authorized" translation:

Then cometh Jesus from Galilee to Jordan unto John, to be baptized of him.

But John forbad him, saying, I have need to be baptized of thee, and comest thou to me?

And Jesus answering said unto him, Suffer it to be so now: for thus it becometh us to fulfill
all righteousness. Then he suffered him.

And Jesus, when he was baptized, went up straightway out of the water: and, lo, the heavens
were opened unto him, and he saw the Spirit of God descending like a dove, and lighting
upon him:

And lo a voice from heaven, saying, This is my beloved Son, in whom I am well pleased. 30

Mark bears witness to this event in like manner:

And straightway coming up out of the water, he saw the heavens opened, and the Spirit like a
dove descending upon him:

And there came a voice from heaven, saying, Thou art my beloved Son, in whom I am well
pleased.

And immediately the spirit driveth him into the wilderness. 31

In these accounts we have the Father, the Son and the Holy Ghost, manifest as three separate
Personages. These passages do not indicate, as many have presumed, that the Holy Ghost
was in the form of a dove, but that he descended in the manner of a dove. The Prophet Joseph
Smith has written:

The Holy Ghost is a personage and is in the form of a personage. It does not confine itself to
the form of the dove, but in the sign of a dove. The Holy Ghost cannot be transformed into
the form of a dove; but the sign of a dove was given to John to signify the truth of the deed
[baptism of Christ], as the dove is an emblem or token of truth and innocence. 32

And Simon Peter answered and said, Thou art the Christ, the Son of the living God.

And Jesus answered and said unto him, Blessed art thou, Simon Barjona: for flesh and blood
hath not revealed it unto thee, but my Father which is in heaven.

And I say also unto thee, That thou art Peter, and upon this rock I will build my church; and
the gates of hell shall not prevail against it. . . .

For the Son of man shall come in the glory of his Father with his angels; and then he shall
reward every man according to his works. 33

And, behold, thou shalt conceive in thy womb, and bring forth a son, and shall call his name
JESUS.

He shall be great, and shall be called the Son of the Highest: and the Lord God shall give
unto him the throne of his father David:

And he shall reign over the house of Jacob for ever; and of his kingdom there shall be no end.

Then said Mary unto the angel, How shall this be, seeing I know not a man?

And the angel answered and said unto her, The Holy Ghost shall come upon thee, and the
power of the Highest shall overshadow thee: therefore also that holy thing which shall be
born of thee shall be called the Son of God. 34

Then said they all, Art thou then the Son of God? And he said unto them, Ye say that I am.

And they said, What need we any further witness? for we ourselves have heard of his own
mouth. 35

For God so loved the world, that he gave his only begotten Son, that whosoever believeth in
him shall not perish, but have everlasting life.

For God sent not his Son into the world to condemn the world; but that the world through
him might be saved.

He that believeth on him is not condemned: but he that believeth not is condemned already,
because he hath not believed in the name of the only begotten Son of God. 36

For he whom God hath sent speaketh the words of God: for God giveth not the Spirit by
measure unto him.

The Father loveth the Son, and hath given all things into his hand.

He that believeth on the Son hath everlasting life: and he that believeth not the Son shall not
see life; but the wrath of God abideth on him. 37

The woman saith unto him, I know that Messias cometh, which is called Christ: when he is
come, he will tell us all things.

Jesus saith unto her, I that speak unto thee am he. 38

But Jesus answered them, My Father worketh hitherto, and I work.

Therefore the Jews sought the more to kill him, because he not only had broken the sabbath,
but said also that God was his Father, making himself equal with God.

Then answered Jesus and said unto them, Verily, verily, I say unto you, The Son can do
nothing of himself, but what he seeth the Father do: for what things soever he doeth, these
also doeth the Son likewise.

For the Father loveth the Son, and sheweth him all things that himself doeth: and he will
shew him greater works than these that ye marvel.

For as the Father raiseth up the dead, and quickened them; even so the Son quickeneth whom
he will.

For the Father judgeth no man, but hath committed all judgment unto the Son:

That all men should honour the Son, even as they honour the Father. He that honoureth not
the Son honoureth not the Father which hath sent him. 39

If this man were not of God, he could do nothing.

They answered and said unto him, Thou wast altogether born in sins, and dost thou teach us?
And they cast him out.

Jesus heard that they had cast him out; and when he had found him, he said unto him, Dost
thou believe on the Son of God?

He answered and said, Who is he, Lord, that I might believe on him?

And Jesus said unto him, Thou hast both seen him, and it is he that talketh with thee.

And he said, Lord, I believe. And he worshipped him. 40

Now is my soul troubled; and what shall I say? Father, save me from this hour: but for this
cause came I unto this hour.

Father, glorify thy name. Then came there a voice from heaven, saying, I have both glorified
it, and will glorify it again.

The people therefore, that stood by, and heard it, said that it thundered: others said, An Angel
spake to him.

Jesus answered and said, This voice came not because of me, but for your sakes. 41

For the Father himself loveth you, because ye have loved me, and have believed that I came
out from God.

I came forth from the Father, and am come into the world: again, I leave the world, and go to
the Father. 42

And this is life eternal, that they might know thee the only true God, and Jesus Christ, whom
thou hast sent.

I have glorified thee on the earth: I have finished the work which thou gavest me to do.

And now, O Father, glorify thou me with thine own self with the glory which I had with thee
before the world was. 43

Jesus saith unto her, Touch me not; for I am not yet ascended to my Father: but go to my
brethren, and say unto them, I ascend unto my Father, and your Father; and to my God, and
your God. 44

Then saith he to Thomas, Reach hither thy finger, and behold my hands; and reach hither thy
hand, and thrust it into my side: and be not faithless, but believing.

And Thomas answered and said unto him, My Lord and my God. 45

But these are written, that ye might believe that Jesus is the Christ, the Son of God; and that
believing ye might have life through his name. 46

God, who at sundry times and in divers manners spake in time past unto the fathers by the
prophets,

Hath in these last days spoken unto us by his Son, whom he hath appointed heir of all things,
by whom also he made the worlds;

Who being the brightness of his glory, and the express image of his person, and upholding all
things by the word of his power, when he had by himself purged our sins, sat down on the
right hand of the Majesty on high;

Being made so much better than the angels, as he hath by inheritance obtained a more
excellent name than they.

For unto which of the angels said he at any time, Thou art my son, this day have I begotten
thee? And again, I will be to him a Father, and he shall be to me a Son?

And again, when he bringeth in the firstbegotten into the world, he saith, And let all the
angels of God worship him. 47

And the Word was made flesh, and dwelt among us, (and we beheld his glory, the glory as of
the only begotten of the Father,) full of grace and truth. 48

Thomas saith unto him, Lord, we know not whither thou goest; and how can we know the
way?

Jesus saith unto him, I am the way, the truth, and the life: no man cometh unto the Father, but
by me.

If ye had known me, ye should have known my Father also: and from henceforth ye know
him, and have seen him.

Philip saith unto him, Lord, shew us the Father, and it sufficeth us.

Jesus saith unto him, Have I been so long time with you, and yet hast thou not known me,
Philip? he that hath seen me that seen the Father; and how sayest thou then, Shew us the
Father?

Believest thou not that I am in the Father, and the Father in me? the words that I speak unto
you I speak not of myself: but the Father that dwelleth in me, he doeth the works. 49

But if our gospel be hid, it is hid to them that are lost:

In whom the god of this world hath blinded the minds of them which believe not, lest the
light of the glorious gospel of Christ, who is the image of God, should shine unto them. 50

Giving thanks unto the Father, which hath made us meet to be partakers of the inheritance of
the saints in light:

Who hath delivered us from the power of darkness, and hath translated us into the kingdom
of his dear Son:

In whom we have redemption through his blood, even the forgiveness of sins:

Who is the image of the invisible God, the firstborn of every creature:

For by him were all things created, that are in heaven, and that are in earth, visible, and
invisible, whether they be thrones, or dominions, or principalities, or powers: all things were
created by him, and for him:

And he is before all things, and by him all things consist.

And he is the head of the body, the church: who is the beginning, the firstborn from the dead;
that in all things he might have the preeminence.

For it pleased the Father that in him should all fulness dwell. 51

Blessed be the God and Father of our Lord Jesus Christ, which according to his abundant
mercy hath begotten us again unto a lively hope by the resurrection of Jesus Christ from the
dead. 52

This is quite an array of scriptural references that may appear somewhat tedious to some
readers, but they are presented to show the evidence in the scriptures, that the Father, the Son
and the Holy Ghost, are three separate personages. Moreover, to show that Jesus Christ is the
Messiah; the only begotten Son of God in the flesh. I have presented only a few out of the
many scriptural evidences that could be arrayed showing these truths are found abundantly in
the Bible. How strange it is that learned men, filled with the theories of this modern world,
fail to see and comprehend these truths. They have blinded themselves by hugging to their
bosoms impossible and ridiculous theories concerning God based upon organic evolution.
Therefore to maintain their views it becomes necessary for them to reject God as an
anthropomorphic being and the Father of Jesus Christ; and no matter how overwhelming the
evidence of his resurrection, that also must meet with their scoffing and rejection. The
testimonies of those who both heard and saw, who conversed with our Lord; walked withhim in his ministry; saw him go to his death and were eyewitnesses of his resurrection and
ascension into heaven, by these characters must be denied. Their lame excuse, or one of
them, being that the stories told by Matthew, Mark, Luke, John, Peter, Paul and others of the
New Testament writers were written so long after the events were supposed to have taken
place that they were augmented by these miraculous stories which really never happened.
Therefore they refuse to accept Jesus as the Messiah, the giver of life to man through his
resurrection. Many of them are willing to accede that he was a great teacher, but are not
willing to bow their knees and give him reverence. In the hardness of their hearts and in their
love of soul-destroying doctrines they rejoice and mock at his atonement. Here are a few
samples of their pernicious doctrines:

Theology has much to say about original sin. This original sin is neither more nor less than
the brute-inheritance which every man carries with him. 53

If mankind have been slowly developing out of ape-like ancestors, then what is called sin
consists of nothing but the tendencies which they have inherited from these ancestors; there
never was a state of primeval innocence, and all the nations of the world have developed out
of primitive man by processes as natural as those which gave rise to the Jews. 54

As a matter of fact, the high man of today is not worrying about his sins at all, still less about
his punishment. His mission, if he is good for anything, is to be up and doing, and insofar as
he acts wrongfully or unwisely he expects to suffer. He may unconsciously plead for
mitigation on the grounds of good intentions, but never either consciously or unconsciously
will anyone but a cur ask for the punishment to fall on someone else nor rejoice when told
that it has already fallen. 55

This view, growing out of the myths, legends, and theologies of earlier peoples, we also find
embodied in the sacred tradition of the Jews, and especially in one of the documents which
form the impressive poem beginning the books attributed to Moses. As to the Christian
Church, no word of its Blessed Founder indicates that it was committed by him to this theory,
or that he even thought it worthy of his attention. How, like so many other dogmas never
dreamed of by Jesus of Nazareth and those who knew him best, it was developed, it does not
lie within the province of this chapter to point out; nor is it worth our while to dwell upon its
evolution in the early Church, in the Middle Ages, at the Reformation, and in various
branches of the Protestant Church: suffice it that, though among English-speaking nations by
far the most important influence in its favor has come from Milton's inspiration rather than
from that of older sacred books, no doctrine has been more universally accepted "always,
everywhere, and by all," from the earliest fathers of the Church down to the present hour.

On the other hand appeared at an early period the opposite view—that mankind, instead of
having fallen from a high intellectual, moral, and religious condition, has slowly risen from
low and brutal beginnings. 56

In this chapter [i.e. \textit{The "Fall of Man" and Anthropology}] I propose to present some outlines
of the work of Anthropology, especially as assisted by Ethnology, in showing what the
evolution of human civilization has been.

Here, too, the change from the old theological view based upon the letter of our sacred books
to the modern scientific view based upon evidence absolutely irrefragable is complete. Here,too, we are at the beginning of a vast change in the basis and modes of thought upon man—a
change even more striking than that accomplished by Copernicus and Galileo, when they
substituted for a universe in which sun and planets revolved about the earth a universe in
which the earth is but the merest grain or atom revolving with other worlds, larger and
smaller, about the sun; and all these forming but one among innumerable systems.

Ever since the beginning of man's effective thinking upon the great problems around him,
two antagonistic views have existed regarding the life of the human race upon the earth. The
first of these is the belief that man was created "in the beginning" a perfect being, endowed
with the highest moral and intellectual powers, but that there came a "fall," and, as its result,
the entrance into the world of evil, toil, sorrow, and death.

Nothing could be more natural than such an explanation of the existence of evil, in times
when men saw everywhere miracle and nowhere law. It is, under such circumstances, by far
the most easy of explanations, for it is in accordance with the appearances of things: men
adopted it just as naturally as they adopted the theory that the Almighty hangs up the stars as
lights in the solid firmament above the earth, or hides the sun behind a mountain at night, or
wheels the planets around the earth, or flings comets as "signs and wonders" to scare a
wicked world, or allow evil spirits to control thunder, lightning, and storm, and to cause
diseases of body and mind, or open the "windows of heaven" to let down "the waters that be
above the heavens," and thus to give rain upon the earth.

A belief, then, in a primeval period of innocence and perfection—moral, intellectual, and
physical—from which men for some fault fell, is perfectly in accordance with what we
should expect. 57

The church declared that the earth is the central and most important body in the Universe;
that the sun and moon and stars are tributary to it. 58 On these points she was worsted by
astronomy. She affirmed that a universal deluge had covered the earth; that the only
surviving animals were such as had been saved in an ark. In this her error was established by
geology. She taught that there was a first man, who, some six or eight thousand years ago,
was suddenly created or called into existence in a condition of physical and moral perfection,
and from that condition he fell. But anthropology has shown that human beings existed far
back in geological time, and in a savage state but little better than that of the brute.

Many good and well-meaning men have attempted to reconcile the statements of Genesis
with the discoveries of science, but it is in vain. The divergence has increased so much, that it
has become an absolute opposition. One of the antagonists must give way.

May we not, then, be permitted to examine the authenticity of this book, which, since the
second century, has been put forth as the criterion of scientific truth? To maintain itself in a
position so exalted, it must challenge human criticism. 59

Dr. Friedrich Paulsen also proclaimed the false idea that the Church of Jesus Christ and the
Bible were responsible for all the erroneous doctrines that crept into the Catholic Church and
he, like Dr. White and many others, condemns the Bible as well as the prophets and apostles,
for the iniquities and foolish teachings of the popes in the time of the great apostasy. Dr.
Paulsen writes:

The church ought to have learned so much at least from her unfortunate conflict with modern
cosmology in the seventeenth century, that it is under no circumstances advisable for her to
affiliate with any scientific system. When the church made the Aristotelian-Ptolemaic
cosmology an article of faith, she applied the axe to the roots of her faith. Every blow that
struck the false theory also struck the church. The same effect is bound to ensue if the church
declares a certain biological view as a part of her doctrines. The persons who see in
Darwinism the final destruction of religion well illustrate this fact. By removing the Mosaic
account of creation, and Adam and Eve, they say, Darwin has, at the same time, made
superfluous for biology, "The hypothesis of a God," which cosmology had long ago
abandoned. From youth they have been taught to regard the existence of God as proved and
assured by the teleological argument; now they no longer have confidence in the old proof
and consequently reject the thing itself. Nothing is more dangerous to a good cause than false
arguments.

It seems that Darwin himself underwent the same experience. He lost his religion when he
lost confidence in Paley's evidences. He says: "The old argument from design in Nature, as
given by Paley, which formerly seemed to me so conclusive, fails, now that the law of
natural selection has been discovered. We can no longer argue that, for instance, the beautiful
hinge of a bivalve shell must have been made by an intelligent being, like the hinge of a door
by man." "At the present day," he continues, "the most usual argument for the existence of an
intelligent God is drawn from the deep inward conviction and feeling which are experienced
by most persons." Formerly he was led by feelings such as those just referred to, to the firm
conviction of the existence of God and of the immortality of the soul. The grandeur of the
Brazillian forest, he says, used to inspire him with religious awe. "But now the grandest
scenes would not cause any such convictions and feelings to arise in my mind. It may be
truly said that I am like a man who has become color-blind." In another passage he mentions
the fact that his love for poetry has gradually disappeared—a proof of the withering effect
which continual scientific investigation may exert upon the soul! His state was, however,
evidently preconditioned by the original intellectualistic bent of his religious convictions,
formed by his early instruction. He has a feeling of having been cheated by false theories and
proofs, and therefore looks with distrust upon the entire church. This is an every-day
occurrence. Consequently it is a vital question for the church to assume a proper attitude
towards science. The mutual distrust existing between science and the church is fatal to her.
(\textit{Introduction to Philosophy}, by Dr. Friedrich Paulsen, pp. 159-160.)

The quotations in relation to the loss of faith of Charles Darwin is taken from the book,
\textit{Charles Darwin's Life}, by his son, Francis Darwin, page 63. This advice to the church by Dr.
Paulsen, is based, of course, on his understanding gained from the attitude of an uninspired
ecclesiastical organization. He is correct, however, in his conclusion that one who follows the
theories of Darwin, will eventually, like Darwin, lose all faith in God the Eternal Creator. A
person cannot believe that bivalve shells come by chance and hinges of a door have to come
by the act of an intelligent being, and be sound in his thinking. Verily, those who insistently
follow the evolutionary theories, cannot at the same time accept and worship an intelligent
anthropomorphic God!

From the doctrines of these scientific men, we see that they not only make their attack upon
the fall of Adam, but upon the atonement of Jesus Christ. They condemn the Bible and go out
of their way to destroy it, and thus destroy all that is sacred to every believer in the atoning
blood of our Redeemer. Surely they have put their faith in the arm of flesh and held incontempt the living God—the God of Abraham, Isaac and Jacob, and of Joseph Smith and
every true Latter-day Saint. They speak the truth in saying that "one of the antagonists must
give way." Their doctrines may prevail temporally during this benighted age when
wickedness covers the earth like a flood, but the day is near at hand when all these man-made
doctrines shall come to naught and shall pass away, for the Redeemer of this world has
declared it. For the Latter-day Saints, and all other good people who devoutly believe in the
mission of Jesus Christ, there is but one course to pursue, and that is remain humbly and
prayerfully true to their convictions. "The great and dreadful day of the Lord is near, even at
the doors." 60

Ye look and behold the fig trees, and ye see them with your eyes, and ye say when they begin
to shoot forth, and their leaves are yet tender, that summer is now nigh at hand;

Even so it shall be in that day when they shall see all these things, then shall they know that
the hour is nigh.

And it shall come to pass that he that feareth me shall be looking forth for the great day of the
Lord to come, even for the signs of the coming of the Son of Man.

And they shall see signs and wonders, for they shall be shown forth in the heavens above,
and in the earth beneath.

And they shall behold blood, and fire, and vapors of smoke.

And before the day of the Lord shall come, the sun shall be darkened, and the moon be
turned into blood, and the stars fall from heaven. 61

\newpage
REFERENCES—CHAPTER FOUR

Footnotes

1. John 17:1-5.

2. 1 Nephi 14:25-28; Ether 4:16.

3. John 1:1-5.

4. Millikan, Dr. R. A., \textit{Evolution in Science and Religion}, p. 60.

5. \textit{Ibid.}, p. 88.

6. Paulsen, Dr. Friedrich, U. of Berlin, \textit{Introduction to Philosophy}, p. 160.

7. Millikan, Dr. R. A., \textit{Evolution in Science and Religion}, pp. 73, 78, 79.

8. Draper, Dr. J. W., \textit{Conflict Between Religion and Science}, p. 64.

9. \textit{Ibid.}, p. 71.

10. \textit{Ibid.}, p. 86.

11. White, Dr. A. D., \textit{A History of the Warfare of Science with Theology in Christendom}, p 2.

12. \textit{Ibid.}, pp. 49-50.

13. \textit{Ibid.}, pp. 59-60.

14. D. \& C. 88:6-13.

15. Smith, J. F., \textit{The Progress of Man}, pp. 19-20.

16. D. \& C. 101:32-34.

17. Genesis 1:26-27; 5:1; 9:6; James 3:8-9.

18. Moses 6:5-6, 46.

19. Moses 7:67; D. \& C. 107:57.

20. Moses 1:27-29, 40-41.

21. Abraham 1:31.

22. Moses, Chapters 1-8.

23. Gen. 5:22-24;Moses Chapters 6-8.24. Gen. 17:1; 18:1; Abraham 1:15; 2:6.

25. Gen. 28:13; 35:7.

26. Num. 12:5-8.

27. Isaiah 6:1; 1 Kings 22:19; Acts 7:55-56.

28. Millikan, Dr. R. A., \textit{Evolution in Science and Religion}, pp. 69-71.

29. Bryan, W. J., \textit{In His Image}, pp. 40-41.

30. Matt. 3:13-17.

31. Mark 1:10-12.

32. D. H. C., Vol. 5, p. 261; Smith, J. F., \textit{Essentials in Church History}, pp. 334-335.

33. Matt. 16:16, 18, 27.

34. Luke 1:31-35.

35. Luke 22:70-71.

36. John 3:16-18.

37. John 3:34-36.

38. John 4:25-26.

39. John 5:17-23.

40. John 9:33-38.

41. John 12:27-30.

42. John 16:27-28.

43. John 17:3-5.

44. John 20:17.

45. John 20:27.

46. John 20:31.

47. Heb. 1:1-6.

48. John 1:14.49. John 14:5-10.

50. 2 Cor. 4:3-4.

51. Col. 1:12-19.

52. 1 Peter 1:3.

53. Fisk, John, \textit{The Destiny of Man}, p. 103.

54. McBride, Dr. W. U., \textit{The Modern Churchman}, Sept. 1924, p. 242; Smith, J. F., \textit{Signs of
the Times}, p. 31, revised ed.

55. Lodge, Sir Oliver, \textit{Man and the Mineral}, p. 204; Smith, J. F., \textit{Signs of the Times}, pp. 32-
33.

56. White, Dr. A. D., \textit{A History of the Warfare of Science with Theology in Christendom}, pp.
286-287.

57. \textit{Ibid.}, pp. 284-285.

58. The Church taught no such thing. This was the doctrine taught by science and the
apostate church, which had departed from the teachings of the Gospel as instituted by our
Lord and taught by his apostles in the first century of the Christian Era. These ideas
concerning the earth prevailed during the "dark ages" brought upon the people because of the
departure from the teachings of Jesus Christ.

59. Draper, Dr. A. D., \textit{Conflict Between Religion and Science}, pp. 218-219.

60. D. \& C. 110:16.

61. D. \& C. 45:37-42.

