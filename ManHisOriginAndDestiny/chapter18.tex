\chapter{THE FALL AND INFINITE ATONEMENT}

``THEN Martha, as soon as she heard that Jesus was coming, went and met him: but Mary sat
still in the house.

``Then said Martha unto Jesus, Lord, if thou hadst been here, my brother had not died.

``But I know, that even now, whatsoever thou wilt ask of God, God will give it thee.

``Jesus saith unto her, Thy brother shall rise again.

``Martha saith unto him, I know that he shall rise again in the resurrection at the last day.

``Jesus said unto her, I AM THE RESURRECTION, AND THE LIFE: he that believeth in
me, though he were dead, yet shall he live:

``And whosoever liveth and believeth in me shall never die. Believest thou this?

``She saith unto him, Yea, Lord: I believe that thou art the Christ, the Son of God, which
should come into the world.'' (John 11:20-27.)

Every member of the Church should have the same assurance that Martha had. It is their
privilege to know that Jesus Christ is the Only Begotten Son of God, and is the resurrection
and the life. It is likewise their privilege to know that Adam fell and brought death into the
world. 1 The resurrection is a demonstrated fact, and by faith we know that the atonement of
Jesus Christ came because of the fall.

It is impossible for the carnally minded to understand the reason for the fall and likewise
understand the necessity for the atonement of Jesus Christ. It is true that not all the purposes
of our Eternal Father have been revealed to man and there are some things that have to be
received by faith; but these great truths have been made known and we have the assurance
that through the sacrifice made on the cross all mankind and every other creature, even the
earth itself, are redeemed from death and shall receive the resurrection and be restored to
immortal life. Men receive assurance and knowledge due to their faithfulness and adherence
to the commandments of Jesus Christ. Those who reject their Redeemer and refuse to keep
his commandments cannot know and comprehend these eternal truths. Alma explained this to
Zeezrom in the following words:

And now Alma began to expound these things unto him, saying: It is given unto many to
know the mysteries of God; nevertheless they are laid under a strict command that they shall
not impart only according to the portion of his word which he doth grant unto the children of
men, according to the heed and diligence which they give unto him.

And therefore, he that will harden his heart, the same receiveth the lesser portion of the word;
and he that will not harden his heart, to him is given the greater portion of the word, until it is
given unto him to know the mysteries of God until he know them in full.

And they that will harden their hearts, to them is given the lesser portion of the word until
they know nothing concerning his mysteries; and then they are taken captive by the devil,
and led by his will down to destruction. Now this is what is meant by the chains of hell. 2

Naturally the Lord cannot reveal the mysteries of his kingdom to the scoffer, neither can he
to the member of the Church who is not faithful. If a man does not exercise faith why should
he have the revelations concerning the kingdom of God revealed to him? They cannot
understand them because they are ``fallen'' man, and without the enlightening influence of the
Holy Spirit, they are as the Lord expressed it, ``carnal, sensual and devilish.'' 3 It is written
that when the disciples asked the Savior why he spoke in parables, he answered: ``Because it
is given unto you to know the mysteries of the kingdom of heaven, but to them it is not
given. For whosoever hath, to him shall be given, and he shall have more abundance: but
whosoever hath not, from him shall be taken away even that he hath.'' 4 The Lord further
said, ``Give not that which is holy unto dogs, neither cast ye your pearls before swine, lest
they trample them under their feet, and turn again and rend you.'' 5 There were occasions
when he instructed his disciples not to reveal certain manifestations until after his
resurrection.

Of course, a man who believes that man has descended from lower forms of life, and by
gradual development after an enormous length of time evolved from fish to reptile and then
to ape, can never understand the fall of man and the atonement. These truths are mysteries to
him and only contempt and abuse do they receive from him. Here are a few such expressions.

This from Robert Blatchford:

But no Adam, no Fall; no Fall no Atonement; no Atonement no Savior. Accepting evidence,
how can we believe in a Fall? When did man fall; was it before he ceased to be a monkey, or
after? Was it when he was a tree man, or later? Was it in the Stone Age, or the Bronze Age,
or in the Age of Iron? . . . And if there never was a Fall, why should there be any Atonement?
(\textit{God and My Neighbor}, p. 159, Chicago, 1917.)

This from Durant Drake:

What sort of justice is it that could be satisfied with the punishment of one innocent man and
the free pardon of myriads of guilty men? The theory seems a remnant of the ancient idea
that the gods need to be placated; but by the side of pagan gods, who were content with
humble offerings of flesh and fruit, the Christian God, demanding the suffering and death of
his own Son, appears a monster of cruelty. (\textit{Problems of Religion}, p. 176.)

This from John Fisk:

Theology has much to say about original sin. This original sin is neither more nor less than
the brute-inheritance which every man carries with him. (\textit{The Destiny of Man}, p. 103.)

This from Dr. E. W. McBride, at the Oxford Conference of Modern Churchmen:

If mankind have been slowly developing out of ape-like ancestors, then what is called sin
consists of nothing but the tendencies which they have inherited from these ancestors: there
never was a state of primeval innocence, and all the nations of the world have developed out
of primitive man by processes as natural as those which gave rise to the Jews. (\textit{The Modern
Churchman}, September 1924, p. 232.)

This from Dr. H. D. A. Major, also at the Oxford Conference of Modern Churchmen:

Science has shown us that what is popularly called ``original sin'' . . . consists of man's
inheritance from his brute ancestry. (\textit{The Modern Churchman}, p. 206.)

This from Andrew D. White:

. . . The theory of an evolution process in the formation of the universe and of animated
nature is established, and the theory of direct creation is gone forever. In place of it science
has given us conceptions far more noble, and opened the way to an argument for design
infinitely more, beautiful than any ever developed by theology. (\textit{A History of the Warfare of
Science with Theology in Christendom}, Vol. 1, p. 86.)

With this special attack upon geological science by means of the dogma of Adam's fall, the
more general attack by the literal interpretation of the text was continued. The legendary
husks and rinds of our sacred books were insisted upon as equally precious and nutritious
with the great moral and religious truths which they developed. (\textit{Ibid}, pp. 222-223.)

A belief, then, in a primeval period of innocence and perfection— moral, intellectual, and
physical—from which men for some fault fell, is perfectly in accordance with what we
should expect.

Among the earliest known records of our race we find this view taking shape in the Chaldean
legends of war between the gods, and of a fall of man; both of which seemed necessary to
explain the existence of evil. . . . But there came a ``fall,'' caused by human curiosity.
Pandora, the first woman created, received a vase which, by divine command, was to remain
closed; but she was tempted to open it, and troubles, sorrow, and disease escaped into the
world, hope alone remaining. So too, in Roman mythological poetry the well-known picture
of Ovid is but one among the many exhibitions of this same belief in a primeval golden
age—a Saturnian cycle. \textit{Ibid}, Vol. 1, p. 286.)

In previous chapters we have shown by the revelations the Lord has given us that Adam was
placed on the earth not subject to death. The Lord said to him that if he partook of the fruit of
the tree of the knowledge of good and evil he should die. From the words of Lehi we learn
that Adam could have lived forever and all things on the face of the earth likewise, had he
not partaken of that fruit. President Brigham Young said that by partaking of that fruit, Adam
and Eve ``transgressed a command of the Lord, and through that transgression sin came into
the world.'' (Discourses, pp. 157-158.) President Joseph F. Smith said that Adam was
``immortal'' before the fall. Elder Orson Pratt, in the \textit{Times and Seasons}, 1845, and \textit{Journal of
Discourses} 1:280-284, said the same. We are also taught that, not being subject to death,
Adam had no blood in his veins before the fall. Blood is the life of the mortal body. The Lord
so declared it when commanding Noah after the flood. He said:

And the fear of you and the dread of you shall be upon every beast of the earth, and upon
every fowl of the air, upon all that moveth upon the earth, and upon all the fishes of the sea;
into your hands are they delivered.

Every moving thing that liveth shall be meat for you; even as the green herb have I given you
all things.

But flesh with the life thereof, which is the blood thereof, shall ye not eat.

And surely your blood of your lives will I require; at the hand of every beast will I require it,
and at the hand of man; at the hand of every man's brother will I require the life of man.

Whoso sheddeth man's blood, by man shall his blood be shed: for in the image of God made
he man. 6

In giving the law to Moses for Israel the Lord confirmed this commandment and explained
that the blood is the life of the mortal body:

And whatsoever man there be of the house of Israel, or of the strangers that sojourn among
you, that eateth any manner of blood; I will even set my face against that soul that eateth
blood, and will cut him off from among his people.

For the life of the flesh is in the blood: and I have given it to you upon the altar to make an
atonement for the souls: for it is the blood that maketh an atonement for the soul.

Therefore I said unto the children of Israel, No soul of you shall eat blood, neither shall any
stranger that sojourneth among you eat blood.

And whatsoever man there be of the children of Israel, or of the strangers that sojourn among
you, which hunteth and catcheth any beast or fowl that may be eaten; he shall even pour out
the blood thereof, and cover it with dust.

For it is the life of flesh; the blood of it is for the life thereof; therefore I said unto the
children of Israel, Ye shall eat the blood of no manner of flesh: for the life of all flesh is the
blood thereof: whosoever eateth it shall be cut off.

And every soul that eateth that which died of itself, or that which was torn with beasts,
whether it be one of your own country, or a stranger, he shall both wash his clothes, and
bathe himself in water, and be clean until the even: then shall he be clean. 7

There is no blood in an immortal body, and when Adam transgressed the law and ate the fruit
that had been forbidden there came a drastic change in his body and it was transformed from
the condition where there was no death to a condition where it became subject to death, or
mortality, and from that time forth blood was the life-giving fluid.

Some of the clearest explanations of this change that took place and the consequences which
followed are found in the Book of Mormon and the Pearl of Great Price. Lehi points out the
fact that had there been no fall there could have been no fulfilment of the purposes of the
Lord, that is, to people this earth with his children as he had done in other earths, and this
great prophet says had there been no fall, Adam and Eve would have had no children;
``wherefore they would have remained in a state of innocence, having no joy, for they knew
no misery; doing no good, for they knew no sin. But behold, all things have been done in the
wisdom of him who knoweth all things. Adam fell that men might be; and men are, that they
might have joy.'' 8 This same truth was made known to Adam and Eve, and Eve, when
learning that good was to come out of the fall, rejoiced and said: ``Were it not for our
transgression we never should have had seed, and never should have known good and evil,
and the joy of our redemption, and the eternal life which God giveth unto all the obedient.'' 9
So we learn that the fall became a blessing to mankind. Moreover, we have been taught that
this earth-life was essential to the eternal progress of man. The plan for the peopling of the
earth was received with great joy by the majority of the spirit children of our Eternal Father.
We lived in his presence and were acquainted with him. We beheld his glory, for he was an
exalted man—``Man of Holiness is his name,'' 10 and he had a physical body, the tabernacle
for his glorious spirit. We were promised that in coming down to the earth and proving
faithful to this second estate, we could return with added glory 11 and be like him, 12 and
those who fail to obtain this exaltation, yet they will be blessed with their physical
tabernacles to receive blessings far in advance of what they could attain had they remained in
the spirit world, for they will obtain some degree of glory, 13 the one exception being in the
case of those who having had the light rebel and put Christ to open shame.

We have not had revealed to us all the purposes of the Lord, but we know by our faith, that it
was essential that Adam come to this earth a son of God without being subject to mortality
and that it was essential that he partake of mortality as a step on the way to eternal glory and
to become like God. Moreover, we have learned, and can understand, the need of passing
through a probationary state to be tried and tested to see if in this mortal estate we will be
true to every commandment our Eternal Father gives us here. It is by being tried and proved
that we are prepared to have glory added upon our heads for ever and ever, if we remain
faithful; therefore this is a state of probation. Not only did Adam fall ``that men might be,''
but also that men might prove themselves for an eternal reward after their resurrection. The
Lord has prepared places for his children and they will receive rewards and punishments
according to their works and thus find their place in the eternity to come.

Having transgressed the law under which he existed before the fall in the Garden of Eden,
Adam became subject to Satan; that is to say, he, knowing good and evil, was subject to sin
and temptation. This the Lord has said was essential to man's progress, ``And it must needs be
that the devil should tempt the children of men, or they could not be agents unto themselves;
for if they never should have bitter they could not know the sweet.'' 14 From Adam all of his
posterity have become subject to the same conditions which came upon him, and being
subject to sin and the mortal death, all men were under the dominion of Satan. Moreover,
Adam and his posterity were unable to free themselves from this awful condition, for we all
become subject to death without the power to redeem ourselves. Therefore without the help
of someone not subject to death, when death comes we would have been separated spirit and
body forever. Our bodies would have returned to the dust and our spirits would have become
subject to Satan. Jacob, son of Lehi, has discoursed on this point as follows:

For as death hath passed upon all men, to fulfil the merciful plan of the great Creator, there
must needs be a power of resurrection, and the resurrection must needs come unto man by
reason of the fall; and the fall came by reason of transgression; and because man became
fallen they were cut off from the presence of the Lord.

Wherefore, it must needs be an infinite atonement—save it should be an infinite atonement
this corruption could not put on incorruption. Wherefore, the first judgment which came
upon man must needs have remained to an endless duration. And if so, this flesh must have
laid down to rot and to crumble to its mother earth, to rise no more.

O the wisdom of God, his mercy and grace! For behold, if the flesh should rise no more our
spirits must become subject to that angel who fell from before the presence of the Eternal
God, and became the devil, to rise no more.

And our spirits must have become like unto him, and we become devils, angels to a devil, to
be shut out from the presence of our God, and to remain with the father of lies, in misery, like
unto himself; yea, to that being who beguiled our first parents, who transformeth himself
nigh unto an angel of light, and stirreth up the children of men unto secret combinations of
murder and all manner of secret works of darkness.

O how great the goodness of our God, who prepareth a way for our escape from the grasp of
this awful monster; yea, that monster, death and hell, which I call the death of the body, and
also the death of the spirit.

And because of the way of deliverance of our God, the Holy One of Israel, this death, of
which I have spoken, which is the temporal, shall deliver up its dead; which death is the
grave.

And this death of which I have spoken, which is the spiritual death, shall deliver up its dead;
which spiritual death is hell; wherefore, death and hell must deliver up their dead, and hell
must deliver up its captive spirits, and the grave must deliver up its captive bodies, and the
bodies and the spirits of men will be restored one to the other; and it is by the power of the
resurrection of the Holy One of Israel.

O how great the plan of our God! For on the other hand, the paradise of God must deliver up
the spirits of the righteous, and the grave deliver up the body of the righteous; and the spirit
and the body is restored to itself again, and all men become incorruptible, and immortal, and
they are living souls, having a perfect knowledge like unto us in the flesh, save it be that our
knowledge shall be perfect.

Wherefore, we shall have a perfect knowledge of all our guilt, and our uncleanness, and our
nakedness; and the righteous shall have a perfect knowledge of their enjoyment, and their
righteousness, being clothed with purity, yea, even with the robe of righteousness.

And it shall come to pass that when all men shall have passed from this first death unto life,
insomuch as they have become immortal, they must appear before the judgment-seat of the
Holy One of Israel; and then cometh the judgment, and then must they be judged according
to the holy judgment of God. 15

This is a very clear declaration by Jacob which we all should read. It is stated in the right
spirit concerning the mission of Jesus Christ. It is pitiful to know that men possessed with
some degree of intelligence who should be quickened by the Spirit of Christ which is given
to every man, turn from its promptings and from the mission of Jesus Christ with such
wicked contempt and reviling; but, they cannot follow Satan and have faith in Jesus Christ.
We have seen that in the grand council in heaven before the foundation of the earth was laid,
this plan of salvation was made known, to the sons and daughters of God. It was there that
Jesus Christ volunteered to come to this earth and fulfil his mission by the shedding of his
blood to redeem mankind from the fall. It was there that Adam volunteered, or was
appointed, to come and fall, that man might be, and the purpose of the Father be
accomplished. It is reasonable for us to believe that Adam came in the manner in which he
did, and for Christ to come to redeem him and his posterity. All of this was known before the
earth was formed. It is written in the Bible, we are redeemed ``with the precious blood of
Christ, as of a lamb without blemish and without spot: Who verily was foreordained before
the foundation of the world, but was manifest in these last times for you.'' 16 We may be sure
had there been any other way to bring to pass the fall and the redemption the sacrifice by
Jesus Christ would not have been required. That it was required is evident in the fact that he
was chosen before the foundation of the earth was laid. When John saw Jesus coming for
baptism, he said, ``Behold the Lamb of God, which taketh away the sin of the world.'' 17

We have learned from the writings of Moses in the Pearl of Great Price that sacrifice of oxen,
sheep and goats, was introduced in the days of Adam among the first commandments given
to him after the fall. He was instructed that this sacrifice was ``a similitude of the sacrifice of
the Only Begotten of the Father, which is full of grace and truth,'' 18 and from the very
beginning such sacrifice was offered. It was one of the first things done by Noah after
leaving the ark, and commanded by the Lord in Israel in his word to Moses. This practice,
instituted to remind them of the great sacrifice of the ``Lamb of God,'' continued down to the
time of the crucifixion of Jesus Christ when the practice was discontinued because the great
Sacrifice, to which all other sacrifices pointed, had been accomplished. From that day
forward the Lord instituted the sacrament, pointing back to his death upon the cross. It is
unfortunate that apostates from the Church in the earliest times perverted the covenant of
sacrifice and among these apostate peoples the sacrifice of human beings was offered. No
longer did they remember the great sacrifice of Jesus Christ who was to come, but sacrifice
was made to please and appease their false gods, whom they had substituted for the worship
of our Eternal Father. The fact, however, that sacrifice was offered among people everywhere
on the face of the earth, harks back to the time in which it was rightfully done by
commandment of the Lord.

It was said by one distinguished writer that Jesus Christ knew nothing about the doctrine of
the fall. Surely this man did not understand the scriptures. It was at the time of the coming of
Nicodemus that the Savior called attention to the lifting of the brazen serpent in the camps of
Israel and said that ``as Moses lifted up the serpent in the wilderness, even so must the Son of
man be lifted up; That whosoever believeth in him should not perish, but have eternal life.''
19 Constantly during his ministry he called his disciples' attention to the fact that he was to
lay down his life and take it again. He spoke of himself as the water of life and as the bread
of life, and confounded his enemies who said, ``How can this man give us his flesh to eat?''
20 This was in reference to the introduction of the sacrament which was to replace the law of
animal sacrifice. All of these things were said because he knew he was to be the sacrifice for
the sins of the world.

We have some excellent prophecies concerning the atonement of Jesus Christ in the
teachings of the Book of Mormon prophets. Mosiah, the prophet-king, in speaking of Christ
more than one hundred years before his birth said in his instructions to his people:

For behold, the time cometh, and is not far distant, that with power, the Lord Omnipotent
who reigneth, who was, and is from all eternity to all eternity, shall come down from heaven
among the children of men, and shall dwell in a tabernacle of clay, and shall go forth
amongst men, working mighty miracles, such as healing the sick, raising the dead, causing
the lame to walk, the blind to receive their sight, and the deaf to hear, and curing all manner
of diseases.

And he shall cast out devils, or the evil spirits which dwell in the hearts of the children of
men.

And lo, he shall suffer temptations, and pain of body, hunger, thirst, and fatigue, even more
than man can suffer, except it be unto death; for behold, blood cometh from every pore, so
great shall be his anguish for the wickedness and the abominations of his people.

And he shall be called Jesus Christ, the Son of God, the Father of heaven and earth, the
Creator of all things from the beginning; and his mother shall be called Mary.

And lo, he cometh unto his own, that salvation might come unto the children of men even
through faith on his name; and even after all this they shall consider him a man, and say that
he hath a devil, and shall scourge him, and shall crucify him.

And he shall rise the third day from the dead; and behold, he standeth to judge the world; and
behold, all these things are done that a righteous judgment might come upon the children of
men.

For behold, and also his blood atoneth for the sins of those who have fallen by the
transgression of Adam, who have died not knowing the will of God concerning them, or who
have ignorantly sinned.

But wo, wo unto him who knoweth that he rebelleth against God! for salvation cometh to
none such except it be through repentance and faith on the Lord Jesus Christ. 21

Again King Benjamin continued:

I say unto you, if ye have come to a knowledge of the goodness of God, and his matchless
power, and his wisdom, and his patience, and his long-suffering towards the children of men;
and also, the atonement which has been prepared from the foundation of the world, that
thereby salvation might come to him that should put his trust in the Lord, and should be
diligent in keeping his commandments, and continue in the faith even unto the end of his life,
I mean the life of the mortal body—

I say, that this is the man who receiveth salvation, through the atonement which was prepared
from the foundation of the world for all mankind, which ever were since the fall of Adam, or
who are, or who ever shall be, even unto the end of the world.

And this is the means whereby salvation cometh. And there is none other salvation save this
which hath been spoken of; neither are there any conditions whereby man can be saved
except the conditions which I have told you. 22

Alma answering the question of Antionah, a chief ruler who asked:

What is this that thou hast said, that man should rise from the dead and be changed from this
mortal to an immortal state, that the soul can never die?

What does the scripture mean, which saith that God placed cherubim and a flaming sword on
the east of the garden of Eden, lest our first parents should enter and partake of the fruit of
the tree of life, and live forever? And thus we see that there was no possible chance that they
should live forever?

Now Alma said unto him: This is the thing which I was about to explain. Now we see that
Adam did fall by the partaking of the forbidden fruit, according to the word of God; and thus
we see, that by his fall, all mankind became a lost and fallen people.

And now behold, I say unto you that if it had been possible for Adam to have partaken of the
fruit of the tree of life at that time, there would have been no death, and the word would have
been void, making God a liar, for he said: If thou eat thou shalt surely die.

And we see that death comes upon mankind, yea, the death which has been spoken of by
Amulek, which is the temporal death; nevertheless there was a space granted unto man in
which he might repent; therefore this life became a probationary state; a time to prepare to
meet God; a time to prepare for that endless state which has been spoken of by us, which is
after the resurrection of the dead.

Now, if it had not been for the plan of redemption, which was laid from the foundation of the
world, there could have been no resurrection of the dead; but there was a plan of redemption
laid, which shall bring to pass the resurrection of the dead, of which has been spoken.

And now behold, if it were possible that our first parents could have gone forth and partaken
of the tree of life they would have been forever miserable, having no preparatory state; and
thus the plan of redemption would have been frustrated, and the word of God would have
been void, taking none effect.

But behold, it was not so; but it was appointed unto men that they must die; and after death,
they must come to judgment, even that same judgment of which we have spoken, which is
the end.

And after God had appointed that these things should come unto man, behold, then he saw
that it was expedient that man should know concerning the things whereof he had appointed
unto them;

Therefore he sent angels to converse with them, who caused men to behold of his glory.

And they began from that time forth to call on his name; therefore God conversed with men,
and made known unto them the plan of redemption, which had been prepared from the
foundation of the world; and this he made known unto them according to their faith and
repentance and their holy works. 23

Here is the testimony of Samuel the Lamanite, five years before the birth of Jesus Christ:

And behold, he said unto them: Behold, I give unto you a sign; for five years more cometh,
and behold, then cometh the Son of God to redeem all those who shall believe on his name.

And behold, this will I give unto you for a sign at the time of his coming; for behold, there
shall be great lights in heaven, insomuch that in the night before he cometh there shall be no
darkness, insomuch that it shall appear unto man as if it was day.

Therefore, there shall be one day and a night and a day, as if it were one day and there were
no night; and this shall be unto you for a sign; for ye shall know of the rising of the sun and
also of its setting; therefore they shall know of a surety that there shall be two days and a
night; nevertheless the night shall not be darkened; and it shall be the night before he is born.

And behold, there shall a new star arise, such an one as ye never beheld; and this also shall
be a sign unto you.

Samuel then gave them a sign of the crucifixion and death of the Lord:

For behold, he surely must die that salvation may come; yea, it behooveth him and becometh
expedient that he dieth, to bring to pass the resurrection of the dead, that thereby men may be
brought into the presence of the Lord.

Yea, behold, this death bringeth to pass the resurrection, and redeemth all mankind from the
first death—that spiritual death; for all mankind, by the fall of Adam being cut off from the
presence of the Lord, are considered as dead, both as to things temporal and to things
spiritual.

But behold, the resurrection of Christ redeemeth mankind, yea, even all mankind, and
bringeth them back into the presence of the Lord.

Yea, and it bringeth to pass the condition of repentance, that whosoever repenteth the same is
not hewn down and cast into the fire; but whosoever repenteth not is hewn down and cast
into the fire; and there cometh upon them again a spiritual death, yea, a second death, for
they are cut off again as to things pertaining to righteousness. 24

\textit{The testimony of Moroni:}

Behold, he created Adam, and by Adam came the fall of man. And because of the fall of man
came Jesus Christ, even the Father and the Son; and because of Jesus Christ came the
redemption of man.

And because of the redemption of man, which came by Jesus Christ, they are brought back
into the presence of the Lord; yea, this is wherein all men are redeemed, because the death of
Christ bringeth to pass the resurrection, which bringeth to pass a redemption from an endless
sleep, from which sleep all men shall be awakened by the power of God when the trump shall
sound; and they shall come forth, both small and great, and all shall stand before his bar,
being redeemed and loosed from this eternal band of death, which death is a temporal death.

And then cometh the judgment of the Holy One upon them; and then cometh the time that he
that is filthy shall be filthy still; and he that is righteous shall be righteous still; he that is
happy shall be happy still; and he that is unhappy shall be unhappy still. 25

In our own dispensation we have the word of the Lord from his own mouth in relation to his
atonement. Here are a few such statements which will suffice:

Therefore I command you to repent—repent, lest I smite you by the rod of my mouth, and by
my wrath, and by my anger, and your suffering be sore—how sore you know not, how
exquisite you know not, yea, how hard to bear you know not.

For behold, I God, have suffered these things for all, that they might not suffer if they would
repent;

But if they would not repent they must suffer even as I;

Which suffering caused myself, even God, the greatest of all, to tremble because of pain, and
to bleed at every pore, and to suffer both body and spirit—and would that I might not drink
the bitter cup, and shrink—

Nevertheless, glory be to the Father, and I partook and finished my preparations unto the
children of men. 26

Thus saith the Lord your God, even Jesus Christ, the Great I AM, Alpha and Omega, the
beginning and the end, the same which looked upon the wide expanse of eternity, and all the
seraphic hosts of heaven, before the world was made;

The same which knoweth all things, for all things are present before mine eyes;

I am the same which spake, and the world was made, and all things came by me.

I am the same which have taken the Zion of Enoch into mine own bosom; and verily, I say,
even as many as have believed in my name, for I am Christ, and in mine own name, by the
virtue of the blood which I have spilt, have I pleaded before the Father for them. 27

But God hath made known unto our fathers that all men must repent.

And he called upon our father Adam by his own voice, saying: I am God; I made the world,
and men before they were in the flesh.

And he also said unto him: If thou wilt turn unto me, and hearken unto my voice, and
believe, and repent of all thy transgressions, and be baptized, even in water, in the name of
mine Only Begotten Son, who is full of grace and truth, which is Jesus Christ, the only name
which shall be given under heaven, whereby salvation shall come unto the children of men,
ye shall receive the gift of the Holy Ghost, asking all things in his name, and whatsoever ye
shall ask, it shall be given you. 28

Before the fall Adam was in the presence of God and was not subject to death; he and Eve
could have no children, and they knew not good and evil, for all their knowledge of the pre-
existence had been taken away from them. After the fall, Adam and Eve became subject to
the physical or temporal death and were banished from the presence of the Lord thus
partaking of both the temporal and spiritual, or second death, which is banishment from God.
Through baptism and the gift of the Holy Ghost they were reclaimed from the spiritual death.
Moreover, they became parents of a great posterity. They were capable of knowing good and
evil and gained knowledge and were taught the everlasting Gospel. Adam also found himself
in a condition under the broken law, where he could not pay the debt and repair the broken
law. He could not restore either to himself or give to his children the eternal, or immortal life,
that had been taken away. Justice demanded reparation and the restoration of the life that had
been taken away—life free from the seeds of death.

Blood had become the life-giving fluid in Adam's body, and was inherited by his posterity.
Blood was not only the life of the mortal body, but also contained in it the seed of death
which bring the mortal body to its end. Previously the life force in Adam's body, which is
likewise the sustaining power in every immortal body, was the spirit. In order to restore that
immortal condition and destroy the power of the blood, an infinite sacrifice had to be made.
No one subject to death could pay the price, for all mortal beings were under the curse of
mortality. Therefore it was decreed in the heavens before the world was formed that the Only
Begotten Son of God should come and pay the debt demanded by justice and give to man the
blessing of immortality and eternal life. Jacob, son of Lehi, made this declaration:

And he cometh into the world that he may save all men if they will hearken unto his voice;
for behold, he suffereth the pains of all men, yea, the pains of every living creature, both
men, women, and children, who belong to the family of Adam.

And he suffereth this that the resurrection might pass upon all men, that all might stand
before him at the great and judgment day.

And he commandeth all men that they must repent, and be baptized in his name, having
perfect faith in the Holy One of Israel, or they cannot be saved in the kingdom of God.

And if they will not repent and believe in his name, and be baptized in his name, and endure
to the end, they must be damned; for the Lord God, the Holy One of Israel has spoken it.

Wherefore, he has given a law; and where there is no law given there is no punishment; and
where there is no punishment there is no condemnation; and where there is no condemnation
the mercies of the Holy One of Israel have claim upon them, because of the atonement; for
they are delivered by the power of him.

For the atonement satisfieth the demands of his justice upon all those who have not the law
given to them, that they are delivered from that awful monster, death and hell, and the devil,
and the lake of fire and brimstone, which is endless torment; and they are restored to that
God who gave them breath, which is the Holy One of Israel.

But wo unto him that has the law given, yea, that has all the commandments of God, like
unto us, and that transgresseth them, and that wasteth the days of his probation, for awful is
his state!

O that cunning plan of the evil one! O the vainness, and the frailties, and the foolishness of
men! When they are learned they think they are wise, and they hearken not unto the counsel
of God, for they set it aside, supposing they know of themselves, wherefore, their wisdom is
foolishness and it profiteth them not. And they shall perish.

But to be learned is good if they hearken unto the counsels of God. 29

So we see that the atonement of Jesus Christ not only restores man through the resurrection,
uniting inseparably his spirit and his body, but it also redeems all who repent of their sins,
receive the Gospel and endure in faith to the end. The atonement, therefore, is of twofold
nature; it saves all men from the eternal separation of spirit and body which was inflicted
upon them by the fall, and it also saves all who are willing to be obedient to the plan of
salvation and gives them eternal life to become like God.

These words of our Redeemer are extremely significant in relation to his mission and
atonement:

As the Father knoweth me, even so know I the Father: and I lay down my life for the sheep.

And other sheep I have, which are not of this fold: them also I must bring, and they shall hear
my voice; and there shall be one fold, and one shepherd.

Therefore doth my Father love me, because I lay down my life, that I might take it again.

No man taketh it from me, but I lay it down of myself. I have power to lay it down, and I
have power to take it again. This commandment have I received of my Father. 30

Having been born of a mortal mother and an immortal Father, Jesus had inherited the power
over death. He was never subject to death; therefore, was not under the curse of Adam's
transgression. On other occasions he declared that he had life in himself (John 5:24-26.), so
that he could lay his life down and take it again. Being the Son of Mary, he had obtained
from her his blood and the power to lay down his life and by the power coming from his
Father, to take it again. Thus he became the ``resurrection and the life,'' with power to open
the door to eternity and redeem from Satan's power every living creature. It was necessary
that he die by the shedding of his blood, the life-giving power of mortality, for it was by the
blood that mortality came into the world, and by the atonement of Jesus Christ mortality is
destroyed and the debt paid that came through Adam's fall.

\newpage
REFERENCES—CHAPTER EIGHTEEN

Footnotes

1. 2 Nephi 9:5-6.

2. Alma 12:9-11.

3. Moses 3:13; Mosiah 16:3.

4. Matt. 13:11-12.

5. \textit{Ibid.}, 7:6.

6. Gen. 9:2-6.

7. Leviticus 17:10-15.

8. 2 Nephi 2:23-25.

9. Moses 5:11.

10. \textit{Ibid.}, 6:57.

11. Abraham 3:26

12. 1 John 3:1-3.

13. D. \& C. 76:71, 89, 112.

14. \textit{Ibid.}, 29:39; 2 Nephi 2:15-16.

15. 2 Nephi 9:6-15.

16. 1 Peter 1:19-20.

17. John 1:29.

18. Moses 5:7.

19. John 3:14-15.

20. \textit{Ibid.}, 6:52. (Read verses 30-58.)

21. Mosiah 3:5-12.

22. \textit{Ibid.}, 4:6-8.

23. Alma 12:20-30.24. Helaman 14:2-5; 15-18.

25. Mormon 9:12-14.

26. D. \& C. 19:15-19.

27. \textit{Ibid.}, 38:1-4.

28. Moses 6:50-52.

29. 2 Nephi 9:21-29.

30. John 10:15-18.

