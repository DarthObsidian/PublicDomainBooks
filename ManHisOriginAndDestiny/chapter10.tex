\chapter{THE HYPOTHESIS OF ORGANIC EVOLUTION—4}

PAUL, in his epistle to the Corinthian saints, arguing for the resurrection of the dead, and
bearing witness that Jesus Christ had risen from the dead, said, "But God giveth it a body as
it hath pleased him, and to every seed his own body. All flesh is not the same flesh: but there
is one kind of flesh of men, another flesh of beasts, another of fishes, and another of birds."
(1 Cor. 15:38-39.) We do not consider Paul to be an anatomist, and it may be thought by
many that he was speaking about something he did not know and was merely making a
guess, like evolutionists do. It is very possible that Paul \textit{was} speaking something that he \textit{did}
know and it was not merely a guess, or by accident that he made this positive statement. If it
is shown that these words are true, then it gives a final answer to organic evolution from the
standpoint of anatomical construction. We will here present the findings of one of the most
capable scientists in this field of research whose testimony carries great weight.

Dr. Albert Fleischmann, professor of zoology and comparative anatomy in the University of
Erlangen, Germany, was trained in his school days as a believer in organic evolution on the
Darwin-Wallace-Heackel plan. When he got older and began to make his own research and
study of anatomy he became convinced that these evolutionary ideas were absurd
speculations based on the findings of a few scattered bones, and could not be true. Therefore
he forsook the evolutionary theory of the origin of man from the amoeba or slime of the
ocean through countless stages, because he saw too many fallacies and surmises without any
base in truth as a foundation for them. In a paper delivered before the "771st Ordinary
General Meeting," of \textit{The Victoria Institute or, Philosophical Society of Great Britain}, May
22, 1933, he delivered the following address which is here published in full by permission
and courtesy of the \textit{Victoria Institute} of Great Britain.

771ST ORDINARY GENERAL MEETING

HELD IN COMMITTEE ROOM B, THE CENTRAL HALL, WESTMINSTER, S. W. 1,
ON MONDAY, MAY 22ND, 1933, AT 4:30 P. M.

DOUGLAS DEWAR, ESQ., B.A., F.Z.S., IN THE CHAIR.

THE DOCTRINE OF ORGANIC EVOLUTION IN THE LIGHT OF MODERN
RESEARCH

BY DR. ALBERT FLEISCHMANN, GR.

Professor of Zoology and Comparative Anatomy in the University of Erlangen.

THE OBSOLETE ROOTS OF DARWINISM

The earth, with its living creatures, is an indescribably great wonder. The more it is
investigated in search of its secrets, the less comprehensible does it become. Yet our
contemporaries, especially of the younger generation, have been taught to regard the riddle as
solved. They believe that the animal kingdom has, by the natural selection of fortuitous littleimprovements during millions of years, reached ever greater and greater perfection.
Following Charles Darwin, they regard all animal groups as branches of one gigantic tree.
Few of them realize that this idea of Evolution belongs to the days of our grandfathers and
great-grandfathers, while its roots pertain to the middle of the eighteenth century and stretch
back to G. Leibniz. It is precisely for this reason, however, that the theory suffers from grave
defects, which are becoming more and more apparent as time advances. It can no longer
square with practical scientific knowledge, nor does it suffice for our theoretical grasp of the
facts.

The manner in which the doctrine of organic evolution has fallen behind during the progress
of events may be seen if we briefly review the growth of zoological knowledge. About two
hundred years ago, K. Linne gave zoology its fundamental principles. A hundred years later
(1831) Charles Darwin concluded a three years' tour round the world, returning to England
with a rich store of new observations, and the rudiments of his theory, which, some thirty
years later (1859), roused a delirium of enthusiasm in scientific circles, and finally afforded
to the wider circles of both educated and uneducated society the illusion of a revelation of
natural science.

LINNAEAN CLASSIFICATION

Linne's principles of research are so simple and clear that they have unquestionably served to
guide the work of all subsequent generations up to the present time. He insisted, in the first
place, that statements should be limited to matters of actual fact, all play of the imagination
being avoided. His second principle is implied by the title of his work (1735), named
\textit{Systema animalium}; for he held that the study of animals is facilitated by their proper
arrangement—that is, by their synthesis (or grouping together) into genera, families, orders
and classes, and their antithesis (or separation apart) into unlike animal groups. These two
principles have served zoology throughout its great development during the last two hundred
years. They have enabled the pupils of the great master to classify systematically not only the
species known in his day, but also the vast numbers which have since been discovered; so
that the arrangement of animals according to his system remains to this day the standard
method of registering all special knowledge which we have acquired in regard to them.
Anyone, who would pass judgment on the correctness or otherwise of the doctrine of
Evolution, must first master the details of this arrangement. For most of the laity such a task
is impossible to undertake, owing to the colossal dimensions to which this classification has
now attained. The first edition of his work, compiled by the youthful Linne, dealt with 560
animal species. After a century (1830), some 30,000 were included; and now, after another
century (1933), about a million species. This fundamental work underwent a sudden
expansion at the close of the first hundred years, owing to the recognition of fossils which
had long been known, but disregarded as \textit{Lusus naturae}—as the remains of once living types.
They then had to be inserted in their proper places, among still living types, in the Linnaean
system; and this gave new work to naturalists, and led to manifold observations being made
on the characters of many strange animals which once lived on this earth in countless
numbers.

DARWIN'S DREAM

Charles Darwin's youth was passed during the early years of this great expansion, and he
received from it a strong impression which mastered his whole thought. He expected to find
in fossil types, much information regarding the origin of living things. He regarded fossil
species as the ancestors of living ones, and dreamed of a \textit{genealogical tree} embracing all
species of animals, both past and present.

This fascinating dream has not, however, been confirmed by later discoveries for the fossil
fragments of extinct types are limited to their harder parts (bones, teeth, shells, etc.), while
the softer parts have almost always been entirely lost. Hence the increasing mass of
palaeontological discoveries has only served to multiply our problems and emphasize our
ignorance during the second hundred years, at the same time that increasing knowledge of
the soft parts of living species, and of their minute structure, attained unexpected dimensions,
and swept away the ground from beneath the feet of the evolutionists. Charles Darwin lived
in a day when few people realized the value of detailed anatomical research in regard to
Linnaean groupings of creatures; he consequently acquired comparatively little knowledge of
anatomy, and never heard of modern anatomical methods.

THE PROGRESS OF ANATOMICAL RESEARCH

Indeed, during the first hundred years of zoological work, anatomy had only played a
subordinate part. Linne and his contemporaries had studied the outer appearance of the
animals of their own and foreign countries, and arranged them according to similarities in
such matters. Hence the early classifications were often based upon striking pecularities of
form, and single superficial features; study of the inner structure of the animals concerned
being left severely alone. One might almost say that there was a general aversion to
anatomical research at that time, although the great anatomist G. Cuvier (1769-1832) had
insisted, soon after the death of Linne, that classification would be based upon internal details
as well as on external ones. His chief supporters were found among students of human
anatomy.

A revolution in methods during the second hundred years has succeded in raising anatomical
knowledge to the high status which it holds today. This is realized by experts, although the
general public knows little about it. Hence few adherents of the doctrine of evolution realize
how incompatible their shibboleths are with the leading modern concepts of animal anatomy.

A hundred and fifty years ago, detailed anatomical work was restricted to the study of the
human body, and not extended to zoology in general. Instruction given to doctors of
medicine was mainly in accordance with the syllabus drawn up by A. Vesalius (1514-1565)
in 1543, which spoke of such organs as Bones, Ligaments, Muscles, Blood-vessels, Nerves,
etc. Such a classification, based upon the structure of the human body, could not be utilized
by zoologists in general, who had to deal with very different types of animals (Insecta,
Crustacea, Echinodermata, Vermes, etc.). Cuvier had emphasized this fact in 1804, when he
distinguished four main types or \textit{phyla} of animals (Vertebrata, Articulata, Mollusca and
Radiata). Only the first phylum (Vertebrata) contains creatures whose structure is comparable
with that of man; the other three phyla differ from it fundamentally. In spite of this, for many
decades, the results of research in animal anatomy were still tabulated according to Vesalius'
arrangement of organs. Ultimately, the latter was abandoned; but not until a great increase in
knowledge had led to seeming correspondences being better understood, and anatomical
divisions being more scientifically defined—and before this could happen, the whole
technique of anatomical research had to be fundamentally altered and refined.

THE NEW METHODS AND CONCEPTS

If one desires to study the inner constitution of animals, one can only do so by dissecting, or
progressively dividing up their bodies, which resemble intricate shrines, until one resolves
them into many separate parts, and finds that they appear to be composed of separate organs.
This dissection of bodies is so essential to their study that the whole process of research work
on them is briefly termed a "cutting up" (Anatomy). In place, however, of methods of
dissection which have been followed from very ancient times, new processes and instruments
were introduced during the second hundred years (1830-1930). At first there came the
dissection of frozen bodies by means of a saw into what were still comparatively thick
longitudinal and transverse sections; then followed an increased refinement whereby, with
the help of a razor, very thin sections (0.5 to 0.002 mm.) of parts of bodies, and of small
animals, hardened and embedded in paraffin, were obtained by the microtome invented in
1876. By this latter means the investigation of body structure was revolutionized. Instead of
dealing with bodies divided crudely into thick masses, we can now examine long ribbons of
sections, as thin as may be required, which expose the inner structure without materially
disturbing its arrangement. This new method of cutting sections facilitated an excellent new
method of dealing with anatomical material which, under the name of topographic anatomy,
was first practiced by doctors in England and France. The structure of the body was no
longer regarded from the standpoint of isolated organs, but from that of body regions—head,
trunk, limbs, etc. By this more enlightened practice, a method of dealing with bodily
dispositions was adopted which had long been known to those who had to solve architectural,
geometrical and mathematical problems. Thanks to the microscopically enlarged \textit{sections}, the
eye of the research worker was also enabled to penetrate deeply into the minute structure of
the body and discovered the fact, which had previously been unknown, that all animal
structures are developed from special layers which recall the annual rings of trees.

The growth of knowledge of the \textit{body layers}, affords, in fact, the most remarkable feature in
the progress of zoology during the second century of that science's existence. It provided rich
material for new connections of ideas, to which Darwin and his contemporaries had been
strangers. Likewise, the microscope disclosed the fact that all the body layers are made up of
cells—tiny little building stones from 0.07 to 0.1 mm. in length. Owing to the thorough work
of talented investigators, our knowledge of histology has increased to such an extent that
anatomical relationships are regarded in a very different light today from that in which they
were viewed during the first half of the nineteenth century.

THE IMPORTANCE OF ONTOGENY

At the same time these facts were being revealed, other pioneers of research, headed by K. E.
von Baer (1792-1876), were showing that anatomical work should not be restricted to the
fully-grown body, but that it was necessary to study sections of the body during \textit{all} the
phases of its existence (adult, youth, child and egg). When this is done, an extraordinarily
manifold transformation-scene is witnessed, which runs throughout the whole life of every
individual, and brings about great changes in both its inner and its outer form, often
accompanied by changes in its geometrical proportions. Something of this nature had been
noticed, during the seventeenth and eighteenth centuries, in regard to the easily seen
changing life stages (egg, caterpillar, pupa, imago) of the Lepidoptera and other insects; and
most surprising changes, from simple larvae into highly complex adults, were now
discovered among marine organisms.

Every year assiduous research work revealed more plainly that the course of every animal's
life is, from egg to adolescence and even to death, one continual transformation, be it rapid or
slow. Earlier and later life stages often seemed quite irreconcilable (e.g. tadpole—frog, etc.)
so long as only a few growth stages were known, separated by considerable intervals of time.
But the greater the number of stages of the building up of the body that were placed in
correct series, the greater became the knowledge of their regular logical sequence. A splendid
revelation was thus obtained of the progressive building up of the body, governed by laws of
space and time; and the sequence of life phenomena emerged from their former obscurity like
a continuous cinematograph film, the individual pictures in which follow each other in
necessary order.

Many great transformations are seen to take place; a tiny double cell, the fertilized egg, from
0.5 to 0.2 mm, in diameter, grows into a great adult creature weighing many hundred
kilogrammes. The investigation of this marvel is far more profitable than making
unverifiable guesses regarding the genealogical changes of long-extinct animal species of
former ages, which are only known to us from bits of their skeletons.

EFFECT ON THE CONCEPT OF SPECIES

The concept of the \textit{species} also received, during the course of the second hundred years, a
new far-reaching significance, much beyond Linne's conception. It no longer signifies, to us,
the constant form of a pair of adult individuals, but it rather represents the ceaseless flow of a
determinate change in organization which, beginning with the simple spherical form of the
fertilized egg-cell, is so strictly regulated for each species that one can actually wait, watch in
hand, for the appearances of the destined form conditions. At first, simple structures begin to
appear within the enclosed space of the egg. Soon they emerge from this, especially after
food begins to be absorbed, and the tiny mass unfolds itself like a graduated series of
concentric spheres into the form of a living animal. Exhibiting, the first, only a simple lace
pattern, the fertilized egg-cell becomes, by progressive segmentation, or doubling, split up
into an increasing number of cells (2, 4, 8, 16 . . . 128, 256, 512, 1,024). Then the cells
arrange themselves into three \textit{basic layers}, called "germinal layers," which unfold each other.
In all the animal groups (except the Protozoa) a cylinder-shaped structure then arises, which
consists of an outer single-layered wall (or tube) formed by a stratum of connected cells
known as the \textit{ectoderm}, beneath (or inside) which lies a mass of densely crowded cells called
the \textit{mesoderm}, and lastly comes an innermost single layer (or tube) of cells—the endoderm.
Since these three germinal layers remain distinct throughout life, we are able to trace the
subsequent development, from each layer, of the structures to which it respectively gives rise.

FUNDAMENTAL DISTINCTIONS OF THE PHYLA

The new view-points stimulated, on all sides, assiduous research in the wide field of animal
anatomy. The resulting well-grounded knowledge soon led to a complete change in ideas,
which swept aside the old widespread notion of Darwin's day that the human body supplied
the pattern for all animals, or, as it used to be said, that the organs of all members of the
animal kingdom correspond to those of a dissected man (L. Oken); a preconceived notion
which, by encouraging talk of "the ascending scale" of animal species, has led to great
confusion. In place of this notion, the clear conviction arose that the Invertebrate phyla are,
throughout their history, fundamentally different from the Vertebrata (including man), just as
Cuvier had, with admirable insight, pointed out between the years 1795 and 1832. Now, inthe year 1933, we actually recognize more than a dozen such groups of fundamentally
different types of body structure, namely: Vertebrata, Arthropoda, Crustacea, Annelides,
Rotatoria, Mollusca, Brachiopoda, Echinodermata, Tunicata, Platodes, Bryozoa,
Coelenterata, Protozoa.

Had Darwin lived to witness this advance, he would have abandoned his illusion of a single
great genealogical tree for all species of animals. The layman, however, could not formerly,
and still cannot today, understand why the genealogical tree and the phyla conceptions are so
irreconcilably opposed to each other, because he lacks the comprehensive knowledge of the
development phases of all the phyla, which would make this opposition clear to him.

THE REFERENCE PLANES OF ANATOMICAL MEASUREMENTS

When once the recognition of positions in the germinal layers was realized to be the most
important business of anatomical research, it became obvious that measurements of
sterometric bodies had to be made with reference to the three chief planes (XX, YY, ZZ), in
order to make proper comparisons of those bodies. Since the animal body has an outer and
inner aspect, and a curved instead of a straight boundary surface, the outer boundary is not
taken into consideration because of its extremely manifold modelling. All references are
therefore made to the three chief inner planes. These are allotted definite positions in the
body, in order to determine the relative distances of all points in the germinal layers, and in
the numerous outgrowths from those layers. Most animals clearly bear, in their outer form,
indications of the middle plane (ZZ) of the body, which is witnessed to by the mirror-like
duplication of their right and left sides, so similar in shape, but developed in opposite
directions. Owing to the discovery of the three germinal layers the work of measurement has
been greatly lightened, because the body-complex is no longer regarded as a mass of organs,
but as a co-ordinated combination of the three chief layers. One clearly sees how these
germinal layer masses have developed similarly varying thicknesses on each side of the
middle plane. Each layer shows a certain freedom in disposing of its mass; it may remove
itself further from the three planes, or sink closer to them. In consequence of this, the layers
are at times bent outwards to a greater or less extent; at other times they are bent inwards to
form cavities, pouches, funnels, sometimes alternating with protuberances. There are,
however, always fixed limits to their expansion in height, length and breadth.

The importance of the three chief layers has been incontrovertibly proved, particularly in
cases where anatomical investigation has followed the whole course of life (egg to death),
during which decisive changes of state follow one another in rapid succession. Reference to
the three layers has the great advantage that the animal body is regarded as a whole, all
regions and parts of it being equally observed, while the three chief planes only are taken into
consideration.

MEASUREMENT FIXATIONS OF GROWTH PHENOMENA

Just as the geologist reckons the strata of the earth by stages, so does the anatomist look for
layer differences which characterize successive life phases. Traces of future structures first
appear as exceedingly faint indications in the three-layered complex, and gradually develop
into their final forms. All this results from the multiplication, often to an incredible degree, of
minute cells which—except in rare instances—never become large enough to be seen by the
naked eye. Indeed, this intricate cell structure of the body is one of the chief discoveries ofthe second hundred years. The more carefully we follow the developments of the three
layers, with reference to the three main planes, the more clearly do we appreciate the strict
order of bodily growth down even to its minor details; while, at the same time, we also begin
to realize even more clearly the wonderful regularity of body structures, which had
previously only been recognized in regard to the segments and appendages of Insects,
Arachnoids and Crustaceans. All this has contributed to emphasize the value of the new
methods of treating animal anatomy by counting, reckoning, and (above all) by measuring.

It is due to the study of the three germinal layers that the structure of nearly a million species
has now been fairly well elucidated, in contrast with the darkness which covered the subject a
hundred years ago. We accept those three layers today as our means for accurately estimating
likenesses and differences in the animal world. The new system insists that names, often
incorrectly used in a universal sense (for example, eyes, teeth, stomach, lungs) would be
restricted to the particular phylum; and it endows them with their proper meaning within the
same. The head of an insect, for instance, has a very different derivation from that of a
vertebrate!

The limits of the phyla, in comparing body structures, are now determined by the law of
situation. He who measures the distances of important surfaces and regions from the main
planes, obtains a true \textit{group-picture} of the arrangements in species of all features which
either grow out of each germinal layer like peninsulas, or else are detached as independent
islands and become embedded in the middle layer. The idea of local relationships has
prevailed over the conception of organs, which was universal in Darwin's time. The
textbooks of animal anatomy have likewise acquired a wider outlook, because the large body
areas are now regarded as entities, and comprehensible pictures of the most important
features of the phylum are thereby presented.

RESULTING WHOLE-LIFE VIEW OF SPECIES

As compared with the obsolete methods of procedure of 60 to 100 years ago, the modern one
has the advantage that it takes into consideration not only the fully developed body, but also
all the stages of its growth, from egg to adult. This comprehensive review shows us that the
foundations of the ultimate structure are laid down in the earliest stages of existence, and
development proceeds, as if of logical necessity, to the pre-ordained magnitude and final
conditions. The same identical sequence of earlier and later life stages repeats itself, in the
case of each member of the species, just as if the process of bodily development clung to a
rigid track, along which the germinal layer complex was compelled to travel during life,
through a definite number of fixed intermediate stages to the appointed end. The course of
life of every individual within the phylum traverses a special, native and unchangeable
sequence of phases, which finally produces the fully developed body with all its parts. The
wonderful regularity shown by the course of this development forbids the idea that the mode
of growth within the phylum ever left one track in order to follow another. It is clear that, in
supposing that existing species had sprung from other species, Darwin was only taking adult
structures into consideration. In any case, Darwin's followers must now suppose that the
developments of the germinal layers of earlier species underwent very frequent changes! But
modern knowledge of the constancy of development shown by species lends no countenance
to this.

There is no ambiguity about the general results reached by the clear-cut methods of modern
anatomical research. One certainly sees, the universal appearance of the three germinal layers
and their regular placing with reference to the three chief planes, a general likeness in the
structure of all species of animals; but we nevertheless find that those germinal layers
perform different tasks in each phylum, according to the size and weight of the body and its
inner and outer details. Thus the supporting structures required by the living body are
formed, among insects, arachnids, and crustaceans, from the outer layer, which produces a
calcareous shell; among the vertebrates, on the other hand, the outer layer is unfruitful in this
respect, all the masses of cartilage and bone of their skeletons being derived from the middle
layer. It is certainly true that the calcareous plates and spines found in the phylum
Echinodermata are also derived from the middle layer, but they are derived in quite a
different manner. Hundreds of examples are known of the incredible differences to be found
among the products of the germinal layers, according to the groups concerned.

THE ADDED CERTAINTY IN CLASSIFICATION

As the result of these investigations into the details of structure and development processes of
animal bodies, many new characteristics have been added to the distinctions recognized by
earlier workers and have endowed the conceptions of zoological classification with an
unexpected new element of certainty. Thus the hopes of Cuiver have been fulfilled during the
second century of anatomical work, and Linne's efforts after classification have finally
resulted in a system well grounded on anatomical facts.

Sound work on the structure and connections of the layers must begin by dealing with groups
of the most closely related species. This reveals the regularity and wonderful individuality of
the development of each species, and habituates the mind to think more and more in terms of
anatomical group measurements. Broad facts which Cuvier outlined 130 years ago are now
practically illustrated by group-pictures of the growing layer connections and chief tissue
complexes during the whole life history of individual species; and such evidence affords a
firm foundation on which to base our arrangements of species, each according to the
wonderful shading of its common group features, into well-selected higher groups of like
forms (genera to classes). The phyla thus constituted usually agree, in general, with improved
groupings under the older system of classification. Every recent handbook of zoology places
the classes within the phyla so delineated (for example, the Coleoptera, Diptera, Hemiptera,
Hymenoptera, Lepidoptera, Orthoptera, etc., among the Insecta), and the lesser groups within
the classes, down to the individual species group. If an arrangement originally based upon
external adult features agrees so well (in a general way) with our later classification based on
the whole developmental history of structures, inner as well as outer, it would seem to imply
that those thinkers are right who regard the animal body as a \textit{wonderful self-contained work
of art.}

PHENOMENA OF LAYER-COMBINATION

Modern anatomy clearly emphasizes the indivisability of the parts of the body at all times,
past and present. Cuvier designated this the "Correlation" of the parts; E. Geoffroy St. Hilaire
styled it their "Connection"; I myself have hitherto called it the "Layer-Combination"
("unlosbaren Lageverband"). This expression indicates the fact that anatomical structures
cannot be regarded as results arrived at by accumulations of little accidents, but that each is asuperhuman work of art, living, regulated enigmatically by strict laws, and itself conserving
and producing new life forms.

SPECIFIC CONSTANCY UNAFFECTED BY VARIATION

Study of the higher groups reveals a striking regularity, which was unknown 100 years ago,
and which, in view of the rules of position and form which are obeyed down to the smallest
details, lends no support to the idea that the strict laws of one species could be changed, by
means of minute fortuitous variations, into the structural laws of another species. Seventy
years ago, Darwin could talk as if varietal differences tended to "change the species," and
such talk met with approval; but since the strict orderliness of development has been
discovered the assumption of an evolution of species has encountered insuperable
difficulties. No one can demonstrate that the limits of a species have ever been passed. These
are the Rubicons which evolutionists cannot cross. The fact of variability, on which Darwin
based his ideas of fortuitous differences linking allied species, is countered by the sobering
fact of the law of variation, which expresses the fundamental agreement of measured
characters among the members of a species, as known from the the Statistics of Variations
during the last decade. This shows that the variations are centered round a mean value in the
form of the binomial curve which represents the law of averages, and is constant and true for
one species, but not for related species. The question, therefore, is not whether the species is
variable or invariable. The essential point is that the concept of the species is based upon the
\textit{regular} destiny which is inscribed on the three germinal layers, and the place-form
pecularities of their complexes in the course of life of the individual. Thus accident, caprice
and arbitrariness are eliminated from zoological discussion.

INCONGRUITY OF THE "GENEALOGICAL TREE" CONCEPT

In the same way, the altogether useless concept of the animal \textit{genealogical tree} is found to
disappear. It affords no satisfactory picture of the relationships between the million living
species of animals and the 120,000 known extinct species. For the last 70 years evolutionists
have discussed hundreds of supposed ancestral derivations, without having agreed about a
single one. Attempts to blend together the characters of the fourteen different phyla into one
hypothetical common stock only result in producing an opalescent pattern of body structures,
which proves nothing for the common origin of those phyla.

The so-called pedigree of the animal kingdom is utterly unlike the genealogical trees of
human families, because the latter deal only with members of one species, whereas the
former include multitudes of different species and postulate countless purely hypothetical
links between them. Even the shortened genealogical trees found in popular writings are apt
to dogmatize about the derivations of whole phyla—that is, of anything from 2,000 to
100,000 species at a time.

The family genealogical tree shows a limited number of names, arranged in the semblance of
a tree, of people actually known to have been related by descent. It is a compilation of facts,
like a dictionary. Nothing resembling it is known regarding species connections. When we
come to discuss the latter, we are no longer dealing with first-hand evidence (i.e. with verbal
or written traditions) as to the connections concerned. All is hypothesis. We postulate long
ancestries simply because we do not know the real ones, and because creatures have to be
accounted for somehow. We note the incontrovertible fact that new creatures, born everyyear, experience the same time—and form—regulated fate as their parents; hence the
sequences we see are obviously links in chains or organisms of which neither the beginnings
nor the ends are visible to us. But that does not justify us in supposing that, just because each
individual changes in form while developing from childhood to adolescence, therefore its
remote ancestors must have changed from one species into another. Again, even when we
deal with the members of a single existing species, we find it impossible, on purely
anatomical grounds apart from historic testimony, to demonstrate the connection between
individual parents and their offspring. Among animals, the father is apt to disappear nameless
among the multitude of his species, after taking his brief part in procreation, and science is
powerless to re-identify him. Despite these facts, evolutionists search for "ancestors" in the
graveyards of the past, and arrange fossil fragments (e.g. leg bones, teeth, or skulls) of
various extinct species of horse into hypothetical series, and—in complete disregard of the
rules of group-position and form-believe that these represent real ancestries. Yet the facts
which they quote go no further than, for example, the science of malacology went 200 years
ago, when only empty shells were examined. Malacology has long grown out of that stage,
owing to our increased knowledge of the soft parts of shelled animals; but palaeontologists,
whose researches are of necessity confined to the hard parts of extinct species, still know
nothing about the minute cell-structure of those species.

Nothing is gained by glib talk about "ancestors," "stem-parents," "ancient progenitors," etc.,
as classificatory concepts of extinct species, on the supposition that evidence to prove the
truth of those concepts will be found later on. Our hopes in this respect are very remote,
especially in the case of the thousands of species of minute creatures whose tiny bodies
rapidly decompose after death and leave no enduring hard parts.

CONCLUSION

A survey of the history of zoology thus reveals an actual situation very different from that
generally claimed by the advocates of evolution. The business of classifying animal species
began, in 1735, with very little knowledge. During the course of the second century since that
date, however, about a million species have been mastered by means of a detailed study of
their major and minor body structures throughout their development from the egg, at the
same time that incontrovertible methods of measuring the degrees of likeness have been
invented, and the unvarying form and time stages of the life of animals have been discovered.
On the other hand, the study of palaeontology has not fulfilled the hopes that Darwins and his
contemporaries placed in it. As it happened, they found themselves in much the same
condition in regard to palaeontology, 100 years ago, as Linne had found himself, in regard to
zoology, a century earlier. He had little knowledge to begin with, although zoological science
has since so greatly expanded. But palaeontologists are still confronted by the fatal difficulty
that their field of research lies in the graveyards of the buried past, instead of in the living
world which continually renews its youth. While attempting to deal with similar problems,
the palaeontologist has only a skeleton to work upon, while the zoologist can study the entire
animal in the full vigour of its existence.

The limitation of the palaeontological field of research can obviously never be removed, and
the very antiquity of the fossiliferous strata precludes our attaining certain knowledge
regarding the animals which lived while they were being laid down. All that we can do is to
group the fragmentary remains of these animals as best we may, after careful examination of
all the available evidence, together with existing species. It is obvious that we can nevercompare their minute structure with that of living things, or with that of other fossil types. In
other words, we can never hope to attain adequate knowledge of the fossil world, much less
can we prove its evolution.

Seventy years ago, Darwin ransacked other spheres of practical research work for idea. In
particular, he borrowed his views on selection from T. R. Malthus' ideas regarding the
dangers of over-population, to which he added the facts recorded by breeders regarding the
variability of domestic animals, the results of artificial selection of the best pairs in herds, the
pedigrees of domestic animals, and the improvements of existing races and the development
of new ones, etc. In order to adapt these things to a theory of wild life, he then added the very
reasonable concepts (in J. Kant's opinion) of the struggle for existence and natural selection.
But his whole resulting scheme remains, to this day, foreign to scientifically established
zoology, since actual changes of species by such means are still unknown. On the other hand,
our greatly increased knowledge of specific anatomy throughout life, as well as the new
variation statistics and our increased knowledge of Mendelian laws, have all tended—
especially within the last 30 years—to accumulate evidence against Darwin's theory.

In my opinion, the most serious defect in the Darwinian school of thought is that it is not
based on the knowledge of rigid law. No matter how much eloquence the advocates of
evolution may pour forth, they will not cancel the facts briefly outlined above!
(Note—it is unfortunate that the word "phylum" should imply that very concept of a
genealogical tree to which this paper takes exception. To substitute another another and less
familiar term might, however, lead to misunderstanding, since "phylum" has now acquired
such definite significance, in classification, as referring to one of those great sections of the
animal kingdom whose fundamental structural designs are so distinct from each other. The
term "phylum" is therefore retained in this paper; but it should be clearly understood that it is
here used in the sense only of a great division of organized beings, and not as implying any
doctrine of a common genetic origin. All modern research emphasizes the distinctions not
only between the great divisions themselves, but also between the subdivisions of which each
is composed; and it shows the absence of all factual grounds for postulating genetic
connections between them.)

DISCUSSION

The CHAIRMAN (Mr. Douglas Dewar) moved that the thanks of the Institute be given to the
learned author of the paper, and the same was accorded with acclamation.

Rev. Dr. H. C. MORTON said: "We have listened to a really notable paper by one of the
world's great zoologists, who, especially in the light of anatomical research, finds only one
course open, viz., the emphatic and unfaltering denial of the "illusion" of Darwinian
Evolution, and of "the fascinating dream" of the genealogical tree of the Doctrine of Descent.

I am not an anatomist, and even if I were, this occasion lends itself but little to technical
discussion. But there are two things I want to say: The first is that it is worthy of note that
Professor Fleischmann does not trouble to distinguish between Darwinism and Evolution in
general, but evidently treats Darwinism as the one attractive and widely influential form of
the evolutionary hypothesis. What applies to Darwinism applies also to any other form in
which the same concept, of progress from the lower to the higher forms by long successionof changes, may be embodied. It is this whole concept "which no longer squares with
practical scientific knowledge." Just as Bateson said, in 1921, that forty years ago (that would
be 1881) real scientists had ceased even to talk about evolution, so Fleischmann says that this
concept belongs "to the days of our grandfathers and great-grandfathers." Not merely
Darwinism but "the altogether useless concept of the animal genealogical tree" is found to
disappear.

The second thing I want to say is this: that those who desire to preserve faith in the Bible
have got to deal with Evolution. It is not possible for a logical mind to hold both Bible
teaching and evolutionary teaching at the same time. The main cause of that failing faith
which is bringing down all the levels of our life, and with them the whole structure of British
power, is found in Evolution. The common practice of cramming evolutionary ideas down
the throat of the youth of our schools and colleges and universities, is not only an outrage
upon fairness and justice, but it is hastening that collapse which is so evidently sweeping up
upon us. I believe God is giving us our call and our chance. We have got to make our choice,
and a deliverance like Fleischmann's today should help us to make it.

Sir Arthur Keith has twice publicly given to the British Nation his religious experience. He
began as an Evangelical Christian, then became an evolutionist, and found every belief of the
Christian Faith, slowly perhaps but surely, destroyed within his mind; and he has declared
that the Christian Church has no half-way house, she must accept everything or else reject
everything. The Bible and Evolution represent two absolutely diverse, alien, and hostile
realms of thought. No logical mind will even try to dwell in both at the same time. Some of
us are not logical, but in the long run logic has a wonderful way of asserting itself. If the
Bible does not kill Evolution, Evolution will kill the Bible; and the choice between the two is
big with doom.

Mr. GEORGE BREWER said: Our thanks are due to Dr. Fleischmann for his clear statement
of the result of modern discoveries in confirming the unscientific basis of the theory of
Organic Evolution. He assures us that "modern anatomy clearly emphasizes the indivisibility
of the parts of the body at all times, past and present," and that this "layer combination
indicates the fact that anatomical structures cannot be regarded as results arrived at by
accumulation of little accidents, but that each is a super-human work of art, regulated by
strict laws, and itself conserving and producing new life forms."

Galen, a celebrated physician, who practiced in Pergamos and Rome in the second century,
and the author of a large number of medical works, which formed the chief textbooks of the
medical profession for several centuries, was converted as the result of his dissections, and
compelled to own to a Supreme Being, as the Author of nature's wonderful handiwork. The
Psalmist records a similar conviction that he is "fearfully and wonderfully made," when he
declares: "Thine eyes did see my substance, yet being unperfect; and in Thy book all my
members were written, which in continuance were fashioned, when as yet there was none of
them." (Psalms 139.)

There has been an utter failure on the part of evolutionists to prove their theory. The
arguments from natural selection, embryology and palaeontology have completely broken
down; and the feverish anxiety to find the supposed "missing link" failed, even though not
one, but thousands of links would be in evidence if the theory were true, such zeal shows the
natural desire of man to account for the wonders of creation, apart from the Creator.

It is refreshing to turn from evolutionary fables, based on assumption and speculation and
falsely-called Science, to the inspired record in the Book of Genesis, and the statement of the
Apostle Paul in 1 Cor. 15, "That all flesh is not the same flesh: but there is one kind of flesh
of men, and another of beats, and another of fishes, and and other of birds." And while all
flesh is as grass which withereth away, "the word of the Lord endureth for ever." (1 Peter. 1,
23.)

WRITTEN COMMUNICATIONS

THE PRESIDENT (Sir Ambrose Fleming, F.R.S.) wrote: The Members of the Victoria
Institute will all, no doubt, agree with the opinion that we are fortunate in having secured
from such an eminent naturalist as Dr. Albert Fleischmann, the Professor of Zoology in the
University of Erlangen, a valuable criticism of the theory of Organic Evolution. We have had
many papers read to us in recent years bringing to bear critical insight on the defects and
tendencies of the above-mentioned hypothesis. The Darwinian theory of natural selection for
the production of animal species, and its logical outcome, in the origin of the human species,
is still strongly advocated by writers and speakers who can command public attention. The
serious objections to that theory do not easily obtain a hearing, and hence the general public
are led to believe that no forcible objections or anything but prejudice can be urged against it.

In the press, on the platform, and even in the pulpit, it is taken for granted that the human
race began millions of years ago, as the product of Darwinian Natural Selection operating in
animal ancestors. The grave objections to this hypothesis and its absolute failure to explain
the origin of the ethical, altruistic, religious, and spiritual qualities of mankind, are not given
the weight they demand, whilst its logical consequences are disastrous, in their influence on
human aims and thought. But it is clear that the theory as regards the human species stands or
falls by its correctness as regards animal species, and hence any scientific, learned, and valid
criticism of Darwin's theory is of great importance. Even though we ourselves may not have
sufficient technical knowledge to search out the valid arguments against this popular theory
of Organic Evolution, we can all appreciate the very masterly survey of them which Dr.
Fleischmann has given us in this paper. He has dealt fresh and powerful blows at the theory,
and shown us that, at the bottom, it is in truth destitute of a solid scientific basis. In short, it is
not a scientific theory or explanation in any true sense of the word, but an unverified
hypothesis which has apparent strength but falls to pieces under any really searching
examination. I desire, therefore, to associate myself very strongly with the thanks which will
be offered to Professor Fleischmann for his powerful and useful contribution to our
Proceedings.

Lt.-Col. L. M. DAVIES, F.G.S., wrote: More than 30 years have passed since Professor
Fleischmann roused a storm in biological circles by throwing over his own long-standing
belief in Darwinism and publishing a book, "Die Descendenztheorie" (Leipzig, 1901), in
which he poured scorn upon the whole case for Evolution. What particularly stung his
opponents was the fact that Fleischmann could not be dismissed as an incompetent judge;
even Kellogg admitted him to be a "reputable zoologist," and a "biologist of recognized
position." (\textit{Darwinism Today}, p. 8.)

So the matter was hushed up. When, therefore, people like Bishop Barnes—who, by the way,
is a mathematician, and not a biologist—declare that no competent biologist today questions
the doctrine of Organic Evolution, it should be realized that they are coolly ignoring anexpert—one like Fleischmann—who has held the chair in Zoology and Comparative
Anatomy at a great German University, since days before Barnes was heard of.

I am unable to judge of some of the facts which the Professor stresses in this paper; but it is
useful to have the impression of so good an authority regarding the consistence (as evidenced
in their development, etc.) of specific types, which the evolutionist must assume to be so
mobile. Where he deals with some other points I am better able to confirm the Professor's
remarks. Thus, when he stresses the importance of the regional study of structures, I recall
the nonsense which people, who ignore this principle, have written about the supposed
"human tail." It will be remembered that Darwin, following the old "organ" view of anatomy,
tried to treat all vertebrae below the pelvic girdle as a "tail"—even though they might have
no external existence or functions \textit{as a tail}. The folly of this is seen when we examine the
great apes, which are supposed to link man to the tailed monkey; for those creatures have
been "less tail" (as Sir Arthur Keith admits) than man. Their coccygeal vertebrae are less
developed than our own! To anyone trained to regard structures as a whole, the reason is
obvious: semi-erect creatures, like the apes, require even less of a coccyx than do fully erect
creatures. In other words, our supposed "hidden tail" is not a tail at all; it has functions to
perform (relatively small, since the coccyx itself is small) which are purely internal, and
exactly suited to the needs of an erect structure like man's.

When Professor Fleischmann turns to the subject of Geology, I am glad to see that he stresses
several of the chief points which I tried to emphasize in a paper read before this Institute
seven years ago. Evolution, is essentially, a doctrine of \textit{unbroken genetic connections:} yet,
from the moment that historic testimony is lacking, not a single genetic connection can be
proved by any means known to science. When we deal with fossil forms, we are inevitably
afloat upon a sea of hypothesis. We can believe what we please; but we can actually prove
nothing for descent. Provided that a God exists who can literally create, we have no way of
showing that He has not created. The evolutionist will, of course, go his own way; but it is
good, occasionally, to receive such direct evidence as this paper of Professor Fleischmann's
affords that (despite all assertions to the contrary) first-rate biologists do exist who, knowing
all that their science can say upon this subject, still flatly disbelieve in Evolution.

