\chapter{FUNDAMENTAL DOCTRINES OF THE CHURCH}

THE following doctrines are fundamental to the Gospel of Jesus Christ. They were all given
by revelation. The revelation came by personal visitations from the Lord or from his angels
sent from his presence, or by his word to his servants the prophets. They are revealed truth.
They cannot be changed or modified.

1. God, our Eternal Father, is an immortal exalted Man, with a body of flesh and bones and
eternal spirit, inseparably connected that cannot be divided and cannot die. 1

2. The presiding authority in the Universe is, God the Father, his Son Jesus Christ and the
Holy Ghost. 2

3. This earth on which we dwell at the word of the Father, was created by the Son before he
obtained his tabernacle of flesh and bones. 3

4. Adam was the first man on the earth, and Eve, his wife, the first woman. They were
created in the image of God. 4

5. When Adam and Eve were placed in Eden they were not subject to the power of death and
could have lived, in the state of innocence in which they were, forever had they not violated
the law given them in the Garden. 5

6. The earth also was pronounced good, and would have remained in that same state forever
had it not been changed to meet Adam's fallen condition. 6

7. All things on the face of the earth also would have remained in that same condition, had
not Adam transgressed the law. 7

8. By partaking of the forbidden fruit, and thus violating the law under which he was placed,
his nature was changed, and he became subject to (1) spiritual death, which is banishment
from the presence of God; (2) temporal death, which is separation of spirit and body. This
death also came to Eve his wife. 8

9. Had Adam and Eve not transgressed the law given in Eden, they would have had no
children. 9

10. Because of this transgression bringing mortality, the children of Adam and Eve inherited
mortal bodies and became subject to the mortal death. 10

11. Because Adam transgressed the law, the Lord changed the earth to suit the mortal
condition and all things on the face of the earth became subject to mortality, as did the earth
also. 11

12. To defeat the power which death had gained it became necessary that an infinite
atonement be offered to pay the debt and thereby restore Adam and Eve and all of their
posterity, and all things, to immortal life through the resurrection. 12

13. To accomplish this, Jesus Christ, who created the earth, volunteered and was chosen to
come to earth as that infinite sacrifice. According to eternal law, this sacrifice had to be made
by a God who was not subject to death, yet had the power to die and take up his body again
by inherent right. Being the only begotten Son of God in the flesh, our Redeemer obtained
from his Father the mastery over death, and from his mother he inherited the power to die. 13

14. This atonement by the Son of God is of twofold nature: (1) It redeems all creatures who
are subject to mortality through the fall, and restores them to immortality, without any act
whatsoever on their part. Hence every soul born into this world shall receive the resurrection,
the body and spirit being united never again to be divided. 14 (2) It redeems every soul from
the penalties of his own transgressions, on the condition that he accepts the plan of salvation
and is obedient to all the laws and commandments of God and that he endures in faithfulness
to the end. None but the truly repentant are entitled to this forgiveness of sins. 15

15. There are definite laws which are known as The Gospel of Jesus Christ, that must be
obeyed in order to obtain this salvation.

16. The first, or foundation principles, which must be received and obeyed are these:

a. Faith in God and in the Atonement of Jesus Christ.

b. Repentance from all sin.

c. Baptism, by one having authority, in water by immersion for the remission of sin.

d. Laying on of hands by one holding authority, for the gift of the Holy Ghost.

e. A contrite spirit and a humble heart.

f. Obedience to every other ordinance and principle of the Gospel, appertaining to the
blessing of eternal life, and faithfulness to the end.

17. All men will be judged according to their individual works and none will be required to
pay the debt of another. 16

18. There are graded kingdoms, or degrees of glory, into which the children of men will go
after the resurrection, according to their faithfulness, or lack of faithfulness, in the keeping of
the commandments of God; in other words, according to their individual works while on
earth. Those who have been baptized and confirmed and remain true, having overcome the
world by "faith, and are sealed by the Holy Spirit of promise, which the Father sheds forth
upon all those who are just and true," shall inherit the celestial kingdom, where they will be
"priests and kings," and "sons of God." Honorable men of the earth, who have kept
themselves virtuous, who have been honest, just and fair in their dealings with others but
who would not accept the Gospel when it was offered them, shall become heirs of the
terrestrial kingdom. All those who love wickedness, "who are liars, and sorcerers, andadulterers, and whoremongers, and whosoever loves and makes a lie," shall be sent to the
telestial kingdom. There is still another group composed of those who have had a testimony
of divine truth, who have had the guidance of the Spirit of the Lord, or Holy Ghost, and
afterwards deny the truth and put Jesus Christ to open shame. These shall be cast out into
"outer darkness." They are called sons of perdition. 17 Moreover, there is salvation for all
those who have died without knowing the plan of salvation; who never knew of Jesus Christ
and had no opportunity to repent and receive the remission of sins while living in this mortal
world. By eternal provision, declared in the beginning, all of these have the opportunity to
hear the Gospel in the world of spirits, and all who are willing to accept it there, will be heirs
of salvation, the work essential for that salvation which belongs to this mortal existence, will
be done for them vicariously. 18 Thus our Eternal Father in his great mercy grants salvation
to all who are willing to receive it, both the living and the dead. This great work for the dead
is performed in the temples and shall continue to be performed through the coming
millennium until every soul who is worthy of salvation shall hear and understand the fulness
of the Gospel.

These doctrines we are called on to preach and sustain by the commandment of the Lord to
his Church. It is just as necessary to cry repentance today, as it was in the days of Paul, and
to preach "Christ and him crucified," which is just as much a stumblingblock to our
evolutionist friends today as it was to the Greeks in the days of Paul. But our friends say such
doctrines as I have here presented, are dogmatic. Dr. Millikan, for instance, has said:

As I see it, there are but two points of view to be taken with respect to this whole question of
religion. The one is the point of view of the dogmatist; the other the point of view of the
open-minded seeker after truth. Dogmatism means assertiveness without knowledge. The
attitude of the dogmatist is the attitude of the closed mind. There are two sorts of dogmatists
in the field of religion. One calls himself a fundamentalist; the other calls himself an atheist.
They seem to me to represent much the same type of thinking. Each asserts a definite
knowledge of the ultimate \textit{which he does not possess}. Each has closed his mind to any future
truth. Each has a religion that is fixed. 19

What could be more dogmatic than this expression? Is it true that dogmatism "means
assertiveness without knowledge?" How do you know that the assertiveness is without
knowledge? When the eleven disciples asserted that Christ appeared to them in the upper
room after his resurrection, and they thrust their hands in the wounds in his side and his
hands, was it assertion without knowledge? Their statement is dogmatic, and justly so. True
religion \textit{is} dogmatic. All truth is dogmatic. When Peter said to the Jews: "Be it known unto
you all, and to all the people of Israel, that by the name of Jesus Christ of Nazareth, whom ye
crucified, whom God raised from the dead, even by him doth this man stand here before you
whole," he was very dogmatic. Was this dogmatism without knowledge? Peter was an
eyewitness to the resurrection of the Lord. He not only saw him, but felt his wounds and
heard his voice. We could hardly think that Peter the disciple and witness, could say, "we
thought we saw him, and thought we felt the wounds and heard his voice." Such would be
extremely ridiculous. The prophets were dogmatic, and when they received revelation, had
visions and visitations from heavenly personages, they \textit{knew} it, they were not deceived, and
their assertions were dogmatic, righteously so. There are members of the Church by the
hundreds of thousands today, who can speak with knowledge, and dogmatically and
truthfully say, they know that God lives, that Jesus Christ is their Redeemer, that he was
resurrected from the dead and that the Gospel of Jesus Christ has been revealed from heavenand once more given unto men. They speak dogmatically, they cannot speak any other way.
They have the testimony of the truth from the most positive source from which eternal
knowledge can come. They have not closed their minds against further truth. They are not
asserting these things without knowing full well that they are true. They are not bigots, but
their religion is \textit{fixed} because it is given them by divine revelation. Joseph Smith was
dogmatic in relating his visitation of the Father and the Son, when he said: "I had actually
seen a light, and in the midst of that light I saw two Personages, and they did in reality speak
to me; and though I was hated and persecuted for saying that I had seen a vision, yet it was
true; and while they were persecuting me, reviling me, and speaking all manner of evil
against me falsely for so saying, I was led to say in my heart: Why persecute me for telling
the truth? I have actually seen a vision; and who am I that I can withstand God, or why does
the world think to make me deny what I have actually seen? For I had seen a vision; I knew
it, and I knew that God knew it, and I could not deny it, neither dared I do it; at least I knew
that by so doing I would offend God, and come under condemnation." He was not speaking
something that he did not know. The assertion had to be positive for it was true.

What could be more dogmatic than this expression coming from Dr. Andrew D. White:

Whatever additional factors may be added to natural selection—and Darwin himself fully
admitted that there might be others—the theory of an evolution process in the formation of
the universe and animated nature is established, and the old theory of direct revelation is
gone forever. In place of it science has given us conceptions far more noble, and opened the
way to an argument for design infinitely more beautiful than any ever developed by theology.
20

The whole question of dogmatism is seemingly this: If the dogmatism comes from
evolutionists, it is justified; but if it comes from religion, it is from a "closed mind" and an
"assertion of definite knowledge that is not possessed." The scientist asserts that the earth is a
globular body; that it makes one revolution on its axis every complete day; that it revolves in
its orbit around the sun once each year. All of these teachings and thousands more are
presented dogmatically. We do not deny that they are true. Rain is evaporated water from the
oceans and the face of the earth that has ascended above the earth and then under certain
conditions has been precipitated to the earth again. This doctrine is dogmatically proclaimed.
There are great numbers of laws in nature that have been discovered and are dogmatically
proclaimed. They are true. The religionist does not dispute these things. The trouble with the
scientist is that he denies to the religionist the right, or privilege, to know of spiritual things
which are not manifest to the scientific eye, heard by the scientific ear, or felt by the
scientific touch. But there are things that the Lord has revealed that are just as true as any
truth which may be demonstrated by scientific investigation; yet the scientist, especially if he
is an evolutionist, in his superior wisdom, denies to his religious fellows the right to claim
that he knows certain other truths, which scientifically cannot be discerned.

There is no scientific way ever discovered that can prove that there is a spirit in man; that
when a child is born into this world his eternal spirit which existed from before the
foundations of this earth were laid, has been united with the mortal body. 21 There is no
scientific way to prove that when a man dies, the spirit goes back to him who gave it. 22 It
cannot be proved scientifically that there has been, or will be a resurrection of the dead, but
many of the dead have come forth from their graves, 23 and the edict of the Almighty is that
all shall come forth who have lived, who are living now and will yet live on the earth. 24 Allof these truths have been revealed and have been positively known by the prophets. They are
known to be true by the righteous followers of Jesus Christ today. There are many living
today who have had the knowledge of these things revealed to them and they know them to
be true; but they have not been made known through the physical senses but through the
spiritual and those who have this knowledge know they are true. The mocking of the infidel,
the agnostic, the evolutionist, cannot change the fact. True religion, revealed to man from
Jesus Christ through the Holy Spirit, is dogmatic and those who have the knowledge speak
that which they know. It is not "assertiveness without knowledge"; it is not the assertion of
the ultimate which they do not possess. There is no truth that can be known more positively
than the truth revealed through the Holy Ghost. Moroni knew perfectly well that his promise
which is recorded in the 10th chapter of the Book of Moroni, would be fulfilled, and there are
many thousands who can testify to this truth. 25

The knowledge revealed to the humble believer in Jesus Christ, who has been baptized and
confirmed by the laying on of hands surpasses knowledge in the weight of its conviction
beyond that of any other source. For that reason the Lord gave the following commandment:

Wherefore I say unto you, All manner of sin and blasphemy shall be forgiven unto men: but
the blasphemy against the Holy Ghost shall not be forgiven unto men.

And whosoever speaketh a word against the Son of man, it shall be forgiven him: but
whosoever speaketh against the Holy Ghost, it shall not be forgiven him, neither in this
world, neither in the world to come. 26

Understanding the importance of the testimony of the Spirit, the author of the Hebrews
wrote:

For it is impossible for those who were once enlightened, and have tasted of the heavenly
gift, and were made partakers of the Holy Ghost,

And have tasted the good word of God, and the powers of the world to come,

If they shall fall away, to renew them again unto repentance; seeing they crucify to
themselves the Son of God afresh, and put him to an open shame. 27

Again:

For if we sin wilfully after that we have received the knowledge of the truth, there remaineth
no more sacrifice for sins,

But a certain fearful looking for of judgment and fiery indignation, which shall devour the
adversaries.

He that despised Moses' law died without mercy under two or three witnesses:

Of how much sorer punishment, suppose ye, shall he be thought worthy, who hath trodden
under foot the Son of God, and hath counted the blood of the covenant, wherewith he was
sanctified, an unholy thing, and hath done despite unto the Spirit of grace? 28

Peter also bore witness to this truth:

For if after they have escaped the pollutions of the world through the knowledge of the Lord
and

Saviour Jesus Christ, they are again entangled therein, and overcome, the latter end is worse
with them than the beginning.

For it had been better for them not to have known the way of righteousness, than, after they
have known it, to turn from the holy commandment delivered unto them.

But it is happened unto them according to the true proverb, The dog is turned to his own
vomit again; and the sow that was washed to her wallowing in the mire. 29

It is, however, impossible for men who have never felt the influence of the Spirit of the Lord,
but, to the contrary, have done all in their power to destroy the faith of the people in the Holy
Scriptures, and in the divine mission of Jesus Christ as the only begotten Son of God, to
understand spiritual things. It is a true saying that all such things are foolishness to them.
This is not a condition which prevails in these latter days that did not exist in ancient times.
Human nature has not changed through all the ages since mankind commenced to scatter
over the face of the earth. Today they have the same ambitions, the same weaknesses, the
same attitude of superiority. Job, thousands of years ago, understood this, and what he said to
the men who came to him in the hour of his affliction, can be applied today with equal truth
to all those who boast of their worldly wisdom, who lack humility and faith in God, but boast
in their own strength; "No doubt but ye are the people, and wisdom shall die with you!" 30

\newpage
REFERENCES—CHAPTER THREE

Footnotes

1. \textit{History of the Church}, Vol. 1, p. 305; Genesis 1:26-27; 1 Cor. 11:7. P. of G. P., Moses 1:6;
Heb. 1:3; 2:26-27; D. \& C. 20:18.

2. First Article of Faith; Matt. 3:11, 16; Luke 3:22; John 14:16-17; Matt. 28:18-19; D. \& C.
20:27, 29.

3. Moses 1:32-33; John 1:1-4; Eph. 3:9; Heb. 1:2.

4. Gen. 1:26-27. Moses 3:7.

5. Gen. 2:17; 3:3, 19; 1 Cor. 15:21-22; 2 Nephi 2:22.

6. Gen. 1:31; 2:17-19; 2 Nephi 2:22.

7. 2 Nephi 2:22.

8. D. \& C. 29:40-44.

9. Moses 5:11; 2 Nephi 2:22-25.

10. Gen. 3:17-19; 1 Cor. 15:21-22.

11. Gen. 3:17-19; Pratt, Parley P., \textit{Voice of Warning}, pp. 135-136; Taylor, John, \textit{Government
of God}, pp. 104-109.

12. 1 Cor. 15:22; John 3:14-17; 4:14; 5:24-29; 11:25-26; D. \& C. 29:22-25; Alma 11:40-45;
40:22-23.

13. Moses 4:1-2; Abraham 3:25-28; 2 Nephi.

14. D. \& C. 29:26-28; John 5:25-29; Rev. 20:4, 5, 12, 13.

15. Orson Pratt M. S. 69-70; Alma 42:24; D. \& C. 19:16-19.

16. Second Article of Faith.

17. D. \& C. Sec. 76; 1 Cor. 15:40-42.

18. 1 Cor. 15:29;D. \& C. Sec. 128; \textit{History of the Church}, Vol. 2, p. 380.

19. Millikan, Dr. R. A., \textit{Evolution in Science and Religion}, pp. 86-87.

20. White, Dr. A. D., \textit{History of Warfare of Science with Theology}, Vol. 1, p. 86.

21. Gen. 2:7; Luke 24.22. Alma 40:11.

23. Matt. 27:51-53; 3 Nephi 23:9-13;D. \& C. Sec. 13.

24. D. \& C. 27:12; John 5:25-29; 11:25; Rev. 20:4-5, 12-13; 2 Nephi 9:10-13; Mosiah 15:21-
26.

25. Moroni 10:4-5.

26. Matt. 12:31-32.

27. Hebrews 6:4-6.

28. \textit{Ibid.}, 10:26-29.

29. 2 Peter 2:20-22.

30. Job 12:2.

