\chapter{THE DOCTRINE OF GOD—3}

THE modern Christian world is in great confusion in relation to the Fatherhood of God and
the Sonship of Jesus Christ. The reason for this is that it has followed in large measure, if not
entirely, the creed formulated and adopted in the year 325 A.D., by the religious leaders
under the inspiration of Constantine, emperor of Rome. This action marked the point where
the church as it then existed rejected God and Jesus Christ his Only Begotten Son. From that
day until now the Christian world has confounded the Persons in the Godhead and attempted
to make of them a mysterious essence or spirit which fills the immensity of space. The pagan
scientist is just as sorely confused. God to them may be likened to the forces of nature, or
some unknowable influence which likewise fills the immensity of space. They talk of the
"God of Nature" and the "God of Science." Both the worshipers in their churches and the
scientists in their laboratory have rejected the real anthropomorphic God and in their writings
have ridiculed him. Both teach that such an idea as that God has a physical body after which
man was formed is a "primitive" doctrine harking back to "primitive man." It is essential,
therefore, that the true position of the members of the Church of Jesus Christ of Latter-day
Saints should be clearly defined in relation to their doctrine of the Godhead, more
particularly in relation to the Fatherhood of God and the Sonship of Jesus Christ.

Moreover, there are some members of the Church who are confused because of certain
passages of scripture which refer to Jesus Christ as both Father and Son. For instance, in the
testimony given by Oliver Cowdery, David Whitmer and Martin Harris, published in the
Book of Mormon, they say:

. . . and we know that it is by the grace of God the Father, and our Lord Jesus Christ, that we
beheld and bear record that these things are true. And it is marvelous in our eyes.
Nevertheless, the voice of the Lord commanded us that we should bear record of it;
wherefore, to be obedient unto the commandments of God, we bear testimony of these
things. And we know that if we are faithful in Christ, we shall rid our garments of the blood
of all men, and be found spotless before the judgment-seat of Christ, and shall dwell with
him eternally in the heavens. And the honor be to the Father, and to the Son, and to the Holy
Ghost, which is one God. Amen.

The meaning of this closing phrase is that the Father and the Son and the Holy Ghost
constitute one Godhead, or Governing Presidency, with infinite power and jurisdiction. Then
in the Book of Mormon and in the Doctrine and Covenants are to be found passages in which
Jesus Christ is referred to as God and in the opening sentences of the Gospel of John, it is
written:

In the beginning was the Word, and the Word was with God, and the Word was God.

The same was in the beginning with God.

All things were made by him; and without him was not any thing made that was made.

King Benjamin placed his people under covenant to keep the commandments of the Lord and
take upon them the name of Jesus Christ; then he said to them:

And now, because of the covenant which ye have made ye shall be called the children of
Christ, his sons, and his daughters; for behold, this day he hath spiritually begotten you; for
ye say that your hearts are changed through faith on his name; therefore, ye are born of him
and have become his sons and his daughters.

And under this head ye are made free, and there is no other head whereby ye can be made
free. There is no other name given whereby salvation cometh; therefore, I would that ye
should take upon you the name of Christ, all you that have entered into the covenant with
God that ye should be obedient unto the end of your lives.

And it shall come to pass that whosoever doeth this shall be found at the right hand of God,
for he shall know the name by which he is called; for he shall be called by the name of
Christ.

And now it shall come to pass, that whosoever shall not take upon him the name of Christ
must be called by some other name; therefore, he findeth himself on the left hand of God.
And I would that ye should remember also, that this is the name that I said I should give unto
you that never should be blotted out, except it be through transgression; therefore, take heed
that ye do not transgress, that the name be not blotted out of your hearts. 1

Like the Nephites in King Benjamin's day, we Latter-day Saints have likewise taken upon
ourselves the name of Christ. Each week at the Sacrament service, as we are commanded to
do, we take upon us his name always to remember him and that is what the Nephites
covenanted to do. We are indebted to him for all our blessings. On June 30, 1916, the First
Presidency and the Council of the Twelve, issued an epistle to the Church setting forth the
doctrine of the Church in regard to the relationship of Jesus Christ to the Father and our
relation to both the Father and the Son. This document is of such importance that it is here
reproduced in full:

THE FATHER AND THE SON: A DOCTRINAL EXPOSITION BY THE FIRST
PRESIDENCY AND THE TWELVE—The scriptures plainly and repeatedly affirm that God
is the Creator of the earth and the heavens and all things that in them are. In the sense so
expressed, the Creator is an Organizer. God created the earth as an organized sphere, but He
certainly did not create, in the sense of bringing into primal existence, the ultimate elements
of the materials of which the earth consists, for "the elements are eternal." (D. \& C. 93:33.)

So also life is eternal, and not created; but life, or the vital force, may be infused into
organized matter, though the details of the process have not been revealed unto man. For
illustrative instances see Genesis 2:7; Moses 3:7; and Abraham 5:7. Each of these scriptures
states that God breathed into the body of man the breath of life. See further Moses 3:19, for
the statement that God breathed the breath of life into the bodies of the beasts and birds. God
showed unto Abraham "the intelligences that were organized before the world was;" and by
"intelligences" we are to understand "personal spirits" (Abraham 3:22, 23); nevertheless, we
are expressly told that "Intelligence" that is, "the light of truth, was not created or made,
neither indeed can be." (D. \& C. 93:29.)

The term "father" as applied to Deity occurs in sacred writ with plainly different meanings.
Each of the four significations specified in the following treatment should be carefully
segregated.

1. \textit{"Father" as Literal Parent}—Scriptures embodying the ordinary signification—literally
that of Parent—are too numerous and specific to require citation. The purport of these
scriptures is to the effect that God the Eternal Father, whom we designate by the exalted
name-title "Elohim," is the literal Parent of our Lord and Savior Jesus Christ, and of the
spirits of the human race. Elohim is the Father in every sense in which Jesus Christ is so
designated, and distinctively He is the Father of spirits. Thus we read in the Epistle to the
Hebrews: "Furthermore we have had fathers of our flesh which corrected us, and we gave
them reverence: shall we not much rather be in subjection unto the Father of spirits, and
live?" (Hebrews 12:9.) In view of this fact we are taught by Jesus Christ to pray: "Our Father
which art in heaven, Hallowed be thy name."

Jesus Christ applies to Himself both titles, "Son" and "Father." Indeed, He specifically said to
the brother of Jared: "Behold, I am Jesus Christ. I am the Father and the Son." (Ether 3:14.)
Jesus Christ is the Son of Elohim both as spiritual and bodily offspring; that is to say, Elohim
is literally the Father of the spirit of Jesus Christ and also of the body in which Jesus Christ
performed His mission in the flesh, and which body died on the cross and was afterwards
taken up by the process of resurrection, and is now the immortalized tabernacle of the eternal
spirit of our Lord and Savior. No extended explanation of the title "Son of God" as applied to
Jesus Christ appears necessary.

2. \textit{"Father" as Creator}—A second scriptural meaning of "Father" is that of Creator, e.g. in
passages referring to any one of the Godhead as "The Father of the heavens and of the earth
and all things that in them are." (Ether 4:7; see also Alma 11:38, 39 and Mosiah 15:4.)

God is not the father of the earth as one of the the worlds in space, nor of the heavenly bodies
in whole or in part, nor of the inanimate objects and the plants and the animals upon the
earth, in the literal sense in which He is the Father of the spirits of mankind, Therefore,
scriptures that refer to God in any way as the Father of the heavens and the earth are to be
understood as signifying that God is the Maker, the Organizer, the Creator of the heavens and
the earth.

With this meaning, as the context shows in every case, Jehovah, who is Jesus Christ the Son
of Elohim, is called "The Father" and even "the very eternal Father of heaven and of earth"
(see passages before cited, and also Mosiah 16:15.) With analogous meaning Jesus Christ is
called "The Everlasting Father" (Isaiah 9:6; compare 2 Nephi 19:6.) The descriptive titles
"Everlasting" and "Eternal" in the foregoing texts are synonymous.

That Jesus Christ, whom we also know as Jehovah, was the executive of the Father, Elohim,
in the work of creation is set forth in the book, \textit{Jesus the Christ}, Chapter 4. Jesus Christ,
being the Creator is consistently called the Father of heaven and earth in the sense explained
above; and since His creations are of Eternal quality He is very properly called the Eternal
Father of heaven and earth.

3. \textit{Jesus Christ the "Father" of Those Who Abide in His Gospel}—A second sense in which
Jesus Christ is regarded as the "Father" has reference to the relationship between Him andthose who accept His Gospel and thereby become heirs of eternal life. Following are a few of
the scriptures illustrating this meaning.

In the fervent prayer offered just prior to His entrance into Gethsemane, Jesus Christ
supplicated His Father in behalf of those whom the Father had given unto Him, specifically
the apostles, and more generally, all who would accept and abide in the Gospel through the
ministry of the apostles. Read in our Lord's own words the solemn affirmation that those for
whom He particularly prayed were His own, and that His Father had given them unto Him; "I
have manifested thy name unto the men which thou gavest me out of the world: thine they
were and thou gavest them me; and they have kept thy word. Now they have known that all
things whatsoever thou hast given me are of thee. For I have given unto them the words
which thou gavest me; and they have received them, and have known surely that I came out
from thee, and they have believed that thou didst send me. I pray for them: I pray not for the
world, but for them which thou hast given me; for they are mine. And all mine are thine, and
thine are mine; and I am glorified in them. And now I am no more in the world, but these are
in the world, and I come to thee. Holy Father, keep through thine own name those whom
thou hast given me, that they may be one, as we are. While I was with them in the world, I
kept them in thy name: those that thou gavest me I have kept, and none of them is lost, but
the son of perdition; that the scripture might be fulfilled." (John 17:6-12.)

And further: "Neither pray I for these alone, but for them also which shall believe on me
through their word; That they all may be one; as thou, Father, art in me, and I in thee, that
they also may be one in us: that the world may believe that thou hast sent me. And the glory
which thou gavest me I have given them; that they may be one, even as we are one: I in them,
and thou in me, that they may be made perfect in one; and that the world may know that thou
hast sent me, and hast loved them, as thou hast loved me. Father, I will that they also, whom
thou hast given me, be with me where I am; that they may behold my glory, which thou hast
given me: for thou lovedst me before the foundation of the world." (John 17:20-24.)

To His faithful servants in the present dispensation the Lord has said: "Fear not, little
children, for you are mine, and I have overcome the world, and you are of them that my
Father hath given me." (D. \& C. 50:41.)

Salvation is attainable only through compliance with the laws and ordinances of the Gospel;
and all who are thus saved become sons and daughters unto God in a distinctive sense. In a
revelation given through Joseph the Prophet to Emma Smith the Lord Jesus addressed the
woman as "My daughter," and said: "for verily I say unto you, all those who receive my
gospel are sons and daughters in my kingdom." (D. \& C. 25:1.) In many instances the Lord
has addressed men as His sons (e.g. D. \& C. 9:1; 34:3; 121:7.)

That by obedience to the Gospel men may become sons of God, both as sons of Jesus Christ,
and, through Him, as sons of His Father, is set forth in many revelations given in the current
dispensation. Thus we read in an utterance of the Lord Jesus Christ to Hyrum Smith in 1829:
"Behold, I am Jesus Christ, the Son of God. I am the life and the light of the world. I am the
same who came unto mine own and mine own received me not; But verily, verily, I say unto
you, that as many as receive me, to them will I give power to become the sons of God, even
to them that believe on my name. Amen." D. \& C. 11:28-30.) To Orson Pratt the Lord spoke
through Joseph the Seer, in 1830: "My son Orson, hearken and hear and behold what I, the
Lord God, shall say unto you, even Jesus Christ your Redeemer; The light and the life of theworld, a light which shineth in darkness and the darkness comprehendeth it not; Who so
loved the world that he gave his own life, that as many as would believe might become the
sons of God. Wherefore you are my son." (D. \& C. 34:1-3.) In 1830 the Lord thus addressed
Joseph Smith and Sidney Rigdon: "Listen to the voice of the Lord your God, even Alpha and
Omega, the beginning and the end, whose course is one eternal round, the same today as
yesterday, and for ever. I am Jesus Christ, the Son of God, who was crucified for the sins of
the world, even as many as will believe on my name, that they may become the sons of God,
even one in me as I am one in the Father, as the Father is one in me, that we may be one." (D.
\& C. 35:1-2.) Consider also the following given in 1831: "Hearken and listen to the voice of
him who is from all eternity to all eternity, the Great I AM, even Jesus Christ—The light and
the life of the world; a light which shineth in darkness and the darkness comprehendeth it
not: The same which came in the meridian of time unto my own, and my own received me
not; But to as many as received me, gave I power to become my sons, and even so will I give
unto as many as will receive me, power to become my sons." (D. \& C. 39:1-4.) In a
revelation given through Joseph Smith in March, 1831 we read: "For verily I say unto you
that I am Alpha and Omega, the beginning and the end, the light and the life of the world—a
light that shineth in darkness and the darkness comprehendeth it not. I came unto my own,
and my own received me not; but unto as many as received me, gave I power to do many
miracles, and to become the sons of God; and even unto them that believed on my name gave
I power to obtain eternal life." (D. \& C. 45:7-8.)

A forceful exposition of this relationship between Jesus Christ as the Father and those who
comply with the requirements of the Gospel as His children was given by Abinadi, centuries
before our Lord's birth in the flesh: "And now I say unto you, Who shall declare his
generation? Behold, I say unto you, that when his soul has been made an offering for sin, he
shall see his seed. And now what say ye? And who shall be his seed? Behold I say unto you,
that whosoever has heard the words of the prophets, yea, all the holy prophets who have
prophesied concerning the coming of the Lord—I say unto you, that all those who have
hearkened unto their words, and believed that the Lord would redeem his people, and have
looked forward to that day for a remission of their sins, I say unto you, that these are his seed,
or they are the heirs of the kingdom of God. For these are they whose sins he has borne; these
are they for whom he has died to redeem them from their transgressions. And now, are they
not his seed? Yea, and are not the prophets, every one that has opened his mouth to prophesy,
that has not fallen into transgression, I mean all the holy prophets ever since the world
began? I say unto you that they are his seed." (Mosiah 15:10-13.)

In tragic contrast with the blessed state of those who become children of God through
obedience to the Gospel of Jesus Christ is that of the unregenerate, who are specifically
called the children of the devil. Note the words of Christ, while in the flesh, to certain wicked
Jews who boasted of their Abrahamic lineage: "If ye were Abraham's children, ye would do
the works of Abraham. . . . Ye do the deeds of your father. . . . If God were your Father, ye
would love me. . . . Ye are of your father the devil, and the lusts of your father ye will do."
(John 8, 39, 41, 42, 44.) Thus Satan is designated as the father of the wicked, though we
cannot assume any personal relationship of parent and children as existing between him and
them. A combined illustration showing that the righteous are the children of God and the
wicked the children of the devil appears in the parable of the Tares: "The good seed are the
children of the kingdom; but the tares are the children of the wicked one." (Matt. 13:38.)

Men may become children of Jesus Christ by being born anew— born of God, as the inspired
word states: "He that committeth sin is of the devil; for the devil sinneth from the beginning.
For this purpose the Son of God was manifested, that he might destroy the works of the devil.
Whosoever is born of God doth not commit sin; for his seed remaineth in him: and he cannot
sin, because he is born of God. In this the children of God are manifest, and the children of
the devil: whosoever doeth not righteousness is not of God, neither he that loveth not his
brother." (1 John 3:8-10.)

Those who have been born unto God through obedience to the Gospel may by valiant
devotion to righteousness obtain exaltation and even reach the status of godhood. Of such we
read: "Wherefore, as it is written, they are gods, even the sons of God." (D. \& C. 76:58;
compare 132:20, and contrast paragraph 17 in same section; see also paragraph 37.) Yet,
though they be gods they are still subject to Jesus Christ as their Father in this exalted
relationship; and so we read in the paragraph following the above quotation: "and they are
Christ's and Christ is God's." (D. \& C. 76:59.)

By the new birth—that of water and the Spirit—mankind may become children of Jesus
Christ, being through the means by Him provided "begotten sons and daughters unto God."
(D. \& C. 76:24.) This solemn truth is further emphasized in the words of the Lord Jesus
Christ given through Joseph Smith in 1833: "And now, verily I say unto you, I was in the
beginning with the Father, and am the Firstborn; And all those who are begotten through me
are partakers of the glory of the same, and are the church of the Firstborn." (D. \& C. 93:21,
22.) For such figurative use of the term "begotten" in application to those who are born unto
God see Paul's explanation: "for in Christ Jesus I have begotten you through the gospel." (1
Cor. 4:15.) An analogous instance of sonship attained by righteous service is found in the
revelation relating to the order and functions of Priesthood, given in 1832: "For whoso is
faithful unto the obtaining these two priesthoods of which I have spoken, and the magnifying
their calling, are sanctified by the Spirit unto the renewing of their bodies. They become the
sons of Moses and of Aaron and the seed of Abraham, and the church and kingdom, and the
elect of God." (D. \& C. 84:33, 34.)

If it be proper to speak of those who accept and abide in the Gospel as Christ's sons and
daughters—and upon this matter the scriptures are explicit and cannot be gainsaid nor
denied—it is consistently proper to speak of Jesus Christ as the Father of the righteous, they
having become His children and He having been made their Father through the second
birth—the baptismal regeneration.

4. \textit{Jesus Christ the "Father" by Divine Investiture of Authority}. A third reason for applying
the title "Father" to Jesus Christ is found in the fact that in all His dealings with the human
family Jesus the Son has represented and yet represents Elohim His Father in power and
authority. This is true of Christ in His preexistent, antemortal, or unembodied state in the
which He was known as Jehovah; also during His embodiment in the flesh; and during His
labors as a disembodied spirit in the realm of the dead; and since that period in His
resurrected state. To the Jews He said: "I and my Father are one," (John 10:30; see also
17:11, 22.); yet He declared "My Father is greater than I," (John 14:28.); and further, "I am
come in my Father's name." (John 5:43; see also 10:25.) The same truth was declared by
Christ Himself to the Nephites (see 3 Nephi 20:35 and 28:10.), and has been reaffirmed by
revelation in the present dispensation. (D. \& C. 50:43.) Thus the Father placed His name
upon the Son; and Jesus Christ spoke and ministered in and through the Father's name; andso far as power, authority and Godship are concerned His words and acts were and are those
of the Father.

We read, by way of analogy, that God placed his name upon or in the Angel who was
assigned to special ministry unto the people of Israel during the exodus. Of that Angel the
Lord said: "Beware of him, and obey his voice, provoke him not; for he will not pardon your
transgressions: for my name is in him." (Exodus 23:21.)

The ancient apostle, John, was visited by an angel who ministered and spoke in the name of
Jesus Christ. As we read: "The Revelation of Jesus Christ, which God gave unto him, to shew
unto his servants things which must shortly come to pass; and he sent and signified it by his
angel unto his servant John." (Revelation 1:1.) John was about to worship the angelic being
who spoke in the name of the Lord Jesus Christ, but was forbidden: "And I John saw these
things, and heard them. And when I had heard and seen, I fell down to worship before the
feet of the angel which shewed me these things. Then saith he unto me, See thou do it not:
for I am thy fellowservant, and of thy brethren the prophets, and of them which keep the
sayings of this book; worship God." (Rev. 22:8, 9.) And then the angel continued to speak as
though he were the Lord Himself: "And, behold, I come quickly; and my reward is with me,
to give every man according as his work shall be. I am Alpha and Omega, the beginning and
the end, the first and the last." (verses 12, 13.) The resurrected Lord, Jesus Christ, who had
been exalted to the right hand of God His Father, had placed His name upon the angel sent to
John, and the angel spoke in the first person, saying "I come quickly," "I am Alpha and
Omega," though he meant that Jesus Christ would come, and that Jesus Christ was Alpha and
Omega.

None of these considerations, however, can change in the least degree the solemn fact of the
literal relationship of Father and Son between Elohim and Jesus Christ. Among the spirit
children of Elohim the firstborn was and is Jehovah or Jesus Christ to whom all others are
juniors. Following are affirmative scriptures bearing upon this great truth. Paul, writing to the
Colossians, says of Jesus Christ: "Who is the image of the invisible God, the firstborn of
every creature: For by him were all things created, that are in heaven, and that are in earth,
visible and invisible, whether they be thrones, or dominions, or principalities, or powers: all
things were created by him, and for him: And he is before all things, and by him all things
consist. And he is the head of the body, the church: who is the beginning, the firstborn from
the dead; that in all things he might have the preeminence. For it pleased the Father that in
him should all fulness dwell." (Colossians 1:15-19.) From this scripture we learn that Jesus
Christ was "the firstborn of every creature" and it is evident that the seniority here expressed
must be with respect to antemortal existence, for Christ was not the senior of all mortals in
the flesh. He is further designated as "the firstborn from the dead," this having reference to
Him as the first to be resurrected from the dead, or as elsewhere written "the first fruits of
them that slept." (1 Corinthians 15:20, see also verse 23.); and "the first begotten of the
dead." (Revelation 1:5; compare Acts 26:23.) The writer of the Epistle to the Hebrews
affirms the status of Jesus Christ as the firstborn of the spirit children of His Father, and
extols the preeminence of the Christ when tabernacled in flesh: "And again, when he bringeth
in the firstbegotten into the world, he saith, And let all the angels of God worship him."
(Hebrews 1:6; read the preceding verses.) That the spirits who were juniors to Christ were
predestined to be born in the image of their Elder Brother is thus attested by Paul: "And we
know that all things work together for good to them that love God, to them who are the called
according to his purpose. For whom he did foreknow, he also did predestinate to be
conformed to the image of his Son, that he might be the firstborn among many brethren."
(Romans 8:28, 29.) John the Revelator was commanded to write to the head of the Laodicean
church, as the words of the Lord Jesus Christ: "These things saith the Amen, the faithful and
true witness, the beginning of the creation of God." (Revelation 3:14.) In the course of a
revelation given through Joseph Smith in May, 1833, the Lord Jesus Christ said as before
cited: "And now, verily I say unto you, I was in the beginning with the Father, and am the
Firstborn." (D. \& C. 93:21.) A later verse makes plain the fact that human beings generally
were similarly existent in spirit state prior to their embodiment in the flesh: "Ye were also in
the beginning with the Father; that which is Spirit, even the Spirit of truth." (verse 23.)

There is no impropriety, therefore, in speaking of Jesus Christ as the Elder Brother of the rest
of human kind. That He is by spiritual birth Brother to the rest of us is indicated in Hebrews:
"Wherefore in all things it behoved him to be made like unto his brethren, that he might be a
merciful and faithful high priest in things pertaining to God, to make reconciliation for the
sins of the people." (Hebrews 2:17.) Let it not be forgotten, however, that He is essentially
greater than any and all others, by reason (1) of His seniority as the oldest or firstborn; (2) of
His unique status in the flesh as the offspring of a mortal mother and of an immortal, or
resurrected and glorified, Father; (3) of His selection and fore-ordination as the one and only
Redeemer and Savior of the race; and (4) of His transcendent sinlessness.

Jesus Christ is not the Father of the spirits who have taken or yet shall take bodies upon this
earth, for He is one of them. He is The Son, as they are sons or daughters of Elohim. So far
as the stages of eternal progression and attainment have been made known through divine
revelation, we are to understand that only resurrected and glorified beings can become
parents of spirit offspring. Only such exalted souls have reached maturity in the appointed
course of eternal life; and the spirits born to them in the eternal worlds will pass in due
sequence through the several stages or estates by which the glorified parents have attained
exaltation.

The First Presidency and the Council of the Twelve Apostles of the Church of Jesus Christ of
Latter-day Saints.

Salt Lake City, Utah, June 30, 1916.

\newpage
REFERENCE—CHAPTER SIX

Footnotes

1. Mosiah 5:7:11.
