\chapter{AUTHENTICITY OF THE SCRIPTURES (New Testament)}

THE New Testament, although much nearer our day than the Old Testament, has been
subjected to the same sort of criticism that has been given to the older records. Some of the
critics have maintained that many of the books composing the New Testament were not
written until the second or even the third century of the Christian Era. Doubt has been thrown
on the authorship of the Synoptic Gospels, and by some even on the Gospel of John,
notwithstanding the declaration in the closing verses that the author is "the disciple whom
Jesus loved," and the one of whom Peter said, "What shall this man do?" As a part of the
argument throwing doubt on the authorship of the Synoptic Gospels, particularly Matthew
and Mark, it is said that they are "anonymous," but "ancient and commonly-held tradition has
connected them with Matthew and Mark." This tradition goes back as far as the second
century. The same doubt as to the authorship of the epistles of John and the Apocalypse has
likewise persisted to this day. One reason for the doubt in relation to the two Gospels is that
there is in them no reference to the authorship. The book of Luke is more readily accepted.

Another criticism that prevailed in the nineteenth century, but which has lost its force
through recent discoveries, is the \textit{time} when these Gospels and epistles were written. It was
the custom of critics in the second half of the nineteenth century to conclude that in them
were found words that were not in use until the second or third century. The notion also
prevailed that the authors wrote in classical language, or the language of the educated.
Therefore, since words were found in these gospels and some epistles which the scholars
thought were not coined until the later centuries, it was claimed that these books could not
have been written before these words were in current use. We can illustrate this by giving
this example: If there should be handed to us a document purporting to come from the hands
of the Prophet Joseph Smith which contained in it such words as airplane, automobile,
refrigerator, or referring to electric sewing machines, we would know that it was not written
in his day. It was on similar ground that these critics worked. The great difficulty with their
system, however, was that they were comparing the writings of the authors of the Gospels
and some of the epistles with the classical Greek and not with the vernacular.

Great discoveries were made about the close of the nineteenth century, and have continued to
be made since that time which have thrown a great flood of light upon these scripural
writings, and these discoveries have greatly altered, or destroyed, those former views. In
1897, two young scientists, B. P. Grenfell and Arthur S. Hunt, working for the Egyptian
Exploring Fund, excavating at Behnesa, the ancient Oxyrhynchus in the Nile Valley, some
120 miles south of Cairo, made what Dr. Camden M. Cobern says was a discovery that in
one thousand years had not been equalled. Almost by accident they discovered tons of Greek
papyri, great quantities of which were written in the language of the New Testament. Dr.
Cobern says: "When it is remembered that no one previous to this time had ever read even
one autograph manuscript which had been written by a scribe of the first century in the
language which the common people of Palestine and Egypt used in that era, the sensational
nature of this discovery may be more easily realized." Other papyri were discovered at
various times but none equal to the great discovery by Grenfell and Hunt. Up to 1915 only
about twenty of these documents had been deciphered. When they were examined it was
discovered that they had a wonderful bearing on the New Testament. Dr. Adolf Deissmann in
examining these papyri discovered that they were written exactly in the language of the New
Testament, which was not to be "any longer regarded as an esoteric, sacred language, or a
language to any considerable degree Hebraized by its Jewish authors."

"How it ought to curb the pride of specialists to find that the original discovery which opened
to the world the Egyptian language, the Babylonian language, and the New Testament
language, was in each case made not by a world-renowned expert, but by a shrewd young
man of good sense and insight unbiased by the trammels of scholarly tradition." 1

Dr. Deissmann declared that, with perhaps the exception of two or three books of the New
Testament, "they were written to working men in the tongue of the working man, the Bible
authors freely using the colloquialisms and even the solecisms of the market-place. This was
a theory which at first seemed too good to be true. It means that Wycliffe only did for
England what Matthew and Mark did for the Roman world. Christianity from its beginning
spoke the tongue of the peasant. Its crooked grammar and missed orthography and peculiar
syntax, upon which have been built so many theological castles in the air, are all found
paralleled exactly in the letters and other familiar documents of the first century. This
common Greek was spoken everywhere throughout the entire Roman empire, and even our
early church titles, such as 'bishop,' 'presbyter,' 'deacon,' etc. were well-known official names
used in the trade unions and other corporations, religious and civil, of that era. This
contention, which seemed at first utterly unbelievable, has now inside of twenty years gained
the adhesion of almost every great living Greek scholar and has caused the re-writing of the
New Testament lexicon and grammar." 2 Dr. Deissmann discovered that most of the "alleged
Hebraisms of the Septuagint were probably merely popular Greek expressions common in
the vernacular." Moreover, it is discovered that the books of the New Testament came out of
"the very generation in which the apostles lived, and by one noble effort interpreting results
in such a way that those who have followed him have done little more than supplement his
results, notwithstanding the enormous increase in the material now at the disposal of Greek
scholars." 3

Many of these manuscripts found by Doctors Grenfell and Hunt dated back to the first and
second centuries, and in them were found the expressions which the scholars declared were
unknown in that day. These explorers and archaeologists declare that among these
discoveries are some coming from the two centuries before Christ, and from the first, second
and third centuries following. It is well known today that in the life of our Savior there was a
widespread habit of writing. All classes of the people were able to read and write, and there
is no doubt, "the leading facts were written down and circulated almost as soon as they took
place—though doubtless at first in fragmentary form—so that probably the first account of
the death of Jesus, must be presumed to be written in the year he died.

"As soon as the converts became so many that the original apostles could not easily carry
authoritative facts personally to all Christian communities, a need of records would be felt
which, because of this general habit of note taking and writing could be supplied." 4 And on
the testimony of Dr. Edouard Naville, "these finds in the closing years of the nineteenth
century and the fore-part of the twentieth century, compel us, to place the authors of its
different parts (of the New Testament) in the time when they were said to have lived, among
their readers or their hearers to whom they spoke. This seems to the present writer the best
answer to the radical criticism and the most telling way of showing how insufficient and
often misleading are its results, which are generally brought forward as being above
discussion."

"If we put side by side the gospels, the epistles of Paul, and the writings which have been
discovered of the first century, we shall find in those 'as it were a new autographic
commentary,' the explanation of many expressions showing that 'the New Testament writings
were not theological treatises, but were mostly composed in the non-technical and rather
careless language of the street and home.' This comparative study led Dr. Milligan to declare
that 'in view of all the new light coming upon the question from recent discovery, it is safe to
conclude that with the probable exception of 2 Peter, all our New Testament writings may
now be placed within the first century, though the collection called the New Testament may
be of a later date.'

"This goes a long way to disprove many of the critical theories, attributing parts of a book
like the gospel of John to a later epoch, and cutting it up between various authors, some of
them quite unknown and mere literary creations.

"Archaeology has already done a great deal to modify the ideas or systems based on mere
literary or philological evidence." 5 Dr. Camden M. Cobern says, "The text of no other
ancient book is so certain as that of the New Testament. . . . Aramaic was the native language
of the Jews in Palestine in the first century. It has long been noticed that our Lord, at least in
times of excitement, spoke his native tongue. Delman and others have made this perfectly
clear. 6 It is very likely that all the disciples were not only bi-lingual but also tri-lingual just
as the modern Palestinians are. Syriac was a dialect of Aramaic and 'the first language into
which the New Testament was translated; and as the Greek text itself was written by men
who habitually taught in Syriac, the early versions in this tongue have a clearer affinity with
the original text than those of any other can possibly have, not excepting the Old Latin.'" So
says Dr. Cobern, quoting Dr. Agnes Smith Lewis. 7

Dr. Cobern says that two generations had not passed from the death of the last of the apostles
that four Gospels, "just these four and no others," although they were written by different
persons and in different countries had long been in use and recognized as Christian
scriptures, and were used by heretics as well as orthodox believers. "Thus many of the most
scholarly and weighty arguments ever formulated against the Christian faith have been
rendered obsolete." 8 At this present time there are more than 4,200 Greek manuscripts that
"have been collated, and they all confirm the integrity and purity of the New Testament text."
9 In addition to these documents there has been collected from these early centuries a great
deal of apocryphal writings. While they are filled with errors and are distorted, nevertheless
they do possess some value. Evidently these stories and attempts to quote the Savior and his
apostles have been handed down from those who were eyewitnesses of many incidents that
really occurred, but in the repeating have been distorted just as the stories current in most all
countries of Adam and Eve, the fall, the flood, etc., have been distorted. Many of these
stories, says Dr. Cobern, were not told maliciously. To illustrate: "The man born blind would
have a story to tell his posterity, likewise many of the multitude who were healed, blessed or
who witnessed the crucifixion. It is true that there arose many foolish stories coming largely
from over-zealous priests like we find in the Apocrypha, such as the Savior making clay
birds that flew away when he clapped his hands, stretching boards that his foster-father
Joseph cut too short, showing anger and cursing a boy who accidently bumped into him.
These, however, are all of a more recent date, coming centuries later."

In regard to which of the Evangelists wrote his Gospel first, we need have little concern. It is
a point which in the end is immaterial. Many of the intellectuals have centered on Mark,
endeavoring to show that Matthew and Luke borrowed from him. They have advanced a very
ingenious argument in support of this, but after all, it is speculative. There are other
distinguished writers who maintain that the books were placed in their proper order and that
Matthew was better qualified to write than either of the other synoptic writers. We learn from
the scriptures that Matthew was a tax-gatherer and as such would have to be competent, not
only in reading and writing but in the law. We are informed by very competent authority that
in the days of our Lord, and for that matter centuries before, writing was common among the
people. Especially was this the case in the days of our Savior's ministry. The discoveries at
Oxyrhynchus by Doctors Grenfell and Hunt have revealed that writing was almost as
common among all classes of people as it is today. Moreover, we have learned that as far
back as the days of Adam there was a divine command that records be kept and Adam's
children, "were taught to read and write, having a language which was pure and undefiled,"
10 and in the opening of the dispensation of the Fulness of Times, one of the first
commandments given to the Church was that records should be kept, 11 and "a regular
history" should be maintained and an historian was appointed. 12 The same counsel was
given to the Nephites and when the Lord visited them, he was concerned because some
important things had not been recorded.

All of this being true we may well believe that the Lord did not fail to have scribes appointed
to keep a record of his travels and ministry. For this purpose we know that John, who was so
constantly with the Lord, received such an appointment. 13 What would be more natural than
that a man like Matthew also received such an appointment? Mark and Luke did not have the
close association with the Lord that was given to Matthew and John. This being true, the
question arises, why should Matthew, the close companion of the Lord, who was competent
to make and keep a record of the Lord's sayings, have to depend upon one who obtained his
information from another?

Dr. B. F. C. Atkinson in an article with the title: \textit{The Composition of the Matthew Gospel},
published in Vol. 83 (1951) {Journal of Transactions of the Victoria Institute}, has presented a
very clear argument favoring Matthew as the first writer of the Evangelists, and giving
intelligent internal evidence that Matthew, the apostle, was indeed the writer of the book
which bears his name. It was either a matter of modesty that prevented these writers of the
Gospel from placing their names to their manuscripts, or their names, if attached, were not
continued in the numerous copies. Dr. Atkinson shows that there are events which point to
Matthew as the author, moreover, that the very nature of the work, emphasis being given
particularly to matters which could concern Matthew above his fellows, enter into this
Gospel. I merely call attention to this defense, which I think conclusive, but feel that it is
unnecessary to present a defense, because we have the assurance coming from the Lord to
Joseph Smith that this story of the ministry of our Redeemer was written by Matthew, and
the Lord called upon the Prophet to give us a more complete account of some things which
Matthew wrote and which appear in his writings as they have come down to us. This is found
in the Pearl of Great Price.

It is commonly believed that Mark obtained his record at the feet of Peter. In part this may be
true. Let us not lose sight of the fact, however, that there were manuscripts written and
records kept from the very days of Christ's ministry. The scholars are of a mind that there
was another account which they called "Q" from which each of the writers obtained
inspiration. I repeat, if the Lord was so particular with Adam's descendants and with the
Jaredites and Nephites, that a history be kept of all their doings, it is unreasonable for us to
believe that he paid no thought to record keeping when he was here in his ministry, and that
men who wrote the Gospels had to rely on memory and tradition many years later, to obtain
the Savior's words and acts in the most important period of the world's history.

It is more reasonable to believe that there was at least one recorder, perhaps several, among
the apostles who wrote and preserved the important sayings and events of the Lord's
ministry. That we have obtained but a fragmentary part of this history and his counsels is
very apparent, and so declared John. Moreover, those things that were preserved in some
measure have been lost to us by faulty transcribing of records, misinterpretations by scribes
and translators.

Quite generally scholars are content to give Mark and Luke credit for the books which bear
their names. Such criticism as is brought against them is negligible, but in relation to the
writings of John, the story is quite different. The professors and students of the New
Testament are greatly confused and have been led astray, because in the epistles of John and
in the Apocalypse, John the apostle is not named. In the second and third epistles of John he
is spoken of as the \textit{elder}. They cannot feature John the apostle as an elder, so they have
created an imaginery author of these epistles, and as for the Apocalypse, it is so different in
style there are many who maintain that the author of it could not possibly be the author of the
Fourth Gospel. So we find them in great confusion. With some exceptions the Gospel of John
is ascribed to the Apostle John due to the fact that in the concluding paragraphs the author
identifies himself as the disciple, "whom Jesus loved," and which also leaned on his breast at
supper, and said, "Lord, which is he that betrayeth thee?" No other John can possibly fit into
this picture. Strange it is, however, that there are some critics who are willing to assign this
glorious story to some other John.

Sir Frederick Kenyon, the British archaeologist, has written: "The question of the authorship
and date of the Fourth Gospel has been one of the storm-points of New Testament criticism
for over a century. The Tubingen School, which took its rise with F. C. Baur in 1831,
assigned it to the second half of the second century (about A.D. 170), and P. W. Schmiedel at
the beginning of the present century maintained that about A.D. 140 was the earliest possible
date for it. Such things exclude not only the authorship of the Apostle St. John, but also that
of any eyewitness of the events recorded, such as John the Elder, mentioned by Papias. It was
presented as a pseudonymous work, produced more than a century later than our Lord's life,
quite unreliable for historical detail, and embodying a theology of post-apostolic profoundly
tinged with Gnosticism. Other scholars have assigned other dates and they ascribe the book
to the Elder John, such is the case with Dr. Streeter. Others among English scholars, such as
Lightfoot, Wescott and others are firm in the belief that it was written by the Apostle John,
and between A.D. 85 and 95. It is quite generally accepted that the Apostle John is the
author."

In relation to the second and third epistles of John and the Apocalypse there is profound
ignorance. These two epistles commence by stating that they were written—the first by "the
elder unto the elect lady and her children," and the second, by "the elder unto the well
beloved Gaius." Therefore since Papias speaks of "John the Elder," and Papias was a
companion of Polycarp, the tradition arose that some greatly favored person who went by the
name of "John the Elder," wrote these epistles and likewise the Apocalypse. Throughout the
years this controversy goes on, some critics claiming that it is the work of John, the brother
of James and son of Zebedee, and others proclaiming it to be the work of some other John
who was known as the Elder.

Here, again, we see how we are blessed through the restoration of the Gospel. There was no
character called "John the Elder," except John the brother of James and an apostle. The Lord
has made this perfectly clear and it can be understood by every member of the Church who
wishes to know. John who calls himself an elder is the Apostle John. The Lord said to the
Church when defining the officers and their duties at the time of the organization of the
Church that an apostle is an elder. It is recorded in Section 20, verses 38 to 45:

\textit{The duty of the elders, priests, teachers, deacons, and members of the Church of Christ}—An
apostle is an elder, and it is his calling to baptize;

And to ordain other elders, priests, teachers, and deacons;

And to administer bread and wine—the emblems of the flesh and blood of Christ—

And to confirm those who are baptized into the church, by the laying on of hands for the
baptism of fire and the Holy Ghost, according to the scriptures;

And to teach, expound, exhort, baptize, and watch over the church;

And to confirm the church by the laying on of the hands, and the giving of the Holy Ghost;

And to take the lead of all meetings.

Since the restoration of the Gospel the apostles in the Church are called elders. It is by that
title that they greet each other, and so they greet all the men holding the Melchizedek
Priesthood. It is an honorable title. John respected it and as an elder of the Church he wrote
his epistles. This was fully understood in his day. Without doubt the brethren in the
dispensation of the Meridian of Time likewise used this term. This is evident also from the
fact that Peter, as well as John, so refers to himself in one of his own epistles. Writing to "the
strangers scattered throughout Pontus, Galatia, Cappadocia, Asia, and Bithynia, "he said:
"The elders which are among you I exhort, \textit{who am also an elder}, and a witness of the
sufferings of Christ, and also a partaker of the glory that shall be revealed." So there is no
mystery, and no unknown prophet called "The Elder," who wrote the Apocalypse, neither the
epistles of John.

The main argument against John writing the Apocalypse and the Gospel, is centered in the
fact that they are quite dissimilar. Why this should be such a problem is difficult to see. The
Gospel account by John the Beloved, is a narrative. It deals with historical facts, relating the
life and ministry of Jesus Christ. It is written in a spirit of tenderness and love showing in
every sentence, the great love that this disciple had for the Master. He was recounting events
with which he was distinctly familiar. No one was quite as closely intimate with the Lord as
was he. Peter, James and John, the brother of James, were chosen as a presidency among the
twelve. They shared the Lord's confidence on occasions when the others were not present.
John particularly, was near to him, and in writing of him he expressed himself in tenderness
and love. In writing the Apocalypse it was in vision. Events of great moment were before
him and overwhelmed him. They depicted scenes from the beginning of time to the end of
time, covering the periods of temporal existence of the earth, and recording what was to
come to pass. This was the great privilege of John above his fellows.

His mission as the writer of the Gospel and of the Book of Revelation was assigned to him
hundreds of years before he was born. This knowledge was revealed to Nephi nearly six
hundred years before the birth of Christ, when he, Nephi, also saw the visions of heaven and
had revealed to him the great events that were to come. Some of these he was commanded to
write, others he was told not to write, for that privilege had been assigned to another. That
other was John.

Nephi writes:

And it came to pass that the angel spake unto me, saying: Look!

And I looked and beheld a man, and he was dressed in a white robe.

And the angel said unto me: Behold one of the twelve apostles of the Lamb.

Behold, he shall see and write the remainder of these things; yea, and also many things which
have been.

And he shall also write concerning the end of the world.

Wherefore, the things which he shall write are just and true; and behold they are written in
the book which thou beheld proceeding out of the mouth of the Jew; and at the time they
proceeded out of the mouth of the Jew, or, at the time the book proceeded out of the mouth of
the Jew, the things which were written were plain and pure, and most precious and easy to
the understanding of all men.

And behold, the things which this apostle of the Lamb shall write are many things which
thou hast seen; and behold, the remainder shalt thou see.

But the things which thou shalt see hereafter thou shalt not write; for the Lord God hath
ordained the apostle of the Lamb of God that he should write them.

And also others who have been, to them hath he shown all things, and they have written
them; and they are sealed up to come forth in their purity, according to the truth which is in
the Lamb, in the own due time of the Lord, unto the house of Israel.

And I, Nephi, heard and bear record, that the name of the apostle of the Lamb was John,
according to the word of the angel. 14

If the learned men of the world would only hearken to the elders of the Church and heed the
revealed word of the Lord, it would save them from their errors and they would have a
knowledge of the things which so greatly trouble them and over which they stumble.

\newpage
REFERENCES—CHAPTER TWENTY-SEVEN

Footnotes

1. Cobern, Dr. Camden M., \textit{New Archaeological Discoveries}, p. 30.

2. \textit{Ibid.}, pp. 30-31

3. \textit{Ibid.}, pp. 38-39.

4. \textit{Ibid.}, p. 99.

5. \textit{Ibid.}, Preface xxiii.

6. \textit{Ibid.}, p. 174.

7. \textit{Ibid.}, p. 208.

8. \textit{Ibid.}, p. 209.

9. Moses 6:6.

10. D. \& C. 21:1.

11. \textit{Ibid.}, 47:1, 3, 4; 85:21.

12. 1 Nephi 14:20-27.

13. \textit{Ibid.}, 14:18-27.

14. See \textit{The Words of Jesus}, C. H. Delman (1902.)

