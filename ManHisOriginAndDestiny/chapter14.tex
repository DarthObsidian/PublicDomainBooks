\chapter{JESUS THE ADVOCATE}
HOW long Adam and Eve were in the Garden of Eden we do not know, but they were there
long enough to be taught by our heavenly Father and learn his language, which we are
informed was a perfect language. After Adam was driven from the Garden he taught his
children to read and write. Moreover, there was nothing to prevent him from being in the
presence of God his Father, who gave him commandments. After Adam's transgression he
was shut out of the presence of the Father who has remained hidden from his children to this
day, with few exceptions wherein righteous men have been privileged with the glorious
privilege of seeing him. The withdrawal of the Father did not break the communication
between men and God, for another means of approach was instituted and that is through the
ministry of his Beloved Son, Jesus Christ. Since the fall all revelation and commandments
from the Father have come through Jesus Christ. The people of the world do not understand
this, but our Lord and Savior came on the scene as our advocate between man and the Father,
and Adam was commanded to call upon God, but always in the name of Jesus Christ. This
has been the divine order all through the ages. Jesus is our advocate with the Father, and in
the beginning, shortly after Adam had been driven from the Garden, an angel of God came to
him and commanded him saying: "Wherefore, thou shalt do all that thou doest in the name of
the Son, and thou shalt repent and call upon God in the name of the Son forevermore." 1

John wrote to the members of the Church as follows:

My little children, these things write I unto you, that ye sin not. And if any man sin, we have
an advocate with the Father, Jesus Christ the righteous:

And he is the propitiation for our sins: and not for our's only, but also for the sins of the
whole world. 2

Paul also bears testimony to this fact in his epistles; here are some of those expressions:

Who is he that condemneth? It is Christ that died, yea rather, that is risen again, who is even
at the right hand of God, who also maketh intercession for us. 3

For there is one God, and one mediator between God and men, the man Christ Jesus;

Who gave himself a ransom for all, to be testified in due time. 4

Wherefore he is able also to save them to the uttermost that come unto God by him, seeing he
ever liveth to make intercession for them. 5

Then we have the words of our Savior bearing witness to his calling as the one who pleads
the cause of men, in a revelation to the Prophet Joseph Smith:

I am the same which have taken the Zion of Enoch into mine own bosom; and verily, I say,
even as many as have believed in my name, for I am Christ, and in mine own name, by the
virtue of the blood which I have spilt, have I pleaded before the Father for them.

But behold, the residue of the wicked have I kept in chains of darkness until the judgment of
the great day, which shall come at the end of the earth;

And even so will I cause the wicked to be kept, that will not hear my voice but harden their
hearts, and wo, wo, wo, is their doom. 6

Listen to him who is the advocate with the Father, who is pleading your cause before him—

Saying: Father, behold the sufferings and death of him who did no sin, in whom thou wast
well pleased; behold the blood of thy Son which was shed, the blood of him whom thou
gavest that thy-self might be glorified;

Wherefore, Father, spare these my brethren that believe on my name, that they may come
unto me and have everlasting life. 7

These quotations are from the Book of Mormon:

Wherefore, my beloved brethren, have miracles ceased because Christ hath ascended into
heaven, and hath sat down on the right hand of God, to claim of the Father his rights of
mercy which he hath upon the children of men?

For he hath answered the ends of the law, and he claimeth all those who have faith in him;
and they who have faith in him will cleave unto every good thing; wherefore he advocateth
the cause of the children of men; and he dwelleth eternally in the heavens. 8

Wherefore, he is the first-fruits unto God, inasmuch as he shall make intercession for all the
children of men; and they that believe in him shall be saved.

And because of the intercession for all, all men come unto God; wherefore, they stand in the
presence of him, to be judged of him according to the truth and holiness which is in him.
Wherefore, the ends of the law which the Holy One hath given, unto the inflicting of the
punishment which is affixed, which punishment that is affixed is in opposition to that of the
happiness which is affixed, to answer the ends of the atonement.

For it must needs be that there is an opposition in all things. 9

All of these quotations from the scriptures bear witness to the fact that Jesus Christ is the
advocate between man and his Eternal Father and he is pleading the cause of man before the
Father. There has grown up during these later generations, and particularly since the
introduction of the iniquitous doctrine that man has descended through animal ancestry, that
anciently peoples—each nation and tribe—had a multitude of gods which they worshiped,
and that the Israelites under Moses, Joshua and the later judges were not different from all
other peoples of the surrounding nations. Some of these advocates prate about these "tribal
gods," and say that "Jahweh" as they prefer to call Jehovah, the God of Israel, was no
different from the heathen gods. In other words he was but the creation of the minds of the
people and was vengeful and loved the shedding of blood. These advocates of these modern
theories teach that down through the centuries the idea of God changed and gradually he
became gentle, more sympathetic and merciful, until the coming of Jesus Christ who taught
that his Father was a God of love. So they "take the God of Jesus as their norm, not the God
of Joshua." For instance, Dr. Robert Andrew Millikan has written:

If I can assist ever so little by presenting some of my own reflections upon the place of
evolution in religion, I shall consider myself amply justified for having the temerity to speak
with no sort of authority.

I shall state my conclusion at the outset when I say that religion itself is one of the most
striking possible examples of evolution. In so saying I am uttering nothing that is in any way
heretical, nothing that is not said in every theological seminary of importance in every
denomination in the United States, nothing that is not said in every group of people who do
any reflecting at all, or who have any sort of familiarity with history and its interpretation.
For nothing stands out more clearly, even in Bible history, than the fact that religion, as we
find it in the world today, has evolved up to its present state from the crudest sort of
beginnings, and I propose to run rapidly over four stages of that evolution. 10

The eminent scientist then mentions his "four stages of evolution" to be:

(1) "Primitive man just beginning to come into consciousness of himself, to act not altogether
instinctively as the lower animals for the most part do, but with a little bit of reflection."

(2) "He personifies nature. He sees a spirit in the storm, a god, very like his powerful
enemy."

(3) To appease his god he offers sacrifice. Their conception of God is that he is still
extraordinarily man-like. Then came the teachings through Mohammed, Buddha, and finally
through Jesus who taught that 'the kingdom of heaven is within you.' Jesus struck the most
mortal blow that has ever been struck at all childish literalisms, at all the ideas which
underlie modern so-called fundamentalism, when he changed the realistic interpretation of
the Jewish scriptures, the anthropomorphic conception of God prevalent up to his time, and
saw God no longer merely as a powerful human being, but a being whose qualities
transcended all human qualities; when he cried, 'It hath been written . . . \textit{but} I say unto you';
when he saw a great benevolence behind the universe; when he taught, 'God is a spirit'; when
he said, 'The kingdom of heaven is within you.'"

(4) This fourth stage is the one we are in now. "A stage that is ushered in through the growth
of another sublime idea or through a new revelation from God to man, in the idea that has
come in human thinking out of the utilization of Galileo's method in the study of geology, of
biology, of physics, of palaeontology, of history, an idea in the development of which
Darwin has been one of many outstanding figures." 11

These views show how far a man can get from the truth when he is without the inspiration of
the Spirit of the Lord. Briefly let me say, that Jesus \textit{did not} teach that the "kingdom of God is
\textit{within} you." There is one passage in Luke (17:20-21.) where the Pharisees asked the Lord
when the kingdom of God would come, and he answered, according to the Authorized
Version: "The kingdom of God cometh not with observation: Neither shall they say, Lo here!
or, lo there! for, behold, the kingdom of God is within you"; but in the margin we have
"among" which is the proper interpretation of this passage. Constantly he spoke of his
kingdom and described it as a literal government. Luke also records the following, (Ch.
22:16.) at the supper of the Passover when our Lord informed his apostles that he is to leave
them: "For I say unto you, I will not any more eat thereof, until it be fulfilled in the kingdom
of God." Then in the prayer he taught his disciples: "Thy kingdom come. Thy will be done in
earth as it is in heaven." (Matt. 6:10.) How could the kingdom come if it was within them?
Consider also the Lord's answer to Pilate's question: "Art thou the King of the Jews?" Jesus
answered him, "Sayest thou this thing of thyself, or did others tell it thee of me? ..." Jesus
answered: "My kingdom is not of this world: if my kingdom were of this world, then would
my servants fight, that I should not be delivered to the Jews: but now is my kingdom not
from hence." Pilate asked again, "Art thou a king then?" Jesus answered, "Thou sayest that I
am a king. To this end was I born, and for this cause came I into the world, that I should bear
witness unto the truth. Every one that is of the truth heareth my voice." (John 18:33-37.)

How strange it is that these great men are willing to accept the teachings of Jesus Christ
when it suits them, and they laud him as a great teacher; but they inconsistently reject all of
his teachings concerning his divine mission, his declaration that he \textit{is} the Son of God, and the
evidence of his death, burial and resurrection. They ridicule the doctrine that God, the Father
of Jesus Christ, is an anthropomorphic being—such a thing to them is "childish" and they
even accuse Christ of destroying the doctrine. Yet he taught that he was in the express image
of his Father in his answer to Thomas' question. The inconsistency of these followers of
Darwin is almost beyond belief. So, likewise, Dr. Andrew D. White, a bitter opponent of the
Bible, teaches that the Hebrew, or Israelite nation, obtained its doctrines from the Chaldean-
Babylonian sources. 12

Unfortunately there are some professed members of the Church who have had their judgment
warped by these foolish theories supposed to be based on scientific research. The fact is that
the true doctrines of Christianity, which were introduced to Adam in the beginning became
corrupted through the apostasy of Adam's descendants. It is written by Moses that the Lord
said to Adam:

And it is given unto them to know good from evil; wherefore they are agents unto
themselves, and I have given unto you another law and commandment.

Wherefore teach it unto your children, that all men, everywhere, must repent, or they can in
nowise inherit the kingdom of God, for no unclean thing can dwell there, or dwell in his
presence; for, in the language of Adam, Man of Holiness is his name, and the name of his
Only Begotten is the Son of Man, even Jesus Christ, a righteous Judge, who shall come in the
meridian of time.

Therefore I give you a commandment, to teach these things freely unto your children, saying:

That by reason of transgression cometh the fall, \textit{which fall bringeth death}, and inasmuch as
ye were born into the world by water, and blood, and the spirit, which I have made, and so
became of dust a living soul, even so ye must be born again into the kingdom of heaven, of
water, and of the Spirit, and be cleansed by blood, even the blood of mine Only Begotten;
that ye might be sanctified from all sin, and enjoy the words of eternal life in this world, and
eternal life in the world to come, even immortal glory;

For by the water ye keep the commandment; by the Spirit ye are justified, and by the blood
ye are sanctified; 13

Therefore it is given to abide in you; the record of heaven; the Comforter; the peaceable
things of immortal glory; the truth of all things; that which quickeneth all things, which
maketh alive all things; that which knoweth all things, and hath all power, according to
wisdom, mercy, truth, justice, and judgment.

And now, behold, I say unto you: This is the plan of salvation unto all men, through the
blood of mine Only Begotten, who shall come in the meridian of time. 14

To hold such views as those expressed by these men full of worldly wisdom, can only be
done by a complete rejection of the plan of salvation and the revelations of God to his
servants the prophets as recorded in the scriptures. Here we learn that Adam taught his
children of Christ and his mission in the world in the meridian of time as their Redeemer.
They were made acquainted with the plan of salvation and that same truth and Priesthood
was with men in the beginning which is on the earth today through the restoration of the
Gospel. The condition of retrogression which followed was not the beginning of civilization,
but it came because of the rejection of civilization—the civilization established by God the
Father and his Son Jesus Christ. It is written:

And Adam and Eve blessed the name of God, and they made all things known unto their sons
and their daughters.

And Satan came among them, saying: I am also a son of God; and he commanded them,
saying: Believe it not; and they believed it not, and they loved Satan more than God. And
men began from that time forth to be carnal, sensual, and devilish. 15

It was the rejection of the principles that Adam taught and turning to doctrines of Satan that
brought to pass the wickedness in the world and turned men away from the truth to the
worship of forces of nature and false gods. So the "evolution" that followed instead of being
what these great men claim it to be was the "evolution" of retrogression, and this is still going
on because of the evolutionary theories, and instead of progress in the knowledge of God,
and the proper worship, men are following theories which lead them farther and farther from
the doctrines of Jesus Christ and the eternal plan of salvation given to Adam in the
beginning.

It is also a misunderstanding prevalent everywhere that the God of Israel known as Jehovah
was someone different from Jesus Christ. Even among members of the Church there are
many who believe that it was the Father, and not Jesus who spoke to Enoch, who
commanded Noah to build an ark and who talked with Abraham and the ancient prophets. In
some of the more recent "translations" of the scriptures, the name of Jehovah is used instead
of saying the Lord. And there is confusion because Jehovah, even among believers is thought
to be God the Father. As I stated in the beginning of this chapter, the Father withdrew from
having personal contact with his children and Jesus Christ as the advocate and mediator
between God the Father and mankind comes upon the scene. Here is an article written by
President George Q. Cannon many years ago, in answer to this question, showing clearly that
it was our Redeemer who delivered messages and led Israel and the prophets in ancient
times:

JESUS THE GOD OF ANCIENT ISRAEL

The following question is asked by an Ogden correspondent: In the leaflet of January 7th,
subject, God—are we to understand that God the Father spoke to Moses face to face on Mt.
Sinai, or are we to understand that it was Jesus Christ?

There is in modern Christendom a strong tendency to ascribe to the Father visits and
communications with mankind that were really made by the Lord Jesus. There is even a
respectable percentage of the members of His Church, established in these days, who have
the idea that it was the Father and not the Son who appeared to the patriarchs and prophets of
old, who delivered Israel from Egypt, who gave the law on Sinai, and who was the guide and
inspirer of the ancient seers. This was not the understanding of the true servants of God either
before or after His coming. Those who preceded the advent of the Messiah understood that
He whom they worshiped as Jehovah should in due time tabernacle in the flesh, and the
writings of Justin Martyr and others of the early fathers show that this was the belief of the
early Christian Church on the eastern continent. The writings of the Hebrew prophets, as we
have them in the Bible, are perhaps not as plain on this point as are those of the Nephite seers
that are revealed to us in the Book of Mormon. But we have in this latter record some
quotations from the earlier Hebrew prophets that make this point very clear. Nephi writes, (1
Nephi 19:10.):

"And the God of our fathers, who were led out of Egypt, out of bondage, and also were
preserved in the wilderness by him, yea, the God of Abraham, and of Isaac, and the God of
Jacob, yieldeth himself, according to the words of the angel, as a man, into the hands of
wicked men, to be lifted up, according to the words of Zenock, and to be crucified, according
to the words of Neum, and to be buried in a sepulchre, according to the words of Zenos."
Here we have the testimony of Zenock, Neum and Zenos that the God of Abraham, Isaac and
Jacob was by wicked men to be lifted up, crucified and afterwards buried in a sepulchre,
showing that these ancient worthies understood that it was the God of Israel who would come
to His own. Nephi who himself was a Hebrew and the son of a prophet of the same race, also
testifies in the above passage that it was the same God of their fathers who led them out of
Egypt and preserved them in the wilderness. About four hundred years later another Nephite
seer, King Benjamin, testifies that an angel came to him and made this glorious promise:

"For behold, the time cometh, and is not far distant, that with power, the Lord Omnipotent
who reigneth, who was, and is from all eternity to all eternity, shall come down from heaven
among the children of men, and shall dwell in a tabernacle of clay, and shall go forth
amongst men, working mighty miracles, such as healing the sick, raising the dead, causing
the lame to walk, the blind to receive their sight, and the deaf to hear, and curing all manner
of diseases."

A little later he says:

"And he shall be called Jesus Christ, the Son of God, the Father of heaven and earth, the
Creator of all things from the begining; and his mother shall be called Mary.

"And lo, he cometh unto his own, that salvation might come unto the children of men, even
through faith on his name; and even after all this they shall consider him a man, and say that
he hath a devil, and shall scourge him, and shall crucify him." (Mosiah 3:5, 8, 9.)

But we have the word of the Savior himself on this point that puts controversy to an end.
When, after his resurrection and ascension into heaven, He first appeared to His Nephite
disciples on this land, He declared, "Behold, I am Jesus Christ, whom the prophets testified
shall come into the world. . . . I am the God of Israel, and the God of the whole earth, and
have been slain for the sins of the world." (3 Nephi 11:10, 14.) Later during his ministry
among the Nephites he affirms: "Behold, I say unto you, that the law is fulfilled that was
given unto Moses. Behold, I am he that gave the law, and I am he who covenanted with my
people Israel; therefore, the law in me is fulfilled." (3 Nephi 15:4, 5.)

Should any still have a lingering doubt that the Jehovah who revealed Himself to Abraham,
to Moses and to others was any other than He whom we know in the flesh as Jesus Christ,
that doubt is set at rest by the revelations given in these days. In the vision seen by the
Prophet Joseph Smith and Oliver Cowdery in the Kirtland Temple, 3rd of April, 1836, the
following appears:

"We saw the Lord standing upon the breast-work of the pulpit, before us; and under his feet
was a paved work of pure gold, in color like amber.

"His eyes were as a flame of fire; the hair of his head was white like the pure snow; his
countenance shown above the brightness of the sun; and his voice was as the sound of the
rushing of great waters, even the voice of Jehovah, saying:

"I am the first and the last; I am he who liveth, I am he who was slain; I am your advocate
with the Father." (D. \& C. Sec. 110:2-4.)

Somewhat curiously an ancient Syriac manuscript has within the past few months been
unearthed that is known as the Gospel of the Twelve Apostles. Whether the Twelve Apostles
had anything to do with writing it has nothing to do with the point under consideration. The
writing was originally in Hebrew, and what we wish to draw attention to is that whenever
this manuscript was first written, the writers of the original believed that Jesus was He who
spake with the ancient Israelites. It commences:

"The beginning of the Gospel of Jesus Christ, the Son of the living God, according as it was
said by the Holy Ghost, I send an angel before his face, who shall prepare his way.

"It came to pass in the 309th, year of Alexander, the son of Philip the Macedonian, in the
reign of Tiberius Caesar, in the government of Herod, the ruler of the Jews, that the angel
Gabriel, the chief of the angels, by command of God went down to Nazareth to a virgin
called Miriam, of the tribe of Judah the son of Israel (her who was betrothed to Joseph the
Just), and he appeared to her and said, 'Lo! there ariseth from thee the one who spake with
our fathers, and he shall be a Savior to Israel; and they who do not confess him shall perish,
for his authority is in the lofty heights, and his kingdom does not pass away.'" (\textit{Juvenile
Instructor}, Vol. 35, pp. 90-91.)

In Exodus, chapter 6, verses one to three, according to the Authorized Version, we find the
following:

Then the Lord said unto Moses, Now shalt thou see what I will do to Pharaoh: for with a
strong hand shall he let them go, and with a strong hand shall be drive them out of his land.

And God spake unto Moses, and said unto him, I am the Lord:

And I appeared unto Abraham, unto Isaac, and unto Jacob, by the name of God Almighty,
but by my name JEHOVAH was I not known to them.

This passage should be a sweet morsel to Dr. White and those who seem to delight in finding
contradictions in the Bible, and this is a contradiction. He says that in the first chapter of
Genesis, the Elohistic account, it states that the waters brought forth the fowl, and in the
second chapter, the Jehovistic account, it says that the "land animals and birds are declared to
have been created not out of water, but \textit{'out of the ground'}." Therefore the story of creation
cannot be correct. 16 It seems that there is one point that the learned doctor overlooked. He
did not stop to consider that we do not have an original record of the Book of Genesis and
have to rely on a copy, presumably after having been copied a score of times. It is rather
childish to raise a question whether the fowl came from the sea or the land, when we have to
depend on faulty translations which fact is admitted by Bible scholars. So this passage in
Exodus, while it confirms the doctrine that the God of Abraham, Isaac and Jacob, was
Jehovah, yet it is one of the passages incorrectly translated. Correctly it should read:

And God spake unto Moses, and said unto him, I am the Lord:

And I appeared unto Abraham, unto Isaac, and unto Jacob. I am the Lord God Almighty; the
Lord JEHOVAH. And was not my name known unto them?

Yea, and I have also established my covenant with them, which I made with them, to give
them the land of Canaan, the land of their pilgrimage, wherein they were strangers. 17

These evolutionists; these mightymen of renown, like those mentioned in the days before the
flood, "the same became mighty men which were of old, men of renown," had one great
defect, their "wickedness was great," and the "imagination of the thoughts of their hearts was
only evil continually." Any man, no matter how renowned he becomes who endeavors to
destroy faith in God, in Jesus Christ as the Redeemer of the world and the Savior of men—
the Only Begotten Son of God—is a wicked man. I care not how well he may be received
and honored by his fellow men, if he takes a course of that kind, which will tend to lead
admiring persons away from the worship of the true God, Elohim, and his Beloved Son,
Jesus Christ, and will ridicule the sacred writings of the scriptures, he is a wicked man. I
repeat, the greatest crime of all is to destroy faith in God and in the living, unchangeable
saving principles of the Gospel.

And I now give unto you a commandment to beware concerning yourselves, to give diligent
heed to the words of eternal life.

For you shall live by every word that proceedeth forth from the mouth of God.

For the word of the Lord is truth, and whatsoever is truth is light, and whatsoever is light is
Spirit, even the Spirit of Jesus Christ.

And the Spirit giveth light to every man that cometh into the world; and the Spirit
enlighteneth every man through the world, that hearkeneth to the voice of the Spirit.

And every one that hearkeneth to the voice of the Spirit cometh unto God, even the Father.

And the Father teacheth him of the covenant which he has renewed and confirmed upon you,
which is confirmed upon you for your sakes, and not for your sakes only, but for the sake of
the whole world.

And the whole world lieth in sin, and groaneth under darkness and under the bondage of sin.

And by this you may know they are under the bondage of sin, because they come not unto
me.

For whoso cometh not unto me is under the bondage of sin.

And whoso receiveth not my voice is not acquainted with my voice, and is not of me.

And by this you may know the righteous from the wicked, and that the whole world groaneth
under sin and darkness even now. 18

\newpage
REFERENCES—CHAPTER FOURTEEN

Footnotes

1. Moses 5:8.

2. 1 John 2:1-2.

3. Romans 8:34.

4. 1 Tim. 2:5-6.

5. Heb. 7:25.

6. D. \& C. 38:4-6.

7. \textit{Ibid.}, 45:3-5.

8. \textit{Ibid.}, 45:3-5.

9. 2 Nephi 2:9-11.

10. Millikan, Dr. R. A., \textit{Evolution in Science and Religion}, pp. 65-66.

11. \textit{Ibid.}, pp. 67-80.

12. White, Dr. A. D., \textit{Warfare of Science with Theology}, Vol. 1, p. 20.

13. Moses 6:56-62

14. \textit{Ibid.}, 5:12-13.

15. White, Dr. A. D., \textit{Warfare of Science with Theology}, Vol. 1, p. 50.

16. D. \& C. 84:43-53

17. Compare 1 John 5:4-12.

18. From the Prophet Joseph Smith's revision, Exodus 6:2.4.

