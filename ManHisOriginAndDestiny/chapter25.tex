\chapter{AUTHENTICITY OF THE SCRIPTURES (Old Testament—Part One)}

IT appears to be a habit with critics of the Christian religion to blame the Lord and the Bible
for all the mistakes that have been made during the days of spiritual darkness, known as the
"dark ages," which came upon the Christian world following the death of the apostles. Such
doctrines were taught by the catholic Church and some of the "early fathers," as part of
orthodox Christianity, which had been taken from the pagan world. Every Latter-day Saint
knows that following the death of the apostles, Paul's prophecy was fulfilled, for there were
many "grievous wolves" that entered the flock, and men arose "speaking perverse things," so
that the doctrines were changed and the true Church of Jesus Christ ceased to be on the earth.
1 For this reason there had to come a restoration of the Church and a new revelation and
bestowal of divine authority. The Church of Jesus Christ and the Holy Scriptures are,
therefore, not responsible for the changed doctrines and unscientific teachings of those times,
when uninspired ecclesiastics controlled the thinking of the people. The Bible is not in
conflict with authenticated principles taught in science. It should not be forgotten that the
false doctrines concerning science in those days of spiritual darkness were not peculiar to the
church that then existed. These doctrines were just as much the views of the advocates of
science until some inspired individual arose to correct the errors. For men of science to point
the finger of accusation at the uninspired ecclesiastics is hardly fair; for they were merely
following the scientific teachings of their day. It was not the corrupted church that was
responsible for these false theories, for they had borrowed them from those who were
teaching, supposedly, scientific truth. The great fault of the ecclesiastics was that they
maintained, falsely, that the prevailing notions of their time were doctrines in harmony with
the scriptures. In their ignorance they punished any who denied their cherished theories.

It was the universal idea of the "dark ages" that the earth was flat and the center of the
universe; the firmament, or sky, a solid dome. It was argued that if the earth was round
everything would fall off on the under side, and if they, by some means, were able to remain
they would be standing top-side down. Such a thing was to them scientifically sound. I
repeat, it is unfair to ascribe this doctrine solely to the church of that day, for it was equally
the doctrine of the teachers of science, until someone, not always a man of science, and most
always a member of the church, took steps to demonstrate the falsity of those notions. Yet
many writers who deny the Bible to be the divine work of God, and who bitterly oppose its
teachings, attempt to lay the blame for these false doctrines upon the scriptures. In doing so
these scientific men also make many mistakes. It should be remembered that the theories of
science have been undergoing radical changes ever since man became scientifically minded.
In this present age, many of the scientific theories that have stood the test apparently for
centuries, are threatened with radical change. Many others that are comparatively new, that
is, that have been advanced within the past one hundred years, are finding their way into the
discard.

Dr. John William Draper, one of the leading critics of religious teachings, has said:

A divine revelation of science admits of no improvement, no change, no advance. It
discourages as needless, and indeed as presumptious, all new discovery, considering it as an
unlawful prying into things which it was the intention of God to conceal.

What, then, is that sacred, that revealed them from those who were teaching, supposedly,
scien-knowledge?

It likened all phenomena, natural and spiritual, to human acts. It saw in the Almighty, the
Eternal, only a gigantic man.

As to the earth, it affirmed that it is a flat surface, over which the sky is spread like a dome,
or, as St. Augustine tells us, is stretched like a skin. In this the sun and moon and stars move,
so that they may give light by day and by night to man. The earth was made of matter created
by God out of nothing, and, with all the tribes of animals and plants inhabiting it, was
finished in six days. Above the sky or firmament is heaven; in the dark and fiery space
beneath the earth is hell. The earth is the central and most important body of the universe, all
other things being intended for and subservient to it. 2

It is hard to see how a man could make more mistakes in so brief a space. Yet this is typical
of the attacks that are made by these critics of religion and the sacred scriptures. Dr. Andrew
D. White also gloats over what he thinks are the mistakes of revelation and the inspiration of
the Biblical prophets. In a similar trend of thought this eminent scholar has said:

We have already noted that there are generally three periods or phases in a theological attack
upon any science. The first of these is marked by the general use of scriptural texts and
statements against the new scientific doctrine; the third by attempts at compromise by means
of far-fetched reconciliations of textual statements with ascertained fact; but the second or
intermediate period between these two is frequently marked by the pitting against science of
some great doctrine in theology. 3

These gentlemen, we repeat, have fallen into the error of confusing divine revelation with the
doctrines of the Catholic popes and priests. Naturally opinion and interpretations of
uninspired priests and ministers cannot be taken as the criterion by which the revelations and
doctrines of the holy scriptures are to be tried. Permit me to offer a correction of some of the
conclusions reached by these great men.

It is true that a divine revelation admits of no change, but it may admit of additional
knowledge or development and information. It may, in fact, for cause, be revoked. The Lord
does not always reveal the fulness of a principle at first and he certainly has the right to
reserve to himself other and greater knowledge. His word to man comes in steps, peacemeal,
as his servants are prepared to receive it. But there will be no conflict between the part first
revealed, and the latter part revealed, they will harmonize. The revelation of some scientific
truth does not preclude any "new discovery," or addition to what went before. The Church of
Jesus Christ is not now, and never was, bound by the uninspired utterances of the "fathers,"
who were relying on their own judgment, after the falling away from the Gospel. Nor is the
Church of Jesus Christ bound by any statement in the scriptures, bearing a false interpretation
and translation by uninspired men. The true Church does not "liken all phenomena, natural
and spiritual, to human acts." It does accept God the Father as an "exalted man," but not a
"gigantic man." The Church of Jesus Christ does not now, and never did, accept or believe,
that the earth is flat, the sky a dome, or that matter was "created out of nothing." No such
thing is declared in the scriptures. These are nonsensical views that crept into apostate
Christianity. No such thing is recorded in the Bible. From this point let us pay attention to
some of the specific charges hurled against the Church and its revelations.

1. \textit{The flat earth}. Attention has already been called to the fact revealed to the Prophet Joseph
Smith that Methuselah, through the gift of God, was an astronomer and was acquainted with
the heavenly bodies. Kolob's position among them was known to him, and that it is a "fixed"
star and one of the governing bodies. 4 From the Book of Abraham we discover that
Abraham taught astronomy to the Egyptians, and that he was acquainted with the shape of
the earth and the heavenly bodies. We know from the writings of Mormon that the Nephites
were acquainted with the heavenly bodies and knew that the earth revolved around the sun. 5
The great calendar stone of Mexico is evidence that the ancient peoples on this continent
were acquainted with the stars and the revolutions of the earth and knew that the earth is not
flat. The enemies of truth have to stretch a point to find in the Bible any evidence pointing to
a flat earth. When the Psalmist wrote: "When I consider the heavens, the work of thy fingers,
the moon and the stars, which thou hast ordained; What is man, that thou art mindful of him?
and the son of man, that thou visitest him?" And again, "The heavens declare the glory of
God; and the firmament sheweth his handywork. Day unto day uttereth speech, and night
unto night sheweth knowledge. There is no speech nor language, where their voice is not
heard." These sayings imply an understanding of the nature of the stars and planets. Then in
Job 9:8: "Which maketh Arcturus, Orion, and Pleiades, and the chambers of the south,"
indicates that in that ancient day there was an understanding of the heavens. And then in the
same book, the words of the Lord to Job: "Canst thou bind the sweet influences of Pleiades,
or loose the bands of Orion?" convinces us that in that day they were not ignorant of the
heavenly bodies and astronomy.

Dr. Andrew D. White ridicules sayings in the Bible such as the "four corners of the earth,"
and an angel holding the "four winds of the earth" (Rev. 7:1; Dan. 7:2.), "the pillars of
heaven," the "doors of heaven," "windows of heaven." 6 Such criticisms are childish in the
extreme, and show a profound ignorance of the poetical nature of the Hebrew mind. Should
we cut out of modern poetry and classical literature all the figures of speech, there would be
only the wooden remains left, inert and lifeless. We find in our scriptures some of the most
expressive and meaningful language that was ever written. Our modern translators through
their tampering with these writings, have robbed them of their beauty and color. Do we take
literally the words of the Psalmist when he says, "The fool hath said in his heart, there is no
God?" Does he say it in his heart? Change the expression and half the meaning is lost. Again:
"Save me, O God; for the waters are come in unto my soul. I sink in deep mire, where there
is no standing; I am come into deep waters, where the floods overflow me." Do we not
comprehend the meaning without taking every word with a literal meaning? So what is
wrong, or what conveys a false impression in speaking of the "four winds of heaven," or of
the "pillars of heaven," or of the "doors of heaven?" Do any of these sayings convey to an
intelligent person the idea that these ancient poets were speaking literally? Do we not speak
today the same way, and are we misunderstood? Yet so bitter is Mr. White that he can see
nothing but ignorance in the facts clothed in such expressions. We also speak of the four
corners of the earth, and we are not confused into thinking that the earth is a square body
with four corners. Today we say, "the sun sets; the moon rises." We know that both are
wrong. Let us try to be fair with these beautiful sayings.

2. \textit{The firmament of heaven}. Writes Mr. White: "In both accounts (i.e. Chaldean and Hebrew)
there is placed over the whole creation a solid, concave firmament; in both, light is created
first, and the heavenly bodies are afterwards placed 'for signs and for seasons,' and this is said
in ridicule of the term \textit{firmament} as used in the Bible." 7 Ecclesiastical writers as well as
scientists may have believed this condition to have been true, and it may have been the
opinion of the translators of some manuscripts; but it is an unwarranted fallacy to proclaim
that the ancient Hebrews believed that the earth was covered by "a solid, concave
firmament," or dome. No such statement is found in the ancient Hebrew, and certainly such
thoughts were not entertained by the prophets of old. The word "firmament" has appeared in
various printed editions of the Bible for so many years that we have come to look upon this
word with no misunderstanding of the fact which it represents. Even our dictionaries
recognize the word "firmament" as referring to "the expanse of heaven." Not one word in the
Bible can be pointed out as declaring that the heaven above the earth is a solid dome. One
passage called in question by these over-zealous critics is as follows:

"And God said, Let there be a firmament in the midst of the waters, and let it divide the
waters from the waters.

"And God made the firmament, and divided the waters which were under the firmament from
the waters which were above the firmament: and it was so." (Gen. 1:6-7.)

Then in verse 20: "And God said, Let the waters bring forth abundantly the moving creature
that hath life, and fowl that may fly above the earth in the \textit{open firmament} of heaven." How
could they fly in a solid dome? Dr. Adam Clark in his \textit{Commentary} says of this translation:

And God said, Let there be a firmament] Our translators, by following the firmamentum of
the Vulgate, which is a translation of the Septuagint, have deprived this passage of all sense
of meaning. The Hebrew word \textit{rakeed} from Raka, to \textit{spread out as the curtains of a tent or
pavillion}, simply signifies an expanse of space, and consequently, the circumambient space
or expansion, separating the clouds which are in the higher regions of it, from the seas, etc.,
which are below it. This we call the \textit{atmosphere}, the orb of atoms, or inconclusively small
particles; but the word appears to have been used by Moses in a more extensive sense, and to
include the whole of the planetary vortex, or the space which is occupied by the whole solar
system.

Dr. D. E. Hart-Davies, in an article published in the \textit{Journal of Transactions} of The Victoria
Institute, discussing the word firmament has this to say:

But, as a matter of fact, the idea expressed by the English word "firmament" from the Latin
\textit{firmamentum}, which does denote something strong and solid, is not found in the original
Hebrew. The word there is (\textit{raquia}), which means that which is stretched out, attenuated, or
extended. The verbal form of the root was used to describe the beating-out of gold into thin
wires or threads fine enough to be sewn into the priestly garment. The extremely thin gold-
leaf which remains after the goldsmith has finished his task represents the \textit{raquia} of the piece
of pure metal with which he began. The noun, therefore, denotes extension. Hence the R. V.
rendering is "expanse," which is correct. The Hebrew is a strictly accurate term. The word
"firmament" is a mistranslation due to the false astronomy of Alexandria in the third century
B.C. The Greeks believed that the sky was a solid crystalline sphere. Hence the \textit{raquia} of the
Hebrew was rendered in the Greek Septuagint version by the word \textit{stereoma}, which was
again translated in the Latin Vulgate by \textit{Firmamentum}, from which the A. V. word
"Firmament" was derived. 8

3. \textit{And God came down}. Another saying in the scriptures that meets with the hilarious
merriment of our critics is the saying in various parts of the scriptures, that "God came
down." He came down "in the cool of the day" to rebuke Adam. He came down to see the
building of Babel. He came down to destroy Sodom and Gomorrah, and the Lord said, "I will
go down now, and see whether they have done altogether according to the cry of it, which is
come unto me; and if not, I will know." He came down to talk to Moses and give him
commandments, and spoke to the ancient prophets. Of such things Dr. Andrew D. White has
this to say:

Myths having this geographical idea as their germ developed in luxuriance through thousands
of years. Ascensions to heaven and descents from it, "translations," "assumptions,"
"annunciations," mortals "caught up" into it and returning, angels flying between it and the
earth, thunderbolts hurled down from it, mighty winds issuing from its corners, voices
speaking from the upper floor to men on the lower, temporary openings of the floor of
heaven to reveal the blessedness of the good, "signs and wonders" hung out from it to warn
the wicked, interventions of every kind—from the heathen gods coming down on every sort
of errand, and Jehovah coming down to walk in Eden in the cool of the day, to St. Mark
swooping down into the market-place of Venice to break the shackles of a slave—all these
are but features in a vast evolution of myths arising largely from the geographical germ. 9

To these men who laugh at the thought of an anthropomorphic God, one who can descend
and ascend after communing with his prophets, such a thing is extremely absurd. Such
thoughts as God coming down, angels descending from heaven and returning again, holy
men having visions such as was given to Stephen, is all too much for their scientific minds to
understand. How can a god who is ethereal and who fills the immensity of space, or a god
who is merely a thought dwelling in the human mind, come down from heaven and return
again? Of course their god could do no such thing; but the God of Israel could and did. He
came down and met Moses on the top of Sinai and gave him the commandments for Israel.
He came down and rebuked Aaron and Miriam. It is written: "And the Lord came down in
the pillar of the cloud, and stood in the door of the tabernacle, and called Aaron and Miriam:
and they both came forth. And he said, Hear now my words: If there be a prophet among
you, I the Lord will make myself known unto him in a vision, and will speak unto him in a
dream. My servant Moses is not so, who is faithful in all mine house. With him will I speak
mouth to mouth, even apparently, and not in dark speeches; and the similitude of the Lord
shall be behold: wherefore then were ye not afraid to speak against my servant Moses? And
the anger of the Lord was kindled against them; and he departed." 10

The Lord came down and dwelt among men for some 33 years, and was then taken and
crucified by wicked men and thus redeemed all mankind from death and gave them the
blessing of the resurrection. Both the Father and the Son came down to a boy in the year
1820, and gave him commandments and made it known to him that all the ideas prevailing
about God, and which, unfortunately persist that he is an ethereal, intangible, immaterial
force which fills the immensity of space, is a false conception without any warrant in
scripture and utterly beyond the realm of reason. Moreover, this same God sent from time to
time from his presence angels who also came down to his servants the prophets and
conversed with them. The evidence of this fact has been given to us repeatedly in this
dispensation of the Fulness of Times. Moroni, a resurrected being, appeared to the Prophet
Joseph Smith, to Oliver Cowdery, David Whitmer and Martin Harris and others, Peter, James
and John, the apostles of Jesus Christ, came down and gave to Joseph Smith and Oliver
Cowdery the holy Melchizedek Priesthood, and John the Baptist came down and restored his
authority and thus was established among men that which had been taken away because of
corruption, the plan of salvation, even the fulness of the Gospel with all of its powers and
authority. Men may laugh about God coming down, angels descending and men "being
caught up" to heaven, but their scoffing and smug superior intelligence does not change the
fact.

4. \textit{Jesus ignorant of the Fall}. "A belief, then, in a primeval period of innocence and
perfection—moral, intellectual, and physical—from which men for some fault fell, is
perfectly in accordance with what we should expect. Among the earliest known records of
our race we find this view taking shape in the Chaldean legends of war between the gods, and
of a fall of man; both of which seemed necessary to explain the existence of evil. . . .

"This view, growing out of the myths, legends, and theologies of earlier peoples, we also find
embodied in the sacred tradition of the Jews, and especially in one of the documents which
form the impressive poem beginning the books attributed to Moses. As to the Christian
Church, no word of its Blessed Founder indicates that it was committed by him to this theory,
or that he even thought it worthy of his attention. How like so many other dogmas never
dreamed of by Jesus of Nazareth and those who knew him best, it was developed, it does not
lie within the province of this chapter to point out; nor is it worth our while to dwell upon its
evolution in the early church, in the Middle Ages, at the Reformation, and in various
branches of the Protestant Church; suffice it that, though among English-speaking nations by
far the most important influence in its favor has come from Milton's inspiration rather than
from that of older sacred books, no doctrine has been more universally accepted, 'always
everywhere, and by all,' from the earliest fathers of the church down to the present hour.

"On the other hand appeared at an early period the opposite view — that mankind, instead of
having fallen from a high intellectual, moral, and religious condition, has slowly risen from
low and brutal beginnings." 11

Of course, all of this is untrue. This writer, like the apostles of old before the Lord inspired
them following his resurrection, is ignorant of the mission of Jesus Christ in this world.
When he says that our Lord was ignorant of Adam's fall, and that this was a myth, a legend,
that crept into the theologies of the Jews and early Christians, he speaks without authority
and of that which he did not know. He was blinded by the foolish doctrines of the modern
world in relation to the beginning of things. Our Savior understood fully and perfectly why
he came into this world. He tried to impress this fact upon the minds of his disciples, but they
failed in the beginning to understand. It is sufficient to refute this ignorant attack by a
reference to a few of the expressions of our Lord.

And he came to Nazareth, where he had been brought up; and, as his custom was, he went
into the synagogue on the sabbath day, and stood up for to read.

And there was delivered unto him the book of the prophet Esaias. And when he had opened
the book, he found the place where it was written,

The Spirit of the Lord is upon me, because he hath anointed me to preach the gospel to the
poor; he hath sent me to heal the brokenhearted, to preach deliverance to the captives, and
recovering of sight to the blind, to set at liberty them that are bruised,

To preach the acceptable year of the Lord. (Luke 4:16-19.)

And as Moses lifted up the serpent in the wilderness, even so must the Son of man be lifted
up:

That whosoever believeth in him should not perish, but have eternal life.

For God so loved the world, that he gave his only begotten Son, that whosoever believeth in
him should not perish, but have everlasting life.

For God sent not his Son into the world to condemn the world; but that the world through
him might be saved.

He that believeth on him is not condemned: but he that believeth not is condemned already,
because he hath not believed in the name of the only begotten Son of God. (John 3:14-18.)

Verily, verily, I say unto you, He that heareth my word, and believeth on him that sent me,
hath everlasting life, and shall not come into condemnation; but is passed from death unto
life.

Verily, verily, I say unto you, The hour is coming, and now is, when the dead shall hear the
voice of the Son of God: and they that hear shall live.

For as the Father hath life in himself; so hath he given to the Son to have life in himself;

And hath given him authority to execute judgment also, because he is the Son of man. (John
5:25-27.)

Jesus said unto her, I am the resurrection, and the life: he that believeth in me, though he
were dead, yet shall he live:

And whosoever liveth and believeth in me shall never die. Believest thou this? (John 11:25-
26.)

And I, if I be lifted up from the earth, will draw all men unto me.

This he said, signifying what death he should die.

The people answered him, We have heard out of the law that Christ abideth for ever: and
how sayest thou, The Son of man must be lifted up? who is this Son of man? (John 12:32-
34.)

These are a few sayings of Jesus as they are recorded in the New Testament. Each of these
and many more of like import, testify to the fact that Jesus knew of Adam's fall, and that his
mission was to repair the broken law and restore mankind through his death on the cross
from death unto life everlasting. It is in the darkness of unbelief that these scholars fail to see
and understand the truth that Jesus was fully aware of his mission and spoke frequently of it
to his disciples and pointed out to them the fact, as on occasions he did to the Jews, that he
was to die on the cross in order to "draw all men unto him," or, in other words, give unto
them through his sacrifice the resurrection and power over the physical, or mortal death.
There is one other passage from his lips uttered in this dispensation that should be presented
here:

For behold, I, God, have suffered these things for all, that they might not suffer if they would
repent;

But if they would not repent they must suffer even as I;

Which suffering caused myself, even God, the greatest of all, to tremble because of pain, and
to bleed at every pore, and to suffer both body and spirit—and would that I might not drink
the bitter cup, and shrink—

Nevertheless, glory be to the Father, and I partook and finished my preparations unto the
children, of men. (D. \& C. 19:16-19.)

5. \textit{The Gospel preached in all the world}. Another doctrine that is ridiculed by the critics is
the saying as found in the 19th Psalm, verses 3 and 4, and Romans 9:17 and 10:18. These
passages declare that the word of the Lord had been proclaimed throughout all the earth. I
quote from Paul, Romans 10:18:

But I say, Have they not heard? Yes verily, their sound went into all the earth, and their
words unto the ends of the world.

Our critics strike at the scriptures by referring to the doctrines taught by some of the "early
fathers," principally St. Augustine. Here, again, they blame the Church for the interpretations
of Paul's sayings and those in the Psalms, by St. Augustine and priests of the Catholic
Church, the inference being that these men were the rightful interpreters of the holy
scriptures, for the Church of Jesus Christ. Since the general view in the day of the "fathers,"
from about the third century to the days of the Protestant revolution, was that the whole
world was that portion of Europe, Asia and Africa known to them; then there could not be
peoples living in other far distant lands unknown to them. Therefore, these wise men ridicule
the doctrine, so universally taught at that time, which came in conflict with knowledge
revealed by the explorers beginning with Columbus. So, Dr. White and Dr. Draper, as leaders
in this controversy, chuckle at the discomfiture of the Catholic Church when its leaders were
forced to reconstruct their thinking and their interpretations of the scriptures, and they, the
critics, have to switch their erroneous doctrines from these uninspired ecclesiastics, to the
inspired writers of the holy scriptures.

It is reasonable to believe that the writer of the 19th Psalm, St. Paul and St. Peter, knew what
they were talking about by the inspiration of the Lord. It is also a fact that what they wrote is
verily true.

Dr. White writes: "In summing up the action of the Church upon geography, we must say,
then, that the dogmas developed in strict adherence to scripture and the conceptions held in
the Church, during many centuries "always, everywhere, and by all," were, on the whole,
steadily hostile to truth; but it is only just to make a distinction here between the religious
and the theological spirit." 12 Then in defense of the religious spirit, he commends
Columbus, Prince John of Portugal and many of the explorers who were religiously devout;
but the theological leaders, even those who wrote the scriptures, were universally ignorant of
the truth. Again attention is called to the fact that these men who had partaken of the spirit of
apostasy from the teachings of Jesus Christ and the prophets, cannot be referred to as
authorized servants of the Lord to proclaim his doctrine. The statement that they were only
teaching with "strict adherence with the scriptures," we justly challenge. The prophets spoke
the truth. This Gospel had been proclaimed in all the earth. It was the Lord who scattered the
people in the beginning. We have the evidence of this in the Book of Mormon. It was at the
building of the tower when the Lord decreed the scattering of the people "abroad upon the
face of all the earth." (Gen. 11:9.) One of these colonies was led by the Lord to the western
hemisphere and was known as Jaredites. From the writings of the prophets, we learn the Lord
fulfilled his promise and scattered the people as he said he would to all parts of the earth.
Other colonies were also sent forth by divine direction. It should be remembered also that it
was not until several hundred years after the flood when the earth was divided. Before that
time, and while men were being divinely scattered, all the land surface of the earth \textit{was in
one place}. (Gen. 1:9.)

Jacob, brother of Nephi, in his parable of the Olive Tree, has shown how the Lord scattered
the people to all parts of the earth. Moreover, the Lord gave to many of these people, if not
all, prophets. (Alma 29:8 and 3 Nephi ch. 16). It is not hard therefore for a member of the
Church to understand that the Lord kept this promise made at Babel.

In a discourse preached unto the Jews but a short time before his final meeting with the
apostles, Jesus called attention to the nature of his mission. He declared that he was the door
of the sheep, and all who came before him claiming Messiahship were "thieves and robbers:
but the sheep did not hear them." Continuing he said:

I am the door: by me if any man enter in, he shall be saved, and shall go in and out, and find
pasture.

The thief cometh not, but for to steal, and to kill, and to destroy: I am come that they might
have life, and that they might have it more abundantly.

I am the good shepherd: the good shepherd giveth his life for the sheep.

But he that is an hireling, and not the shepherd, whose own the sheep are not, seeth the wolf
coming, and leaveth the sheep, and fleeth: and the wolf catcheth them, and scattereth the
sheep.

The hireling fleeth, because he is an hireling, and careth not for the sheep.

I am the good shepherd, and know my sheep, and am known of mine.

As the Father knoweth me, even so know I the Father: and I lay down my life for the sheep.

\textit{And other sheep I have, which are not of this fold: them also I must bring, and they shall
hear my voice; and there shall be one fold, and one shepherd. (John 10:9-16.)}

When Christ visited the Nephites on the American continent he said:

And verily, verily, I say unto you that I have other sheep, which are not of this land, neither
of the land of Jerusalem, neither in any parts of that land round about whither I have been to
minister.

For they of whom I speak are they who have not as yet heard my voice; neither have I at any
time manifested myself unto them.

But I have received a commandment of the Father that I shall go unto them, and that they
shall hear my voice, and shall be numbered among my sheep, that there may be one fold and
one shepherd; therefore I go to show myself unto them." (3 Nephi 16:1-3.)

He also declared that it was of the Nephites and other peoples that had been scattered of
whom he spoke as recorded in John 10:16. He revealed to the Nephites the fulness of the
Gospel, and informed them that to other peoples he would go that they also should hear his
words. So Christ himself carried the message to all the world and this is in harmony with
what Paul declared.

6. \textit{The miracles of the scriptures}. In the attacks made upon the Bible many of the darts of
criticism have been hurled at the miracles related in the scriptures. Whether it has been the
healing of the sick, causing the lame to walk, the filling of the widow's vessels with oil, the
calling down fire from heaven, or whatever it may be, there has been a full measure of
unbelief, argument and ridicule to the contrary. When the apostles came to the Lord and said
unto him, "Increase our faith," he answered them, "If ye had faith as a grain of mustard seed,
ye might say unto this sycamine tree, Be thou plucked up by the root, and be thou planted in
the sea; and it should obey you." The scoffers say, it would not obey you! Only on natural
and known principles in accordance with law can anything be accomplished. Disease may be
cured by the aid of medical skill and science, but not by faith and prayers. So they speak and
write. They say these wonderful stories recorded in the Bible of mighty deeds, where the
dead were raised by faith, where fire came down through prayer to consume Elijah's
sacrifice, and to destroy cities, where waters were turned out of their course, mouths of lions
closed, men thrown into a fiery furnace coming out unharmed, are all part and parcel of the
myths which crept into the record through the aid of fertile brains.

Andrew D. White has written: "Legends of miracles have thus grown about the lives of all
great benefactors of humanity in early ages, and about saints and devotees. Throughout
human history the lives of such personages, almost without exception, have been
accompanied or followed by a literature in which legends of miraculous powers form a very
important part—a part constantly increasing until a different mode of looking at nature and of
weighing testimony causes miracles to disappear. While modern thought holds the testimony
to the vast mass of such legends in all ages as worthless, it is very widely acknowledged that
great and gifted beings who endow the earth with higher religious ideas, gaining the deepest
hold upon the hearts and minds of multitudes, may at times exercise such influence upon
those about them that the sick in mind or body are helped or healed." 13

This is an age when faith and the power of God should be greatly increased, but to the
contrary it is diminished and men boast in their own strength; yet we see every day of our
lives, the greatest of miracles. The flying of the airplane, the voice on the radio, the picture
on the screen and television. There are thousands of miracles performed today, wonders that
would astound our grandfathers could they suddenly see them. These miracles are as great as
turning water into wine, raising the dead or anything else. A miracle is not, as many believe,
the setting aside or overruling natural laws. Every miracle performed in Biblical days or now,
is done on natural principles and in obedience to natural law. The healing of the sick, the
raising of the dead, giving eyesight to the blind, whatever it may be that is done by the power
of God, is in accordance with natural law. Because we do not understand how it is done, does
not argue for the impossibility of it. Our Father in heaven knows many laws that are hidden
from us. Man today has learned of many laws that our grandfathers did not understand. It is
small business for the critics to condemn the miracles in scriptures as though all the laws of
God have been revealed, and there could be no powers which they do not understand. Moroni
has spoken by inspiration in relation to these things:

And again I speak unto you who deny the revelations of God, and say that they are done
away, that there are no revelations, nor prophecies, nor gifts, nor healing, nor speaking with
tongues, and the interpretation of tongues;

Behold I say unto you, he that denieth these things knoweth not the gospel of Christ; yea, he
has not read the scriptures; if so, he does not understand them.

For do we not read that God is the same yesterday, today, and forever, and in him there is no
variableness neither shadow of changing?

And now, if ye have imagined up unto yourselves a god who doth vary, and in whom there is
shadow of changing, then have ye imagined up unto yourselves a god who is not a God of
miracles.

But behold, I will show unto you a God of miracles, even the God of Abraham, and the God
of Isaac, and the God of Jacob; and it is that same God who created the heavens and the
earth, and all things that in them are.

Behold, he created Adam, and by Adam came the fall of man. And because of the fall of man
came Jesus Christ, even the Father and the Son; and because of Jesus Christ came the
redemption of man. . . .

And now, O all ye that have imagined up unto yourselves a god who can do no miracles, I
would ask of you, have all these things passed, of which I have spoken? Has the end come
yet? Behold I say unto you, Nay; and God has not ceased to be a God of miracles.

Behold, are not the things that God hath wrought marvelous in our eyes? Yea, and who can
comprehend the marvelous works of God?

Who shall say that it was not a miracle that by his word the heaven and the earth should be;
and by the power of his word man was created of the dust of the earth; and by the power of
his word have miracles been wrought.

And who shall say that Jesus Christ did not do many mighty miracles? And there were many
mighty miracles wrought by the hands of the apostles.

And if there were miracles wrought then, why has God ceased to be a God of miracles and
yet be an unchangeable Being? And behold, I say unto you he changeth not; if so he would
cease to be God; and he ceaseth not to be God, and is a God of miracles.

And the reason why he ceaseth to do miracles among the children of men is because that they
dwindle in unbelief, and depart from the right way, and know not the God in whom they
should trust.

Behold, I say unto you that whoso believeth in Christ, doubting nothing, whatsoever he shall
ask the Father in the name of Christ it shall be granted him; and this promise is unto all, even
unto the ends of the earth. (Mormon 9:7-12; 15-21.)

7. \textit{Evil spirits and spirit-possession}. Dr. Howard W. Haggard has written: "The early and
mediaeval Christians accepted the doctrine of the power of demons in the lives of men; they
saw this power particularly in the demoniac production of disease. They believe in miracles
and especially in the miraculous healing of disease. The demonological belief of the
Christians was inherited from the doctrine of the Jews, who were firm believers in demons
and the 'possession by devils.' Thus the logical cure of disease consisted in the exorcism of
devils." 14

Andrew D. White in his fight against the belief in evil spirits and possession made this
remark: "Nothing is more simple and natural, in the early stages of civilization, than belief in
occult, self-conscious powers of evil. Troubles and calamities come upon man; his ignorance
of physical laws forbids him to attribute them to physical causes; he therefore attributes them
sometimes to the wrath of a good being, but more frequently to the malice of an evil being.

"Especially is this the case with diseases. The real causes of disease are so intricate that they
are reached only after ages of scientific labor; hence they, above all, have been attributed to
the influence of evil spirits." 15

We are aware of the fact that among many peoples and especially in the dark ages there
existed strange doctrines regarding devils, witches and belief in magic. These doctrines
continued even in the United States in its infancy and innocent persons were accused and
punished as witches, but this should not cause these scientific men to class all cases of illness
and derangement to the physical conditions of the body. We all realize that there are diseases
of the mind as well as of the organs and other parts of the human body, and it may be that at
times mankind have ascribed many if not all of these to the possession or influence of evil
spirits. The fact remains however, that the cases of spirit-possession as recorded in the New
Testament, are true. It is also true that under some conditions Satan has bound the bodies of
individuals by his power. This is not only true of such conditions in the days of our Savior
and his apostles, but we have the evidence of such being true in this dispensation in which we
now live. We must not discount the power of the adversary of all righteousness. There are
scores of cases, fully attested in our own day of demon influence and possession. Cases
which were not caused by derangement of the mind, but by actual overpowering of the
individual and taking possession of his body. One of the most outstanding was the case
where Satan and his cohorts endeavored to destroy the work of the Lord when it was opened
in Great Britain. The story is recorded in the \textit{Life of Heber C. Kimball}, and occurred in the
presence of at least four individuals. The reader is referred to Chapter 13, pages 300 to 303,
wherein this incident is related.

In the Book of Moses (Pearl of Great Price) we find recorded the encounters that Moses had
with Lucifer, and it is foolish for any man to argue that the stories told in the New Testament
of the casting out of devils by our Lord and his apostles are not true. These same sort of
things have occurred scores of times in the present day.

And when he had called unto him his twelve disciples, he gave them power against unclean
spirits, to cast them out, and to heal all manner of sickness and all manner of disease. (Matt.
10:1.)

And as you go, preach, saying, The kingdom of heaven is at hand.

Heal the sick, cleanse the lepers, raise the dead, cast out devils: freely ye have received,
freely give. (Matt. 10:7-8.)

Then he called his twelve disciples together, and gave them power and authority over all
devils, and to cure diseases. (Luke 9:1.)

After these things the Lord appointed other seventy also, and sent them two and two before
his face into every city and place, whither he himself would come. (Luke 10:1.)

And the seventy returned again with joy, saying, Lord, even the devils are subject unto us
through thy name. (Luke 10:17.)

\newpage
REFERENCES—CHAPTER TWENTY-FIVE

Footnotes

1. Acts 20:29-30; 1 Tim. 4:1-3.

2. Draper, Dr. J. W., \textit{Conflict Between Religion and Science}, pp. 62-63.

3. White, Dr. A. D., \textit{History of the Warfare of Science with Theology}, Vol. 1, p. 218.

4. \textit{Egyptian Alphabet}, Historian's Office.

5. Helaman 12:12-15.

6. White, Dr. A. D., \textit{History of Warfare of Science with Theology}, Vol. 1, p. 90.

7. \textit{Ibid.}, p. 324.

8. \textit{Journal of the Transactions}—Victoria Institute, Vol. 70, p. 57.

9. White, Dr. A. D., \textit{History of the Warfare of Science with Theology}, Vol. 1, p. 96.

10. Numbers 12:5-9.

11. White, Dr. A. D., \textit{History of the Warfare of Science with Theology}, Vol. 1, pp. 285-286.

12. \textit{Ibid.}, Vol. 1, p. 113.

13. White, Dr. A. D., \textit{History of the Warfare of Science with Theology}, Vol. 2, p. 5.

14. Haggard, Dr. Howard W., \textit{Devils, Drugs and Doctors}, p. 297.

15. White, Dr. A. D., \textit{History of the Warfare of Science with Theology}, Vol. 2, p. 97.

