\chapter{ADAM'S PLACE IN THE EARTH'S DESTINY—2}

IN the Doctrine and Covenants, Section 107, we have a wonderful revelation on Priesthood.
This revelation was given at the request of the recently ordained apostles. They were about to
go forth on missions assigned to them and they desired some guidance by revelation, so the
Lord granted this request and instructed them in matters pertaining to the Priesthood. In the
course of this divine communication the Lord gave definite instruction in relation to Adam
and the patriarchs living before the flood. In it we are informed that the Priesthood was first
given to Adam and he in the course of years, ordained his son Seth, Enos, Cainan,
Mahalaleel, Jared, Enoch and Methuselah. Be it remembered that Adam, after the fall, lived
for 930 years. Seth ordained Lamech and Noah was ordained by Methuselah. This is the
descent of Priesthood from the beginning to the time of the flood.

Three years before the death of Adam, or in the year 927 from the date of the fall, this great
Patriarch and father of the human family called Seth, Enos, Cainan, Mahaleel, Jared, Enoch
and Methuselah, who were all high priests, with the residue of his faithful descendants, now
numerous, into the valley of Adam-ondi-Ahman, the place where Adam dwelt, and there
bestowed upon them his last blessing.

And the Lord appeared unto them, and they rose up and blessed Adam, and called him
Michael, the prince, the archangel.

And the Lord administered comfort unto Adam, and said unto him: I have set thee to be at
the head; a multitude of nations shall come of thee, and thou art a prince over them forever.

And Adam stood up in the midst of the congregation; and, notwithstanding he was bowed
down with age, being full of the Holy Ghost, predicted whatsoever should befall his posterity
unto the latest generation.

These things were all written in the book of Enoch, and are to be testified of in due time. 1

Again in Section 84, we have another reference to the descent of the Priesthood, this time
tracing it back from Moses, Abraham, Melchizedek, Noah, Enoch, Able to Adam, and in
verses 16 and 17 we read:

And from Enoch to Abel, who was slain by the conspiracy of his brother, who received the
priesthood by the commandments of God, by the hand of his father Adam, \textit{who was the first
man—}

Which priesthood continueth in the church of God in all generations, and is without
beginning of days or end of years.

We are here informed again that Adam was the first man, and that the Priesthood held by the
prophets of old came down through his lineage from him to the days of Moses, and from
Moses it continued on to the days of the coming of our Lord and Savior Jesus Christ. In the
Doctrine and Covenants, Section 117, we are informed that Adam dwelt in Adam-ondi-Ahman and the plains of Olaha Shinehah, which places have been made known to have been
on the western hemisphere, and in what is now known as the State of Missouri. This is well
for us to remember since it is the general impression that civilization commenced on the
eastern hemisphere. Then, in a revelation given to President Brigham Young at Winter
Quarters, January 14, 1847, the Lord said:

For they killed the prophets, and them that were sent unto them; and they have shed innocent
blood, which crieth from the ground against them.

Therefore, marvel not at these things, for ye are not yet pure; ye cannot yet bear my glory;
but ye shall behold it if ye are faithful in keeping all my words that I have given you, from
the days of Adam to Abraham, from Abraham to Moses, from Moses to Jesus and his
apostles, and from Jesus and his apostles to Joseph Smith, whom I did call upon by mine
angels, my ministering servants, and by mine own voice out of the heavens, to bring forth my
work. 2

I have now given testimony from the Pearl of Great Price, the Book of Mormon and the
Doctrine and Covenants, all confirming what is written in the Bible in regard to Adam, his
creation, his fall, his Priesthood, and indicating to us a very definite time when he lived; of
his ministry and when he died in full fellowship with the Lord, honored by his righteous
posterity. One other passage completes this list, that from Section 78, verses 15-16:

That you may come up unto the crown prepared for you, and be made rulers over many
kingdoms, saith the Lord God, the Holy One of Zion, who hath established the foundations of
Adam-ondi-Ahman; Who hath appointed Michael your prince, and established his feet, and
set him upon high, and given unto him the keys of salvation under the counsel and direction
of the Holy One, who is without beginning of days or end of life.

In this scripture we receive the knowledge that Michael, who is Adam, not only stands at the
head ruling over his posterity as a prince unto them forever; but also he has been appointed to
hold the keys of salvation for the benefit of the people of all the world who will truly repent
and receive the Gospel, and this great honor is bestowed upon him to act under the direction
of Jesus Christ who is the Only Begotten Son of God and the Holy One of Israel. Therefore,
all ye mockers and you mighty men take heed unto yourselves, for the day will come when
you will have to face this first man who has been crowned with honor to stand next to Jesus
Christ holding the keys of salvation; and you will not pass through the gates into that
kingdom without his consent! When that day comes and if you are permitted to pass through
those gates it will be only because you sorely repented and have humbly apologized to this
great man whom you have so mercilessly maligned.

The following teachings in regard to Adam are taken from the discourses and writings of the
Prophet Joseph Smith, who was better prepared to speak than any other man since the days of
Jesus Christ, on the life and authority of Adam. His knowledge was not obtained from books
or the writings of the worldly wise of the present day. What he learned and revealed was
given him by divine revelation.

The Prophet Joseph Smith had this to say about Adam and the Priesthood:

The Priesthood was first given to Adam; he obtained the First Presidency, and held the keys
of it from generation to generation. He obtained it in the Creation, before the world was
formed, as in Genesis 1:26, 27, 28. He had dominion given him over every living creature.
He is Michael the Archangel, spoken of in the scriptures. Then to Noah, who is Gabriel; he
stands next in authority to Adam in the Priesthood; he was called of God to this office, and
was the father of all living in his day, and to him was given the dominion. These men held
keys first on earth, and then in heaven.

The Priesthood is an everlasting principle, and existed with God from eternity, and will to
eternity, without beginning of days or end of years. The keys have to be brought from heaven
whenever the Gospel is sent. When they are revealed from heaven, it is by Adam's authority.

Daniel in his seventh chapter speaks of the Ancient of Days; he means the oldest man, our
Father Adam, Michael, he will call his children together and hold a council with them to
prepare them for the coming of the Son of Man. He (Adam) is the father of the human
family, and presides over the spirits of all men, and all that have had the keys must stand
before him in this grand council. This may take place before some of us leave this stage of
action. The Son of Man stands before him, and here is given him glory and dominion. Adam
delivers up his stewardship to Christ, that which was delivered to him as holding the keys of
the universe, but retains his standing as head of the human family.

The spirit of man is not a created being; it existed from eternity, and will exist to eternity.
Anything created cannot be eternal; and earth, water, etc., had their existence in an
elementary state, from eternity. Our Savior speaks of children and says, Their angels always
stand before my Father. The Father called all spirits before him at the creation of man, and
organized them. He (Adam) is the head and was told to multiply. The keys were first given to
him, and by him to others. He will have to give an account of his stewardship, and they to
him.

The Priesthood is everlasting. The Savior, Moses, and Elias, gave the keys to Peter, James
and John, on the mount, when they were transfigured before him. The Priesthood is
everlasting—without beginning of days or end of years; without father, mother, etc. If there
is no change of ordinances, there is no change of Priesthood. Wherever the ordinances of the
Gospel are administered, there is the Priesthood.

DESCENT OF PRIESTHOOD

How have we come at the Priesthood in the last days? It came down, in regular succession.
Peter, James and John, had it given to them and they gave it to others. Christ is the Great
High Priest; Adam next. Paul speaks of the Church coming to an innumerable company of
angels . . . to God the Judge of all—the spirits of just men made perfect; to Jesus the mediator
of the new covenant. (Hebrews 12:23.)

I saw Adam in the valley of Adam-ondi-Ahman. He called together his children and blessed
them with a patriarchal blessing. The Lord appeared in their midst, and he (Adam) blessed
them all, and foretold what would befall them to the latest generation.

This is why Adam blessed his posterity; he wanted to bring them into the presence of God.
They looked for a city, etc., "whose builder and maker is God." (Hebrews 11:10.) Mosessought to bring the children of Israel into the presence of God, through the power of the
Priesthood, but he could not. \textit{In the first ages of the world} they tried to establish the same
thing; and there were Eliases raised up who tried to restore these very glories, but did not
obtain them; but they prophesied of a day when this glory would be revealed. Paul spoke of
the dispensation of the fulness of times, when God would gather together all things in one,
etc.; and those men to whom these keys have been given, will have to be there; and they
without us cannot be made perfect.

These men are in heaven, but their children are on the earth. Their bowels yearn over us. God
sends down men for this reason. "The Son of man shall send forth his angels, and they shall
gather out of his kingdom all things that offend, and them which do iniquity." (Matthew
13:41.) All these authoritative characters will come down and join hand in hand in bringing
about this work. 3

At the October conference in Nauvoo in 1840, the Prophet gave further instruction regarding
the Priesthood and in the course of his remarks had the following to say about Adam:

Commencing with Adam, who was the first man, who is spoken of in Daniel as being the
"Ancient of Days," or in other words, the first and oldest of all, the great, grand progenitor of
whom it is said in another place he is Michael, because he was the first and father of all, not
only by progeny, but the first to hold the spiritual blessings, to whom was made known the
plan of ordinances for the salvation of his posterity unto the end, and to whom Christ was
first revealed, and through whom Christ has been revealed from heaven, and will continue to
be revealed from henceforth. Adam holds the keys of the dispensation of the fulness of times;
i.e., the dispensation of all the times have been and will be revealed through him from the
beginning to Christ, and from Christ to the end of the dispensations that are to be revealed.
"Having made known unto us the mystery of his will, according to his good pleasure which
he hath purposed in himself: That in the dispensation of the fulness of times he might gather
together in one all things in Christ, both which are in heaven, and which are on earth; even in
him." (Eph. 1:9-10.)

Now the purpose in himself in the winding up scene of the last dispensation is that all things
pertaining to that dispensation should be conducted precisely in accordance with the
preceding dispensations.

And again, God purposed in himself that there should not be an eternal fulness until every
dispensation should be fulfilled and gathered together in one, and that all things whatever,
that should be gathered together in one in those dispensations unto the same fulness and
eternal glory, should be in Christ Jesus; therefore he set the ordinances to be the same forever
and ever, and set Adam to watch over them, to reveal them from heaven to man, or to send
angels to reveal them. "Are they not all ministering spirits, sent forth to minister for them
who shall be heirs of salvation?" (Hebrews 1:14.)

These angels are under the direction of Michael or Adam, who acts under the direction of the
Lord. From the above quotation we learn that Paul perfectly understood the purposes of God
in relation to his connection with man, and that glorious and perfect order which he
established in himself, whereby he sent forth power, revelations, and glory.

ADAM RECEIVED COMMANDMENTS FROM GOD

God will not acknowledge that which he has not called, ordained, and chosen. In the
beginning God called Adam by his own voice. "And the Lord God called unto Adam, and
said unto him, Where art thou? And he said, I heard thy voice in the garden, and I was afraid
because I was naked; and hid myself." (Genesis 3:9-10.) Adam received commandments and
instructions from God: this was the order from the beginning.

That he received revelations, commandments and ordinances at the beginning is beyond the
power of controversy; else how did they begin to offer sacrifices to God in an acceptable
manner. And if they offered sacrifices they must be authorized by ordination. We read in
Genesis 4:4, that Abel brought of the firstlings of the flock and the fat thereof, and the Lord
had respect to Abel and to his offering. And, again, "By faith Abel offered unto God a more
excellent sacrifice than Cain, by which he obtained witness that he was righteous, God
testifying of his gifts: and by it he being dead yet speaketh." (Hebrews 11:4.) How doth he
yet speak? Why he magnified the Priesthood which was conferred upon him, and died a
righteous man, and therefore has become an angel of God by receiving his body from the
dead, holding still the keys of his dispensation; and was sent down from heaven unto Paul to
minister consoling words, and to commit unto him a knowledge of the mysteries of
godliness.

And if this was not the case, I would ask, how did Paul know so much about Abel, and why
should he talk about his speaking after he was dead? Hence, that he spoke after he was dead
must be by being sent down out of heaven to administer.

This, then, is the nature of the Priesthood; every man holding the Presidency of his
dispensation, and one man holding the Presidency of them all, even Adam; and Adam
receiving his Presidency and authority from the Lord, but cannot receive a fulness until
Christ shall present the kingdom to the Father, which shall be at the end of the last
dispensation. . . .

If Cain had fulfilled the law of righteousness as did Enoch, he could have walked with God
all the days of his life, and never failed of a blessing. "And Enoch walked with God after he
begat Methuselah three hundred years, and begat sons and daughters: And all the days of
Enoch were three hundred sixty and five years: And Enoch walked with God, and he was not,
for God took him." (Gen. 5:22-24.) Now this Enoch God reserved unto himself, that he
should not die at that time, and appointed unto him a ministry unto terrestrial bodies, of
whom there has been little revealed. He is reserved also unto the presidency of a
dispensation, and more shall be said of him and terrestrial bodies in another treatise. He is a
ministering angel, to minister to those who shall be heirs of salvation, and appeared unto
Jude as Abel did unto Paul; therefore Jude spoke of him (14-15 verses). And Enoch, the
seventh from Adam, revealed these sayings: "Behold, the Lord cometh with ten thousands of
his saints." 4

In the early part of the year 1912, Elder Samuel O. Bennion, then presiding in the Central
States Mission, wrote to the First Presidency for a statement answering the enemies of the
Church who were falsely quoting President Brigham Young. The letter of Presidency to
Elder Bennion is as follows:

Salt Lake City, Utah

February 20, 1912

Pres. Samuel O. Bennion,

Independence, Missouri.

Dear Brother:

Your question concerning Adam has not been answered because of pressure of important
business. We now respond briefly but, we hope, plainly. You speak of "the assertion made by
Brigham Young that Jesus was begotten of the Father in the flesh by our father Adam, and
that Adam is the father of Jesus Christ and not the Holy Ghost, and you say that elders are
challenged by certain critics to prove this.

If you will carefully examine the sermon to which you refer, in the \textit{Journal of Discourses,}
Vol. 1, you will discover that, while President Young denied that Jesus was "begotten by the
Holy Ghost," he did not affirm, in so many words, that "Adam is the father of Jesus Christ in
the flesh." He said, "Jesus, our elder brother, was begotten in the flesh by the same character
that was in the garden of Eden and who is our Father in Heaven." Here is what President
Young said about him, "Our Father in Heaven begat all the spirits that ever were or ever will
be upon this earth and they were born spirits in the eternal world. Then the Lord by his power
and wisdom organized the mortal tabernacle of man." Was he in the garden of Eden?

Surely; he gave commandment to Adam and Eve; he was their Father in Heaven; they
worshiped him and taught their children after the fall to worship and obey him in the name of
the Son who was to come.

But President Young went on to show that our father Adam—that is, our earthly father—the
progenitor of the race of man, stands at the head, being "Michael the Archangel, the Ancient
of Days," and that he was not fashioned from earth like an adobe, but begotten by his Father
in Heaven.

Adam is called in the Bible "the son of God." (Luke 3:38.) It was our Father in Heaven who
begat the spirit of him who was the Firstborn of all spirits that come to this earth and who
was also his Father by the Virgin Mary, making him "the Only Begotten in the flesh." Read
Luke 1:26-35. Where is Jesus called "the Only Begotten of the Holy Ghost?" He is always
singled out as "the Only Begotten of the Father." (John 14:3-16-18, etc.). The Holy Ghost
came upon Mary, her conception was under that influence, even of the spirit of life; our
Father in Heaven was the Father of the Son of Mary, to whom the Savior prayed, as did our
earthly father Adam.

When President Young asked, "Who is the father?" he was speaking of Adam as the father of
our earthly bodies, who is at our head as revealed in Doctrine and Covenants, Sec. 107,
verses 53-56. In that sense he is one of the Gods referred to in numerous scriptures, and
particularly by Christ. (John 10:34-36.) He is the great Patriarch, the Ancient of Days, who
will stand in his place as "a prince over us forever," and with whom we shall have to do," as
each family will have to do with its head, according to the Holy Patriarchal order. Our father
Adam, perfected and glorified as a God, will be a being who will carry out the behests of the
great Elohim in relation to his posterity.

While, as Paul puts it, There be gods many and lords many (whether in heaven or in earth),
unto us there is but one God the Father, of whom are all things, and one Lord Jesus Christ.
Latter-day Saints worship him and him alone, who is the Father of Jesus Christ, whom he
worshiped, whom Adam worshiped and who is God the Eternal Father of us all.

Your brethren,

Joseph F. Smith

Anthon H. Lund

Charles W. Penrose

First Presidency President Brigham Young has borne testimony concerning Adam and his
place in the world as Michael, the Archangel, who will stand at the head of his posterity
forever having jurisdiction over them under Jesus Christ. In one of his discourses he said:

We are safe in saying that from the day that Adam was created and placed in the garden of
Eden to this day, the plan of salvation and the revelations of the will of God to man are
unchanged, although mankind have not for many ages been favored there-with, in
consequence of apostasy and wickedness. There is no evidence to be found in the Bible that
the Gospel should be one thing in the days of the Israelites, and another in the days of Christ
and his apostles, and another in the 19th century, but, on the contrary, we are instructed that
God is the same in every age, and that his plan of saving his children is the same. The plan of
salvation is one, from the beginning of the world to the end thereof. 5

Adam was as conversant with his Father who placed him upon the earth as we are conversant
with our earthly parents. The Father frequently came to visit his son Adam, and walked with
him; and the children of Adam were more or less acquainted with him; and the things that
pertain to God and to heaven were as familiar among mankind in the first ages of their
existence on the earth, as these mountains are to our mountain boys, as gardens are to our
wives and children, or as the road to the Western Ocean is to the experienced traveler. From
this source mankind received their religious traditions. 6

The youth of Israel should remember that the Prophet Joseph Smith was in communication
with the heavens constantly. He was instructed by angels and by the Son of God himself. For
four years he was tutored by the Angel Moroni before he was privileged to obtain the plates
of the Book of Mormon and after that he was frequently visited by heavenly messengers. He
and Oliver Cowdery stood in the presence of John the Baptist and under his hands received
the Aaronic Priesthood, and later under the hands of Peter, James and John received the
Melchizedek Priesthood and were commanded to organize the Church. The ancient prophets
from Adam to Peter, James and John in the dispensation of the Meridian of Time, came and
manifested the keys of their dispensations to these two men. The Prophet Joseph Smith saw
Adam as well as these many other ancient prophets; he speaks by authority for he had the
knowledge. He knew that Adam lived and that he is the "first man," the "Ancient of Days,"
so called because he was the "oldest of all." I have presented the testimonies of the Nephite
prophets, the ancient prophets of the Israelites as the knowledge has come to us through
revelation and recorded in the Pearl of Great Price and the writings of Abraham, all
confirming the stories related in the Bible. We discover that organic evolution mocks at all of
this.

Now, my beloved brethren and sisters, and especially you younger members of the Church, is
it not better to hearken to these brethren who had personal knowledge than to accept the
insecure doctrines of those who reject their Redeemer and his servants and endeavor to put
them to open shame?

\newpage
REFERENCES—CHAPTER SIXTEEN

Footnotes

1. D. \& C. 107:41-57.

2. \textit{Ibid.}, 136:36-37.

3. \textit{Teachings of the Prophet Joseph Smith}, pp. 157-159.

4. \textit{Ibid.}, pp. 167-170.

5. \textit{Journal of Discourses}, Vol. 10, p. 324.

6. \textit{Ibid.}, Vol. 9, p. 148.

