\chapter{CONFLICT BETWEEN SCIENCE AND RELIGION—2}

DURING the past century and the first half of the present century there has arisen a multitude
of critics who have taken upon themselves the task of destroying, if possible, the divine
inspiration of the Holy Scriptures. These critics are divided into two groups, each claiming
scientific learning. One of these groups deals with the history and authorship of the several
books in the Bible. They call themselves "higher critics" of the sacred record; but in fact they
are destructive critics. They proclaim that the books of the Bible are without divine
inspiration and were not written at the time indicated by the record, and in many instances
were written by others than those whose names they bear. The other group deals with the
geological history of the earth and "organic evolution" of all living things upon her face.
Their contention is that the earth is more than seven billion years old; that all life upon it
"evolved" from some infinitismal life in the far distant past by spontaneous generation. These
two groups are parts of the same general class, and each is bent upon the destruction of the
story of creation and the development of humanity as this story is told in the Bible.

Both groups have tried to show that the Bible is a book filled with impossible stories
regarding creation, folklore akin to fairy tales coming from ignorant, simpleminded people,
who either worshiped the forces of nature or imaginary gods with passions like their own.
These critics know that the ancient Egyptians, Chaldeans, Babylonians, Syrians, Greeks and
Romans, worshiped many gods, both male and female. These gods loved and hated, were
jealous of each other, were guilty of all kinds of shortcomings which we find so prevalent
among the peoples of the earth. They have classed the prophets of the Bible and the God of
Enoch, Noah, Abraham and Moses, in the same category with the pagan nations. Many of
these writers have said that the gods were only the creations of the minds of primitive men.
They were tribal gods, and the Hebrews in their worship were no different from these other
nations. They flippantly, in derision, utter the refrain, "Man has created God in his own
image," 1 or, that the "Jaweh of the Book of Joshua, is not the Father-God of Jesus." 2 Their
doctrine is that as man "evolved" and became more intelligent, his gods became more kind
and merciful; less selfish in the sense of dispensing justice and equity among their devotees.
According to this doctrine, the God of Moses and the prophets was vengeful, loving the
shedding of blood and the offering of sacrifice. This class of intelligentsia, wise in their own
conceit, with the wisdom which Isaiah says will perish, is very glib in the pronouncements
that the doctrine of God is a progressive development and that the prophets of the Scriptures
imagined that they were divinely inspired.

Among the most outstanding critics of religion, or theology, we find Dr. John William
Draper and Dr. Andrew Dickson White, although the number who have taken it upon
themselves to bring mankind into a modernistic view is legion. Another eminent person
whom we may mention is Dr. Robert Andrew Millikan, the great scientist, although he
admits that he has gone out of his field in order to correct the fateful superstitions and errors,
as he views them, in the doctrines of the present Christian world. All three of these men are
honorable and presumably honest in their convictions. The stand they have taken has been in
the spirit of freeing religion, especially that of the Christian denominations, from what these
scholars think are fatal errors to the Christian cause. Unfortunately each of these gentlemen
views Christianity as he sees it, in its apostate setting, wherein it is a vastly different religionfrom that which existed in the days of our Lord and his apostles, for since their day the
doctrines and practices of the Primitive Christians have been changed beyond recognition. It
is not however, merely the changed doctrines that come under careful consideration, but the
fundamental doctrines which the present world has discarded and which were taught by the
prophets and by the Son of God himself.

Dr. John William Draper was born in Liverpool, England, May 5, 1811. He received his early
schooling in the English schools and at the University of London. He came to America in
1832 and studied medicine and science at the University of Pennsylvania. He wrote several
books on history and science. His outstanding works are \textit{History of the Intellectual
Development of Europe; History of the American Civil War; Treatise on Human Physiology;
and The Conflict Between Religion and Science}. For some time he was a professor in the
University of New York. He died in the year 1882.

Dr. Andrew Dickson White was born in the year 1832 at Homer, New York. He was an
American diplomat and served with the United States Delegation at St. Petersburg during the
Crimean War. He served in the New York State Senate at the time of the Civil War and was
minister to Germany in 1879 to 1881; minister to Russia 1892-1894 and again in 1897-1902,
and was chairman of the American delegation to the Hague Peace Conference in 1899. He
assisted Ezra Cornell in the founding of Cornell University and became its first president and
gave to that school financial help and his extensive library. It was while traveling as a
diplomat in foreign countries that he gathered his material for his work, \textit{A History of the
Warfare of Science with Theology in Christendom}.

Dr. Robert Andrew Millikan, one of the most distinguished of American scientists, as
previously stated, acknowledged that the study of religion was not in his field, yet at the
instance of others he took it upon himself to advance some thoughts which must, in this
work, receive attention. It is unnecessary to go very far beyond the writing of these
prominent advocates of science in their criticisms, for all other criticisms follow in great
measure the same general pattern. It is necessary at this point to give only a sample of the
modernistic doctrines which have been taught and are now being taught, to confuse an
uninspired religious world and drive it further from the light of truth. These men are
advocates of the doctrine that the prophets only imagined that they were inspired and
conversed with the Lord and his angels. In their imaginations these critics go back in the far
distant past and attempt to portray our ancient fathers as \textit{probably} "seeing a spirit in a storm,"
a "god in the image of a powerful enemy, in the thunder," a "nymph in a stream," and if he
happened to be a "monotheist" every "caprice in nature is attributed to one Great Spirit," and
if he has but one god he tries to get him in a favorable mood rather than a hostile one by the
offering of sacrifice. 3 In this manner the attempt is made to destroy faith in God and to turn
people away from the revealed word of the Lord. It has become a very serious matter and
many professed ministers preaching from their pulpits have fallen prey to these pernicious
and soul-destroying doctrines. The result of this is that there has come to pass a rejection of
Jesus Christ as the Son of God, the Redeemer of the world, and Savior of those who will
repent and accept his Gospel. Many ministers weak in faith have become fearful less these
theories of evolution and scriptural criticisms are true. It becomes necessary therefore that we
who know the truth should raise our voices in defense of revealed religion and speak that
which we do know to be true, that the faith of members of the Church, at least, shall not fail.The evidence that the God of the Old Testament is the God of the New is well established.
The truth of the story of Joseph Smith is too well attested to be successfully refuted. The
Lord declared and taught his disciples that faith would dwindle, men would turn away from
the truth as they did in the days of Noah before the flood, and that a similar condition would
exist preceding the coming of our Lord to take his place as the rightful Ruler of the earth.
Strong delusions have come just as our Lord and Peter and Paul predicted they would. All of
these things give evidence of the near approach of the Millennial reign. The words of the
prophets are being fulfilled.

These learned men all profess great friendship for religious people of the Christian
denominations. They say they do not wish to do anything that is harmful, but desire to clear
away the superstitions and rubbish that have accumulated through many years. For instance,
Dr. Andrew D. White says in the preface to the first volume of his book:

My work in this book is like that of the Russian \textit{mujik} on the Neva. I simply try to aid in
letting the light of historical truth into that decaying mass of outworn thought which attaches
the modern world to mediaeval conceptions of Christianity, and which still lingers among
us—a most serious barrier to religion and morals, and a menace to the whole normal
evolution of society.

For behind this barrier also the flood is rapidly rising—the flood of increased knowledge and
new thought; and this barrier also, though honeycombed and in many places thin, creates a
danger—danger of a sudden breaking away, distressing and clamitous, sweeping before it not
only outworn creeds and noxious dogmas, but cherished principles and ideals, and even
wrenching out most precious religious and moral foundations of the whole social and
political fabric.

My hope is to aid—even if it be but a little—in the gradual and healthful dissolving away of
this mass of unreason, that the stream of "religion pure and undefiled" may flow on broad
and clear, a blessing to humanity. . . .

It had certainly never entered into the mind of either of us that in all this we were doing
anything irreligious or unchristian. Mr. Cornell was reared a member of the Society of
Friends; he had from his fortune liberally aided every form of Christian effort which he
found going on about him, and among the permanent trustees of the public library which he
had already founded, he had named all the clergymen of the town—Catholic and Protestant.
As for myself, I had been bred a churchman, had recently been elected a trustee of one
church college, and a professor in another; those nearest and dearest to me were devoutly
religious. . . . So far from wishing to injure Christianity, we both hoped to promote it; but we
did not confound religion with sectarianism, and we saw in the sectarian character of
American colleges and universities, as a whole, a reason for the poverty of the advanced
instruction then given in so many of them. 4

Dr. Draper, Dr. Millikan and others take similar views. Their fight is not so much against
religion or theology as they state it, as it is against what they call the superstitions,
corruptions, and the perpetuation of mythological beliefs that have found their way into
church organizations perpetuated by "ignorant and infuriated ecclesiastics, parasites,
eunuchs, and slaves." 5

It is a pity that these capable men failed to come in contact with the restored Gospel, for they
could have been of great value if they had been converted to the truth. The fact in the case of
each of these distinguished scientists is pointed out by Paul in his First Epistle to the
Corinthians:

For ye see your calling, brethren, how that not many wise men after the flesh, not many
mighty, not many noble, are called:

For God hath chosen the foolish things of the world to confound the wise; and God hath
chosen the weak things of the world to confound the things which are mighty;

And base things of the world, and things which are despised, hath God chosen, yea, and
things which are not, to bring to nought things that are:

That no flesh should glory in his presence. 6

Much of the difficulty experienced by these scientists and many others, is the fact that they
confound apostate Christianity with the Gospel of Jesus Christ. They recognized fully that
great changes gathered from the pagan world, have come into the churches, but they were
unable to discern the truth from the darkness, and having been led into the pitfalls of organic
evolution and the mis-interpretations and confusion which came through the destructive
criticism, they were unable to see the light. Therefore they discarded the history of the
scriptures as it had been given by revelation, and lost all faith in the miracles and classed
them among the mythology of the nations with whom the Israelites were surrounded. They
looked through colored glasses that distorted all things out of proportion, and hence they
became easy prey to the "strong delusions, that they should believe a lie." 7

The fact of the apostasy, however, was still visible to their eyes. It would have been to the
advantage of restored Christianity had they pointed out the fact of the great apostasy, without
confusing the departures from the truth with the correct teaching of the prophets; but being
without the guidance of the Holy Spirit, they were unable to discover these differences. Truth
and error too frequently appeared to them to be of the same substance, and therefore what
was written by inspiration failed to register as revelation and commandment from the Lord.
Dr. Millikan bears record that Jesus, although he does not accept him as the Only Begotten
Son of God, "taught the gospel of a beneficent creator, whose most outstanding attribute was
love, and that conception of course made love, unselfishness, the first duty of man," then he
adds, "And through all the next thousand years of horrible strife and disaster the life and the
spirit and, to an extent, the conception of Jesus was kept before the whole western world as
the basis of its religion." Then he adds: "I would not at all overlook the backward steps which
religion took during this period, for let us frankly admit that it did take backward steps. It
became deeply encrusted with superstition." 8

Dr. John William Draper senses more fully the departure of the Catholic Church from the
original teachings of our Savior and his apostles, and while he has much criticism of the old
prophets and historians, he has confined the greater part of his criticism to the changes and
doctrines existing in the corrupted church from the third century on, coming principally since
the time of the rule of the popes. The story he tells of apostate Christianity is as vivid as it
could be had he been familiar with the restored Gospel. Since his criticisms deal with these
changes, it may be in perfect order to let him bear witness and confirm the position of theChurch of Jesus Christ as it has been restored and given by divine revelation. In the preface
to his work he says:

In speaking of Christianity, reference is generally made to the Roman Church, partly because
the demands are the most pretentious, and partly because it has commonly sought to enforce
those demands by the civil power. None of the Protestant Churches has ever occupied a
position so imperious—none has ever had such widespread political influence. For the most
part they have been adverse to constraint, and except in very few instances their opposition
has not passed beyond the exciting of theological odium.

As to Science, she has never sought to ally herself to civil power. She has never attempted to
throw odium or inflict social ruin on any human being. She has never subjected any one to
mental torment, physical torture, least of all to death, for the purpose of upholding or
promoting her ideas. She presents herself unstained by cruelties and crimes. But in the
Vatican—we have only to recall the Inquisition—the hands that are now raised in appeals to
the Most Merciful are crimsoned. They have been steeped in blood!

In selecting and arranging the topics now to be presented, I have been guided in part by "the
Confession" of the late Vatican Council, and in part by the order of events in history. Not
without interest will the reader remark that the subjects offer themselves to us now as they
did to the old philosophers of Greece. We still deal with the same questions about which they
disputed. What is God? What is soul? What is the world? How is it governed? Have we any
standard or criterion of truth? And the thoughtful reader will earnestly ask, "Are our solutions
of these problems any better than theirs?" 9

In his second chapter, dealing with "The Origin of Christianity.—Its Transformation on
Attaining Imperial Power.—Its Relation to Science," Dr. Draper has this to say:

For many years Christianity manifested itself as a system enjoining three things—toward
God veneration, in personal life purity, in social life benevolence. In its early days of
feebleness it made proselytes only by persuasion, but, as it increased in numbers and
influence, it began to exhibit political tendencies, a disposition to form a government within
the government, an empire within the empire. These tendencies it has never since lost. They
are, in truth, the logical result of its development. The Roman emperors, discovering that it
was absolutely incompatible with the imperial system, tried to put it down by force. This was
in accordance with the spirit of their military maxims, which had no other means but force
for the establishment of conformity.

Place, profit, power—these were in view of whoever now joined the conquering sect. Crowds
of worldly persons, who cared nothing about its religious ideas, became its warmest
supporters. Pagan at heart, their influence was soon manifested in the paganization of
Christianity that forthwith ensued. The emperor, no better than they, did nothing to check
their proceedings. But he did not personally conform to the ceremonial requirements of the
Church until the close of his evil life, A.D. 337.

That we may clearly appreciate the modifications now impressed on Christianity—
modifications which eventually brought it in conflict with science—we must have as a means
of comparison, a statement of what it was in its purer days. Such, fortunately, we find in the
"Apology or Defense of the Christians against the Accusations of the Gentiles," written byTertullian, at Rome, during the persecution of Severus. He addressed it, not to the emperor,
but to the magistrates who sat in judgment on the accused. It is a solemn and most earnest
expostulation, setting forth all that could be said in explanation of the subject, a
representation of the belief and cause of the Christians made in the imperial city in the face
of the whole world, not a querulous or passionate ecclesiastical appeal, but a grave historical
document. It has ever been looked upon as one of the ablest of the early Christian works. Its
date is about A.D. 200. 10

From Tertullian's able work we see what Christianity was while it was suffering persecution
and struggling for existence. We have now to see what it became when in possession of
imperial power. Great is the difference between Christianity under Severus and Christianity
after Constantine. Many of the doctrines which at the latter period were preeminent, in the
former were unknown.

Two causes led to the amalgamation of Christianity with paganism: 1. The political
necessities of the new dynasty; 2. The policy adopted by the new religion to insure its spread.

1. Though the Christian party had proved itself sufficiently strong to give a master to the
empire, it was never sufficiently strong to destroy its antagonist, paganism. The issue of the
struggle between them was an amalgamation of the principles of both. In this, Christianity
differed from Mohammedanism, which absolutely annihilated its antagonist, and spread its
own doctrines without adulteration.

Constantine continually showed by his acts that he felt he must be the impartial sovereign of
all his people, not merely the representative of a successful faction. Hence if he built
Christian churches, he also restored pagan temples; if he listened to the clergy, he also
consulted the haruspices; if he summoned the Council of Nicea, he also honored the statute
of Fortune; if he accepted the rite of baptism he also struck a medal bearing his title of
"God." His statue, on the top of the great porphyry pillar at Constantinople, consisted of an
ancient image of Apollo, whose fortunes were replaced by those of the emperor, and its head
surrounded by the nails feigned to have been used at the crucifixion of Christ, arranged so as
to form a crown of glory.

Feeling that there must be concessions to the defeated pagan party, in accordance with its
ideas, he looked with favor on the idolatrous movements of his court. In fact the leaders of
these movements were persons of his own family.

2. To the emperor—a mere worlding—a man without any religious convictions, doubtless it
appeared best for himself, best for the empire, and best for the contending parties, Christian
and pagan, to promote their union or amalgamation as much as possible. Even sincere
Christians do not seem to have been averse to this; perhaps they believed that the new
doctrines would diffuse most thoroughly by incorporation in themselves ideas borrowed from
the old, that Truth would assert herself in the end, and the impurity be cast off. In
accomplishing this amalgamation, Helena, the empress-mother, aided by the court ladies, led
the way. For her gratification there were discovered, in a cavern at Jerusalem, wherein they
had lain buried for more than three centuries, the Savior's cross, and those of the two thieves,
the inscription, and the nails that had been used. They were identified by miracle. A true
relic-worship set in. The superstition of the old Greek times reappeared; the times when the
tools with which the Trojan horse was made might still be seen in Metapontum, the sceptreof Pelops at Chaeroneia, the spear of Achilles at Phaselis, the sword of Memnon at
Nicomedia, when the Tegeates could show the head of the Calydonian boar and very many
cities boasted their possession of the true palladium of Troy; when there were statues of
Minerva that could brandish spears, paintings that could blush, images that could sweat, and
endless shrines and sanctuaries at which miracle-cures could be performed.

As years passed on, the faith described by Tertullian was transmuted into one more
fashionable and more debased. It was incorporated with the old Greek mythology. Olympus
was restored, but the divinities passed under other names. The more powerful provinces
insisted on the adoption of their time-honored conceptions. Views of the Trinity, in
accordance with Egyptian traditions, were established. Not only was the adoration of Isis
under a new name restored, but even her image, standing on the crescent moon, reappeared.
The well-known effigy of that goddess, with the infant Horus in her arms, had descended to
our days in the beautiful, artistic creations of the Madonna and Child. Such restorations of
old conceptions under novel forms were everywhere received with delight. When it was
announced to the Ephesians that the Council of that place, headed by Cyril, had decreed that
the Virgin should be called the "Mother of God," with tears of joy they embraced the knees
of their bishop; it was the old instinct peeping out; their ancestors would have done the same
to Diana.

This attempt to conciliate worldly converts, by adopting their ideas and practices, did not
pass without remonstrance from those whose intelligence discerned the motive. "You have,"
says Faustus to Augustine, "substituted your agapae for the sacrifice of the pagans; for their
idols your martyrs, whom you serve with the very same honors. You appease the shades of
the dead with wine and feasts; you celebrate the solemn festivities of the Gentiles, their
calends, and their solstices; and, as to their manners, those you have retained without any
alteration. Nothing distinguishes you from the pagans, except that you hold your assemblies
apart from them." Pagan observances were everywhere introduced. At weddings it was the
custom to sing hymns to Venus.

Let us pause here a moment, and see, in anticipation, to what a depth of intellectual
degradation this policy of paganization eventually led. Heathen rites were adopted, a
pompous and splendid ritual, gorgeous robes, mitres, tiaras, wax-tapers, processional
services, lustrations, gold and silver vases, were introduced. The Roman lituus, the chief
ensign of the augurs, became the crozier. Churches were built over the tombs of martyrs, and
consecrated with rites borrowed from the ancient laws of the Roman pontiffs. Festivals and
commemorations of martyrs multiplied with the numberless fictitious discoveries of remains.
Fasting became the grand means of repelling the devil and appeasing God; celibacy the
greatest of the virtues. Pilgrimages were made to Palestine and the tombs of the martyrs.
Quantities of dust and earth were brought from the Holy Land and sold at enormous prices,
as antidotes against devils. The virtues of consecrated water were upheld. Images and relics
were introduced into the churches, and worshiped after the fashion of the heathen gods. It
was given out that prodigies and miracles were to be seen in certain places, as in the heathen
times. The happy souls of departed Christians were invoked; it was believed that they were
wandering about the world, or haunting their graves. There was a multiplication of temples,
altars, and penitential garments. The festival of the purification of the Virgin was invented to
remove the uneasiness of heathen converts on account of the loss of their Lupercalia, or
feasts of Pan. The worship of images, of fragments of the cross, or bones, nails, and other
relics, a true fetish worship, was cultivated. Two arguments were relied on for theauthenticity of these objects—the authority of the Church, and the working of miracles. Even
the worn-out clothing of the saints and the earth of their graves were venerated. From
Palestine were brought what were affirmed to be the skeletons of St. Mark and St. James, and
other ancient worthies. The apotheosis of the old Roman times was replaced by canonization;
titulary saints succeeded to local mythological divinities. Then came the mystery of
transubstantiation, or the conversion of bread and wine by the priests into flesh and blood of
Christ. As centuries passed, the paganization became more and more complete. Festivals
sacred to the memory of the lance with which the Savior's side was pierced, the nails that
fastened him to the cross, and the crown of thorns, were instituted. Though there were several
abbeys that possessed this last peerless relic, no one dared to say that it was impossible they
could all be authentic. . . . The worshiping and adoring of the dead in their sepulchres,
shrines, and relics; the consecrating and bowing down to images; the attributing of
miraculous powers and virtues to idols; the setting up of little oratories, altars, and statues in
the streets and highways, and on the tops of mountains, the carrying of images and relics in
pompous procession, with numerous lights and music and singing; flagellations at solemn
sessions under the notion of penance; a great variety of religious orders and fraternities of
priests; the shaving of priests, or the tonsure as it is called, on the crown of their heads; the
imposing of celibacy and vows of chastity on the religious of both sexes—all these and many
more rites and ceremonies are equally parts of pagan and popish superstition. Nay, the very
same temples, the very same images, which were once consecrated to Jupiter and the other
demons, are now consecrated to the Virgin Mary, and the other saints. 11

There came out of the transformation of Isis the practice of praying to and the worship of the
Virgin Mary, and she became the mediator between man and God. 12

Considering all these changes which came transforming the Church of Jesus Christ, which he
established in its purity, can we wonder that intelligent men become confused in relation to
religion, and in their confusion and bewilderment turn away from that which is true in
disgust? This corruption of the truth becomes revolting and the revelations of the Lord are
more readily looked upon as figments coming from fanatical and over-wrought minds. I
repeat that the deliberate teaching of false doctrines, which corrupt the minds of many, by
those professing divine authority and to be ministers of Jesus Christ, is a crime of the greatest
magnitude. This is likewise true of those who put forth false theories of science as truth.

It is, however, within the power of every soul to know the truth. All that is necessary is to
"do the will" of the Father and his Son Jesus Christ. We are taught that the Lord has not
forsaken mankind and left them to grope blindly trying to find their way back into his
presence. For wise reasons the Lord withdrew his presence from his children when they came
into this mortal life and called on them to walk by faith and thus to be proved whether under
all the circumstances of mortality, they would be willing to keep his commandments which
he, through his servants the prophets, has given them. Moreover, he has given to every man
guidance through the Spirit of Truth, or Light of Christ. This is not the Holy Ghost, but
another Spirit which, if it is heeded, will lead men to the truth. In relation to this Spirit the
Lord has given this definite information:

For you shall live by every word that proceedeth forth from the mouth of God.

For the word of the Lord is truth, and whatsoever is truth is light, and whatsoever is light is
Spirit, even the Spirit of Jesus Christ.

And the Spirit giveth light to every man that cometh into the world; and the Spirit
enlighteneth every man through the world, that hearkeneth to the voice of the Spirit.

And every one that hearkeneth to the voice of the Spirit cometh unto God, even the Father.
13
To Moroni the Lord revealed this same truth as follows:

For behold, the Spirit of Christ is given to every man, that he may know good from evil;
wherefore, I show unto you the way to judge; for everything which inviteth to do good, and
to persuade to believe in Christ, is sent forth by the power and gift of Christ; wherefore ye
may know with a perfect knowledge it is of God. 14

How true it is that the majority of men reject the truth when it is offered them by revelation
preferring the philosophies and speculations of their uninspired fellows.

For God sent not his Son into the world to condemn the world; but that the world through
him might be saved.

He that believeth on him is not condemned: but he that believeth not is condemned already,
because he hath not believed in the name of the only begotten Son of God.

And this is the condemnation, that light is come into the world, and men loved darkness
rather than light, because their deeds were evil.

For every one that doeth evil hateth the light, neither cometh to the light, lest his deeds
should be reproved

But he that doeth truth cometh to the light, that his deeds may be made manifest, that they are
wrought in God. 15

\newpage
REFERENCES—CHAPTER TWO

Footnotes

1. Smith, J. F., \textit{Signs of the Times}, pp. 76-77.

2. Ibid, p. 78.

3. Millikan, Dr. R. A., \textit{Evolution in Science and Religion}, pp. 67-68.

4. White, Dr. A. D., Introduction VI-VII, \textit{History of the Conflict of Science with Theology}.

5. Draper, Dr. J. W., Preface VI-VIII, \textit{Conflict Between Religion and Science.}

6. 1 Cor. 1:26-29.

7. 2 Thess. 2:11.

8. Millikan, Dr. R. A., \textit{Evolution in Science and Religion}, pp 74-75.

9. Draper, Dr. J. W., \textit{Conflict Between Religion and Science}, X-XII.

10. \textit{Ibid.}, Chapter 2:38-40.

11. \textit{Ibid.}, Chapter 2:45-50.

12. \textit{Ibid.}, Chapter 2:55.

13. D. \& C. 84:45-47. Moroni 7:16.

14. Moroni 7:16.

15. John 3:16-21.

