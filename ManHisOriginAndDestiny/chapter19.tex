\chapter{TESTIMONY OF EARLY BRETHREN}

THE Tenth Article of Faith reads as follows:

"We believe in the literal gathering of Israel and in the restoration of the Ten Tribes; that
Zion will be built upon this [the American] continent; that Christ will reign personally upon
the earth; and, that the earth will be renewed and receive its paradisiacal glory."

We are also taught that we are living in the "Dispensation of the Fulness of Times." This is
the dispensation into which all other dispensations flow. It is spoken of in the scriptures as
"the times of restitution of all things, which God hath spoken by the mouth of all his holy
prophets since the world began." 1 If the earth is to be renewed, to what is it to be renewed?
It must be to some condition which prevailed in the beginning when the Lord pronounced it
"good." Isaiah, in the 65th chapter of his book gives us the story of what this restoration will
be. Likewise in the Doctrine and Covenants, Section 101, verses 23 to 30, we are given a
similar account, and in the same book, Section 133, the Lord reveals in some detail other
things pertaining to this restoration. This work of restoration commenced many years ago,
when the Lord prepared for the restoration of the Church in this dispensation, and received its
impetus when the Lord commenced his "marvelous work and a wonder." 2

We learn in Section 133, that the Ten Tribes are to come to the children of Ephraim to
receive their blessings and be restored; the Lamb shall come and stand on Mt. Zion, and on
the Mt. of Olives and "utter his voice out of Zion, and he shall speak from Jerusalem, and his
voice shall be heard among all people." His voice shall break down the mountains, the "great
deep" shall be driven \textit{back} into the North countries, and the islands shall become one land
and Jerusalem and Zion shall be turned back to their own place, "and the earth shall be like as
it was in the days before it was divided. And the Lord, even the Savior, shall stand in the
midst of his people, and shall reign over all flesh."

Some of our brethren who lived in the days of the Prophet Joseph Smith have written
interesting accounts of this condition which was in the beginning and what it will be like in
the restoration. First we will present parts of the story as related by Elder Parley P. Pratt, in
his \textit{Voice of Warning} and as it is re-published by John Taylor, in his \textit{The Government of God.}
President Taylor introduces the quotation from Elder Parley P. Pratt's writings with the
following sentence:

Now, restoration signifies a bringing back, and must refer to something which existed before;
for if it did not exist before, it could not be restored. I cannot describe this better than Parley
P. Pratt has done in his \textit{Voice of Warning}, and shall therefore make the following extract:—. .
.

Now, we can never understand precisely what is meant by restoration, unless we understand
what is lost or taken away; for instance, when we offer to restore any thing to a man, it is as
much as to say he once possessed it, but had lost it, and we propose to replace or put him in
possession of that which he once had; therefore, when a prophet speaks of the restoration of
\textit{all things}, he means \textit{all things} have undergone a change, and are to be again restored to their
primitive order even as they first existed.

First, then, it becomes necessary for us to take a view of creation as it rolled in purity from
the hand of its Creator; and if we can discover the true state in which it then existed, and
understand the changes that have taken place since, then we shall be able to understand what
is to be restored; and thus our minds being prepared, we shall be looking for the very things
which will come, and shall be in no danger of lifting our puny arm, in ignorance, to oppose
the things of God.

First, then, we will take a view of the earth, as to its surface, local situation, and productions.

When God had created the heavens and the earth, and separated the light from the darkness,
his next command was to the waters, Gen. 1:9—And God said, "Let the waters under the
heaven be gathered together into \textit{one place}, and let the dry land appear: and it was so!" From
this we learn a marvelous fact, which very few ever realized or believed in this benighted
age; we learn that the waters, which are now divided into oceans, seas, and lakes, were then
all gathered together, into one vast ocean; and, consequently, that the land, which is now torn
asunder, and divided into continents and islands, almost innumerable, was then \textit{one} vast
continent or body, \textit{not} separated as it is now.

Second, we hear the Lord God pronounce the earth, as well as every thing else, \textit{very good.}
From this we learn that there were neither deserts, barren places, stagnant swamps, rough,
broken, ragged hills, nor vast mountains covered with eternal snows; and no part of it was
located in the frigid zones, so as to render its climate dreary and unproductive, subject to
eternal frost, or everlasting chains of ice,—

Where no sweet flowers the dreary landscape cheer, Nor plenteous harvests crown the
passing year;

but the whole earth was probably one vast plain, or interspersed with gently rising hills, and
sloping vales, well calculated for cultivation; while its climate was delightfully varied with
the moderate changes of heat and cold, of wet and dry, which only tended to crown the
varied year, with the greater variety of productions, all for the good of man, animal, fowl, or
creeping thing; while from the flowery plain, or spicy grove, sweet odors were wafted on
every breeze; and all the vast creation of animated beings breathed naught but health, peace,
and joy.

Next, we learn from Genesis 1:29-30, "And God said, Behold, I have given you every herb
bearing seed, which is upon the face of all the earth, and every tree, in which is the fruit of a
tree, yielding seed; to you it shall be for meat. And to every beast of the earth, and to every
fowl of the air, and to every thing that creepeth upon the earth, wherein there is life, I have
given every green herb for meat: and it was so." From these verses we learn that the earth
yielded neither noxious weeds nor poisonous plants, nor useless thorns and thistles; indeed,
everything that grew was just calculated for the food of man, beast, fowl, and creeping thing;
and their food was all vegetable; flesh and blood were never sacrificed to glut their souls, or
gratify their appetites; the beasts of the earth were all in perfect harmony with each other; the
lion ate straw like the ox—the wolf dwelt with the lamb—the leopard lay down with the
kid—the cow and bear fed together, in the same pasture . . . in perfect security, under the
shade of the same trees; all was peace and harmony, and nothing to hurt nor disturb, in all the
holy mountain.

And to crown the whole, we behold man created in the image of God, and exalted in dignity
and power, having dominion over all the vast creation of animated beings, which swarmed
through the earth, while at the same time, he inhabits a beautiful and well-watered garden, in
the midst of which stood the tree of life, to which he had free access; while he stood in the
presence of his Maker, conversed with him face to face, and gazed upon his glory, without a
dimming veil between. O reader, contemplate, for a moment, this beautiful creation, clothed
with peace and plenty; the earth teeming with harmless animals, rejoicing over all the plain,
the air swarming with delightful birds, whose never ceasing notes filled the air with varied
melody; and all in subjection to their rightful sovereign who rejoiced over them; while, in a
delightful garden—the capital of creation,—man was seated on the throne of his vast empire,
swaying his scepter over all the earth, with undisputed right; while legions of angels
encamped round about him, and joined their glad voices in grateful songs of praise, and
shouts of joy; neither a sign nor a groan was heard, throughout the vast expanse; neither was
there sorrow, tears, pain, weeping, sickness, nor death; neither contention, wars, nor
bloodshed; but peace crowned the seasons as they rolled, and life, joy, and love reigned over
all his works. But, O! how changed the scene.

It now becomes my painful duty, to trace some of the important changes, which have taken
place, and the causes which have conspired to reduce the earth and its inhabitants to their
present state.

First, man fell from his standing before God, by giving heed to temptation; and this fall
affected the whole creation, as well as man, and caused various changes to take place; he was
banished from the presence of his Creator, and a veil was drawn between them, and he was
driven from the garden of Eden, to till the earth, which was cursed for man's sake, and should
bring forth thorns and thistles; and in the sweat of his face should earn his bread, and in
sorrow eat of it, all the days of his life, and finally return to dust. But as to Eve, her curse was
a great multiplicity of sorrow and conception; and between her and the seed of the serpent,
there was to be a constant enmity; it should bruise the serpent's head, and the serpent should
bruise his heel.

Now, reader, contemplate the change. This scene, which was so beautiful a little while
before, had now become the abode of sorrow and toil, of death and mourning; the earth
groaned with its production of accursed thorns and thistles; man and beast at enmity; the
serpent slily creeping away, fearing lest his head should get the deadly bruise; and man
starting amid the thorny path, in fear, lest the serpent's fangs should pierce his heel; while the
lamb yields his blood upon the smoking altar. Soon man begins to persecute, hate, and
murder his fellow; until at length the earth is filled with violence; all flesh becomes corrupt,
the powers of darkness prevail; and it repented Noah that God had made man, and it grieved
him at his heart, because the Lord should come out in vengeance, and cleanse the earth by
water.

How far the flood may have contributed, to produce the various changes, as to the division of
the earth into broken fragments, islands and continents, mountains and valleys, we have not
been informed; the change must have been considerable. But after the flood, in the days of
Peleg, the earth was divided.—See Gen. 10:25,—a short history, to be sure, of so great anevent; but still it will account for the mighty revolution, which rolled the sea from its own
place in the north, and brought it to interpose between different portions of the earth, which
were thus parted asunder, and moved into something near their present form; this, together
with the earthquakes, revolutions, and commotions which have since taken place, have all
contributed to reduce the face of the earth to its present state; while the great curses which
have fallen upon different portions, because of the wickedness of men, will account for the
the stagnant swamps, the sunken lakes, the dead seas, and great deserts.

Then speaking of the restoration we have a continuation as follows:

Thus you see, every mountain being laid low, and every valley exalted, and the rough places
being made plain, and the crooked straight, that these mighty revolutions will begin to restore
the face of the earth to its former beauty. But all this done, we have not yet gone through our
restoration; there are many more great things to be done, in order to restore all things. . . .

Thus, having cleansed the earth, and glorified it with the knowledge of God, as the waters
cover the sea, and having poured out his Spirit upon all flesh, both men and beast becoming
perfectly harmless, as they were in the beginning, and feeding on vegetable food only, while
nothing is left to hurt or destroy in all the vast creation, the prophets then proceed to give us
many glorious descriptions of the enjoyment of its inhabitants. "They shall build houses and
inhabit them; they shall plant vineyards, and drink the wine of them; they shall plant gardens
and eat the fruit of them; they shall not build and another inhabit; they shall not plant and
another eat; for as the days of a tree are the days of my people, and mine elect shall enjoy the
work of their hands. They shall not labor in vain, nor bring forth in trouble; for they are the
seed of the blessed of the Lord, and their offspring with them; and it shall come to pass, that
before they call I will answer, and while they are yet speaking I will hear." In this happy state
of existence it seems that all people will live to the full age of a tree, and this too without
pain or sorrow, and whatsoever they ask will be immediately answered, and even all their
wants will be anticipated. Of course, then, none of them will sleep in the dust, for they will
prefer to be translated; that is, changed in the twinkling of an eye, from mortal to immortal;
after which they will continue to reign with Jesus on the earth.

A great council will then be held to adjust the affairs of the world, from the commencement,
over which Father Adam will preside as head and representative of the human family. There
have been, in different ages of the world, communications opened between the heavens and
the earth. (\textit{Voice of Warning and Government of God}, pages 106-115.)

TESTIMONY OF ORSON PRATT

At the funeral of Caroline Grant Smith, wife of William B. Smith, in Nauvoo, May 24, 1845,
Elder Orson Pratt gave the following in his discourse:

In the morning of creation all things were pronounced good by the Creator, as they rolled
into organized existence unsullied and without a curse. Man, the last and noblest of God's
creations, was placed in the garden of Eden, being governed by laws and restricted by
commandments, not being subject to sickness, disease, or death. Adam was placed upon the
earth an immortal being. He was placed in the garden to dress, beautify and adorn it, and to
hold the supremacy of power over all the things of God's creation.

Instead of our first parents eating animal food, they subsisted upon herbs and the fruits of the
earth, which were originally designed for the food of man, and had they not transgressed they
would have both been living upon the earth at the present day, as fair, as healthy, as beautiful
and as free from sickness and death, as they were previous to the transgression. What was
that transgression? it was violating a single commandment of God, and disregarding the
counsel of those immortal beings who stood above them in authority. . . . His was a simple
commandment; but the violation of it subjected Adam to the fall from his exalted station in
the favor of God. Consequently a curse was placed upon all created things, and in the
posterity of Adam were sown the seeds of dissolution. . . . That transgression subjected him
to a curse and that was a fall from a state of immortality to that of mortality; consequently
you see that it was through his agency that death entered the world. (\textit{Times and Seasons}, Vol.
6, pp. 918-919.)

August 29, 1852, the First Presidency asked Orson Pratt to give a discourse on marriage. In
this discourse Elder Pratt said:

The Lord himself solemnized the first marriage pertaining to this globe, and pertaining to
flesh and bones here upon this earth. I do not say pertaining to mortality; for when the first
marriage was celebrated, no mortality was here. The first marriage that we have any account
of, was between two immortal beings—old father Adam and old mother Eve; they were
immortal beings: death had no dominion nor power over them; and they were capable of
enduring forever and ever in their organization. . . .

What would you consider, my hearers, if a marriage was to be celebrated between beings not
subject to death? Would you consider them joined together for a certain number of years, and
that then all their covenants were to cease for ever, and the marriage contract to be dissolved?
Would it look reasonable and consistent? Every heart would say that the work of God is
perfect in and of itself, and inasmuch as sin had not brought imperfection upon the globe,
what God joined together could not be dissolved, and destroyed and torn asunder by any
power beneath the celestial world, consequently it was eternal; the sealing of the great
Jehovah upon Adam and Eve was eternal in its nature. (\textit{Journal of Discourses} Vol. 1, p. 58.)

Again, July 25, 1852, Elder Orson Pratt preached a wonderful discourse, designated as "A
funeral sermon of all Saints and Sinners; also of the heavens and the Earth." This entire
discourse which is printed in the Millennial Star and other publications, should be read by
every member of the Church. It cannot be produced here in its fulness, but the following
taken from it has to do with the subject of Adam and the fall.

I will take a text, which you will find recorded in the 51st chapter of the prophecy of Isaiah,
and the sixth verse: "Lift up your eyes to the heavens, and look upon the earth beneath: for
the heavens shall vanish away like smoke, and the earth shall wax old like a garment, and
they that dwell therein shall die in like manner: but my salvation shall be forever, and my
righteousness shall not be abolished!"

All things with which we are acquainted, pertaining to this earth of ours, are subject to
change; not only man, so far as his temporal body is concerned, but the beasts of the field,
the fowls of the air, the fishes of the sea, and every living thing with which we are
acquainted—all are subject to pain and distress, and finally die and pass away; death seems
to have universal dominion in our creation. It certainly is a curious world; it certainly does
not look like a world constructed in such manner as to produce eternal happiness; and it
would be very far from the truth, I think, for any being at the present time to pronounce it
very good; everything seems to show us that goodness, in a great degree, has fled from this
creation. If we partake of the elements, death is there in all its forms and varieties; and when
we desire to rejoice, sorrow is there, mingling itself in every cup; and woe, and
wretchedness, and misery, seem to be our present doom.

There is something, however, in man, that is constantly reaching forward after happiness,
after pleasure, after something to satisfy the longing desire that dwells within his bosom.
Why is it that we have such a desire? And why is it that it is not satisfied? Why is it that this
creation is so constructed? And why is it that death reigns universally over all living earthly
beings? Did the great Author of creation construct this little globe of ours subject to all these
changes, which are calculated to produce sorrow and death among the beings that inhabit it?
Was this the original condition of our creation? I answer, no; it was not so constructed. But
how was it made in the beginning? All things that were made pertaining to this earth were
pronounced "very good." Where there is pain, where there is sickness, where there is sorrow,
and where there is death, this saying can not be understood in its literal sense; things cannot
be very good where something very evil reigns and has universal dominion.

We are, therefore, constrained to believe, that in the first formation of our globe, as far as the
Mosaic history gives us the information, everything was perfect in its formation; that there
was nothing in the air, or in the waters or in the solid elements that was calculated to produce
misery, wretchedness, unhappiness, or death, in the way that it was then organized; not but
what the same elements, organized a little differently, would produce all these effects; but as
it was then constructed, we must admit that every particle of air, of water, and earth, was so
organized as to be capable of diffusing life and immortality through all the varied species of
animated existence—immortality reigned in every department of creation; hence it was
pronounced "very good."

When the Lord made the fowls of the air, and the fishes of the sea, to people the atmospheric
heavens, or the watery elements, these fowls and fishes were so constructed in their nature as
to be capable of eternal existence. To imagine anything different from this, would be to
suppose the Almighty to form that which was calculated to produce wretchedness and
misery. What says the Psalmist David upon the subject? He says that all the works of the
Lord shall endure forever. Did not the Lord make the fish? Did he not make the fowls of the
heavens? Yes. Did he not make the beasts of the field, and the creeping things, and the
insects? Yes. Do they endure forever? They apparently do not; and yet David says all his
works are constructed upon that principle. Is this a contradiction? No. God has given some
other particulars in relation to these works. He has permitted the destroyer to visit them, who
has usurped a certain dominion and authority, carrying desolation and ruin on every hand; the
perfections of the original organizations have ceased. But will the Lord for ever permit these
destructions to reign? No. His power exists, and the power of the destroyer exists. His power
exists, and the power of death exists; but his power exceeds all other powers; and
consequently wherever a usurper comes in and lays waste any of his works, he will repair
these wastes, build up the old ruins, and make all things new: Even the fish of the sea, and
the fowls of the heavens, and the beasts of the earth, must yet, in order to carry out the
designs of the Almighty, be so constructed as to be capable of eternal existence.

It would be interesting to know something about the situation of things when they were first
formed, and how this destroyer happened to make inroads upon this fair creation; what the
causes were, and why it was permitted.

Man, when he was first placed upon this earth, was an immortal being, capable of eternal
endurance; his flesh and bones, as well as his spirit, were immortal and eternal in their
nature; and it was just so with all the inferior creation—the lion, the leopard, the kid, and the
cow; it was so with the feathered tribes of creation, as well as those that swim in the vast
ocean of waters; all were immortal and eternal in their nature; and the earth itself, as a living
being, was immortal and eternal in its nature. What! is the earth alive too? If it were not, how
could the words of our text be fulfilled, where it speaks of the earth's dying? How can that
die that has no life? "Lift up your eyes to the heavens above, say the Lord, and look upon the
earth beneath: for the heavens shall vanish away like smoke, and the earth shall wax old like
a garment, and they that dwell therein shall die in like manner." What! the earth and the
heavens to die? Yes, the material heavens and the earth must all undergo this change which
we call death; and if so, the earth must be alive as well as we. The earth was so constructed
that it was capable of existing as a living being to all eternity, with all the swarms of animals,
fowls, and fishes that were first placed upon the face thereof. But how can it be proved that
man was an immortal being? We will refer you to what the Apostle Paul has written upon
this subject: he says that by one man came death; and he tells us how it came: It was by the
transgression of one individual that death was introduced here. But did transgression bring in
all these diseases and this sorrow, this misery and wretchedness, over the whole face of this
creation? Is it by the transgression of one person that the very heavens are to vanish away as
smoke, and the earth is to wax old like a garment? Yes, it is by the transgression of one; and
if it had not been for his transgression, the earth never would have been subject to death.
Why? Because the works of the Lord are so constructed as to exist for ever; and if death had
come in without a cause, and destroyed the earth, and laid waste the material heavens, and
produced a general and utter overthrow and ruin in this fair creation, then the works of the
Lord would have ceased to endure according to the promise, being imperfect in their
construction, and consequently not very good.

But what was the sin, and what was the nature of it? I will tell you what it was; it was merely
the partaking of a certain kind of fruit. But, says one, "I should think there is no harm in
eating fruit." There would not be unless God gave a command upon the subject. There are
things in nature that would be evil without a commandment: If there were no commandment,
it would be evil for you to murder an innocent being, and your own conscience would tell
you it was an evil thing. It is an evil for any individual to injure another, or to infringe upon
the rights of another, independent of any revealed law; for the savage, or that being who has
never heard of the written laws of heaven—who has never heard of the revealed laws of
God—with regard to these principles—as well as the Saint, knows that it is an evil to infringe
upon the rights of another; the very nature of the thing shows that it is an evil; but not so in
regard to many other things that are evil; which are only made evil by commandment.

For instance, here is the Sabbath day; a person who never heard the revealed law of God
upon the subject, never could conceive that it was an evil to work on the Sabbath day; he
would consider it just as right to work on the first day of the week, as on the seventh; he
would perceive nothing in the nature of the thing by which he could distinguish it to be an
evil. So with regard to eating certain fruit, it was the commandment of the Great God that
made it an evil. He said to Adam and Eve, "Here are all the fruits of the garden; you may eat
of them freely except this one tree that stands in the midst of the garden; now beware for in
the day you eat thereof you shall surely die." Don't we perceive that the commandment made
this an evil? Had it not been for this commandment, Adam would have walked forth and
freely partaken of every tree, without any remorse of conscience; just as the savage, who
never had heard the revealed will of God, would work on the Sabbath, the same as on any
other day, and have no conscience about the matter. But when a man murders, he knows it to
be an injury, and he has a conscience about it, though he never heard of God; and so with
thousands of evils. But why did the Lord place man under these peculiar circumstances?
Why did he not withhold the commandment, if the partaking of the fruit, after the
commandment was given, was sin? Why should there have been a commandment upon the
subject at all, inasmuch as there was no evil in the nature of the thing to be perceived or
understood? The Lord had a purpose in view; though he constructed this fair creation, as we
have told you, subject to immortality, and capable of eternal endurance, and though he had
constructed men capable of living forever, yet he had an object in view in regard to that man,
and the creation he inhabited. What was the object? And when shall this object be
accomplished?

Why, the Lord wanted this intelligent being called man, to prove himself; inasmuch as he
was an agent. He desired that he should show himself approved before his Creator.

How could this be done without a commandment? Can you devise any possible means? Is
there any person in this congregation having wisdom sufficient to devise any means by which
an intelligent being can show himself approved before a superior intelligence, unless it be by
administering to that man certain laws to be kept? No. Without law, without commandment
or rule, there could be no possible way of showing his integrity; it could not be said that he
would keep all the laws that govern superior orders of beings, unless he had been placed in a
position to be tried, and thus proved whether he would keep them or not. Then it was wisdom
to try the man and the woman, so the Lord gave them this commandment; if he had not
intended the man should be tried by this commandment, he never would have planted that
tree. He never would have placed it in the midst of the garden. Now the very fact that he
planted it where the man could have easy access to it, shows that he intended man should be
tried by it, and thus prove whether he would keep his commandments or not. The penalty of
disobedience to this law was death.

But could he not give a commandment, without affixing a penalty? He could not; it would be
folly, even worse than folly, for God to give a law to an intelligent being, without affixing a
penalty to it if it were broken. Why? Because all intelligent beings would discard the very
idea of a law being given, which might be broken at pleasure, without the individuals
breaking it being punished for their transgression. They would say—"Where is the principle
of justice in the giver of the law? It is not here: we do not reverence him nor his law; justice
does not have an existence in his bosom. He does not regard his own laws, for he suffers
them to be broken with impunity, and trampled under foot by those whom he has made;
therefore we care not for him or his laws; nor his pretended justice. We will rebel against it.
Where would have been the use of it if there had been no penalty affixed?

But what is the nature of this penalty? It was wisely ordained to be of such a nature as to
instruct man. Penalties inflicted upon human beings here, by governors, kings, or rulers, are
generally of such a nature as to benefit them.

Adam was appointed lord of this creation; a great governor, swaying the scepter of power
over the whole earth. When the governor, the person who was placed to reign over this fair
creation, had transgressed, all in his dominion had to feel the effects of it, the same as a
father or a mother, who transgress certain laws, frequently transmit the effects thereof to the
latest generations.

How often do we see certain diseases becoming hereditary, being handed down from father
to son for generations? Why? Because in the first instance there was a transgression, and the
children partook of the effects of it. And what was the fullest extent of the penalty of Adam's
transgression? I will tell you—it was death. The death of the immortal tabernacle—of the
tabernacle where the seeds of death had not been, that was wisely framed, and pronounced
very good; the seeds of death were introduced into it. How and in what manner? Some say
there was something in the nature of the fruit that introduced mortality. Be this as it may, one
thing is certain, death entered into the system; it came there by some means, and sin was the
main spring by which this monster was introduced. If there had been no sin, our father Adam
would at this day have been in the garden of Eden, as bright and as blooming, as fresh and as
fair, as ever, together with his lovely consort Eve, dwelling in all the beauty of youth.

By one man came death—the death of the body. What becomes of the spirit when the body
dies? Will it be perfectly happy? Would old father Adam's spirit have gone back into the
presence of God, and dwelt there eternally, enjoying all the felicities and glories of heaven,
after his body had died? No; for the penalty of that transgression was not limited to the body
alone. When he sinned, it was with both the body and the spirit that he sinned. It was not only
the body that ate of the fruit, but the spirit gave the will to eat; the spirit sinned therefore as
well as the body; they were agreed in partaking of that fruit. Was not the spirit to suffer then
as well as the body? Yes. How long? To all ages of eternity, without any end, while the body
was to return back to its mother earth, and there slumber to all eternity. That was the effect of
the fall, leaving out the plan of redemption; so that, if there had been no plan of redemption
prepared from before the foundation of the world, man would have been subject to an eternal
dissolution of the body and spirit—the one to lie mingling with its mother earth, to all ages of
eternity, and the other to be subject, throughout all future duration, to the power that deceived
him, and led them astray; to be completely miserable, or as the Book of Mormon says, "dead
as to things pertaining to righteousness." (\textit{Journal of Discourses}, Vol. 1, pp. 280-284.)

From an epistle by the Prophet to the Elders in Missouri, sent January 22, 1834, the
following is taken:

Though man in his own supposed wisdom would not admit the influence of a power superior
to his own, yet for wise and great purposes, for the good and happiness of his creatures, God
has instructed man to form wise and wholesome laws, since he had departed from him and
refused to be governed by those laws which God had given by his own voice from on high in
the beginning. But notwithstanding the transgression, by which man had cut himself off from
an immediate intercourse with his maker without a Mediator, it appears that the great and
glorious plan of his redemption was previously provided: the sacrifice prepared; the
atonement wrought out in the mind and purpose of God, even in the person of the Son,
through whom man was now to look for acceptance and through whose merits he was now
taught that he alone could find redemption, since the word had been pronounced, Unto dust
thou shalt return.

But that man was not able himself to erect a system, or plan with power sufficient to free him
from a destruction which awaited him is evident from the fact that God, as before remarked,
prepared a sacrifice in the gift of his own Son who should be sent in due time, to prepare a
way, or open a door through which man might enter into the Lord's presence, whence he had
been cast out for disobedience. From time to time these glad tidings were sounded in the ears
of men in different ages of the world down to the time of Messiah's coming. By faith in the
atonement or plan of redemption, Abel offered to God a sacrifice that was accepted, which
was the firstlings of the flock. Cain offered of the fruit of the ground, and was not accepted,
because he could not do it in faith; he could have no faith, or could not exercise faith contrary
to the plan of heaven. It must be shedding of blood of the Only Begotten to atone for man;
for this was the plan of redemption; and without the shedding of blood was no remission; and
as the sacrifice was instituted as a type, by which man was to discern the great Sacrifice
which God had prepared; to offer a sacrifice contrary to that, no faith could be exercised,
because redemption was not purchased in that way, nor the power of atonement instituted
after that order; consequently Cain could have no faith; and whatsoever is not faith is sin.
(\textit{Teachings of the Prophet Joseph Smith}, pp. 57-58.)

\newpage
Footnotes

1. Acts 3:21.

2. D. \& C. Sec. 4.

