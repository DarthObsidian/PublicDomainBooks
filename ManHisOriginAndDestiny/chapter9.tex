\chapter{THE HYPOTHESIS OF ORGANIC EVOLUTION—3}

IN the Third and Fourth \textit{Chapters of his Descent of Man}, Charles Darwin indulges in a rather
difficult struggle to close the gap between the mental powers of ``man and those of the lower
animals.'' He admits that there is an ``immense'' distance between the ``highest ape,'' whatever
that happens to be, ``and the lowest savage,'' whatever that happens to be, ``yet they have
certain instincts in common.'' His opening paragraph (chapter 3) is as follows:

We have seen in the last two chapters that man bears in his bodily structure clear traces of his
descent from some lower form; but it may be urged that, as man differs so greatly in his
mental powers from all other animals, there must be some error in this conclusion. No doubt
the difference in this respect is enormous, even if we compare the mind of one of the lowest
savages, who has no words to express any number higher than four, and who uses hardly any
abstract terms for common objects or for the affections, with that of the most highly
organized ape. The difference would, no doubt, still remain immense, even if one of the
higher apes had been improved or civilized as much as a dog has been in comparison with its
parent-form, the wolf or jackal. The Fuegians rank amongst the lowest barbarians; but I was
continually struck with surprise how closely the three natives on board H.M.S. ``Beagle,''
who had lived some years in England, and could talk a little English, resembled us in
disposition and in most of our mental faculties. If no organic being excepting man had
possessed any mental power, or if his powers had been of a wholly different nature from
those of the lower animals, then we should never have been able to convince ourselves that
our high faculties had been gradually developed. But it can be shewn that there is no
fundamental difference of this kind. We must also admit that there is a much wider interval
in mental power between one of the lowest fishes, as a lamprey or lancelet, and one of the
higher apes, than between an ape and man; yet this interval is filled up by numberless
gradations. 1

After this introduction the honorable gentleman, who erroneously glories in the thought that
he has an amoeba, or perhaps rather, ``the larvae of existing Ascidians,'' 2 as very distant
grandparents, laboriously endeavors to show that this great dividing gulf should not be
considered as beyond bridging in course of time. He states that these ``lower animals'' and
man have certain instincts in common, ``even as it is.'' Then he discourses on these
``instincts,'' which are enumerated as ``sexual-love, the love of mother for her new-born
offspring, the desire possessed by the latter to suck, and so forth.'' Was this advocate of
organic evolution unaware of the fact that it was written in the scriptures that the Lord
``created great whales, and \textit{every living creature that moveth}, which the waters brought forth
abundantly, \textit{after their kind,} and every winged fowl \textit{after his kind:} and God saw that it was
good?'' Likewise he brought forth, ``cattle, and creeping things, and beasts of the earth after
his kind''; and that all of these were commanded to multiply and ``fill the earth?'' How could
they do all of this and keep this commandment \textit{if} they had not been endowed with these
``instincts in common with man?'' What has this to do with the organic evolution theory? Of
course it is natural for a mother to have love for her young, for the Lord endowed her with it;
and the poor little animal would starve to death if it did not have the ``common instinct'' to
seek its food! In all of this these ``lower animals'' are faithfully performing the commandment
the Lord had given them.

Then the discourse goes on to enumerate other ``instincts''—and with this we have no
controversy with our opponents—such as love and hate, which the animals manifest. For all I
know animals may be able to count more than four, at least they have the sense to distinguish
between two, three, and a herd. We freely admit that many of them, in their wild state gather
in flocks and herds, and it is possible that they have some social aim in doing so, as well as
considering it a means of protection. Latter-day Saints, at least, do not take the view that
animals have no reason, and cannot think. We have divine knowledge that each possesses a
spirit in the likeness of its body, 3 and that each was created spiritually before it was
naturally, or given a body 4 on the earth. Naturally, then, there is some measure of
intelligence in members of the animal kingdom. The fact remains, however, that they
received their place and their bounds by divine decree, \textit{which they cannot pass.} We admit
that many have a language, or some power of communication, whether it be the elephant, the
bear, the fox, the bee, the ant, or the spider. And be it remembered that the amount, or extent,
of the intelligence \textit{does not depend on the size of the brain!} Judged by performance the lowly
ant manifests greater intelligence than Mr. Darwin's ``highest ape.'' The busy bee can travel
for many miles and knows where it is going and its way home. Bees are organized and live in
communities and work harmoniously together; so do ants. Some animals may have some
``sense of beauty,'' and Mr. Darwin is right in saying birds build nests without having
previous instruction. The snake-like eel and the humble salmon follow the habits of their
ancestors without having been taught and this is more than man can do! The eel from the
streams of England, Holland or other places, finds its way to the deep waters of the
Galapagos sea and back again from whence it started, if I am rightly informed. The salmon
leaves the waters where it was spawned for a life in the sea and then returns to the same
stream from whence it went in its youth. It comes home to spawn and die. Is this
intelligence? Some call it ``instinct,'' a debatable question.

Attention has been called to the fact that ``All kingdoms have a law given. . . . And unto every
kingdom is given a law; and unto every law there are certain bounds also and conditions.''
Man, in the beginning was given laws pertaining to his being, temporally as well as
spiritually. So were the beast, the fowl, the fish—all creatures were given laws and
commandments, \textit{which they cannot pass}. Among these laws was the measure of intelligence
which each possesses and beyond the bounds of this decree they cannot go. Therefore the
beast of the forest and the domesticated animal remain within the bounds divinely set at the
beginning. Their measure of intelligence is fixed and limited; not so with man, for he is the
offspring of God and has been commanded to keep his commandments with the promise that
he may become perfect, even as his Father, etc. (Matt. 5:48.)

Many animals have superlative power of smell and of hearing far beyond the power of man.
The bloodhound has a keen sense of smell and is used in tracing criminals. This is a power
given which men call ``scent'' or ``spoor.'' Sportsmen hunting wild game endeavor to keep free
from wind blowing in the direction of their game, lest the air carry evidence of their
presence. So the work of nature goes on and man is called on to marvel at the powers in the
so-called ``lower animals.'' All this is true, and to use a common phrase—``So what!'' There is
nothing here, in any of these things, that in the least gives evidence of any relationship
between beast and man. All such thinking originates in the minds of foolish, misguided men.
The Almighty, however, has not placed the requirement or responsibility upon the animal
world—beast, reptile, fowl or fish, to bow down and worship him. He placed each in its
sphere, gave it commandments commensurate with its position. They have been commanded
to multiply, not to pray. To increase \textit{after their kind,} not pay homage, and to these
commandments they are true and faithful. They have not been commanded to believe in God
or be damned. These commandments, to believe in God, to obey his commandments, to pray
to him in the name of Jesus Christ \textit{have} been given to men everywhere throughout the world.
When men and nations have rebelled against his authority and his commandments, they have
perished. We, the human family, have been commanded to be obedient and worship him, the
Creator of the earth and all things therein, and yet we find men, otherwise intelligent, in open
rebellion and offering to him the greatest insult, denying his existence, his Fatherhood, his
right to command and direct, and in their open rebellion they choose to worship an amoeba, a
fish, a reptile, a baboon and place them in their ancestral, or genealogical tree!

Yes, we who have faith in the Supreme Being—God the Eternal Father, whose offspring we
are, are willing to concede to the animals some measure of intelligence. We are happy to
know, or believe, that they have some mental powers; that they can think, that they may have
a sense of beauty, that they may exercise the spirit of pleasure and happiness, that they may
become angry and remember a wrong committed against them, etc. etc., but all of this does
not make them the offspring of God, nor does it indicate that the Almighty had nothing to do
with their being.

Speaking of the powers of animals which have not been given to man, calls to mind the
following from the pen of Camille Flammarion, in his book, \textit{The Unknown}, wherein he says:

Take the pendulum beating each second in the air. If we double its beats we obtain the
following:

1 Degree .....................2

2 Degrees ....................4

3 `` .....................8

4 `` ...................16

5 `` ...................32

6 `` ...................64

7 `` .................128

8 `` ...... ...........256 Sound.

9 `` ..................512

10 `` ...............1,024

15 `` .............32,768

At the fifth degree, after the beginning to 32 vibrations in a second we enter the region where
the vibration of the atmosphere is revealed to us under the name of sound. We there find the
lowest musical note. If among musical notes the most solemn is chosen—for instance, the
lowest octave of the organ—it will be perceived that elementary sensations, though forming a
continuous whole, which is essential that the sound may remain musical, are nevertheless
distinct to a certain degree. ``The lower the note is,'' says Helmholtz, ``the better does the ear
distinguish in it the successive pulsations of the air.''

In the six following degrees the vibrations in each second increase from 32 to 32,768; each
doubling reproduces the same note in a higher octave. The normal diapason, which gives us
the note \textit{la} (or F), is a vibration of 455 a second, and has 870 vibrations when doubled. The
sharpest sound has about 56,000 vibrations, and the region of sound ends there, so far as the
human ear is concerned. But probably some animals, better gifted than ourselves, may hear
sound too acute for our organs—that is, sounds the rapidity of whose vibrations overpass our
limits. 5

We are willing to admit all of this, and also that there are waves that we cannot hear and
cannot understand without external aid and that some of these waves are, apparently, audible
to the delicate hearing of animals. This does not justify anyone in saying that man belongs to
the races of ``lower animals,'' nor does it show that animals by training can be brought up to
the standard even of ``savage'' men.

It is extremely doubtful that there have been in the past, and may be now, any people so low
in the scale of intelligence that they have no sound to convey a thought of more than four.
Mr. Darwin refers to the Fuegians of Tierra Del Fuego, and says, ``The Fuegians rank
amongst the lowest barbarians.'' He then adds later on, (pp. 143-144) ``The Fuegians were
probably compelled by other conquering hordes to settle in their inhospitable country, and
they may have become in consequence somewhat more degraded; but it would be difficult to
prove that they have fallen much below the Botocudos, which inhabit the finest parts of
Brazil.'' 6 The thought here as I view it, is to convey the idea that the Fuegians, who were
amongst the ``lowest barbarians,'' and the Botocudos, who were still lower, but living in the
choice land of Brazil, were primitive peoples who had not yet come out of the ``Stone Age''
or some other supposed geological age, and were that far behind their fellows in other parts
of the ``savage'' or barbarian world. It is too late to say anything that will do Mr. Darwin any
good—he has gone where he has the opportunity to learn of his folly—but to others who are
inclined to follow his lead, permit me to say that we have the evidence that these Fuegians
and the lower Botocudos, are descendants of a once white and delightsome people; an
intelligent people, who had the guidance of the Lord and his holy prophets among them. It
was because of their wickedness and rebellion against the Living God that they were brought
to this deplorable condition. The same condition came upon many other peoples because
Satan came among them, after they had been taught faith in God and had been given his
commandments, and told them to believe it not, and we are informed that ``men began from
that time forth to be carnal, sensual, and devilish.'' 7

In speaking of the Fuegians and other ``barbarian'' peoples of ``primitive'' times, Mr. Darwin
infers that such peoples are limited in their speech to a few simple words and they are unable
in extreme cases to count beyond four for lack of words to express the meaning. In this he
runs afoul of the world's most distinguished philologists who maintain that it is a mistaken
idea that the earliest inhabitants were forced by their ignorance to choose monosyllable
words based upon the imitation of sounds such as the barking of dogs, the crowing of cocks,
and like sources. Said Dr. Otto Jespersen of the University of Copenhagen, a world authority
on language:

We find that the ancient languages of our family, Sanskrit, Zend, etc., abound in very long
words; the further back we go, the greater the number of sesquipedalia. We have seen how
the current theory, according to which every language started with monosyllable roots, fails
at every point to account for actual facts and breaks down before the established truth of
linguistic history. 8

Dr. J. Vendryes, Professor in the University of Paris in \textit{Language, A Linguistic Introduction
to History}, has written:

Some languages have been proved to be older than others, and certain of our modern tongues
are known to us in forms more than two thousand years old. But the oldest known languages,
the ``parent languages,'' as they are sometimes called, have nothing of the primitive about
them. Differ though they may from our modern tongues, they only furnish us with an
indication of the changes which language has undergone, they do not tell us how language
originated. 9

Mr. Darwin and his advocates speak glibly of the ``numberless gradations'' all ranging from
amoeba, lamprey or other infinitismal form of life up through the apes to ``higher apes'' and
then to man. They speak as though it is a proved fact that the whale, the shark and other
inhabitants of the deep, as well as all manner of life upon the land, have shown ``numberless
gradations,'' to the final state in which man finds himself today; but when challenged to show
these gradations they failed to do so and beg the question. If there have been these
``numberless gradations,'' then on the earth the missing links connecting us with the amoeba
or worm, all the way up to the ``higher apes,'' would be filling the earth today with their kind
and it would not be necessary for our scientific brethren to search the deserts and the
mountains and the depths of the sea to find these links which have never been produced. But
when these remains are examined what do we find? A piece of a skull, a jaw bone, a tooth,
etc., and these objects found in caves or gravel pits, have been dug up from distant places
many yards apart. Yet they are brought together and constructed by the help of vivid
imaginations into ``ape-men,'' or ``missing links.'' Yet the scientists have not been agreed.
Some of these bones have been declared by experts not to belong to the same body and some
of them have belonged to lower forms of life. In the desperation to prove a wicked cause,
these deceptions have been, and are now, practiced upon the gullible and the ignorant and
thus the advocates strive to destroy faith in the word of our Creator, and destroy his work
which he has established upon the earth.

THE CONSCIENCE OF MAN

Let us now discuss another phase of this question which has been unanswered by organic
evolutionists, although Mr. Darwin and his followers have made the futile effort to overcome
the breach. I refer to the conscience of man, and his moral understanding. In the Doctrine and
Covenants, Section 84, verses 44-48, the Lord revealed to his Church the following:

For the word of the Lord is truth, and whatsoever is truth is light, and whatsoever is light is
Spirit, even the Spirit of Jesus Christ.

\textit{And the Spirit giveth light to every man that cometh into the world; and the Spirit
enlighteneth every man through the world, that hearkeneth to the voice of the Spirit.}

And every one that hearkeneth to the voice of the Spirit cometh unto God, even the Father.

And the Father teacheth him of the covenant which he has renewed and confirmed upon you,
which is confirmed upon you for your sakes, and not for your sakes only, but for the sake of
the whole world.

These words by revelation were given to the Prophet Joseph Smith and his brethren as they
were in council, April 30, 1832. They are of the greatest import and have a vital bearing on
the question which is before us, that is whether man and animals are related and have the
same ancestry. The Lord revealed the same truth to Moroni and he wrote:

For behold, the \textit{Spirit of Christ is given to every man}, that he may know good from evil;
wherefore, I show unto you the way to judge; for every thing which inviteth to do good, and
to persuade to believe in Christ, is sent forth by the power and gift of Christ; wherefore ye
may know with a perfect knowledge it is of God.

But whatsoever thing persuadeth men to do evil, and believeth not in Christ, and deny him,
and serve not God, then ye may know with a perfect knowledge it is of the devil; for after
this manner doth the devil work, for he persuadeth no man to do good, no, not one; neither do
his angels; neither do they who subject themselves unto him.

And now, my brethren, seeing that ye know the light by which ye may judge, which light is
the light of Christ, see that ye do not judge wrongfully; for with that same judgment which ye
judge ye shall also be judged. 10

This is in harmony with the words of John:

In the beginning was the Word, and the Word was with God, and the Word was God.

The same was in the beginning with God.

All things were made by him; and without him was not anything made that was made.

\textit{In him was life; and the life was the light of men.}

And the light shineth in darkness; and the darkness comprehendeth it not. . . .

That was the true Light, which lighteth every man that cometh into the world.

He was in the world, and the world was made by him, and the world knew him not. 11

Then we have the word of Jesus Christ bearing witness to this same truth:

Then spake Jesus again unto them, saying, I am the light of the world: He that followeth me
shall not walk in darkness, but shall have the light of life. 12

Here are passages of scripture from three sources, which should be believed and accepted by
every member of the Church. The first quotation from the Doctrine and Covenants, the
second from the Book of Mormon and the third, from the New Testament, the last being the
words of our Lord himself confirming the revelation to the Prophet Joseph Smith and also to
Moroni. What, now, do we learn from these scriptures? That Jesus Christ is the light of the
world, that there emanates from him and his Father a Spirit which is called the ``Spirit of
Christ''; and that \textit{every man} born into this world is endowed with this Spirit. If every man is
so endowed. then the ``lowly savage'' is endowed with some portion of this Spirit, otherwise
it is not given to \textit{every man}. These learned men, who have learned the philosophies and
wisdom of the world, which Isaiah said will perish, have the guidance, or the right to the
guidance, of this Spirit. This is not the Holy Ghost, for the Holy Ghost is \textit{not} given to every
man, but only to those who repent and are baptized for the remission of their sins and receive
the gift of the Holy Ghost by the laying on of hands by those who have the authority and are
appointed to bestow this gift. 13 These evolutionists then, who deny the divine creation of
man, nevertheless have their minds quickened by the Spirit of Christ. We read further:

This is the light of Christ. As also he [i.e. his authority] is in the sun, and the light of the sun,
and the power thereof by which it was made.

As also he is in the moon, and is the light of the moon, and the power by which it was made;

As also he is in the moon, and is the light of the moon, and the power by which it was made;

As also the light of the stars, and the power thereof by which they are made;

And the earth also, and the power thereof, even the earth upon which you stand.

And the light which shineth, which giveth you light, is through him who enlighteneth your
eyes, which is the same light that quickeneth your understandings;

Which light proceedeth forth from the presence of God to fill the immensity of space—

The light which is in all things, \textit{which giveth life to all things}, which is the law by which all
things are governed, even the power of God who sitteth upon his throne, who is in the bosom
of eternity, who is in the midst of all things. 14

Therefore these men who deny Christ receive from him the power of enlightenment which
they possess. Notwithstanding their rebellion it is by his grace that they exercise the measure
of intelligence which they have, but in their blind stubbornness they have turned against the
source of their intellectual light, for it is by the light of Christ that their understanding is
quickened. 15 Therefore they have added to their ingratitude by turning away from Jesus
Christ and endeavoring to put him to shame in their contempt for him and his great mission
for the salvation of mankind. If they would hearken to this Spirit—and it speaks to them
through their consciences—they would understand the truth which makes men free, and give
praise and honor to him who by his Spirit leads them to a correct understanding of the
workings of the Lord. However, they have rebelled and by doing so they have shown their
contempt for the very power by which their minds are quickened. When men reject this
inspiration coming from this Spirit, that inspiration is taken away, for the Lord has said: ``I,
the Lord, am angry with the wicked; I am holding my Spirit from the inhabitants of the
earth.'' 16 In other words, the full influence of the light which emanates from this Spirit is
diminished though it is not entirely withdrawn, for if this should be the case their spiritual
existence would come to an end, for it is written, ``He that hath the Son hath life; and he that
hath not the Son of God hath not life.'' 17

This great gift of ``conscience,'' which is an outward manifestation of the Spirit of Christ
given to every man, which quickens their minds and gives them intelligence and leads those
who hearken to it to the divine truth, \textit{was not given to the animal world!} The Lord does not
require of them repentance from sin, \textit{for they do not sin}. It requires intelligence and a
knowledge of right and wrong, in order for a man to sin. While animals prey on each other
there is no violation of conscience, for they have not the gift of conscience. There is no moral
obligation for a lion or a bear, or any other carniverous animal to kill and prey on other
animals, for the Lord did not give to them the light of truth, or the guidance of the Spirit of
Christ, and therefore when they trespass upon the rights of other and weaker animals and kill
and lay waste, there is no obligation or divine commandment that they should repent and
restore that which they have taken. They have no moral obligation, no understanding of right
and wrong. As Mr. Darwin has stated it, according to the rules and regulations prevailing in
the animal world, it is a matter of the ``survival of the strongest.'' So the large fish prey upon
the smaller ones; the eagle swoops down and seizes the rabbit; the hawk, the slower traveling
bird; the lion, the weaker sheep, and so on it goes; but there is no feeling of wrongdoing on
the part of any of these. To take advantage of others, to drive them away from their food and
consume it, does not enter into the mind of the large fish, the bear, lion, eagle or hawk, that
they are doing wrong. There is no moral question troubling them. They have no moral sense,
or understanding of justice, right or wrong. The thought of honesty, doing to others as you
would be done by never enters the head of the beast. You ask why? Because the Creator did
not give to him these moral commandments or make him responsible for his depredations on
others. He is not directed by the ``light of truth,'' and therefore is not morally, religiously or
intellectually, responsible for his deeds.

Even the ``lowest savages,'' of whom Mr. Darwin likes to speak, had a code of right and
wrong. It may have been very defective judged by a more enlightened view, but nevertheless
they were subject to some sort of laws by which the individual and the community were
governed. he had a conscience, must have had, for the word of the Lord is true, for ``In him
was life; and the life was the light of men. . . . That was the true light, which lighteth every
man that cometh into the world.'' 18 He may have had a code that permitted him to kill his
enemy, steal his substance, drive him from his possessions, but he was, nevertheless, subject
to and guided by some law recognized by him and his community. Can we say that he is far
different from the so-called civilized races? They have done quite the same all down through
the ages. Weak nations have had to yield to the strong, who have tried to ease their
consciences by the evolutionary doctrine of the survival of the fittest. But they, as with
individuals, have to come to the day of compensation in the end, for the day of retribution
arrives to all those who have a conscience and violate it, whether it be men or nations.

THE POWER OF SPEECH

Another distinctive difference between all members of the animal kingdom and man is the
power of spoken and written speech, expressing definite thoughts. We have said that animals
have some means of communication, but they have no spoken or written language. They
have not the power to converse by signs or leave a written history. The ``lowly savage,''
whether he lived in a hut or in a cave, learned to draw pictures and by certain marks and
designs tell at least a crude story. In regard to the origin of language more will be said at a
later time. Never was the ``heathen'' so low that he could not express himself by the use of
words, and always with human beings, whether living in what some are pleased to call a
``savage'' state or a ``civilized community'' they had a spoken language and a manner of
communication by words or signs. This is not the case in any community of ``higher apes''—
whatever is meant by that term—or fish, fowl or beast. If there were ever a time when the
``higher apes'' became savage men, with some sort of crude language, we would find these
``missing links'' living in all parts of the earth, forming crude sentences, but they are not
found!

Dr. Harold C. Morton, in an article published in the \textit{Journal of Transactions, of the
Philosophical Society of Great Britain,} April 24, 1933, has shown definitely the gap between
the animal world and humanity in relation to the moral and intellectual status of each. The
title of this article is \textit{The Supposed Evolutionary Origin of the Moral Imperative.} He states
that ``one indisputable 'urge' in human thinking has been the 'urge' to get rid of God, the
Almighty Creator.'' In the question of ``the Moral Imperative,'' as he states it, ``evolution
meets one of its 'acid tests,' and to fail here is to be discredited altogether. This tests
Evolution in the realm of Life, and that is strictly its only sphere. . . . From this non-moral
race Man is supposed to have come: and Man's moral nature is his distinctive human
attribute.''

Dr. Morton continues:

Man is man, not because he walks the world of the body, the world where mechanistic cause
and effect and physico-chemical forces abound, but because he knows himself to be a citizen
of a higher realm, the realm of the Spirit, the realm of moral values—where Right has
authority; where Obligation, not mechanical or chemical, but Moral, reigns; where he hears a
Sovereign Voice, ``Thou shalt,'' and knows that the victory and glory of life lies in obedience
to that voice. His Mind is aware that Moral Law must be obeyed because it is Moral Law and
for that reason alone. . . .

It is universal in normal humanity. However much moral ideals and moral life vary (e.g.,
some communities even praise theft, provided it is theft from enemies) the Moral Imperative
is always there. I believe it can be maintained that the great moral laws—Truth, Justice,
Honesty, Industry, Kindness, and so forth—are, and have been, universally known in normal
human life; and that any ignorance is to be attributed to the debasement of human nature,
false training, and the sway of evil ideals. Conscience, which perceives the Law, hears the
voice, feels the obligation, \textit{may} become ``seared as with a hot iron.'' Even if, with what is
called the ``New Institutionism,'' we had to admit that knowledge of detailed laws is not
universal, we still should affirm the universal sense of Moral Obligation to follow after
whatever is allowed to be ``the Good.'' In some form or other the moral fact is always there,
and generally as we know it today. How has this come to pass? How has the non-moral
``tangle of apes'' been transmuted into moral Man? Evolution has to tell us; and, if she cannot,
her cause can only be adjudged lost. 19

Dr. Morton closes his article with the following paragraph:

Thus Emergent Evolution offers no \textit{explanation} of the Moral Imperative, nor of any other
``emergent qualities.'' It simply asks us to accept without explanation, without any ``power
that works changes,'' the assumption that these qualities did emerge, and in an order which
fits in with evolutionary speculation. All this we are to accept with ``natural piety!'' Surely it
is not for us to accept with natural piety, but to reject with supernatural energy, a philosophy
which gets rid of both God and Cause in order to effect its purpose. Emergent Evolution is an
admission of the failure to show cause for the origin of the Moral Imperative; and still the
great Imperative of our Moral Life sounds forth, unexplained and unexplainable save on this
one foundation: ``And God said, Let Us make Man in Our Image, After Our likeness. 20

Mr. Douglas Dewar commenting on the lecture by Dr. Morton says:

Those who delight to give rein to their imagination, especially those who suffer from
Theophobia, have from time immemorial toyed with the idea of Evolution. The theories of
these persons never obtained general acceptance because they do not fit in with the fact that
you cannot get out of anything more than has been put into it; as Dr. Morton well says,
theories of Evolution resemble the conjurer's trick of producing the required article out of
nothing. . . . One of the many difficulties encountered by Evolutionists is that with which Dr.
Morton has so ably dealt, viz., the origin of the Moral Imperative. Practical men, as opposed
to mere theorists, attempt to discover in the lower animals the rudiments of this, and to show
how this characteristic has developed to its present condition in Man. Such assert that any
character tending to the preservation, vitality or happiness of a tribe or herd will tend to be
preserved and passed on to subsequent generations and gradually become amplified until we
arrive at the Moral Imperative. Dr. Morton has shown that this line of argument has met with
little success. Sir Arthur Thompson tacitly admits this in his article entitled ``Evolutionary
Ethics'' in the latest edition of the \textit{Encyclopedia Britanica.} 21

Mr. George Brewer adds this comment:

Dr. Morton has, I think, shown us that the doctrine of the Evolutionary origin of the Moral
Imperative has not only no foundation in fact, but is contrary to history and experience. Like
similar teaching in connection with the organic and inorganic realms of nature, it is based on
assumption, buttressed by speculation, and built up from fragments of human imagination.

According to Professor Alexander, man, evolved from protoplasm through a series of lower
animals, will eventually emerge into Deity; so that, in place of the simple revelation given to
us in His Word ``that God made man in His own image,'' we are asked to accept with \textit{``natural
piety''} the impious proposition that man is making God. That principle of Moral
Consciousness implanted in Man by God Himself, which even the corruption consequent
upon the Fall has failed to obliterate, and which we call Conscience, is in evidence
throughout the ages, and is certified by the Apostle Paul in his epistle to Romans (ch. 2:14,
15): ``For when the Gentiles, who have not the law, do by nature the things contained in the
law, these, having not the law, are a law unto themselves: which show the work of the law
written in their hearts, their conscience also bearing witness, and their thoughts the
meanwhile accusing or else excusing one another.'' . . . The great crises of life arise when this
Moral Imperative called Conscience, issues one command, and self-interest, passion, or some
outside authority issues another, and the individual has to decide which command is to be
obeyed. What Conscience commands may be apparently against our material interests,
contrary to our inclination, opposed by the advice of friends and popular judgment, and may
even be contrary to the decrees of the ruling power; yet it refused to withdraw, or modify its
claim.

The Utilitarian and Emergent theories, put forward to support the cause of Evolution, fail
entirely to account for Conscience, for history records that men have, at the dictates of this
moral force, chosen to act contrary to self-interest and inclination, and even to suffer torture
and death rather than violate the judgment of Conscience. Further, when the human will is
called upon to decide upon one of two courses in which a moral principle is involved, the
individual becomes conscious, whether he professes to believe it or not, of his obligation to a
Supreme Being, to whom he will be answerable, having power to approve a right decision
and to inflict punishment for a wrong one. 22

Dr. Friedrich Paulsen, formerly professor of philosophy in the University of Berlin, in his
\textit{Introduction to Philosophy}, and who was inclined towards the theory of evolution, comes to
our assistance in the discussion of the moral urge in man as a quality peculiar to mankind. I
quote from his work, page 69:

That mind is to mind the object of greatest interest in the world is clearly shown by the
division of scientific labor into the two spheres of reality, nature and history. If we were to
banish from our libraries everything that pertains to the mental life of man, everything that
belongs to history and philology, politics and morals, theology and philosophy, sociology
and jurisprudence, medicine and technics, we should have left a very modest remainder. Or,
suppose we should strike out of our large voluminous encyclopedias and lexica the same
subjects, retaining only what pertains to astronomy and physics, chemistry and mineralogy;
the remainder would fill a small thin volume. And this will most likely always be the case.
The human mind will ever regard the human mind as the most important object of reality.

My second remark will consider the question concerning the consequences of materialism for
morality and mode of life. The view is widely circulated that materialism has consequences
dangerous to morality. In destroying religion, it is held, it also destroys morality and faith in
ideals; its practical conclusion is: Virtue is an empty dream, conscience a freak, and the
moral law the invention of priests; true wisdom consists in enjoying life and getting what we
can get.

I do not believe that this view can be accepted, at least not in the form in which it is stated. A
man's conduct is not determined by his metaphysical ideas concerning the nature of reality,
but essentially by natural impulses, temperament, education and condition of life. If,
however, there is any connection between theoretical and what is called practical
materialism, it is brought about, not because a man's metaphysics determines his life, but
because his life determines his metaphysics. An empty and low life has the immediate
tendency to produce a nihilistic conception of life; its features are a low estimate of life and
its destiny, a depreciation and scorn of the nobler phases of man's nature, a loss of reverence
for moral and spiritual greatness, disbelief and derision as regards all ideal aspirations. And
such a nihilistic view of life naturally tends to a materialistic philosophy. It will welcome the
``results of science'' that nature as well as history is the play of meaningless chance, the blind
forces combine atoms and carelessly scatter them again at the next instant. Conversely, an
active and honorable, a good and great life naturally tends to an idealistic metaphysics; it is
exalted and pacified by a view that represents its highest aims and ideals as the underlying
forces of reality. From the striving after great ends grows the belief in the supremacy of
ideas, in the governance of Providence in the historical life of man, and this belief finds a
theoretical basis in the thought that reality as a whole is founded on ideas, that the world is
the work of God. 23

Alma, one of the ancient Nephite prophets, tried to teach a wayward son the futility of
seeking pleasure in sinful practices, and his labors were not in vain. He taught the youth that
\textit{wickedness never was happiness} and the Lord had decreed certain fundamental principles
which all should follow, for there would come a retribution and there would be restored to
every man according to his works. In this discourse he said that this plan of restoration was
requisite with the justice of God and if the works of men were evil then evil would be
restored to them again, for the ``decrees of God are unalterable; therefore, the way is prepared
that whosoever will may walk therein and be saved.'' Moreover ``that restoration is to bring
back again evil for evil, or carnal for carnal, or devilish for devilish—good for that which is
good; righteous for that which is righteous; just for that which is just; merciful for that which
is merciful. . . . For that which ye do send out shall return unto you again, and be restored;
therefore the word restoration more fully condemneth the sinner, and justifieth him not at
all.'' 24

How foolish it would be to teach that the beasts of the field would be subject to such law?
The Almighty has not placed them under the law to observe the golden rule. They do not
have the capacity of intellect nor the power of understanding; the moral code which has been
given to man, and this great gulf which separated them from human beings has always
existed and has never been changed. But of man, whether he believes in God or is opposed to
him, the edict has gone forth that he shall be true to the law of chastity, of truth, and the
practice of righteous principles; and none can escape the consequences of his acts. Never has
there been a man who has not had a conscience. It may be true that it has been seared, for the
impressions that come to every man are given him of God, and the Spirit of the Almighty
will not always strive with rebellious men. Nevertheless every man will pay the price of his
wrong-doing for he has been endowed with the guiding Spirit and with the power of
understanding to know what is just and true, and if he rebels he cannot plead lack of
understanding of the divine law. His conscience is his warning signal against sin and it is the
Spirit of the Lord which directs him until he loses all sense of divine justice and obligation
and obedience to eternal law through his sins.

We live in a day when many philosophies and hypotheses are taught in the world. The
hypothesis of organic evolution is one of the most cunningly devised among the fables. It
strikes at the soul of man. It denies his divine origin as a child of God, as clearly declared by
Paul to the Greeks; and pronounces the eternal death of all living creatures and their
assignment to everlasting oblivion. It proclaims to all who accept it that there are no rewards
or punishments after death. It encourages the gratification of every urge and passion on the
theory that there can come no punishment for sin. In fact, as stated by Sir Oliver Lodge, those
who accept this theory are not worrying about their sins at all. This hypothesis teaches that
Mercy is a fallacy, Justice a dream, and there can come no retribution or punishment for
crime after death intervenes.

Organic evolution mocks at retributive justice. Its philosophy is diametrically opposed to that
proclaimed by Alma (Alma Ch. 41), and Ralph Waldo Emerson who said:

The dice of God are always loaded. The world looks like a multiplication table, or a
mathematical equation, which turn it how you will, balances itself. Take what figure you
will, its exact value, nor more nor less, still returns to you. Every secret is told, every virtue
rewarded, every wrong redressed, in silence and certainty. What we call retribution is the
universal necessity by which the whole appears wherever a part appears. If you see smoke,
there must be fire. If you see a hand or a limb you know that the trunk to which it belongs is
there behind. 25

And the rebellious shall be pierced with much sorrow; for their iniquities shall be spoken
upon the housetops, and their secret acts shall be revealed. (D. \& C. 1:3.)
Dr. Friedrich Paulsen has also treated this phase of the matter of conscience. I quote as
follows:

Whoever disregards the laws of statics will see his structure fall to ruin, he may think of these
laws what he will. Whoever transgresses the laws of medicinal dietetics will pay the penalty
with indisposition and disease, whether he believes in the validity of these laws or not.
Similarly, whoever violates the laws of morality will pay for it with his own life's happiness,
regardless of what he may think of them. Whoever disregards the duties which he owes
himself, whoever abandons himself to intemperance and dissipation, destroys the
fundamental conditions of his own welfare. Whoever surrenders himself to idleness and love
of pleasure, expecting in this way to find his happiness, will ultimately perish in satiety and
disgust; that is a biological law of human nature as well as the other law that successful
activity is followed by pleasure, and that capacities grow through exercise. Finally, whoever
disobeys the commands of social morality disturbs the life of others, and suffers for it himself
as a social being. Whoever treats his surroundings inconsiderately, haughtily, and meanly,
arouses aversion and hatred and the behavior corresponding to these feelings, his views
concerning the nature of moral laws to the contrary notwithstanding. No one exists, however,
to whom these things are altogether indifferent; there is not a man in the world who can do
without the love and confidence of his fellows, to whom distrust and hatred are not painful in
themselves and destructive in their consequences. And even if anyone should succeed in
perpetrating wrong and baseness, undiscovered and with impunity, he could not escape the
reactions: the fear of discovery would remain. For it is a strange fact that the man who has
something to conceal always believes himself to be watched and seen by others.
Consciousness of guilt makes a man lonely. And should anyone succeed in shaking off all
relations with others, he would not be secure against one—the judge in his own heart.
Blinded by passion, he may momentarily delude himself into the belief that he has torn out
his conscience by the very roots; it will come again some day and audibly speak to him.
When the passionate desire is satisfied, when recollection and reflection reawake, or when,
with increasing age, strength and courage fail, then the image of past deeds arises before the
soul and causes anxiety. There is perhaps no man who could look back upon a life full of
emptiness and baseness, full of falsehood and cowardice, full of wickedness and depravity,
with feelings of satisfaction. At any rate, it would not be advisable for anyone to make the
trial. The lives of so-called men of the world and their female partners, or of blacklegs and
scoundrels, little and big ones, are not apt to be described at length and openly either by
themselves or others. Should it be done, and perhaps it would not be a useless task, it is not
likely that anyone would lay aside the book with the feeling: that was a happy and enviable
life. And if such a life had achieved an apparent success, if it had committed everything and
enjoyed everything with impunity, nevertheless it would not easily strike an observer as a
beautiful and desirable lot.

Hence, as long as the world is what it is and human nature remains what it has been, the
moral laws will remain in force, whether we conceive reality as composed of atoms or
immaterial substances or what not. 26

Our Lord declared in one of his parables, when speaking of the wicked, ``So shall it be at the
end of the world: the angels shall come forth, and sever the wicked from amongst the just,
And shall cast them into the furnace of fire: there shall be wailing and gnashing of teeth.'' 27
This is to be interpreted that in that day the wicked shall feel the torment of their sins and the
torment likened to fire is the torment of their minds. The most wicked man of all is the one
who destroys faith in God and the keeping of his commandments, thus turning souls from the
truth to partake of this eternal torment. Now we realize that the judgment the conscience will
inflict will be confined to the human family. It cannot be applied to the beasts of the field, the
fowl of the air or the fishes of the waters. They do not understand the Moral Commandment
and the quality of right and wrong, justice and mercy. They never did and they never will
while mortality endures. The Moral Commandment is another stumbling block in the path of
organized evolution, a chasm that has never been crossed by any in the animal kingdom. It
separated man from the beast by divine decree and it is purely imaginary thinking and wistful
wishing that causes men to seek for such a bridging of the gulf.

We now come to another vital point in the discussion—the inseparable gap separating man
from all other living creatures. This is the gift of intelligence which he has inherited as the
offspring of God, and by which he may advance from a small degree to a higher
intellectually. If faithful to the commandments given by our Eternal Father, and he endure in
that faithfulness to the end of his mortal life, he will become a son of God and sit upon a
throne. 28 This is in accordance with the eternal plan prepared before the foundation of this
earth was laid. 29 Every human infant comes into the world more helpless and dependent
than the offspring of any other creature. If left alone following birth in a few hours he would
die. For the first year he is outstripped by every other newly born creature. When that length
of time has elapsed most creatures in the animal kingdom are ready to take care of
themselves, but their advancement after they are weaned comes to an end. They have reached
the full maturity, with very few exceptions. Their parents have forsaken them; the filial love
has come to an end. In that length of time the offspring are on their own. For the first year the
human child is still fully dependent on its parents, but it has accomplished great things. It has
learned to walk, but the animal learned to do that many months ago, even from the day, or
shortly after the day, it was born. The human child has learned to express himself with a few
common words and phrases. This the animal has not learned to do. This the parents before it
had not learned to do, and so on as far back as the knowledge of man can go. The gift of
language is inherited by the offspring of God and has been denied the animal kingdom. This
bridge they have not crossed.

When the child is about five or six years old it is ready for school and from that point on for
fifteen or more years, he may graduate with honors and degrees. He may become a noted
engineer, architect, scientist, or skilled in some other field. He may be looked upon as being a
master in his calling, having through the keenness of his intellect—and the help of the Spirit
of the Lord—outstripped all who have gone before him. Thus he becomes a benefactor of the
race, an inventor, a discoverer of natures secrets; a builder of towering structures, monuments
and machinery. He may have made it possible to counteract and overcome disease, and then
again he may waste his life in idleness, or seek unrighteous power and dominion through
greedy ambition. He may be an instructor, a religious teacher endeavoring to create faith in
the hearts of his fellows—a prophet of God endowed with power from on high, or he may be
one who delights in destroying faith in God, who denies the source of his intelligence.
However it may be, and no matter what course he may pursue, the fact remains that his
powers and intelligence which he manifests, are the gift to him from God. This he may deny
and boast in his own strength to his eternal damnation. In all of these gifts and graces man
stands out preeminently. These qualities have never been given to the beast of the field or the
forest, and create a mark of separation between the animal world and the human family that
has not and cannot be crossed.

Sir Ambrose Fleming, a gentleman of the highest integrity and accomplishments, a scientist
and mathematician of outstanding ability, in his book, \textit{Evolution or Creation}, has given much
food for thought on this question separating man from the animal kingdom. In his chapter,
\textit{The Failure of Evolution to Account for Life, Mind, and Man,} he has said:

Without aspiring to supply any definition in detail, we can note at once certain qualities in
the human species not the smallest trace of which appear in the animal species. Thus no
animal has ever made any weapon or tool to help its bodily endowments. It fights with teeth
and claws, horns, tusks, or hoofs, but it makes no military weapon of any kind. Nor has any
animal made a tool—spade, rake, knife, hatchet, axe, or saw. No animal makes itself any
artificial dress, hat or coat, shoes or ornament, to improve its appearance, nor does it dress or
arrange the hair on its head. But all the very earliest true human beings do these things. No
animal has discovered how to produce fire, or even to maintain it.

The explorer, Du Chaillu, says he has seen monkeys sitting round a dying fire left by a hunter
in a forest and warming their paws, but they have not sufficient intelligence to put sticks on
the fire to keep it alive.

The animal mind or intellect is static, or limited. It never progresses beyond a certain point.
Domestic animals which have been in contact with man for thousands of years are no further
forward intellectually than at the beginning.

On the other hand, the human mind is extremely progressive, self-educative, and
assimilative. Uncultured races of men brought in contact with more advanced races adopt
quickly their achievements, customs, modes of thought, and habits, and unfortunately also
their vices. Animals undoubtedly can communicate with each other, conveying information,
but they have not developed the powers of speech or rational thought to anything even
remotely approaching that in the case of man. 30

\newpage
REFERENCES—CHAPTER NINE

Footnotes

1. Darwin, Charles, \textit{The Descent of Man}, ed. 1897, p. 65.

2. Ibid., p. 164.

3. D. \& C. 77:1-2.

4. Moses 3:5-9.

5. Flammarion, Camille, \textit{The Unknown}, pp. 15-17.

6. Darwin, Charles, \textit{The Descent of Man}, pp. 143-144.

7. Pearl of Great Price 5:13.

8. Jespersen, Dr. Otto, \textit{Language, Its Nature, Development and Origin,} p. 420.

9. Vendryes, Dr. J., Language, \textit{A Linguistic Introduction,} p. 5.

10. Moroni 7:16-18.

11. John 1:1-5.

12. \textit{Ibid}., 8:12.

13. \textit{Ibid}., 14:17.

14. D. \& C. 88:7-13.

15. \textit{Ibid}., 88:12-13.

16. \textit{Ibid}., 63:32.

17. 1 John 5:11-12.

18. John 1:4, 9.

19. Morton, Dr. Harold C., \textit{The Moral Imperative}, Victoria Transactions 1933, pp. 149-153.

20. \textit{Ibid}., p. 164.

21. \textit{Ibid}., p. 166.

22. \textit{Ibid}., pp. 168-169.

23. Paulsen, Dr. Friedrich, \textit{Introduction to Philosophy}, pp. 69-70.24. Alma Ch. 41.

25. Emerson, Ralph Waldo, \textit{Essay on Compensation}.

26. Paulsen, Dr. Friedrich, \textit{Introduction to Philosophy}, pp. 72-73.

27. Matt. 13:49-50.

28. Rev. 21:7; John 10:34-36; D. \& C. 76:53-60; 121:29.

29. Abraham 3:22-26.

30. Fleming, Sir Ambrose, \textit{Evolution or Creation}, pp. 72-73.

