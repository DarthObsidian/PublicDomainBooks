\chapter{THE EARTH'S TEMPORAL EXISTENCE}

"FOR my thoughts are not your thoughts, neither are your ways my ways, saith the Lord.

"For as the heavens are higher than the earth, so are my ways higher than your ways, and my
thoughts than your thoughts." 1

So said the Lord through Isaiah to faltering and wayward Israel. The conditions and attitude
of mankind today are no different from those of Israel in the days of Isaiah. The wise men of
the earth still boast in their own strength. We have reached an age when the wisdom of God
is no longer needed in the scientific world, or, for that matter, hardly anywhere else. Through
their research and from their deductions many of the learned attempt to tell us when and how
this earth came into existence billions of years ago; how long, presumably, it will last before
it will wear out and become cold and useless. This, however, is not according to the wisdom
and thinking of the Lord. He has other plans and has revealed them to his servants, the
prophets. This earth is destined to follow the course of the millions of other earths which
have gone on before it to fill their final destiny. They were created as this earth was created,
passed through a fall and a temporal existence, just as our earth has fallen and is now passing
through a temporal existence, and have gone on to their appointed places to endure forever as
the habitations of the resurrected children of God. To accomplish this he has said worlds are
created, and it is his work and his glory "to bring to pass the immortality and eternal life of
man." 2

When this earth on which we dwell was created it was according to the eternal plan of the
Father, and likewise in accordance with his reckoning. This he has definitely declared to the
knowledge and understanding of all those who are, through their faith, willing to accept what
he has given by revelation to his prophets. Abraham was among other things an astronomer.
He knew more about the heavens, when they were created, and the purpose of their creation,
than all the astronomers in the world today put together. The Lord revealed these things to
him by Urim and Thummim and also talked to him, "face to face, as one man talketh with
another," and told him "of the works which his hands had made," and they were multiplied
before Abraham's eyes and he could not see the end thereof. 3

The Lord made known to him the following facts: That Kolob is the first creation, and is
nearest to the celestial, or the residence of God. It is the first in government, the last
pertaining to the measurement of time. This measurement is according to celestial time. One
day in Kolob is equal to a thousand years according to the measurement of this earth, which
by the Egyptians was called Jah-oh-eh. Oliblish, so called by the Egyptians, stands next to
Kolob in the grand governing creation near the celestial, or place where God resides. This
great star is also a governing star and is equal to Kolob in its revolutions and in its measuring
of time. 4 Other grand governing stars were also revealed to Abraham, and the Lord said to
him:

These are the governing ones; and the name of the great one is Kolob, because it is near unto
me, for I am the Lord thy God: I have set this one to govern all those which belong to the
same order as that upon which thou standest.

And the Lord said unto me, by the Urim and Thummim, that Kolob was after the manner of
the Lord, according to its times and seasons in the revolutions thereof; that one revolution
was a day unto the Lord, after his manner of reckoning, it being one thousand years
according to the time appointed unto that whereon thou standest. This is the reckoning of the
Lord's time, according to the reckoning of Kolob.

And the Lord said unto me: The planet which is the lesser light, lesser than that which is to
rule the day, even the night, is above or greater than that upon which thou standest in point of
reckoning, for it moveth in order more slow; this is in order because it standeth above the
earth upon which thou standest, therefore the reckoning of its time is not so many as to its
number of days, and of months, and of years.

And the Lord said unto me: Now, Abraham, these two facts exist, behold thine eyes see it; it
is given unto thee to know the times of reckoning, and the set time, yea, the set time of the
earth upon which thou standest, and the set time of the greater light which is set to rule the
day, and the set time of the lesser light which is set to rule the night.

Now the set time of the lesser light is a longer time as to its reckoning than the reckoning of
the time of the earth upon which thou standest.

And where these two facts exist, there shall be another fact above them, that is, there shall be
another planet whose reckoning of time shall be longer still;

And thus there shall be the reckoning of the time of one planet above another, until thou
come nigh unto Kolob, which Kolob is after the reckoning of the Lord's time; which Kolob is
set nigh unto the throne of God, to govern all those planets which belong to the same order as
that upon which thou standest.

And it is given unto thee to know the set time of all the stars that are set to give light, until
thou come near unto the throne of God. 5

From this revelation given to Abraham in relation to the heavenly bodies, we have
discovered that the governing star of the universe is Kolob, "the first creation," and the Lord's
time is the time of Kolob, "which is celestial time."

We also learn that Peter understood this fact when he said: "But, beloved, be not ignorant of
this one thing, that one day is with the Lord as a thousand years, and a thousand years as one
day." 6 Here is the information which throws light upon the days of creation, and again the
Lord revealed to Abraham that the creation was according to God's time, and he wrote
regarding the commandment given to Adam:

But of the tree of knowledge of good and evil, thou shalt not eat of it; for in the time that thou
eatest thereof, thou shalt surely die. Now I, Abraham, saw that it was after the Lord's time,
which was after the time of Kolob; for as yet the Gods had not appointed unto Adam his
reckoning. 7

Here again we have the information that the creation of this earth was according to Kolob's
time, which is celestial time.

We learn from the Book of Genesis and also from the Book of Moses in the Pearl of Great
Price, that when the earth and its heavens were finished, the Father declared that "all things
that I had made were very good," and on the seventh day, he ended his work. If the work
when finished was "very good," then in it was no imperfection, and it, like man, and "all
things which were created must have remained in the same state in which they were after
they were created; and they must have remained forever, and had no end," if Adam had not
fallen. 8

When Adam fell, the earth and all things upon it partook of the fall, and were henceforth
subject to mortal, or temporal, conditions. The Lord said to Adam, "Because thou hast
hearkened unto the voice of thy wife, and hast eaten of the tree, of which I commanded thee,
saying, Thou shalt not eat of it: cursed is the ground for thy sake; in sorrow shalt thou eat of
it all the days of thy life"; 9 and so the earth became suited to Adam's condition and became a
temporal earth, or subject to all the conditions of mortality and death. After it has filled the
measure of its temporal existence it will die and since it and all creatures upon it have been
redeemed through the blood of Jesus Christ, it will rise again, receiving the resurrection and
will become a glorious celestial habitation for the righteous. 10

Early in March 1832, while the Prophet Joseph Smith was revising the scriptures by
revelation, the Lord gave him answers concerning certain things in the Revelation of John.
Among these questions and answers are the following:

6. Q. What are we to understand by the book which John saw, which was sealed on the back
with seven seals?

A. We are to understand that it contains the revealed will, mysteries, and works of God; the
hidden things of his economy concerning this earth \textit{during the seven thousand years of its
continuance, or its temporal existence.} (My italics.)

Question Twelve and answer are as follows:

Q. What are we to understand by the sounding of the trumpets, mentioned in the 8th chapter
of Revelation?

A. We are to understand that as God made the world in six days, and on the seventh day he
finished his work, and sanctified it, and also formed man out of the dust of the earth, even so,
in the beginning of the seventh thousand years will the Lord God sanctify the earth, and
complete the salvation of man, and judge all things, and shall redeem all things, except that
which he hath not put into his power, when he shall have sealed all things, unto the end of all
things; and the sounding of the trumpets of the seven angels are the preparing and finishing
of his work, in the beginning of the seventh thousand years—the preparing of the way before
the time of his coming. 11

This revelation confirms the fact that the days of creation were celestial days, and this earth
is passing through \textit{one week of temporal (mortal) existence}, after which it will die and
receive its resurrection.

The vision to John of the opening of the seven seals is extremely interesting when we get this
understanding. This vision is recorded in the Book of Revelation in chapters five to ten, and
is confirmed in the Doctrine and Covenants, Section 88, verses 92 to 114. Each angel shall
sound his trumpet and reveal the acts of men during each of the six thousand years, or six
days of the temporal existence of the earth, and the seventh angel, who is Michael, will
gather his armies and the devil will gather his armies, "And then will come the battle of the
great God; and the devil and his armies shall be cast away into their own place, that they
shall not have power over the saints any more at all."

Here is a very interesting statement by the Prophet Joseph Smith which has a bearing on this
important subject:

". . . the Lord shall be King over the whole earth," and "Jerusalem his throne." "The law shall
go forth from Zion, and the word of the Lord from Jerusalem."

This is the only thing that can bring about the "restitution of all things spoken of by all the
holy prophets since the world was" ". . . the dispensation of the fulness of times, when God
shall gather together all things in one." Other attempts to promote universal peace and
happiness in the human family have proved abortive; every effort has failed; every plan and
design has fallen to the ground; it needs the wisdom of God, the intelligence of God, and the
power of God to accomplish this. \textit{The world has had a fair trial for six thousand years; the
Lord will try the seventh thousand himself.} (My italics.) "He whose right it is, will possess
the kingdom, and reign until he has put all things under his feet;" iniquity will hide its hoary
head, Satan will be bound, and the works of darkness destroyed; righteousness will be put to
the line, and judgment to the plummet, and "he that fears the Lord will alone be exalted in
that day." To bring about this state of things, there must of necessity be great confusion
among the nations of the earth; "distress of nations with perplexity." Am I asked what is the
cause of the present distress? I would answer: "Shall there be evil in a city and the Lord hath
not done it?"

The earth is groaning under corruption, oppression, tyranny and bloodshed; and God is
coming out of his hiding place, as he said he would do, to vex the nations of the earth.
Daniel, in his vision, saw convulsion upon convulsion; he "beheld till the thrones were cast
down, and the Ancient of Days did sit;" and one was brought before him like unto the Son of
Man; and all nations, kindred, tongues, and peoples, did serve and obey him. It is for us to be
righteous, that we may be wise and understand; for none of the wicked shall understand; but
the wise shall understand, and they that turn many to righteousness shall shine as the stars for
ever and ever. 12

\newpage
REFERENCES—CHAPTER TWENTY-FOUR

Footnotes

1. Isaiah 55:8-9.

2. Moses 1:37-39.

3. Abraham 3:1-10.

4. \textit{Ibid.} of cut, Book of Abraham.

5. \textit{Ibid.}, 3:3-10.

6. 2 Peter 3:8.

7. Abraham 5:13.

8. 2 Nephi 2:22.

9. Gen. 3:17; Moses 4:23.

10. D. \& C. 88:15-28.

11. \textit{Ibid.}, 77:6, 12; D. H. C., Vol. 1, pp. 253-4.

12. \textit{Teachings of the Prophet Joseph Smith}, pp. 252-253.

