\begin{flushleft}
\fontsize{35}{35}\sffamily
\textbf{Man, His Origin and Destiny}
\Large{by Joseph Fielding Smith}
\end{flushleft}

\section{FOREWORD}
\thispagestyle{empty}

Conflicting attitudes expressed concerning science and religion have confused many people.
Especially has this been true in the class room where hypotheses have been set forth
erroneously as facts and where deductions made from those theories have been regarded as
established truth.

Many of the followers of Darwin, for instance, carried his views to the extreme of
materialistic atheism, declaring not only that creation occurred without the aid of any
Intelligent Creator, but that as a matter of fact, no such Being even exists.
Both science and religion have suffered as a result. The greatest damage, however, has been
among students who have lost their faith in God through accepting these man-made theories
as facts.

But time changes things. Whereas for years atheistic deductions were made from scientific
research, now true scientists, armed with what they term "the new knowledge," are revising
their "hasty first conclusions" as Sir James Jeans expressed it, and have discovered "evidence
of a designing or controlling power that has something in common with our individual
minds."

The present day attitude of top scientists was expressed recently by Dr. Joseph W. Barker,
president and chairman of the Research Corporation of America, and formerly dean of the
engineering school at Columbia University, in an address at Ripon University. He explained
there that scientists of the nineteenth century were misled by certain of their observations,
and as a result came to conclusions which were definitely atheistic.

"But now," said Dr. Barker, "even the most pragmatic materialist, in the face of present day
scientific knowledge, is led to the inevitable conclusion that the heavens declare the glory of
God and the firmament showeth his handiwork."

Dr. Barker's concluding remarks to the students were: "As the children of Israel foreswore
the worship of the golden calf and returned to the faith of Jehovah, so have we foresworn the
crass mechanistic materialism and returned to that faith in God of which the Psalmist of old
sang. The Earth is the Lord's and all that therein is."

Knowing the great need to provide Latter-day Saint students of science with material which
would help them to preserve their faith and coordinate in their minds the pure truth of both
science and revelation, some of us have hoped for a book which could make the facts readily
available to them.

Many have recognized in President Joseph Fielding Smith of the Council of the Twelve the
profound student of scripture which he is, but not so many were acquainted with the fact that
he also is a deep student of science, widely read in various phases of the subject.

Recognizing his possession of this superb knowledge of both science and religion, some of
us urged him to write a book on the creation of the world and the origin of man, setting forth
both the up-to-date views of science, and the facts provided through revelation.

The present volume is the result. It is a most remarkable presentation of material from both
sources under discussion. It will fill a great need in the Church, and will be particularly
invaluable to students who have become confused by the misapplication of information
derived from scientific experimentation.

It will be an outstanding addition to a list of this author's books which already have stabilized
the faith of countless thousands the world around.

\vspace{\onelineskip}
-MARK E. PETERSEN.

\newpage
\section{AN INTRODUCTION}

\begin{flushleft}
BY DR. MELVIN A. COOK
\end{flushleft}

Theory plays an important role in all arts and sciences (1) by providing a means for the
unification and classification of available knowledge, and (2) by suggesting and prescribing
the design of experimental studies that will broaden the scope of knowledge. Failure to
accomplish either of these objectives necessitates modifications in the theory or substitution
of an alternate one. For this reason the basic concepts are continually undergoing change in a
healthy and forward-moving science. We are living in a world of great endeavor and
achievement in which the scientific or objective application of theory, whether true or simply
the best that can be devised to represent as faithfully as possible all known facts, has an
important place. Unfortunately, owing to the strong desire of scientists to display their
brilliance and ingenuity, there is a tendency for theory to become the objective instead of a
means to the end. Theory then not only loses its real value, but actually becomes a stumbling
block to progress. Its inventor and disciples become so engrossed in the theory that they lose
sight of its fundamental purpose, the quest for truth. This condition was shockingly
illustrated in my presence at a meeting of scientists when one of great renown met a factual
objection with the statement, "I am more concerned with the elegance of the theory than the
truth of it."

One need not look far into science to discover it consists too generally of a maze of facts and
theory so closely interwoven that even the most learned and honorable scientist (to say
nothing of the intellectually dishonest one or the novice) may have difficulty in
distinguishing readily between truth and theory. While this weakness of science is serious
enough in fields which are not closely related to the primary purposes of mortality, in the
fields more closely related, the difficulties of discerning fact and theory may well prove
disastrous. This is particularly true as regards the development of spirituality in those who
place science foremost.

The principles of the Gospel of Jesus Christ provide faithful members of the Church with
wonderful and inspiring principles of truth directly applicable in distinguishing between
fundamental truth and error in all fields of arts and science. This application requires a clear
recognition of the pre-eminence of the gospel and its "eternal scientists" of which the author
of this book stands high among the great ones in mortality. The paramount key to this
important application of "eternal science" is that every principle of the baser sciences must
square with the revealed truths.

Few fields of science come into such direct conflict with the revealed scriptures as the
palaeo-sciences—historical geology, palaeoethnology, paleontology, and palaeogeography.
The factual or experimental components of these sciences have contributed much to our
knowledge and culture and their scientists are indispensable in practical applications dealing
with the structural and dynamic features of the earth's crust, the discovery of valuable
minerals and the evaluation of natural resources, and description and classification of plants
and animals. With the author of this book, I believe that much of the theoretical structure of
these sciences is incorrect because it is not only in disagreement with the scriptures but is in
direct opposition to them. Moreover, I believe that when these sciences are denuded of their
theoretical superstructure, they are not found to conflict with the revealed truths of thescriptures.
For those who have the patience to await the great event, when the final chapters
of theory in these and other sciences are written, I am confident that they also will square
with the pre-eminent science of our Savior. The great challenge thus confronts the scientist
with faith in divine revelation to attempt each in his own field to write his theories to include
not only the facts of direct experimental observation but also those generally more significant
ones revealed by the Omnipotent Scientist, the Creator of the world and Savior of mankind.

As one frequently confronted with questions from perplexed students of the sciences, I am
deeply grateful for this documental and scientifically accurate volume to which one may turn
for answers to technical questions as well as for inspiration to continue steadfast in the
gospel. This study reveals its author to be one well versed in the scientific method and a
strong supporter of true science and those scientists who apply theory and observation
objectively in the search for truth and toward creative contributions to civilization. If he
seems impatient toward those whose objective is elegance in the manipulation of theory
rather than the discovery of truth, it is because of his deep love for mankind and a passion to
see him on the way to eternal life.

\vspace{\onelineskip}
MELVIN A. COOK

\vspace{\onelineskip}
Professor of Metallurgy

\vspace{\onelineskip}
University of Utah

\newpage
\section{PREFACE}
The following pages are the result of many months of reflection and conviction that
something should be written to strengthen the faith of some weak members of the Church,
and our students in the public schools and colleges, who are constantly exposed to the
theories of organic evolution and the higher criticism, so-called.

These hypotheses are not confined to the schools, for they find their way into the press and
current magazines expressed with a finality as though they had been definitely proved. They
are but guesses. They can never be more than guesses, for they lie beyond the possibility of
proof. Moreover, being in conflict with the revelations of the Lord to his servants the
prophets, and the teachings of our Redeemer, they are ever destructive of faith.

It has been my wish for several years that something might be done to counteract these false
teachings, so destructive of faith in God. I have mentioned this many times to my associates
and it is with their constant urging that I have undertaken this work.

To Elders Mark E. Petersen, Marion G. Romney of the Council of the Twelve; Elders Milton
R. Hunter and Bruce R. McConkie of the First Council of Seventy, I am deeply indebted for
the encouragement and help which they have given. Equally am I indebted to Dr. Melvin A.
Cook and Elder A. Wm. Lund, assistant Church Historian, for their assistance and their
valuable suggestions. Nor must I forget the aid of my secretary, Mrs. Rubie McKinlay
Egbert, and my wife, Jessie Evans Smith, for the typing and reading of the proof, and my
son, Joseph Fielding Smith Jr., who set the type and offered many helpful suggestions.

\vspace{\onelineskip}
—JOSEPH FIELDING SMITH.

\newpage
\section{ACKNOWLEDGMENTS}
Sincere appreciation and thanks are given to the following publishers for the privilege
granted to use quotations from the copyright works here listed which have been of great
assistance in the publication of this work.

The Victoria Institute or Philosophical Society of Great Britain, for numerous quotations
from several volumes of the \textit{Journal of Transactions}, including the entire lecture of Dr.
Albert Fleischmann, professor of Zoology and Comparative Anatomy in the University of
Erlangen, Germany, Volume 65, for the year 1933.

Augsburg Publishing House, Minneapolis: \textit{After Its Kind} and \textit{The Deluge Story in Stone}, by
Byron C. Nelson.

Pacific Press Publishing Association, Mountain View, California: \textit{The New Geology}, by
Professor George McCready Price.

The Devin-Adair Company, New York: \textit{God—Or Gorilla}, by Alfred Watterson McCann.

Funk \& Wagnalls Company, New York and London: \textit{The New Archaeological Discoveries},
Dr. Camden M. Cobern.

William Heinemann Ltd., London: \textit{The Accuracy of the Bible}, Dr. A. S. Yahuda.

Fleming H. Revell Company, London and Edinburgh: \textit{New Bible Evidences}, Sir Charles
Marston.

The following works, in addition to those mentioned above, will be of great benefit to any
who are confused by the hypothesis of organic evolution:

\textit{The Mammoth and the Flood; The Glacial Nightmare and the Flood}, (two volumes); \textit{Ice or
Water}, (two volumes), by Sir Henry Howorth.

\textit{The Phantom of Organic Evolution; The Geological Hoax}, by Professor George McCready
Price.

\textit{The Origin of Mankind; Evolution or Creation}, Sir Ambrose Fleming.

\newpage
\section{INTRODUCTION}

FOR a long time I have wished that someone more capable than I would write a defense of
the fundamental principles of the Gospel for the benefit of our youth who are confronted in
their studies in high schools and universities with the modern theories of so-called science
and philosophy which are in conflict with the revealed doctrines of the Church. I realize that
many books and articles have been published in defense of the faith, but not one that deals
with these pernicious doctrines which have become so universally accepted even in what we
are pleased to call our Christian nation. There cannot be any conflict between truth revealed
from heaven and truth revealed through the research of man; for truth is a unit and never is
found in conflict with itself. Unfortunately we live in an age when many theories which have
not been proved are accepted as truth. These theories have been changed from time to time
and are still subject to great modification; yet they persist, and their advocates present them
as if they have been definitely demonstrated. We find them deeply embedded in most
textbooks in geology, astronomy, psychology, sociology, biology, anthropology, and even in
the histories which are used in our schools.

Our children are taught in their homes, in our Auxiliary organizations and in our Priesthood
quorums, to believe in the restoration of the Gospel of Jesus Christ. They are taught that the
Father and the Son appeared and gave instruction to the Prophet Joseph Smith in answer to
his prayer when he sought for light and truth to guide him in and through a confused
religious world. They have been taught that the Son of God advised him what to do and later
other heavenly messengers came and revealed to him the Book of Mormon, instructed him
and conferred upon him the Holy Priesthood and under the direction of these messengers sent
from the presence of the Lord, the Church of Jesus Christ of Latter-day Saints was organized.
They have been taught that man is the offspring of God and that through the \textit{fall} of Adam
death came into the world and passed upon every creature, through Adam's transgression.
They have been taught that this transgression required an infinite atonement making it
necessary for our Heavenly Father to send into this world his Only Begotten Son Jesus Christ
to be a sacrifice to cleanse the world from the penalty of death and to give unto all creatures
the resurrection and immortal life, thus gaining the mastery over death. Moreover, they have
been taught that through this atonement all men may be redeemed from their individual sins
on conditions of true repentance and come back into the presence of God, from whence they
came. 1

In the home parents are commanded by revelation to teach their children these principles of
the Gospel and the necessity of baptism for the remission of sins in the following words:

And again, inasmuch as parents have children in Zion, or in any of her stakes which are
organized, that teach them not to understand the doctrine of repentance, faith in Christ the
Son of the living God, and of baptism and the gift of the Holy Ghost by the laying on of the
hands, when eight years old, the sin be upon the heads of the parents.

For this shall be a law unto the inhabitants of Zion, or in any of her stakes which are
organized.

And their children shall be baptized for the remission of their sins when eight years old, and
receive the laying on of the hands.

And they shall also teach their children to pray, and to walk uprightly before the Lord.

And the inhabitants of Zion shall also observe the Sabbath day to keep it holy. 2

In this manner they are instructed in the home. Then they go to school and find these glorious
principles ridiculed and denied by the doctrines of men founded on foolish theories which
deny that man is the offspring of God and that when we pray to him as our Father, our words
are meaningless and that man is the offspring of some worm or \textit{amoeba} that in some
unknown way multiplied to fill the earth with all its plants and animal life. It is true that not
all teachers believe and teach these foolish doctrines; but these theories do dominate the
secular education of our youth. They are constantly published in our newspapers, in
magazines and other periodicals, and those who believe in God and his divine revelations
frequently sit supinely by without raising any voice of protest. Under these adverse
conditions is there any wonder that the student becomes confused? He does not know
whether to believe what his parents and the Church have taught him, or to believe what the
teacher says and what is written in the textbook he is given to study. Naturally students have
confidence in their teachers and as that confidence increases, there comes a lack of
confidence in the doctrines of the Church and the parental instruction. These are critical years
and every effort should be made in the Sunday School, Mutual Improvement and all the
Auxiliary organizations and Priesthood quorums, to strengthen the faith of these young
people. Bishops and other presiding officers should see to it that only men and women who
are converted and full of faith are appointed to teach. Too frequently, I regret to say,
unwittingly presiding officers in wards and quorums choose teachers that have scholastic
training without discovering whether or not they are converted and in full faith in the
doctrines of the Church. When this happens and a teacher is appointed who is filled with
modernistic doctrines conflicting with what the Lord has revealed, and these theories he
presents before the class, confusion is the result and we find confusion from within. Under
such conditions, with enemies in our ranks, the influence of both Church and home is further
weakened and our youth more seriously impressed with these false theories.

According to our constitutional government denominational religion cannot be taught in our
public schools because our citizenry is composed of so many different faiths, and in justice to
the religious freedom of all no one faith can be singled out with special privileges. This law
has been universally respected by the various churches. In the scholastic world, however, no
man's faith is respected. From one end of the land to the other it is assumed by most teachers
with scholastic degrees, that these degrees place those who bear them in a superior class with
academic freedom to teach what they will and to criticise and condemn, by virtue of this
freedom, any doctrine or theory destructive of the faith of religious people. This idea that the
teacher belongs to a superior class and his learning grants him immunity from showing
respect for religious doctrines is a fallacy not sustained by justice nor constitutional law.
Most of the textbooks written today boldly and impudently contradict the doctrines in the
Bible and its history. Instead thereof, the students are confronted with unproved, and in many
cases, unprovable theories. In truth, no number of scholastic degrees convey the right on the
part of teachers to attack religion in the public schools. This custom is assumed, but because
the protests made against it are impotent the work of destruction of faith goes on. We are
taught that eternal life is the greatest gift of God. This truth requires, or should require, no
argument. God lives. He has decreed that all those who obey his will and are true to his
commandments having to do with salvation and eternal life, shall receive eternal life. They
are to dwell in his presence and be endowed with the fullness of his kingdom. They will
become his sons and his daughters, and joint heirs with Jesus Christ. 3

That man who leads his fellows away from the path to eternal life, commits the greatest of all
crimes! I cannot see how, for this offense against man and God, there can be any forgiveness.
If a man murders a human being in cold blood, he will be damned. He is denied a place in the
celestial kingdom, yet, he has deprived a fellow of a few years of mortal existence who in
course of time would die, for the mortal death is decreed for all; but he who leads a fellow
being away from eternal life, deprives that soul of the greatest gift that our Eternal Father can
bestow.

These theories taught in our schools should be taught \textit{only as theories} for they can be nothing
more. Unfortunately as previously said, they are presented by many instructors as though
they were well established facts, with a positive assurance that belongs only to established
truth. Between belief in God and the fact that he has directed and does direct his servants by
revelation, vision, and personal visitation, and the theories based on organic evolution, there
is a gulf that can never be bridged. These theories are man-made deductions but the
testimony of the prophets are actual facts, attested by sufficient witnesses, according to the
decree of the Almighty, and thus it becomes incumbent upon every soul unto whom these
testimonies come to carefully weigh them in the spirit of humility and prayer by which the
knowledge of the truth may be received, and then accepted. The Savior gave us a formula by
which divine truth may be known. Said he:

My doctrine is not mine, but his that sent me.

If any man will do his will, he shall know of the doctrine, whether it be of God, or whether I
speak of myself.

He that speaketh of himself seeketh his own glory: but he that seeketh his glory that sent him,
the same is true, and no unrighteousness is in him. 4

This is a true saying. Every man who will do the will of the Father as taught by Jesus Christ
will know the truth; but men harden their hearts and refuse to heed his sayings. I know that
our Eternal Father has spoken and revealed his truth to righteous men, and that his truth is
eternal. In these last days the Almighty has opened the heavens and given commandments to
men:

Proving to the world that the holy scriptures are true, and that God does inspire men and call
them to his holy work in this age and generation, as well as in generations of old;

Thereby showing that he is the same God yesterday, today, and forever. Amen.

Therefore, having so great witnesses, by them shall the world be judged, even as many as
shall hereafter come to a knowledge of his work.

Those who receive it in faith, and work righteousness, shall receive a crown of eternal life;

But those who harden their hearts in unbelief, and reject it, it shall turn to their own
condemnation—

For the Lord God has spoken it; and we, the elders of the church, have heard and bear
witness to the words of the glorious Majesty on high, to whom be glory forever and ever.
Amen.

By these things we know that there is a God in heaven, who is infinite and eternal, from
everlasting to everlasting the same unchangeable God, the framer of heaven and earth, and all
things that are in them;

And that he created man, male and female, after his own image and in his own likeness,
created he them;

And gave unto them commandments that they should love and serve him, the only living and
true God, and that he should be the only being whom they should worship.

But by transgression of these holy laws man became sensual and devilish, and became fallen
man. 5

The words of Jacob, brother of Nephi: "Remember, to be carnally-minded is death, and to be
spiritually-minded is life eternal." 6 Which is the same truth stated by our Lord to
Nicodemus:

For God sent not his Son into the world to condemn the world; but that the world through
him might be saved.

He that believeth on him is not condemned: but he that believeth not is condemned already,
because he hath not believed in the name of the only begotten Son of God.

And this is the condemnation, that light is come into the world, and men loved darkness
rather than light, because their deeds were evil.

For everyone that doeth evil hateth the light, neither cometh to the light, lest his deeds should
be reproved.

But he that doeth truth cometh to the light, that his deeds may be made manifest, that they are
wrought in God. 7

It is a very strange thing, but verily true, that almost any false doctrine, philosophy or
hypothesis, will be readily received. Charlatans and false religious leaders seemingly have
little trouble to gain a following and become popular, but the truth has had to fight its way
through the most severe opposition. It is now (1954), nearly 134 years since the Prophet
Joseph Smith had a visitation from the Father and the Son. The pronouncement of this
visitation brought ridicule, persecution, lying reports that have persisted to this day. Nearly
every missionary who has declared the message of the restored Gospel, has had to face bitter
opposition and enemies of the truth have gnashed their teeth in bitter denunciation of them.
But, with a little thought every intelligent man could testify that false faiths and doctrines that
have come into circulation within the past 134 years have existed without serious opposition.
The same is true of philosophies and scientific theories. The only sure way to know the truth
and have the gift of discernment and be able to distinguish between truth and error is by
following the admonition of our Lord Jesus Christ, and then we will know the truth which
will make us free from error. Members of the Church have been baptized and confirmed and
they have the right to the companionship of the Holy Ghost. This gift is bestowed upon them,
but only those who are contrite in spirit, obedient in the keeping of divine commandments,
who are faithful and true, will have this great gift of discernment. If they comply with the
laws of the kingdom of God and earnestly, faithfully, seek to know the truth, they shall find it
and will not be deceived. The great trouble with so many members of the Church is that they
do not live in strict accordance with divine law, therefore they have not freed themselves
from darkness, and they are unable to distinguish the truths from heaven from the theories
and doctrines of men. The word of the Lord will never fail the honest humble person who
will do the will of the Father, he will be given an abiding knowledge that no theory or false
doctrine can destroy. This is the promise of our Lord whose promises do not fail.

President Joseph F. Smith once said:

The Church holds to the definite authority of divine revelation which must be the standard;
and that, so-called "science" has changed from age to age in its deductions, and as divine
revelation is truth, and must abide forever, views as to the lesser should conform to the
positive statements of the greater; and, further, that in institutions founded by the Church for
the teaching of theology, as well as other branches of education, its instructors must be in
harmony in their teachings with its principles of doctrine. . .

A good motto for young people to adopt, who are determined to delve into philosophic
theories is to search all things, but be careful to hold only to that which is true. The truth
persists, but the theories of philosophers change and are overthrown. What men use today as
a scaffolding for scientific purposes from which to reach out into the unknown for truth, may
be torn down tomorrow, having served its purpose; but faith is an eternal principle through
which the humble believer may secure everlasting solace. It is the only way to find God. 8

At the October General Conference, (1952) I made the following remarks: 9

So far as the philosophy and wisdom of the world are concerned, they mean nothing unless
they conform to the revealed word of God. Any doctrine, whether it comes in the name of
religion, science, philosophy, or whatever it may be, if it is in conflict with the revealed word
of the Lord, will fail. It may appear plausible. It may be put before you in language that
appeals and which you may not be able to answer. It may appear to be established by
evidence that you cannot controvert, but all you need to do is to abide your time. Time will
level all things. You will find that every doctrine, every principle, no matter how universally
believed, if it is not in accord with the divine word of the Lord to his servants, will perish.
Nor is it necessary for us to try to stretch the word of the Lord, in a vain attempt to make it
conform to these theories and teachings. The word of the Lord shall not pass away
unfulfilled, but these false doctrines and theories will all fail. Truth and only truth, will
remain when all else has perished. The Lord has said, "And truth is knowledge of things as
they are, and as they were, and as they are to come." 10

Frequently some young student comes to me greatly disturbed because some statement made
by a teacher has expressed doubt of or has discredited, some principle of the Gospel or some
fact recorded in the Bible. Most of these young people are at a receptive age. They have been
taught to believe the scriptures are of divine origin, that our Eternal Father has spoken and
does speak to man and that the books of the Bible are of divine inspiration. Then to have a
teacher ridicule some scriptural incident, or doctrinal teaching, is to them very disturbing.
Having some confidence in their teachers they find themselves torn by a mental conflict. Are
their parents deceived? Is the teacher right? They look upon the teacher as a person of
reliability and integrity. This feeling is augmented by the confirmation given in the textbook
to what the teacher has said. These conflicts are most serious indeed and the student begins to
accept the theories and to reject the teachings of the Church and his parents. If they continue
in school with this conflict to contend with, the conviction is strengthened that the text and
the confirmation by the teacher cannot be wrong.

In fairness, let me say that there are many teachers who have faith and who are able to guide
their students correctly through the rapids of doubt and unbelief, but these instructors are,
today, numbered among the minority, and the odds are against the student who is taking a
high school or college course. I know of no history published today dealing with ancient
peoples that does not start out with a false conception in relation to the origin of man, the age
of the earth, and the historical development of the human race. Under these conditions it
takes a strong will and a secure faith to weather the storms while passing through these
adolescent and early years of manhood and not be influenced by these unstable and unproved
doctrines of men. It is well for our young people to have the experience of a mission where
they can be grounded in the truth before they finish college courses; however, because of the
wickedness of the world at this time, it is impossible for this to be accomplished, for our
youth are taken into military camps and to other military duties where all the finer things of
life are forgotten and where they are left face to face with the most insidious and
unwholesome trials and temptations. In these activities they are furnished tobacco and other
harmful things and where virtue too frequently is laughed at with contempt.

The Lord has revealed that in our day there are many spirits abroad that lie in wait to deceive.
Therefore members of the Church should be "doing all things with prayer and thanksgiving,"
that they may not "be seduced by evil spirits, or doctrines of devils, or the commandments of
men; for some are of men, and others of devils. Wherefore, beware lest ye are deceived." 11

Only a short time before this writing a young girl came to me in some excitement because
her professor in the class had ridiculed the story of Jonah saying that such an incident was
impossible and a legendary story that had found its way into the Bible and that it could not
have happened. She said, "I have always believed that this story was true; what am I to
believe?" I answered, "Do not let what your professor said worry you. You believe in Jesus
Christ do you not?" "I do most certainly." "Then," I said, "our Savior believed it and gave
this story as a sign to the corrupt Jews that he would be three days and three nights in the
earth and then would come forth again." 12 This seemed to satisfy her and with a better spirit
she departed.

On another occasion a young man who had filled a mission came in the office agitated over
some teachings that had been given in his class dealing with some of the fundamental
principles of the Gospel and the following conversation followed:

"You filled a mission did you not?"

"Yes."

"Did you receive a testimony while on the mission that what you were teaching is true?"

"Yes."

"Have you changed your mind; do you believe now that what you taught in the mission field
is not true?"

"No! I still believe it is true."

"Then why are you greatly concerned by the teachings of your professor?"

"Well, you see, I will have to take an examination in his class and what can I say? If I do not
answer as he teaches us, I will get a poor mark."

"Answer his questions according to the text, if you have to; but say it is what is given in the
text. You do not have to say that you believe it. Do not forsake your prayers while you are
studying, or your study of the scriptures, or your activities in the Church, and all will be well
with you."

A few years ago the parents of a young man who was studying scientific courses came to me
in great alarm. Their son was doubting some of the doctrines of the Church. He declared that
they could not be true for they were in conflict with the teachings given in his classes. They
wished me to have a talk with their son. This I did and we went into the matters at some
length. I tried to convince him that there were other textbooks and other scientists which do
not hold to the views he was being taught. That what he was being taught was merely a
theory and not a proved fact. Just what effect my conversation had upon him I do not know.
Others talked to him. One day he came to the office and said he was going on a mission, and
thanked me and others for what had been done for him. He filled an honorable mission and
came home fully convinced with a testimony of the Gospel.

One day I spoke before a congregation of Church members and in the course of my remarks
mentioned the story of Joshua commanding the sun and moon to stand still. I said I did not
know just how this happened, but I believed it happened; and I quoted the words of Mormon
in the twelfth chapter of Helaman: "Yea, and if he say unto the earth—Move—it is moved.
Yea, if he say unto the earth—Thou shalt go back, that it lengthen out the day for many
hours—it is done. And thus, according to his word the earth goeth back, and it appeareth unto
man that the sun standeth still; yea, and behold, this is so; for surely it is the earth that
moveth and not the sun." I said that this presented a plausible reason how that miracle in the
days of Joshua may have been done. This was published and it brought into my office a
teacher of science with whom I had gone to school in earlier days. He took me to task for my
remarks and said: "Why, do you not know that if the earth slowed up for part of a day that it
would create such a terrific wind that everything on the face of the earth would be swept
off?" I looked at him and with a smile said: "My goodness! Is it not too bad that the Lord
would not know this?" The conversation ended. Then I thought of the scripture where it is
written that before the great day of the coming of the Lord the earth would "reel to and fro as
a drunkard," 13 and what then, would be the nature of the wind.

On another occasion a young man that I had known in the mission field came to see me. In
the mission as a boy he was very active. He had wonderful parents, two brothers and a sister,
all of whom were very active. His mother was a noble woman, faithful and true. I had not
seen this boy for several years. He was now a young man. He came to me seeking a favor. In
the course of our conversation he said he was not active in the Church; in fact could no
longer accept the teachings of the Church, and then followed this conversation:

"You mean to say that you have lost your faith in the Church? Does your mother know how
you feel?"

"I have not told her. You see, I have learned a great deal since I have been to school. I don't
believe anything that I cannot see or feel."

"Do you see that high mountain through the window?"

"Yes."

"Do you see anything between us and the mountain top?"

"No."

"Do you know that there are hundreds of thousands of tons of air between us and that
mountain peak? That there is a pressure of some 12 or more pounds of air to every square
inch?"

"I don't know; it may be so."

"Can you see or feel that immense weight of air?"

"No."

"Then your philosophy is all wrong. There are thousands of things that you and I cannot see,
feel, smell, taste or hear."

"Do you know that there are many tons of water suspended in the air between us and the
mountain top?"

"I don't know."

"Well there are. You neither see this water, feel it, taste it or smell it, but it is there
nevertheless. I think you better forsake your false philosophy."

The poor fellow thought that he had gained wisdom. He had heard the doctrines of the
Church criticized and had been taught fragments of some modern philosophy. He wanted a
demonstration, a tangible evidence for everything, like the Pharisees of old, and perhaps for
the same reason. The fact remains, and is acknowledged by all experienced scientists that
there are thousands of things around about us and everywhere in the universe that cannot be
explained by any of the ordinary senses. We know they are true, but they remain unknown,
their secrets have not been discovered. For instance, scientists do not know what light is.
They have theories, but all they know is confined to theory. The rate that light travels is
measured, that it travels with terrific speed is established. Professors Erich Hausmann and
Edgar P. Slack have written, "Light is radiant energy which is capable of affecting the eye to
produce vision. Its exact nature, as in the case of gravitation and electricity, is not fully
understood, but much has been learned about the way it is produced and propagated." 14
Here is an admission by two noted scientists that we do not understand light, neither do we
fully understand gravitation or electricity. Dr. Charles E. Dill has said: "It seems a strange
paradox to say that physicists are in greater darkness concerning the true nature of light than
they are in regard to almost any other topic." 15

Dr. Oswald Blackwood says: "A question often arises over whether or not light exists in the
depths of space where no eye is present to observe it. To this question there are two correct
and yet contradictory answers, their correctness depending on whether the question is
answered by a psychologist or a physicist. Some psychologists define light as a sensation;
hence they would say that no light exists where there is no eye to perceive it. The physicist,
on the other hand, defines it as the cause of that sensation, and he is more interested in the
behavior of light as an objective phenomenon than in its subjective perception." 16

A scientist is able to understand the structure of a brain and the nervous system but who is
able to tell whence comes a thought? What makes the heart beat? Why will two rose bushes
only two feet apart, drawing nourishment from the same soil bear roses one deep red and the
other pure white? Where and how comes the delicate coloring of the pansy or violet out of
the same soil? Why are snow crystals always formed in six-pointed stars or sides, never in
five or seven? One scientist has said that, "Water and sugar and the complex minerals which
make the granite rocks all follow laws which are utterly unchangeable, but which are, as far
as we can see, without any special reason: it is as profitable to speculate why the chlorophyll
of vegetation is green and why the blood of animals is red. . . . Science knows why snow is
white, and why it is beneficent; but it cannot explain the law of six." A black hen will lay a
white egg and another hen either white or black will lay a brown egg. The eggs of some birds
are blue, some are brown, some are white and some are speckled. William J. Bryan once
said: why can "a black cow eat green grass and then give white milk with yellow butter in
it?" Who can explain why these things are so?

There is no saying of greater truth than that of Paul in writing to the saints at Corinth:

But as it is written, Eye hath not seen, nor ear heard, neither have entered into the heart of
man, the things which God hath prepared for them that love him.

But God hath revealed them unto us by his Spirit: for the Spirit searcheth all things, yea, the
deep things of God.

For what man knoweth the things of a man, save the spirit of man which is in him? even so
the things of God knoweth no man, but the Spirit of God.

Now we have received, not the spirit of the world, but the spirit which is of God; that we
might know the things that are freely given to us of God.

Which things also we speak, not in the words which man's wisdom teacheth, but which the
Holy Ghost teacheth; comparing spiritual things with spiritual.

But the natural man receiveth not the things of the Spirit of God: for they are foolishness
unto him: neither can he know them, because they are spiritually discerned.
But he that is spiritual judgeth all things, yet he himself is judged by no man.
For who hath known the mind of the Lord, that he may instruct him? But we have the mind
of Christ." (1 Cor. 2:9-16.)

Zophar, the Naamathite, said to Job, "Canst thou by searching find out God? Canst thou find
out the Almighty unto perfection?" 17 The answer is, without the Spirit of the Lord, No! The
scientific mind which dwells constantly on the physical and temporal things of the universe,
endeavoring to fathom all of the laws of nature, but who ignores the spiritual guidance which
he could have if he sought in faith for it, will never find out God. He will invariably find and
follow false gods, worshiping the substance and ignoring the Maker. This will be
demonstrated as we proceed with this treatise. We have been promised by the Lord that all
those who do the will of the Father shall know of the doctrine and all who would continue in
his word should know the truth, and the truth would make them free. 18 Moreover he gave to
his disciples the gift of the Holy Ghost that they might be taught and directed in all truth and
so great did he consider the guiding power of this Holy Spirit, that he declared though all
other sins and blasphemy may be forgiven men, yet the man who commits blasphemy against
the Holy Ghost "it shall not be forgiven him, neither in this world, neither in the world to
come." 19

It is unfortunate that so many scientists know nothing of spiritual things and such expressions
as these of our Lord, are meaningless to them. All they have is, as Paul puts it, knowledge
limited to "the spirit of man."

There are spiritual influences that are just as deep and meaningful as anything that is tangible
to the natural senses; yet they cannot be described or explained. They come through the still
small voice of the Spirit. They are penetrating but cannot be described any more than the
feelings of love, sympathy, friendship, can be defined and fathomed. One thing that a
member of the Church may know is most assuredly that God lives, that Jesus is in very deed
the Only Begotten Son of God; the Redeemer of the world and the Savior of all those who
obey him and keep his commandments. Moroni has promised that every soul who will
sincerely, in faith and humility, read the Book of Mormon shall know by the power of the
Holy Ghost that it is true. "And by the power of the Holy Ghost ye may know the truth of all
things." 20 Members of the Church by the many thousands can sincerely, truthfully, testify
that this promise has literally been fulfilled and that these words are true. They know the
truth as thoroughly as they know that there is sunshine and rain upon the earth, that the wind
at times blows, that we are subject to cold and heat and have many other sensations common
to other men. These manifestations are just as true and more enduring than are the
manifestations that come through the ordinary senses of man. They cannot, however, be
explained to the understanding of the unbelieving person who has no experiences with which
there can be made a comparison. It is just as impossible to make the hardened materialist
understand the spiritual manifestations as it is to make a man born blind understand the color
of blue, or red, or yellow, for he has no experience by which a comparison can be made. Yet
it is just as foolish for the materialist to deny the spiritual manifestations that come to a
humble member of the Church with a contrite spirit and broken heart, as it would be for the
person born blind to deny that there are such colors as red, blue or yellow, that those who see
can visualize, because he has never seen them and does not know what they are like.

By scientific investigation no man can demonstrate and prove the resurrection of the dead.
How can a body that is burned to ashes be restored, or one that has turned to dust in the
grave? This, nevertheless is the great promise made by our Lord, Jesus Christ, that all who
have lived, who are now living and who will yet live in mortal life upon the earth, shall come
forth from the dead receiving immortality or eternal life. 21 This promise in part has been
fulfilled, for the righteous dead who lived from the days of Adam to the time of the ministry
of Jesus Christ, came forth from the dead after his resurrection. 22 Every true Latter-day
Saint knows that Peter, James, Moroni and other former apostles and prophets, came in their
immortal resurrected bodies to Joseph Smith and Oliver Cowdery and others.

Since the advent of wireless telegraphy, the radio and television, it has been impressed upon
the minds of all that there are innumerable waves passing over the face of the earth in all
directions. We cannot see them, we cannot feel them, yet we know they exist. Some of these
waves scientists have been able to intercept, or at least with them make contact. Previously
these waves were unknown to all except a few. We know that a dog and other animals can
understand sounds that reach their ears that human ears cannot hear. The scientist and
astronomer, Camille Flammarion has given us these stimulating thoughts:

Auguste Comte and Littre' have apparently striven to trace out for science its definite, its
"positive" way. They tell us we are only to admit what we can see, or can touch, or what we
have heard; we are to receive nothing except on the clear evidence of our own senses, and are
not to endeavor to know what is unknowable. For half a century these have been the rules
which have regulated science in the world.

But see now. In analyzing the testimony of our senses we find that they can deceive us
absolutely. We see the sun, the moon, the stars revolving, as it seems to us, round us. That is
all false. We feel that the earth is motionless. That is false too. We see the sun rise above the
horizon. It is beneath us. We touch what we think is a solid body. There is no such thing. We
hear harmonious sounds; but the air has only brought us silently undulations that are silent
themselves. . . .

Nor is this all. Furthermore our five poor senses are insufficient. They only enable us to feel
a very small number of the movements which make up the life of the universe. To give an
idea of this here, I will repeat what I wrote in \textit{Lumen}, a third of a century ago. "Between the
last acoustic sensation perceived by our ears, and due to 36,850 vibrations per second, to the
first optical sensation perceived by our eye, which is due to 400,000,000,000,000 vibrations
in the same space of time, we perceive nothing. There is an enormous interval with which no
one of our senses brings us into relation. If we had other chords to our lyre, ten, one hundred,
or a thousand, the harmony of nature would be transmitted to us more complete than it is
now, by making these chords all feel the influence of vibrations. On one hand our senses
deceive us, on the other their testimony is very incomplete. Thus we have no cause to be
vainglorious, or to set up our so-called positive philosophy as a principle. 23

That there are influences and contacts that may be made that are far beyond the powers of
mortal man, unaided by the Spirit of the Lord, every member of the Church may know. The
whispering of the Still Small Voice, the impressions that come from the guidance of the Holy
Ghost are felt, but they can only be received by the person with a pure heart, a contrite spirit,
for there must be these in order to complete the contact, just as we have to comply with
certain definite laws to become attuned to the message of the radio, or of television.
Members of the Church should so live as to be worthy of these manifestations.

Suppose an airplane travels at the rate of 300 miles per hour from Quito in Ecuador to Belem
in Brazil, not far from the equator. Each hour the airplane goes better than 1300 miles.
Suppose it takes the return journey at the same rate of speed, then it covers about 700 miles
going eastward while making the 300 miles westward. Should it travel at the same rate of
speed from Quito to Panama, it would make the 300 miles northward according to schedule,
but at the same time would be going east at the rate of 1000 miles per hour. Who ever stops
to think of this? To all appearances it is not true.

If members of the Church will obey divine commandments they may be in perfect accord
with the Spirit of the Lord, then they will not be deceived and that Spirit will enlighten their
minds and quicken their spirits and they will not be deceived in relation to the great
principles of truth which prevail in and govern the Kingdom of God.

\newpage
REFERENCES—INTRODUCTION

Footnotes

1. Pearl of Great Price, Moses 5:6-16. 2 Nephi 9:19-38.

2. D. \& C. 68:25-29.

3. Romans 8:14-17. Rev. 21:7. D. \& C. 14:7. 76:53-59.

4. John 7:16-17.

5. D. \& C. 20:11-20.

6. 2 Nephi 9:39.

7. John 3:19-21.

8. \textit{Improvement Era}, Vol. 4:548-551.

9. \textit{Conference Pamphlet}, Oct. 1952, \textit{Era}, Dec. 1952.

10. D. \& C. 93:24.

11. \textit{Ibid.}, 46:7-8.

12. Matthew 12:30-40.

13. Isaiah 29:20. D. \& C. 45:48. 49:23.

14. Hausmann and Slack, \textit{Modern Physics}, p. 583.

15. \textit{Ibid.}, p. 321.

16. Blackwood, Dr. Oswald, \textit{General Physics}, p. 307.

17. Job 11:7.

18. John 7:17. 8:32.

19. Matthew 12:31-32.

20. Moroni 10:4-5.

21. John 5:25-29. 11:25. Alma 11:41-45.

22. Mathew 27:52-53. 3 Nephi 23:9-13.

23. Flammarion, Camille, \textit{The Unknown}, pp. 11-12.

