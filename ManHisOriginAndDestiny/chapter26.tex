\chapter{AUTHENTICITY OF THE SCRIPTURES (Old Testament—Part Two)}

DURING the second half of the nineteenth century there was a determined effort launched on
the part of certain scholars to tear asunder and destroy the authenticity of the holy scriptures.
They were influenced by the same spirit which prompted the organic evolutionists. This plan
has been called "Higher Criticism," but in reality it should be called "destructive criticism."
The advocates of this theory assumed to have the wisdom by which they could discover,
without Urim or Thummim, but by their own wisdom, a difference in style in the various
books of the Bible. This difference they proclaimed was discovered within paragraphs as
well as in chapters of the various books. Moreover, they taught that in many of the books,
particularly the five books of Moses and Joshua, Isaiah and others, there was evidence that
indicated that parts could not have been written at the time indicated by these books, but at
some later date. In this manner of criticism unknown writers had to be provided to take care
of these theories. Some of the passages, like that dealing with Isaiah's prophecy naming
Cyrus (Isaiah 44:28 and 45:1-4), they maintained were written by authors after the prophetic
events had taken place. To their way of thinking even God could not predict the birth of a
man over one hundred years before he was born. In the Book of Genesis they thought they
discovered combinations of writers, and that the account of the creation and of Adam's
advent in the Garden of Eden was in conflict with itself. These critics were not like the
prophets of old of whom Peter speaks: "Knowing this first, that no prophecy of the scripture
is of any private interpretation. For the prophecy came not in old time by the will of man: but
holy men of God spake as they were moved by the Holy Ghost." (2 Peter 1:20-21.) These
learned men do not claim to have the guidance of the Holy Ghost, but by their scientific
training they have spoken and given utterance.

In their contention, through their uninspired skill, they maintain that they were able to
discover that the five books of Moses were not the works of Moses. So they concluded to
give these books to several authors living at various times. Genesis, said they, was compiled
by some enterprising scribes hundreds of years later. So the Pentateuch and the Book of
Joshua had to be assigned to writers of various and later times. Moreover, their doctrine was
that the stories of creation, the Garden of Eden, Adam's fall, recorded in Genesis, were taken
from the myths and legends of the Assyrians and Babylonians. It was the accepted view at
that time that "the Mosaic age was outside the scope of written records." 1

In addition to this severe criticism of the Pentateuch these critics assigned the Book of Isaiah
to three, at least, different writers. The entire study was, of course, speculative, and could be
nothing more. Examples have been given where some of these experts who claim to decipher
these ancient writings, have failed to find the line of demarkation in the writings of men in
our own day. Sir Charles Marston, a renowned archaeologist, illustrates this in his most
excellent work, \textit{New Bible Evidence}. He says that the work of various writers of \textit{The New
York Times} which are doubtless amended by editors and subeditors, yet so far as he is aware,
"No textual critic ever pretended to be able to distinguish one writer from another, nor to
identify the amendations of the editorial staff. If methods of textual criticism are powerless to
analyze contemporary press composition, how can they correctly analyze documents
composed more than two thousand years ago, and written in a dead language? Yet it is the
very fact that the documents are so ancient and the language so old, that \textit{seems to be
responsible for the supposition that the critic can do so}, and to sustain their supreme
confidence." (pp. 232-233.)

He further says that "the so-called textual criticism of the Old Testament is an endeavor to
extract internal evidence from a sacred text. Such a method cannot be applied to
contemporary literature. He relates how a case of plagiarism of a modern book led to the
investigation of "Higher Criticism," in a Canadian law court in 1931. The judge called it
"solemn nonsense." Then the case was taken to the Appellate Division of the Supreme Court
of Ontario, and the "Higher Criticism" was there called by Justice Riddel, "almost an insult to
common sense." Then finally the case was taken to the Judicial Committee of the Privy
Council of England, which is the highest tribunal of England, or the British Empire, in
October 1832. Judge Atkin, the presiding judge, described "Higher Criticism" as "fantastic
hypotheses." In addition to this attempt to find multiple authors for the scriptures, this
"research" has also resulted in these advocates classing many of the earlier stories in the
realm of fiction and myths, which they have claimed were current for many years antecedent
to the days of Adam. The story of creation, the temptation and the fall had to go into the
discard as having no divine sanction, but came out of the Assyrio-Babylonian myths. This
also, they say, is true of the flood and the story of the confusion of tongues. One of these
writers upholding this theory has said: "The events covered in Genesis are pre-history," 2 and
many others are relegated to the category of "pure romance," and as such are not to be
accepted as true. The following will explain this view:

In the childhood of the human race, many thousand years ago, there came a time when men
began to feel the need of help against unfriendly things around them. The world was filled
with forces they could not understand. In their efforts to range these powers on their side and
so find life and happiness, men originated and continued to practice ways of behavior which
strike us as strange, if not irrational. In this way \textit{magic} was developed and out of it, possibly,
\textit{ethic religion}. Some of the behavior patterns from the period the student will already have
heard about—sympathetic magic, the worship of idols, animal sacrifice, necromancy,
circumcision, religious rites and observances of many kinds. We know a great deal—through
archaeology, history, and the study of primitive tribes on the earth today—about these old
practices. 3

Their search led these scholars to divide the books of Moses into four grand divisions: the "J"
(Jehovistic), the "E" (Elohistic), the "P" (Priestly), and "D" (Deuteronomistic). These
writings were made at different times, according to this story, and by enterprising scribes,
who compiled and placed them in the books as we have them now. They maintain that in the
first chapters of Genesis there are contradictions. This conclusion is reached, in part at least,
due to the fact that in the translations as the world has them of Genesis, these great men do
not have the knowledge that there were two creations. First, the spiritual and second, the
physical. Again, they have no correct understanding that Jehovah and Elohim are separate
personages, one being the Father and the other the Son. There is no contradiction and it is a
case where the things of God are not understood by the spirit of man, and this great truth was
made clear by Paul. 4

These four accounts, according to the theory, run concurrently throughout most of the Old
Testament. Therefore they have the "J," the "E," the "P" and the "D" authors of these
chapters in Genesis and many other parts of the Bible. 5 Isaiah, say they, was composed by
"First," "Second" and "Third" Isaiah. In other words there were three writers who made the
book, perhaps there were four. One of these is supposed to be a writer of the fifth century, or
later of the pre-Christian era. He is supposed to have lived in Babylon and furthermore, he it
was who wrote of Cyrus, for, as noticed before, God could not predict the coming of Cyrus to
be a deliverer of Israel. This story will be told in another place. And so these modernistic
critics have cut the book of Isaiah up and thrown it to the four winds. The book of
Deuteronomy is also relegated to this later date. 6

My purpose is not, however, to present these theories, but to give expert testimony by other
learned men who are competent to speak, who contradict all of these speculative conclusions,
which have, within the last part of this century, been proved to be false. First we will give the
testimony of Dr. A. S. Yahuda, one of the outstanding students and researchers of ancient
records:

A. S. YAHUDA

\textit{Destructive Methods of Biblical Criticism}.—No one, and the present writer least of all, would
make the slightest attempt to belittle the great merits and achievements of Biblical criticism.
But it must be said that, so long as moderate views prevailed, there was a sane and sound
method of Biblical research. Unfortunately this method has since deteriorated through the
more radical views adopted by the modern school of High Criticism, especially under
Wellhausen and his followers.

The whole system has degenerated into a mass of far-fetched hypotheses and haphazard
theories, which only fitted with a frame of preconceived ideas about the history, the
development and the composition of the scriptures. In the long run it became customary to
consider it as highly scientific to challenge everything Biblical and to alter the texts to one's
heart's desire.

The whole Pentateuch is represented as a conglomoration of various sources. In many cases
one chapter is attributed to two, three, or more sources. Even in each one of these sources
two or more underlayers are discerned. Thus, taking the whole Pentateuch as it is made to
appear, the impression is left of a patchwork stuck together by stupid authors and ignorant
scribes, the result being a most disproportionate and inharmonious composition.

Indeed, the mania of seeing everywhere a wrong text and detecting all kinds of
interpolations, glosses and anachronisms, and likewise the zeal to heap emendations upon
corrections resulted in creating a new speciality for speculative 'experts' to exert themselves
in the art of text alterations and source-hunting. (Introduction—\textit{The Accuracy of the Bible}, p.
xxi.)

This also occurs in the \textit{Introduction}:

In taking up the task of proving the Hebrew-Egyptian relationship from a wider and broader
angle than has ever been done hitherto, it is not intended to substitute the pan-Babylonistic
method of deriving everything Biblical from Assyro-Babylonian sources—a method which
was so much in vogue and is still dominating Biblical research—by a similar one-sided pan-
Egyptianism."

Doubting Biblical statements became a standard of scientific method in Biblical research,
and critics practicing that method earned recognition and acquired great authority. The
greater the doubts raised the more was appreciation expected; and the more numerous the
hypotheses brought forward to discredit Biblical statements, the more credit was granted to
the scientific soundness and critical sagacity of the sceptics. All these methods and
arguments only betray the superficiality with which the Biblical documents are treated by
Biblical critics, and indicate their embarrassment in attempting to maintain arbitrary theories
which can be proved neither documentary evidence nor by logical reasoning. 7

A still more radical standpoint was adopted with regard to the early history of Israel,
especially that of the Patriarchs and the sojourn of Israel in Egypt. Indeed the whole story of
the Patriarchs was declared as more or less legendary, and that of the sojourn of Israel in
Egypt was represented as the product of much later periods containing only very pale
reminiscences of vague old memories of the Egyptian epoch, and episodes invented with the
object of substantiating later conceptions by earlier supposed events, which according to
those sceptics had never occurred.

The notable finds yielded by the excavations in Assyria and Babylonia, which confirm the
Biblical records, have been employed rather to shake the authority of the Bible than to
uphold it. Because some of the Genesis stories bear a remarkable resemblance to Assyro-
Babylonian myths, of which the story of the Great Flood is the best example, it was assumed
that they were written during the Babylonian exile, in the sixth century B.C., and that only
certain portions were of two or three centuries earlier. Yet, on closer examination of the
Genesis stories from a linguistic point of view, I have found that the Assyro-Babylonian
traces were much fewer than was supposed, and that these stories can by no means have been
composed in the Babylonian exile nor in the ninth or eighth century B.C., but that they must
belong to the time of the great civilization of Ur, in the time of the Patriarchs. 8

This distinguished scientist declares that the same tactics have been employed by these
destructive critics in other parts of the Bible history. The Joseph-Exodus narratives were
twisted to "prove exactly the opposite of that which should actually be proved, and were
employed rather to obscure the Egyptian background of the narratives. It was asserted that
their author or authors had very little knowledge or none at all of Egyptial matters, and that
even such features which, according to their views, still preserve certain Egyptian coloring,
had been supplied by tourists or Hebrew mercenaries in Pharaoh's army who happened to be
in Egypt! These people have just snatched a few things from Egyptian life, picked up a few
words from the Egyptian language and brought them home for the benefit of scribes who
utilized them for literary exercises." 9 He most emphatically declares that the archaeological
and linguistic evidence, of these narratives bear all the evidence of "information and
confirmation" upholding the Biblical account. The more recent excavations by Mr. Leonard
Wooley in Ur also present fresh evidence of the truth of the flood story in the Pentateuch and
these have been further confirmed by the Biblical data discovered by Professor Garstang at
Jericho, where the walls disclosed the secret which remained for "thousands of years buried
in their cracks—the secret, namely, that the fall of the walls, hitherto looked upon as a mere
legend, \textit{was a real historical event}."

Dr. Yahuda further declares that the Hebrew language shows evidence likewise of the
accuracy of the Biblical history and this evidence can be traced through the years of their
contact with other races. "All this will contribute to demonstrate that the presence ofEgyptian elements in the Pentateuch is the best indication that the Books of Moses have
actually been composed in the epoch, in which the Hebrews were still under the immediate
influence of their connections with the Egyptians, just as it is stated in the Pentateuch itself."
10

The story of Pharaoh's dream, says Dr. Yahuda, could not fit in any other place except in
Egypt. It was in Egypt where the goddess Hathor was worshiped in the form of a cow. There
are pictures in the monuments of Egypt of the seven kine. This is seen in the tomb of
Nefretiry. There were seven districts each having its \textit{Hathor cow}. This is shown in the Book
of the Dead in mural reliefs of the temple of Hatschepsut in Dair-al-Bahri. The author of the
Joseph narrative was acquainted with the Egyptian customs of the time. "Joseph was the sole
'vizier' over the whole country of the 'two lands.' This is of historical interest from the fact
that before the 'New Kingdom' there was only one vizier for both territories, but in the New
Kingdom two viziers came upon the scene, one for Upper Egypt with the title 'Vizier of the
South,' and the other for Lower Egypt, 'Vizier of the North.' When Joseph was made sole
vizier over the whole country it is written over the 'two lands'." And says the Doctor, if the
narrative had been written without the narrator knowing this fact, he would have said merely
that Joseph was "installed as vizier of Egypt, without emphasizing on every occasion that the
whole country of the 'two lands,' were under his rule." It is consequently not possible to admit
that Joseph's appointment for him is a mere legend. On the contrary he records "it as a
positive historical fact, illustrated by such features as could only be rightly understood and
appreciated in the light of changes introduced in state institutions much later than the Joseph
period." 11

On page 116 of his work, \textit{The Accuracy of the Bible}, Dr. Yahuda says that the Exodus
account is treated with the same "distrust" by these modernistic Bible critics, and a "few
Egyptologists," who seem to find in it legendary and mythical details. Of course anything
that borders on the miraculous, such as the dividing of the Red Sea, the cloud that led the
Israelites, the plagues in Egypt, and other marvelous stories, will appear to these skeptics as
being legendary and mythical. Dr. Yahuda says this same method is not applied by those
same scholars to non-Biblical documents, "even when permeated with mythical details. The
discrimination against the Joseph and Exodus stories, perfectly fits within the attitude
pursued by scholars adhering to Biblical criticism when they write the history of Israel, or of
the literature of the Bible, but thoroughly differs from true and sound scientific methods of
historiography. Unprejudiced writers of ancient history base their views on the documents
transmitted by the people themselves, and as a whole, accept ancient records as the essential
foundation for historical reconstruction."

EDOUARD NAVILLE

Dr. Edouard Naville, recent Professor of Archaeology at the University of Geneva,
Switzerland, is considered to be one of the most outstanding scholars in the archaeological
field. He did considerable research in Egypt and other places and was honored by his
associates and his wisdom and learning were sought by others in his chosen field. He added
much to the present knowledge by his researches. These are his views in relation to this
modernistic criticism. Speaking of the many records that have been found:

Their bearing on the books of the Bible has not been adequately shown, the reason being that
most Biblical scholars are still tied down to the methods of the destructive criticism. A book
of scripture is taken, a minute philological analysis is made of it, with often a great amount of
scholarship, but this analysis necessarily leads to the discovery of apparent inconsistencies,
of disconnections, of repetitions, which have been interpreted as showing the hands of
different writers. The whole process has been disintegration of the books, resulting in the
creation of a great number of authors, for the existence of whom no historical proofs
whatever can be adduced. 12

SIR CHARLES MARSTON

Another leading archaeologist of great renown, Sir Charles Marston, adds his testimony to
the accuracy of the scriptures. He was with Professor John Garstang at the excavations at
Jericho, where evidence was found confirming dates of many of the great events of Israelite
history, contrary to the destructive criticism which became so popular and is related to the
theories of organic evolution. Dr. Marston, in the \textit{Preface} to his very excellent work, \textit{New
Bible Evidence}, makes the following comment:

. . . George Bernard Shaw calls the Bible an old collection of myths and fairy tales, and there
appears to be quite an impression abroad that this is the case. But if only because the
description cuts right across the whole course of American history, it is unacceptable to us.
Mr. Shaw would have us believe that the Pilgrim Fathers and the great makers of America
\textit{believed a lie}! It seems incredible that the civilization of the United States was begun and
built up on mere myths and fairy tales.

Once again: The Old Testament largely concerns the history and religion of the Jewish
people. In the countless vicissitudes through which this race has come during the past
eighteen centuries, had it succumbed, or ceased to exist, or sunk into obscurity, it might be
said, with some reason, that faith in the truth of the Book was unfounded. But the contrary
was proved to be the case—the continued existence and present prosperity of the Jewish race
constitute a living witness to a reality which must underlie the Old Testament.

These considerations call for an examination of the Old Testament in the light of recent
archaeological discoveries.

No matter what attitude (conservative or advanced) a man may adopt, he has to face the fact
that there are sentences in the Bible which do not make sense. On the other hand such
examples serve to demonstrate its great age. The same characteristics occur in less ancient
writings—even in the works of Shakespeare, composed only some three centuries or so ago,
and in the English language. But the Bible, so far as the New Testament is concerned, was
written down more than eighteen hundred years ago in the colloquial Greek of that time;
while the Old Testament was composed from 2500 to 3500 years ago, in the ancient Hebrew
language.

There is abundant evidence that, in course of ages, little mistakes have been made by scribes
in copying the sacred texts. But in no instance has it been proven to have effected the
fundamental sense and harmony of the truth. 13 In the case of the Old Testament those
acquainted with the Hebrew alphabet will be aware that it has no vowel sounds. Some of the
letters so closely resemble others that mistakes in transmission are easy; and a comparison
with the Septuagint, or Greek translation of the Hebrew text made three centuries before
Christ, suggests that such mistakes have occurred even since that time. Such considerationsweigh against the assumption of the word-for-word and letter-for-letter correctness of the
English Bible. 14

Those who employ critical methods in the study of the Bible, have got into the habit of
representing those who advance proofs to the contrary, of being prejudiced. Is there any
reason why those who are led by evidence to adopt an orthodox attitude to the Bible, should
be more prejudiced than those who describe it as a collection of fairy tales? . . .

There is another class of prejudice which the archaeologist has to face; it comes from those
who cannot disentangle themselves from their past academic environment. Thus, when
Schliemann uncovered the remains of Troy in 1870, the scholars laughed him to scorn. Or,
again; so bewildered were German professors when Dr. Hilprecht, on behalf of the
University of Pennsylvania, laid bare a great temple platform at Nippur in Mesopotamia,
built of blocks inscribed with the name of a monarch which the critics had treated as mythical
that the excavator was positively accused of perpetrating the forgery of a whole Babylonian
temple platform. 15

Dr. Marston has this to say about the attempt to dissect the Bible, by Cannon Samuel R.
Driver in respect to the first six books of the Old Testament. Driver said: "The two earliest
narratives are doubtedly those by J. and E.; these are based upon the oral traditions current in
the eighth and ninth centuries." 16

The statement that purports to be made by Moses five or six centuries earlier, were oral
traditions of the eighth and ninth centuries; and, in order to complete the critical diagnosis of
dates, were first committed to writing B.C. 621. Dr. Marston then launches on a defense of
the Bible. At this point all we need to do is to refer to the words of the Lord in the Book of
Mormon and the Pearl of Great Price which fully and completely refute these fancy
imaginary stories. We have the sure word of the Lord revealed to us in these two standard
works, that the accounts in Genesis and the history of the first six books of the Bible are true
and that these accounts were written by Moses in the first five books, from revelations given
to him, and without a doubt from records which had been handed down from the fathers to
Abraham who declared that he would hand them down to his posterity. Moreover, Lehi had
these books in his possession on the Brass Plates, before there could have been any
"imaginary prophet of the exile" to write them.

Dr. Marston also pays his respects to the part that evolution has played in the twisting of the
Bible accounts of things in the beginning. It is needless to consider every foolish criticism
brought against the Bible coming from this destructive source. However, there are many
statements of importance in Dr. Marston's writings. I commend to every member of the
Church the study of his book, \textit{New Bible Evidence}. Here are other important statements:

The alternative of following the guidance and authority of critics and commentators in the
light of recent knowledge can but bewilder the issues and lead us all astray. It is quite
obvious that the complete assurance, with which many have written, is entirely unjustified,
even as it is out of harmony with the scientific outlook of the present day. Yet the mass of
people are not aware of this fact, and the erroneous belief that scholars and scientists knew
all there was to be known about the Old Testament, has had a blighting effect upon
Christianity.

The vast archaeological discoveries of the past eight years, resemble the fragments of some
immense jig-saw puzzle; they need a guide to fit them together. The Old Testament has
proved an excellent guide to the geography of the Holy Land; may it not also be of service in
elucidating its history? . . .

Men are still trying to weigh and measure the Bible by imperfect historical evidence, and by
materialistic conceptions of the unseen, which science has already discarded. 17

Dr. Marston pays his respects to our historians who waste their time with "primitive man,"
according to evolutionary doctrine, which is all conjecture. "If all the time wasted, he says,
"on minute dissection of the Bible text—on straining at gnats and swallowing camels—had
been spent in learning to read cuneiform tablets, there would be a far clearer knowledge of
ancient history today, than at present is the case." He says also, "what is the use of teaching
students an ancient history of the world, still largely based on conjecture, when there are in
existence original contemporary documents to tell us what the people of Abraham's day
studied, or the people of Moses' time believed; what were their customs; how they bought
and sold; what laws their rulers made, how long they reigned, and what they did."

Dr. Marston was with Dr. John Garstang at the excavation of the old city of Jericho, and was
also at the excavations of other parts of Palestine, where ancient cities were uncovered, in
which new evidence was discovered. For instance, in the city of Jericho the excavations
revealed the truth of the Bible story. That city was destroyed suddenly by an earthquake
which threw down the walls. Moreover, these scientists found in the pottery evidence which
had not been known in the first quarter of this century. They found numerous Egyptian
scarabs which fixed the dates, and thus they discovered the time of the destruction of Jericho
and other settlements, which forced the scientists to correct their chronology in relation to the
time the Israelites were in Egypt, the time they left and when they arrived in the promised
land. These scarabs, or seals, had inscribed upon them the names of Pharaohs whose dates
were known which confirmed the stories in the Books of Moses. These recent discoveries
have revealed that alphabetical writing was in existence in Sinai long before the time of
Moses, and the keeping of records in the wilderness by Moses is assured, 18 and that worship
of one God is as old as historical knowledge goes. The best and most reliable archaeologists
now affirm that the oldest worship was the worship of one Supreme Being, not many gods
which so many of our imaginary historians are pleased to tell us.

In connection with these discoveries I present here an article taken from the San Francisco
Chronicle of March 16, 1937, near the time that these other great discoveries were made.

JERUSALEM, March 15—(Palcor Agency)

Twelve pieces of broken pottery found on the site of ancient Lachish destroy the very
foundation of Biblical "Higher Criticism," Dr. E. L. Sukenik, professor of archaeology at the
Hebrew University of Jerusalem, said today.

Professor Henry Toreznyer, expert on Semitic languages at the university, definitely ascribed
the potsherds, with their inscriptions in the ancient Hebrew language, to the time of Jeremiah,
2,500 years ago.

Professor Toreznyer called it "the most valuable find ever made in the period of the first
temple (Solomon's)" and Dr. Sukenik termed the potsherds the greatest discovery since the
Siloam inscriptions in 1890. 19

A remarkable feature of these writings, Professor Toreznyer asserted, is the fact that they
appear to be written in ordinary ink. Dr. Sukenik declared their most amazing characteristic
the fact that many of the words and names used are spelled exactly as they are in the
traditional Masoretic text of the Pentateuch. (The Masoretic text is the form of the Hebrew
used today by Jews through the world.)

That the spelling found at Tel Adduweir corresponds exactly with that in use today would
indicate, Dr. Sukenik declared, that the Bible was written by scribes during the time in which
the events in Chronicles actually accurred, and that the scribes were eyewitnesses of the
incidents they reported.

"This would tend to destroy the theory of 'Higher Criticism' that the narrative was written
many centuries later," he said.

DR. ARCHIBALD C. SAYCE

Dr. Archibald C. Sayce was one of England's outstanding archaeologists with a natural
ability to learn and master ancient languages. In his younger days he was denied the position
of professor of Hebrew at Oxford, although considered eminently qualified, except that he
had a tendency towards the Assyrian-Babylonian theories of "Higher Criticism," and Dr.
Samuel Driver who received the chair was considered to be orthodox. In denying the position
to Dr. Sayce, although he had been recommended by Dr. Pusey who held the chair for some
64 years, Prime Minister Gladstone said he recognized the ability of Dr. Sayce but because
he was a leader in the German critical theology, therefore he was considered not to be "safe."
The result was, in the end, that Dr. Driver became the advocate of the destructive criticism,
and Dr. Sayce repented of the evil and returned to the defense of the Bible.

Dr. Andrew D. White, the persistent enemy of the Bible and Christianity as he knew it, in his
two volumes, \textit{A History of the Warfare of Science with Theology in Christendom}, on several
pages refers to Dr. Sayce in glowing terms as a great advocate of Biblical criticism and as an
outstanding archaeologist. One page 51, Vol. 1, for instance, he says this:

The Rev. Prof. Sayce, of Oxford, than whom no English-speaking scholar carries more
weight in a matter of this kind, has recently declared his belief that the Chaldaeo-Babylonian
theory was the undoubted source of the similar theory propounded by the Ionic philosopher
Anaximander—the Greek thinkers deriving this view from Babylonians through the
Phoenicians; he also allows that from the same source its main features were adopted into
both the accounts given in the first of the sacred books, and in the general view that most
eminent Christian Assyriologists concur.

It is true that these sacred accounts of ours contradict each other. In the part of the first or
Elohistic account given in the first chapter of Genesis the \textit{waters} bring forth fishes, marine
animals, and birds (Genesis 1:20.), but in that part of the second or Jehovistic account given
in the second chapter of Genesis both the land animals and the birds are declared to have
been created not out of the water, \textit{but out of the ground}. (Genesis 2:19.)

Considering the fact that we do not have any original documents, it is rather childish for Dr.
White to raise the question whether the fowl came from the sea or the land, when we have to
depend on faulty translations which is admitted by all Bible scholars. There are two matters,
however, in this statement by Dr. White in relation to Dr. Sayce that I cannot permit to pass
unnoticed. The first is that the scholarly and efficient Professor Sayce who was, as all admit,
a keen Bible student and archaeologist, discounted the Bible account and agreed that the
Hebrew story of the creation came from Chaldaeo-Babylonian sources. The second is that in
the Bible we have the Elohistic and Jehovistic accounts which do not agree. Let us consider
the case of Dr. Sayce first.

It is true that he was led astray by his earlier research and joined the Chaldaeo-Babylonian
group who based their conclusions, not on evidence, but false deductions. But Dr. Sayce was
big enough when he discovered his error to openly forsake it and return to his defense of the
Bible. It was before he entered Queen's College, Oxford, in 1865, that he took up with what
he called the "German theories." It was after "Hupfeld had published his dissection of
Genesis," and Sayce was impressed by it, and "Colenso had issued his first volume criticizing
the Pentateuch." In later years when Dr. Sayce had made further research, had done some
excavating and had translated other records, he began to change his views. Then came the
discovery of the Tel El-Amarna tablets from the Nile, the Siloam Pool manuscripts, and other
discoveries which gave a flood of light upon the ancient records which convinced him that
the Bible records were anterior to any Babylonian or Assyrian legends. He was convinced
that the art of writing and the making of written records antedated anything previously
believed to be known and that the records of Moses were ancient workmanship and evidently
written at the time of the occurrence of the events. He wrote: "The only winter which I did
not spend on the Nile was the one when the famous cuneiform tablets were found by the
fellahin at Tel El-Amarna." Some of these fell into his hands, others he had the privilege of
examining. About one third of them had been carelessly destroyed. He said: "Next to the
historical books of the Old Testament, the Tel El-Amarna tablets have proved to be the most
valuable records which the ancient civilized world of the East has bequeathed to us. What we
now have is an index of what we should have possessed had the collection been preserved
uninjured and intact." Higher criticism had determined that there could have been no Semitic
literature before the epoch of King David. The study of these tablets became the turning point
with Dr. Sayce and revolutionized his thinking, or turned him back again from the "German
theories." He learned that in the Mosaic age people were educated. 20

Dr. Sayce and Dr. Pinches translated some of the Babylonian Dynastic Tablets where they
discovered that the view that Belshazzar could not be discovered in profane history was false.
It had been the idea that the Book of Daniel was wrong in mentioning him and therefore that
the book could not be correct. I quote from an article by T. W. Fawthrop, entitled, \textit{The Stones
Cry Out: Scriptural Confirmation Often Overlooked}. This was published in the Transactions
of the Victoria Institute, Vol. 72, pages 137-148.

When certain professors were unable to find Belshazzar in profane history, they discarded the
Book of Daniel. Dean Farrar said, "History knows of no such king." But foundation-cycles
from Ur contain prayers of King Nabonidus for Belshazzar, his son. Other inscriptions record
Belshazzar's business transactions, and his death when the Persians entered Babylon.
Professors Sayce and Pinches show that as Solomon was co-king with David, so Belshazzar
reigned with Nabonidus, his father; one captained the troops in the field, the other defended
the city. So Belshazzar is found. Professor Sayce declared, "The higher criticism is now
bankrupt"; and Professor Pinches wrote, "I am glad to think, in the face of archaeology, with
regard to the Book of Daniel, that the higher criticism is, in fact, buried." Dr. Orr adds, "So
Professor McFadyen's apparent revellings in the inaccuracies of Daniel are all out-worn and
answered. Daniel's history is authentic. He knew Belshazzar because they both dwelt in
Babylon. Herodotus and Zenophone did not know him because they lived far away."
(\textit{Transactions}, Vol. 72, p. 146.)

SIR FREDERICK KENYON

In 1882 Dr. Kenyon became assistant in the Department of Manuscripts in the British
Museum and later became director and principal librarian. He was the author of numerous
books and articles on the Bible and archaeology and kindred subjects. He succeeded Sir
Charles Marston as the president of the Victoria Institute. He was a firm defender of the
Bible against the attacks of destructive criticism, yet liberal in his views that the Bible,
having passed through many hands, having been transcribed numerous times naturally would
contain some errors. On this point he said: "It is clear that the Bible records have not reached
us without some corruption in passing through human hands. There are in the first place
variations and not unimportant variations in which they have reached different peoples. The
Jew has them in the Masoretic Old Testament; the Greek Church in the Septuagint Old
Testament, and the New Testament which is often not in accordance with the oldest MSS.;
the Roman Church in the Vulgate; the Abyssinian in the Ethiopic version; we ourselves both
in our Authorized and in the Revised Version; and all of these are dependent upon hundreds
of manuscripts, no two of which have an absolutely identical text. Which of these is the
authoritative form of the Divine Revelation?" 21 This is the natural conclusion that a scholar
of the Bible would have to take. It is in perfect harmony with the doctrine of the Church:
"We believe the Bible to be the word of God as far as it is translated correctly." Moreover,
the angel informed Nephi that many changes would be made before it should reach our day,
and that important doctrines would be eliminated. And so Sir Frederick George Kenyon has
said, this is an insurmountable difficulty that the destructive critics have to face when they
endeavor to tell us which passages and even sentences are placed there by which author. It is
regrettable to us, one and all, that the Holy Scriptures have come down to us with many
corruptions, but for men of learning to examine them and by their natural ability, without the
aid of the Divine Spirit, which they do not have, and do not profess to have, dissect them into
fragments and assign each fragment to a certain time and writer is too much to believe.

Sir Frederick, however, believes the Bible to be the word of God expressed in his language
just as members of the Church do—as far as it is correctly translated. In his mild manner he
has come to the defense of the Holy Scriptures. In an address given May 22, 1950, in Claxton
Hall, Westminster, he said:

In the latter years of the nineteenth century the champions of Christianity were mainly on the
defensive. Natural Science was in the heydey of its progress which took rise in the
discoveries and doctrines of Darwin, and there were many who believed that Natural Science
held the key to all the problems of existence and that the day of religious belief was over. At
the same time, within the sphere of religious study itself, a school of thought asserted itself
which questioned the authenticity and trust-worthiness of the fundamental documents of
Christianity and applied the utmost freedom of scepticism to the narratives. "Advanced"
thought, as it called itself, flourished rampantly, and orthodoxy was pushed aside as an
outworn tradition, discredited by modern science and by modern scholarship. And against
this attitude the state of our knowledge of biblical archaeology did not supply arguments
which could effectively convince those who did not wish to be convinced. The advocate of
the Christian faith fought at a disadvantage and on the defensive.

Now all this is changed, and the point which I wish to make is that we are no longer on the
defensive. It is no longer the Christian scholar that is out of date. The up-to-date scholars are
now those who recognize the authenticity and authority of the Christian literature; it is the
critics who formerly claimed to be "advanced" who are now belated and behind the time. The
last half-century has been a period of wonderful, almost sensational, advance in our
knowledge of the conditions under which our religion took its form and in which the books
which contain its credentials were produced; and discovery after discovery has tended to
establish the essential soundness of the traditions which from the point of human scholarship,
are the title-deeds of our faith." 22

This renowned scientist continues and says that this great change came, in regard to the Old
Testament, "in the years lying around the turn of the century. Previously our knowledge of
the area lying between the Euphrates and the Nile was, except for the books of the Old
Testament, practically a blank. It was the accepted view that writing was unknown in all that
part of the world before the beginning of the first millennium." Grote, for Greece, put its
origin as late as the seventh century. Wellhausen, for the Hebrews, had it not earlier than the
ninth. "The Mosaic age was supposed to be far outside the scope of written records." Then
came the discovery of the Tel El-Amarna tablets in Egypt in 1887. These tablets proved that
writing was "habitual" as far back as the fourteenth century B.C., "but far more decisive were
the discoveries made in Babylonia where sites such as Telloh, Nippur, Ur, Kish, Warka and
others yielded thousands of tablets dating back as far as the third millennium B.C., or even
earlier." These, said Sir Frederick Kenyon, contain many literary and semi-literary works,
including the story of the flood. These records, "established beyond question two things of
vital importance for Old Testament scholarship—the early use of writing and the existence of
elaborate codes of laws far beyond the age of Moses."

"All of these discoveries have thrown a flood of light on the Old Testament literature, and
particularly on that part of it which was considered as historically the least reliable, namely,
the Pentateuch. . . . The boot is now, in fact, on the other leg. Instead of the Mosaic
legislation being whittled down to a few verses and regarding all the rest as later accretions,
the presumption now must be in favor of the antiquity and authenticity of the Mosaic
legislation."

This distinguished scholar also refers to the Dead Sea manuscripts and says of them: "Still
more recently we have received illuminating evidence which strengthens our confidence in
the reliability of the Old Testament as it has come down to us. I refer, of course, to the
discovery of Hebrew manuscripts in a cave near the Dead Sea. These include a nearly
complete copy of the Book of Isaiah, which is assigned by those who have studied it to the
late second or early part of the first century B.C. Hitherto the pedigree of the Hebrew text
could be carried back no further than the so-called Synod of Jamnia, in the last years of the
first century A.D." 23

There are scores of other scientists, archaeologists and educators today who bear this same
kind of testimony, but it is needless and beyond the scope of this writing to attempt to include
them all. However, from the competent advocates of Biblical authenticity, we may make a
brief summary. We have learned:

That writing and record keeping, from the archaeological discoveries were long anterior to
Abraham.

That many parts of the Bible formerly considered legendary and mythical, have been
confirmed by archaeological research.

That many events recorded in Genesis, previously viewed skeptically, are now confirmed by
records that have been discovered.

That Moses did not get the accounts of the Garden of Eden, the fall, the flood, from Assyrian
or Babylonian sources.

That the "four accounts," J., E., P. and D., are without any foundation in fact.

That there is no sound and justifiable reason for dividing the Book of Isaiah among two or
three authors, and placing a great part of it as late as the sixth or fifth centuries B.C.

That the earliest religion known was not worshiping multiple gods. The worship of idols and
multiple gods came later, contrary to what is written in every school textbook on ancient
history at the present day. The earliest religion was the worship of one God.

That the so-called "higher criticism," which is destructive criticism, is based on "unsound
assumptions," that are unreliable. 24

Above all of this, members of the Church have double assurance. We have the word of the
Lord that Isaiah wrote the book that bears his name. Chapters called in question by these
critics are found, or quoted in part, in the Book of Mormon. The Five Books of Moses were
in possession of the Nephites on this continent, and therefore Deuteronomy or any other part
could not have been written after Lehi left Jerusalem.

Our Savior quoted constantly from Isaiah, and the books of Moses and other parts of the Old
Testament. In his conflict with the devil in the wilderness the Savior gave all three quotations
from Deuteronomy. After his resurrection, when talking with the two disciples on the way to
Emmaus, it is written: "And beginning at Moses and all the prophets, he expounded unto
them in all the scriptures the things concerning himself." (Luke 24:27.) Later when he met
with the disciples in an upper room, it is written: "And he said unto them, These are the
words which I spake unto you, while I was yet with you, that all things must be fulfilled,
which were written in the law of Moses, and the prophets, and in the psalms concerning me."
(Luke 24:44.) This quotation the destructive critics should memorize:

And it came to pass, that the begger died, and was carried by the angels into Abraham's
bosom: the rich man also died, and was buried;

And in hell he lift up his eyes, being in torments, and seeth Abraham afar off, and Lazarus in
his bosom.

And he cried and said, Father Abraham, have mercy on me, and send Lazarus, that he may
dip the tip of his finger in water, and cool my tongue; for I am tormented in this flame.

But Abraham said, Son, remember that thou in thy lifetime receivedst thy good things, and
likewise Lazarus evil things: but now he is comforted, and thou art tormented.

And beside all this, between us and you there is a great gulf fixed: so that they which would
pass from hence to you cannot; neither can they pass to us, that would come from thence.

Then he said, I pray thee therefore, father, that thou wouldst send him to my father's house:

For I have five brethren; that he may testify unto them, lest they also come into this place of
torment.

Abraham saith unto him, They have Moses and the prophets; let them hear them.

And he said, Nay, father Abraham: but if one went unto them from the dead, they will repent.

And he said unto him, If they hear not Moses and the prophets, neither will they be
persuaded, though one rose from the dead. (Luke 16:22-31.)

\newpage
REFERENCES—CHAPTER TWENTY-SIX

Footnotes

1. Snell, Dr. H. C., \textit{Ancient Israel, Its Story and Meaning}, p. 12.

2. \textit{Ibid.}, p. 47.

3. 1 Cor. 2:10-12.

4. Snell, Dr. H. C., \textit{Ancient Israel, Its Story and Meaning}, pp. 5, 6, 24.

5. \textit{Ibid.}, Footnote, p. 24.

6. Yahuda, Dr. A. S., \textit{Accuracy of the Bible}, Introduction xxii

7. \textit{Ibid.}, p. xxiii.

8. \textit{Ibid.}, p. xxv.

9. \textit{Ibid.}, p. xxxii.

10. \textit{Ibid.}, p. 23.

11. Cobern, Dr. Camden M., \textit{Introduction to the New Archaeological Discovery, Introduction}
xv.

12. Marston, Sir Charles, \textit{New Bible Evidences, Preface}, p. 1-6.

13. \textit{Ibid.}, Preface, p. 9.

14. \textit{Ibid.}, pp. 24-25.

15. \textit{Ibid.}, p. 178.

16. \textit{Transactions, Victoria Institute}, Vol. 72:137-149; 77:104-111.

17. \textit{Ibid.}, Vol. 79, p. 222.

18. \textit{Ibid.}, Vol. 82, p. 224.

19. \textit{Ibid.}, Vol. 82, 227-228.

20. Marston, Sir Charles, \textit{New Bible Evidences}, p. 232.

21. Kenyon, Sir Frederick G., \textit{Victoria Transactions}, Vol. 82 (1950), page 225.22. In addition to what Dr. Marston has said, which none can successfully deny, we have the
word of the Lord to Nephi that by design many of the "plain and precious things" were taken
away from the book of the "Lamb of God." (1 Nephi 13:24-29.)

23. Dr. Marston wrote to President Rudger Clawson, from New York, Dec. 27, 1934, as
follows:

"Rev. Rudger Clawson

47 E. South Temple

Salt Lake City,

"Dear Sir:

"I am constrained to write a message for this New Year on the eve of my departure to
Palestine. It has become my mission in life to ascertain the reality or otherwise of the Bible.
For the past ten years I have been spending money on excavations in Bible lands. The time
has now come when the work already accomplished, and the results already attained, must be
made known. I invite your cooperation.

"You will have noticed the rapid advance of modern knowledge, and remarked how it has
overturned a great deal of what had previously been taught as assured truth. For example,
textbooks on the science of Physics have been rendered obsolete by the discovery of
Relativity. Now a similar fate has overtaken the so-called 'Scientific' criticism of the Old
Testament. The conjectures and speculations on which this criticism was based have proved
to be unsound, and incapable of sustaining the shocks of recent archaeological discoveries in
Bible lands. For details see my book, \textit{New Bible Evidence}, published by Fleming H. Revel
Company, of this city.

"The far reaching effect of the critical collapse can be better imagined than described in a
brief communication, such as this letter. The discoveries do not so far prove that the Old
Testament, as we understand it, is correct, word for word, or letter for letter; yet they tend to
satisfy the suggestion that it is substantially true. And the existing 'scientific' Bible criticism,
even as applied to the New Testament, must be regarded with grave suspicion.

"In face of the now recognized fact of the Reality of the Unseen, the bogy which has been
made of Bible miracles, is also being dissipated. The new science of Physical Research is
throwing a good deal of light on such happenings, and leaders of thought are recognizing we
do not yet fully understand.

"As you are aware, the Bible has been the basis on which the great civilization of this country
was reared. For years a materialistic school of speculative thought has been allowed to
confuse our minds as to the substantial reality of its contents. My message—a layman's
message—to America today is that this darkness is passing away, and that the evidence of
things 'unseen' is no longer to be relegated entirely to the region of Faith."

"Yours very truly,

Charles Marston."

24. The discovery of the writings at the Pool of Siloam, near Jerusalem, was made in 1880,
by some boys who were playing in a tunnel adjacent to the pool. This incident came to the
attention of Dr. Schick of Jerusalem, but he was incapable of reading them. Dr. A. C. Sayce
was in Jerusalem in 1881 and learning of these writings went to the Pool of Siloam, and by a
lighted candle and in the mud and water, made a copy of some of the writings cut in a stone
which he translated. It proved to be the oldest example of Hebrew writing up to that time
discovered. Through him these inscriptions, found by the boys, became known to the world.
It gave the record how the rock beneath Zion was tunneled simultaneously from the two ends
in order to bring the waters from the spring outside the city within the walls. The workmen
from the opposite ends exhibited great engineering skill in meeting "pick to pick." This
discovery confirms the story in 2nd Chronicles 32:20, which occurred in the reign of
Hezekiah. The passage is as follows:

"This same Hezekiah also stopped the upper watercourse of Gihon, and brought it straight
down to the west side of the city of David. And Hezekiah prospered in all his works."

This discovery was another sad blow to the destructive criticism.

