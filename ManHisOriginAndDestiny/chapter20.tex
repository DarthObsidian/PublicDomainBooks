\chapter{THE HISTORICITY OF JESUS}

IT is very strange that many scientists freely admit that Jesus Christ really did exist and they
are willing to acknowledge the superior quality of his teachings. Notwithstanding this they
studiously avoid reference to his crucifixion and resurrection. All reference to his identity,
even when he spoke of himself and declared himself to be the Son of God, they carefully
avoid. All the testimony given by his disciples and recorded in the epistles, is ignored, or
relegated to a later century as legendary lore. Of course they realize that to admit his divinity,
his resurrection, and the testimonies of Peter, John and Paul, would be a mortal blow to their
organic evolutionary theories. The attitude they take is only one step removed from denying
his earth existence. As it is, they look upon the Bible, as being a collection of ``old wives'
tales,'' and Dr. White in his two-volume work on the \textit{Warfare of Science with Theology},
ridicules and denies the miraculous stories, not only of the prophets in Israel, but also the
miraculous accounts in relation to the Savior himself. So, we may expect in a short time, if
conditions continue as they are today, that more and more the Only Begotten Son of God will
be eliminated entirely from their picture and will be as much a myth as the stories of the
Greek and Roman gods.

Mr. White has taken one step in this direction in his endeavor to do away with the brilliant
star which appeared at the birth of our Lord, by saying that such stories have been told of
Buddha, Chrishna in India, Yu, and Lao-tse in China, and in some Jewish traditions even at
the birth of Abraham and Moses. Likewise Kersey Graves, in his work, \textit{The World's Sixteen
Crucified Saviors}, goes to great effort to show that this story is legendary, and the whole
story of the birth and death of Jesus Christ is but the repetition of similar stories in all parts of
the world. He mocks at the birth of our Messiah. It may be true that in many nations there
was a story of the appearing of a great star at the birth of some outstanding religious teacher;
likewise of the birth and crucifixion of a god long before the birth of Jesus Christ. Like Sir
Henry H. Howorth said about the traditions of the flood, so likewise the fact that in India,
China, and many other countries, long before the birth of Jesus Christ, stories of this kind
were told and have crept into their mythology. Nor did they come there by chance, but from
the fact that the birth of Jesus Christ was known by revelation from the days of Adam.
Ancient prophets spoke of his coming, such as Enoch, Abraham, Moses, Isaiah and many
others. The story of his birth, the marvelous occurrences that would take place and likewise
the darkness etc. which would prevail at his death, were revealed to these ancient prophets.
We know that there was written on the original records which were copied on the Brass
Plates obtained by Nephi, some of the ancient prophecies, from Isaiah, Zenock, Neum, Zenos
and Nephi. There can be no question that the account of the star appearing, like the sacrifice
which he should make, were fully revealed in the beginning. Such accounts were recorded in
the Book of Enoch, of which we have obtained a glimpse, but which we are promised will in
the due time of the Lord be revealed. These stories concerning the coming of our Lord, his
crucifixion and resurrection, were taught to the people of Noah and the prophets who
succeeded him, and as the people turned away from the truth and scattered over the earth
these stories were carried with them. In course of time the names and circumstances changed,
as stories will, until they became legendary and various names were used instead of that of
Messiah, or Jesus Christ. We know that the stories of Adam and Eve, the fall, the flood, etc.,
are found hidden in the legendary mythology of most races.

These learned men, however, not knowing these truths, accept the stories of mythology and
place them in a position of contradiction to the true stories as they were revealed to the
ancient prophets. There are, nevertheless, a large number of influential men, educators and
scientists, who accept the story of Jesus Christ as recorded in the New Testament. There are
many others who have rejected the Bible, both Old Testament and New, who deny the very
existence of Jesus Christ in person on the earth. This list is gradually growing as the theories
of men lend impetus to such conclusions. Knowing this condition to exist and that the
number of skeptics was growing, Elder J. M. Sjodahl, a Hebrew scholar, wrote an article
which was published in the \textit{Improvement Era} in 1930, under the title: \textit{Jesus As A Personage
In History.} This was followed a short time later by another by Dr. William J. Snow, of the
Brigham Young University. These articles are timely and are here reproduced in full.

``JESUS AS A PERSONAGE IN HISTORY''

``Sometimes the question is asked whether there is any historical evidence, outside the books
of the New Testament, relating to the life and activity of Jesus on this earth. Such a question
might seem superfluous, but the fact is that some, who consider themselves scientifically
competent, have actually denied the reality of the life of our Lord in Palestine, and asserted
that the biographies of the evangelists are myths.

``In a close examination of this question it is important to remember that the kingdom of
Judah, at the time of the Savior, was of minor importance in the political geography of the
world. Riots and executions were numerous, and the appearance on the scene of the Son of a
carpenter from a village in far off Galilee, and his tragic fate, did not attract general attention
at first. Judea was so insignificant immediately after the Babylonian dispersion that Greek
historians hardly ever heard of it. The Macabees tried to restore it to its former importance,
but what they had gained was almost entirely lost during the Roman iron rule. We cannot
expect many historical references to Jesus in works from this time, except those penned by
his own followers.

``There are, however, some. Thus, in the Talmud, there are more or less mythical stories
concerning Ben Stada, Ben Pandera, Pappus Ben Jehuda, Miriam M'Gadd'la, Neshaya, and
Yeshu, which are by some supposed to refer to Jesus of Nazareth and some of his
contemporaries. The stories may be distorted, but even so, they prove that there is an
historical basis.

``There is also a book, \textit{Toledoth Yeshu} (Biography of Jesus) which possibly was circulated
among the Jews as early as the fifth century of our era, and which undoubtedly contains some
items from an earlier day, but the narrative is so distorted that it is worthless for historical
purposes.

``Josephus is a better witness. Some have supposed that what little he says of Jesus is
interpolated by early Christians, but that view is no longer generally accepted by scientific
criticism. Dr. Joseph Klausner, a Jewish scholar in his history of \textit{Jesus of Nazareth}, says of
this question:

```There are not sufficient grounds for supposing this whole to be spurious. Josephus treats of
the life and death of John the Baptist at fair length, and what he says does not at all
correspond with the gospel account, and there is no reason, therefore, to suspect Christian
copyists of interpolating this section as well, as does Graetz. According even to Shurer, ``the
genuiness of this passage is only rarely open to question.'' It is remarkable that Josephus tries
his hardest to conceal from his readers that John preached the coming of the Messiah (the
reason which we have mentioned); in order to make the episode comprehensible to Greek
readers he describes John the Baptist as ``a good man who commanded the Jews to exercise
virtue, both as to righteousness towards one another, and piety towards God, and so come to
baptism.'' Even the three Jewish parties, the Sadducees, the Pharisees and the Essenes,
Josephus explains in terms of philosophic schools, all with a view of making himself
understood by his Gentile readers.

```And he did precisely the same with Jesus; he described him as a ``wise man'' just as he
described John the Baptist as ``a good man``; he described Jesus as a ``teacher of such men as
received the truth with pleasure,'' just as he described John the Baptist as one who ``called
upon the Jews to exercise virtue, etc.,'' and he described Jesus as a ``doer of wonderful works''
(for Josephus himself was a firm believer in miracles). He could say of Jesus that ``he drew
after him many Jews and also Greeks,'' because the church contained many Greeks at the
time of writing, 93 C.E., and ancient historians had the habit of judging earlier conditions
from later times. It was also Josephus who wrote that ``they who loved him at the first did not
cease to do so even after Pilate had condemned him to crucifixion at the suggestion of the
principal men among us,'' and that the ``race'' (or tribe) of Christians, so named from him, are
not extinct to this day.'''

``The quotation from Josephus is as follows:

```Now there was about this time, Jesus, a wise man [if it be lawful to call him a man.] For he
was a doer of wonderful works, a teacher of such men as received the truth with pleasure. He
drew over to him both many of the Jews and many of the Gentiles. [He was the Messiah];
and when Pilate, at the suggestion of the principal men among us, had condemned him to the
cross, those that loved him at the first ceased not, [for he appeared to them alive again the
third day, as the divine prophets had foretold these and ten thousand other wonderful things
concerning him]; and the race of Christians, so named from him, are not extinct even now.'

``These words with brackets are universally admitted to be interpolations, but the other words
are now ascribed to Josephus, and from a powerful testimony for the historical character of
Jesus, our Lord.

``In another connection, Josephus tells how Annas, the son of Annas, the high priest, brought
before the Sanhedrin one by the name of James, 'the brother of Jesus who was called the
Messiah,' and others whom he regards as law-breakers. This, however, cost the high priest
his office. According to Josephus, he was deposed by Agrippa II, and another was appointed
in his stead.

``Among other witnesses for the historical existence of Jesus is Suetonius, the Roman. He
relates that the Jews were expelled from Rome on account of a tumult concerning one
'Christus.' This is supposed to have happened in the year 49 A.D.''

``THE HISTORICITY OF JESUS''

By Dr. William J. Snow, of the Brigham Young University

``In the March (1930) \textit{Improvement Era}, J. M. Sjodahl presents a rather timely and suggestive
article on Jesus as an Historical Personage. Perhaps some further elaboration of this subject
would be helpful to many of the readers of the \textit{Era}. At least the question raised for discussion
is one that has confronted the Christian world in recent years and aroused controversy as to
the evidence in the case.

``The writer of this brief article had this matter forcibly called to his attention when, in answer
to a phone call a short time ago, he received the query, 'Do you know of any proof outside
the New Testament that Jesus ever lived?' Further discussion disclosed the fact that an
energetic teacher of a senior class in Mutual had been challenged by this very question.
Having perfect faith in the life and mission of Jesus Christ, the teacher had never entertained
a thought that doubts of his historicity existed anywhere.

``However, it is generally known among students of comparative religions and of the Genesis
of Christianity, that the whole gospel story has been challenged and seriously discussed for
more than half a century. Fortunately, for doubting Thomases 1 and happily for those who,
with faith unshaken, still desire to see their assurances confirmed by extra Biblical evidence,
the scholars quite generally, both in Europe and America have reached the conclusion that
the traditional story of Christ's life is essentially true, that he did actually live, and that his
influence has reached down through the centuries to the present. Says Carpenter, an English
scholar of great renown who still remains in the camp of the skeptics, 'Nevertheless, I need
hardly remark that large and learned as the body of opinion here represented is,' (he has been
quoting authorities who purport to prove the Jesus story a myth) 'a still larger (but less
learned body) fight desperately for the actual historicity of Jesus.' 2

``It probably can be said that this larger body is constantly increasing. Hopkins of Yale, an
Assyrian scholar and profound student of unquestioned learning, declares emphatically, 'The
story of Christ is no myth.' 3 This is typical of Bible scholars whose sole desire is to set forth
the truth deduced after carefully evaluating the evidence.

``It is apropos at this point to suggest the basis of the negative position. There are two
grounds of attack; one growing out of and depending for its validity on the other. The basic
assumption then is that since there is so little mention of Jesus outside the New Testament
story the probability is that he is an invention of Christian writers. Arthur Drews, a professor
in the University of Karlsruhe, 4 wrote a book in 1910 entitled, \textit{The Christ Myth}. In this work
he mentions the German, Bauer, as contending that Jesus was a pure invention of Mark's.
Then follows a discussion of the legendary theory and the authorities who have given a
reasoned exposition of this theory. Among them is the American, W. Benjamin Smith, author
of \textit{The Pre-Christian Jesus}. (1906).

``Briefly then, what is the theory? It is based essentially upon the fact that Christianity
evolved in a world in which belief in a Savior God was general. To satisfy the longing of the
people for a religion of redemption, various pre-Christian Saviors had appeared. These
Savior Gods came to earth, took bodies, died and were resurrected and finally were to return
to raise the dead and annihilate all evil. Such where Adonis, Osiris, Ceyble, Krishna, and
Mithra. The last named was miraculously born from a rock, came with a great redeeming
light like the sun, initiated devotees by lustrations of water and blood (baptism), brought his
earthly career to an end in a last supper; and then ascended to heaven where he continued his
supernatural help to his devotees, and from whence he would come to resurrect and redeem
them all in the last days. 5

``Now these various deities, accompanied as they were by mysterious cults and practices,
were nevertheless mythical; they had no actual existence, Christianity arising in the world
filled with such beliefs, must likewise have a founder, hence Christ, the Messiah, was
invented. 6 Such was the argument. Case faces all these arguments fairly and estimates their
value. He says, 'When all the evidence brought against Jesus' historicity is surveyed it is
found to contain no elements of strength. All theories that would explain the rise of the New
Testament literature by making it a purely fictitious product, fail.' 7

``Now as to the evidence of lay historians in the Roman world. It must be admitted there are
but few authenticated references. This is not to be wondered at, however, as an obscure
character in a remote part of the Roman empire would not arouse great interest or concern
among Pagan writers. Moreover, it must be granted that our chief evidence is the gospel
narratives and the writings of Paul. These, however, stand a most rigid test. Says Hopkins, 8
'Within almost a generation of his death, the words and activities of Jesus and his immediate
followers were committed to writing. This account is too near the event to justify doubt as to
the historicity of Jesus.'

``But there is supporting testimony worthy of consideration—extra Biblical evidence that
cannot be well gainsaid. Clement of Rome, writing near the end of the first century declared,
'The apostles received the gospel for us from the Lord, Jesus Christ.' Of course the Christian
tradition generally accepted Christ without question. Heretics, against whom Ignatius and
others warned the followers, did not question at all the actual appearance of Jesus on earth.

``That this Christian tradition was accepted by Roman writers who at least casually mention
Christ is of vital importance. Pliny in a letter to Trajan (112 A.D.) asks anxiously about his
duties with reference to Christians in his province of Bithynia. He seems to think there is
little danger, from them, that the superstition is dying out. He found some, he said, who
offered incense to Caesar and cursed Christ. He writes as though the actuality of Christ's life
is well known. He tells us nothing in particular about him, but finds that the center of
Christian worship is Christ to whom they sing hymns of praise.

``Suetonius, in his lives of the Twelve Caesars (Ca. 120 A.D.) twice apparently refers to
Christianity. While there is considerable vagueness about his reference to one Christus who
created a disturbance among the Jews in Rome, it is fair to presume that he knew there were
followers of a character named Christ.

``Be that as it may, Tacitus 9—a Roman historian of great note, refers explicitly and
definitely to Christ after whom the Christians whom Nero persecuted were named. Moreover,
he gives the information that Jesus (clearly the Jesus of the gospel history) was put to death
by Pontius Pilate in the reign of Tiberius Caesar. Tacitus lived in the latter part of the first
century and the early second century. His annals date about 115 A.D.

``Here then are three Roman writers who may be cited in support of the gospel story of the
historic Jesus, viz., Pliny, Suetonius, and Tacitus. For Jewish writers—Extra Biblical—
Josephus is now pretty generally accredited as a supporting witness as set forth by Elder
Sjodahl in the March \textit{Era}.

``In the face, then, of a well accredited Jesus, the labored efforts of some great scholars to
give a legendary account of him falls to the ground.'' 10

In the September \textit{Era} of that same year (1930) I added another article to the two previously
given from which I quote the following:

``I would like to add a few reflections for the benefit of our young people who may be
disturbed by this modern criticism of the historicity of Jesus Christ, for it is admitted that
some distinguished scholars have advocated this astonishing, and, to us, impossible view.
Evidence produced of the time of our Savior and immediately following, has been presented
conclusively in the articles mentioned. It is not my purpose to cover the same ground, except
to say that many who deny the divinity of Jesus Christ are convinced of his historicity. One
of the most persistent and determined foes of Jesus Christ in modern times admits that the
evidence is beyond reasonable dispute and that Jesus Christ lived and taught the people in
Judea. Moreover, he declares Paul, the chief writer of the epistles and advocate of Jesus
Christ, was a real personality who came in contact with the Christians within the first decade
after the death of Christ.

```Paul . . . habitually speaks of Cephas and others who were actual companions of Jesus. We
have to deny the genuineness of all the epistles to doubt this. . . . So he (Paul) joined the
Christian body and mingled with them in Jerusalem, within less than ten years of the
execution of Jesus. No Jew there seems to have told him that Jesus was a mere myth. In all
the bitter strife of Jew and Christian the idea seems to have occurred to nobody. Setting aside
the Gospels entirely, ignoring all the Latin writers are supposed to have said in the second
century, we have a large and roughly organized body of Christians at the time when men
were still alive who remembered events of the fourth decade of the century.

```I conclude that it is more reasonable to believe in the historicity of Jesus. There is no
parallel in history to the sudden growth of a myth and its conversion into a human personage
in one generation. . . . From the earliest moment that we catch sight of Christians in history
the essence of their belief is that Jesus was an incarnation, in Judea, of the great God of the
universe. . . . So it seems to me far more reasonable, far more scientific, far more consonant
with the fact of religious history which we know, to conclude that Jesus was a man who was
gradually turned into a God. 11

``he point I wish to make, however, is that we have `a more sure word of prophecy,' as Peter
might put it, `whereunto we do well that we take heed,' by which we may know that Jesus
Christ lives and is indeed the Only Begotten Son of God.

``The Book of Mormon, while an ancient record, has come to light within the knowledge of
this generation. We all know how it was revealed and how it was translated, and that the
Lord raised up witnesses, 'as seemeth him good,' who testified 'to the truth of the Book and
the things therein.'

``Moreover the Book of Mormon was preserved as it is recorded, to come forth in the latter
days to bear witness of the truth of the record of the Jews (Bible), and to bear witness 'to the
convincing of the Jew and Gentile that JESUS is the CHRIST the ETERNAL GOD,
manifesting himself unto all nations.' The Book of Mormon bears record of the personality
and reality of Jesus Christ, both by prophecy uttered hundreds of years before he was born
and also of his personal appearance among the ancient people on this American continent. In
this sacred volume we have his words recorded and the testimony of witnesses who saw him
and unto whom he administered after his resurrection.

``However, we are not dependent upon the writings and testimony of those who lived and
wrote in ancient times. Although we accept their sayings. We have the testimony of
witnesses of our own time, Joseph Smith, Oliver Cowdery, Sidney Rigdon, and others have
borne witness to the world—as they were commanded to do—that they saw Jesus Christ,
conversed with him, were ministered to by him and received from him instruction. These
facts are recorded as they were written at the time. This testimony has gone forth into all the
world and has been before the world for over one hundred years. Joseph Smith and Oliver
Cowdery were in the presence of the Lord Jesus Christ in the Kirtland Temple, April 3, 1836,
and heard his voice. Joseph Smith and Sidney Rigdon were in his presence February 16,
1832, and have given their testimony as follows:

```And now, after the many testimonies which have been given of him: this is the testimony,
last of all, which we give of him, \textit{That he lives!}

```For we saw him, even on the right hand of God; and we heard the voice bearing record that
he is the Only Begotten of the Father—

```That by him, and through him, and of him, the worlds are and were created, and the
inhabitants thereof are begotten sons and daughters unto God.' 12

``This testimony has gone forth unto all the world. There are thousands who know it is true
for they too have had witness borne in upon their souls. There are thousands who believe in
the promise of the Lord, 'That every soul who forsaketh his sins and cometh unto me, and
calleth on my name, and obeyeth my voice, and keepeth my commandments, shall see my
face and know that I am; And that I am the true light that lighteth every man that cometh into
the world.''' 13

Another very remarkable prophecy concerning the attitude of the people in the world in the
latter days towards Jesus Christ and the authenticity of the scriptures, is recorded in the Book
of Mormon. The Lord revealed to Nephi between five and six hundreds years before the birth
of Jesus Christ that there would be an apostasy in the latter days and people would be
denying the predictions of the prophets, the record coming forth from the Jews, and the
divine mission of the Son of God. In fact the revelations to Nephi are very plain that these
conditions would arise. Moreover, the coming forth of the Book of Mormon as a new witness
to the world in the dispensation of the Fulness of Times, was in large measure to bear witness
of the authenticity of the Bible and to bear record to the divinity of Jesus Christ. From the
great vision given him in the presence of an angel, the following is recorded:

``For, behold, saith the Lamb: I will manifest myself unto thy seed, that they shall write many
things which I shall minister unto them, which shall be plain and precious; and after thy seed
shall be destroyed, and dwindle in unbelief, and also the seed of thy brethren, behold, these
things shall be hid up, to come forth unto the Gentiles, by the gift and power of the Lamb.

``And in them shall be written my gospel, saith the Lamb, and my rock and my salvation.

``And blessed are they who shall seek to bring forth my Zion at that day, for they shall have
the gift and the power of the Holy Ghost; and if they endure unto the end they shall be lifted
up at the last day, and shall be saved in the everlasting kingdom of the Lamb; and whoso
shall publish peace, yea, tidings of great joy, how beautiful upon the mountains shall they be.

``And it came to pass that I beheld the remnant of the seed of my brethren, and also the book
of the Lamb of God, which had proceeded forth from the mouth of the Jew, that it came forth
from the Gentiles unto the remnant of the seed of my brethren.

``And after it had come forth unto them I beheld other books, which came forth by the power
of the Lamb, from the Gentiles unto them, unto the convincing of the Gentiles and the
remnant of the seed of my brethren, and also the Jews who were scattered upon all the face of
the earth, that the records of the prophets and of the twelve apostles of the Lamb are true.

``And the angel spake unto me, saying: \textit{These last records, which thou hast seen among the
Gentiles, shall establish the truth of the first, which are of the twelve apostles of the Lamb,
and shall make known the plain and precious things which have been taken away from them;
and shall make known to all kindreds, tongues, and people, that the Lamb of God is} THE
SON OF THE ETERNAL FATHER, \textit{and the Savior of the world; and that all men must
come unto him, or they cannot be saved.}

``And they must come according to the words which shall be established by the mouth of the
Lamb; and the words of the Lamb shall be made known in the records of thy seed, as well as
in the records of the twelve apostles of the Lamb; wherefore they both shall be established in
one; for there is one God and one Shepherd over all the earth.

``And the time cometh that he shall manifest himself unto all nations, both unto the Jews and
also unto the Gentiles; and after he has manifested himself unto the Jews and also unto the
Gentiles, then he shall manifest himself unto the Gentiles and also unto the Jews, and the last
shall be first, and the first shall be last.'' 14

These \textit{last records} which were to come forth to bear witness of the ``book of the Lamb of
God,'' which is the Bible, are the Book of Mormon, the Doctrine and Covenants, and the
revelations of the Lord to Joseph Smith. It is evident, in this prophecy that they were to come
forth in a day when people would be denying the authority and authenticity of the books of
the Bible. When they would be criticizing them, taking from them all divine inspiration and
declaring that Jesus Christ is not the Only Begotten Son of God! Therefore the Lord would
establish his Marvelous Work, open the heavens and take out of the earth the record of his
ancient people of the tribes of Joseph, which would speak ``out of the dust,'' and bear witness
for the Bible and the Son of God!

All through the history of the Nephite nation, when they were serving the Lord, they talked
and wrote about the glorious day of this restoration, and the last words recorded by Moroni,
who sealed the records up and buried them in the dust, dealt with this theme:

``Wherefore, it is an abridgment of the record of the people of Nephi, and also of the
Lamanites—Written to the Lamanites, who are a remnant of the house of Israel; and also to
Jew and Gentile—Written by way of commandment, and also by the spirit of prophecy and
of revelation—Written and sealed up, and hid up unto the Lord, that they might not be
destroyed—To come forth by the gift and power of God unto the interpretation thereof—
Sealed by the hand of Moroni, and hid up unto the Lord, to come forth in due time by way of
the Gentile—The interpretation thereof by the gift of God.

``An abridgment taken from the Book of Ether also, which is a record of the people of Jared,
who were scattered at the time the Lord confounded the language of the people, when they
were building a tower to get to heaven—Which is to show unto the remnant of the House of
Israel what great things the Lord hath done for their fathers; and that they may know the
covenants of the Lord, that they are not cast off forever—\textit{And also to the convincing of the
Jew and Gentile} that JESUS is the CHRIST, the ETERNAL GOD, \textit{manifesting himself unto
all nations}—And now, if there are faults they are the mistakes of men; wherefore, condemn
not the things of God, that ye may be found spotless at the judgment seat of Christ.'' 15

\newpage
REFERENCES—CHAPTER TWENTY

Footnotes

1. It has been common practice to call persons who show any spirit of doubt, ``doubting
Thomases.'' This is, of course, based on the fact that Thomas, the apostle, declared when told
by his brethren that the risen Lord had appeared to them in his absence, that he would not
believe their words, ``Except I shall see in his hands the print of the nails, and put my finger
into the print of the nails, and thrust my hand into his side, I will not believe.'' Eight days
later the Lord came again and here we have one of the most touching stories in the New
Testament. Thomas was with them. Jesus came, the doors being shut, and stood in the midst
of them. Then said he to Thomas: ``Reach hither thy finger, and behold my hands; and reach
hither thy hand, and thrust it into my side: and be not faithless, but believing.''

``And Thomas answered and said unto him, My Lord and my God.''

``Jesus saith unto him, Thomas, because thou hast seen me, thou hast believed: blessed are
they that have not seen, and yet have believed.''

This is unfair to say of Thomas because he was no different than the other disciples and
apostles. Matthew records: ``And when they saw him, they worshiped him: but some
doubted.'' And Luke wrote: ``It was Mary Magdalene, and Joanna, and Mary the mother of
James, and other women that were with them, which told these things unto the apostles. And
their words seemed to them as idle tales, and they believed them not.'' It seems we have been
unfair to Thomas.

2. Carpenter, Edward—\textit{Pagan and Christian Creeds—Their Origin and Meaning}, p. 210.

3. Hopkins, E. Washburn—\textit{History of Religion}, p. 552.

4. Vide Carpenter, \textit{op-cit.}, p. 209.

5. For an interesting discussion of this subject see Angus S., \textit{Mystery Religions and
Christianity}, passim, Cf. Case, Shirley Jackson, \textit{Experience with the Supernatural in Early
Christian Times}, Chap. IV—\textit{Heroic Redeemers.}

6. Carpenter, \textit{op. cit}. pp. 210-221, Cf. Case, Shirley Jackson, The Historicity of Jesus,
Chapters II, III and IV.

7. See Case \textit{op. cit}. pp. 130-132. Case allows the leading negative scholars to play their
trump cards and then turns the trick with scholarly evidence.

8. Hopkins, E. W., \textit{The History of Religions}, p. 552, Cf. Drake, Durant. \textit{Problems of Religion},
pp. 63-64.

9. Tacitus, \textit{Annal} XV—44.

10. For a thorough discussion of the whole question, see Case, S. J., \textit{The Historicity of Jesus.}

11. McCabe, Joseph, \textit{The Story of Religious Controversy}, p. 228.

12. D. \& C. 76:22-24.

13. \textit{Ibid.}, 93:1-2.

14. 1 Nephi 13:35-42.

15. \textit{Title Page}, Book of Mormon.

