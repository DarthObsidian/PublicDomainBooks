\chapter{IMMORTALITY AND ETERNAL LIFE}

THE Lord informed Moses that his work and his glory are ``to bring to pass the immortality
and eternal life of man.'' In these words our eternal Father reveals the final destiny of man. It
is his great pleasure that man may prove himself obedient to the divine laws which govern
his celestial kingdom and by doing so obtain a fulness of joy. Lehi, in giving counsel to his
son Jacob, said:

But behold, all things have been done in the wisdom of him who knoweth all things.

Adam fell that men might be; and men are, that they might have joy.

And the Messiah cometh in the fulness of time, that he may redeem the children of men from
the fall. And because that they are redeemed from the fall they have become free forever,
knowing good from evil; to act for themselves and not to be acted upon, save it be by the
punishment of the law at the great and last day, according to the commandments which God
hath given.

Wherefore, men are free according to the flesh; and all things are given them which are
expedient unto man. And they are free to choose liberty and eternal life, through the great
mediation of all men, or to choose captivity and death, according to the captivity and power
of the devil; for he seeketh that all men might be miserable like unto himself. 1

This mortal life is a probationary state allotted to man to prove him to see if through the
experiences obtained here he will abide in the laws and commandments which pertain to the
exaltation. 2 Our contacts in this mortal life, their pleasure and pains, temptations and
resistance to temptations and sin, all count towards our final destiny. If we are willing to
abide in divine law as it has been revealed from heaven by messengers from the Father and
the Son, we shall receive the highest reward. This reward is eternal life, ``which gift is the
greatest gift of God.'' 3 He who obtains eternal life will become a son of God, a joint heir
with Jesus Christ, 4 and the Father promises him the fulness of the blessings of his kingdom.
5

Our Savior made these truths clear in his discourses and instructions to the Jews. Here are a
few quotations:

He that honoureth not the Son honoureth not the Father which hath sent him.

Verily, verily, I say unto you, He that heareth my word, and believeth on him that sent me,
hath everlasting life, and shall not come into condemnation; but is passed from death unto
life. 6

Verily, verily, I say unto you, He that believeth on me hath everlasting life.

I am that bread of life.

Your fathers did eat manna in the wilderness, and are dead.

This is the bread which cometh down from heaven, that a man may eat thereof, and not die.

I am the living bread which came down from heaven; if any man eat of this bread, he shall
live forever: and the bread that I will give is my flesh, which I will give for the life of the
world. 7

These words spake Jesus, and lifted up his eyes to heaven, and said, Father, the hour is come;
glorify thy Son, that thy Son also may glorify thee:

As thou hast given him power over all flesh, that he should give eternal life to as many as
thou hast given him.

And this is life eternal, that they might know thee the only true God, and Jesus Christ, whom
thou hast sent. 8

In saying that those who believe and obey him will pass from death unto life, and shall never
die, he had no reference to the separation of the spirit and the body in the temporal death, but
that they will never partake of the second death. The second death, which will be pronounced
on the wicked, is banishment from the presence of God. 9 This is spiritual death in which
those who par-take of it are denied any guidance of the Holy Spirit, and are placed in outer
darkness.

In the resurrection the spirit and body of every creature will be united again, never to be
divided, thus they become immortal. This immortality will come to all, both the wicked and
the righteous; but only those who overcome their sins and accept and keep all of the
commandments, will receive eternal life. The Lord said to the disciples whom he chose
among the Nephites:

And my Father sent me that I might be lifted up upon the cross; and after that I had been
lifted up upon the cross, that I might draw all men unto me, that as I have been lifted up by
men even so should men be lifted up by the Father, to stand before me, to be judged of their
works, whether they be good or whether they be evil—

And for this cause have I been lifted up; therefore, according to the power of the Father I will
draw all men unto me, that they may be judged according to their works.

And it shall come to pass, that whoso repenteth and is baptized in my name shall be filled;
and if he endureth to the end, behold, him will I hold guiltless before my Father at that day
when I shall stand to judge the world.

And he that endureth not unto the end, the same is he that is also hewn down and cast into the
fire, from whence they can no more return, because of the justice of the Father.

And this is the word which he hath given unto the children of men. And for this cause he
fulfilleth the words which he hath given, and he lieth not, but fulfilleth all his words.

And no unclean thing can enter into his kingdom; therefore nothing entereth into his rest save
it be those who have washed their garments in my blood, because of their faith, and the
repentance of all their sins, and their faithfulness unto the end. 10

The exaltation to the celestial kingdom is so great that the Father is fully justified in making
it dependent upon strict obedience to \textit{all} of his commandments. The celestial kingdom is a
kingdom of perfection. All who enter there must be thoroughly tried and proved and become
perfect to inherit it. The Lord has said that through their obedience those who enter must be
sanctified from all unrighteousness. Every law governing it must be obeyed. There can be no
opposition to divine law, nor could any one receiving this reward have any desire to change
or disagree with anything prevailing there, for these laws are perfect. As well may a man in
the mortal world object to the law of gravity or any of the other fixed laws of nature, as to
object to the laws of the celestial kingdom. They have been tried, proved and are eternal.
This being the fact there can be nothing but peace and joy in that kingdom.

Exalted beings, because they have proved themselves by obedience to ``every word that
proceedeth forth from the mouth of God,'' will become perfect and be like him, and as heirs
will become gods themselves. The history of mankind has revealed most clearly, that from
the beginning men have been rebellious, with few exceptions, disobeying the laws of God
that would bring them to perfection. The words of the Savior in the Sermon on the Mount are
full of meaning; but they have been ignored and in many instances misunderstood. As an
example, when he said to those who were present on the memorable occasion, ``Be ye
therefore perfect, even as your Father which is in heaven is perfect,'' some argue that he did
not mean just what he said. It has been maintained by some members of the Church that he
meant this relatively, for we cannot be perfect as God is perfect. The fact is, however, that he
intended it to mean just what he said—for those who believe on him to seek the same kind of
perfection which his Father has. He was not speaking as pertaining to mortality, but with the
larger view embracing eternity itself. We well understand that mortal man cannot be perfect,
but the immortal man can. To reach that condition will take time and we have eternity for it,
for we are destined to live forever. In the revelations given to the Church in this last
dispensation this matter of perfection, yet to come, is made very clear. One of the most
profound thoughts ever given by revelation is this given to the Prophet Joseph Smith:

And that which doth not edify is not of God, and is darkness.

That which is of God is light; and he that receiveth light, and continueth in God, receiveth
more light; and that light groweth brighter and brighter until the perfect day. 11

It is here made perfectly plain that it is possible for man, \textit{if he will continue in God}, to obtain
eventually the fulness of light and this light is knowledge and wisdom. But this will not come
in the few years allotted to man in mortality. Again the Lord said:

And this greater priesthood [i.e. Melchizedek Priesthood] administereth the gospel and
holdeth the keys of the mysteries of the kingdom, even the keys of the knowledge of God.

Therefore, in the ordinances thereof, the power of godliness is manifest.

And without the ordinances thereof, and the authority of the priesthood, the power of
godliness is not manifest unto men in the flesh;

For without this no man can see the face of God, even the Father, and live. 12

Once again the Lord said:

The Spirit of truth is of God. I am the Spirit of truth, and John bore record of me, saying: He
receiveth a fulness of truth, yea, even of all truth;

And no man receiveth a fulness unless he keepeth his commandments.

He that keepeth his commandments receiveth truth and light, until he is glorified in truth and
knoweth all things. 13

In these scriptures the Lord most emphatically declared that it is impossible for man to
become like God without the Priesthood and obedience to his commandments. Man has the
power to know all things, to become perfect and be bathed in light, knowledge and wisdom,
if he will only humble himself and walk in the light and truth. The man who refuses and lives
bound within his own wisdom can never attain to these great blessings of exaltation and
progression. A man must have, and be obedient to, the power of the Priesthood; he must be in
full harmony and fellowship with God from whom all knowledge, wisdom and power come.
No matter how much knowledge a man may gain, in this life or in the life to come, he cannot
obtain the fulness unless he holds and magnifies the Priesthood and \textit{continueth in God!} The
power, knowledge and wisdom in their fulness, will never be exercised by those who reject
the counsels and covenants of the Gospel of Jesus Christ. These are the possessions to be
given to the just and true, who become members of the Church of the Firstborn.

And who overcome by faith, and are sealed by the Holy Spirit of promise, which the Father
sheds forth upon all those who are just and true.

They are they who are the Church of the Firstborn.

They are they into whose hands the Father has given all things—

They are they who are priests and kings, who have received of his fulness, and of his glory;

And are priests of the Most High, after the order of Melchizedek, which was after the order
of Enoch, was was after the order of the Only Begotten Son.

Wherefore, as it is written, they are gods, even the sons of God—

Wherefore, all things are theirs, whether life or death, or things present, or things to come, all
are theirs and they are Christ's and Christ is God's.

And they shall overcome all things. 14

Salvation in the kingdom of God requires far more than the mere confessing him with the
lips, or believing in him with the mouth. It requires a constant and faithful adherence to his
commandments. There are many commandments, but they are not difficult to bear. He said:

Come unto me, all ye that labour and are heavy laden, and I will give you rest. Take my yoke
upon you, and learn of me; for I am meek and lowly in heart: and ye shall find rest unto your
souls.

For my yoke is easy, and my burden is light.

Moreover he said:

He that loveth father or mother more than me is not worthy of me: and he that loveth son or
daughter more than me is not worthy of me.

And he that taketh not his cross, and followeth after me, is not worthy of me. 15

We learn from the words of the Lord, that in order to obtain a celestial glory we must live a
celestial law. Nothing short of this will suffice, and also we must endure in faith to the end of
our lives. The Prophet Joseph Smith has given the world some valuable advice in relation to
their salvation. Some of his sayings are here given:

We consider that God has created man with a mind capable of instruction, and a faculty
which may be enlarged in proportion to the heed and diligence given to the light
communicated from heaven to the intellect; and that the nearer man approaches perfection,
the clearer are his views, and the greater his enjoyments, till he has overcome the evils of his
life and lost every desire for sin; and like the ancients, arrives at the point of faith where he is
wrapped in the power and glory of his Maker and is caught up to dwell with him. But we
consider that this is a station to which no man ever arrived in a moment; he must have been
instructed in the government and laws of that kingdom by proper degrees, until his mind is
capable in some measure of comprehending the propriety, justice, equality, and consistency
of the same. 16

Add to your faith knowledge, etc. This principle of knowledge is the principle of salvation.
This principle can be comprehended by the faithful and diligent; and every one that does not
obtain knowledge sufficient to be saved will be condemned. The principle of salvation is
given us through the knowledge of Jesus Christ. 17

There are a great many wise men and women too in our midst who are too wise to be taught;
therefore they must die in their ignorance, and in the resurrection they will find their mistake.
Many seal up the door to heaven by saying, ``So far God may reveal and I will believe.''
All men who become heirs of God and joint heirs with Jesus Christ will have to receive the
fulness of the ordinances of his kingdom; and those who will not receive all the ordinances
will come short of the fulness of that glory, if they do not lose the whole. 18

We are only capable of comprehending that certain things exist, which we may acquire by
certain fixed principles. If men would acquire salvation, they have got to be subject, before
they leave this world, to certain rules and principles, which were fixed by an unalterable
decree before the world was.

The disappointment of hopes and expectations at the resurrection would be indescribably
dreadful. 19

The question is frequently asked, ``Can we not be saved without going through with all these
ordinances?'' I would answer, No, not the fulness of salvation. Jesus said, ``There are many
mansions in my Father's house, and I will go and prepare a place for you.'' \textit{House} here named
should have been translated kingdom; and any person who is exalted to the highest mansion
has to abide a celestial law, and the whole law too.

But there has been a great difficulty in getting anything into the heads of this generation. It
has been like splitting hemlock knots with a corn-dodger for a wedge, and a pumpkin for a
beetle. Even the saints are slow to understand. 20

There is throughout the Christian world the prevailing thought that heaven where the
righteous will go is some place far off on some glorious sphere which is the home of our
Eternal Father. He revealed to Abraham that his throne is near to Kolob, the great governing
star of our universe. The heaven to which the righteous will go who dwell upon this earth,
will be right here upon this earth, for the Lord said, ``Blessed are the meek: for they shall
inherit the earth.'' The meek are those who have kept his commandments and have never
been permitted to inherit much of this earth during the six thousand years that it has been
ruled by man since the fall of Adam. They will inherit it during the one thousand years of its
regeneration which is referred to quite generally as the Millennium. That will be the time
when Jesus will sit on his throne and his apostles will sit on thrones and judge the twelve
tribes of Israel. 21 That will be the day when the earth will be renewed as spoken of by
Isaiah: a new heaven and a new earth in the day of this restoration. 22 This, however, is not
the final glory of our earth. After the Millennium it will die and then be raised in the
resurrection, purified and celestialized. This will be the time that it will become the eternal
abode of the righteous.

And the redemption of the soul is through him that quickeneth all things, in whose bosom it
is decreed that the poor and the meek of the earth shall inherit it.

Therefore, it must needs be sanctified from all unrighteousness, that it may be prepared for
the celestial glory;

For after it hath filled the measure of its creation, it shall be crowned with glory, even with
the presence of the Father;

That bodies who are of the celestial kingdom may possess it forever and ever; for, for this
intent was it made and created, and for this intent are they sanctified.

And they who are not sanctified through the law which I have given unto you, even the law
of Christ, must inherit another kingdom, even that of a terrestrial kingdom, or that of a
telestial kingdom.

For he who is not able to abide the law of a celestial kingdom cannot abide a celestial glory.

And he who cannot abide the law of a terrestrial kingdom cannot abide a terrestrial glory.

And he who cannot abide the law of a telestial kingdom cannot abide a telestial glory;
therefore he is not meet for a kingdom of glory. Therefore he must abide a kingdom which is
not a kingdom of glory.

And again, verily I say unto you, the earth abideth the law of a celestial kingdom, for it filleth
the measure of its creation, and transgresseth not the law—

Wherefore, it shall be sanctified; yea, notwithstanding it shall die, it shall be quickened again,
and shall abide the power by which it is quickened, and the righteous shall inherit it.

For notwithstanding they die, they also shall rise again, a spiritual body.

They who are of a celestial spirit shall receive the same body which was a natural body; even
ye shall receive your bodies, and your glory shall be that glory by which your bodies are
quickened.

Ye who are quickened by a portion of the celestial glory shall then receive of the same, even
a fulness.

And they who are quickened by a portion of the terrestrial glory shall then receive of the
same, even a fulness.

And also they who are quickened by a portion of the telestial glory shall then receive of the
same, even a fulness.

And they who remain shall also be quickened; nevertheless, they shall return again to their
own place, to enjoy that which they are willing to receive, because they were not willing to
enjoy that which the might have received.

For what doth it profit a man if a gift is bestowed upon him, and he receiveth not the gift?
Behold, he rejoices not in that which is given unto him, neither rejoices in him who is the
giver of the gift. 23

When this celestialized earth comes, then only those of the celestial kingdom will inherit it.
Those who have lived a terrestrial law will be assigned to a terrestrial kingdom on some other
globe. Those who have lived a telestial law will have to go to a telestial sphere suited to their
condition. Where these worlds are the Lord has not revealed to us, however they are spheres
now being prepared for them. Justice demands that every man shall receive a reward
according to his works. Those who do not attain to eternal life, which is to become sons and
daughters of God and joint heirs with Jesus Christ, will receive the gift of immortality.
Immortality means that they will live forever. The bodies of all the children of men, both the
righteous and the unrighteous, even sons of perdition, will come forth in the resurrection,
their spirits and bodies being united inseparably, and they shall live forever. Eternal life has a
deeper meaning than immortality, and all those who receive it become like God. 24 They will
inherit the fulness of the Father's kingdom, 25 all things will be given to them and they
become sons and daughters of God. 26 In the celestial kingdom those who receive the
exaltation will remain husbands and wives. The family organization will not be broken and
will endure forever and they will have eternal increase.

In the terrestrial and telestial kingdoms, there will be no marriage, hence no continuation of
the lives, 27 for they remain in these kingdoms separately and singly through all eternity.
This the Lord calls ``the deaths,'' 28 because there is no increase. The question frequently
arises: ``If men and women live singly in the terrestrial and the telestial kingdoms, then what
will prevent them from living promiscuously?'' The Lord has given us the answer to that
question. They will be quickened by different kind of bodies. They shall receive back their
natural body, but they will be terrestrial bodies and telestial bodies and their bodies will be
suited to the conditions prevailing in those kingdoms. Elder Orson Pratt has given an
excellent explanation as follows:

In every species of animals and plants there are many resemblances in the general outlines,
and many specific differences characterizing the individuals of each species. So in the
resurrection: There will be several classes of resurrected bodies; some celestial, some
terrestrial, some telestial, and some sons of perdition. Each of these classes will differ from
the others by prominent and marked distinctions; yet in each, considered by itself, there will
be found many resemblances as well as distinctions. There will be some physical peculiarity
by which each individual in every class can be identified. 29

Those who inherit the terrestrial kingdom and the telestial kingdom do not go into the
presence of God our Eternal Father. Those who inherit the telestial kingdom will be
ministered to by those of the terrestrial, and those of the terrestrial by those of the celestial.
The inhabitants of the terrestrial will have visitations from Jesus Christ, but not the Father.
This kingdom will be inhabited by those who have lived good moral lives, who have been
honest, honorable and just, but who would not receive the Gospel with its covenants. The
telestial kingdom will be the place for the wicked, those ``who are liars, and sorcerers, and
adulterers, and whoremongers, and whosoever loves and makes a lie. These are they who
suffer the wrath of God on the earth. These are they who suffer the vengeance of eternal fire.
These are they who are cast down to hell and suffer the wrath of Almighty God, until the
fulness of times, when Christ shall have subdued all enemies under his feet, and shall have
perfected his work; When he shall deliver up the kingdom, and present it unto the Father,
spotless, saying: I have overcome and have trodden the wine-press alone, even the wine-
press of the fierceness of the wrath of Almighty God.'' 30

There is still another group, comparatively few, who, after receiving the fulness of the
Gospel and the testimony of Jesus, then deny him and put him to open shame by turning
against his work and denying his power. These are called sons of perdition and they go away
into outer darkness. 31

So the Lord in his great mercy does for all men just the best that he can. Even the wicked,
after they pay the price, and they will have to pay a dreadful price, will be placed in a
kingdom where they can be made as happy as circumstances will permit. Through their
intense suffering while they wait for the resurrection at the end of the earth's temporal
existence, they will have learned to be obedient to law, for this will be a requirement in each
of the kingdoms, but where God and Christ are they cannot come worlds without end. 32

\newpage
REFERENCES—CHAPTER TWENTY-EIGHT

Footnotes

1. 2 Nephi 2:24-27.

2. \textit{Ibid.}, 2:21; 9:27.

3. D. \& C. 14:7.

4. Romans 8:13-17; 1 John 3:1-3; Moses 6:66-68.

5. Rev. 21:7; D. \& C. 76:53-59; 84:38.

6. John 5:23-24.

7. \textit{Ibid.}, 6:47-51.

8. \textit{Ibid.}, 17:1-3.

9. D. \& C. 29:40-41.

10. 3 Nephi 27:14-19.

11. D. \& C. 50:23-24.

12. \textit{Ibid.}, 84:19-22.

13. \textit{Ibid.}, 93:26-28.

14. \textit{Ibid.}, 76:53-60.

15. Matt. 10:37-40.

16. \textit{Teachings of the Prophet Joseph Smith}, p. 51.

17. \textit{Ibid.}, p. 297.

18. \textit{Ibid.}, p. 309.

19. \textit{Ibid.}, p. 324-425.

20. \textit{Ibid.}, p. 331.

21. Matt. 19:28.

22. Isaiah 65:17-25; D. \& C. 101:23-32.

23. D. \& C. 88:17-20, 33.24. 1 John 3:1-3.

25. D. \& C. 76:53-60; 84:37-38; Rev. 21:7.

26. Romans 8:19-20.

27. D. \& C. 132:19-20.

28. \textit{Ibid.}, 132:25.

29. Pratt, Orson, The Seer, p. 274.

30. D. \& C. 76:103-107.

31. \textit{Ibid.}, 76:31-33.

32. \textit{Ibid.}, 76:112.

