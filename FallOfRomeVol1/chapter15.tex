\chapter{Progress Of The Christian Religion.}
\section{Part \thesection.}

\textit{The Progress Of The Christian Religion, And The Sentiments,
Manners, Numbers, And Condition Of The Primitive Christians.}\textsuperscript{101}
\vspace{\onelineskip}

\pagenote[101]{In spite of my resolution, Lardner led me to look
through the famous fifteenth and sixteenth chapters of Gibbon. I
could not lay them down without finishing them. The causes
assigned, in the fifteenth chapter, for the diffusion of
Christianity, must, no doubt, have contributed to it materially;
but I doubt whether he saw them all. Perhaps those which he
enumerates are among the most obvious. They might all be safely
adopted by a Christian writer, with some change in the language
and manner. Mackintosh see Life, i. p. 244.—M.}

A candid but rational inquiry into the progress and establishment
of Christianity may be considered as a very essential part of the
history of the Roman empire. While that great body was invaded by
open violence, or undermined by slow decay, a pure and humble
religion gently insinuated itself into the minds of men, grew up
in silence and obscurity, derived new vigor from opposition, and
finally erected the triumphant banner of the Cross on the ruins
of the Capitol. Nor was the influence of Christianity confined to
the period or to the limits of the Roman empire. After a
revolution of thirteen or fourteen centuries, that religion is
still professed by the nations of Europe, the most distinguished
portion of human kind in arts and learning as well as in arms. By
the industry and zeal of the Europeans, it has been widely
diffused to the most distant shores of Asia and Africa; and by
the means of their colonies has been firmly established from
Canada to Chili, in a world unknown to the ancients.

But this inquiry, however useful or entertaining, is attended
with two peculiar difficulties. The scanty and suspicious
materials of ecclesiastical history seldom enable us to dispel
the dark cloud that hangs over the first age of the church. The
great law of impartiality too often obliges us to reveal the
imperfections of the uninspired teachers and believers of the
gospel; and, to a careless observer, \textit{their} faults may seem to
cast a shade on the faith which they professed. But the scandal
of the pious Christian, and the fallacious triumph of the
Infidel, should cease as soon as they recollect not only \textit{by
whom}, but likewise \textit{to whom}, the Divine Revelation was given.
The theologian may indulge the pleasing task of describing
Religion as she descended from Heaven, arrayed in her native
purity. A more melancholy duty is imposed on the historian. He
must discover the inevitable mixture of error and corruption,
which she contracted in a long residence upon earth, among a weak
and degenerate race of beings.\textsuperscript{102}

\pagenote[102]{The art of Gibbon, or at least the unfair
impression produced by these two memorable chapters, consists in
confounding together, in one undistinguishable mass, the origin
and apostolic propagation of the Christian religion with its
later progress. The main question, the divine origin of the
religion, is dexterously eluded or speciously conceded; his plan
enables him to commence his account, in most parts, below the
apostolic times; and it is only by the strength of the dark
coloring with which he has brought out the failings and the
follies of succeeding ages, that a shadow of doubt and suspicion
is thrown back on the primitive period of Christianity. Divest
this whole passage of the latent sarcasm betrayed by the
subsequent one of the whole disquisition, and it might commence a
Christian history, written in the most Christian spirit of
candor.—M.}

Our curiosity is naturally prompted to inquire by what means the
Christian faith obtained so remarkable a victory over the
established religions of the earth. To this inquiry, an obvious
but satisfactory answer may be returned; that it was owing to the
convincing evidence of the doctrine itself, and to the ruling
providence of its great Author. But as truth and reason seldom
find so favorable a reception in the world, and as the wisdom of
Providence frequently condescends to use the passions of the
human heart, and the general circumstances of mankind, as
instruments to execute its purpose, we may still be permitted,
though with becoming submission, to ask, not indeed what were the
first, but what were the secondary causes of the rapid growth of
the Christian church. It will, perhaps, appear, that it was most
effectually favored and assisted by the five following causes:

I. The inflexible, and if we may use the expression, the
intolerant zeal of the Christians, derived, it is true, from the
Jewish religion, but purified from the narrow and unsocial
spirit, which, instead of inviting, had deterred the Gentiles
from embracing the law of Moses.\textsuperscript{1023}

II. The doctrine of a future life, improved by every additional
circumstance which could give weight and efficacy to that
important truth. III. The miraculous powers ascribed to the
primitive church. IV. The pure and austere morals of the
Christians.

V. The union and discipline of the Christian republic, which
gradually formed an independent and increasing state in the heart
of the Roman empire.

\pagenote[1023]{Though we are thus far agreed with respect to the
inflexibility and intolerance of Christian zeal, yet as to the
principle from which it was derived, we are, toto cœlo, divided
in opinion. You deduce it from the Jewish religion; I would refer
it to a more adequate and a more obvious source, a full
persuasion of the truth of Christianity. Watson. Letters Gibbon,
i. 9.—M.}

I. We have already described the religious harmony of the ancient
world, and the facility with which the most different and even
hostile nations embraced, or at least respected, each other’s
superstitions. A single people refused to join in the common
intercourse of mankind. The Jews, who, under the Assyrian and
Persian monarchies, had languished for many ages the most
despised portion of their slaves,\textsuperscript{1} emerged from obscurity under
the successors of Alexander; and as they multiplied to a
surprising degree in the East, and afterwards in the West, they
soon excited the curiosity and wonder of other nations.\textsuperscript{2} The
sullen obstinacy with which they maintained their peculiar rites
and unsocial manners seemed to mark them out as a distinct
species of men, who boldly professed, or who faintly disguised,
their implacable habits to the rest of human kind.\textsuperscript{3} Neither the
violence of Antiochus, nor the arts of Herod, nor the example of
the circumjacent nations, could ever persuade the Jews to
associate with the institutions of Moses the elegant mythology of
the Greeks.\textsuperscript{4} According to the maxims of universal toleration,
the Romans protected a superstition which they despised.\textsuperscript{5} The
polite Augustus condescended to give orders, that sacrifices
should be offered for his prosperity in the temple of Jerusalem; \textsuperscript{6}
whilst the meanest of the posterity of Abraham, who should have
paid the same homage to the Jupiter of the Capitol, would have
been an object of abhorrence to himself and to his brethren.

But the moderation of the conquerors was insufficient to appease
the jealous prejudices of their subjects, who were alarmed and
scandalized at the ensigns of paganism, which necessarily
introduced themselves into a Roman province.\textsuperscript{7} The mad attempt of
Caligula to place his own statue in the temple of Jerusalem was
defeated by the unanimous resolution of a people who dreaded
death much less than such an idolatrous profanation.\textsuperscript{8} Their
attachment to the law of Moses was equal to their detestation of
foreign religions. The current of zeal and devotion, as it was
contracted into a narrow channel, ran with the strength, and
sometimes with the fury, of a torrent. This facility has not
always prevented intolerance, which seems inherent in the
religious spirit, when armed with authority. The separation of
the ecclesiastical and civil power, appears to be the only means
of at once maintaining religion and tolerance: but this is a very
modern notion. The passions, which mingle themselves with
opinions, made the Pagans very often intolerant and persecutors;
witness the Persians, the Egyptians even the Greeks and Romans.

1st. The Persians.—Cambyses, conqueror of the Egyptians,
condemned to death the magistrates of Memphis, because they had
offered divine honors to their god. Apis: he caused the god to be
brought before him, struck him with his dagger, commanded the
priests to be scourged, and ordered a general massacre of all the
Egyptians who should be found celebrating the festival of the
statues of the gods to be burnt. Not content with this
intolerance, he sent an army to reduce the Ammonians to slavery,
and to set on fire the temple in which Jupiter delivered his
oracles. See Herod. iii. 25—29, 37. Xerxes, during his invasion
of Greece, acted on the same principles: l c destroyed all the
temples of Greece and Ionia, except that of Ephesus. See Paus. l.
vii. p. 533, and x. p. 887.

Strabo, l. xiv. b. 941. 2d. The Egyptians.—They thought
themselves defiled when they had drunk from the same cup or eaten
at the same table with a man of a different belief from their
own. “He who has voluntarily killed any sacred animal is punished
with death; but if any one, even involuntarily, has killed a cat
or an ibis, he cannot escape the extreme penalty: the people drag
him away, treat him in the most cruel manner, sometimes without
waiting for a judicial sentence. * * * Even at the time when King
Ptolemy was not yet the acknowledged friend of the Roman people,
while the multitude were paying court with all possible attention
to the strangers who came from Italy * * a Roman having killed a
cat, the people rushed to his house, and neither the entreaties
of the nobles, whom the king sent to them, nor the terror of the
Roman name, were sufficiently powerful to rescue the man from
punishment, though he had committed the crime involuntarily.”
Diod. Sic. i 83. Juvenal, in his 13th Satire, describes the
sanguinary conflict between the inhabitants of Ombos and of
Tentyra, from religious animosity. The fury was carried so far,
that the conquerors tore and devoured the quivering limbs of the
conquered.

Ardet adhuc Ombos et Tentyra, summus utrinque Inde furor vulgo,
quod numina vicinorum Odit uterque locus; quum solos credat
habendos Esse Deos quos ipse colit. Sat. xv. v. 85.

3d. The Greeks.—“Let us not here,” says the Abbé Guénée, “refer
to the cities of Peloponnesus and their severity against atheism;
the Ephesians prosecuting Heraclitus for impiety; the Greeks
armed one against the other by religious zeal, in the
Amphictyonic war. Let us say nothing either of the frightful
cruelties inflicted by three successors of Alexander upon the
Jews, to force them to abandon their religion, nor of Antiochus
expelling the philosophers from his states. Let us not seek our
proofs of intolerance so far off. Athens, the polite and learned
Athens, will supply us with sufficient examples. Every citizen
made a public and solemn vow to conform to the religion of his
country, to defend it, and to cause it to be respected. An
express law severely punished all discourses against the gods,
and a rigid decree ordered the denunciation of all who should
deny their existence. * * * The practice was in unison with the
severity of the law. The proceedings commenced against
Protagoras; a price set upon the head of Diagoras; the danger of
Alcibiades; Aristotle obliged to fly; Stilpo banished; Anaxagoras
hardly escaping death; Pericles himself, after all his services
to his country, and all the glory he had acquired, compelled to
appear before the tribunals and make his defence; * * a priestess
executed for having introduced strange gods; Socrates condemned
and drinking the hemlock, because he was accused of not
recognizing those of his country, \&c.; these facts attest too
loudly, to be called in question, the religious intolerance of
the most humane and enlightened people in Greece.” Lettres de
quelques Juifs a Mons. Voltaire, i. p. 221. (Compare Bentley on
Freethinking, from which much of this is derived.)—M.

4th. The Romans.—The laws of Rome were not less express and
severe. The intolerance of foreign religions reaches, with the
Romans, as high as the laws of the twelve tables; the
prohibitions were afterwards renewed at different times.
Intolerance did not discontinue under the emperors; witness the
counsel of Mæcenas to Augustus. This counsel is so remarkable,
that I think it right to insert it entire. “Honor the gods
yourself,” says Mæcenas to Augustus, “in every way according to
the usage of your ancestors, and compel others to worship them.
Hate and punish those who introduce strange gods, not only for
the sake of the gods, (he who despises them will respect no one,)
but because those who introduce new gods engage a multitude of
persons in foreign laws and customs. From hence arise unions
bound by oaths and confederacies, and associations, things
dangerous to a monarchy.” Dion Cass. l. ii. c. 36. (But, though
some may differ from it, see Gibbon’s just observation on this
passage in Dion Cassius, ch. xvi. note 117; impugned, indeed, by
M. Guizot, note in loc.)—M.

Even the laws which the philosophers of Athens and of Rome wrote
for their imaginary republics are intolerant. Plato does not
leave to his citizens freedom of religious worship; and Cicero
expressly prohibits them from having other gods than those of the
state. Lettres de quelques Juifs a Mons. Voltaire, i. p. 226.—G.

According to M. Guizot’s just remarks, religious intolerance will
always ally itself with the passions of man, however different
those passions may be. In the instances quoted above, with the
Persians it was the pride of despotism; to conquer the gods of a
country was the last mark of subjugation. With the Egyptians, it
was the gross Fetichism of the superstitious populace, and the
local jealousy of neighboring towns. In Greece, persecution was
in general connected with political party; in Rome, with the
stern supremacy of the law and the interests of the state. Gibbon
has been mistaken in attributing to the tolerant spirit of
Paganism that which arose out of the peculiar circumstances of
the times. 1st. The decay of the old Polytheism, through the
progress of reason and intelligence, and the prevalence of
philosophical opinions among the higher orders.

2d. The Roman character, in which the political always
predominated over the religious party. The Romans were contented
with having bowed the world to a uniformity of subjection to
their power, and cared not for establishing the (to them) less
important uniformity of religion.—M.

\pagenote[1]{Dum Assyrios penes, Medosque, et Persas Oriens fuit,
despectissima pars servientium. Tacit. Hist. v. 8. Herodotus, who
visited Asia whilst it obeyed the last of those empires, slightly
mentions the Syrians of Palestine, who, according to their own
confession, had received from Egypt the rite of circumcision. See
l. ii. c. 104.}

\pagenote[2]{Diodorus Siculus, l. xl. Dion Cassius, l. xxxvii. p.
121. Tacit Hist. v. 1—9. Justin xxxvi. 2, 3.}

\pagenote[3]{Tradidit arcano quæcunque volumine Moses, Non
monstrare vias cadem nisi sacra colenti, Quæsitum ad fontem solos
deducere verpas. The letter of this law is not to be found in the
present volume of Moses. But the wise, the humane Maimonides
openly teaches that if an idolater fall into the water, a Jew
ought not to save him from instant death. See Basnage, Histoire
des Juifs, l. vi. c. 28. * Note: It is diametrically opposed to
its spirit and to its letter, see, among other passages, Deut. v.
18. 19, (God) “loveth the stranger in giving him food and
raiment. Love ye, therefore, the stranger: for ye were strangers
in the land of Egypt.” Comp. Lev. xxiii. 25. Juvenal is a
satirist, whose strong expressions can hardly be received as
historic evidence; and he wrote after the horrible cruelties of
the Romans, which, during and after the war, might give some
cause for the complete isolation of the Jew from the rest of the
world. The Jew was a bigot, but his religion was not the only
source of his bigotry. After how many centuries of mutual wrong
and hatred, which had still further estranged the Jew from
mankind, did Maimonides write?—M.}

\pagenote[4]{A Jewish sect, which indulged themselves in a sort
of occasional conformity, derived from Herod, by whose example
and authority they had been seduced, the name of Herodians. But
their numbers were so inconsiderable, and their duration so
short, that Josephus has not thought them worthy of his notice.
See Prideaux’s Connection, vol. ii. p. 285. * Note: The Herodians
were probably more of a political party than a religious sect,
though Gibbon is most likely right as to their occasional
conformity. See Hist. of the Jews, ii. 108.—M.}

\pagenote[5]{Cicero pro Flacco, c. 28. * Note: The edicts of
Julius Cæsar, and of some of the cities in Asia Minor (Krebs.
Decret. pro Judæis,) in favor of the nation in general, or of the
Asiatic Jews, speak a different language.—M.}

\pagenote[6]{Philo de Legatione. Augustus left a foundation for a
perpetual sacrifice. Yet he approved of the neglect which his
grandson Caius expressed towards the temple of Jerusalem. See
Sueton. in August. c. 93, and Casaubon’s notes on that passage.}

\pagenote[7]{See, in particular, Joseph. Antiquitat. xvii. 6,
xviii. 3; and de Bell. Judiac. i. 33, and ii. 9, edit. Havercamp.
* Note: This was during the government of Pontius Pilate. (Hist.
of Jews, ii. 156.) Probably in part to avoid this collision, the
Roman governor, in general, resided at Cæsarea.—M.}

\pagenote[8]{Jussi a Caio Cæsare, effigiem ejus in templo locare,
arma potius sumpsere. Tacit. Hist. v. 9. Philo and Josephus gave
a very circumstantial, but a very rhetorical, account of this
transaction, which exceedingly perplexed the governor of Syria.
At the first mention of this idolatrous proposal, King Agrippa
fainted away; and did not recover his senses until the third day.
(Hist. of Jews, ii. 181, \&c.)}

This inflexible perseverance, which appeared so odious or so
ridiculous to the ancient world, assumes a more awful character,
since Providence has deigned to reveal to us the mysterious
history of the chosen people. But the devout and even scrupulous
attachment to the Mosaic religion, so conspicuous among the Jews
who lived under the second temple, becomes still more surprising,
if it is compared with the stubborn incredulity of their
forefathers. When the law was given in thunder from Mount Sinai,
when the tides of the ocean and the course of the planets were
suspended for the convenience of the Israelites, and when
temporal rewards and punishments were the immediate consequences
of their piety or disobedience, they perpetually relapsed into
rebellion against the visible majesty of their Divine King,
placed the idols of the nations in the sanctuary of Jehovah, and
imitated every fantastic ceremony that was practised in the tents
of the Arabs, or in the cities of Phœnicia.\textsuperscript{9} As the protection
of Heaven was deservedly withdrawn from the ungrateful race,
their faith acquired a proportionable degree of vigor and purity.

The contemporaries of Moses and Joshua had beheld with careless
indifference the most amazing miracles. Under the pressure of
every calamity, the belief of those miracles has preserved the
Jews of a later period from the universal contagion of idolatry;
and in contradiction to every known principle of the human mind,
that singular people seems to have yielded a stronger and more
ready assent to the traditions of their remote ancestors, than to
the evidence of their own senses.\textsuperscript{10}

\pagenote[9]{For the enumeration of the Syrian and Arabian
deities, it may be observed, that Milton has comprised in one
hundred and thirty very beautiful lines the two large and learned
syntagmas which Selden had composed on that abstruse subject.}

\pagenote[10]{“How long will this people provoke me? and how long
will it be ere they believe me, for all the signs which I have
shown among them?” (Numbers xiv. 11.) It would be easy, but it
would be unbecoming, to justify the complaint of the Deity from
the whole tenor of the Mosaic history. Note: Among a rude and
barbarous people, religious impressions are easily made, and are
as soon effaced. The ignorance which multiplies imaginary
wonders, would weaken and destroy the effect of real miracle. At
the period of the Jewish history, referred to in the passage from
Numbers, their fears predominated over their faith,—the fears of
an unwarlike people, just rescued from debasing slavery, and
commanded to attack a fierce, a well-armed, a gigantic, and a far
more numerous race, the inhabitants of Canaan. As to the frequent
apostasy of the Jews, their religion was beyond their state of
civilization. Nor is it uncommon for a people to cling with
passionate attachment to that of which, at first, they could not
appreciate the value. Patriotism and national pride will contend,
even to death, for political rights which have been forced upon a
reluctant people. The Christian may at least retort, with
justice, that the great sign of his religion, the resurrection of
Jesus, was most ardently believed, and most resolutely asserted,
by the eye witnesses of the fact.—M.}

The Jewish religion was admirably fitted for defence, but it was
never designed for conquest; and it seems probable that the
number of proselytes was never much superior to that of
apostates. The divine promises were originally made, and the
distinguishing rite of circumcision was enjoined, to a single
family. When the posterity of Abraham had multiplied like the
sands of the sea, the Deity, from whose mouth they received a
system of laws and ceremonies, declared himself the proper and as
it were the national God of Israel; and with the most jealous
care separated his favorite people from the rest of mankind. The
conquest of the land of Canaan was accompanied with so many
wonderful and with so many bloody circumstances, that the
victorious Jews were left in a state of irreconcilable hostility
with all their neighbors. They had been commanded to extirpate
some of the most idolatrous tribes, and the execution of the
divine will had seldom been retarded by the weakness of humanity.

With the other nations they were forbidden to contract any
marriages or alliances; and the prohibition of receiving them
into the congregation, which in some cases was perpetual, almost
always extended to the third, to the seventh, or even to the
tenth generation. The obligation of preaching to the Gentiles the
faith of Moses had never been inculcated as a precept of the law,
nor were the Jews inclined to impose it on themselves as a
voluntary duty.

In the admission of new citizens that unsocial people was
actuated by the selfish vanity of the Greeks, rather than by the
generous policy of Rome. The descendants of Abraham were
flattered by the opinion that they alone were the heirs of the
covenant, and they were apprehensive of diminishing the value of
their inheritance by sharing it too easily with the strangers of
the earth. A larger acquaintance with mankind extended their
knowledge without correcting their prejudices; and whenever the
God of Israel acquired any new votaries, he was much more
indebted to the inconstant humor of polytheism than to the active
zeal of his own missionaries.\textsuperscript{11} The religion of Moses seems to
be instituted for a particular country as well as for a single
nation; and if a strict obedience had been paid to the order,
that every male, three times in the year, should present himself
before the Lord Jehovah, it would have been impossible that the
Jews could ever have spread themselves beyond the narrow limits
of the promised land.\textsuperscript{12} That obstacle was indeed removed by the
destruction of the temple of Jerusalem; but the most considerable
part of the Jewish religion was involved in its destruction; and
the Pagans, who had long wondered at the strange report of an
empty sanctuary,\textsuperscript{13} were at a loss to discover what could be the
object, or what could be the instruments, of a worship which was
destitute of temples and of altars, of priests and of sacrifices.

Yet even in their fallen state, the Jews, still asserting their
lofty and exclusive privileges, shunned, instead of courting, the
society of strangers. They still insisted with inflexible rigor
on those parts of the law which it was in their power to
practise. Their peculiar distinctions of days, of meats, and a
variety of trivial though burdensome observances, were so many
objects of disgust and aversion for the other nations, to whose
habits and prejudices they were diametrically opposite. The
painful and even dangerous rite of circumcision was alone capable
of repelling a willing proselyte from the door of the synagogue. \textsuperscript{14}

\pagenote[11]{All that relates to the Jewish proselytes has been
very ably by Basnage, Hist. des Juifs, l. vi. c. 6, 7.}

\pagenote[12]{See Exod. xxiv. 23, Deut. xvi. 16, the
commentators, and a very sensible note in the Universal History,
vol. i. p. 603, edit. fol.}

\pagenote[13]{When Pompey, using or abusing the right of
conquest, entered into the Holy of Holies, it was observed with
amazement, “Nulli intus Deum effigie, vacuam sedem et inania
arcana.” Tacit. Hist. v. 9. It was a popular saying, with regard
to the Jews, “Nil præter nubes et coeli numen adorant.”}

\pagenote[14]{A second kind of circumcision was inflicted on a
Samaritan or Egyptian proselyte. The sullen indifference of the
Talmudists, with respect to the conversion of strangers, may be
seen in Basnage Histoire des Juifs, l. xi. c. 6.}

Under these circumstances, Christianity offered itself to the
world, armed with the strength of the Mosaic law, and delivered
from the weight of its fetters. An exclusive zeal for the truth
of religion, and the unity of God, was as carefully inculcated in
the new as in the ancient system; and whatever was now revealed
to mankind concerning the nature and designs of the Supreme Being
was fitted to increase their reverence for that mysterious
doctrine. The divine authority of Moses and the prophets was
admitted, and even established, as the firmest basis of
Christianity. From the beginning of the world, an uninterrupted
series of predictions had announced and prepared the
long-expected coming of the Messiah, who, in compliance with the
gross apprehensions of the Jews, had been more frequently
represented under the character of a King and Conqueror, than
under that of a Prophet, a Martyr, and the Son of God. By his
expiatory sacrifice, the imperfect sacrifices of the temple were
at once consummated and abolished. The ceremonial law, which
consisted only of types and figures, was succeeded by a pure and
spiritual worship equally adapted to all climates, as well as to
every condition of mankind; and to the initiation of blood was
substituted a more harmless initiation of water. The promise of
divine favor, instead of being partially confined to the
posterity of Abraham, was universally proposed to the freeman and
the slave, to the Greek and to the barbarian, to the Jew and to
the Gentile. Every privilege that could raise the proselyte from
earth to heaven, that could exalt his devotion, secure his
happiness, or even gratify that secret pride which, under the
semblance of devotion, insinuates itself into the human heart,
was still reserved for the members of the Christian church; but
at the same time all mankind was permitted, and even solicited,
to accept the glorious distinction, which was not only proffered
as a favor, but imposed as an obligation. It became the most
sacred duty of a new convert to diffuse among his friends and
relations the inestimable blessing which he had received, and to
warn them against a refusal that would be severely punished as a
criminal disobedience to the will of a benevolent but
all-powerful Deity.

\section{Part \thesection.}

The enfranchisement of the church from the bonds of the synagogue
was a work, however, of some time and of some difficulty. The
Jewish converts, who acknowledged Jesus in the character of the
Messiah foretold by their ancient oracles, respected him as a
prophetic teacher of virtue and religion; but they obstinately
adhered to the ceremonies of their ancestors, and were desirous
of imposing them on the Gentiles, who continually augmented the
number of believers. These Judaizing Christians seem to have
argued with some degree of plausibility from the divine origin of
the Mosaic law, and from the immutable perfections of its great
Author. They affirmed, that if the Being, who is the same through
all eternity, had designed to abolish those sacred rites which
had served to distinguish his chosen people, the repeal of them
would have been no less clear and solemn than their first
promulgation: \textit{that}, instead of those frequent declarations,
which either suppose or assert the perpetuity of the Mosaic
religion, it would have been represented as a provisionary scheme
intended to last only to the coming of the Messiah, who should
instruct mankind in a more perfect mode of faith and of worship:\textsuperscript{15}
\textit{that} the Messiah himself, and his disciples who conversed
with him on earth, instead of authorizing by their example the
most minute observances of the Mosaic law,\textsuperscript{16} would have
published to the world the abolition of those useless and
obsolete ceremonies, without suffering Christianity to remain
during so many years obscurely confounded among the sects of the
Jewish church. Arguments like these appear to have been used in
the defence of the expiring cause of the Mosaic law; but the
industry of our learned divines has abundantly explained the
ambiguous language of the Old Testament, and the ambiguous
conduct of the apostolic teachers. It was proper gradually to
unfold the system of the gospel, and to pronounce, with the
utmost caution and tenderness, a sentence of condemnation so
repugnant to the inclination and prejudices of the believing
Jews.

\pagenote[15]{These arguments were urged with great ingenuity by
the Jew Orobio, and refuted with equal ingenuity and candor by
the Christian Limborch. See the Amica Collatio, (it well deserves
that name,) or account of the dispute between them.}

\pagenote[16]{Jesus... circumcisus erat; cibis utebatur Judaicis;
vestitu simili; purgatos scabie mittebat ad sacerdotes; Paschata
et alios dies festos religiose observabat: Si quos sanavit
sabbatho, ostendit non tantum ex lege, sed et exceptis
sententiis, talia opera sabbatho non interdicta. Grotius de
Veritate Religionis Christianæ, l. v. c. 7. A little afterwards,
(c. 12,) he expatiates on the condescension of the apostles.}

The history of the church of Jerusalem affords a lively proof of
the necessity of those precautions, and of the deep impression
which the Jewish religion had made on the minds of its sectaries.
The first fifteen bishops of Jerusalem were all circumcised Jews;
and the congregation over which they presided united the law of
Moses with the doctrine of Christ.\textsuperscript{17} It was natural that the
primitive tradition of a church which was founded only forty days
after the death of Christ, and was governed almost as many years
under the immediate inspection of his apostle, should be received
as the standard of orthodoxy. The distant churches very
frequently appealed to the authority of their venerable Parent,
and relieved her distresses by a liberal contribution of alms.
But when numerous and opulent societies were established in the
great cities of the empire, in Antioch, Alexandria, Ephesus,
Corinth, and Rome, the reverence which Jerusalem had inspired to
all the Christian colonies insensibly diminished.\textsuperscript{18b} The Jewish
converts, or, as they were afterwards called, the Nazarenes, who
had laid the foundations of the church, soon found themselves
overwhelmed by the increasing multitudes, that from all the
various religions of polytheism enlisted under the banner of
Christ: and the Gentiles, who, with the approbation of their
peculiar apostle, had rejected the intolerable weight of the
Mosaic ceremonies, at length refused to their more scrupulous
brethren the same toleration which at first they had humbly
solicited for their own practice. The ruin of the temple of the
city, and of the public religion of the Jews, was severely felt
by the Nazarenes; as in their manners, though not in their faith,
they maintained so intimate a connection with their impious
countrymen, whose misfortunes were attributed by the Pagans to
the contempt, and more justly ascribed by the Christians to the
wrath, of the Supreme Deity. The Nazarenes retired from the ruins
of Jerusalem\textsuperscript{18} to the little town of Pella beyond the Jordan,
where that ancient church languished above sixty years in
solitude and obscurity.\textsuperscript{19} They still enjoyed the comfort of
making frequent and devout visits to the \textit{Holy City}, and the
hope of being one day restored to those seats which both nature
and religion taught them to love as well as to revere. But at
length, under the reign of Hadrian, the desperate fanaticism of
the Jews filled up the measure of their calamities; and the
Romans, exasperated by their repeated rebellions, exercised the
rights of victory with unusual rigor. The emperor founded, under
the name of Ælia Capitolina, a new city on Mount Sion,\textsuperscript{20} to
which he gave the privileges of a colony; and denouncing the
severest penalties against any of the Jewish people who should
dare to approach its precincts, he fixed a vigilant garrison of a
Roman cohort to enforce the execution of his orders. The
Nazarenes had only one way left to escape the common
proscription, and the force of truth was on this occasion
assisted by the influence of temporal advantages. They elected
Marcus for their bishop, a prelate of the race of the Gentiles,
and most probably a native either of Italy or of some of the
Latin provinces. At his persuasion, the most considerable part of
the congregation renounced the Mosaic law, in the practice of
which they had persevered above a century. By this sacrifice of
their habits and prejudices, they purchased a free admission into
the colony of Hadrian, and more firmly cemented their union with
the Catholic church.\textsuperscript{21}

\pagenote[17]{Pæne omnes Christum Deum sub legis observatione
credebant Sulpicius Severus, ii. 31. See Eusebius, Hist.
Ecclesiast. l. iv. c. 5.}

\pagenote[18b]{Footnote 18b: Mosheim de Rebus Christianis ante
Constantinum Magnum, page 153. In this masterly performance,
which I shall often have occasion to quote he enters much more
fully into the state of the primitive church than he has an
opportunity of doing in his General History.}

\pagenote[18]{This is incorrect: all the traditions concur in
placing the abandonment of the city by the Christians, not only
before it was in ruins, but before the seige had commenced.
Euseb. loc. cit., and Le Clerc.—M.}

\pagenote[19]{Eusebius, l. iii. c. 5. Le Clerc, Hist. Ecclesiast.
p. 605. During this occasional absence, the bishop and church of
Pella still retained the title of Jerusalem. In the same manner,
the Roman pontiffs resided seventy years at Avignon; and the
patriarchs of Alexandria have long since transferred their
episcopal seat to Cairo.}

\pagenote[20]{Dion Cassius, l. lxix. The exile of the Jewish
nation from Jerusalem is attested by Aristo of Pella, (apud
Euseb. l. iv. c. 6,) and is mentioned by several ecclesiastical
writers; though some of them too hastily extend this interdiction
to the whole country of Palestine.}

\pagenote[21]{Eusebius, l. iv. c. 6. Sulpicius Severus, ii. 31.
By comparing their unsatisfactory accounts, Mosheim (p. 327, \&c.)
has drawn out a very distinct representation of the circumstances
and motives of this revolution.}

When the name and honors of the church of Jerusalem had been
restored to Mount Sion, the crimes of heresy and schism were
imputed to the obscure remnant of the Nazarenes, which refused to
accompany their Latin bishop. They still preserved their former
habitation of Pella, spread themselves into the villages adjacent
to Damascus, and formed an inconsiderable church in the city of
Berœa, or, as it is now called, of Aleppo, in Syria.\textsuperscript{22} The name
of Nazarenes was deemed too honorable for those Christian Jews,
and they soon received, from the supposed poverty of their
understanding, as well as of their condition, the contemptuous
epithet of Ebionites.\textsuperscript{23} In a few years after the return of the
church of Jerusalem, it became a matter of doubt and controversy,
whether a man who sincerely acknowledged Jesus as the Messiah,
but who still continued to observe the law of Moses, could
possibly hope for salvation. The humane temper of Justin Martyr
inclined him to answer this question in the affirmative; and
though he expressed himself with the most guarded diffidence, he
ventured to determine in favor of such an imperfect Christian, if
he were content to practise the Mosaic ceremonies, without
pretending to assert their general use or necessity. But when
Justin was pressed to declare the sentiment of the church, he
confessed that there were very many among the orthodox
Christians, who not only excluded their Judaizing brethren from
the hope of salvation, but who declined any intercourse with them
in the common offices of friendship, hospitality, and social
life.\textsuperscript{24} The more rigorous opinion prevailed, as it was natural
to expect, over the milder; and an eternal bar of separation was
fixed between the disciples of Moses and those of Christ. The
unfortunate Ebionites, rejected from one religion as apostates,
and from the other as heretics, found themselves compelled to
assume a more decided character; and although some traces of that
obsolete sect may be discovered as late as the fourth century,
they insensibly melted away, either into the church or the
synagogue.\textsuperscript{25}

\pagenote[22]{Le Clerc (Hist. Ecclesiast. p. 477, 535) seems to
have collected from Eusebius, Jerome, Epiphanius, and other
writers, all the principal circumstances that relate to the
Nazarenes or Ebionites. The nature of their opinions soon divided
them into a stricter and a milder sect; and there is some reason
to conjecture, that the family of Jesus Christ remained members,
at least, of the latter and more moderate party.}

\pagenote[23]{Some writers have been pleased to create an Ebion,
the imaginary author of their sect and name. But we can more
safely rely on the learned Eusebius than on the vehement
Tertullian, or the credulous Epiphanius. According to Le Clerc,
the Hebrew word Ebjonim may be translated into Latin by that of
Pauperes. See Hist. Ecclesiast. p. 477. * Note: The opinion of Le
Clerc is generally admitted; but Neander has suggested some good
reasons for supposing that this term only applied to poverty of
condition. The obscure history of their tenets and divisions, is
clearly and rationally traced in his History of the Church, vol.
i. part ii. p. 612, \&c., Germ. edit.—M.}

\pagenote[24]{See the very curious Dialogue of Justin Martyr with
the Jew Tryphon. The conference between them was held at Ephesus,
in the reign of Antoninus Pius, and about twenty years after the
return of the church of Pella to Jerusalem. For this date consult
the accurate note of Tillemont, Memoires Ecclesiastiques, tom.
ii. p. 511. * Note: Justin Martyr makes an important distinction,
which Gibbon has neglected to notice. * * * There were some who
were not content with observing the Mosaic law themselves, but
enforced the same observance, as necessary to salvation, upon the
heathen converts, and refused all social intercourse with them if
they did not conform to the law. Justin Martyr himself freely
admits those who kept the law themselves to Christian communion,
though he acknowledges that some, not the Church, thought
otherwise; of the other party, he himself thought less favorably.
The former by some are considered the Nazarenes the atter the
Ebionites—G and M.}

\pagenote[25]{Of all the systems of Christianity, that of
Abyssinia is the only one which still adheres to the Mosaic
rites. (Geddes’s Church History of Æthiopia, and Dissertations de
La Grand sur la Relation du P. Lobo.) The eunuch of the queen
Candace might suggest some suspicious; but as we are assured
(Socrates, i. 19. Sozomen, ii. 24. Ludolphus, p. 281) that the
Æthiopians were not converted till the fourth century, it is more
reasonable to believe that they respected the sabbath, and
distinguished the forbidden meats, in imitation of the Jews, who,
in a very early period, were seated on both sides of the Red Sea.
Circumcision had been practised by the most ancient Æthiopians,
from motives of health and cleanliness, which seem to be
explained in the Recherches Philosophiques sur les Americains,
tom. ii. p. 117.}

While the orthodox church preserved a just medium between
excessive veneration and improper contempt for the law of Moses,
the various heretics deviated into equal but opposite extremes of
error and extravagance. From the acknowledged truth of the Jewish
religion, the Ebionites had concluded that it could never be
abolished. From its supposed imperfections, the Gnostics as
hastily inferred that it never was instituted by the wisdom of
the Deity. There are some objections against the authority of
Moses and the prophets, which too readily present themselves to
the sceptical mind; though they can only be derived from our
ignorance of remote antiquity, and from our incapacity to form an
adequate judgment of the divine economy. These objections were
eagerly embraced and as petulantly urged by the vain science of
the Gnostics.\textsuperscript{26} As those heretics were, for the most part,
averse to the pleasures of sense, they morosely arraigned the
polygamy of the patriarchs, the gallantries of David, and the
seraglio of Solomon. The conquest of the land of Canaan, and the
extirpation of the unsuspecting natives, they were at a loss how
to reconcile with the common notions of humanity and justice.\textsuperscript{261}
But when they recollected the sanguinary list of murders, of
executions, and of massacres, which stain almost every page of
the Jewish annals, they acknowledged that the barbarians of
Palestine had exercised as much compassion towards their
idolatrous enemies, as they had ever shown to their friends or
countrymen.\textsuperscript{27} Passing from the sectaries of the law to the law
itself, they asserted that it was impossible that a religion
which consisted only of bloody sacrifices and trifling
ceremonies, and whose rewards as well as punishments were all of
a carnal and temporal nature, could inspire the love of virtue,
or restrain the impetuosity of passion. The Mosaic account of the
creation and fall of man was treated with profane derision by the
Gnostics, who would not listen with patience to the repose of the
Deity after six days’ labor, to the rib of Adam, the garden of
Eden, the trees of life and of knowledge, the speaking serpent,
the forbidden fruit, and the condemnation pronounced against
human kind for the venial offence of their first progenitors.\textsuperscript{28}
The God of Israel was impiously represented by the Gnostics as a
being liable to passion and to error, capricious in his favor,
implacable in his resentment, meanly jealous of his superstitious
worship, and confining his partial providence to a single people,
and to this transitory life. In such a character they could
discover none of the features of the wise and omnipotent Father
of the universe.\textsuperscript{29} They allowed that the religion of the Jews
was somewhat less criminal than the idolatry of the Gentiles; but
it was their fundamental doctrine that the Christ whom they
adored as the first and brightest emanation of the Deity appeared
upon earth to rescue mankind from their various errors, and to
reveal a new system of truth and perfection. The most learned of
the fathers, by a very singular condescension, have imprudently
admitted the sophistry of the Gnostics.\textsuperscript{291} Acknowledging that
the literal sense is repugnant to every principle of faith as
well as reason, they deem themselves secure and invulnerable
behind the ample veil of allegory, which they carefully spread
over every tender part of the Mosaic dispensation.\textsuperscript{30}

\pagenote[26]{Beausobre, Histoire du Manicheisme, l. i. c. 3, has
stated their objections, particularly those of Faustus, the
adversary of Augustin, with the most learned impartiality.}

\pagenote[261]{On the “war law” of the Jews, see Hist. of Jews,
i. 137.—M.}

\pagenote[27]{Apud ipsos fides obstinata, misericordia in
promptu: adversus amnes alios hostile odium. Tacit. Hist. v. 4.
Surely Tacitus had seen the Jews with too favorable an eye. The
perusal of Josephus must have destroyed the antithesis. * Note:
Few writers have suspected Tacitus of partiality towards the
Jews. The whole later history of the Jews illustrates as well
their strong feelings of humanity to their brethren, as their
hostility to the rest of mankind. The character and the position
of Josephus with the Roman authorities, must be kept in mind
during the perusal of his History. Perhaps he has not exaggerated
the ferocity and fanaticism of the Jews at that time; but
insurrectionary warfare is not the best school for the humaner
virtues, and much must be allowed for the grinding tyranny of the
later Roman governors. See Hist. of Jews, ii. 254.—M.}

\pagenote[28]{Dr. Burnet (Archæologia, l. ii. c. 7) has discussed
the first chapters of Genesis with too much wit and freedom. *
Note: Dr. Burnet apologized for the levity with which he had
conducted some of his arguments, by the excuse that he wrote in a
learned language for scholars alone, not for the vulgar. Whatever
may be thought of his success in tracing an Eastern allegory in
the first chapters of Genesis, his other works prove him to have
been a man of great genius, and of sincere piety.—M}

\pagenote[29]{The milder Gnostics considered Jehovah, the
Creator, as a Being of a mixed nature between God and the Dæmon.
Others confounded him with an evil principle. Consult the second
century of the general history of Mosheim, which gives a very
distinct, though concise, account of their strange opinions on
this subject.}

\pagenote[291]{The Gnostics, and the historian who has stated
these plausible objections with so much force as almost to make
them his own, would have shown a more considerate and not less
reasonable philosophy, if they had considered the religion of
Moses with reference to the age in which it was promulgated; if
they had done justice to its sublime as well as its more
imperfect views of the divine nature; the humane and civilizing
provisions of the Hebrew law, as well as those adapted for an
infant and barbarous people. See Hist of Jews, i. 36, 37, \&c.—M.}

\pagenote[30]{See Beausobre, Hist. du Manicheisme, l. i. c. 4.
Origen and St. Augustin were among the allegorists.}

It has been remarked with more ingenuity than truth, that the
virgin purity of the church was never violated by schism or
heresy before the reign of Trajan or Hadrian, about one hundred
years after the death of Christ.\textsuperscript{31} We may observe with much more
propriety, that, during that period, the disciples of the Messiah
were indulged in a freer latitude, both of faith and practice,
than has ever been allowed in succeeding ages. As the terms of
communion were insensibly narrowed, and the spiritual authority
of the prevailing party was exercised with increasing severity,
many of its most respectable adherents, who were called upon to
renounce, were provoked to assert their private opinions, to
pursue the consequences of their mistaken principles, and openly
to erect the standard of rebellion against the unity of the
church. The Gnostics were distinguished as the most polite, the
most learned, and the most wealthy of the Christian name; and
that general appellation, which expressed a superiority of
knowledge, was either assumed by their own pride, or ironically
bestowed by the envy of their adversaries. They were almost
without exception of the race of the Gentiles, and their
principal founders seem to have been natives of Syria or Egypt,
where the warmth of the climate disposes both the mind and the
body to indolent and contemplative devotion. The Gnostics blended
with the faith of Christ many sublime but obscure tenets, which
they derived from oriental philosophy, and even from the religion
of Zoroaster, concerning the eternity of matter, the existence of
two principles, and the mysterious hierarchy of the invisible
world.\textsuperscript{32} As soon as they launched out into that vast abyss, they
delivered themselves to the guidance of a disordered imagination;
and as the paths of error are various and infinite, the Gnostics
were imperceptibly divided into more than fifty particular sects,\textsuperscript{33}
of whom the most celebrated appear to have been the
Basilidians, the Valentinians, the Marcionites, and, in a still
later period, the Manichæans. Each of these sects could boast of
its bishops and congregations, of its doctors and martyrs;\textsuperscript{34}
and, instead of the Four Gospels adopted by the church,\textsuperscript{341} the
heretics produced a multitude of histories, in which the actions
and discourses of Christ and of his apostles were adapted to
their respective tenets.\textsuperscript{35} The success of the Gnostics was rapid
and extensive.\textsuperscript{36} They covered Asia and Egypt, established
themselves in Rome, and sometimes penetrated into the provinces
of the West. For the most part they arose in the second century,
flourished during the third, and were suppressed in the fourth or
fifth, by the prevalence of more fashionable controversies, and
by the superior ascendant of the reigning power. Though they
constantly disturbed the peace, and frequently disgraced the
name, of religion, they contributed to assist rather than to
retard the progress of Christianity. The Gentile converts, whose
strongest objections and prejudices were directed against the law
of Moses, could find admission into many Christian societies,
which required not from their untutored mind any belief of an
antecedent revelation. Their faith was insensibly fortified and
enlarged, and the church was ultimately benefited by the
conquests of its most inveterate enemies.\textsuperscript{37}

\pagenote[31]{Hegesippus, ap. Euseb. l. iii. 32, iv. 22. Clemens
Alexandrin Stromat. vii. 17. * Note: The assertion of Hegesippus
is not so positive: it is sufficient to read the whole passage in
Eusebius, to see that the former part is modified by the matter.
Hegesippus adds, that up to this period the church had remained
pure and immaculate as a virgin. Those who labored to corrupt the
doctrines of the gospel worked as yet in obscurity—G}

\pagenote[32]{In the account of the Gnostics of the second and
third centuries, Mosheim is ingenious and candid; Le Clerc dull,
but exact; Beausobre almost always an apologist; and it is much
to be feared that the primitive fathers are very frequently
calumniators. * Note The Histoire du Gnosticisme of M. Matter is
at once the fairest and most complete account of these sects.—M.}

\pagenote[33]{See the catalogues of Irenæus and Epiphanius. It
must indeed be allowed, that those writers were inclined to
multiply the number of sects which opposed the unity of the
church.}

\pagenote[34]{Eusebius, l. iv. c. 15. Sozomen, l. ii. c. 32. See
in Bayle, in the article of Marcion, a curious detail of a
dispute on that subject. It should seem that some of the Gnostics
(the Basilidians) declined, and even refused the honor of
Martyrdom. Their reasons were singular and abstruse. See Mosheim,
p. 539.}

\pagenote[341]{M. Hahn has restored the Marcionite Gospel with
great ingenuity. His work is reprinted in Thilo. Codex. Apoc.
Nov. Test. vol. i.—M.}

\pagenote[35]{See a very remarkable passage of Origen, (Proem. ad
Lucam.) That indefatigable writer, who had consumed his life in
the study of the Scriptures, relies for their authenticity on the
inspired authority of the church. It was impossible that the
Gnostics could receive our present Gospels, many parts of which
(particularly in the resurrection of Christ) are directly, and as
it might seem designedly, pointed against their favorite tenets.
It is therefore somewhat singular that Ignatius (Epist. ad Smyrn.
Patr. Apostol. tom. ii. p. 34) should choose to employ a vague
and doubtful tradition, instead of quoting the certain testimony
of the evangelists. Note: Bishop Pearson has attempted very
happily to explain this singularity.’ The first Christians were
acquainted with a number of sayings of Jesus Christ, which are
not related in our Gospels, and indeed have never been written.
Why might not St. Ignatius, who had lived with the apostles or
their disciples, repeat in other words that which St. Luke has
related, particularly at a time when, being in prison, he could
have the Gospels at hand? Pearson, Vind Ign. pp. 2, 9 p. 396 in
tom. ii. Patres Apost. ed. Coteler—G.}

\pagenote[36]{Faciunt favos et vespæ; faciunt ecclesias et
Marcionitæ, is the strong expression of Tertullian, which I am
obliged to quote from memory. In the time of Epiphanius (advers.
Hæreses, p. 302) the Marcionites were very numerous in Italy,
Syria, Egypt, Arabia, and Persia.}

\pagenote[37]{Augustin is a memorable instance of this gradual
progress from reason to faith. He was, during several years,
engaged in the Manichæar sect.}

But whatever difference of opinion might subsist between the
Orthodox, the Ebionites, and the Gnostics, concerning the
divinity or the obligation of the Mosaic law, they were all
equally animated by the same exclusive zeal, and by the same
abhorrence for idolatry, which had distinguished the Jews from
the other nations of the ancient world. The philosopher, who
considered the system of polytheism as a composition of human
fraud and error, could disguise a smile of contempt under the
mask of devotion, without apprehending that either the mockery,
or the compliance, would expose him to the resentment of any
invisible, or, as he conceived them, imaginary powers. But the
established religions of Paganism were seen by the primitive
Christians in a much more odious and formidable light. It was the
universal sentiment both of the church and of heretics, that the
dæmons were the authors, the patrons, and the objects of
idolatry.\textsuperscript{38} Those rebellious spirits who had been degraded from
the rank of angels, and cast down into the infernal pit, were
still permitted to roam upon earth, to torment the bodies, and to
seduce the minds, of sinful men. The dæmons soon discovered and
abused the natural propensity of the human heart towards
devotion, and artfully withdrawing the adoration of mankind from
their Creator, they usurped the place and honors of the Supreme
Deity. By the success of their malicious contrivances, they at
once gratified their own vanity and revenge, and obtained the
only comfort of which they were yet susceptible, the hope of
involving the human species in the participation of their guilt
and misery. It was confessed, or at least it was imagined, that
they had distributed among themselves the most important
characters of polytheism, one dæmon assuming the name and
attributes of Jupiter, another of Æsculapius, a third of Venus,
and a fourth perhaps of Apollo;\textsuperscript{39} and that, by the advantage of
their long experience and ærial nature, they were enabled to
execute, with sufficient skill and dignity, the parts which they
had undertaken. They lurked in the temples, instituted festivals
and sacrifices, invented fables, pronounced oracles, and were
frequently allowed to perform miracles. The Christians, who, by
the interposition of evil spirits, could so readily explain every
præternatural appearance, were disposed and even desirous to
admit the most extravagant fictions of the Pagan mythology. But
the belief of the Christian was accompanied with horror. The most
trifling mark of respect to the national worship he considered as
a direct homage yielded to the dæmon, and as an act of rebellion
against the majesty of God.

\pagenote[38]{The unanimous sentiment of the primitive church is
very clearly explained by Justin Martyr, Apolog. Major, by
Athenagoras, Legat. c. 22. \&c., and by Lactantius, Institut.
Divin. ii. 14—19.}

\pagenote[39]{Tertullian (Apolog. c. 23) alleges the confession
of the dæmons themselves as often as they were tormented by the
Christian exorcists}

\section{Part \thesection.}

In consequence of this opinion, it was the first but arduous duty
of a Christian to preserve himself pure and undefiled by the
practice of idolatry. The religion of the nations was not merely
a speculative doctrine professed in the schools or preached in
the temples. The innumerable deities and rites of polytheism were
closely interwoven with every circumstance of business or
pleasure, of public or of private life, and it seemed impossible
to escape the observance of them, without, at the same time,
renouncing the commerce of mankind, and all the offices and
amusements of society.\textsuperscript{40} The important transactions of peace and
war were prepared or concluded by solemn sacrifices, in which the
magistrate, the senator, and the soldier, were obliged to preside
or to participate.\textsuperscript{41} The public spectacles were an essential
part of the cheerful devotion of the Pagans, and the gods were
supposed to accept, as the most grateful offering, the games that
the prince and people celebrated in honor of their peculiar
festivals.\textsuperscript{42} The Christians, who with pious horror avoided the
abomination of the circus or the theatre, found himself
encompassed with infernal snares in every convivial
entertainment, as often as his friends, invoking the hospitable
deities, poured out libations to each other’s happiness.\textsuperscript{43} When
the bride, struggling with well-affected reluctance, was forced
in hymenæal pomp over the threshold of her new habitation,\textsuperscript{44} or
when the sad procession of the dead slowly moved towards the
funeral pile,\textsuperscript{45} the Christian, on these interesting occasions,
was compelled to desert the persons who were the dearest to him,
rather than contract the guilt inherent to those impious
ceremonies. Every art and every trade that was in the least
concerned in the framing or adorning of idols was polluted by the
stain of idolatry;\textsuperscript{46} a severe sentence, since it devoted to
eternal misery the far greater part of the community, which is
employed in the exercise of liberal or mechanic professions. If
we cast our eyes over the numerous remains of antiquity, we shall
perceive, that besides the immediate representations of the gods,
and the holy instruments of their worship, the elegant forms and
agreeable fictions consecrated by the imagination of the Greeks,
were introduced as the richest ornaments of the houses, the
dress, and the furniture of the Pagans.\textsuperscript{47} Even the arts of music
and painting, of eloquence and poetry, flowed from the same
impure origin. In the style of the fathers, Apollo and the Muses
were the organs of the infernal spirit; Homer and Virgil were the
most eminent of his servants; and the beautiful mythology which
pervades and animates the compositions of their genius, is
destined to celebrate the glory of the dæmons. Even the common
language of Greece and Rome abounded with familiar but impious
expressions, which the imprudent Christian might too carelessly
utter, or too patiently hear.\textsuperscript{48}

\pagenote[40]{Tertullian has written a most severe treatise
against idolatry, to caution his brethren against the hourly
danger of incurring that guilt. Recogita sylvam, et quantæ
latitant spinæ. De Corona Militis, c. 10.}

\pagenote[41]{The Roman senate was always held in a temple or
consecrated place. (Aulus Gellius, xiv. 7.) Before they entered
on business, every senator dropped some wine and frankincense on
the altar. Sueton. in August. c. 35.}

\pagenote[42]{See Tertullian, De Spectaculis. This severe
reformer shows no more indulgence to a tragedy of Euripides, than
to a combat of gladiators. The dress of the actors particularly
offends him. By the use of the lofty buskin, they impiously
strive to add a cubit to their stature. c. 23.}

\pagenote[43]{The ancient practice of concluding the
entertainment with libations, may be found in every classic.
Socrates and Seneca, in their last moments, made a noble
application of this custom. Postquam stagnum, calidæ aquæ
introiit, respergens proximos servorum, addita voce, libare se
liquorem illum Jovi Liberatori. Tacit. Annal. xv. 64.}

\pagenote[44]{See the elegant but idolatrous hymn of Catullus, on
the nuptials of Manlius and Julia. O Hymen, Hymenæe Io! Quis huic
Deo compararier ausit?}

\pagenote[45]{The ancient funerals (in those of Misenus and
Pallas) are no less accurately described by Virgil, than they are
illustrated by his commentator Servius. The pile itself was an
altar, the flames were fed with the blood of victims, and all the
assistants were sprinkled with lustral water.}

\pagenote[46]{Tertullian de Idololatria, c. 11. * Note: The
exaggerated and declamatory opinions of Tertullian ought not to
be taken as the general sentiment of the early Christians. Gibbon
has too often allowed himself to consider the peculiar notions of
certain Fathers of the Church as inherent in Christianity. This
is not accurate.—G.}

\pagenote[47]{See every part of Montfaucon’s Antiquities. Even
the reverses of the Greek and Roman coins were frequently of an
idolatrous nature. Here indeed the scruples of the Christian were
suspended by a stronger passion. Note: All this scrupulous nicety
is at variance with the decision of St. Paul about meat offered
to idols, 1, Cor. x. 21— 32.—M.}

\pagenote[48]{Tertullian de Idololatria, c. 20, 21, 22. If a
Pagan friend (on the occasion perhaps of sneezing) used the
familiar expression of “Jupiter bless you,” the Christian was
obliged to protest against the divinity of Jupiter.}

The dangerous temptations which on every side lurked in ambush to
surprise the unguarded believer, assailed him with redoubled
violence on the days of solemn festivals. So artfully were they
framed and disposed throughout the year, that superstition always
wore the appearance of pleasure, and often of virtue. Some of the
most sacred festivals in the Roman ritual were destined to salute
the new calends of January with vows of public and private
felicity; to indulge the pious remembrance of the dead and
living; to ascertain the inviolable bounds of property; to hail,
on the return of spring, the genial powers of fecundity; to
perpetuate the two memorable æras of Rome, the foundation of the
city and that of the republic; and to restore, during the humane
license of the Saturnalia, the primitive equality of mankind.
Some idea may be conceived of the abhorrence of the Christians
for such impious ceremonies, by the scrupulous delicacy which
they displayed on a much less alarming occasion. On days of
general festivity it was the custom of the ancients to adorn
their doors with lamps and with branches of laurel, and to crown
their heads with a garland of flowers. This innocent and elegant
practice might perhaps have been tolerated as a mere civil
institution. But it most unluckily happened that the doors were
under the protection of the household gods, that the laurel was
sacred to the lover of Daphne, and that garlands of flowers,
though frequently worn as a symbol either of joy or mourning, had
been dedicated in their first origin to the service of
superstition. The trembling Christians, who were persuaded in
this instance to comply with the fashion of their country, and
the commands of the magistrate, labored under the most gloomy
apprehensions, from the reproaches of his own conscience, the
censures of the church, and the denunciations of divine
vengeance.\textsuperscript{49} \textsuperscript{50}

\pagenote[49]{Consult the most labored work of Ovid, his
imperfect Fasti. He finished no more than the first six months of
the year. The compilation of Macrobius is called the Saturnalia,
but it is only a small part of the first book that bears any
relation to the title.}

\pagenote[50]{Tertullian has composed a defence, or rather
panegyric, of the rash action of a Christian soldier, who, by
throwing away his crown of laurel, had exposed himself and his
brethren to the most imminent danger. By the mention of the
emperors, (Severus and Caracalla,) it is evident, notwithstanding
the wishes of M. de Tillemont, that Tertullian composed his
treatise De Corona long before he was engaged in the errors of
the Montanists. See Memoires Ecclesiastiques, tom. iii. p. 384.
Note: The soldier did not tear off his crown to throw it down
with contempt; he did not even throw it away; he held it in his
hand, while others were it on their heads. Solus libero capite,
ornamento in manu otioso.—G Note: Tertullian does not expressly
name the two emperors, Severus and Caracalla: he speaks only of
two emperors, and of a long peace which the church had enjoyed.
It is generally agreed that Tertullian became a Montanist about
the year 200: his work, de Corona Militis, appears to have been
written, at the earliest about the year 202 before the
persecution of Severus: it may be maintained, then, that it is
subsequent to the Montanism of the author. See Mosheim, Diss. de
Apol. Tertull. p. 53. Biblioth. Amsterd. tom. x. part ii. p. 292.
Cave’s Hist. Lit. p. 92, 93.—G. ——The state of Tertullian’s
opinions at the particular period is almost an idle question.
“The fiery African” is not at any time to be considered a fair
representative of Christianity.—M.}

Such was the anxious diligence which was required to guard the
chastity of the gospel from the infectious breath of idolatry.
The superstitious observances of public or private rites were
carelessly practised, from education and habit, by the followers
of the established religion. But as often as they occurred, they
afforded the Christians an opportunity of declaring and
confirming their zealous opposition. By these frequent
protestations their attachment to the faith was continually
fortified; and in proportion to the increase of zeal, they
combated with the more ardor and success in the holy war, which
they had undertaken against the empire of the demons.

II. The writings of Cicero\textsuperscript{51} represent in the most lively colors
the ignorance, the errors, and the uncertainty of the ancient
philosophers with regard to the immortality of the soul. When
they are desirous of arming their disciples against the fear of
death, they inculcate, as an obvious though melancholy position,
that the fatal stroke of our dissolution releases us from the
calamities of life; and that those can no longer suffer, who no
longer exist. Yet there were a few sages of Greece and Rome who
had conceived a more exalted, and, in some respects, a juster
idea of human nature, though it must be confessed, that in the
sublime inquiry, their reason had been often guided by their
imagination, and that their imagination had been prompted by
their vanity. When they viewed with complacency the extent of
their own mental powers, when they exercised the various
faculties of memory, of fancy, and of judgment, in the most
profound speculations, or the most important labors, and when
they reflected on the desire of fame, which transported them into
future ages, far beyond the bounds of death and of the grave,
they were unwilling to confound themselves with the beasts of the
field, or to suppose that a being, for whose dignity they
entertained the most sincere admiration, could be limited to a
spot of earth, and to a few years of duration. With this
favorable prepossession they summoned to their aid the science,
or rather the language, of Metaphysics. They soon discovered,
that as none of the properties of matter will apply to the
operations of the mind, the human soul must consequently be a
substance distinct from the body, pure, simple, and spiritual,
incapable of dissolution, and susceptible of a much higher degree
of virtue and happiness after the release from its corporeal
prison. From these specious and noble principles, the
philosophers who trod in the footsteps of Plato deduced a very
unjustifiable conclusion, since they asserted, not only the
future immortality, but the past eternity, of the human soul,
which they were too apt to consider as a portion of the infinite
and self-existing spirit, which pervades and sustains the
universe.\textsuperscript{52} A doctrine thus removed beyond the senses and the
experience of mankind might serve to amuse the leisure of a
philosophic mind; or, in the silence of solitude, it might
sometimes impart a ray of comfort to desponding virtue; but the
faint impression which had been received in the schools was soon
obliterated by the commerce and business of active life. We are
sufficiently acquainted with the eminent persons who flourished
in the age of Cicero and of the first Cæsars, with their actions,
their characters, and their motives, to be assured that their
conduct in this life was never regulated by any serious
conviction of the rewards or punishments of a future state. At
the bar and in the senate of Rome the ablest orators were not
apprehensive of giving offence to their hearers by exposing that
doctrine as an idle and extravagant opinion, which was rejected
with contempt by every man of a liberal education and
understanding.\textsuperscript{53}

\pagenote[51]{In particular, the first book of the Tusculan
Questions, and the treatise De Senectute, and the Somnium
Scipionis, contain, in the most beautiful language, every thing
that Grecian philosophy, on Roman good sense, could possibly
suggest on this dark but important object.}

\pagenote[52]{The preexistence of human souls, so far at least as
that doctrine is compatible with religion, was adopted by many of
the Greek and Latin fathers. See Beausobre, Hist. du Manicheisme,
l. vi. c. 4.}

\pagenote[53]{See Cicero pro Cluent. c. 61. Cæsar ap. Sallust. de
Bell. Catilis n 50. Juvenal. Satir. ii. 149. ——Esse aliquid
manes, et subterranea regna, —————Nec pueri credunt, nisi qui
nondum æree lavantæ.}

Since therefore the most sublime efforts of philosophy can extend
no further than feebly to point out the desire, the hope, or, at
most, the probability, of a future state, there is nothing,
except a divine revelation, that can ascertain the existence and
describe the condition, of the invisible country which is
destined to receive the souls of men after their separation from
the body. But we may perceive several defects inherent to the
popular religions of Greece and Rome, which rendered them very
unequal to so arduous a task. 1. The general system of their
mythology was unsupported by any solid proofs; and the wisest
among the Pagans had already disclaimed its usurped authority. 2.
The description of the infernal regions had been abandoned to the
fancy of painters and of poets, who peopled them with so many
phantoms and monsters, who dispensed their rewards and
punishments with so little equity, that a solemn truth, the most
congenial to the human heart, was oppressed and disgraced by the
absurd mixture of the wildest fictions.\textsuperscript{54} 3. The doctrine of a
future state was scarcely considered among the devout polytheists
of Greece and Rome as a fundamental article of faith. The
providence of the gods, as it related to public communities
rather than to private individuals, was principally displayed on
the visible theatre of the present world. The petitions which
were offered on the altars of Jupiter or Apollo expressed the
anxiety of their worshippers for temporal happiness, and their
ignorance or indifference concerning a future life.\textsuperscript{55} The
important truth of the immortality of the soul was inculcated
with more diligence, as well as success, in India, in Assyria, in
Egypt, and in Gaul; and since we cannot attribute such a
difference to the superior knowledge of the barbarians, we must
ascribe it to the influence of an established priesthood, which
employed the motives of virtue as the instrument of ambition.\textsuperscript{56}

\pagenote[54]{The xith book of the Odyssey gives a very dreary
and incoherent account of the infernal shades. Pindar and Virgil
have embellished the picture; but even those poets, though more
correct than their great model, are guilty of very strange
inconsistencies. See Bayle, Responses aux Questions d’un
Provincial, part iii. c. 22.}

\pagenote[55]{See xvith epistle of the first book of Horace, the
xiiith Satire of Juvenal, and the iid Satire of Persius: these
popular discourses express the sentiment and language of the
multitude.}

\pagenote[56]{If we confine ourselves to the Gauls, we may
observe, that they intrusted, not only their lives, but even
their money, to the security of another world. Vetus ille mos
Gallorum occurrit (says Valerius Maximus, l. ii. c. 6, p. 10)
quos, memoria proditum est pecunias montuas, quæ his apud inferos
redderentur, dare solitos. The same custom is more darkly
insinuated by Mela, l. iii. c. 2. It is almost needless to add,
that the profits of trade hold a just proportion to the credit of
the merchant, and that the Druids derived from their holy
profession a character of responsibility, which could scarcely be
claimed by any other order of men.}

We might naturally expect that a principle so essential to
religion, would have been revealed in the clearest terms to the
chosen people of Palestine, and that it might safely have been
intrusted to the hereditary priesthood of Aaron. It is incumbent
on us to adore the mysterious dispensations of Providence,\textsuperscript{57}
when we discover that the doctrine of the immortality of the soul
is omitted in the law of Moses; it is darkly insinuated by the
prophets; and during the long period which elapsed between the
Egyptian and the Babylonian servitudes, the hopes as well as
fears of the Jews appear to have been confined within the narrow
compass of the present life.\textsuperscript{58} After Cyrus had permitted the
exiled nation to return into the promised land, and after Ezra
had restored the ancient records of their religion, two
celebrated sects, the Sadducees and the Pharisees, insensibly
arose at Jerusalem.\textsuperscript{59} The former, selected from the more opulent
and distinguished ranks of society, were strictly attached to the
literal sense of the Mosaic law, and they piously rejected the
immortality of the soul, as an opinion that received no
countenance from the divine book, which they revered as the only
rule of their faith. To the authority of Scripture the Pharisees
added that of tradition, and they accepted, under the name of
traditions, several speculative tenets from the philosophy or
religion of the eastern nations. The doctrines of fate or
predestination, of angels and spirits, and of a future state of
rewards and punishments, were in the number of these new articles
of belief; and as the Pharisees, by the austerity of their
manners, had drawn into their party the body of the Jewish
people, the immortality of the soul became the prevailing
sentiment of the synagogue, under the reign of the Asmonæan
princes and pontiffs. The temper of the Jews was incapable of
contenting itself with such a cold and languid assent as might
satisfy the mind of a Polytheist; and as soon as they admitted
the idea of a future state, they embraced it with the zeal which
has always formed the characteristic of the nation. Their zeal,
however, added nothing to its evidence, or even probability: and
it was still necessary that the doctrine of life and immortality,
which had been dictated by nature, approved by reason, and
received by superstition, should obtain the sanction of divine
truth from the authority and example of Christ.

\pagenote[57]{The right reverend author of the Divine Legation of
Moses as signs a very curious reason for the omission, and most
ingeniously retorts it on the unbelievers. * Note: The hypothesis
of Warburton concerning this remarkable fact, which, as far as
the Law of Moses, is unquestionable, made few disciples; and it
is difficult to suppose that it could be intended by the author
himself for more than a display of intellectual strength. Modern
writers have accounted in various ways for the silence of the
Hebrew legislator on the immortality of the soul. According to
Michaelis, “Moses wrote as an historian and as a lawgiver; he
regulated the ecclesiastical discipline, rather than the
religious belief of his people; and the sanctions of the law
being temporal, he had no occasion, and as a civil legislator
could not with propriety, threaten punishments in another world.”
See Michaelis, Laws of Moses, art. 272, vol. iv. p. 209, Eng.
Trans.; and Syntagma Commentationum, p. 80, quoted by Guizot. M.
Guizot adds, the “ingenious conjecture of a philosophic
theologian,” which approximates to an opinion long entertained by
the Editor. That writer believes, that in the state of
civilization at the time of the legislator, this doctrine, become
popular among the Jews, would necessarily have given birth to a
multitude of idolatrous superstitions which he wished to prevent.
His primary object was to establish a firm theocracy, to make his
people the conservators of the doctrine of the Divine Unity, the
basis upon which Christianity was hereafter to rest. He carefully
excluded everything which could obscure or weaken that doctrine.
Other nations had strangely abused their notions on the
immortality of the soul; Moses wished to prevent this abuse:
hence he forbade the Jews from consulting necromancers, (those
who evoke the spirits of the dead.) Deut. xviii. 11. Those who
reflect on the state of the Pagans and the Jews, and on the
facility with which idolatry crept in on every side, will not be
astonished that Moses has not developed a doctrine of which the
influence might be more pernicious than useful to his people.
Orat. Fest. de Vitæ Immort. Spe., \&c., auct. Ph. Alb. Stapfer, p.
12 13, 20. Berne, 1787. ——Moses, as well from the intimations
scattered in his writings, the passage relating to the
translation of Enoch, (Gen. v. 24,) the prohibition of
necromancy, (Michaelis believes him to be the author of the Book
of Job though this opinion is in general rejected; other learned
writers consider this Book to be coeval with and known to Moses,)
as from his long residence in Egypt, and his acquaintance with
Egyptian wisdom, could not be ignorant of the doctrine of the
immortality of the soul. But this doctrine if popularly known
among the Jews, must have been purely Egyptian, and as so,
intimately connected with the whole religious system of that
country. It was no doubt moulded up with the tenet of the
transmigration of the soul, perhaps with notions analogous to the
emanation system of India in which the human soul was an efflux
from or indeed a part of, the Deity. The Mosaic religion drew a
wide and impassable interval between the Creator and created
human beings: in this it differed from the Egyptian and all the
Eastern religions. As then the immortality of the soul was thus
inseparably blended with those foreign religions which were
altogether to be effaced from the minds of the people, and by no
means necessary for the establishment of the theocracy, Moses
maintained silence on this point and a purer notion of it was
left to be developed at a more favorable period in the history of
man.—M.}

\pagenote[58]{See Le Clerc (Prolegomena ad Hist. Ecclesiast.
sect. 1, c. 8) His authority seems to carry the greater weight,
as he has written a learned and judicious commentary on the books
of the Old Testament.}

\pagenote[59]{Joseph. Antiquitat. l. xiii. c. 10. De Bell. Jud.
ii. 8. According to the most natural interpretation of his words,
the Sadducees admitted only the Pentateuch; but it has pleased
some modern critics to add the Prophets to their creed, and to
suppose that they contented themselves with rejecting the
traditions of the Pharisees. Dr. Jortin has argued that point in
his Remarks on Ecclesiastical History, vol. ii. p. 103.}

When the promise of eternal happiness was proposed to mankind on
condition of adopting the faith, and of observing the precepts,
of the gospel, it is no wonder that so advantageous an offer
should have been accepted by great numbers of every religion, of
every rank, and of every province in the Roman empire. The
ancient Christians were animated by a contempt for their present
existence, and by a just confidence of immortality, of which the
doubtful and imperfect faith of modern ages cannot give us any
adequate notion. In the primitive church, the influence of truth
was very powerfully strengthened by an opinion, which, however it
may deserve respect for its usefulness and antiquity, has not
been found agreeable to experience. It was universally believed,
that the end of the world, and the kingdom of heaven, were at
hand. \textsuperscript{591} The near approach of this wonderful event had been
predicted by the apostles; the tradition of it was preserved by
their earliest disciples, and those who understood in their
literal senses the discourse of Christ himself, were obliged to
expect the second and glorious coming of the Son of Man in the
clouds, before that generation was totally extinguished, which
had beheld his humble condition upon earth, and which might still
be witness of the calamities of the Jews under Vespasian or
Hadrian. The revolution of seventeen centuries has instructed us
not to press too closely the mysterious language of prophecy and
revelation; but as long as, for wise purposes, this error was
permitted to subsist in the church, it was productive of the most
salutary effects on the faith and practice of Christians, who
lived in the awful expectation of that moment, when the globe
itself, and all the various race of mankind, should tremble at
the appearance of their divine Judge.\textsuperscript{60}

\pagenote[591]{This was, in fact, an integral part of the Jewish
notion of the Messiah, from which the minds of the apostles
themselves were but gradually detached. See Bertholdt,
Christologia Judæorum, concluding chapters—M.}

\pagenote[60]{This expectation was countenanced by the
twenty-fourth chapter of St. Matthew, and by the first epistle of
St. Paul to the Thessalonians. Erasmus removes the difficulty by
the help of allegory and metaphor; and the learned Grotius
ventures to insinuate, that, for wise purposes, the pious
deception was permitted to take place. * Note: Some modern
theologians explain it without discovering either allegory or
deception. They say, that Jesus Christ, after having proclaimed
the ruin of Jerusalem and of the Temple, speaks of his second
coming and the sings which were to precede it; but those who
believed that the moment was near deceived themselves as to the
sense of two words, an error which still subsists in our versions
of the Gospel according to St. Matthew, xxiv. 29, 34. In verse
29, we read, “Immediately after the tribulation of those days
shall the sun be darkened,” \&c. The Greek word signifies all at
once, suddenly, not immediately; so that it signifies only the
sudden appearance of the signs which Jesus Christ announces not
the shortness of the interval which was to separate them from the
“days of tribulation,” of which he was speaking. The verse 34 is
this “Verily I say unto you, This generation shall not pass till
all these things shall be fulfilled.” Jesus, speaking to his
disciples, uses these words, which the translators have rendered
by this generation, but which means the race, the filiation of my
disciples; that is, he speaks of a class of men, not of a
generation. The true sense then, according to these learned men,
is, In truth I tell you that this race of men, of which you are
the commencement, shall not pass away till this shall take place;
that is to say, the succession of Christians shall not cease till
his coming. See Commentary of M. Paulus on the New Test., edit.
1802, tom. iii. p. 445,—446.—G. ——Others, as Rosenmuller and
Kuinoel, in loc., confine this passage to a highly figurative
description of the ruins of the Jewish city and polity.—M.}

\section{Part \thesection.}

The ancient and popular doctrine of the Millennium was intimately
connected with the second coming of Christ. As the works of the
creation had been finished in six days, their duration in their
present state, according to a tradition which was attributed to
the prophet Elijah, was fixed to six thousand years.\textsuperscript{61} By the
same analogy it was inferred, that this long period of labor and
contention, which was now almost elapsed,\textsuperscript{62} would be succeeded
by a joyful Sabbath of a thousand years; and that Christ, with
the triumphant band of the saints and the elect who had escaped
death, or who had been miraculously revived, would reign upon
earth till the time appointed for the last and general
resurrection. So pleasing was this hope to the mind of believers,
that the \textit{New Jerusalem}, the seat of this blissful kingdom, was
quickly adorned with all the gayest colors of the imagination. A
felicity consisting only of pure and spiritual pleasure would
have appeared too refined for its inhabitants, who were still
supposed to possess their human nature and senses. A garden of
Eden, with the amusements of the pastoral life, was no longer
suited to the advanced state of society which prevailed under the
Roman empire. A city was therefore erected of gold and precious
stones, and a supernatural plenty of corn and wine was bestowed
on the adjacent territory; in the free enjoyment of whose
spontaneous productions the happy and benevolent people was never
to be restrained by any jealous laws of exclusive property.\textsuperscript{63}
The assurance of such a Millennium was carefully inculcated by a
succession of fathers from Justin Martyr,\textsuperscript{64} and Irenæus, who
conversed with the immediate disciples of the apostles, down to
Lactantius, who was preceptor to the son of Constantine.\textsuperscript{65}
Though it might not be universally received, it appears to have
been the reigning sentiment of the orthodox believers; and it
seems so well adapted to the desires and apprehensions of
mankind, that it must have contributed in a very considerable
degree to the progress of the Christian faith. But when the
edifice of the church was almost completed, the temporary support
was laid aside. The doctrine of Christ’s reign upon earth was at
first treated as a profound allegory, was considered by degrees
as a doubtful and useless opinion, and was at length rejected as
the absurd invention of heresy and fanaticism.\textsuperscript{66} A mysterious
prophecy, which still forms a part of the sacred canon, but which
was thought to favor the exploded sentiment, has very narrowly
escaped the proscription of the church.\textsuperscript{67}

\pagenote[61]{See Burnet’s Sacred Theory, part iii. c. 5. This
tradition may be traced as high as the the author of Epistle of
Barnabas, who wrote in the first century, and who seems to have
been half a Jew. * Note: In fact it is purely Jewish. See
Mosheim, De Reb. Christ. ii. 8. Lightfoot’s Works, 8vo. edit.
vol. iii. p. 37. Bertholdt, Christologia Judæorum ch. 38.—M.}

\pagenote[62]{The primitive church of Antioch computed almost
6000 years from the creation of the world to the birth of Christ.
Africanus, Lactantius, and the Greek church, have reduced that
number to 5500, and Eusebius has contented himself with 5200
years. These calculations were formed on the Septuagint, which
was universally received during the six first centuries. The
authority of the vulgate and of the Hebrew text has determined
the moderns, Protestants as well as Catholics, to prefer a period
of about 4000 years; though, in the study of profane antiquity,
they often find themselves straitened by those narrow limits. *
Note: Most of the more learned modern English Protestants, Dr.
Hales, Mr. Faber, Dr. Russel, as well as the Continental writers,
adopt the larger chronology. There is little doubt that the
narrower system was framed by the Jews of Tiberias; it was
clearly neither that of St. Paul, nor of Josephus, nor of the
Samaritan Text. It is greatly to be regretted that the chronology
of the earlier Scriptures should ever have been made a religious
question—M.}

\pagenote[63]{Most of these pictures were borrowed from a
misrepresentation of Isaiah, Daniel, and the Apocalypse. One of
the grossest images may be found in Irenæus, (l. v. p. 455,) the
disciple of Papias, who had seen the apostle St. John.}

\pagenote[64]{See the second dialogue of Justin with Triphon, and
the seventh book of Lactantius. It is unnecessary to allege all
the intermediate fathers, as the fact is not disputed. Yet the
curious reader may consult Daille de Uus Patrum, l. ii. c. 4.}

\pagenote[65]{The testimony of Justin of his own faith and that
of his orthodox brethren, in the doctrine of a Millennium, is
delivered in the clearest and most solemn manner, (Dialog. cum
Tryphonte Jud. p. 177, 178, edit. Benedictin.) If in the
beginning of this important passage there is any thing like an
inconsistency, we may impute it, as we think proper, either to
the author or to his transcribers. * Note: The Millenium is
described in what once stood as the XLIst Article of the English
Church (see Collier, Eccles. Hist., for Articles of Edw. VI.) as
“a fable of Jewish dotage.” The whole of these gross and earthly
images may be traced in the works which treat on the Jewish
traditions, in Lightfoot, Schoetgen, and Eisenmenger; “Das
enthdeckte Judenthum” t. ii 809; and briefly in Bertholdt, i. c.
38, 39.—M.}

\pagenote[66]{Dupin, Bibliotheque Ecclesiastique, tom. i. p. 223,
tom. ii. p. 366, and Mosheim, p. 720; though the latter of these
learned divines is not altogether candid on this occasion.}

\pagenote[67]{In the council of Laodicea, (about the year 360,)
the Apocalypse was tacitly excluded from the sacred canon, by the
same churches of Asia to which it is addressed; and we may learn
from the complaint of Sulpicius Severus, that their sentence had
been ratified by the greater number of Christians of his time.
From what causes then is the Apocalypse at present so generally
received by the Greek, the Roman, and the Protestant churches?
The following ones may be assigned. 1. The Greeks were subdued by
the authority of an impostor, who, in the sixth century, assumed
the character of Dionysius the Areopagite. 2. A just apprehension
that the grammarians might become more important than the
theologians, engaged the council of Trent to fix the seal of
their infallibility on all the books of Scripture contained in
the Latin Vulgate, in the number of which the Apocalypse was
fortunately included. (Fr. Paolo, Istoria del Concilio
Tridentino, l. ii.) 3. The advantage of turning those mysterious
prophecies against the See of Rome, inspired the Protestants with
uncommon veneration for so useful an ally. See the ingenious and
elegant discourses of the present bishop of Litchfield on that
unpromising subject. * Note: The exclusion of the Apocalypse is
not improbably assigned to its obvious unfitness to be read in
churches. It is to be feared that a history of the interpretation
of the Apocalypse would not give a very favorable view either of
the wisdom or the charity of the successive ages of Christianity.
Wetstein’s interpretation, differently modified, is adopted by
most Continental scholars.—M.}

Whilst the happiness and glory of a temporal reign were promised
to the disciples of Christ, the most dreadful calamities were
denounced against an unbelieving world. The edification of a new
Jerusalem was to advance by equal steps with the destruction of
the mystic Babylon; and as long as the emperors who reigned
before Constantine persisted in the profession of idolatry, the
epithet of Babylon was applied to the city and to the empire of
Rome. A regular series was prepared of all the moral and physical
evils which can afflict a flourishing nation; intestine discord,
and the invasion of the fiercest barbarians from the unknown
regions of the North; pestilence and famine, comets and eclipses,
earthquakes and inundations.\textsuperscript{68} All these were only so many
preparatory and alarming signs of the great catastrophe of Rome,
when the country of the Scipios and Cæsars should be consumed by
a flame from Heaven, and the city of the seven hills, with her
palaces, her temples, and her triumphal arches, should be buried
in a vast lake of fire and brimstone. It might, however, afford
some consolation to Roman vanity, that the period of their empire
would be that of the world itself; which, as it had once perished
by the element of water, was destined to experience a second and
a speedy destruction from the element of fire. In the opinion of
a general conflagration, the faith of the Christian very happily
coincided with the tradition of the East, the philosophy of the
Stoics, and the analogy of Nature; and even the country, which,
from religious motives, had been chosen for the origin and
principal scene of the conflagration, was the best adapted for
that purpose by natural and physical causes; by its deep caverns,
beds of sulphur, and numerous volcanoes, of which those of Ætna,
of Vesuvius, and of Lipari, exhibit a very imperfect
representation. The calmest and most intrepid sceptic could not
refuse to acknowledge that the destruction of the present system
of the world by fire was in itself extremely probable. The
Christian, who founded his belief much less on the fallacious
arguments of reason than on the authority of tradition and the
interpretation of Scripture, expected it with terror and
confidence as a certain and approaching event; and as his mind
was perpetually filled with the solemn idea, he considered every
disaster that happened to the empire as an infallible symptom of
an expiring world.\textsuperscript{69}

\pagenote[68]{Lactantius (Institut. Divin. vii. 15, \&c.) relates
the dismal talk of futurity with great spirit and eloquence. *
Note: Lactantius had a notion of a great Asiatic empire, which
was previously to rise on the ruins of the Roman: quod Romanum
nomen animus dicere, sed dicam. quia futurum est tolletur de
terra, et impere. Asiam revertetur.—M.}

\pagenote[69]{On this subject every reader of taste will be
entertained with the third part of Burnet’s Sacred Theory. He
blends philosophy, Scripture, and tradition, into one magnificent
system; in the description of which he displays a strength of
fancy not inferior to that of Milton himself.}

The condemnation of the wisest and most virtuous of the Pagans,
on account of their ignorance or disbelief of the divine truth,
seems to offend the reason and the humanity of the present age. \textsuperscript{70}
But the primitive church, whose faith was of a much firmer
consistence, delivered over, without hesitation, to eternal
torture, the far greater part of the human species. A charitable
hope might perhaps be indulged in favor of Socrates, or some
other sages of antiquity, who had consulted the light of reason
before that of the gospel had arisen.\textsuperscript{71} But it was unanimously
affirmed, that those who, since the birth or the death of Christ,
had obstinately persisted in the worship of the dæmons, neither
deserved nor could expect a pardon from the irritated justice of
the Deity. These rigid sentiments, which had been unknown to the
ancient world, appear to have infused a spirit of bitterness into
a system of love and harmony. The ties of blood and friendship
were frequently torn asunder by the difference of religious
faith; and the Christians, who, in this world, found themselves
oppressed by the power of the Pagans, were sometimes seduced by
resentment and spiritual pride to delight in the prospect of
their future triumph. “You are fond of spectacles,” exclaims the
stern Tertullian; “expect the greatest of all spectacles, the
last and eternal judgment of the universe.\textsuperscript{71b} How shall I
admire, how laugh, how rejoice, how exult, when I behold so many
proud monarchs, so many fancied gods, groaning in the lowest
abyss of darkness; so many magistrates, who persecuted the name
of the Lord, liquefying in fiercer fires than they ever kindled
against the Christians; so many sage philosophers blushing in
red-hot flames with their deluded scholars; so many celebrated
poets trembling before the tribunal, not of Minos, but of Christ;
so many tragedians, more tuneful in the expression of their own
sufferings; so many dancers.”

\textsuperscript{711} But the humanity of the reader will permit me to draw a veil
over the rest of this infernal description, which the zealous
African pursues in a long variety of affected and unfeeling
witticisms.\textsuperscript{72}

\pagenote[70]{And yet whatever may be the language of
individuals, it is still the public doctrine of all the Christian
churches; nor can even our own refuse to admit the conclusions
which must be drawn from the viiith and the xviiith of her
Articles. The Jansenists, who have so diligently studied the
works of the fathers, maintain this sentiment with distinguished
zeal; and the learned M. de Tillemont never dismisses a virtuous
emperor without pronouncing his damnation. Zuinglius is perhaps
the only leader of a party who has ever adopted the milder
sentiment, and he gave no less offence to the Lutherans than to
the Catholics. See Bossuet, Histoire des Variations des Eglises
Protestantes, l. ii. c. 19—22.}

\pagenote[71]{Justin and Clemens of Alexandria allow that some of
the philosophers were instructed by the Logos; confounding its
double signification of the human reason, and of the Divine
Word.}

\pagenote[711]{This translation is not exact: the first sentence
is imperfect. Tertullian says, Ille dies nationibus insperatus,
ille derisus, cum tanta sacculi vetustas et tot ejus nativitates
uno igne haurientur. The text does not authorize the exaggerated
expressions, so many magistrates, so many sago philosophers, so
many poets, \&c.; but simply magistrates, philosophers, poets.—G.
—It is not clear that Gibbon’s version or paraphrase is
incorrect: Tertullian writes, tot tantosque reges item præsides,
\&c.—M.}

\pagenote[71b]{Tertullian, de Spectaculis, c. 30. In order to
ascertain the degree of authority which the zealous African had
acquired it may be sufficient to allege the testimony of Cyprian,
the doctor and guide of all the western churches. (See Prudent.
Hym. xiii. 100.) As often as he applied himself to his daily
study of the writings of Tertullian, he was accustomed to say,
“Da mihi magistrum, Give me my master.” (Hieronym. de Viris
Illustribus, tom. i. p. 284.)}

\pagenote[72]{The object of Tertullian’s vehemence in his
Treatise, was to keep the Christians away from the secular games
celebrated by the Emperor Severus: It has not prevented him from
showing himself in other places full of benevolence and charity
towards unbelievers: the spirit of the gospel has sometimes
prevailed over the violence of human passions: Qui ergo putaveris
nihil nos de salute Cæsaris curare (he says in his Apology)
inspice Dei voces, literas nostras. Scitote ex illis præceptum
esse nobis ad redudantionem, benignitates etiam pro inimicis Deum
orare, et pro persecutoribus cona precari. Sed etiam nominatim
atque manifeste orate inquit (Christus) pro regibus et pro
principibus et potestatibus ut omnia sint tranquilla vobis Tert.
Apol. c. 31.—G. ——It would be wiser for Christianity, retreating
upon its genuine records in the New Testament, to disclaim this
fierce African, than to identify itself with his furious
invectives by unsatisfactory apologies for their unchristian
fanaticism.—M.}

Doubtless there were many among the primitive Christians of a
temper more suitable to the meekness and charity of their
profession. There were many who felt a sincere compassion for the
danger of their friends and countrymen, and who exerted the most
benevolent zeal to save them from the impending destruction.

The careless Polytheist, assailed by new and unexpected terrors,
against which neither his priests nor his philosophers could
afford him any certain protection, was very frequently terrified
and subdued by the menace of eternal tortures. His fears might
assist the progress of his faith and reason; and if he could once
persuade himself to suspect that the Christian religion might
possibly be true, it became an easy task to convince him that it
was the safest and most prudent party that he could possibly
embrace.

III. The supernatural gifts, which even in this life were
ascribed to the Christians above the rest of mankind, must have
conduced to their own comfort, and very frequently to the
conviction of infidels. Besides the occasional prodigies, which
might sometimes be effected by the immediate interposition of the
Deity when he suspended the laws of Nature for the service of
religion, the Christian church, from the time of the apostles and
their first disciples,\textsuperscript{73} has claimed an uninterrupted succession
of miraculous powers, the gift of tongues, of vision, and of
prophecy, the power of expelling dæmons, of healing the sick, and
of raising the dead. The knowledge of foreign languages was
frequently communicated to the contemporaries of Irenæus, though
Irenæus himself was left to struggle with the difficulties of a
barbarous dialect, whilst he preached the gospel to the natives
of Gaul.\textsuperscript{74} The divine inspiration, whether it was conveyed in
the form of a waking or of a sleeping vision, is described as a
favor very liberally bestowed on all ranks of the faithful, on
women as on elders, on boys as well as upon bishops. When their
devout minds were sufficiently prepared by a course of prayer, of
fasting, and of vigils, to receive the extraordinary impulse,
they were transported out of their senses, and delivered in
ecstasy what was inspired, being mere organs of the Holy Spirit,
just as a pipe or flute is of him who blows into it.\textsuperscript{75} We may
add, that the design of these visions was, for the most part,
either to disclose the future history, or to guide the present
administration, of the church. The expulsion of the dæmons from
the bodies of those unhappy persons whom they had been permitted
to torment, was considered as a signal though ordinary triumph of
religion, and is repeatedly alleged by the ancient apoligists, as
the most convincing evidence of the truth of Christianity. The
awful ceremony was usually performed in a public manner, and in
the presence of a great number of spectators; the patient was
relieved by the power or skill of the exorcist, and the
vanquished dæmon was heard to confess that he was one of the
fabled gods of antiquity, who had impiously usurped the adoration
of mankind.\textsuperscript{76} But the miraculous cure of diseases of the most
inveterate or even preternatural kind can no longer occasion any
surprise, when we recollect, that in the days of Irenæus, about
the end of the second century, the resurrection of the dead was
very far from being esteemed an uncommon event; that the miracle
was frequently performed on necessary occasions, by great fasting
and the joint supplication of the church of the place, and that
the persons thus restored to their prayers had lived afterwards
among them many years.\textsuperscript{77} At such a period, when faith could
boast of so many wonderful victories over death, it seems
difficult to account for the scepticism of those philosophers,
who still rejected and derided the doctrine of the resurrection.
A noble Grecian had rested on this important ground the whole
controversy, and promised Theophilus, Bishop of Antioch, that if
he could be gratified with the sight of a single person who had
been actually raised from the dead, he would immediately embrace
the Christian religion. It is somewhat remarkable, that the
prelate of the first eastern church, however anxious for the
conversion of his friend, thought proper to decline this fair and
reasonable challenge.\textsuperscript{78}

\pagenote[73]{Notwithstanding the evasions of Dr. Middleton, it
is impossible to overlook the clear traces of visions and
inspiration, which may be found in the apostolic fathers. * Note:
Gibbon should have noticed the distinct and remarkable passage
from Chrysostom, quoted by Middleton, (Works, vol. i. p. 105,) in
which he affirms the long discontinuance of miracles as a
notorious fact.—M.}

\pagenote[74]{Irenæus adv. Hæres. Proem. p.3 Dr. Middleton (Free
Inquiry, p. 96, \&c.) observes, that as this pretension of all
others was the most difficult to support by art, it was the
soonest given up. The observation suits his hypothesis. * Note:
This passage of Irenæus contains no allusion to the gift of
tongues; it is merely an apology for a rude and unpolished Greek
style, which could not be expected from one who passed his life
in a remote and barbarous province, and was continually obliged
to speak the Celtic language.—M. Note: Except in the life of
Pachomius, an Egyptian monk of the fourth century. (see Jortin,
Ecc. Hist. i. p. 368, edit. 1805,) and the latter (not earlier)
lives of Xavier, there is no claim laid to the gift of tongues
since the time of Irenæus; and of this claim, Xavier’s own
letters are profoundly silent. See Douglas’s Criterion, p. 76
edit. 1807.—M.}

\pagenote[75]{Athenagoras in Legatione. Justin Martyr, Cohort. ad
Gentes Tertullian advers. Marcionit. l. iv. These descriptions
are not very unlike the prophetic fury, for which Cicero (de
Divinat.ii. 54) expresses so little reverence.}

\pagenote[76]{Tertullian (Apolog. c. 23) throws out a bold
defiance to the Pagan magistrates. Of the primitive miracles, the
power of exorcising is the only one which has been assumed by
Protestants. * Note: But by Protestants neither of the most
enlightened ages nor most reasoning minds.—M.}

\pagenote[77]{Irenæus adv. Hæreses, l. ii. 56, 57, l. v. c. 6.
Mr. Dodwell (Dissertat. ad Irenæum, ii. 42) concludes, that the
second century was still more fertile in miracles than the first.
* Note: It is difficult to answer Middleton’s objection to this
statement of Irenæus: “It is very strange, that from the time of
the apostles there is not a single instance of this miracle to be
found in the three first centuries; except a single case,
slightly intimated in Eusebius, from the Works of Papias; which
he seems to rank among the other fabulous stories delivered by
that weak man.” Middleton, Works, vol. i. p. 59. Bp. Douglas
(Criterion, p 389) would consider Irenæus to speak of what had
“been performed formerly.” not in his own time.—M.}

\pagenote[78]{Theophilus ad Autolycum, l. i. p. 345. Edit.
Benedictin. Paris, 1742. * Note: A candid sceptic might discern
some impropriety in the Bishop being called upon to perform a
miracle on demand.—M.}

The miracles of the primitive church, after obtaining the
sanction of ages, have been lately attacked in a very free and
ingenious inquiry,\textsuperscript{79} which, though it has met with the most
favorable reception from the public, appears to have excited a
general scandal among the divines of our own as well as of the
other Protestant churches of Europe.\textsuperscript{80} Our different sentiments
on this subject will be much less influenced by any particular
arguments, than by our habits of study and reflection; and, above
all, by the degree of evidence which we have accustomed ourselves
to require for the proof of a miraculous event. The duty of an
historian does not call upon him to interpose his private
judgment in this nice and important controversy; but he ought not
to dissemble the difficulty of adopting such a theory as may
reconcile the interest of religion with that of reason, of making
a proper application of that theory, and of defining with
precision the limits of that happy period, exempt from error and
from deceit, to which we might be disposed to extend the gift of
supernatural powers. From the first of the fathers to the last of
the popes, a succession of bishops, of saints, of martyrs, and of
miracles, is continued without interruption; and the progress of
superstition was so gradual, and almost imperceptible, that we
know not in what particular link we should break the chain of
tradition. Every age bears testimony to the wonderful events by
which it was distinguished, and its testimony appears no less
weighty and respectable than that of the preceding generation,
till we are insensibly led on to accuse our own inconsistency, if
in the eighth or in the twelfth century we deny to the venerable
Bede, or to the holy Bernard, the same degree of confidence
which, in the second century, we had so liberally granted to
Justin or to Irenæus.\textsuperscript{81} If the truth of any of those miracles is
appreciated by their apparent use and propriety, every age had
unbelievers to convince, heretics to confute, and idolatrous
nations to convert; and sufficient motives might always be
produced to justify the interposition of Heaven. And yet, since
every friend to revelation is persuaded of the reality, and every
reasonable man is convinced of the cessation, of miraculous
powers, it is evident that there must have been \textit{some period} in
which they were either suddenly or gradually withdrawn from the
Christian church. Whatever æra is chosen for that purpose, the
death of the apostles, the conversion of the Roman empire, or the
extinction of the Arian heresy,\textsuperscript{82} the insensibility of the
Christians who lived at that time will equally afford a just
matter of surprise. They still supported their pretensions after
they had lost their power. Credulity performed the office of
faith; fanaticism was permitted to assume the language of
inspiration, and the effects of accident or contrivance were
ascribed to supernatural causes. The recent experience of genuine
miracles should have instructed the Christian world in the ways
of Providence, and habituated their eye (if we may use a very
inadequate expression) to the style of the divine artist. Should
the most skilful painter of modern Italy presume to decorate his
feeble imitations with the name of Raphæl or of Correggio, the
insolent fraud would be soon discovered, and indignantly
rejected.

\pagenote[79]{Dr. Middleton sent out his Introduction in the year
1747, published his Free Inquiry in 1749, and before his death,
which happened in 1750, he had prepared a vindication of it
against his numerous adversaries.}

\pagenote[80]{The university of Oxford conferred degrees on his
opponents. From the indignation of Mosheim, (p. 221,) we may
discover the sentiments of the Lutheran divines. * Note: Yet many
Protestant divines will now without reluctance confine miracles
to the time of the apostles, or at least to the first century.—M}

\pagenote[81]{It may seem somewhat remarkable, that Bernard of
Clairvaux, who records so many miracles of his friend St.
Malachi, never takes any notice of his own, which, in their turn,
however, are carefully related by his companions and disciples.
In the long series of ecclesiastical history, does there exist a
single instance of a saint asserting that he himself possessed
the gift of miracles?}

\pagenote[82]{The conversion of Constantine is the æra which is
most usually fixed by Protestants. The more rational divines are
unwilling to admit the miracles of the ivth, whilst the more
credulous are unwilling to reject those of the vth century. *
Note: All this appears to proceed on the principle that any
distinct line can be drawn in an unphilosophic age between
wonders and miracles, or between what piety, from their
unexpected and extraordinary nature, the marvellous concurrence
of secondary causes to some remarkable end, may consider
providential interpositions, and miracles strictly so called, in
which the laws of nature are suspended or violated. It is
impossible to assign, on one side, limits to human credulity, on
the other, to the influence of the imagination on the bodily
frame; but some of the miracles recorded in the Gospels are such
palpable impossibilities, according to the known laws and
operations of nature, that if recorded on sufficient evidence,
and the evidence we believe to be that of eye-witnesses, we
cannot reject them, without either asserting, with Hume, that no
evidence can prove a miracle, or that the Author of Nature has no
power of suspending its ordinary laws. But which of the
post-apostolic miracles will bear this test?—M.}

Whatever opinion may be entertained of the miracles of the
primitive church since the time of the apostles, this unresisting
softness of temper, so conspicuous among the believers of the
second and third centuries, proved of some accidental benefit to
the cause of truth and religion. In modern times, a latent and
even involuntary scepticism adheres to the most pious
dispositions. Their admission of supernatural truths is much less
an active consent than a cold and passive acquiescence.
Accustomed long since to observe and to respect the invariable
order of Nature, our reason, or at least our imagination, is not
sufficiently prepared to sustain the visible action of the Deity.

But, in the first ages of Christianity, the situation of mankind
was extremely different. The most curious, or the most credulous,
among the Pagans, were often persuaded to enter into a society
which asserted an actual claim of miraculous powers. The
primitive Christians perpetually trod on mystic ground, and their
minds were exercised by the habits of believing the most
extraordinary events. They felt, or they fancied, that on every
side they were incessantly assaulted by dæmons, comforted by
visions, instructed by prophecy, and surprisingly delivered from
danger, sickness, and from death itself, by the supplications of
the church. The real or imaginary prodigies, of which they so
frequently conceived themselves to be the objects, the
instruments, or the spectators, very happily disposed them to
adopt with the same ease, but with far greater justice, the
authentic wonders of the evangelic history; and thus miracles
that exceeded not the measure of their own experience, inspired
them with the most lively assurance of mysteries which were
acknowledged to surpass the limits of their understanding. It is
this deep impression of supernatural truths which has been so
much celebrated under the name of faith; a state of mind
described as the surest pledge of the divine favor and of future
felicity, and recommended as the first, or perhaps the only merit
of a Christian. According to the more rigid doctors, the moral
virtues, which may be equally practised by infidels, are
destitute of any value or efficacy in the work of our
justification.

\section{Part \thesection.}

IV. But the primitive Christian demonstrated his faith by his
virtues; and it was very justly supposed that the divine
persuasion, which enlightened or subdued the understanding, must,
at the same time, purify the heart, and direct the actions, of
the believer. The first apologists of Christianity who justify
the innocence of their brethren, and the writers of a later
period who celebrate the sanctity of their ancestors, display, in
the most lively colors, the reformation of manners which was
introduced into the world by the preaching of the gospel. As it
is my intention to remark only such human causes as were
permitted to second the influence of revelation, I shall slightly
mention two motives which might naturally render the lives of the
primitive Christians much purer and more austere than those of
their Pagan contemporaries, or their degenerate successors;
repentance for their past sins, and the laudable desire of
supporting the reputation of the society in which they were
engaged.\textsuperscript{83}

\pagenote[83]{These, in the opinion of the editor, are the most
uncandid paragraphs in Gibbon’s History. He ought either, with
manly courage, to have denied the moral reformation introduced by
Christianity, or fairly to have investigated all its motives; not
to have confined himself to an insidious and sarcastic
description of the less pure and generous elements of the
Christian character as it appeared even at that early time.—M.}

It is a very ancient reproach, suggested by the ignorance or the
malice of infidelity, that the Christians allured into their
party the most atrocious criminals, who, as soon as they were
touched by a sense of remorse, were easily persuaded to wash
away, in the water of baptism, the guilt of their past conduct,
for which the temples of the gods refused to grant them any
expiation. But this reproach, when it is cleared from
misrepresentation, contributes as much to the honor as it did to
the increase of the church. The friends of Christianity may
acknowledge without a blush that many of the most eminent saints
had been before their baptism the most abandoned sinners. Those
persons, who in the world had followed, though in an imperfect
manner, the dictates of benevolence and propriety, derived such a
calm satisfaction from the opinion of their own rectitude, as
rendered them much less susceptible of the sudden emotions of
shame, of grief, and of terror, which have given birth to so many
wonderful conversions. After the example of their divine Master,
the missionaries of the gospel disdained not the society of men,
and especially of women, oppressed by the consciousness, and very
often by the effects, of their vices. As they emerged from sin
and superstition to the glorious hope of immortality, they
resolved to devote themselves to a life, not only of virtue, but
of penitence. The desire of perfection became the ruling passion
of their soul; and it is well known that, while reason embraces a
cold mediocrity, our passions hurry us, with rapid violence, over
the space which lies between the most opposite extremes.\textsuperscript{83b}

\pagenote[83b]{The imputations of Celsus and Julian, with the
defence of the fathers, are very fairly stated by Spanheim,
Commentaire sur les Cesars de Julian, p. 468.}

When the new converts had been enrolled in the number of the
faithful, and were admitted to the sacraments of the church, they
found themselves restrained from relapsing into their past
disorders by another consideration of a less spiritual, but of a
very innocent and respectable nature. Any particular society that
has departed from the great body of the nation, or the religion
to which it belonged, immediately becomes the object of universal
as well as invidious observation. In proportion to the smallness
of its numbers, the character of the society may be affected by
the virtues and vices of the persons who compose it; and every
member is engaged to watch with the most vigilant attention over
his own behavior, and over that of his brethren, since, as he
must expect to incur a part of the common disgrace, he may hope
to enjoy a share of the common reputation. When the Christians of
Bithynia were brought before the tribunal of the younger Pliny,
they assured the proconsul, that, far from being engaged in any
unlawful conspiracy, they were bound by a solemn obligation to
abstain from the commission of those crimes which disturb the
private or public peace of society, from theft, robbery,
adultery, perjury, and fraud.\textsuperscript{84} \textsuperscript{841} Near a century afterwards,
Tertullian, with an honest pride, could boast, that very few
Christians had suffered by the hand of the executioner, except on
account of their religion.\textsuperscript{85} Their serious and sequestered life,
averse to the gay luxury of the age, inured them to chastity,
temperance, economy, and all the sober and domestic virtues. As
the greater number were of some trade or profession, it was
incumbent on them, by the strictest integrity and the fairest
dealing, to remove the suspicions which the profane are too apt
to conceive against the appearances of sanctity. The contempt of
the world exercised them in the habits of humility, meekness, and
patience. The more they were persecuted, the more closely they
adhered to each other. Their mutual charity and unsuspecting
confidence has been remarked by infidels, and was too often
abused by perfidious friends.\textsuperscript{86}

\pagenote[84]{Plin. Epist. x. 97. * Note: Is not the sense of
Tertullian rather, if guilty of any other offence, he had thereby
ceased to be a Christian?—M.}

\pagenote[841]{And this blamelessness was fully admitted by the
candid and enlightened Roman.—M.}

\pagenote[85]{Tertullian, Apolog. c. 44. He adds, however, with
some degree of hesitation, “Aut si aliud, jam non Christianus.” *
Note: Tertullian says positively no Christian, nemo illic
Christianus; for the rest, the limitation which he himself
subjoins, and which Gibbon quotes in the foregoing note,
diminishes the force of this assertion, and appears to prove that
at least he knew none such.—G.}

\pagenote[86]{The philosopher Peregrinus (of whose life and death
Lucian has left us so entertaining an account) imposed, for a
long time, on the credulous simplicity of the Christians of
Asia.}

It is a very honorable circumstance for the morals of the
primitive Christians, that even their faults, or rather errors,
were derived from an excess of virtue. The bishops and doctors of
the church, whose evidence attests, and whose authority might
influence, the professions, the principles, and even the practice
of their contemporaries, had studied the Scriptures with less
skill than devotion; and they often received, in the most literal
sense, those rigid precepts of Christ and the apostles, to which
the prudence of succeeding commentators has applied a looser and
more figurative mode of interpretation. Ambitious to exalt the
perfection of the gospel above the wisdom of philosophy, the
zealous fathers have carried the duties of self-mortification, of
purity, and of patience, to a height which it is scarcely
possible to attain, and much less to preserve, in our present
state of weakness and corruption. A doctrine so extraordinary and
so sublime must inevitably command the veneration of the people;
but it was ill calculated to obtain the suffrage of those worldly
philosophers who, in the conduct of this transitory life, consult
only the feelings of nature and the interest of society.\textsuperscript{87}

\pagenote[87]{See a very judicious treatise of Barbeyrac sur la
Morale des Peres.}

There are two very natural propensities which we may distinguish
in the most virtuous and liberal dispositions, the love of
pleasure and the love of action. If the former is refined by art
and learning, improved by the charms of social intercourse, and
corrected by a just regard to economy, to health, and to
reputation, it is productive of the greatest part of the
happiness of private life. The love of action is a principle of a
much stronger and more doubtful nature. It often leads to anger,
to ambition, and to revenge; but when it is guided by the sense
of propriety and benevolence, it becomes the parent of every
virtue, and if those virtues are accompanied with equal
abilities, a family, a state, or an empire may be indebted for
their safety and prosperity to the undaunted courage of a single
man. To the love of pleasure we may therefore ascribe most of the
agreeable, to the love of action we may attribute most of the
useful and respectable, qualifications. The character in which
both the one and the other should be united and harmonized would
seem to constitute the most perfect idea of human nature. The
insensible and inactive disposition, which should be supposed
alike destitute of both, would be rejected, by the common consent
of mankind, as utterly incapable of procuring any happiness to
the individual, or any public benefit to the world. But it was
not in \textit{this} world that the primitive Christians were desirous
of making themselves either agreeable or useful.\textsuperscript{871}

\pagenote[871]{El que me fait cette homelie semi-stoicienne,
semi-epicurienne? t’on jamais regarde l’amour du plaisir comme
l’un des principes de la perfection morale? Et de quel droit
faites vous de l’amour de l’action, et de l’amour du plaisir, les
seuls elemens de l’etre humain? Est ce que vous faites
abstraction de la verite en elle-meme, de la conscience et du
sentiment du devoir? Est ce que vous ne sentez point, par
exemple, que le sacrifice du moi a la justice et a la verite, est
aussi dans le coeur de l’homme: que tout n’est pas pour lui
action ou plaisir, et que dans le bien ce n’est pas le mouvement,
mais la verite, qu’il cherche? Et puis * * Thucy dide et Tacite.
ces maitres de l’histoire, ont ils jamais introduits dans leur
recits un fragment de dissertation sur le plaisir et sur
l’action. Villemain Cours de Lit. Franc part ii. Lecon v.—M.}

The acquisition of knowledge, the exercise of our reason or
fancy, and the cheerful flow of unguarded conversation, may
employ the leisure of a liberal mind. Such amusements, however,
were rejected with abhorrence, or admitted with the utmost
caution, by the severity of the fathers, who despised all
knowledge that was not useful to salvation, and who considered
all levity of discours as a criminal abuse of the gift of speech.
In our present state of existence the body is so inseparably
connected with the soul, that it seems to be our interest to
taste, with innocence and moderation, the enjoyments of which
that faithful companion is susceptible. Very different was the
reasoning of our devout predecessors; vainly aspiring to imitate
the perfection of angels, they disdained, or they affected to
disdain, every earthly and corporeal delight.\textsuperscript{88} Some of our
senses indeed are necessary for our preservation, others for our
subsistence, and others again for our information; and thus far
it was impossible to reject the use of them. The first sensation
of pleasure was marked as the first moment of their abuse. The
unfeeling candidate for heaven was instructed, not only to resist
the grosser allurements of the taste or smell, but even to shut
his ears against the profane harmony of sounds, and to view with
indifference the most finished productions of human art. Gay
apparel, magnificent houses, and elegant furniture, were supposed
to unite the double guilt of pride and of sensuality; a simple
and mortified appearance was more suitable to the Christian who
was certain of his sins and doubtful of his salvation. In their
censures of luxury the fathers are extremely minute and
circumstantial;\textsuperscript{89} and among the various articles which excite
their pious indignation we may enumerate false hair, garments of
any color except white, instruments of music, vases of gold or
silver, downy pillows, (as Jacob reposed his head on a stone,)
white bread, foreign wines, public salutations, the use of warm
baths, and the practice of shaving the beard, which, according to
the expression of Tertullian, is a lie against our own faces, and
an impious attempt to improve the works of the Creator.\textsuperscript{90} When
Christianity was introduced among the rich and the polite, the
observation of these singular laws was left, as it would be at
present, to the few who were ambitious of superior sanctity. But
it is always easy, as well as agreeable, for the inferior ranks
of mankind to claim a merit from the contempt of that pomp and
pleasure which fortune has placed beyond their reach. The virtue
of the primitive Christians, like that of the first Romans, was
very frequently guarded by poverty and ignorance.

\pagenote[88]{Lactant. Institut. Divin. l. vi. c. 20, 21, 22.}

\pagenote[89]{Consult a work of Clemens of Alexandria, entitled
The Pædagogue, which contains the rudiments of ethics, as they
were taught in the most celebrated of the Christian schools.}

\pagenote[90]{Tertullian, de Spectaculis, c. 23. Clemens
Alexandrin. Pædagog. l. iii. c. 8.}

The chaste severity of the fathers, in whatever related to the
commerce of the two sexes, flowed from the same principle; their
abhorrence of every enjoyment which might gratify the sensual,
and degrade the spiritual nature of man. It was their favorite
opinion, that if Adam had preserved his obedience to the Creator,
he would have lived forever in a state of virgin purity, and that
some harmless mode of vegetation might have peopled paradise with
a race of innocent and immortal beings.\textsuperscript{91} The use of marriage
was permitted only to his fallen posterity, as a necessary
expedient to continue the human species, and as a restraint,
however imperfect, on the natural licentiousness of desire. The
hesitation of the orthodox casuists on this interesting subject,
betrays the perplexity of men, unwilling to approve an
institution which they were compelled to tolerate.\textsuperscript{92} The
enumeration of the very whimsical laws, which they most
circumstantially imposed on the marriage-bed, would force a smile
from the young and a blush from the fair. It was their unanimous
sentiment that a first marriage was adequate to all the purposes
of nature and of society. The sensual connection was refined into
a resemblance of the mystic union of Christ with his church, and
was pronounced to be indissoluble either by divorce or by death.
The practice of second nuptials was branded with the name of a
legal adultery; and the persons who were guilty of so scandalous
an offence against Christian purity, were soon excluded from the
honors, and even from the alms, of the church.\textsuperscript{93} Since desire
was imputed as a crime, and marriage was tolerated as a defect,
it was consistent with the same principles to consider a state of
celibacy as the nearest approach to the divine perfection. It was
with the utmost difficulty that ancient Rome could support the
institution of six vestals;\textsuperscript{94} but the primitive church was
filled with a number of persons of either sex, who had devoted
themselves to the profession of perpetual chastity.\textsuperscript{95} A few of
these, among whom we may reckon the learned Origen, judged it the
most prudent to disarm the tempter.\textsuperscript{96} Some were insensible and
some were invincible against the assaults of the flesh.
Disdaining an ignominious flight, the virgins of the warm climate
of Africa encountered the enemy in the closest engagement; they
permitted priests and deacons to share their bed, and gloried
amidst the flames in their unsullied purity. But insulted Nature
sometimes vindicated her rights, and this new species of
martyrdom served only to introduce a new scandal into the church. \textsuperscript{97}
Among the Christian ascetics, however, (a name which they soon
acquired from their painful exercise,) many, as they were less
presumptuous, were probably more successful. The loss of sensual
pleasure was supplied and compensated by spiritual pride. Even
the multitude of Pagans were inclined to estimate the merit of
the sacrifice by its apparent difficulty; and it was in the
praise of these chaste spouses of Christ that the fathers have
poured forth the troubled stream of their eloquence.\textsuperscript{98} Such are
the early traces of monastic principles and institutions, which,
in a subsequent age, have counterbalanced all the temporal
advantages of Christianity.\textsuperscript{99}

\pagenote[91]{Beausobro, Hist. Critique du Manicheisme, l. vii.
c. 3. Justin, Gregory of Nyssa, Augustin, \&c., strongly incline
to this opinion. Note: But these were Gnostic or Manichean
opinions. Beausobre distinctly describes Autustine’s bias to his
recent escape from Manicheism; and adds that he afterwards
changed his views.—M.}

\pagenote[92]{Some of the Gnostic heretics were more consistent;
they rejected the use of marriage.}

\pagenote[93]{See a chain of tradition, from Justin Martyr to
Jerome, in the Morale des Peres, c. iv. 6—26.}

\pagenote[94]{See a very curious Dissertation on the Vestals, in
the Memoires de l’Academie des Inscriptions, tom. iv. p. 161—227.
Notwithstanding the honors and rewards which were bestowed on
those virgins, it was difficult to procure a sufficient number;
nor could the dread of the most horrible death always restrain
their incontinence.}

\pagenote[95]{Cupiditatem procreandi aut unam scimus aut nullam.
Minutius Fælix, c. 31. Justin. Apolog. Major. Athenagoras in
Legat. c 28. Tertullian de Cultu Foemin. l. ii.}

\pagenote[96]{Eusebius, l. vi. 8. Before the fame of Origen had
excited envy and persecution, this extraordinary action was
rather admired than censured. As it was his general practice to
allegorize Scripture, it seems unfortunate that in this instance
only, he should have adopted the literal sense.}

\pagenote[97]{Cyprian. Epist. 4, and Dodwell, Dissertat.
Cyprianic. iii. Something like this rash attempt was long
afterwards imputed to the founder of the order of Fontevrault.
Bayle has amused himself and his readers on that very delicate
subject.}

\pagenote[98]{Dupin (Bibliotheque Ecclesiastique, tom. i. p. 195)
gives a particular account of the dialogue of the ten virgins, as
it was composed by Methodius, Bishop of Tyre. The praises of
virginity are excessive.}

\pagenote[99]{The Ascetics (as early as the second century) made
a public profession of mortifying their bodies, and of abstaining
from the use of flesh and wine. Mosheim, p. 310.}

The Christians were not less averse to the business than to the
pleasures of this world. The defence of our persons and property
they knew not how to reconcile with the patient doctrine which
enjoined an unlimited forgiveness of past injuries, and commanded
them to invite the repetition of fresh insults. Their simplicity
was offended by the use of oaths, by the pomp of magistracy, and
by the active contention of public life; nor could their humane
ignorance be convinced that it was lawful on any occasion to shed
the blood of our fellow-creatures, either by the sword of
justice, or by that of war; even though their criminal or hostile
attempts should threaten the peace and safety of the whole
community.\textsuperscript{100} It was acknowledged that, under a less perfect
law, the powers of the Jewish constitution had been exercised,
with the approbation of heaven, by inspired prophets and by
anointed kings. The Christians felt and confessed that such
institutions might be necessary for the present system of the
world, and they cheerfully submitted to the authority of their
Pagan governors. But while they inculcated the maxims of passive
obedience, they refused to take any active part in the civil
administration or the military defence of the empire. Some
indulgence might, perhaps, be allowed to those persons who,
before their conversion, were already engaged in such violent and
sanguinary occupations;\textsuperscript{101a} but it was impossible that the
Christians, without renouncing a more sacred duty, could assume
the character of soldiers, of magistrates, or of princes.\textsuperscript{102b}
This indolent, or even criminal disregard to the public welfare,
exposed them to the contempt and reproaches of the Pagans, who
very frequently asked, what must be the fate of the empire,
attacked on every side by the barbarians, if all mankind should
adopt the pusillanimous sentiments of the new sect.\textsuperscript{103} To this
insulting question the Christian apologists returned obscure and
ambiguous answers, as they were unwilling to reveal the secret
cause of their security; the expectation that, before the
conversion of mankind was accomplished, war, government, the
Roman empire, and the world itself, would be no more. It may be
observed, that, in this instance likewise, the situation of the
first Christians coincided very happily with their religious
scruples, and that their aversion to an active life contributed
rather to excuse them from the service, than to exclude them from
the honors, of the state and army.

\pagenote[100]{See the Morale des Peres. The same patient
principles have been revived since the Reformation by the
Socinians, the modern Anabaptists, and the Quakers. Barclay, the
Apologist of the Quakers, has protected his brethren by the
authority of the primitive Christian; p. 542-549}

\pagenote[101a]{Tertullian, Apolog. c. 21. De Idololatria, c. 17,
18. Origen contra Celsum, l. v. p. 253, l. vii. p. 348, l. viii.
p. 423-428.}

\pagenote[102b]{Tertullian (de Corona Militis, c. 11) suggested
to them the expedient of deserting; a counsel which, if it had
been generally known, was not very proper to conciliate the favor
of the emperors towards the Christian sect. * Note: There is
nothing which ought to astonish us in the refusal of the
primitive Christians to take part in public affairs; it was the
natural consequence of the contrariety of their principles to the
customs, laws, and active life of the Pagan world. As Christians,
they could not enter into the senate, which, according to Gibbon
himself, always assembled in a temple or consecrated place, and
where each senator, before he took his seat, made a libation of a
few drops of wine, and burnt incense on the altar; as Christians,
they could not assist at festivals and banquets, which always
terminated with libations, \&c.; finally, as “the innumerable
deities and rites of polytheism were closely interwoven with
every circumstance of public and private life,” the Christians
could not participate in them without incurring, according to
their principles, the guilt of impiety. It was then much less by
an effect of their doctrine, than by the consequence of their
situation, that they stood aloof from public business. Whenever
this situation offered no impediment, they showed as much
activity as the Pagans. Proinde, says Justin Martyr, (Apol. c.
17,) nos solum Deum adoramus, et vobis in rebus aliis læti
inservimus.—G. ——-This latter passage, M. Guizot quotes in Latin;
if he had consulted the original, he would have found it to be
altogether irrelevant: it merely relates to the payment of
taxes.—M. — —Tertullian does not suggest to the soldiers the
expedient of deserting; he says that they ought to be constantly
on their guard to do nothing during their service contrary to the
law of God, and to resolve to suffer martyrdom rather than submit
to a base compliance, or openly to renounce the service. (De Cor.
Mil. ii. p. 127.) He does not positively decide that the military
service is not permitted to Christians; he ends, indeed, by
saying, Puta denique licere militiam usque ad causam coronæ.—G.
——M. Guizot is. I think, again unfortunate in his defence of
Tertullian. That father says, that many Christian soldiers had
deserted, aut deserendum statim sit, ut a multis actum. The
latter sentence, Puta, \&c, \&c., is a concession for the sake of
argument: wha follows is more to the purpose.—M. Many other
passages of Tertullian prove that the army was full of
Christians, Hesterni sumus et vestra omnia implevimus, urbes,
insulas, castella, municipia, conciliabula, castra ipsa. (Apol.
c. 37.) Navigamus et not vobiscum et militamus. (c. 42.) Origen,
in truth, appears to have maintained a more rigid opinion, (Cont.
Cels. l. viii.;) but he has often renounced this exaggerated
severity, perhaps necessary to produce great results, and he
speaks of the profession of arms as an honorable one. (l. iv. c.
218.)— G. ——On these points Christian opinion, it should seem,
was much divided Tertullian, when he wrote the De Cor. Mil., was
evidently inclining to more ascetic opinions, and Origen was of
the same class. See Neander, vol. l part ii. p. 305, edit.
1828.—M.}

\pagenote[103]{As well as we can judge from the mutilated
representation of Origen, (1. viii. p. 423,) his adversary,
Celsus, had urged his objection with great force and candor.}

\section{Part \thesection.}

V. But the human character, however it may be exalted or
depressed by a temporary enthusiasm, will return by degrees to
its proper and natural level, and will resume those passions that
seem the most adapted to its present condition. The primitive
Christians were dead to the business and pleasures of the world;
but their love of action, which could never be entirely
extinguished, soon revived, and found a new occupation in the
government of the church. A separate society, which attacked the
established religion of the empire, was obliged to adopt some
form of internal policy, and to appoint a sufficient number of
ministers, intrusted not only with the spiritual functions, but
even with the temporal direction of the Christian commonwealth.
The safety of that society, its honor, its aggrandizement, were
productive, even in the most pious minds, of a spirit of
patriotism, such as the first of the Romans had felt for the
republic, and sometimes of a similar indifference, in the use of
whatever means might probably conduce to so desirable an end. The
ambition of raising themselves or their friends to the honors and
offices of the church, was disguised by the laudable intention of
devoting to the public benefit the power and consideration,
which, for that purpose only, it became their duty to solicit. In
the exercise of their functions, they were frequently called upon
to detect the errors of heresy or the arts of faction, to oppose
the designs of perfidious brethren, to stigmatize their
characters with deserved infamy, and to expel them from the bosom
of a society whose peace and happiness they had attempted to
disturb. The ecclesiastical governors of the Christians were
taught to unite the wisdom of the serpent with the innocence of
the dove; but as the former was refined, so the latter was
insensibly corrupted, by the habits of government. In the church
as well as in the world, the persons who were placed in any
public station rendered themselves considerable by their
eloquence and firmness, by their knowledge of mankind, and by
their dexterity in business; and while they concealed from
others, and perhaps from themselves, the secret motives of their
conduct, they too frequently relapsed into all the turbulent
passions of active life, which were tinctured with an additional
degree of bitterness and obstinacy from the infusion of spiritual
zeal.

The government of the church has often been the subject, as well
as the prize, of religious contention. The hostile disputants of
Rome, of Paris, of Oxford, and of Geneva, have alike struggled to
reduce the primitive and apostolic model\textsuperscript{1041} to the respective
standards of their own policy. The few who have pursued this
inquiry with more candor and impartiality, are of opinion,\textsuperscript{105}
that the apostles declined the office of legislation, and rather
chose to endure some partial scandals and divisions, than to
exclude the Christians of a future age from the liberty of
varying their forms of ecclesiastical government according to the
changes of times and circumstances. The scheme of policy, which,
under their approbation, was adopted for the use of the first
century, may be discovered from the practice of Jerusalem, of
Ephesus, or of Corinth. The societies which were instituted in
the cities of the Roman empire were united only by the ties of
faith and charity. Independence and equality formed the basis of
their internal constitution. The want of discipline and human
learning was supplied by the occasional assistance of the
\textit{prophets},\textsuperscript{106} who were called to that function without
distinction of age, of sex,\textsuperscript{1061} or of natural abilities, and
who, as often as they felt the divine impulse, poured forth the
effusions of the Spirit in the assembly of the faithful. But
these extraordinary gifts were frequently abused or misapplied by
the prophetic teachers. They displayed them at an improper
season, presumptuously disturbed the service of the assembly,
and, by their pride or mistaken zeal, they introduced,
particularly into the apostolic church of Corinth, a long and
melancholy train of disorders.\textsuperscript{107} As the institution of prophets
became useless, and even pernicious, their powers were withdrawn,
and their office abolished. The public functions of religion were
solely intrusted to the established ministers of the church, the
\textit{bishops} and the \textit{presbyters;} two appellations which, in their
first origin, appear to have distinguished the same office and
the same order of persons. The name of Presbyter was expressive
of their age, or rather of their gravity and wisdom. The title of
Bishop denoted their inspection over the faith and manners of the
Christians who were committed to their pastoral care. In
proportion to the respective numbers of the faithful, a larger or
smaller number of these \textit{episcopal presbyters} guided each infant
congregation with equal authority and with united counsels.\textsuperscript{108}

\pagenote[1041]{The aristocratical party in France, as well as in
England, has strenuously maintained the divine origin of bishops.
But the Calvinistical presbyters were impatient of a superior;
and the Roman Pontiff refused to acknowledge an equal. See Fra
Paolo.}

\pagenote[105]{In the history of the Christian hierarchy, I have,
for the most part, followed the learned and candid Mosheim.}

\pagenote[106]{For the prophets of the primitive church, see
Mosheim, Dissertationes ad Hist. Eccles. pertinentes, tom. ii. p.
132—208.}

\pagenote[1061]{St. Paul distinctly reproves the intrusion of
females into the prophets office. 1 Cor. xiv. 34, 35. 1 Tim. ii.
11.—M.}

\pagenote[107]{See the epistles of St. Paul, and of Clemens, to
the Corinthians. * Note: The first ministers established in the
church were the deacons, appointed at Jerusalem, seven in number;
they were charged with the distribution of the alms; even females
had a share in this employment. After the deacons came the elders
or priests, charged with the maintenance of order and decorum in
the community, and to act every where in its name. The bishops
were afterwards charged to watch over the faith and the
instruction of the disciples: the apostles themselves appointed
several bishops. Tertullian, (adv. Marium, c. v.,) Clement of
Alexandria, and many fathers of the second and third century, do
not permit us to doubt this fact. The equality of rank between
these different functionaries did not prevent their functions
being, even in their origin, distinct; they became subsequently
still more so. See Plank, Geschichte der Christ. Kirch.
Verfassung., vol. i. p. 24.—G. On this extremely obscure subject,
which has been so much perplexed by passion and interest, it is
impossible to justify any opinion without entering into long and
controversial details.——It must be admitted, in opposition to
Plank, that in the New Testament, several words are sometimes
indiscriminately used. (Acts xx. v. 17, comp. with 28 Tit. i. 5
and 7. Philip. i. 1.) But it is as clear, that as soon as we can
discern the form of church government, at a period closely
bordering upon, if not within, the apostolic age, it appears with
a bishop at the head of each community, holding some superiority
over the presbyters. Whether he was, as Gibbon from Mosheim
supposes, merely an elective head of the College of Presbyters,
(for this we have, in fact, no valid authority,) or whether his
distinct functions were established on apostolic authority, is
still contested. The universal submission to this episcopacy, in
every part of the Christian world appears to me strongly to favor
the latter view.—M.}

\pagenote[108]{Hooker’s Ecclesiastical Polity, l. vii.}

But the most perfect equality of freedom requires the directing
hand of a superior magistrate: and the order of public
deliberations soon introduces the office of a president, invested
at least with the authority of collecting the sentiments, and of
executing the resolutions, of the assembly. A regard for the
public tranquillity, which would so frequently have been
interrupted by annual or by occasional elections, induced the
primitive Christians to constitute an honorable and perpetual
magistracy, and to choose one of the wisest and most holy among
their presbyters to execute, during his life, the duties of their
ecclesiastical governor. It was under these circumstances that
the lofty title of Bishop began to raise itself above the humble
appellation of Presbyter; and while the latter remained the most
natural distinction for the members of every Christian senate,
the former was appropriated to the dignity of its new president. \textsuperscript{109}
The advantages of this episcopal form of government, which
appears to have been introduced before the end of the first
century,\textsuperscript{110} were so obvious, and so important for the future
greatness, as well as the present peace, of Christianity, that it
was adopted without delay by all the societies which were already
scattered over the empire, had acquired in a very early period
the sanction of antiquity,\textsuperscript{111} and is still revered by the most
powerful churches, both of the East and of the West, as a
primitive and even as a divine establishment.\textsuperscript{112} It is needless
to observe, that the pious and humble presbyters, who were first
dignified with the episcopal title, could not possess, and would
probably have rejected, the power and pomp which now encircles
the tiara of the Roman pontiff, or the mitre of a German prelate.
But we may define, in a few words, the narrow limits of their
original jurisdiction, which was chiefly of a spiritual, though
in some instances of a temporal nature.\textsuperscript{113} It consisted in the
administration of the sacraments and discipline of the church,
the superintendency of religious ceremonies, which imperceptibly
increased in number and variety, the consecration of
ecclesiastical ministers, to whom the bishop assigned their
respective functions, the management of the public fund, and the
determination of all such differences as the faithful were
unwilling to expose before the tribunal of an idolatrous judge.
These powers, during a short period, were exercised according to
the advice of the presbyteral college, and with the consent and
approbation of the assembly of Christians. The primitive bishops
were considered only as the first of their equals, and the
honorable servants of a free people. Whenever the episcopal chair
became vacant by death, a new president was chosen among the
presbyters by the suffrage of the whole congregation, every
member of which supposed himself invested with a sacred and
sacerdotal character.\textsuperscript{114}

\pagenote[109]{See Jerome and Titum, c. i. and Epistol. 85, (in
the Benedictine edition, 101,) and the elaborate apology of
Blondel, pro sententia Hieronymi. The ancient state, as it is
described by Jerome, of the bishop and presbyters of Alexandria,
receives a remarkable confirmation from the patriarch Eutychius,
(Annal. tom. i. p. 330, Vers Pocock;) whose testimony I know not
how to reject, in spite of all the objections of the learned
Pearson in his Vindiciæ Ignatianæ, part i. c. 11.}

\pagenote[110]{See the introduction to the Apocalypse. Bishops,
under the name of angels, were already instituted in the seven
cities of Asia. And yet the epistle of Clemens (which is probably
of as ancient a date) does not lead us to discover any traces of
episcopacy either at Corinth or Rome.}

\pagenote[111]{Nulla Ecclesia sine Episcopo, has been a fact as
well as a maxim since the time of Tertullian and Irenæus.}

\pagenote[112]{After we have passed the difficulties of the first
century, we find the episcopal government universally
established, till it was interrupted by the republican genius of
the Swiss and German reformers.}

\pagenote[113]{See Mosheim in the first and second centuries.
Ignatius (ad Smyrnæos, c. 3, \&c.) is fond of exalting the
episcopal dignity. Le Clerc (Hist. Eccles. p. 569) very bluntly
censures his conduct, Mosheim, with a more critical judgment, (p.
161,) suspects the purity even of the smaller epistles.}

\pagenote[114]{Nonne et Laici sacerdotes sumus? Tertullian,
Exhort. ad Castitat. c. 7. As the human heart is still the same,
several of the observations which Mr. Hume has made on
Enthusiasm, (Essays, vol. i. p. 76, quarto edit.) may be applied
even to real inspiration. * Note: This expression was employed by
the earlier Christian writers in the sense used by St. Peter, 1
Ep ii. 9. It was the sanctity and virtue not the power of
priesthood, in which all Christians were to be equally
distinguished.—M.}

Such was the mild and equal constitution by which the Christians
were governed more than a hundred years after the death of the
apostles. Every society formed within itself a separate and
independent republic; and although the most distant of these
little states maintained a mutual as well as friendly intercourse
of letters and deputations, the Christian world was not yet
connected by any supreme authority or legislative assembly. As
the numbers of the faithful were gradually multiplied, they
discovered the advantages that might result from a closer union
of their interest and designs. Towards the end of the second
century, the churches of Greece and Asia adopted the useful
institutions of provincial synods,\textsuperscript{1141} and they may justly be
supposed to have borrowed the model of a representative council
from the celebrated examples of their own country, the
Amphictyons, the Achæan league, or the assemblies of the Ionian
cities. It was soon established as a custom and as a law, that
the bishops of the independent churches should meet in the
capital of the province at the stated periods of spring and
autumn. Their deliberations were assisted by the advice of a few
distinguished presbyters, and moderated by the presence of a
listening multitude.\textsuperscript{115} Their decrees, which were styled Canons,
regulated every important controversy of faith and discipline;
and it was natural to believe that a liberal effusion of the Holy
Spirit would be poured on the united assembly of the delegates of
the Christian people. The institution of synods was so well
suited to private ambition, and to public interest, that in the
space of a few years it was received throughout the whole empire.
A regular correspondence was established between the provincial
councils, which mutually communicated and approved their
respective proceedings; and the catholic church soon assumed the
form, and acquired the strength, of a great fœderative republic. \textsuperscript{116}

\pagenote[1141]{The synods were not the first means taken by the
insulated churches to enter into communion and to assume a
corporate character. The dioceses were first formed by the union
of several country churches with a church in a city: many
churches in one city uniting among themselves, or joining a more
considerable church, became metropolitan. The dioceses were not
formed before the beginning of the second century: before that
time the Christians had not established sufficient churches in
the country to stand in need of that union. It is towards the
middle of the same century that we discover the first traces of
the metropolitan constitution. (Probably the country churches
were founded in general by missionaries from those in the city,
and would preserve a natural connection with the parent
church.)—M. ——The provincial synods did not commence till towards
the middle of the third century, and were not the first synods.
History gives us distinct notions of the synods, held towards the
end of the second century, at Ephesus at Jerusalem, at Pontus,
and at Rome, to put an end to the disputes which had arisen
between the Latin and Asiatic churches about the celebration of
Easter. But these synods were not subject to any regular form or
periodical return; this regularity was first established with the
provincial synods, which were formed by a union of the bishops of
a district, subject to a metropolitan. Plank, p. 90. Geschichte
der Christ. Kirch. Verfassung—G}

\pagenote[115]{Acta Concil. Carthag. apud Cyprian. edit. Fell, p.
158. This council was composed of eighty-seven bishops from the
provinces of Mauritania, Numidia, and Africa; some presbyters and
deacons assisted at the assembly; præsente plebis maxima parte.}

\pagenote[116]{Aguntur præterea per Græcias illas, certis in
locis concilia, \&c Tertullian de Jejuniis, c. 13. The African
mentions it as a recent and foreign institution. The coalition of
the Christian churches is very ably explained by Mosheim, p. 164
170.}

As the legislative authority of the particular churches was
insensibly superseded by the use of councils, the bishops
obtained by their alliance a much larger share of executive and
arbitrary power; and as soon as they were connected by a sense of
their common interest, they were enabled to attack, with united
vigor, the original rights of their clergy and people. The
prelates of the third century imperceptibly changed the language
of exhortation into that of command, scattered the seeds of
future usurpations, and supplied, by scripture allegories and
declamatory rhetoric, their deficiency of force and of reason.
They exalted the unity and power of the church, as it was
represented in the episcopal office, of which every bishop
enjoyed an equal and undivided portion.\textsuperscript{117} Princes and
magistrates, it was often repeated, might boast an earthly claim
to a transitory dominion; it was the episcopal authority alone
which was derived from the Deity, and extended itself over this
and over another world. The bishops were the vicegerents of
Christ, the successors of the apostles, and the mystic
substitutes of the high priest of the Mosaic law. Their exclusive
privilege of conferring the sacerdotal character invaded the
freedom both of clerical and of popular elections; and if, in the
administration of the church, they still consulted the judgment
of the presbyters, or the inclination of the people, they most
carefully inculcated the merit of such a voluntary condescension.
The bishops acknowledged the supreme authority which resided in
the assembly of their brethren; but in the government of his
peculiar diocese, each of them exacted from his \textit{flock} the same
implicit obedience as if that favorite metaphor had been
literally just, and as if the shepherd had been of a more exalted
nature than that of his sheep.\textsuperscript{118} This obedience, however, was
not imposed without some efforts on one side, and some resistance
on the other. The democratical part of the constitution was, in
many places, very warmly supported by the zealous or interested
opposition of the inferior clergy. But their patriotism received
the ignominious epithets of faction and schism; and the episcopal
cause was indebted for its rapid progress to the labors of many
active prelates, who, like Cyprian of Carthage, could reconcile
the arts of the most ambitious statesman with the Christian
virtues which seem adapted to the character of a saint and
martyr.\textsuperscript{119}

\pagenote[117]{Cyprian, in his admired treatise De Unitate
Ecclesiæ. p. 75—86}

\pagenote[118]{We may appeal to the whole tenor of Cyprian’s
conduct, of his doctrine, and of his epistles. Le Clerc, in a
short life of Cyprian, (Bibliotheque Universelle, tom. xii. p.
207—378,) has laid him open with great freedom and accuracy.}

\pagenote[119]{If Novatus, Felicissimus, \&c., whom the Bishop of
Carthage expelled from his church, and from Africa, were not the
most detestable monsters of wickedness, the zeal of Cyprian must
occasionally have prevailed over his veracity. For a very just
account of these obscure quarrels, see Mosheim, p. 497—512.}

The same causes which at first had destroyed the equality of the
presbyters introduced among the bishops a preeminence of rank,
and from thence a superiority of jurisdiction. As often as in the
spring and autumn they met in provincial synod, the difference of
personal merit and reputation was very sensibly felt among the
members of the assembly, and the multitude was governed by the
wisdom and eloquence of the few. But the order of public
proceedings required a more regular and less invidious
distinction; the office of perpetual presidents in the councils
of each province was conferred on the bishops of the principal
city; and these aspiring prelates, who soon acquired the lofty
titles of Metropolitans and Primates, secretly prepared
themselves to usurp over their episcopal brethren the same
authority which the bishops had so lately assumed above the
college of presbyters.\textsuperscript{120} Nor was it long before an emulation of
preeminence and power prevailed among the Metropolitans
themselves, each of them affecting to display, in the most
pompous terms, the temporal honors and advantages of the city
over which he presided; the numbers and opulence of the
Christians who were subject to their pastoral care; the saints
and martyrs who had arisen among them; and the purity with which
they preserved the tradition of the faith, as it had been
transmitted through a series of orthodox bishops from the apostle
or the apostolic disciple, to whom the foundation of their church
was ascribed.\textsuperscript{121} From every cause, either of a civil or of an
ecclesiastical nature, it was easy to foresee that Rome must
enjoy the respect, and would soon claim the obedience, of the
provinces. The society of the faithful bore a just proportion to
the capital of the empire; and the Roman church was the greatest,
the most numerous, and, in regard to the West, the most ancient
of all the Christian establishments, many of which had received
their religion from the pious labors of her missionaries. Instead
of one apostolic founder, the utmost boast of Antioch, of
Ephesus, or of Corinth, the banks of the Tyber were supposed to
have been honored with the preaching and martyrdom of the two
most eminent among the apostles;\textsuperscript{122} and the bishops of Rome very
prudently claimed the inheritance of whatsoever prerogatives were
attributed either to the person or to the office of St. Peter. \textsuperscript{123}
The bishops of Italy and of the provinces were disposed to
allow them a primacy of order and association (such was their
very accurate expression) in the Christian aristocracy.\textsuperscript{124} But
the power of a monarch was rejected with abhorrence, and the
aspiring genius of Rome experienced from the nations of Asia and
Africa a more vigorous resistance to her spiritual, than she had
formerly done to her temporal, dominion. The patriotic Cyprian,
who ruled with the most absolute sway the church of Carthage and
the provincial synods, opposed with resolution and success the
ambition of the Roman pontiff, artfully connected his own cause
with that of the eastern bishops, and, like Hannibal, sought out
new allies in the heart of Asia.\textsuperscript{125} If this Punic war was
carried on without any effusion of blood, it was owing much less
to the moderation than to the weakness of the contending
prelates. Invectives and excommunications were \textit{their} only
weapons; and these, during the progress of the whole controversy,
they hurled against each other with equal fury and devotion. The
hard necessity of censuring either a pope, or a saint and martyr,
distresses the modern Catholics whenever they are obliged to
relate the particulars of a dispute in which the champions of
religion indulged such passions as seem much more adapted to the
senate or to the camp.\textsuperscript{126}

\pagenote[120]{Mosheim, p. 269, 574. Dupin, Antiquæ Eccles.
Disciplin. p. 19, 20.}

\pagenote[121]{Tertullian, in a distinct treatise, has pleaded
against the heretics the right of prescription, as it was held by
the apostolic churches.}

\pagenote[122]{The journey of St. Peter to Rome is mentioned by
most of the ancients, (see Eusebius, ii. 25,) maintained by all
the Catholics, allowed by some Protestants, (see Pearson and
Dodwell de Success. Episcop. Roman,) but has been vigorously
attacked by Spanheim, (Miscellanes Sacra, iii. 3.) According to
Father Hardouin, the monks of the thirteenth century, who
composed the Æneid, represented St. Peter under the allegorical
character of the Trojan hero. * Note: It is quite clear that,
strictly speaking, the church of Rome was not founded by either
of these apostles. St. Paul’s Epistle to the Romans proves
undeniably the flourishing state of the church before his visit
to the city; and many Roman Catholic writers have given up the
impracticable task of reconciling with chronology any visit of
St. Peter to Rome before the end of the reign of Claudius, or the
beginning of that of Nero.—M.}

\pagenote[123]{It is in French only that the famous allusion to
St. Peter’s name is exact. Tu es Pierre, et sur cette pierre.—The
same is imperfect in Greek, Latin, Italian, \&c., and totally
unintelligible in our Tentonic languages. * Note: It is exact in
Syro-Chaldaic, the language in which it was spoken by Jesus
Christ. (St. Matt. xvi. 17.) Peter was called Cephas; and cepha
signifies base, foundation, rock—G.}

\pagenote[124]{Irenæus adv. Hæreses, iii. 3. Tertullian de
Præscription. c. 36, and Cyprian, Epistol. 27, 55, 71, 75. Le
Clere (Hist. Eccles. p. 764) and Mosheim (p. 258, 578) labor in
the interpretation of these passages. But the loose and
rhetorical style of the fathers often appears favorable to the
pretensions of Rome.}

\pagenote[125]{See the sharp epistle from Firmilianus, bishop of
Cæsarea, to Stephen, bishop of Rome, ap. Cyprian, Epistol. 75.}

\pagenote[126]{Concerning this dispute of the rebaptism of
heretics, see the epistles of Cyprian, and the seventh book of
Eusebius.}

The progress of the ecclesiastical authority gave birth to the
memorable distinction of the laity and of the clergy, which had
been unknown to the Greeks and Romans.\textsuperscript{127} The former of these
appellations comprehended the body of the Christian people; the
latter, according to the signification of the word, was
appropriated to the chosen portion that had been set apart for
the service of religion; a celebrated order of men, which has
furnished the most important, though not always the most
edifying, subjects for modern history. Their mutual hostilities
sometimes disturbed the peace of the infant church, but their
zeal and activity were united in the common cause, and the love
of power, which (under the most artful disguises) could insinuate
itself into the breasts of bishops and martyrs, animated them to
increase the number of their subjects, and to enlarge the limits
of the Christian empire. They were destitute of any temporal
force, and they were for a long time discouraged and oppressed,
rather than assisted, by the civil magistrate; but they had
acquired, and they employed within their own society, the two
most efficacious instruments of government, rewards and
punishments; the former derived from the pious liberality, the
latter from the devout apprehensions, of the faithful.

\pagenote[127]{For the origin of these words, see Mosheim, p.
141. Spanheim, Hist. Ecclesiast. p. 633. The distinction of
Clerus and Iaicus was established before the time of Tertullian.}

\section{Part \thesection.}

I. The community of goods, which had so agreeably amused the
imagination of Plato,\textsuperscript{128} and which subsisted in some degree
among the austere sect of the Essenians,\textsuperscript{129} was adopted for a
short time in the primitive church. The fervor of the first
proselytes prompted them to sell those worldly possessions, which
they despised, to lay the price of them at the feet of the
apostles, and to content themselves with receiving an equal share
out of the general distribution.\textsuperscript{130} The progress of the
Christian religion relaxed, and gradually abolished, this
generous institution, which, in hands less pure than those of the
apostles, would too soon have been corrupted and abused by the
returning selfishness of human nature; and the converts who
embraced the new religion were permitted to retain the possession
of their patrimony, to receive legacies and inheritances, and to
increase their separate property by all the lawful means of trade
and industry. Instead of an absolute sacrifice, a moderate
proportion was accepted by the ministers of the gospel; and in
their weekly or monthly assemblies, every believer, according to
the exigency of the occasion, and the measure of his wealth and
piety, presented his voluntary offering for the use of the common
fund.\textsuperscript{131} Nothing, however inconsiderable, was refused; but it
was diligently inculcated that, in the article of Tithes, the
Mosaic law was still of divine obligation; and that since the
Jews, under a less perfect discipline, had been commanded to pay
a tenth part of all that they possessed, it would become the
disciples of Christ to distinguish themselves by a superior
degree of liberality,\textsuperscript{132} and to acquire some merit by resigning
a superfluous treasure, which must so soon be annihilated with
the world itself.\textsuperscript{133} It is almost unnecessary to observe, that
the revenue of each particular church, which was of so uncertain
and fluctuating a nature, must have varied with the poverty or
the opulence of the faithful, as they were dispersed in obscure
villages, or collected in the great cities of the empire. In the
time of the emperor Decius, it was the opinion of the
magistrates, that the Christians of Rome were possessed of very
considerable wealth; that vessels of gold and silver were used in
their religious worship, and that many among their proselytes had
sold their lands and houses to increase the public riches of the
sect, at the expense, indeed, of their unfortunate children, who
found themselves beggars, because their parents had been saints. \textsuperscript{134}
We should listen with distrust to the suspicions of strangers
and enemies: on this occasion, however, they receive a very
specious and probable color from the two following circumstances,
the only ones that have reached our knowledge, which define any
precise sums, or convey any distinct idea. Almost at the same
period, the bishop of Carthage, from a society less opulent than
that of Rome, collected a hundred thousand sesterces, (above
eight hundred and fifty pounds sterling,) on a sudden call of
charity to redeem the brethren of Numidia, who had been carried
away captives by the barbarians of the desert.\textsuperscript{135} About a
hundred years before the reign of Decius, the Roman church had
received, in a single donation, the sum of two hundred thousand
sesterces from a stranger of Pontus, who proposed to fix his
residence in the capital.\textsuperscript{136} These oblations, for the most part,
were made in money; nor was the society of Christians either
desirous or capable of acquiring, to any considerable degree, the
encumbrance of landed property. It had been provided by several
laws, which were enacted with the same design as our statutes of
mortmain, that no real estates should be given or bequeathed to
any corporate body, without either a special privilege or a
particular dispensation from the emperor or from the senate;\textsuperscript{137}
who were seldom disposed to grant them in favor of a sect, at
first the object of their contempt, and at last of their fears
and jealousy. A transaction, however, is related under the reign
of Alexander Severus, which discovers that the restraint was
sometimes eluded or suspended, and that the Christians were
permitted to claim and to possess lands within the limits of Rome
itself.\textsuperscript{138} The progress of Christianity, and the civil confusion
of the empire, contributed to relax the severity of the laws; and
before the close of the third century many considerable estates
were bestowed on the opulent churches of Rome, Milan, Carthage,
Antioch, Alexandria, and the other great cities of Italy and the
provinces.

\pagenote[128]{The community instituted by Plato is more perfect
than that which Sir Thomas More had imagined for his Utopia. The
community of women, and that of temporal goods, may be considered
as inseparable parts of the same system.}

\pagenote[129]{Joseph. Antiquitat. xviii. 2. Philo, de Vit.
Contemplativ.}

\pagenote[130]{See the Acts of the Apostles, c. 2, 4, 5, with
Grotius’s Commentary. Mosheim, in a particular dissertation,
attacks the common opinion with very inconclusive arguments. *
Note: This is not the general judgment on Mosheim’s learned
dissertation. There is no trace in the latter part of the New
Testament of this community of goods, and many distinct proofs of
the contrary. All exhortations to almsgiving would have been
unmeaning if property had been in common—M.}

\pagenote[131]{Justin Martyr, Apolog. Major, c. 89. Tertullian,
Apolog. c. 39.}

\pagenote[132]{Irenæus ad Hæres. l. iv. c. 27, 34. Origen in Num.
Hom. ii Cyprian de Unitat. Eccles. Constitut. Apostol. l. ii. c.
34, 35, with the notes of Cotelerius. The Constitutions introduce
this divine precept, by declaring that priests are as much above
kings as the soul is above the body. Among the tithable articles,
they enumerate corn, wine, oil, and wool. On this interesting
subject, consult Prideaux’s History of Tithes, and Fra Paolo
delle Materie Beneficiarie; two writers of a very different
character.}

\pagenote[133]{The same opinion which prevailed about the year
one thousand, was productive of the same effects. Most of the
Donations express their motive, “appropinquante mundi fine.” See
Mosheim’s General History of the Church, vol. i. p. 457.}

\pagenote[134]{Tum summa cura est fratribus (Ut sermo testatur
loquax.) Offerre, fundis venditis Sestertiorum millia. Addicta
avorum prædia Foedis sub auctionibus, Successor exheres gemit
Sanctis egens Parentibus. Hæc occuluntur abditis Ecclesiarum in
angulis. Et summa pietas creditur Nudare dulces
liberos.——Prudent. Hymn 2. The subsequent conduct of the deacon
Laurence only proves how proper a use was made of the wealth of
the Roman church; it was undoubtedly very considerable; but Fra
Paolo (c. 3) appears to exaggerate, when he supposes that the
successors of Commodus were urged to persecute the Christians by
their own avarice, or that of their Prætorian præfects.}

\pagenote[135]{Cyprian, Epistol. 62.}

\pagenote[136]{Tertullian de Præscriptione, c. 30.}

\pagenote[137]{Diocletian gave a rescript, which is only a
declaration of the old law; “Collegium, si nullo speciali
privilegio subnixum sit, hæreditatem capere non posse, dubium non
est.” Fra Paolo (c. 4) thinks that these regulations had been
much neglected since the reign of Valerian.}

\pagenote[138]{Hist. August. p. 131. The ground had been public;
and was row disputed between the society of Christians and that
of butchers. Note *: Carponarii, rather victuallers.—M.}

The bishop was the natural steward of the church; the public
stock was intrusted to his care without account or control; the
presbyters were confined to their spiritual functions, and the
more dependent order of the deacons was solely employed in the
management and distribution of the ecclesiastical revenue.\textsuperscript{139} If
we may give credit to the vehement declamations of Cyprian, there
were too many among his African brethren, who, in the execution
of their charge, violated every precept, not only of evangelical
perfection, but even of moral virtue. By some of these unfaithful
stewards the riches of the church were lavished in sensual
pleasures; by others they were perverted to the purposes of
private gain, of fraudulent purchases, and of rapacious usury. \textsuperscript{140}
But as long as the contributions of the Christian people were
free and unconstrained, the abuse of their confidence could not
be very frequent, and the general uses to which their liberality
was applied reflected honor on the religious society. A decent
portion was reserved for the maintenance of the bishop and his
clergy; a sufficient sum was allotted for the expenses of the
public worship, of which the feasts of love, the \textit{agapæ}, as they
were called, constituted a very pleasing part. The whole
remainder was the sacred patrimony of the poor. According to the
discretion of the bishop, it was distributed to support widows
and orphans, the lame, the sick, and the aged of the community;
to comfort strangers and pilgrims, and to alleviate the
misfortunes of prisoners and captives, more especially when their
sufferings had been occasioned by their firm attachment to the
cause of religion.\textsuperscript{141} A generous intercourse of charity united
the most distant provinces, and the smaller congregations were
cheerfully assisted by the alms of their more opulent brethren. \textsuperscript{142}
Such an institution, which paid less regard to the merit than
to the distress of the object, very materially conduced to the
progress of Christianity. The Pagans, who were actuated by a
sense of humanity, while they derided the doctrines, acknowledged
the benevolence, of the new sect.\textsuperscript{143} The prospect of immediate
relief and of future protection allured into its hospitable bosom
many of those unhappy persons whom the neglect of the world would
have abandoned to the miseries of want, of sickness, and of old
age. There is some reason likewise to believe that great numbers
of infants, who, according to the inhuman practice of the times,
had been exposed by their parents, were frequently rescued from
death, baptized, educated, and maintained by the piety of the
Christians, and at the expense of the public treasure.\textsuperscript{144}

\pagenote[139]{Constitut. Apostol. ii. 35.}

\pagenote[140]{Cyprian de Lapsis, p. 89. Epistol. 65. The charge
is confirmed by the 19th and 20th canon of the council of
Illiberis.}

\pagenote[141]{See the apologies of Justin, Tertullian, \&c.}

\pagenote[142]{The wealth and liberality of the Romans to their
most distant brethren is gratefully celebrated by Dionysius of
Corinth, ap. Euseb. l. iv. c. 23.}

\pagenote[143]{See Lucian iu Peregrin. Julian (Epist. 49) seems
mortified that the Christian charity maintains not only their
own, but likewise the heathen poor.}

\pagenote[144]{Such, at least, has been the laudable conduct of
more modern missionaries, under the same circumstances. Above
three thousand new-born infants are annually exposed in the
streets of Pekin. See Le Comte, Memoires sur la Chine, and the
Recherches sur les Chinois et les Egyptians, tom. i. p. 61.}

II. It is the undoubted right of every society to exclude from
its communion and benefits such among its members as reject or
violate those regulations which have been established by general
consent. In the exercise of this power, the censures of the
Christian church were chiefly directed against scandalous
sinners, and particularly those who were guilty of murder, of
fraud, or of incontinence; against the authors or the followers
of any heretical opinions which had been condemned by the
judgment of the episcopal order; and against those unhappy
persons, who, whether from choice or compulsion, had polluted
themselves after their baptism by any act of idolatrous worship.
The consequences of excommunication were of a temporal as well as
a spiritual nature. The Christian against whom it was pronounced
was deprived of any part in the oblations of the faithful. The
ties both of religious and of private friendship were dissolved:
he found himself a profane object of abhorrence to the persons
whom he the most esteemed, or by whom he had been the most
tenderly beloved; and as far as an expulsion from a respectable
society could imprint on his character a mark of disgrace, he was
shunned or suspected by the generality of mankind. The situation
of these unfortunate exiles was in itself very painful and
melancholy; but, as it usually happens, their apprehensions far
exceeded their sufferings. The benefits of the Christian
communion were those of eternal life; nor could they erase from
their minds the awful opinion, that to those ecclesiastical
governors by whom they were condemned, the Deity had committed
the keys of Hell and of Paradise. The heretics, indeed, who might
be supported by the consciousness of their intentions, and by the
flattering hope that they alone had discovered the true path of
salvation, endeavored to regain, in their separate assemblies,
those comforts, temporal as well as spiritual, which they no
longer derived from the great society of Christians. But almost
all those who had reluctantly yielded to the power of vice or
idolatry were sensible of their fallen condition, and anxiously
desirous of being restored to the benefits of the Christian
communion.

With regard to the treatment of these penitents, two opposite
opinions, the one of justice, the other of mercy, divided the
primitive church. The more rigid and inflexible casuists refused
them forever, and without exception, the meanest place in the
holy community, which they had disgraced or deserted; and leaving
them to the remorse of a guilty conscience, indulged them only
with a faint ray of hope that the contrition of their life and
death might possibly be accepted by the Supreme Being.\textsuperscript{145} A
milder sentiment was embraced, in practice as well as in theory,
by the purest and most respectable of the Christian churches.\textsuperscript{146}
The gates of reconciliation and of heaven were seldom shut
against the returning penitent; but a severe and solemn form of
discipline was instituted, which, while it served to expiate his
crime, might powerfully deter the spectators from the imitation
of his example. Humbled by a public confession, emaciated by
fasting and clothed in sackcloth, the penitent lay prostrate at
the door of the assembly, imploring with tears the pardon of his
offences, and soliciting the prayers of the faithful.\textsuperscript{147} If the
fault was of a very heinous nature, whole years of penance were
esteemed an inadequate satisfaction to the divine justice; and it
was always by slow and painful gradations that the sinner, the
heretic, or the apostate, was readmitted into the bosom of the
church. A sentence of perpetual excommunication was, however,
reserved for some crimes of an extraordinary magnitude, and
particularly for the inexcusable relapses of those penitents who
had already experienced and abused the clemency of their
ecclesiastical superiors. According to the circumstances or the
number of the guilty, the exercise of the Christian discipline
was varied by the discretion of the bishops. The councils of
Ancyra and Illiberis were held about the same time, the one in
Galatia, the other in Spain; but their respective canons, which
are still extant, seem to breathe a very different spirit. The
Galatian, who after his baptism had repeatedly sacrificed to
idols, might obtain his pardon by a penance of seven years; and
if he had seduced others to imitate his example, only three years
more were added to the term of his exile. But the unhappy
Spaniard, who had committed the same offence, was deprived of the
hope of reconciliation, even in the article of death; and his
idolatry was placed at the head of a list of seventeen other
crimes, against which a sentence no less terrible was pronounced.
Among these we may distinguish the inexpiable guilt of
calumniating a bishop, a presbyter, or even a deacon.\textsuperscript{148}

\pagenote[145]{The Montanists and the Novatians, who adhered to
this opinion with the greatest rigor and obstinacy, found
themselves at last in the number of excommunicated heretics. See
the learned and copious Mosheim, Secul. ii. and iii.}

\pagenote[146]{Dionysius ap. Euseb. iv. 23. Cyprian, de Lapsis.}

\pagenote[147]{Cave’s Primitive Christianity, part iii. c. 5. The
admirers of antiquity regret the loss of this public penance.}

\pagenote[148]{See in Dupin, Bibliotheque Ecclesiastique, tom.
ii. p. 304—313, a short but rational exposition of the canons of
those councils, which were assembled in the first moments of
tranquillity, after the persecution of Diocletian. This
persecution had been much less severely felt in Spain than in
Galatia; a difference which may, in some measure account for the
contrast of their regulations.}

The well-tempered mixture of liberality and rigor, the judicious
dispensation of rewards and punishments, according to the maxims
of policy as well as justice, constituted the \textit{human} strength of
the church. The Bishops, whose paternal care extended itself to
the government of both worlds, were sensible of the importance of
these prerogatives; and covering their ambition with the fair
pretence of the love of order, they were jealous of any rival in
the exercise of a discipline so necessary to prevent the
desertion of those troops which had enlisted themselves under the
banner of the cross, and whose numbers every day became more
considerable. From the imperious declamations of Cyprian, we
should naturally conclude that the doctrines of excommunication
and penance formed the most essential part of religion; and that
it was much less dangerous for the disciples of Christ to neglect
the observance of the moral duties, than to despise the censures
and authority of their bishops. Sometimes we might imagine that
we were listening to the voice of Moses, when he commanded the
earth to open, and to swallow up, in consuming flames, the
rebellious race which refused obedience to the priesthood of
Aaron; and we should sometimes suppose that we heard a Roman
consul asserting the majesty of the republic, and declaring his
inflexible resolution to enforce the rigor of the laws. “If such
irregularities are suffered with impunity,” (it is thus that the
bishop of Carthage chides the lenity of his colleague,) “if such
irregularities are suffered, there is an end of EPISCOPAL VIGOR; \textsuperscript{149}
an end of the sublime and divine power of governing the
Church, an end of Christianity itself.” Cyprian had renounced
those temporal honors which it is probable he would never have
obtained; but the acquisition of such absolute command over the
consciences and understanding of a congregation, however obscure
or despised by the world, is more truly grateful to the pride of
the human heart than the possession of the most despotic power,
imposed by arms and conquest on a reluctant people.

\pagenote[1491]{Gibbon has been accused of injustice to the
character of Cyprian, as exalting the “censures and authority of
the church above the observance of the moral duties.”
Felicissimus had been condemned by a synod of bishops, (non
tantum mea, sed plurimorum coepiscorum, sententia condemnatum,)
on the charge not only of schism, but of embezzlement of public
money, the debauching of virgins, and frequent acts of adultery.
His violent menaces had extorted his readmission into the church,
against which Cyprian protests with much vehemence: ne pecuniæ
commissæ sibi fraudator, ne stuprator virginum, ne matrimoniorum
multorum depopulator et corruptor, ultra adhuc sponsam Christi
incorruptam præsentiæ suæ dedecore, et impudica atque incesta
contagione, violaret. See Chelsum’s Remarks, p. 134. If these
charges against Felicissimus were true, they were something more
than “irregularities,” A Roman censor would have been a fairer
subject of comparison than a consul. On the other hand, it must
be admitted that the charge of adultery deepens very rapidly as
the controversy becomes more violent. It is first represented as
a single act, recently detected, and which men of character were
prepared to substantiate: adulterii etiam crimen accedit. quod
patres nostri graves viri deprehendisse se nuntiaverunt, et
probaturos se asseverarunt. Epist. xxxviii. The heretic has now
darkened into a man of notorious and general profligacy. Nor can
it be denied that of the whole long epistle, very far the larger
and the more passionate part dwells on the breach of
ecclesiastical unity rather than on the violation of Christian
holiness.—M.}

\pagenote[149]{Cyprian Epist. 69.}

\pagenote[1492]{This supposition appears unfounded: the birth and
the talents of Cyprian might make us presume the contrary.
Thascius Cæcilius Cyprianus, Carthaginensis, artis oratoriæ
professione clarus, magnam sibi gloriam, opes, honores
acquisivit, epularibus cænis et largis dapibus assuetus, pretiosa
veste conspicuus, auro atque purpura fulgens, fascibus oblectatus
et honoribus, stipatus clientium cuneis, frequentiore comitatu
officii agminis honestatus, ut ipse de se loquitur in Epistola ad
Donatum. See De Cave, Hist. Liter. b. i. p. 87.—G. Cave has
rather embellished Cyprian’s language.—M.}

In the course of this important, though perhaps tedious inquiry,
I have attempted to display the secondary causes which so
efficaciously assisted the truth of the Christian religion. If
among these causes we have discovered any artificial ornaments,
any accidental circumstances, or any mixture of error and
passion, it cannot appear surprising that mankind should be the
most sensibly affected by such motives as were suited to their
imperfect nature. It was by the aid of these causes, exclusive
zeal, the immediate expectation of another world, the claim of
miracles, the practice of rigid virtue, and the constitution of
the primitive church, that Christianity spread itself with so
much success in the Roman empire. To the first of these the
Christians were indebted for their invincible valor, which
disdained to capitulate with the enemy whom they were resolved to
vanquish. The three succeeding causes supplied their valor with
the most formidable arms. The last of these causes united their
courage, directed their arms, and gave their efforts that
irresistible weight, which even a small band of well-trained and
intrepid volunteers has so often possessed over an undisciplined
multitude, ignorant of the subject and careless of the event of
the war. In the various religions of Polytheism, some wandering
fanatics of Egypt and Syria, who addressed themselves to the
credulous superstition of the populace, were perhaps the only
order of priests\textsuperscript{150} that derived their whole support and credit
from their sacerdotal profession, and were very deeply affected
by a personal concern for the safety or prosperity of their
tutelar deities. The ministers of Polytheism, both in Rome and in
the provinces, were, for the most part, men of a noble birth, and
of an affluent fortune, who received, as an honorable
distinction, the care of a celebrated temple, or of a public
sacrifice, exhibited, very frequently at their own expense, the
sacred games,\textsuperscript{151} and with cold indifference performed the
ancient rites, according to the laws and fashion of their
country. As they were engaged in the ordinary occupations of
life, their zeal and devotion were seldom animated by a sense of
interest, or by the habits of an ecclesiastical character.
Confined to their respective temples and cities, they remained
without any connection of discipline or government; and whilst
they acknowledged the supreme jurisdiction of the senate, of the
college of pontiffs, and of the emperor, those civil magistrates
contented themselves with the easy task of maintaining in peace
and dignity the general worship of mankind. We have already seen
how various, how loose, and how uncertain were the religious
sentiments of Polytheists. They were abandoned, almost without
control, to the natural workings of a superstitious fancy. The
accidental circumstances of their life and situation determined
the object as well as the degree of their devotion; and as long
as their adoration was successively prostituted to a thousand
deities, it was scarcely possible that their hearts could be
susceptible of a very sincere or lively passion for any of them.

\pagenote[150]{The arts, the manners, and the vices of the
priests of the Syrian goddess are very humorously described by
Apuleius, in the eighth book of his Metamorphosis.}

\pagenote[151]{The office of Asiarch was of this nature, and it
is frequently mentioned in Aristides, the Inscriptions, \&c. It
was annual and elective. None but the vainest citizens could
desire the honor; none but the most wealthy could support the
expense. See, in the Patres Apostol. tom. ii. p. 200, with how
much indifference Philip the Asiarch conducted himself in the
martyrdom of Polycarp. There were likewise Bithyniarchs,
Lyciarchs, \&c.}

When Christianity appeared in the world, even these faint and
imperfect impressions had lost much of their original power.
Human reason, which by its unassisted strength is incapable of
perceiving the mysteries of faith, had already obtained an easy
triumph over the folly of Paganism; and when Tertullian or
Lactantius employ their labors in exposing its falsehood and
extravagance, they are obliged to transcribe the eloquence of
Cicero or the wit of Lucian. The contagion of these sceptical
writings had been diffused far beyond the number of their
readers. The fashion of incredulity was communicated from the
philosopher to the man of pleasure or business, from the noble to
the plebeian, and from the master to the menial slave who waited
at his table, and who eagerly listened to the freedom of his
conversation. On public occasions the philosophic part of mankind
affected to treat with respect and decency the religious
institutions of their country; but their secret contempt
penetrated through the thin and awkward disguise; and even the
people, when they discovered that their deities were rejected and
derided by those whose rank or understanding they were accustomed
to reverence, were filled with doubts and apprehensions
concerning the truth of those doctrines, to which they had
yielded the most implicit belief. The decline of ancient
prejudice exposed a very numerous portion of human kind to the
danger of a painful and comfortless situation. A state of
scepticism and suspense may amuse a few inquisitive minds. But
the practice of superstition is so congenial to the multitude,
that if they are forcibly awakened, they still regret the loss of
their pleasing vision. Their love of the marvellous and
supernatural, their curiosity with regard to future events, and
their strong propensity to extend their hopes and fears beyond
the limits of the visible world, were the principal causes which
favored the establishment of Polytheism. So urgent on the vulgar
is the necessity of believing, that the fall of any system of
mythology will most probably be succeeded by the introduction of
some other mode of superstition. Some deities of a more recent
and fashionable cast might soon have occupied the deserted
temples of Jupiter and Apollo, if, in the decisive moment, the
wisdom of Providence had not interposed a genuine revelation,
fitted to inspire the most rational esteem and conviction,
whilst, at the same time, it was adorned with all that could
attract the curiosity, the wonder, and the veneration of the
people. In their actual disposition, as many were almost
disengaged from their artificial prejudices, but equally
susceptible and desirous of a devout attachment; an object much
less deserving would have been sufficient to fill the vacant
place in their hearts, and to gratify the uncertain eagerness of
their passions. Those who are inclined to pursue this reflection,
instead of viewing with astonishment the rapid progress of
Christianity, will perhaps be surprised that its success was not
still more rapid and still more universal. It has been observed,
with truth as well as propriety, that the conquests of Rome
prepared and facilitated those of Christianity. In the second
chapter of this work we have attempted to explain in what manner
the most civilized provinces of Europe, Asia, and Africa were
united under the dominion of one sovereign, and gradually
connected by the most intimate ties of laws, of manners, and of
language. The Jews of Palestine, who had fondly expected a
temporal deliverer, gave so cold a reception to the miracles of
the divine prophet, that it was found unnecessary to publish, or
at least to preserve, any Hebrew gospel.\textsuperscript{152} The authentic
histories of the actions of Christ were composed in the Greek
language, at a considerable distance from Jerusalem, and after
the Gentile converts were grown extremely numerous.\textsuperscript{153} As soon
as those histories were translated into the Latin tongue, they
were perfectly intelligible to all the subjects of Rome,
excepting only to the peasants of Syria and Egypt, for whose
benefit particular versions were afterwards made. The public
highways, which had been constructed for the use of the legions,
opened an easy passage for the Christian missionaries from
Damascus to Corinth, and from Italy to the extremity of Spain or
Britain; nor did those spiritual conquerors encounter any of the
obstacles which usually retard or prevent the introduction of a
foreign religion into a distant country. There is the strongest
reason to believe, that before the reigns of Diocletian and
Constantine, the faith of Christ had been preached in every
province, and in all the great cities of the empire; but the
foundation of the several congregations, the numbers of the
faithful who composed them, and their proportion to the
unbelieving multitude, are now buried in obscurity, or disguised
by fiction and declamation. Such imperfect circumstances,
however, as have reached our knowledge concerning the increase of
the Christian name in Asia and Greece, in Egypt, in Italy, and in
the West, we shall now proceed to relate, without neglecting the
real or imaginary acquisitions which lay beyond the frontiers of
the Roman empire.

\pagenote[152]{The modern critics are not disposed to believe
what the fathers almost unanimously assert, that St. Matthew
composed a Hebrew gospel, of which only the Greek translation is
extant. It seems, however, dangerous to reject their testimony. *
Note: Strong reasons appear to confirm this testimony. Papias,
contemporary of the Apostle St. John, says positively that
Matthew had written the discourses of Jesus Christ in Hebrew, and
that each interpreted them as he could. This Hebrew was the
Syro-Chaldaic dialect, then in use at Jerusalem: Origen, Irenæus,
Eusebius, Jerome, Epiphanius, confirm this statement. Jesus
Christ preached himself in Syro-Chaldaic, as is proved by many
words which he used, and which the Evangelists have taken the
pains to translate. St. Paul, addressing the Jews, used the same
language: Acts xxi. 40, xxii. 2, xxvi. 14. The opinions of some
critics prove nothing against such undeniable testimonies.
Moreover, their principal objection is, that St. Matthew quotes
the Old Testament according to the Greek version of the LXX.,
which is inaccurate; for of ten quotations, found in his Gospel,
seven are evidently taken from the Hebrew text; the threo others
offer little that differ: moreover, the latter are not literal
quotations. St. Jerome says positively, that, according to a copy
which he had seen in the library of Cæsarea, the quotations were
made in Hebrew (in Catal.) More modern critics, among others
Michaelis, do not entertain a doubt on the subject. The Greek
version appears to have been made in the time of the apostles, as
St. Jerome and St. Augustus affirm, perhaps by one of them.—G.
——Among modern critics, Dr. Hug has asserted the Greek original
of St. Matthew, but the general opinion of the most learned
biblical writer, supports the view of M. Guizot.—M.}

\pagenote[153]{Under the reigns of Nero and Domitian, and in the
cities of Alexandria, Antioch, Rome, and Ephesus. See Mill.
Prolegomena ad Nov. Testament, and Dr. Lardner’s fair and
extensive collection, vol. xv. Note: This question has, it is
well known, been most elaborately discussed since the time of
Gibbon. The Preface to the Translation of Schleier Macher’s
Version of St. Luke contains a very able summary of the various
theories.—M.}

\section{Part \thesection.}

The rich provinces that extend from the Euphrates to the Ionian
Sea were the principal theatre on which the apostle of the
Gentiles displayed his zeal and piety. The seeds of the gospel,
which he had scattered in a fertile soil, were diligently
cultivated by his disciples; and it should seem that, during the
two first centuries, the most considerable body of Christians was
contained within those limits. Among the societies which were
instituted in Syria, none were more ancient or more illustrious
than those of Damascus, of Berea or Aleppo, and of Antioch. The
prophetic introduction of the Apocalypse has described and
immortalized the seven churches of Asia; Ephesus, Smyrna,
Pergamus, Thyatira,\textsuperscript{154} Sardes, Laodicea, and Philadelphia; and
their colonies were soon diffused over that populous country. In
a very early period, the islands of Cyprus and Crete, the
provinces of Thrace and Macedonia, gave a favorable reception to
the new religion; and Christian republics were soon founded in
the cities of Corinth, of Sparta, and of Athens.\textsuperscript{155} The
antiquity of the Greek and Asiatic churches allowed a sufficient
space of time for their increase and multiplication; and even the
swarms of Gnostics and other heretics serve to display the
flourishing condition of the orthodox church, since the
appellation of heretics has always been applied to the less
numerous party. To these domestic testimonies we may add the
confession, the complaints, and the apprehensions of the Gentiles
themselves. From the writings of Lucian, a philosopher who had
studied mankind, and who describes their manners in the most
lively colors, we may learn that, under the reign of Commodus,
his native country of Pontus was filled with Epicureans and
\textit{Christians}.\textsuperscript{156} Within fourscore years after the death of
Christ,\textsuperscript{157} the humane Pliny laments the magnitude of the evil
which he vainly attempted to eradicate. In his very curious
epistle to the emperor Trajan, he affirms that the temples were
almost deserted, that the sacred victims scarcely found any
purchasers, and that the superstition had not only infected the
cities, but had even spread itself into the villages and the open
country of Pontus and Bithynia.\textsuperscript{158}

\pagenote[154]{The Alogians (Epiphanius de Hæres. 51) disputed
the genuineness of the Apocalypse, because the church of Thyatira
was not yet founded. Epiphanius, who allows the fact, extricates
himself from the difficulty by ingeniously supposing that St.
John wrote in the spirit of prophecy. See Abauzit, Discours sur
l’Apocalypse.}

\pagenote[155]{The epistles of Ignatius and Dionysius (ap. Euseb.
iv. 23) point out many churches in Asia and Greece. That of
Athens seems to have been one of the least flourishing.}

\pagenote[156]{Lucian in Alexandro, c. 25. Christianity however,
must have been very unequally diffused over Pontus; since, in the
middle of the third century, there was no more than seventeen
believers in the extensive diocese of Neo-Cæsarea. See M. de
Tillemont, Memoires Ecclesiast. tom. iv. p. 675, from Basil and
Gregory of Nyssa, who were themselves natives of Cappadocia.
Note: Gibbon forgot the conclusion of this story, that Gregory
left only seventeen heathens in his diocese. The antithesis is
suspicious, and both numbers may have been chosen to magnify the
spiritual fame of the wonder-worker.—M.}

\pagenote[157]{According to the ancients, Jesus Christ suffered
under the consulship of the two Gemini, in the year 29 of our
present æra. Pliny was sent into Bithynia (according to Pagi) in
the year 110.}

\pagenote[158]{Plin. Epist. x. 97.}

Without descending into a minute scrutiny of the expressions or
of the motives of those writers who either celebrate or lament
the progress of Christianity in the East, it may in general be
observed that none of them have left us any grounds from whence a
just estimate might be formed of the real numbers of the faithful
in those provinces. One circumstance, however, has been
fortunately preserved, which seems to cast a more distinct light
on this obscure but interesting subject. Under the reign of
Theodosius, after Christianity had enjoyed, during more than
sixty years, the sunshine of Imperial favor, the ancient and
illustrious church of Antioch consisted of one hundred thousand
persons, three thousand of whom were supported out of the public
oblations.\textsuperscript{159} The splendor and dignity of the queen of the East,
the acknowledged populousness of Cæsarea, Seleucia, and
Alexandria, and the destruction of two hundred and fifty thousand
souls in the earthquake which afflicted Antioch under the elder
Justin,\textsuperscript{160} are so many convincing proofs that the whole number
of its inhabitants was not less than half a million, and that the
Christians, however multiplied by zeal and power, did not exceed
a fifth part of that great city. How different a proportion must
we adopt when we compare the persecuted with the triumphant
church, the West with the East, remote villages with populous
towns, and countries recently converted to the faith with the
place where the believers first received the appellation of
Christians! It must not, however, be dissembled, that, in another
passage, Chrysostom, to whom we are indebted for this useful
information, computes the multitude of the faithful as even
superior to that of the Jews and Pagans.\textsuperscript{161} But the solution of
this apparent difficulty is easy and obvious. The eloquent
preacher draws a parallel between the civil and the
ecclesiastical constitution of Antioch; between the list of
Christians who had acquired heaven by baptism, and the list of
citizens who had a right to share the public liberality. Slaves,
strangers, and infants were comprised in the former; they were
excluded from the latter.

\pagenote[159]{Chrysostom. Opera, tom. vii. p. 658, 810, (edit.
Savil. ii. 422, 329.)}

\pagenote[160]{John Malala, tom. ii. p. 144. He draws the same
conclusion with regard to the populousness of antioch.}

\pagenote[161]{Chrysostom. tom. i. p. 592. I am indebted for
these passages, though not for my inference, to the learned Dr.
Lardner. Credibility of the Gospel of History, vol. xii. p. 370.
* Note: The statements of Chrysostom with regard to the
population of Antioch, whatever may be their accuracy, are
perfectly consistent. In one passage he reckons the population at
200,000. In a second the Christians at 100,000. In a third he
states that the Christians formed more than half the population.
Gibbon has neglected to notice the first passage, and has drawn
by estimate of the population of Antioch from other sources. The
8000 maintained by alms were widows and virgins alone—M.}

The extensive commerce of Alexandria, and its proximity to
Palestine, gave an easy entrance to the new religion. It was at
first embraced by great numbers of the Theraputæ, or Essenians,
of the Lake Mareotis, a Jewish sect which had abated much of its
reverence for the Mosaic ceremonies. The austere life of the
Essenians, their fasts and excommunications, the community of
goods, the love of celibacy, their zeal for martyrdom, and the
warmth though not the purity of their faith, already offered a
very lively image of the primitive discipline.\textsuperscript{162} It was in the
school of Alexandria that the Christian theology appears to have
assumed a regular and scientific form; and when Hadrian visited
Egypt, he found a church composed of Jews and of Greeks,
sufficiently important to attract the notice of that inquisitive
prince.\textsuperscript{163} But the progress of Christianity was for a long time
confined within the limits of a single city, which was itself a
foreign colony, and till the close of the second century the
predecessors of Demetrius were the only prelates of the Egyptian
church. Three bishops were consecrated by the hands of Demetrius,
and the number was increased to twenty by his successor Heraclas. \textsuperscript{164}
The body of the natives, a people distinguished by a sullen
inflexibility of temper,\textsuperscript{165} entertained the new doctrine with
coldness and reluctance; and even in the time of Origen, it was
rare to meet with an Egyptian who had surmounted his early
prejudices in favor of the sacred animals of his country.\textsuperscript{166} As
soon, indeed, as Christianity ascended the throne, the zeal of
those barbarians obeyed the prevailing impulsion; the cities of
Egypt were filled with bishops, and the deserts of Thebais
swarmed with hermits.

\pagenote[162]{Basnage, Histoire des Juifs, l. 2, c. 20, 21, 22,
23, has examined with the most critical accuracy the curious
treatise of Philo, which describes the Therapeutæ. By proving
that it was composed as early as the time of Augustus, Basnage
has demonstrated, in spite of Eusebius (l. ii. c. 17) and a crowd
of modern Catholics, that the Therapeutæ were neither Christians
nor monks. It still remains probable that they changed their
name, preserved their manners, adopted some new articles of
faith, and gradually became the fathers of the Egyptian
Ascetics.}

\pagenote[163]{See a letter of Hadrian in the Augustan History,
p. 245.}

\pagenote[164]{For the succession of Alexandrian bishops, consult
Renaudot’s History, p. 24, \&c. This curious fact is preserved by
the patriarch Eutychius, (Annal. tom. i. p. 334, Vers. Pocock,)
and its internal evidence would alone be a sufficient answer to
all the objections which Bishop Pearson has urged in the Vindiciæ
Ignatianæ.}

\pagenote[165]{Ammian. Marcellin. xxii. 16.}

\pagenote[166]{Origen contra Celsum, l. i. p. 40.}

A perpetual stream of strangers and provincials flowed into the
capacious bosom of Rome. Whatever was strange or odious, whoever
was guilty or suspected, might hope, in the obscurity of that
immense capital, to elude the vigilance of the law. In such a
various conflux of nations, every teacher, either of truth or
falsehood, every founder, whether of a virtuous or a criminal
association, might easily multiply his disciples or accomplices.
The Christians of Rome, at the time of the accidental persecution
of Nero, are represented by Tacitus as already amounting to a
very great multitude,\textsuperscript{167} and the language of that great
historian is almost similar to the style employed by Livy, when
he relates the introduction and the suppression of the rites of
Bacchus. After the Bacchanals had awakened the severity of the
senate, it was likewise apprehended that a very great multitude,
as it were \textit{another people}, had been initiated into those
abhorred mysteries. A more careful inquiry soon demonstrated that
the offenders did not exceed seven thousand; a number indeed
sufficiently alarming, when considered as the object of public
justice.\textsuperscript{168} It is with the same candid allowance that we should
interpret the vague expressions of Tacitus, and in a former
instance of Pliny, when they exaggerate the crowds of deluded
fanatics who had forsaken the established worship of the gods.
The church of Rome was undoubtedly the first and most populous of
the empire; and we are possessed of an authentic record which
attests the state of religion in that city about the middle of
the third century, and after a peace of thirty-eight years. The
clergy, at that time, consisted of a bishop, forty-six
presbyters, seven deacons, as many sub-deacons, forty-two
acolytes, and fifty readers, exorcists, and porters. The number
of widows, of the infirm, and of the poor, who were maintained by
the oblations of the faithful, amounted to fifteen hundred.\textsuperscript{169}
From reason, as well as from the analogy of Antioch, we may
venture to estimate the Christians of Rome at about fifty
thousand. The populousness of that great capital cannot perhaps
be exactly ascertained; but the most modest calculation will not
surely reduce it lower than a million of inhabitants, of whom the
Christians might constitute at the most a twentieth part.\textsuperscript{170}

\pagenote[167]{Ingens multitudo is the expression of Tacitus, xv.
44.}

\pagenote[168]{T. Liv. xxxix. 13, 15, 16, 17. Nothing could
exceed the horror and consternation of the senate on the
discovery of the Bacchanalians, whose depravity is described, and
perhaps exaggerated, by Livy.}

\pagenote[169]{Eusebius, l. vi. c. 43. The Latin translator (M.
de Valois) has thought proper to reduce the number of presbyters
to forty-four.}

\pagenote[170]{This proportion of the presbyters and of the poor,
to the rest of the people, was originally fixed by Burnet,
(Travels into Italy, p. 168,) and is approved by Moyle, (vol. ii.
p. 151.) They were both unacquainted with the passage of
Chrysostom, which converts their conjecture almost into a fact.}

The western provincials appeared to have derived the knowledge of
Christianity from the same source which had diffused among them
the language, the sentiments, and the manners of Rome.

In this more important circumstance, Africa, as well as Gaul was
gradually fashioned to the imitation of the capital. Yet
notwithstanding the many favorable occasions which might invite
the Roman missionaries to visit their Latin provinces, it was
late before they passed either the sea or the Alps;\textsuperscript{171} nor can
we discover in those great countries any assured traces either of
faith or of persecution that ascend higher than the reign of the
Antonines.\textsuperscript{172} The slow progress of the gospel in the cold
climate of Gaul, was extremely different from the eagerness with
which it seems to have been received on the burning sands of
Africa. The African Christians soon formed one of the principal
members of the primitive church. The practice introduced into
that province of appointing bishops to the most inconsiderable
towns, and very frequently to the most obscure villages,
contributed to multiply the splendor and importance of their
religious societies, which during the course of the third century
were animated by the zeal of Tertullian, directed by the
abilities of Cyprian, and adorned by the eloquence of Lactantius.

But if, on the contrary, we turn our eyes towards Gaul, we must
content ourselves with discovering, in the time of Marcus
Antoninus, the feeble and united congregations of Lyons and
Vienna; and even as late as the reign of Decius we are assured,
that in a few cities only, Arles, Narbonne, Thoulouse, Limoges,
Clermont, Tours, and Paris, some scattered churches were
supported by the devotion of a small number of Christians.\textsuperscript{173}
Silence is indeed very consistent with devotion; but as it is
seldom compatible with zeal, we may perceive and lament the
languid state of Christianity in those provinces which had
exchanged the Celtic for the Latin tongue, since they did not,
during the three first centuries, give birth to a single
ecclesiastical writer. From Gaul, which claimed a just
preeminence of learning and authority over all the countries on
this side of the Alps, the light of the gospel was more faintly
reflected on the remote provinces of Spain and Britain; and if we
may credit the vehement assertions of Tertullian, they had
already received the first rays of the faith, when he addressed
his apology to the magistrates of the emperor Severus.\textsuperscript{174} But
the obscure and imperfect origin of the western churches of
Europe has been so negligently recorded, that if we would relate
the time and manner of their foundation, we must supply the
silence of antiquity by those legends which avarice or
superstition long afterwards dictated to the monks in the lazy
gloom of their convents.\textsuperscript{175} Of these holy romances, that of the
apostle St. James can alone, by its singular extravagance,
deserve to be mentioned. From a peaceful fisherman of the Lake of
Gennesareth, he was transformed into a valorous knight, who
charged at the head of the Spanish chivalry in their battles
against the Moors. The gravest historians have celebrated his
exploits; the miraculous shrine of Compostella displayed his
power; and the sword of a military order, assisted by the terrors
of the Inquisition, was sufficient to remove every objection of
profane criticism.\textsuperscript{176}

\pagenote[171]{Serius trans Alpes, religione Dei suscepta.
Sulpicius Severus, l. ii. With regard to Africa, see Tertullian
ad Scapulam, c. 3. It is imagined that the Scyllitan martyrs were
the first, (Acta Sincera Rumart. p. 34.) One of the adversaries
of Apuleius seems to have been a Christian. Apolog. p. 496, 497,
edit. Delphin.}

\pagenote[172]{Tum primum intra Gallias martyria visa. Sulp.
Severus, l. ii. These were the celebrated martyrs of Lyons. See
Eusebius, v. i. Tillemont, Mem. Ecclesiast. tom. ii. p. 316.
According to the Donatists, whose assertion is confirmed by the
tacit acknowledgment of Augustin, Africa was the last of the
provinces which received the gospel. Tillemont, Mem. Ecclesiast.
tom. i. p. 754.}

\pagenote[173]{Raræ in aliquibus civitatibus ecclesiæ, paucorum
Christianorum devotione, resurgerent. Acta Sincera, p. 130.
Gregory of Tours, l i. c. 28. Mosheim, p. 207, 449. There is some
reason to believe that in the beginning of the fourth century,
the extensive dioceses of Liege, of Treves, and of Cologne,
composed a single bishopric, which had been very recently
founded. See Memoires de Tillemont, tom vi. part i. p. 43, 411.}

\pagenote[174]{The date of Tertullian’s Apology is fixed, in a
dissertation of Mosheim, to the year 198.}

\pagenote[175]{In the fifteenth century, there were few who had
either inclination or courage to question, whether Joseph of
Arimathea founded the monastery of Glastonbury, and whether
Dionysius the Areopagite preferred the residence of Paris to that
of Athens.}

\pagenote[176]{The stupendous metamorphosis was performed in the
ninth century. See Mariana, (Hist. Hispan. l. vii. c. 13, tom. i.
p. 285, edit. Hag. Com. 1733,) who, in every sense, imitates
Livy, and the honest detection of the legend of St. James by Dr.
Geddes, Miscellanies, vol. ii. p. 221.}

The progress of Christianity was not confined to the Roman
empire; and according to the primitive fathers, who interpret
facts by prophecy, the new religion, within a century after the
death of its divine Author, had already visited every part of the
globe. “There exists not,” says Justin Martyr, “a people, whether
Greek or Barbarian, or any other race of men, by whatsoever
appellation or manners they may be distinguished, however
ignorant of arts or agriculture, whether they dwell under tents,
or wander about in covered wagons, among whom prayers are not
offered up in the name of a crucified Jesus to the Father and
Creator of all things.”\textsuperscript{177} But this splendid exaggeration, which
even at present it would be extremely difficult to reconcile with
the real state of mankind, can be considered only as the rash
sally of a devout but careless writer, the measure of whose
belief was regulated by that of his wishes. But neither the
belief nor the wishes of the fathers can alter the truth of
history. It will still remain an undoubted fact, that the
barbarians of Scythia and Germany, who afterwards subverted the
Roman monarchy, were involved in the darkness of paganism; and
that even the conversion of Iberia, of Armenia, or of Æthiopia,
was not attempted with any degree of success till the sceptre was
in the hands of an orthodox emperor.\textsuperscript{178} Before that time, the
various accidents of war and commerce might indeed diffuse an
imperfect knowledge of the gospel among the tribes of Caledonia, \textsuperscript{179}
and among the borderers of the Rhine, the Danube, and the
Euphrates.\textsuperscript{180} Beyond the last-mentioned river, Edessa was
distinguished by a firm and early adherence to the faith.\textsuperscript{181}
From Edessa the principles of Christianity were easily introduced
into the Greek and Syrian cities which obeyed the successors of
Artaxerxes; but they do not appear to have made any deep
impression on the minds of the Persians, whose religious system,
by the labors of a well-disciplined order of priests, had been
constructed with much more art and solidity than the uncertain
mythology of Greece and Rome.\textsuperscript{182}

\pagenote[177]{Justin Martyr, Dialog. cum Tryphon. p. 341.
Irenæus adv. Hæres. l. i. c. 10. Tertullian adv. Jud. c. 7. See
Mosheim, p. 203.}

\pagenote[178]{See the fourth century of Mosheim’s History of the
Church. Many, though very confused circumstances, that relate to
the conversion of Iberia and Armenia, may be found in Moses of
Chorene, l. ii. c. 78—89. Note: Mons. St. Martin has shown that
Armenia was the first nation that embraced Christianity. Memoires
sur l’Armenie, vol. i. p. 306, and notes to Le Beæ. Gibbon,
indeed had expressed his intention of withdrawing the words “of
Armenia” from the text of future editions. (Vindication, Works,
iv. 577.) He was bitterly taunted by Person for neglecting or
declining to fulfil his promise. Preface to Letters to
Travis.—M.}

\pagenote[179]{According to Tertullian, the Christian faith had
penetrated into parts of Britain inaccessible to the Roman arms.
About a century afterwards, Ossian, the son of Fingal, is said to
have disputed, in his extreme old age, with one of the foreign
missionaries, and the dispute is still extant, in verse, and in
the Erse language. See Mr. Macpher son’s Dissertation on the
Antiquity of Ossian’s Poems, p. 10.}

\pagenote[180]{The Goths, who ravaged Asia in the reign of
Gallienus, carried away great numbers of captives; some of whom
were Christians, and became missionaries. See Tillemont, Memoires
Ecclesiast. tom. iv. p. 44.}

\pagenote[181]{The legends of Abgarus, fabulous as it is, affords
a decisive proof, that many years before Eusebius wrote his
history, the greatest part of the inhabitants of Edessa had
embraced Christianity. Their rivals, the citizens of Carrhæ,
adhered, on the contrary, to the cause of Paganism, as late as
the sixth century.}

\pagenote[182]{According to Bardesanes (ap. Euseb. Præpar.
Evangel.) there were some Christians in Persia before the end of
the second century. In the time of Constantine (see his epistle
to Sapor, Vit. l. iv. c. 13) they composed a flourishing church.
Consult Beausobre, Hist. Cristique du Manicheisme, tom. i. p.
180, and the Bibliotheca Orietalis of Assemani.}

\section{Part \thesection.}

From this impartial though imperfect survey of the progress of
Christianity, it may perhaps seem probable, that the number of
its proselytes has been excessively magnified by fear on the one
side, and by devotion on the other. According to the
irreproachable testimony of Origen,\textsuperscript{183} the proportion of the
faithful was very inconsiderable, when compared with the
multitude of an unbelieving world; but, as we are left without
any distinct information, it is impossible to determine, and it
is difficult even to conjecture, the real numbers of the
primitive Christians. The most favorable calculation, however,
that can be deduced from the examples of Antioch and of Rome,
will not permit us to imagine that more than a twentieth part of
the subjects of the empire had enlisted themselves under the
banner of the cross before the important conversion of
Constantine. But their habits of faith, of zeal, and of union,
seemed to multiply their numbers; and the same causes which
contributed to their future increase, served to render their
actual strength more apparent and more formidable.

\pagenote[183]{Origen contra Celsum, l. viii. p. 424.}

Such is the constitution of civil society, that, whilst a few
persons are distinguished by riches, by honors, and by knowledge,
the body of the people is condemned to obscurity, ignorance and
poverty. The Christian religion, which addressed itself to the
whole human race, must consequently collect a far greater number
of proselytes from the lower than from the superior ranks of
life. This innocent and natural circumstance has been improved
into a very odious imputation, which seems to be less strenuously
denied by the apologists, than it is urged by the adversaries, of
the faith; that the new sect of Christians was almost entirely
composed of the dregs of the populace, of peasants and mechanics,
of boys and women, of beggars and slaves, the last of whom might
sometimes introduce the missionaries into the rich and noble
families to which they belonged. These obscure teachers (such was
the charge of malice and infidelity) are as mute in public as
they are loquacious and dogmatical in private. Whilst they
cautiously avoid the dangerous encounter of philosophers, they
mingle with the rude and illiterate crowd, and insinuate
themselves into those minds whom their age, their sex, or their
education, has the best disposed to receive the impression of
superstitious terrors.\textsuperscript{184}

\pagenote[184]{Minucius Felix, c. 8, with Wowerus’s notes. Celsus
ap. Origen, l. iii. p. 138, 142. Julian ap. Cyril. l. vi. p. 206,
edit. Spanheim.}

This unfavorable picture, though not devoid of a faint
resemblance, betrays, by its dark coloring and distorted
features, the pencil of an enemy. As the humble faith of Christ
diffused itself through the world, it was embraced by several
persons who derived some consequence from the advantages of
nature or fortune. Aristides, who presented an eloquent apology
to the emperor Hadrian, was an Athenian philosopher.\textsuperscript{185} Justin
Martyr had sought divine knowledge in the schools of Zeno, of
Aristotle, of Pythagoras, and of Plato, before he fortunately was
accosted by the old man, or rather the angel, who turned his
attention to the study of the Jewish prophets.\textsuperscript{186} Clemens of
Alexandria had acquired much various reading in the Greek, and
Tertullian in the Latin, language. Julius Africanus and Origen
possessed a very considerable share of the learning of their
times; and although the style of Cyprian is very different from
that of Lactantius, we might almost discover that both those
writers had been public teachers of rhetoric. Even the study of
philosophy was at length introduced among the Christians, but it
was not always productive of the most salutary effects; knowledge
was as often the parent of heresy as of devotion, and the
description which was designed for the followers of Artemon, may,
with equal propriety, be applied to the various sects that
resisted the successors of the apostles. “They presume to alter
the Holy Scriptures, to abandon the ancient rule of faith, and to
form their opinions according to the subtile precepts of logic.
The science of the church is neglected for the study of geometry,
and they lose sight of heaven while they are employed in
measuring the earth. Euclid is perpetually in their hands.
Aristotle and Theophrastus are the objects of their admiration;
and they express an uncommon reverence for the works of Galen.
Their errors are derived from the abuse of the arts and sciences
of the infidels, and they corrupt the simplicity of the gospel by
the refinements of human reason.”\textsuperscript{187}

\pagenote[185]{Euseb. Hist. Eccles. iv. 3. Hieronym. Epist. 83.}

\pagenote[186]{The story is prettily told in Justin’s Dialogues.
Tillemont, (Mem Ecclesiast. tom. ii. p. 384,) who relates it
after him is sure that the old man was a disguised angel.}

\pagenote[187]{Eusebius, v. 28. It may be hoped, that none,
except the heretics, gave occasion to the complaint of Celsus,
(ap. Origen, l. ii. p. 77,) that the Christians were perpetually
correcting and altering their Gospels. * Note: Origen states in
reply, that he knows of none who had altered the Gospels except
the Marcionites, the Valentinians, and perhaps some followers of
Lucanus.—M.}

Nor can it be affirmed with truth, that the advantages of birth
and fortune were always separated from the profession of
Christianity. Several Roman citizens were brought before the
tribunal of Pliny, and he soon discovered, that a great number of
persons of \textit{every order} of men in Bithynia had deserted the
religion of their ancestors.\textsuperscript{188} His unsuspected testimony may,
in this instance, obtain more credit than the bold challenge of
Tertullian, when he addresses himself to the fears as well as the
humanity of the proconsul of Africa, by assuring him, that if he
persists in his cruel intentions, he must decimate Carthage, and
that he will find among the guilty many persons of his own rank,
senators and matrons of noblest extraction, and the friends or
relations of his most intimate friends.\textsuperscript{189} It appears, however,
that about forty years afterwards the emperor Valerian was
persuaded of the truth of this assertion, since in one of his
rescripts he evidently supposes that senators, Roman knights, and
ladies of quality, were engaged in the Christian sect.\textsuperscript{190} The
church still continued to increase its outward splendor as it
lost its internal purity; and, in the reign of Diocletian, the
palace, the courts of justice, and even the army, concealed a
multitude of Christians, who endeavored to reconcile the
interests of the present with those of a future life.

\pagenote[188]{Plin. Epist. x. 97. Fuerunt alii similis amentiæ,
cives Romani—-Multi enim omnis ætatis, omnis ordinis, utriusque
sexus, etiam vocuntur in periculum et vocabuntur.}

\pagenote[189]{Tertullian ad Scapulum. Yet even his rhetoric
rises no higher than to claim a tenth part of Carthage.}

\pagenote[190]{Cyprian. Epist. 70.}

And yet these exceptions are either too few in number, or too
recent in time, entirely to remove the imputation of ignorance
and obscurity which has been so arrogantly cast on the first
proselytes of Christianity.\textsuperscript{1901} Instead of employing in our
defence the fictions of later ages, it will be more prudent to
convert the occasion of scandal into a subject of edification.
Our serious thoughts will suggest to us, that the apostles
themselves were chosen by Providence among the fishermen of
Galilee, and that the lower we depress the temporal condition of
the first Christians, the more reason we shall find to admire
their merit and success. It is incumbent on us diligently to
remember, that the kingdom of heaven was promised to the poor in
spirit, and that minds afflicted by calamity and the contempt of
mankind, cheerfully listen to the divine promise of future
happiness; while, on the contrary, the fortunate are satisfied
with the possession of this world; and the wise abuse in doubt
and dispute their vain superiority of reason and knowledge.

\pagenote[1901]{This incomplete enumeration ought to be increased
by the names of several Pagans converted at the dawn of
Christianity, and whose conversion weakens the reproach which the
historian appears to support. Such are, the Proconsul Sergius
Paulus, converted at Paphos, (Acts xiii. 7—12.) Dionysius, member
of the Areopagus, converted with several others, al Athens, (Acts
xvii. 34;) several persons at the court of Nero, (Philip. iv 22;)
Erastus, receiver at Corinth, (Rom. xvi.23;) some Asiarchs, (Acts
xix. 31) As to the philosophers, we may add Tatian, Athenagoras,
Theophilus of Antioch, Hegesippus, Melito, Miltiades, Pantænus,
Ammenius, all distinguished for their genius and learning.—G.}

We stand in need of such reflections to comfort us for the loss
of some illustrious characters, which in our eyes might have
seemed the most worthy of the heavenly present. The names of
Seneca, of the elder and the younger Pliny, of Tacitus, of
Plutarch, of Galen, of the slave Epictetus, and of the emperor
Marcus Antoninus, adorn the age in which they flourished, and
exalt the dignity of human nature. They filled with glory their
respective stations, either in active or contemplative life;
their excellent understandings were improved by study; Philosophy
had purified their minds from the prejudices of the popular
superstitions; and their days were spent in the pursuit of truth
and the practice of virtue. Yet all these sages (it is no less an
object of surprise than of concern) overlooked or rejected the
perfection of the Christian system. Their language or their
silence equally discover their contempt for the growing sect,
which in their time had diffused itself over the Roman empire.
Those among them who condescended to mention the Christians,
consider them only as obstinate and perverse enthusiasts, who
exacted an implicit submission to their mysterious doctrines,
without being able to produce a single argument that could engage
the attention of men of sense and learning.\textsuperscript{191}

\pagenote[191]{Dr. Lardner, in his first and second volumes of
Jewish and Christian testimonies, collects and illustrates those
of Pliny the younger, of Tacitus, of Galen, of Marcus Antoninus,
and perhaps of Epictetus, (for it is doubtful whether that
philosopher means to speak of the Christians.) The new sect is
totally unnoticed by Seneca, the elder Pliny, and Plutarch.}

It is at least doubtful whether any of these philosophers perused
the apologies\textsuperscript{1911} which the primitive Christians repeatedly
published in behalf of themselves and of their religion; but it
is much to be lamented that such a cause was not defended by
abler advocates. They expose with superfluous wit and eloquence
the extravagance of Polytheism. They interest our compassion by
displaying the innocence and sufferings of their injured
brethren. But when they would demonstrate the divine origin of
Christianity, they insist much more strongly on the predictions
which announced, than on the miracles which accompanied, the
appearance of the Messiah. Their favorite argument might serve to
edify a Christian or to convert a Jew, since both the one and the
other acknowledge the authority of those prophecies, and both are
obliged, with devout reverence, to search for their sense and
their accomplishment. But this mode of persuasion loses much of
its weight and influence, when it is addressed to those who
neither understand nor respect the Mosaic dispensation and the
prophetic style.\textsuperscript{192} In the unskilful hands of Justin and of the
succeeding apologists, the sublime meaning of the Hebrew oracles
evaporates in distant types, affected conceits, and cold
allegories; and even their authenticity was rendered suspicious
to an unenlightened Gentile, by the mixture of pious forgeries,
which, under the names of Orpheus, Hermes, and the Sibyls,\textsuperscript{193}
were obtruded on him as of equal value with the genuine
inspirations of Heaven. The adoption of fraud and sophistry in
the defence of revelation too often reminds us of the injudicious
conduct of those poets who load their \textit{invulnerable} heroes with
a useless weight of cumbersome and brittle armor.

\pagenote[1911]{The emperors Hadrian, Antoninus \&c., read with
astonishment the apologies of Justin Martyr, of Aristides, of
Melito, \&c. (See St. Hieron. ad mag. orat. Orosius, lviii. c.
13.) Eusebius says expressly, that the cause of Christianity was
defended before the senate, in a very elegant discourse, by
Apollonius the Martyr.—G. ——Gibbon, in his severer spirit of
criticism, may have questioned the authority of Jerome and
Eusebius. There are some difficulties about Apollonius, which
Heinichen (note in loc. Eusebii) would solve, by suppose lag him
to have been, as Jerome states, a senator.—M.}

\pagenote[192]{If the famous prophecy of the Seventy Weeks had
been alleged to a Roman philosopher, would he not have replied in
the words of Cicero, “Quæ tandem ista auguratio est, annorum
potius quam aut rænsium aut dierum?” De Divinatione, ii. 30.
Observe with what irreverence Lucian, (in Alexandro, c. 13.) and
his friend Celsus ap. Origen, (l. vii. p. 327,) express
themselves concerning the Hebrew prophets.}

\pagenote[193]{The philosophers who derided the more ancient
predictions of the Sibyls, would easily have detected the Jewish
and Christian forgeries, which have been so triumphantly quoted
by the fathers, from Justin Martyr to Lactantius. When the
Sibylline verses had performed their appointed task, they, like
the system of the millennium, were quietly laid aside. The
Christian Sybil had unluckily fixed the ruin of Rome for the year
195, A. U. C. 948.}

But how shall we excuse the supine inattention of the Pagan and
philosophic world, to those evidences which were represented by
the hand of Omnipotence, not to their reason, but to their
senses? During the age of Christ, of his apostles, and of their
first disciples, the doctrine which they preached was confirmed
by innumerable prodigies. The lame walked, the blind saw, the
sick were healed, the dead were raised, dæmons were expelled, and
the laws of Nature were frequently suspended for the benefit of
the church. But the sages of Greece and Rome turned aside from
the awful spectacle, and, pursuing the ordinary occupations of
life and study, appeared unconscious of any alterations in the
moral or physical government of the world. Under the reign of
Tiberius, the whole earth,\textsuperscript{194} or at least a celebrated province
of the Roman empire,\textsuperscript{195} was involved in a preternatural darkness
of three hours. Even this miraculous event, which ought to have
excited the wonder, the curiosity, and the devotion of mankind,
passed without notice in an age of science and history.\textsuperscript{196} It
happened during the lifetime of Seneca and the elder Pliny, who
must have experienced the immediate effects, or received the
earliest intelligence, of the prodigy. Each of these
philosophers, in a laborious work, has recorded all the great
phenomena of Nature, earthquakes, meteors, comets, and eclipses,
which his indefatigable curiosity could collect.\textsuperscript{197} Both the one
and the other have omitted to mention the greatest phenomenon to
which the mortal eye has been witness since the creation of the
globe. A distinct chapter of Pliny\textsuperscript{198} is designed for eclipses
of an extraordinary nature and unusual duration; but he contents
himself with describing the singular defect of light which
followed the murder of Cæsar, when, during the greatest part of a
year, the orb of the sun appeared pale and without splendor. The
season of obscurity, which cannot surely be compared with the
preternatural darkness of the Passion, had been already
celebrated by most of the poets\textsuperscript{199} and historians of that
memorable age.\textsuperscript{200}

\pagenote[194]{The fathers, as they are drawn out in battle array
by Dom Calmet, (Dissertations sur la Bible, tom. iii. p.
295—308,) seem to cover the whole earth with darkness, in which
they are followed by most of the moderns.}

\pagenote[195]{Origen ad Matth. c. 27, and a few modern critics,
Beza, Le Clerc, Lardner, \&c., are desirous of confining it to the
land of Judea.}

\pagenote[196]{The celebrated passage of Phlegon is now wisely
abandoned. When Tertullian assures the Pagans that the mention of
the prodigy is found in Arcanis (not Archivis) vestris, (see his
Apology, c. 21,) he probably appeals to the Sibylline verses,
which relate it exactly in the words of the Gospel. * Note:
According to some learned theologians a misunderstanding of the
text in the Gospel has given rise to this mistake, which has
employed and wearied so many laborious commentators, though
Origen had already taken the pains to preinform them. The
expression does not mean, they assert, an eclipse, but any kind
of obscurity occasioned in the atmosphere, whether by clouds or
any other cause. As this obscuration of the sun rarely took place
in Palestine, where in the middle of April the sky was usually
clear, it assumed, in the eyes of the Jews and Christians, an
importance conformable to the received notion, that the sun
concealed at midday was a sinister presage. See Amos viii. 9, 10.
The word is often taken in this sense by contemporary writers;
the Apocalypse says the sun was concealed, when speaking of an
obscuration caused by smoke and dust. (Revel. ix. 2.) Moreover,
the Hebrew word ophal, which in the LXX. answers to the Greek,
signifies any darkness; and the Evangelists, who have modelled
the sense of their expressions by those of the LXX., must have
taken it in the same latitude. This darkening of the sky usually
precedes earthquakes. (Matt. xxvii. 51.) The Heathen authors
furnish us a number of examples, of which a miraculous
explanation was given at the time. See Ovid. ii. v. 33, l. xv. v.
785. Pliny, Hist. Nat. l. ii. c 30. Wetstein has collected all
these examples in his edition of the New Testament. We need not,
then, be astonished at the silence of the Pagan authors
concerning a phenomenon which did not extend beyond Jerusalem,
and which might have nothing contrary to the laws of nature;
although the Christians and the Jews may have regarded it as a
sinister presage. See Michaelis Notes on New Testament, v. i. p.
290. Paulus, Commentary on New Testament, iii. p. 760.—G.}

\pagenote[197]{Seneca, Quæst. Natur. l. i. 15, vi. l. vii. 17.
Plin. Hist. Natur. l. ii.}

\pagenote[198]{Plin. Hist. Natur. ii. 30.}

\pagenote[199]{Virgil. Georgic. i. 466. Tibullus, l. i. Eleg. v.
ver. 75. Ovid Metamorph. xv. 782. Lucan. Pharsal. i. 540. The
last of these poets places this prodigy before the civil war.}

\pagenote[200]{See a public epistle of M. Antony in Joseph.
Antiquit. xiv. 12. Plutarch in Cæsar. p. 471. Appian. Bell.
Civil. l. iv. Dion Cassius, l. xlv. p. 431. Julius Obsequens, c.
128. His little treatise is an abstract of Livy’s prodigies.}

