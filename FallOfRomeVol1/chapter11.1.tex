\chapter{Reign Of Claudius, Defeat Of The Goths.}
\section{Part \thesection.}

\textit{Reign Of Claudius. — Defeat Of The Goths. — Victories, Triumph, And
Death Of Aurelian.}
\vspace{\onelineskip}

Under the deplorable reigns of Valerian and Gallienus, the empire
was oppressed and almost destroyed by the soldiers, the tyrants,
and the barbarians. It was saved by a series of great princes,
who derived their obscure origin from the martial provinces of
Illyricum. Within a period of about thirty years, Claudius,
Aurelian, Probus, Diocletian and his colleagues, triumphed over
the foreign and domestic enemies of the state, reëstablished,
with the military discipline, the strength of the frontiers, and
deserved the glorious title of Restorers of the Roman world.

The removal of an effeminate tyrant made way for a succession of
heroes. The indignation of the people imputed all their
calamities to Gallienus, and the far greater part were, indeed,
the consequence of his dissolute manners and careless
administration. He was even destitute of a sense of honor, which
so frequently supplies the absence of public virtue; and as long
as he was permitted to enjoy the possession of Italy, a victory
of the barbarians, the loss of a province, or the rebellion of a
general, seldom disturbed the tranquil course of his pleasures.
At length, a considerable army, stationed on the Upper Danube,
invested with the Imperial purple their leader Aureolus; who,
disdaining a confined and barren reign over the mountains of
Rhætia, passed the Alps, occupied Milan, threatened Rome, and
challenged Gallienus to dispute in the field the sovereignty of
Italy. The emperor, provoked by the insult, and alarmed by the
instant danger, suddenly exerted that latent vigor which
sometimes broke through the indolence of his temper. Forcing
himself from the luxury of the palace, he appeared in arms at the
head of his legions, and advanced beyond the Po to encounter his
competitor. The corrupted name of Pontirolo\footnotemark[1] still preserves the
memory of a bridge over the Adda, which, during the action, must
have proved an object of the utmost importance to both armies.
The Rhætian usurper, after receiving a total defeat and a
dangerous wound, retired into Milan. The siege of that great city
was immediately formed; the walls were battered with every engine
in use among the ancients; and Aureolus, doubtful of his internal
strength, and hopeless of foreign succors already anticipated the
fatal consequences of unsuccessful rebellion.

\footnotetext[1]{Pons Aureoli, thirteen miles from Bergamo, and
thirty-two from Milan. See Cluver. Italia, Antiq. tom. i. p. 245.
Near this place, in the year 1703, the obstinate battle of
Cassano was fought between the French and Austrians. The
excellent relation of the Chevalier de Folard, who was present,
gives a very distinct idea of the ground. See Polybe de Folard,
tom. iii. p. 233-248.}

His last resource was an attempt to seduce the loyalty of the
besiegers. He scattered libels through the camp, inviting the
troops to desert an unworthy master, who sacrificed the public
happiness to his luxury, and the lives of his most valuable
subjects to the slightest suspicions. The arts of Aureolus
diffused fears and discontent among the principal officers of his
rival. A conspiracy was formed by Heraclianus, the Prætorian
præfect, by Marcian, a general of rank and reputation, and by
Cecrops, who commanded a numerous body of Dalmatian guards. The
death of Gallienus was resolved; and notwithstanding their desire
of first terminating the siege of Milan, the extreme danger which
accompanied every moment’s delay obliged them to hasten the
execution of their daring purpose. At a late hour of the night,
but while the emperor still protracted the pleasures of the
table, an alarm was suddenly given, that Aureolus, at the head of
all his forces, had made a desperate sally from the town;
Gallienus, who was never deficient in personal bravery, started
from his silken couch, and without allowing himself time either
to put on his armor, or to assemble his guards, he mounted on
horseback, and rode full speed towards the supposed place of the
attack. Encompassed by his declared or concealed enemies, he
soon, amidst the nocturnal tumult, received a mortal dart from an
uncertain hand. Before he expired, a patriotic sentiment rising
in the mind of Gallienus, induced him to name a deserving
successor; and it was his last request, that the Imperial
ornaments should be delivered to Claudius, who then commanded a
detached army in the neighborhood of Pavia. The report at least
was diligently propagated, and the order cheerfully obeyed by the
conspirators, who had already agreed to place Claudius on the
throne. On the first news of the emperor’s death, the troops
expressed some suspicion and resentment, till the one was
removed, and the other assuaged, by a donative of twenty pieces
of gold to each soldier. They then ratified the election, and
acknowledged the merit of their new sovereign.\footnotemark[2]

\footnotetext[2]{On the death of Gallienus, see Trebellius Pollio in
Hist. August. p. 181. Zosimus, l. i. p. 37. Zonaras, l. xii. p.
634. Eutrop. ix. ll. Aurelius Victor in Epitom. Victor in Cæsar.
I have compared and blended them all, but have chiefly followed
Aurelius Victor, who seems to have had the best memoirs.}

The obscurity which covered the origin of Claudius, though it was
afterwards embellished by some flattering fictions,\footnotemark[3]
sufficiently betrays the meanness of his birth. We can only
discover that he was a native of one of the provinces bordering
on the Danube; that his youth was spent in arms, and that his
modest valor attracted the favor and confidence of Decius. The
senate and people already considered him as an excellent officer,
equal to the most important trusts; and censured the inattention
of Valerian, who suffered him to remain in the subordinate
station of a tribune. But it was not long before that emperor
distinguished the merit of Claudius, by declaring him general and
chief of the Illyrian frontier, with the command of all the
troops in Thrace, Mæsia, Dacia, Pannonia, and Dalmatia, the
appointments of the præfect of Egypt, the establishment of the
proconsul of Africa, and the sure prospect of the consulship. By
his victories over the Goths, he deserved from the senate the
honor of a statue, and excited the jealous apprehensions of
Gallienus. It was impossible that a soldier could esteem so
dissolute a sovereign, nor is it easy to conceal a just contempt.
Some unguarded expressions which dropped from Claudius were
officiously transmitted to the royal ear. The emperor’s answer to
an officer of confidence describes in very lively colors his own
character, and that of the times. “There is not any thing capable
of giving me more serious concern, than the intelligence
contained in your last despatch;\footnotemark[4] that some malicious
suggestions have indisposed towards us the mind of our friend and
\textit{parent} Claudius. As you regard your allegiance, use every means
to appease his resentment, but conduct your negotiation with
secrecy; let it not reach the knowledge of the Dacian troops;
they are already provoked, and it might inflame their fury. I
myself have sent him some presents: be it your care that he
accept them with pleasure. Above all, let him not suspect that I
am made acquainted with his imprudence. The fear of my anger
might urge him to desperate counsels.”\footnotemark[5] The presents which
accompanied this humble epistle, in which the monarch solicited a
reconciliation with his discontented subject, consisted of a
considerable sum of money, a splendid wardrobe, and a valuable
service of silver and gold plate. By such arts Gallienus softened
the indignation and dispelled the fears of his Illyrian general;
and during the remainder of that reign, the formidable sword of
Claudius was always drawn in the cause of a master whom he
despised. At last, indeed, he received from the conspirators the
bloody purple of Gallienus: but he had been absent from their
camp and counsels; and however he might applaud the deed, we may
candidly presume that he was innocent of the knowledge of it.\footnotemark[6]
When Claudius ascended the throne, he was about fifty-four years
of age.

\footnotetext[3]{Some supposed him, oddly enough, to be a bastard of
the younger Gordian. Others took advantage of the province of
Dardania, to deduce his origin from Dardanus, and the ancient
kings of Troy.}

\footnotetext[4]{Notoria, a periodical and official despatch which
the emperor received from the frumentarii, or agents dispersed
through the provinces. Of these we may speak hereafter.}

\footnotetext[5]{Hist. August. p. 208. Gallienus describes the plate,
vestments, etc., like a man who loved and understood those
splendid trifles.}

\footnotetext[6]{Julian (Orat. i. p. 6) affirms that Claudius
acquired the empire in a just and even holy manner. But we may
distrust the partiality of a kinsman.}

The siege of Milan was still continued, and Aureolus soon
discovered that the success of his artifices had only raised up a
more determined adversary. He attempted to negotiate with
Claudius a treaty of alliance and partition. “Tell him,” replied
the intrepid emperor, “that such proposals should have been made
to Gallienus; \textit{he}, perhaps, might have listened to them with
patience, and accepted a colleague as despicable as himself.”\footnotemark[7]
This stern refusal, and a last unsuccessful effort, obliged
Aureolus to yield the city and himself to the discretion of the
conqueror. The judgment of the army pronounced him worthy of
death; and Claudius, after a feeble resistance, consented to the
execution of the sentence. Nor was the zeal of the senate less
ardent in the cause of their new sovereign. They ratified,
perhaps with a sincere transport of zeal, the election of
Claudius; and, as his predecessor had shown himself the personal
enemy of their order, they exercised, under the name of justice,
a severe revenge against his friends and family. The senate was
permitted to discharge the ungrateful office of punishment, and
the emperor reserved for himself the pleasure and merit of
obtaining by his intercession a general act of indemnity.\footnotemark[8]

\footnotetext[7]{Hist. August. p. 203. There are some trifling
differences concerning the circumstances of the last defeat and
death of Aureolus}

\footnotetext[8]{Aurelius Victor in Gallien. The people loudly prayed
for the damnation of Gallienus. The senate decreed that his
relations and servants should be thrown down headlong from the
Gemonian stairs. An obnoxious officer of the revenue had his eyes
torn out whilst under examination. Note: The expression is
curious, “terram matrem deosque inferos impias uti Gallieno
darent.”—M.}

Such ostentatious clemency discovers less of the real character
of Claudius, than a trifling circumstance in which he seems to
have consulted only the dictates of his heart. The frequent
rebellions of the provinces had involved almost every person in
the guilt of treason, almost every estate in the case of
confiscation; and Gallienus often displayed his liberality by
distributing among his officers the property of his subjects. On
the accession of Claudius, an old woman threw herself at his
feet, and complained that a general of the late emperor had
obtained an arbitrary grant of her patrimony. This general was
Claudius himself, who had not entirely escaped the contagion of
the times. The emperor blushed at the reproach, but deserved the
confidence which she had reposed in his equity. The confession of
his fault was accompanied with immediate and ample restitution.\footnotemark[9]

\footnotetext[9]{Zonaras, l. xii. p. 137.}

In the arduous task which Claudius had undertaken, of restoring
the empire to its ancient splendor, it was first necessary to
revive among his troops a sense of order and obedience. With the
authority of a veteran commander, he represented to them that the
relaxation of discipline had introduced a long train of
disorders, the effects of which were at length experienced by the
soldiers themselves; that a people ruined by oppression, and
indolent from despair, could no longer supply a numerous army
with the means of luxury, or even of subsistence; that the danger
of each individual had increased with the despotism of the
military order, since princes who tremble on the throne will
guard their safety by the instant sacrifice of every obnoxious
subject. The emperor expiated on the mischiefs of a lawless
caprice, which the soldiers could only gratify at the expense of
their own blood; as their seditious elections had so frequently
been followed by civil wars, which consumed the flower of the
legions either in the field of battle, or in the cruel abuse of
victory. He painted in the most lively colors the exhausted state
of the treasury, the desolation of the provinces, the disgrace of
the Roman name, and the insolent triumph of rapacious barbarians.
It was against those barbarians, he declared, that he intended to
point the first effort of their arms. Tetricus might reign for a
while over the West, and even Zenobia might preserve the dominion
of the East.\footnotemark[10] These usurpers were his personal adversaries; nor
could he think of indulging any private resentment till he had
saved an empire, whose impending ruin would, unless it was timely
prevented, crush both the army and the people.

\footnotetext[10]{Zonaras on this occasion mentions Posthumus but the
registers of the senate (Hist. August. p. 203) prove that
Tetricus was already emperor of the western provinces.}

The various nations of Germany and Sarmatia, who fought under the
Gothic standard, had already collected an armament more
formidable than any which had yet issued from the Euxine. On the
banks of the Niester, one of the great rivers that discharge
themselves into that sea, they constructed a fleet of two
thousand, or even of six thousand vessels;\footnotemark[11] numbers which,
however incredible they may seem, would have been insufficient to
transport their pretended army of three hundred and twenty
thousand barbarians. Whatever might be the real strength of the
Goths, the vigor and success of the expedition were not adequate
to the greatness of the preparations. In their passage through
the Bosphorus, the unskilful pilots were overpowered by the
violence of the current; and while the multitude of their ships
were crowded in a narrow channel, many were dashed against each
other, or against the shore. The barbarians made several descents
on the coasts both of Europe and Asia; but the open country was
already plundered, and they were repulsed with shame and loss
from the fortified cities which they assaulted. A spirit of
discouragement and division arose in the fleet, and some of their
chiefs sailed away towards the islands of Crete and Cyprus; but
the main body, pursuing a more steady course, anchored at length
near the foot of Mount Athos, and assaulted the city of
Thessalonica, the wealthy capital of all the Macedonian
provinces. Their attacks, in which they displayed a fierce but
artless bravery, were soon interrupted by the rapid approach of
Claudius, hastening to a scene of action that deserved the
presence of a warlike prince at the head of the remaining powers
of the empire. Impatient for battle, the Goths immediately broke
up their camp, relinquished the siege of Thessalonica, left their
navy at the foot of Mount Athos, traversed the hills of
Macedonia, and pressed forwards to engage the last defence of
Italy.

\footnotetext[11]{The Augustan History mentions the smaller, Zonaras
the larger number; the lively fancy of Montesquieu induced him to
prefer the latter.}

We still posses an original letter addressed by Claudius to the
senate and people on this memorable occasion. “Conscript
fathers,” says the emperor, “know that three hundred and twenty
thousand Goths have invaded the Roman territory. If I vanquish
them, your gratitude will reward my services. Should I fall,
remember that I am the successor of Gallienus. The whole republic
is fatigued and exhausted. We shall fight after Valerian, after
Ingenuus, Regillianus, Lollianus, Posthumus, Celsus, and a
thousand others, whom a just contempt for Gallienus provoked into
rebellion. We are in want of darts, of spears, and of shields.
The strength of the empire, Gaul, and Spain, are usurped by
Tetricus, and we blush to acknowledge that the archers of the
East serve under the banners of Zenobia. Whatever we shall
perform will be sufficiently great.”\footnotemark[12] The melancholy firmness
of this epistle announces a hero careless of his fate, conscious
of his danger, but still deriving a well-grounded hope from the
resources of his own mind.

\footnotetext[12]{Trebell. Pollio in Hist. August. p. 204.}

The event surpassed his own expectations and those of the world.
By the most signal victories he delivered the empire from this
host of barbarians, and was distinguished by posterity under the
glorious appellation of the Gothic Claudius. The imperfect
historians of an irregular war\footnotemark[13] do not enable us to describe
the order and circumstances of his exploits; but, if we could be
indulged in the allusion, we might distribute into three acts
this memorable tragedy. I. The decisive battle was fought near
Naissus, a city of Dardania. The legions at first gave way,
oppressed by numbers, and dismayed by misfortunes. Their ruin was
inevitable, had not the abilities of their emperor prepared a
seasonable relief. A large detachment, rising out of the secret
and difficult passes of the mountains, which, by his order, they
had occupied, suddenly assailed the rear of the victorious Goths.

The favorable instant was improved by the activity of Claudius.
He revived the courage of his troops, restored their ranks, and
pressed the barbarians on every side. Fifty thousand men are
reported to have been slain in the battle of Naissus. Several
large bodies of barbarians, covering their retreat with a movable
fortification of wagons, retired, or rather escaped, from the
field of slaughter.

II. We may presume that some insurmountable difficulty, the
fatigue, perhaps, or the disobedience, of the conquerors,
prevented Claudius from completing in one day the destruction of
the Goths. The war was diffused over the province of Mæsia,
Thrace, and Macedonia, and its operations drawn out into a
variety of marches, surprises, and tumultuary engagements, as
well by sea as by land. When the Romans suffered any loss, it was
commonly occasioned by their own cowardice or rashness; but the
superior talents of the emperor, his perfect knowledge of the
country, and his judicious choice of measures as well as
officers, assured on most occasions the success of his arms. The
immense booty, the fruit of so many victories, consisted for the
greater part of cattle and slaves. A select body of the Gothic
youth was received among the Imperial troops; the remainder was
sold into servitude; and so considerable was the number of female
captives that every soldier obtained to his share two or three
women. A circumstance from which we may conclude, that the
invaders entertained some designs of settlement as well as of
plunder; since even in a naval expedition, they were accompanied
by their families.

III. The loss of their fleet, which was either taken or sunk, had
intercepted the retreat of the Goths. A vast circle of Roman
posts, distributed with skill, supported with firmness, and
gradually closing towards a common centre, forced the barbarians
into the most inaccessible parts of Mount Hæmus, where they found
a safe refuge, but a very scanty subsistence. During the course
of a rigorous winter in which they were besieged by the emperor’s
troops, famine and pestilence, desertion and the sword,
continually diminished the imprisoned multitude. On the return of
spring, nothing appeared in arms except a hardy and desperate
band, the remnant of that mighty host which had embarked at the
mouth of the Niester.

\footnotetext[13]{Hist. August. in Claud. Aurelian. et Prob. Zosimus,
l. i. p. 38-42. Zonaras, l. xii. p. 638. Aurel. Victor in Epitom.
Victor Junior in Cæsar. Eutrop. ix ll. Euseb. in Chron.}

The pestilence which swept away such numbers of the barbarians,
at length proved fatal to their conqueror. After a short but
glorious reign of two years, Claudius expired at Sirmium, amidst
the tears and acclamations of his subjects. In his last illness,
he convened the principal officers of the state and army, and in
their presence recommended Aurelian,\footnotemark[14] one of his generals, as
the most deserving of the throne, and the best qualified to
execute the great design which he himself had been permitted only
to undertake. The virtues of Claudius, his valor, affability,
justice, and temperance, his love of fame and of his country,
place him in that short list of emperors who added lustre to the
Roman purple. Those virtues, however, were celebrated with
peculiar zeal and complacency by the courtly writers of the age
of Constantine, who was the great-grandson of Crispus, the elder
brother of Claudius. The voice of flattery was soon taught to
repeat, that gods, who so hastily had snatched Claudius from the
earth, rewarded his merit and piety by the perpetual
establishment of the empire in his family.\footnotemark[15]

\footnotetext[14]{According to Zonaras, (l. xii. p. 638,) Claudius,
before his death, invested him with the purple; but this singular
fact is rather contradicted than confirmed by other writers.}

\footnotetext[15]{See the Life of Claudius by Pollio, and the
Orations of Mamertinus, Eumenius, and Julian. See likewise the
Cæsars of Julian p. 318. In Julian it was not adulation, but
superstition and vanity.}

Notwithstanding these oracles, the greatness of the Flavian
family (a name which it had pleased them to assume) was deferred
above twenty years, and the elevation of Claudius occasioned the
immediate ruin of his brother Quintilius, who possessed not
sufficient moderation or courage to descend into the private
station to which the patriotism of the late emperor had condemned
him. Without delay or reflection, he assumed the purple at
Aquileia, where he commanded a considerable force; and though his
reign lasted only seventeen days,\footnotemark[151] he had time to obtain the
sanction of the senate, and to experience a mutiny of the troops.

As soon as he was informed that the great army of the Danube had
invested the well-known valor of Aurelian with Imperial power, he
sunk under the fame and merit of his rival; and ordering his
veins to be opened, prudently withdrew himself from the unequal
contest.\footnotemark[16]

\footnotetext[151]{Such is the narrative of the greater part of the
older historians; but the number and the variety of his medals
seem to require more time, and give probability to the report of
Zosimus, who makes him reign some months.—G.}

\footnotetext[16]{Zosimus, l. i. p. 42. Pollio (Hist. August. p. 107)
allows him virtues, and says, that, like Pertinax, he was killed
by the licentious soldiers. According to Dexippus, he died of a
disease.}

The general design of this work will not permit us minutely to
relate the actions of every emperor after he ascended the throne,
much less to deduce the various fortunes of his private life. We
shall only observe, that the father of Aurelian was a peasant of
the territory of Sirmium, who occupied a small farm, the property
of Aurelius, a rich senator. His warlike son enlisted in the
troops as a common soldier, successively rose to the rank of a
centurion, a tribune, the præfect of a legion, the inspector of
the camp, the general, or, as it was then called, the duke, of a
frontier; and at length, during the Gothic war, exercised the
important office of commander-in-chief of the cavalry. In every
station he distinguished himself by matchless valor,\footnotemark[17] rigid
discipline, and successful conduct. He was invested with the
consulship by the emperor Valerian, who styles him, in the
pompous language of that age, the deliverer of Illyricum, the
restorer of Gaul, and the rival of the Scipios. At the
recommendation of Valerian, a senator of the highest rank and
merit, Ulpius Crinitus, whose blood was derived from the same
source as that of Trajan, adopted the Pannonian peasant, gave him
his daughter in marriage, and relieved with his ample fortune the
honorable poverty which Aurelian had preserved inviolate.\footnotemark[18]

\footnotetext[17]{Theoclius (as quoted in the Augustan History, p.
211) affirms that in one day he killed with his own hand
forty-eight Sarmatians, and in several subsequent engagements
nine hundred and fifty. This heroic valor was admired by the
soldiers, and celebrated in their rude songs, the burden of which
was, mille, mile, mille, occidit.}

\footnotetext[18]{Acholius (ap. Hist. August. p. 213) describes the
ceremony of the adoption, as it was performed at Byzantium, in
the presence of the emperor and his great officers.}

The reign of Aurelian lasted only four years and about nine
months; but every instant of that short period was filled by some
memorable achievement. He put an end to the Gothic war, chastised
the Germans who invaded Italy, recovered Gaul, Spain, and Britain
out of the hands of Tetricus, and destroyed the proud monarchy
which Zenobia had erected in the East on the ruins of the
afflicted empire.

It was the rigid attention of Aurelian, even to the minutest
articles of discipline, which bestowed such uninterrupted success
on his arms. His military regulations are contained in a very
concise epistle to one of his inferior officers, who is commanded
to enforce them, as he wishes to become a tribune, or as he is
desirous to live. Gaming, drinking, and the arts of divination,
were severely prohibited. Aurelian expected that his soldiers
should be modest, frugal, and laborious; that their armor should
be constantly kept bright, their weapons sharp, their clothing
and horses ready for immediate service; that they should live in
their quarters with chastity and sobriety, without damaging the
cornfields, without stealing even a sheep, a fowl, or a bunch of
grapes, without exacting from their landlords either salt, or
oil, or wood. “The public allowance,” continues the emperor, “is
sufficient for their support; their wealth should be collected
from the spoils of the enemy, not from the tears of the
provincials.”\footnotemark[19] A single instance will serve to display the
rigor, and even cruelty, of Aurelian. One of the soldiers had
seduced the wife of his host. The guilty wretch was fastened to
two trees forcibly drawn towards each other, and his limbs were
torn asunder by their sudden separation. A few such examples
impressed a salutary consternation. The punishments of Aurelian
were terrible; but he had seldom occasion to punish more than
once the same offence. His own conduct gave a sanction to his
laws, and the seditious legions dreaded a chief who had learned
to obey, and who was worthy to command.

\footnotetext[19]{Hist. August, p. 211 This laconic epistle is truly
the work of a soldier; it abounds with military phrases and
words, some of which cannot be understood without difficulty.
Ferramenta samiata is well explained by Salmasius. The former of
the words means all weapons of offence, and is contrasted with
Arma, defensive armor The latter signifies keen and well
sharpened.}

