\section{Part \thesection.}
\thispagestyle{simple}

When the Roman princes had lost sight of the senate and of their
ancient capital, they easily forgot the origin and nature of
their legal power. The civil offices of consul, of proconsul, of
censor, and of tribune, by the union of which it had been formed,
betrayed to the people its republican extraction. Those modest
titles were laid aside;\footnotemark[97] and if they still distinguished their
high station by the appellation of Emperor, or Imperator, that
word was understood in a new and more dignified sense, and no
longer denoted the general of the Roman armies, but the sovereign
of the Roman world. The name of Emperor, which was at first of a
military nature, was associated with another of a more servile
kind. The epithet of Dominus, or Lord, in its primitive
signification, was expressive not of the authority of a prince
over his subjects, or of a commander over his soldiers, but of
the despotic power of a master over his domestic slaves.\footnotemark[98]
Viewing it in that odious light, it had been rejected with
abhorrence by the first Cæsars. Their resistance insensibly
became more feeble, and the name less odious; till at length the
style of \textit{our Lord and Emperor} was not only bestowed by
flattery, but was regularly admitted into the laws and public
monuments. Such lofty epithets were sufficient to elate and
satisfy the most excessive vanity; and if the successors of
Diocletian still declined the title of King, it seems to have
been the effect not so much of their moderation as of their
delicacy. Wherever the Latin tongue was in use, (and it was the
language of government throughout the empire,) the Imperial
title, as it was peculiar to themselves, conveyed a more
respectable idea than the name of king, which they must have
shared with a hundred barbarian chieftains; or which, at the
best, they could derive only from Romulus, or from Tarquin. But
the sentiments of the East were very different from those of the
West. From the earliest period of history, the sovereigns of Asia
had been celebrated in the Greek language by the title of
Basileus, or King; and since it was considered as the first
distinction among men, it was soon employed by the servile
provincials of the East, in their humble addresses to the Roman
throne.\footnotemark[99] Even the attributes, or at least the titles, of the
DIVINITY, were usurped by Diocletian and Maximian, who
transmitted them to a succession of Christian emperors.\footnotemark[100] Such
extravagant compliments, however, soon lose their impiety by
losing their meaning; and when the ear is once accustomed to the
sound, they are heard with indifference, as vague though
excessive professions of respect.

\footnotetext[97]{See the 12th dissertation in Spanheim’s excellent
work de Usu Numismatum. From medals, inscriptions, and
historians, he examines every title separately, and traces it
from Augustus to the moment of its disappearing.}

\footnotetext[98]{Pliny (in Panegyr. c. 3, 55, \&c.) speaks of Dominus
with execration, as synonymous to Tyrant, and opposite to Prince.
And the same Pliny regularly gives that title (in the tenth book
of the epistles) to his friend rather than master, the virtuous
Trajan. This strange contradiction puzzles the commentators, who
think, and the translators, who can write.}

\footnotetext[99]{Synesius de Regno, edit. Petav. p. 15. I am
indebted for this quotation to the Abbé de la Bleterie.}

\footnotetext[100]{Soe Vandale de Consecratione, p. 354, \&c. It was
customary for the emperors to mention (in the preamble of laws)
their numen, sacreo majesty, divine oracles, \&c. According to
Tillemont, Gregory Nazianzen complains most bitterly of the
profanation, especially when it was practised by an Arian
emperor. * Note: In the time of the republic, says Hegewisch,
when the consuls, the prætors, and the other magistrates appeared
in public, to perform the functions of their office, their
dignity was announced both by the symbols which use had
consecrated, and the brilliant cortege by which they were
accompanied. But this dignity belonged to the office, not to the
individual; this pomp belonged to the magistrate, not to the man.
* * The consul, followed, in the comitia, by all the senate, the
prætors, the quæstors, the ædiles, the lictors, the apparitors,
and the heralds, on reentering his house, was served only by
freedmen and by his slaves. The first emperors went no further.
Tiberius had, for his personal attendance, only a moderate number
of slaves, and a few freedmen. (Tacit. Ann. iv. 7.) But in
proportion as the republican forms disappeared, one after
another, the inclination of the emperors to environ themselves
with personal pomp, displayed itself more and more. ** The
magnificence and the ceremonial of the East were entirely
introduced by Diocletian, and were consecrated by Constantine to
the Imperial use. Thenceforth the palace, the court, the table,
all the personal attendance, distinguished the emperor from his
subjects, still more than his superior dignity. The organization
which Diocletian gave to his new court, attached less honor and
distinction to rank than to services performed towards the
members of the Imperial family. Hegewisch, Essai, Hist. sur les
Finances Romains. Few historians have characterized, in a more
philosophic manner, the influence of a new institution.—G.——It is
singular that the son of a slave reduced the haughty aristocracy
of Home to the offices of servitude.—M.}

From the time of Augustus to that of Diocletian, the Roman
princes, conversing in a familiar manner among their
fellow-citizens, were saluted only with the same respect that was
usually paid to senators and magistrates. Their principal
distinction was the Imperial or military robe of purple; whilst
the senatorial garment was marked by a broad, and the equestrian
by a narrow, band or stripe of the same honorable color. The
pride, or rather the policy, of Diocletian engaged that artful
prince to introduce the stately magnificence of the court of
Persia.\footnotemark[101] He ventured to assume the diadem, an ornament
detested by the Romans as the odious ensign of royalty, and the
use of which had been considered as the most desperate act of the
madness of Caligula. It was no more than a broad white fillet set
with pearls, which encircled the emperor’s head. The sumptuous
robes of Diocletian and his successors were of silk and gold; and
it is remarked with indignation that even their shoes were
studded with the most precious gems. The access to their sacred
person was every day rendered more difficult by the institution
of new forms and ceremonies. The avenues of the palace were
strictly guarded by the various schools, as they began to be
called, of domestic officers. The interior apartments were
intrusted to the jealous vigilance of the eunuchs, the increase
of whose numbers and influence was the most infallible symptom of
the progress of despotism. When a subject was at length admitted
to the Imperial presence, he was obliged, whatever might be his
rank, to fall prostrate on the ground, and to adore, according to
the eastern fashion, the divinity of his lord and master.\footnotemark[102]
Diocletian was a man of sense, who, in the course of private as
well as public life, had formed a just estimate both of himself
and of mankind; nor is it easy to conceive that in substituting
the manners of Persia to those of Rome he was seriously actuated
by so mean a principle as that of vanity. He flattered himself
that an ostentation of splendor and luxury would subdue the
imagination of the multitude; that the monarch would be less
exposed to the rude license of the people and the soldiers, as
his person was secluded from the public view; and that habits of
submission would insensibly be productive of sentiments of
veneration. Like the modesty affected by Augustus, the state
maintained by Diocletian was a theatrical representation; but it
must be confessed, that of the two comedies, the former was of a
much more liberal and manly character than the latter. It was the
aim of the one to disguise, and the object of the other to
display, the unbounded power which the emperors possessed over
the Roman world.

\footnotetext[101]{See Spanheim de Usu Numismat. Dissert. xii.}

\footnotetext[102]{Aurelius Victor. Eutropius, ix. 26. It appears by
the Panegyrists, that the Romans were soon reconciled to the name
and ceremony of adoration.}

Ostentation was the first principle of the new system instituted
by Diocletian. The second was division. He divided the empire,
the provinces, and every branch of the civil as well as military
administration. He multiplied the wheels of the machine of
government, and rendered its operations less rapid, but more
secure. Whatever advantages and whatever defects might attend
these innovations, they must be ascribed in a very great degree
to the first inventor; but as the new frame of policy was
gradually improved and completed by succeeding princes, it will
be more satisfactory to delay the consideration of it till the
season of its full maturity and perfection.\footnotemark[103] Reserving,
therefore, for the reign of Constantine a more exact picture of
the new empire, we shall content ourselves with describing the
principal and decisive outline, as it was traced by the hand of
Diocletian. He had associated three colleagues in the exercise of
the supreme power; and as he was convinced that the abilities of
a single man were inadequate to the public defence, he considered
the joint administration of four princes not as a temporary
expedient, but as a fundamental law of the constitution. It was
his intention that the two elder princes should be distinguished
by the use of the diadem, and the title of \textit{Augusti;} that, as
affection or esteem might direct their choice, they should
regularly call to their assistance two subordinate colleagues;
and that the \textit{Cæsars}, rising in their turn to the first rank,
should supply an uninterrupted succession of emperors. The empire
was divided into four parts. The East and Italy were the most
honorable, the Danube and the Rhine the most laborious stations.
The former claimed the presence of the \textit{Augusti}, the latter were
intrusted to the administration of the \textit{Cæsars}. The strength of
the legions was in the hands of the four partners of sovereignty,
and the despair of successively vanquishing four formidable
rivals might intimidate the ambition of an aspiring general. In
their civil government the emperors were supposed to exercise the
undivided power of the monarch, and their edicts, inscribed with
their joint names, were received in all the provinces, as
promulgated by their mutual councils and authority.
Notwithstanding these precautions, the political union of the
Roman world was gradually dissolved, and a principle of division
was introduced, which, in the course of a few years, occasioned
the perpetual separation of the Eastern and Western Empires.

\footnotetext[103]{The innovations introduced by Diocletian are
chiefly deduced, 1st, from some very strong passages in
Lactantius; and, 2dly, from the new and various offices which, in
the Theodosian code, appear already established in the beginning
of the reign of Constantine.}

The system of Diocletian was accompanied with another very
material disadvantage, which cannot even at present be totally
overlooked; a more expensive establishment, and consequently an
increase of taxes, and the oppression of the people. Instead of a
modest family of slaves and freedmen, such as had contented the
simple greatness of Augustus and Trajan, three or four
magnificent courts were established in the various parts of the
empire, and as many Roman \textit{kings} contended with each other and
with the Persian monarch for the vain superiority of pomp and
luxury. The number of ministers, of magistrates, of officers, and
of servants, who filled the different departments of the state,
was multiplied beyond the example of former times; and (if we may
borrow the warm expression of a contemporary) “when the
proportion of those who received exceeded the proportion of those
who contributed, the provinces were oppressed by the weight of
tributes.”\footnotemark[104] From this period to the extinction of the empire,
it would be easy to deduce an uninterrupted series of clamors and
complaints. According to his religion and situation, each writer
chooses either Diocletian, or Constantine, or Valens, or
Theodosius, for the object of his invectives; but they
unanimously agree in representing the burden of the public
impositions, and particularly the land tax and capitation, as the
intolerable and increasing grievance of their own times. From
such a concurrence, an impartial historian, who is obliged to
extract truth from satire, as well as from panegyric, will be
inclined to divide the blame among the princes whom they accuse,
and to ascribe their exactions much less to their personal vices,
than to the uniform system of their administration.\footnotemark[1041] The
emperor Diocletian was indeed the author of that system; but
during his reign, the growing evil was confined within the bounds
of modesty and discretion, and he deserves the reproach of
establishing pernicious precedents, rather than of exercising
actual oppression.\footnotemark[105] It may be added, that his revenues were
managed with prudent economy; and that after all the current
expenses were discharged, there still remained in the Imperial
treasury an ample provision either for judicious liberality or
for any emergency of the state.

\footnotetext[104]{Lactant. de M. P. c. 7.}

\footnotetext[1041]{The most curious document which has come to light
since the publication of Gibbon’s History, is the edict of
Diocletian, published from an inscription found at Eskihissar,
(Stratoniccia,) by Col. Leake. This inscription was first copied
by Sherard, afterwards much more completely by Mr. Bankes. It is
confirmed and illustrated by a more imperfect copy of the same
edict, found in the Levant by a gentleman of Aix, and brought to
this country by M. Vescovali. This edict was issued in the name
of the four Cæsars, Diocletian, Maximian, Constantius, and
Galerius. It fixed a maximum of prices throughout the empire, for
all the necessaries and commodities of life. The preamble
insists, with great vehemence on the extortion and inhumanity of
the venders and merchants. Quis enim adeo obtunisi (obtusi)
pectores (is) et a sensu inhumanitatis extorris est qui ignorare
potest immo non senserit in venalibus rebus quævel in mercimoniis
aguntur vel diurna urbium conversatione tractantur, in tantum se
licen liam defusisse, ut effrænata libido rapien—rum copia nec
annorum ubertatibus mitigaretur. The edict, as Col. Leake clearly
shows, was issued A. C. 303. Among the articles of which the
maximum value is assessed, are oil, salt, honey, butchers’ meat,
poultry, game, fish, vegetables, fruit the wages of laborers and
artisans, schoolmasters and skins, boots and shoes, harness,
timber, corn, wine, and beer, (zythus.) The depreciation in the
value of money, or the rise in the price of commodities, had been
so great during the past century, that butchers’ meat, which, in
the second century of the empire, was in Rome about two denaril
the pound, was now fixed at a maximum of eight. Col. Leake
supposes the average price could not be less than four: at the
same time the maximum of the wages of the agricultural laborers
was twenty-five. The whole edict is, perhaps, the most gigantic
effort of a blind though well-intentioned despotism, to control
that which is, and ought to be, beyond the regulation of the
government. See an Edict of Diocletian, by Col. Leake, London,
1826. Col. Leake has not observed that this Edict is expressly
named in the treatise de Mort. Persecut. ch. vii. Idem cum variis
iniquitatibus immensam faceret caritatem, legem pretiis rerum
venalium statuere conatus.—M}

\footnotetext[105]{Indicta lex nova quæ sane illorum temporum
modestia tolerabilis, in perniciem processit. Aurel. Victor., who
has treated the character of Diocletian with good sense, though
in bad Latin.}

It was in the twenty first year of his reign that Diocletian
executed his memorable resolution of abdicating the empire; an
action more naturally to have been expected from the elder or the
younger Antoninus, than from a prince who had never practised the
lessons of philosophy either in the attainment or in the use of
supreme power. Diocletian acquired the glory of giving to the
world the first example of a resignation,\footnotemark[106] which has not been
very frequently imitated by succeeding monarchs. The parallel of
Charles the Fifth, however, will naturally offer itself to our
mind, not only since the eloquence of a modern historian has
rendered that name so familiar to an English reader, but from the
very striking resemblance between the characters of the two
emperors, whose political abilities were superior to their
military genius, and whose specious virtues were much less the
effect of nature than of art. The abdication of Charles appears
to have been hastened by the vicissitudes of fortune; and the
disappointment of his favorite schemes urged him to relinquish a
power which he found inadequate to his ambition. But the reign of
Diocletian had flowed with a tide of uninterrupted success; nor
was it till after he had vanquished all his enemies, and
accomplished all his designs, that he seems to have entertained
any serious thoughts of resigning the empire. Neither Charles nor
Diocletian were arrived at a very advanced period of life; since
the one was only fifty-five, and the other was no more than
fifty-nine years of age; but the active life of those princes,
their wars and journeys, the cares of royalty, and their
application to business, had already impaired their constitution,
and brought on the infirmities of a premature old age.\footnotemark[107]

\footnotetext[106]{Solus omnium post conditum Romanum Imperium, qui
extanto fastigio sponte ad privatæ vitæ statum civilitatemque
remearet, Eutrop. ix. 28.}

\footnotetext[107]{The particulars of the journey and illness are
taken from Laclantius, c. 17, who may sometimes be admitted as an
evidence of public facts, though very seldom of private
anecdotes.}

Notwithstanding the severity of a very cold and rainy winter,
Diocletian left Italy soon after the ceremony of his triumph, and
began his progress towards the East round the circuit of the
Illyrian provinces. From the inclemency of the weather, and the
fatigue of the journey, he soon contracted a slow illness; and
though he made easy marches, and was generally carried in a close
litter, his disorder, before he arrived at Nicomedia, about the
end of the summer, was become very serious and alarming. During
the whole winter he was confined to his palace: his danger
inspired a general and unaffected concern; but the people could
only judge of the various alterations of his health, from the joy
or consternation which they discovered in the countenances and
behavior of his attendants. The rumor of his death was for some
time universally believed, and it was supposed to be concealed
with a view to prevent the troubles that might have happened
during the absence of the Cæsar Galerius. At length, however, on
the first of March, Diocletian once more appeared in public, but
so pale and emaciated, that he could scarcely have been
recognized by those to whom his person was the most familiar. It
was time to put an end to the painful struggle, which he had
sustained during more than a year, between the care of his health
and that of his dignity. The former required indulgence and
relaxation, the latter compelled him to direct, from the bed of
sickness, the administration of a great empire. He resolved to
pass the remainder of his days in honorable repose, to place his
glory beyond the reach of fortune, and to relinquish the theatre
of the world to his younger and more active associates.\footnotemark[108]

\footnotetext[108]{Aurelius Victor ascribes the abdication, which had
been so variously accounted for, to two causes: 1st, Diocletian’s
contempt of ambition; and 2dly, His apprehension of impending
troubles. One of the panegyrists (vi. 9) mentions the age and
infirmities of Diocletian as a very natural reason for his
retirement. * Note: Constantine (Orat. ad Sanct. c. 401) more
than insinuated that derangement of mind, connected with the
conflagration of the palace at Nicomedia by lightning, was the
cause of his abdication. But Heinichen. in a very sensible note
on this passage in Eusebius, while he admits that his long
illness might produce a temporary depression of spirits,
triumphantly appeals to the philosophical conduct of Diocletian
in his retreat, and the influence which he still retained on
public affairs.—M.}

The ceremony of his abdication was performed in a spacious plain,
about three miles from Nicomedia. The emperor ascended a lofty
throne, and in a speech, full of reason and dignity, declared his
intention, both to the people and to the soldiers who were
assembled on this extraordinary occasion. As soon as he had
divested himself of his purple, he withdrew from the gazing
multitude; and traversing the city in a covered chariot,
proceeded, without delay, to the favorite retirement which he had
chosen in his native country of Dalmatia. On the same day, which
was the first of May,\footnotemark[109] Maximian, as it had been previously
concerted, made his resignation of the Imperial dignity at Milan.

Even in the splendor of the Roman triumph, Diocletian had
meditated his design of abdicating the government. As he wished
to secure the obedience of Maximian, he exacted from him either a
general assurance that he would submit his actions to the
authority of his benefactor, or a particular promise that he
would descend from the throne, whenever he should receive the
advice and the example. This engagement, though it was confirmed
by the solemnity of an oath before the altar of the Capitoline
Jupiter,\footnotemark[110] would have proved a feeble restraint on the fierce
temper of Maximian, whose passion was the love of power, and who
neither desired present tranquility nor future reputation. But he
yielded, however reluctantly, to the ascendant which his wiser
colleague had acquired over him, and retired, immediately after
his abdication, to a villa in Lucania, where it was almost
impossible that such an impatient spirit could find any lasting
tranquility.

\footnotetext[109]{The difficulties as well as mistakes attending the
dates both of the year and of the day of Diocletian’s abdication
are perfectly cleared up by Tillemont, Hist. des Empereurs, tom.
iv. p 525, note 19, and by Pagi ad annum.}

\footnotetext[110]{See Panegyr. Veter. vi. 9. The oration was
pronounced after Maximian had resumed the purple.}

Diocletian, who, from a servile origin, had raised himself to the
throne, passed the nine last years of his life in a private
condition. Reason had dictated, and content seems to have
accompanied, his retreat, in which he enjoyed, for a long time,
the respect of those princes to whom he had resigned the
possession of the world.\footnotemark[111] It is seldom that minds long
exercised in business have formed any habits of conversing with
themselves, and in the loss of power they principally regret the
want of occupation. The amusements of letters and of devotion,
which afford so many resources in solitude, were incapable of
fixing the attention of Diocletian; but he had preserved, or at
least he soon recovered, a taste for the most innocent as well as
natural pleasures, and his leisure hours were sufficiently
employed in building, planting, and gardening. His answer to
Maximian is deservedly celebrated. He was solicited by that
restless old man to reassume the reins of government, and the
Imperial purple. He rejected the temptation with a smile of pity,
calmly observing, that if he could show Maximian the cabbages
which he had planted with his own hands at Salona, he should no
longer be urged to relinquish the enjoyment of happiness for the
pursuit of power.\footnotemark[112] In his conversations with his friends, he
frequently acknowledged, that of all arts, the most difficult was
the art of reigning; and he expressed himself on that favorite
topic with a degree of warmth which could be the result only of
experience. “How often,” was he accustomed to say, “is it the
interest of four or five ministers to combine together to deceive
their sovereign! Secluded from mankind by his exalted dignity,
the truth is concealed from his knowledge; he can see only with
their eyes, he hears nothing but their misrepresentations. He
confers the most important offices upon vice and weakness, and
disgraces the most virtuous and deserving among his subjects. By
such infamous arts,” added Diocletian, “the best and wisest
princes are sold to the venal corruption of their courtiers.”\footnotemark[113]
A just estimate of greatness, and the assurance of immortal fame,
improve our relish for the pleasures of retirement; but the Roman
emperor had filled too important a character in the world, to
enjoy without alloy the comforts and security of a private
condition. It was impossible that he could remain ignorant of the
troubles which afflicted the empire after his abdication. It was
impossible that he could be indifferent to their consequences.
Fear, sorrow, and discontent, sometimes pursued him into the
solitude of Salona. His tenderness, or at least his pride, was
deeply wounded by the misfortunes of his wife and daughter; and
the last moments of Diocletian were imbittered by some affronts,
which Licinius and Constantine might have spared the father of so
many emperors, and the first author of their own fortune. A
report, though of a very doubtful nature, has reached our times,
that he prudently withdrew himself from their power by a
voluntary death.\footnotemark[114]

\footnotetext[111]{Eumenius pays him a very fine compliment: “At enim
divinum illum virum, qui primus imperium et participavit et
posuit, consilii et fact isui non poenitet; nec amisisse se putat
quod sponte transcripsit. Felix beatusque vere quem vestra,
tantorum principum, colunt privatum.” Panegyr. Vet. vii. 15.}

\footnotetext[112]{We are obliged to the younger Victor for this
celebrated item. Eutropius mentions the thing in a more general
manner.}

\footnotetext[113]{Hist. August. p. 223, 224. Vopiscus had learned
this conversation from his father.}

\footnotetext[114]{The younger Victor slightly mentions the report.
But as Diocletian had disobliged a powerful and successful party,
his memory has been loaded with every crime and misfortune. It
has been affirmed that he died raving mad, that he was condemned
as a criminal by the Roman senate, \&c.}

Before we dismiss the consideration of the life and character of
Diocletian, we may, for a moment, direct our view to the place of
his retirement. Salona, a principal city of his native province
of Dalmatia, was near two hundred Roman miles (according to the
measurement of the public highways) from Aquileia and the
confines of Italy, and about two hundred and seventy from
Sirmium, the usual residence of the emperors whenever they
visited the Illyrian frontier.\footnotemark[115] A miserable village still
preserves the name of Salona; but so late as the sixteenth
century, the remains of a theatre, and a confused prospect of
broken arches and marble columns, continued to attest its ancient
splendor.\footnotemark[116] About six or seven miles from the city Diocletian
constructed a magnificent palace, and we may infer, from the
greatness of the work, how long he had meditated his design of
abdicating the empire. The choice of a spot which united all that
could contribute either to health or to luxury did not require
the partiality of a native. “The soil was dry and fertile, the
air is pure and wholesome, and, though extremely hot during the
summer months, this country seldom feels those sultry and noxious
winds to which the coasts of Istria and some parts of Italy are
exposed. The views from the palace are no less beautiful than the
soil and climate were inviting. Towards the west lies the fertile
shore that stretches along the Adriatic, in which a number of
small islands are scattered in such a manner as to give this part
of the sea the appearance of a great lake. On the north side lies
the bay, which led to the ancient city of Salona; and the country
beyond it, appearing in sight, forms a proper contrast to that
more extensive prospect of water, which the Adriatic presents
both to the south and to the east. Towards the north, the view is
terminated by high and irregular mountains, situated at a proper
distance, and in many places covered with villages, woods, and
vineyards.”\footnotemark[117]

\footnotetext[115]{See the Itiner. p. 269, 272, edit. Wessel.}

\footnotetext[116]{The Abate Fortis, in his Viaggio in Dalmazia, p.
43, (printed at Venice in the year 1774, in two small volumes in
quarto,) quotes a Ms account of the antiquities of Salona,
composed by Giambattista Giustiniani about the middle of the
xvith century.}

\footnotetext[117]{Adam’s Antiquities of Diocletian’s Palace at
Spalatro, p. 6. We may add a circumstance or two from the Abate
Fortis: the little stream of the Hyader, mentioned by Lucan,
produces most exquisite trout, which a sagacious writer, perhaps
a monk, supposes to have been one of the principal reasons that
determined Diocletian in the choice of his retirement. Fortis, p.
45. The same author (p. 38) observes, that a taste for
agriculture is reviving at Spalatro; and that an experimental
farm has lately been established near the city, by a society of
gentlemen.}

Though Constantine, from a very obvious prejudice, affects to
mention the palace of Diocletian with contempt,\footnotemark[118] yet one of
their successors, who could only see it in a neglected and
mutilated state, celebrates its magnificence in terms of the
highest admiration.\footnotemark[119] It covered an extent of ground consisting
of between nine and ten English acres. The form was quadrangular,
flanked with sixteen towers. Two of the sides were near six
hundred, and the other two near seven hundred feet in length. The
whole was constructed of a beautiful freestone, extracted from
the neighboring quarries of Trau, or Tragutium, and very little
inferior to marble itself. Four streets, intersecting each other
at right angles, divided the several parts of this great edifice,
and the approach to the principal apartment was from a very
stately entrance, which is still denominated the Golden Gate. The
approach was terminated by a \textit{peristylium} of granite columns, on
one side of which we discover the square temple of Æsculapius, on
the other the octagon temple of Jupiter. The latter of those
deities Diocletian revered as the patron of his fortunes, the
former as the protector of his health. By comparing the present
remains with the precepts of Vitruvius, the several parts of the
building, the baths, bedchamber, the \textit{atrium}, the \textit{basilica},
and the Cyzicene, Corinthian, and Egyptian halls have been
described with some degree of precision, or at least of
probability. Their forms were various, their proportions just;
but they all were attended with two imperfections, very repugnant
to our modern notions of taste and conveniency. These stately
rooms had neither windows nor chimneys. They were lighted from
the top, (for the building seems to have consisted of no more
than one story,) and they received their heat by the help of
pipes that were conveyed along the walls. The range of principal
apartments was protected towards the south-west by a portico five
hundred and seventeen feet long, which must have formed a very
noble and delightful walk, when the beauties of painting and
sculpture were added to those of the prospect.

\footnotetext[118]{Constantin. Orat. ad Coetum Sanct. c. 25. In this
sermon, the emperor, or the bishop who composed it for him,
affects to relate the miserable end of all the persecutors of the
church.}

\footnotetext[119]{Constantin. Porphyr. de Statu Imper. p. 86.}

Had this magnificent edifice remained in a solitary country, it
would have been exposed to the ravages of time; but it might,
perhaps, have escaped the rapacious industry of man. The village
of Aspalathus,\footnotemark[120] and, long afterwards, the provincial town of
Spalatro, have grown out of its ruins. The Golden Gate now opens
into the market-place. St. John the Baptist has usurped the
honors of Æsculapius; and the temple of Jupiter, under the
protection of the Virgin, is converted into the cathedral church.

For this account of Diocletian’s palace we are principally
indebted to an ingenious artist of our own time and country, whom
a very liberal curiosity carried into the heart of Dalmatia.\footnotemark[121]
But there is room to suspect that the elegance of his designs and
engraving has somewhat flattered the objects which it was their
purpose to represent. We are informed by a more recent and very
judicious traveller, that the awful ruins of Spalatro are not
less expressive of the decline of the art than of the greatness
of the Roman empire in the time of Diocletian.\footnotemark[122] If such was
indeed the state of architecture, we must naturally believe that
painting and sculpture had experienced a still more sensible
decay. The practice of architecture is directed by a few general
and even mechanical rules. But sculpture, and, above all,
painting, propose to themselves the imitation not only of the
forms of nature, but of the characters and passions of the human
soul. In those sublime arts the dexterity of the hand is of
little avail, unless it is animated by fancy, and guided by the
most correct taste and observation.

\footnotetext[120]{D’Anville, Geographie Ancienne, tom. i. p. 162.}

\footnotetext[121]{Messieurs Adam and Clerisseau, attended by two
draughtsmen visited Spalatro in the month of July, 1757. The
magnificent work which their journey produced was published in
London seven years afterwards.}

\footnotetext[122]{I shall quote the words of the Abate Fortis.
“E’bastevolmente agli amatori dell’ Architettura, e dell’
Antichita, l’opera del Signor Adams, che a donato molto a que’
superbi vestigi coll’abituale eleganza del suo toccalapis e del
bulino. In generale la rozzezza del scalpello, e’l cattivo gusto
del secolo vi gareggiano colla magnificenz del fabricato.” See
Viaggio in Dalmazia, p. 40.}

It is almost unnecessary to remark, that the civil distractions
of the empire, the license of the soldiers, the inroads of the
barbarians, and the progress of despotism, had proved very
unfavorable to genius, and even to learning. The succession of
Illyrian princes restored the empire without restoring the
sciences. Their military education was not calculated to inspire
them with the love of letters; and even the mind of Diocletian,
however active and capacious in business, was totally uninformed
by study or speculation. The professions of law and physic are of
such common use and certain profit that they will always secure a
sufficient number of practitioners endowed with a reasonable
degree of abilities and knowledge; but it does not appear that
the students in those two faculties appeal to any celebrated
masters who have flourished within that period. The voice of
poetry was silent. History was reduced to dry and confused
abridgments, alike destitute of amusement and instruction. A
languid and affected eloquence was still retained in the pay and
service of the emperors, who encouraged not any arts except those
which contributed to the gratification of their pride, or the
defence of their power.\footnotemark[123]

\footnotetext[123]{The orator Eumenius was secretary to the emperors
Maximian and Constantius, and Professor of Rhetoric in the
college of Autun. His salary was six hundred thousand sesterces,
which, according to the lowest computation of that age, must have
exceeded three thousand pounds a year. He generously requested
the permission of employing it in rebuilding the college. See his
Oration De Restaurandis Scholis; which, though not exempt from
vanity, may atone for his panegyrics.}

The declining age of learning and of mankind is marked, however,
by the rise and rapid progress of the new Platonists. The school
of Alexandria silenced those of Athens; and the ancient sects
enrolled themselves under the banners of the more fashionable
teachers, who recommended their system by the novelty of their
method, and the austerity of their manners. Several of these
masters, Ammonius, Plotinus, Amelius, and Porphyry,\footnotemark[124] were men
of profound thought and intense application; but by mistaking the
true object of philosophy, their labors contributed much less to
improve than to corrupt the human understanding. The knowledge
that is suited to our situation and powers, the whole compass of
moral, natural, and mathematical science, was neglected by the
new Platonists; whilst they exhausted their strength in the
verbal disputes of metaphysics, attempted to explore the secrets
of the invisible world, and studied to reconcile Aristotle with
Plato, on subjects of which both these philosophers were as
ignorant as the rest of mankind. Consuming their reason in these
deep but unsubstantial meditations, their minds were exposed to
illusions of fancy. They flattered themselves that they possessed
the secret of disengaging the soul from its corporal prison;
claimed a familiar intercourse with demons and spirits; and, by a
very singular revolution, converted the study of philosophy into
that of magic. The ancient sages had derided the popular
superstition; after disguising its extravagance by the thin
pretence of allegory, the disciples of Plotinus and Porphyry
became its most zealous defenders. As they agreed with the
Christians in a few mysterious points of faith, they attacked the
remainder of their theological system with all the fury of civil
war. The new Platonists would scarcely deserve a place in the
history of science, but in that of the church the mention of them
will very frequently occur.

\footnotetext[124]{Porphyry died about the time of Diocletian’s
abdication. The life of his master Plotinus, which he composed,
will give us the most complete idea of the genius of the sect,
and the manners of its professors. This very curious piece is
inserted in Fabricius Bibliotheca Græca tom. iv. p. 88—148.}

