\section{Part \thesection.}
\thispagestyle{simple}

Pestilence and famine contributed to fill up the measure of the
calamities of Rome.\footnotemark[25] The first could be only imputed to the
just indignation of the gods; but a monopoly of corn, supported
by the riches and power of the minister, was considered as the
immediate cause of the second. The popular discontent, after it
had long circulated in whispers, broke out in the assembled
circus. The people quitted their favorite amusements for the more
delicious pleasure of revenge, rushed in crowds towards a palace
in the suburbs, one of the emperor’s retirements, and demanded,
with angry clamors, the head of the public enemy. Cleander, who
commanded the Prætorian guards,\footnotemark[26] ordered a body of cavalry to
sally forth, and disperse the seditious multitude. The multitude
fled with precipitation towards the city; several were slain, and
many more were trampled to death; but when the cavalry entered
the streets, their pursuit was checked by a shower of stones and
darts from the roofs and windows of the houses. The foot guards,\footnotemark[27]
who had been long jealous of the prerogatives and insolence of
the Prætorian cavalry, embraced the party of the people. The
tumult became a regular engagement, and threatened a general
massacre. The Prætorians, at length, gave way, oppressed with
numbers; and the tide of popular fury returned with redoubled
violence against the gates of the palace, where Commodus lay,
dissolved in luxury, and alone unconscious of the civil war. It
was death to approach his person with the unwelcome news. He
would have perished in this supine security, had not two women,
his eldest sister Fadilla, and Marcia, the most favored of his
concubines, ventured to break into his presence. Bathed in tears,
and with dishevelled hair, they threw themselves at his feet; and
with all the pressing eloquence of fear, discovered to the
affrighted emperor the crimes of the minister, the rage of the
people, and the impending ruin, which, in a few minutes, would
burst over his palace and person. Commodus started from his dream
of pleasure, and commanded that the head of Cleander should be
thrown out to the people. The desired spectacle instantly
appeased the tumult; and the son of Marcus might even yet have
regained the affection and confidence of his subjects.\footnotemark[28]

\footnotetext[25]{Herodian, l. i. p. 28. Dion, l. lxxii. p. 1215. The
latter says that two thousand persons died every day at Rome,
during a considerable length of time.}

\footnotetext[26]{Tuneque primum tres præfecti prætorio fuere: inter
quos libertinus. From some remains of modesty, Cleander declined
the title, whilst he assumed the powers, of Prætorian præfect. As
the other freedmen were styled, from their several departments, a
rationibus, ab epistolis, Cleander called himself a pugione, as
intrusted with the defence of his master’s person. Salmasius and
Casaubon seem to have talked very idly upon this passage. * Note:
M. Guizot denies that Lampridius means Cleander as præfect a
pugione. The Libertinus seems to me to mean him.—M.}

\footnotetext[27]{Herodian, l. i. p. 31. It is doubtful whether he
means the Prætorian infantry, or the cohortes urbanæ, a body of
six thousand men, but whose rank and discipline were not equal to
their numbers. Neither Tillemont nor Wotton choose to decide this
question.}

\footnotetext[28]{Dion Cassius, l. lxxii. p. 1215. Herodian, l. i. p.
32. Hist. August. p. 48.}

But every sentiment of virtue and humanity was extinct in the
mind of Commodus. Whilst he thus abandoned the reins of empire to
these unworthy favorites, he valued nothing in sovereign power,
except the unbounded license of indulging his sensual appetites.
His hours were spent in a seraglio of three hundred beautiful
women, and as many boys, of every rank, and of every province;
and, wherever the arts of seduction proved ineffectual, the
brutal lover had recourse to violence. The ancient historians\footnotemark[29]
have expatiated on these abandoned scenes of prostitution, which
scorned every restraint of nature or modesty; but it would not be
easy to translate their too faithful descriptions into the
decency of modern language. The intervals of lust were filled up
with the basest amusements. The influence of a polite age, and
the labor of an attentive education, had never been able to
infuse into his rude and brutish mind the least tincture of
learning; and he was the first of the Roman emperors totally
devoid of taste for the pleasures of the understanding. Nero
himself excelled, or affected to excel, in the elegant arts of
music and poetry: nor should we despise his pursuits, had he not
converted the pleasing relaxation of a leisure hour into the
serious business and ambition of his life. But Commodus, from his
earliest infancy, discovered an aversion to whatever was rational
or liberal, and a fond attachment to the amusements of the
populace; the sports of the circus and amphitheatre, the combats
of gladiators, and the hunting of wild beasts. The masters in
every branch of learning, whom Marcus provided for his son, were
heard with inattention and disgust; whilst the Moors and
Parthians, who taught him to dart the javelin and to shoot with
the bow, found a disciple who delighted in his application, and
soon equalled the most skilful of his instructors in the
steadiness of the eye and the dexterity of the hand.

\footnotetext[29]{Sororibus suis constupratis. Ipsas concubinas suas
sub oculis...stuprari jubebat. Nec irruentium in se juvenum
carebat infamia, omni parte corporis atque ore in sexum utrumque
pollutus. Hist. Aug. p. 47.}

The servile crowd, whose fortune depended on their master’s
vices, applauded these ignoble pursuits. The perfidious voice of
flattery reminded him, that by exploits of the same nature, by
the defeat of the Nemæan lion, and the slaughter of the wild boar
of Erymanthus, the Grecian Hercules had acquired a place among
the gods, and an immortal memory among men. They only forgot to
observe, that, in the first ages of society, when the fiercer
animals often dispute with man the possession of an unsettled
country, a successful war against those savages is one of the
most innocent and beneficial labors of heroism. In the civilized
state of the Roman empire, the wild beasts had long since retired
from the face of man, and the neighborhood of populous cities. To
surprise them in their solitary haunts, and to transport them to
Rome, that they might be slain in pomp by the hand of an emperor,
was an enterprise equally ridiculous for the prince and
oppressive for the people.\footnotemark[30] Ignorant of these distinctions,
Commodus eagerly embraced the glorious resemblance, and styled
himself (as we still read on his medals\footnotemark[31])
the \textit{Roman Hercules}.\footnotemark[311]
The club and the lion’s hide were placed by the side of the
throne, amongst the ensigns of sovereignty; and statues were
erected, in which Commodus was represented in the character, and
with the attributes, of the god, whose valor and dexterity he
endeavored to emulate in the daily course of his ferocious
amusements.\footnotemark[32]

\footnotetext[30]{The African lions, when pressed by hunger, infested
the open villages and cultivated country; and they infested them
with impunity. The royal beast was reserved for the pleasures of
the emperor and the capital; and the unfortunate peasant who
killed one of them though in his own defence, incurred a very
heavy penalty. This extraordinary game-law was mitigated by
Honorius, and finally repealed by Justinian. Codex Theodos. tom.
v. p. 92, et Comment Gothofred.}

\footnotetext[31]{Spanheim de Numismat. Dissertat. xii. tom. ii. p.
493.}

\footnotetext[311]{Commodus placed his own head on the colossal
statue of Hercules with the inscription, Lucius Commodus
Hercules. The wits of Rome, according to a new fragment of Dion,
published an epigram, of which, like many other ancient jests,
the point is not very clear. It seems to be a protest of the god
against being confounded with the emperor. Mai Fragm. Vatican.
ii. 225.—M.}

\footnotetext[32]{Dion, l. lxxii. p. 1216. Hist. August. p. 49.}

Elated with these praises, which gradually extinguished the
innate sense of shame, Commodus resolved to exhibit before the
eyes of the Roman people those exercises, which till then he had
decently confined within the walls of his palace, and to the
presence of a few favorites. On the appointed day, the various
motives of flattery, fear, and curiosity, attracted to the
amphitheatre an innumerable multitude of spectators; and some
degree of applause was deservedly bestowed on the uncommon skill
of the Imperial performer. Whether he aimed at the head or heart
of the animal, the wound was alike certain and mortal. With
arrows whose point was shaped into the form of crescent, Commodus
often intercepted the rapid career, and cut asunder the long,
bony neck of the ostrich.\footnotemark[33] A panther was let loose; and the
archer waited till he had leaped upon a trembling malefactor. In
the same instant the shaft flew, the beast dropped dead, and the
man remained unhurt. The dens of the amphitheatre disgorged at
once a hundred lions: a hundred darts from the unerring hand of
Commodus laid them dead as they run raging round the \textit{Arena}.
Neither the huge bulk of the elephant, nor the scaly hide of the
rhinoceros, could defend them from his stroke. Æthiopia and India
yielded their most extraordinary productions; and several animals
were slain in the amphitheatre, which had been seen only in the
representations of art, or perhaps of fancy.\footnotemark[34] In all these
exhibitions, the securest precautions were used to protect the
person of the Roman Hercules from the desperate spring of any
savage, who might possibly disregard the dignity of the emperor
and the sanctity of the god.\footnotemark[35]

\footnotetext[33]{The ostrich’s neck is three feet long, and composed
of seventeen vertebræ. See Buffon, Hist. Naturelle.}

\footnotetext[34]{Commodus killed a camelopardalis or Giraffe, (Dion,
l. lxxii. p. 1211,) the tallest, the most gentle, and the most
useless of the large quadrupeds. This singular animal, a native
only of the interior parts of Africa, has not been seen in Europe
since the revival of letters; and though M. de Buffon (Hist.
Naturelle, tom. xiii.) has endeavored to describe, he has not
ventured to delineate, the Giraffe. * Note: The naturalists of
our days have been more fortunate. London probably now contains
more specimens of this animal than have been seen in Europe since
the fall of the Roman empire, unless in the pleasure gardens of
the emperor Frederic II., in Sicily, which possessed several.
Frederic’s collections of wild beasts were exhibited, for the
popular amusement, in many parts of Italy. Raumer, Geschichte der
Hohenstaufen, v. iii. p. 571. Gibbon, moreover, is mistaken; as a
giraffe was presented to Lorenzo de Medici, either by the sultan
of Egypt or the king of Tunis. Contemporary authorities are
quoted in the old work, Gesner de Quadrupedibum p. 162.—M.}

\footnotetext[35]{Herodian, l. i. p. 37. Hist. August. p. 50.}

But the meanest of the populace were affected with shame and
indignation when they beheld their sovereign enter the lists as a
gladiator, and glory in a profession which the laws and manners
of the Romans had branded with the justest note of infamy.\footnotemark[36] He
chose the habit and arms of the \textit{Secutor}, whose combat with the
\textit{Retiarius} formed one of the most lively scenes in the bloody
sports of the amphitheatre. The \textit{Secutor} was armed with a
helmet, sword, and buckler; his naked antagonist had only a large
net and a trident; with the one he endeavored to entangle, with
the other to despatch his enemy. If he missed the first throw, he
was obliged to fly from the pursuit of the \textit{Secutor}, till he had
prepared his net for a second cast.\footnotemark[37] The emperor fought in this
character seven hundred and thirty-five several times. These
glorious achievements were carefully recorded in the public acts
of the empire; and that he might omit no circumstance of infamy,
he received from the common fund of gladiators a stipend so
exorbitant that it became a new and most ignominious tax upon the
Roman people.\footnotemark[38] It may be easily supposed, that in these
engagements the master of the world was always successful; in the
amphitheatre, his victories were not often sanguinary; but when
he exercised his skill in the school of gladiators, or his own
palace, his wretched antagonists were frequently honored with a
mortal wound from the hand of Commodus, and obliged to seal their
flattery with their blood.\footnotemark[39] He now disdained the appellation of
Hercules. The name of Paulus, a celebrated Secutor, was the only
one which delighted his ear. It was inscribed on his colossal
statues, and repeated in the redoubled acclamations\footnotemark[40] of the
mournful and applauding senate.\footnotemark[41] Claudius Pompeianus, the
virtuous husband of Lucilla, was the only senator who asserted
the honor of his rank. As a father, he permitted his sons to
consult their safety by attending the amphitheatre. As a Roman,
he declared, that his own life was in the emperor’s hands, but
that he would never behold the son of Marcus prostituting his
person and dignity. Notwithstanding his manly resolution
Pompeianus escaped the resentment of the tyrant, and, with his
honor, had the good fortune to preserve his life.\footnotemark[42]

\footnotetext[36]{The virtuous and even the wise princes forbade the
senators and knights to embrace this scandalous profession, under
pain of infamy, or, what was more dreaded by those profligate
wretches, of exile. The tyrants allured them to dishonor by
threats and rewards. Nero once produced in the arena forty
senators and sixty knights. See Lipsius, Saturnalia, l. ii. c. 2.
He has happily corrected a passage of Suetonius in Nerone, c.
12.}

\footnotetext[37]{Lipsius, l. ii. c. 7, 8. Juvenal, in the eighth
satire, gives a picturesque description of this combat.}

\footnotetext[38]{Hist. August. p. 50. Dion, l. lxxii. p. 1220. He
received, for each time, decies, about 8000l. sterling.}

\footnotetext[39]{Victor tells us, that Commodus only allowed his
antagonists a...weapon, dreading most probably the consequences
of their despair.}

\footnotetext[40]{Footnote 40: They were obliged to repeat, six
hundred and twenty-six times, Paolus first of the Secutors, \&c.}

\footnotetext[41]{Dion, l. lxxii. p. 1221. He speaks of his own
baseness and danger.}

\footnotetext[42]{He mixed, however, some prudence with his courage,
and passed the greatest part of his time in a country retirement;
alleging his advanced age, and the weakness of his eyes. “I never
saw him in the senate,” says Dion, “except during the short reign
of Pertinax.” All his infirmities had suddenly left him, and they
returned as suddenly upon the murder of that excellent prince.
Dion, l. lxxiii. p. 1227.}

Commodus had now attained the summit of vice and infamy. Amidst
the acclamations of a flattering court, he was unable to disguise
from himself, that he had deserved the contempt and hatred of
every man of sense and virtue in his empire. His ferocious spirit
was irritated by the consciousness of that hatred, by the envy of
every kind of merit, by the just apprehension of danger, and by
the habit of slaughter, which he contracted in his daily
amusements. History has preserved a long list of consular
senators sacrificed to his wanton suspicion, which sought out,
with peculiar anxiety, those unfortunate persons connected,
however remotely, with the family of the Antonines, without
sparing even the ministers of his crimes or pleasures.\footnotemark[43] His
cruelty proved at last fatal to himself. He had shed with
impunity the noblest blood of Rome: he perished as soon as he was
dreaded by his own domestics. Marcia, his favorite concubine,
Eclectus, his chamberlain, and Lætus, his Prætorian præfect,
alarmed by the fate of their companions and predecessors,
resolved to prevent the destruction which every hour hung over
their heads, either from the mad caprice of the tyrant,\footnotemark[431] or
the sudden indignation of the people. Marcia seized the occasion
of presenting a draught of wine to her lover, after he had
fatigued himself with hunting some wild beasts. Commodus retired
to sleep; but whilst he was laboring with the effects of poison
and drunkenness, a robust youth, by profession a wrestler,
entered his chamber, and strangled him without resistance. The
body was secretly conveyed out of the palace, before the least
suspicion was entertained in the city, or even in the court, of
the emperor’s death. Such was the fate of the son of Marcus, and
so easy was it to destroy a hated tyrant, who, by the artificial
powers of government, had oppressed, during thirteen years, so
many millions of subjects, each of whom was equal to their master
in personal strength and personal abilities.\footnotemark[44]

\footnotetext[43]{The prefects were changed almost hourly or daily;
and the caprice of Commodus was often fatal to his most favored
chamberlains. Hist. August. p. 46, 51.}

\footnotetext[431]{Commodus had already resolved to massacre them the
following night they determined o anticipate his design. Herod.
i. 17.—W.}

\footnotetext[44]{Dion, l. lxxii. p. 1222. Herodian, l. i. p. 43.
Hist. August. p. 52.}

The measures of the conspirators were conducted with the
deliberate coolness and celerity which the greatness of the
occasion required. They resolved instantly to fill the vacant
throne with an emperor whose character would justify and maintain
the action that had been committed. They fixed on Pertinax,
præfect of the city, an ancient senator of consular rank, whose
conspicuous merit had broke through the obscurity of his birth,
and raised him to the first honors of the state. He had
successively governed most of the provinces of the empire; and in
all his great employments, military as well as civil, he had
uniformly distinguished himself by the firmness, the prudence,
and the integrity of his conduct.\footnotemark[45] He now remained almost alone
of the friends and ministers of Marcus; and when, at a late hour
of the night, he was awakened with the news, that the chamberlain
and the præfect were at his door, he received them with intrepid
resignation, and desired they would execute their master’s
orders. Instead of death, they offered him the throne of the
Roman world. During some moments he distrusted their intentions
and assurances. Convinced at length of the death of Commodus, he
accepted the purple with a sincere reluctance, the natural effect
of his knowledge both of the duties and of the dangers of the
supreme rank.\footnotemark[46]

\footnotetext[45]{Pertinax was a native of Alba Pompeia, in Piedmont,
and son of a timber merchant. The order of his employments (it is
marked by Capitolinus) well deserves to be set down, as
expressive of the form of government and manners of the age. 1.
He was a centurion. 2. Præfect of a cohort in Syria, in the
Parthian war, and in Britain. 3. He obtained an Ala, or squadron
of horse, in Mæsia. 4. He was commissary of provisions on the
Æmilian way. 5. He commanded the fleet upon the Rhine. 6. He was
procurator of Dacia, with a salary of about 1600l. a year. 7. He
commanded the veterans of a legion. 8. He obtained the rank of
senator. 9. Of prætor. 10. With the command of the first legion
in Rhætia and Noricum. 11. He was consul about the year 175. 12.
He attended Marcus into the East. 13. He commanded an army on the
Danube. 14. He was consular legate of Mæsia. 15. Of Dacia. 16. Of
Syria. 17. Of Britain. 18. He had the care of the public
provisions at Rome. 19. He was proconsul of Africa. 20. Præfect
of the city. Herodian (l. i. p. 48) does justice to his
disinterested spirit; but Capitolinus, who collected every
popular rumor, charges him with a great fortune acquired by
bribery and corruption.}

\footnotetext[46]{Julian, in the Cæsars, taxes him with being
accessory to the death of Commodus.}

Lætus conducted without delay his new emperor to the camp of the
Prætorians, diffusing at the same time through the city a
seasonable report that Commodus died suddenly of an apoplexy; and
that the virtuous Pertinax had \textit{already} succeeded to the throne.
The guards were rather surprised than pleased with the suspicious
death of a prince, whose indulgence and liberality they alone had
experienced; but the emergency of the occasion, the authority of
their præfect, the reputation of Pertinax, and the clamors of the
people, obliged them to stifle their secret discontents, to
accept the donative promised by the new emperor, to swear
allegiance to him, and with joyful acclamations and laurels in
their hands to conduct him to the senate house, that the military
consent might be ratified by the civil authority. This important
night was now far spent; with the dawn of day, and the
commencement of the new year, the senators expected a summons to
attend an ignominious ceremony.\footnotemark[461] In spite of all
remonstrances, even of those of his creatures who yet preserved
any regard for prudence or decency, Commodus had resolved to pass
the night in the gladiators’ school, and from thence to take
possession of the consulship, in the habit and with the
attendance of that infamous crew. On a sudden, before the break
of day, the senate was called together in the temple of Concord,
to meet the guards, and to ratify the election of a new emperor.
For a few minutes they sat in silent suspense, doubtful of their
unexpected deliverance, and suspicious of the cruel artifices of
Commodus: but when at length they were assured that the tyrant
was no more, they resigned themselves to all the transports of
joy and indignation. Pertinax, who modestly represented the
meanness of his extraction, and pointed out several noble
senators more deserving than himself of the empire, was
constrained by their dutiful violence to ascend the throne, and
received all the titles of Imperial power, confirmed by the most
sincere vows of fidelity. The memory of Commodus was branded with
eternal infamy. The names of tyrant, of gladiator, of public
enemy resounded in every corner of the house. They decreed in
tumultuous votes,\footnotemark[462] that his honors should be reversed, his
titles erased from the public monuments, his statues thrown down,
his body dragged with a hook into the stripping room of the
gladiators, to satiate the public fury; and they expressed some
indignation against those officious servants who had already
presumed to screen his remains from the justice of the senate.
But Pertinax could not refuse those last rites to the memory of
Marcus, and the tears of his first protector Claudius Pompeianus,
who lamented the cruel fate of his brother-in-law, and lamented
still more that he had deserved it.\footnotemark[47]

\footnotetext[461]{The senate always assembled at the beginning of
the year, on the night of the 1st January, (see Savaron on Sid.
Apoll. viii. 6,) and this happened the present year, as usual,
without any particular order.—G from W.}

\footnotetext[462]{What Gibbon improperly calls, both here and in the
note, tumultuous decrees, were no more than the applauses and
acclamations which recur so often in the history of the emperors.
The custom passed from the theatre to the forum, from the forum
to the senate. Applauses on the adoption of the Imperial decrees
were first introduced under Trajan. (Plin. jun. Panegyr. 75.) One
senator read the form of the decree, and all the rest answered by
acclamations, accompanied with a kind of chant or rhythm. These
were some of the acclamations addressed to Pertinax, and against
the memory of Commodus. Hosti patriæ honores detrahantur.
Parricidæ honores detrahantur. Ut salvi simus, Jupiter, optime,
maxime, serva nobis Pertinacem. This custom prevailed not only in
the councils of state, but in all the meetings of the senate.
However inconsistent it may appear with the solemnity of a
religious assembly, the early Christians adopted and introduced
it into their synods, notwithstanding the opposition of some of
the Fathers, particularly of St. Chrysostom. See the Coll. of
Franc. Bern. Ferrarius de veterum acclamatione in Grævii Thesaur.
Antiq. Rom. i. 6.—W. This note is rather hypercritical, as
regards Gibbon, but appears to be worthy of preservation.—M.}

\footnotetext[47]{Capitolinus gives us the particulars of these
tumultuary votes, which were moved by one senator, and repeated,
or rather chanted by the whole body. Hist. August. p. 52.}

These effusions of impotent rage against a dead emperor, whom the
senate had flattered when alive with the most abject servility,
betrayed a just but ungenerous spirit of revenge.

The legality of these decrees was, however, supported by the
principles of the Imperial constitution. To censure, to depose,
or to punish with death, the first magistrate of the republic,
who had abused his delegated trust, was the ancient and undoubted
prerogative of the Roman senate;\footnotemark[48] but the feeble assembly was
obliged to content itself with inflicting on a fallen tyrant that
public justice, from which, during his life and reign, he had
been shielded by the strong arm of military despotism.\footnotemark[481]

\footnotetext[48]{The senate condemned Nero to be put to death more
majorum. Sueton. c. 49.}

\footnotetext[481]{No particular law assigned this right to the
senate: it was deduced from the ancient principles of the
republic. Gibbon appears to infer, from the passage of Suetonius,
that the senate, according to its ancient right, punished Nero
with death. The words, however, more majerum refer not to the
decree of the senate, but to the kind of death, which was taken
from an old law of Romulus. (See Victor. Epit. Ed. Artzen p. 484,
n. 7.)—W.}

Pertinax found a nobler way of condemning his predecessor’s
memory; by the contrast of his own virtues with the vices of
Commodus. On the day of his accession, he resigned over to his
wife and son his whole private fortune; that they might have no
pretence to solicit favors at the expense of the state. He
refused to flatter the vanity of the former with the title of
Augusta; or to corrupt the inexperienced youth of the latter by
the rank of Cæsar. Accurately distinguishing between the duties
of a parent and those of a sovereign, he educated his son with a
severe simplicity, which, while it gave him no assured prospect
of the throne, might in time have rendered him worthy of it. In
public, the behavior of Pertinax was grave and affable. He lived
with the virtuous part of the senate, (and, in a private station,
he had been acquainted with the true character of each
individual,) without either pride or jealousy; considered them as
friends and companions, with whom he had shared the danger of the
tyranny, and with whom he wished to enjoy the security of the
present time. He very frequently invited them to familiar
entertainments, the frugality of which was ridiculed by those who
remembered and regretted the luxurious prodigality of Commodus.
49

\footnotetext[49]{Dion (l. lxxiii. p. 1223) speaks of these
entertainments, as a senator who had supped with the emperor;
Capitolinus, (Hist. August. p. 58,) like a slave, who had
received his intelligence from one the scullions.}

To heal, as far as it was possible, the wounds inflicted by the
hand of tyranny, was the pleasing, but melancholy, task of
Pertinax. The innocent victims, who yet survived, were recalled
from exile, released from prison, and restored to the full
possession of their honors and fortunes. The unburied bodies of
murdered senators (for the cruelty of Commodus endeavored to
extend itself beyond death) were deposited in the sepulchres of
their ancestors; their memory was justified and every consolation
was bestowed on their ruined and afflicted families. Among these
consolations, one of the most grateful was the punishment of the
Delators; the common enemies of their master, of virtue, and of
their country. Yet even in the inquisition of these legal
assassins, Pertinax proceeded with a steady temper, which gave
every thing to justice, and nothing to popular prejudice and
resentment.

The finances of the state demanded the most vigilant care of the
emperor. Though every measure of injustice and extortion had been
adopted, which could collect the property of the subject into the
coffers of the prince, the rapaciousness of Commodus had been so
very inadequate to his extravagance, that, upon his death, no
more than eight thousand pounds were found in the exhausted
treasury,\footnotemark[50] to defray the current expenses of government, and to
discharge the pressing demand of a liberal donative, which the
new emperor had been obliged to promise to the Prætorian guards.
Yet under these distressed circumstances, Pertinax had the
generous firmness to remit all the oppressive taxes invented by
Commodus, and to cancel all the unjust claims of the treasury;
declaring, in a decree of the senate, “that he was better
satisfied to administer a poor republic with innocence, than to
acquire riches by the ways of tyranny and dishonor.” Economy and
industry he considered as the pure and genuine sources of wealth;
and from them he soon derived a copious supply for the public
necessities. The expense of the household was immediately reduced
to one half. All the instruments of luxury Pertinax exposed to
public auction,\footnotemark[51] gold and silver plate, chariots of a singular
construction, a superfluous wardrobe of silk and embroidery, and
a great number of beautiful slaves of both sexes; excepting only,
with attentive humanity, those who were born in a state of
freedom, and had been ravished from the arms of their weeping
parents. At the same time that he obliged the worthless favorites
of the tyrant to resign a part of their ill-gotten wealth, he
satisfied the just creditors of the state, and unexpectedly
discharged the long arrears of honest services. He removed the
oppressive restrictions which had been laid upon commerce, and
granted all the uncultivated lands in Italy and the provinces to
those who would improve them; with an exemption from tribute
during the term of ten years.\footnotemark[52]

\footnotetext[50]{Decies. The blameless economy of Pius left his
successors a treasure of vicies septies millies, above two and
twenty millions sterling. Dion, l. lxxiii. p. 1231.}

\footnotetext[51]{Besides the design of converting these useless
ornaments into money, Dion (l. lxxiii. p. 1229) assigns two
secret motives of Pertinax. He wished to expose the vices of
Commodus, and to discover by the purchasers those who most
resembled him.}

\footnotetext[52]{Though Capitolinus has picked up many idle tales of
the private life of Pertinax, he joins with Dion and Herodian in
admiring his public conduct.}

Such a uniform conduct had already secured to Pertinax the
noblest reward of a sovereign, the love and esteem of his people.

Those who remembered the virtues of Marcus were happy to
contemplate in their new emperor the features of that bright
original; and flattered themselves, that they should long enjoy
the benign influence of his administration. A hasty zeal to
reform the corrupted state, accompanied with less prudence than
might have been expected from the years and experience of
Pertinax, proved fatal to himself and to his country. His honest
indiscretion united against him the servile crowd, who found
their private benefit in the public disorders, and who preferred
the favor of a tyrant to the inexorable equality of the laws.\footnotemark[53]

\footnotetext[53]{Leges, rem surdam, inexorabilem esse. T. Liv. ii.
3.}

Amidst the general joy, the sullen and angry countenance of the
Prætorian guards betrayed their inward dissatisfaction. They had
reluctantly submitted to Pertinax; they dreaded the strictness of
the ancient discipline, which he was preparing to restore; and
they regretted the license of the former reign. Their discontents
were secretly fomented by Lætus, their præfect, who found, when
it was too late, that his new emperor would reward a servant, but
would not be ruled by a favorite. On the third day of his reign,
the soldiers seized on a noble senator, with a design to carry
him to the camp, and to invest him with the Imperial purple.
Instead of being dazzled by the dangerous honor, the affrighted
victim escaped from their violence, and took refuge at the feet
of Pertinax. A short time afterwards, Sosius Falco, one of the
consuls of the year, a rash youth,\footnotemark[54] but of an ancient and
opulent family, listened to the voice of ambition; and a
conspiracy was formed during a short absence of Pertinax, which
was crushed by his sudden return to Rome, and his resolute
behavior. Falco was on the point of being justly condemned to
death as a public enemy had he not been saved by the earnest and
sincere entreaties of the injured emperor, who conjured the
senate, that the purity of his reign might not be stained by the
blood even of a guilty senator.

\footnotetext[54]{If we credit Capitolinus, (which is rather
difficult,) Falco behaved with the most petulant indecency to
Pertinax, on the day of his accession. The wise emperor only
admonished him of his youth and in experience. Hist. August. p.
55.}

These disappointments served only to irritate the rage of the
Prætorian guards. On the twenty-eighth of March, eighty-six days
only after the death of Commodus, a general sedition broke out in
the camp, which the officers wanted either power or inclination
to suppress. Two or three hundred of the most desperate soldiers
marched at noonday, with arms in their hands and fury in their
looks, towards the Imperial palace. The gates were thrown open by
their companions upon guard, and by the domestics of the old
court, who had already formed a secret conspiracy against the
life of the too virtuous emperor. On the news of their approach,
Pertinax, disdaining either flight or concealment, advanced to
meet his assassins; and recalled to their minds his own
innocence, and the sanctity of their recent oath. For a few
moments they stood in silent suspense, ashamed of their atrocious
design, and awed by the venerable aspect and majestic firmness of
their sovereign, till at length, the despair of pardon reviving
their fury, a barbarian of the country of Tongress\footnotemark[55] levelled
the first blow against Pertinax, who was instantly despatched
with a multitude of wounds. His head, separated from his body,
and placed on a lance, was carried in triumph to the Prætorian
camp, in the sight of a mournful and indignant people, who
lamented the unworthy fate of that excellent prince, and the
transient blessings of a reign, the memory of which could serve
only to aggravate their approaching misfortunes.\footnotemark[56]

\footnotetext[55]{The modern bishopric of Liege. This soldier
probably belonged to the Batavian horse-guards, who were mostly
raised in the duchy of Gueldres and the neighborhood, and were
distinguished by their valor, and by the boldness with which they
swam their horses across the broadest and most rapid rivers.
Tacit. Hist. iv. 12 Dion, l. lv p. 797 Lipsius de magnitudine
Romana, l. i. c. 4.}

\footnotetext[56]{Dion, l. lxxiii. p. 1232. Herodian, l. ii. p. 60.
Hist. August. p. 58. Victor in Epitom. et in Cæsarib. Eutropius,
viii. 16.}

