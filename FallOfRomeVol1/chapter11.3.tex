\section{Part \thesection.}
\thispagestyle{simple}

After a successful expedition against the Gothic plunderers of
Asia, the Palmyrenian prince returned to the city of Emesa in
Syria. Invincible in war, he was there cut off by domestic
treason, and his favorite amusement of hunting was the cause, or
at least the occasion, of his death.\footnotemark[56] His nephew Mæonius
presumed to dart his javelin before that of his uncle; and though
admonished of his error, repeated the same insolence. As a
monarch, and as a sportsman, Odenathus was provoked, took away
his horse, a mark of ignominy among the barbarians, and chastised
the rash youth by a short confinement. The offence was soon
forgot, but the punishment was remembered; and Mæonius, with a
few daring associates, assassinated his uncle in the midst of a
great entertainment. Herod, the son of Odenathus, though not of
Zenobia, a young man of a soft and effeminate temper,\footnotemark[57] was
killed with his father. But Mæonius obtained only the pleasure of
revenge by this bloody deed. He had scarcely time to assume the
title of Augustus, before he was sacrificed by Zenobia to the
memory of her husband.\footnotemark[58]

\footnotetext[56]{Hist. August. p. 192, 193. Zosimus, l. i. p. 36.
Zonaras, l. xii p. 633. The last is clear and probable, the
others confused and inconsistent. The text of Syncellus, if not
corrupt, is absolute nonsense.}

\footnotetext[57]{Odenathus and Zenobia often sent him, from the
spoils of the enemy, presents of gems and toys, which he received
with infinite delight.}

\footnotetext[58]{Some very unjust suspicions have been cast on
Zenobia, as if she was accessory to her husband’s death.}

With the assistance of his most faithful friends, she immediately
filled the vacant throne, and governed with manly counsels
Palmyra, Syria, and the East, above five years. By the death of
Odenathus, that authority was at an end which the senate had
granted him only as a personal distinction; but his martial
widow, disdaining both the senate and Gallienus, obliged one of
the Roman generals, who was sent against her, to retreat into
Europe, with the loss of his army and his reputation.\footnotemark[59] Instead
of the little passions which so frequently perplex a female
reign, the steady administration of Zenobia was guided by the
most judicious maxims of policy. If it was expedient to pardon,
she could calm her resentment; if it was necessary to punish, she
could impose silence on the voice of pity. Her strict economy was
accused of avarice; yet on every proper occasion she appeared
magnificent and liberal. The neighboring states of Arabia,
Armenia, and Persia, dreaded her enmity, and solicited her
alliance. To the dominions of Odenathus, which extended from the
Euphrates to the frontiers of Bithynia, his widow added the
inheritance of her ancestors, the populous and fertile kingdom of
Egypt.\footnotemark[60] The emperor Claudius acknowledged her merit, and was
content, that, while \textit{he} pursued the Gothic war, \textit{she} should
assert the dignity of the empire in the East. The conduct,
however, of Zenobia was attended with some ambiguity; not is it
unlikely that she had conceived the design of erecting an
independent and hostile monarchy. She blended with the popular
manners of Roman princes the stately pomp of the courts of Asia,
and exacted from her subjects the same adoration that was paid to
the successor of Cyrus. She bestowed on her three sons\footnotemark[61] a Latin
education, and often showed them to the troops adorned with the
Imperial purple. For herself she reserved the diadem, with the
splendid but doubtful title of Queen of the East.

\footnotetext[59]{Hist. August. p. 180, 181.}

\footnotetext[60]{[See, in Hist. August. p. 198, Aurelian’s testimony
to her merit; and for the conquest of Egypt, Zosimus, l. i. p.
39, 40.] This seems very doubtful. Claudius, during all his
reign, is represented as emperor on the medals of Alexandria,
which are very numerous. If Zenobia possessed any power in Egypt,
it could only have been at the beginning of the reign of
Aurelian. The same circumstance throws great improbability on her
conquests in Galatia. Perhaps Zenobia administered Egypt in the
name of Claudius, and emboldened by the death of that prince,
subjected it to her own power.—G.}

\footnotetext[61]{Timolaus, Herennianus, and Vaballathus. It is
supposed that the two former were already dead before the war. On
the last, Aurelian bestowed a small province of Armenia, with the
title of King; several of his medals are still extant. See
Tillemont, tom. 3, p. 1190.}

When Aurelian passed over into Asia, against an adversary whose
sex alone could render her an object of contempt, his presence
restored obedience to the province of Bithynia, already shaken by
the arms and intrigues of Zenobia.\footnotemark[62] Advancing at the head of
his legions, he accepted the submission of Ancyra, and was
admitted into Tyana, after an obstinate siege, by the help of a
perfidious citizen. The generous though fierce temper of Aurelian
abandoned the traitor to the rage of the soldiers; a
superstitious reverence induced him to treat with lenity the
countrymen of Apollonius the philosopher.\footnotemark[63] Antioch was deserted
on his approach, till the emperor, by his salutary edicts,
recalled the fugitives, and granted a general pardon to all who,
from necessity rather than choice, had been engaged in the
service of the Palmyrenian Queen. The unexpected mildness of such
a conduct reconciled the minds of the Syrians, and as far as the
gates of Emesa, the wishes of the people seconded the terror of
his arms.\footnotemark[64]

\footnotetext[62]{Zosimus, l. i. p. 44.}

\footnotetext[63]{Vopiscus (in Hist. August. p. 217) gives us an
authentic letter and a doubtful vision, of Aurelian. Apollonius
of Tyana was born about the same time as Jesus Christ. His life
(that of the former) is related in so fabulous a manner by his
disciples, that we are at a loss to discover whether he was a
sage, an impostor, or a fanatic.}

\footnotetext[64]{Zosimus, l. i. p. 46.}

Zenobia would have ill deserved her reputation, had she
indolently permitted the emperor of the West to approach within a
hundred miles of her capital. The fate of the East was decided in
two great battles; so similar in almost every circumstance, that
we can scarcely distinguish them from each other, except by
observing that the first was fought near Antioch,\footnotemark[65] and the
second near Emesa.\footnotemark[66] In both the queen of Palmyra animated the
armies by her presence, and devolved the execution of her orders
on Zabdas, who had already signalized his military talents by the
conquest of Egypt. The numerous forces of Zenobia consisted for
the most part of light archers, and of heavy cavalry clothed in
complete steel. The Moorish and Illyrian horse of Aurelian were
unable to sustain the ponderous charge of their antagonists. They
fled in real or affected disorder, engaged the Palmyrenians in a
laborious pursuit, harassed them by a desultory combat, and at
length discomfited this impenetrable but unwieldy body of
cavalry. The light infantry, in the mean time, when they had
exhausted their quivers, remaining without protection against a
closer onset, exposed their naked sides to the swords of the
legions. Aurelian had chosen these veteran troops, who were
usually stationed on the Upper Danube, and whose valor had been
severely tried in the Alemannic war.\footnotemark[67] After the defeat of
Emesa, Zenobia found it impossible to collect a third army. As
far as the frontier of Egypt, the nations subject to her empire
had joined the standard of the conqueror, who detached Probus,
the bravest of his generals, to possess himself of the Egyptian
provinces. Palmyra was the last resource of the widow of
Odenathus. She retired within the walls of her capital, made
every preparation for a vigorous resistance, and declared, with
the intrepidity of a heroine, that the last moment of her reign
and of her life should be the same.

\footnotetext[65]{At a place called Immæ. Eutropius, Sextus Rufus,
and Jerome, mention only this first battle.}

\footnotetext[66]{Vopiscus (in Hist. August. p. 217) mentions only
the second.}

\footnotetext[67]{Zosimus, l. i. p. 44—48. His account of the two
battles is clear and circumstantial.}

Amid the barren deserts of Arabia, a few cultivated spots rise
like islands out of the sandy ocean. Even the name of Tadmor, or
Palmyra, by its signification in the Syriac as well as in the
Latin language, denoted the multitude of palm-trees which
afforded shade and verdure to that temperate region. The air was
pure, and the soil, watered by some invaluable springs, was
capable of producing fruits as well as corn. A place possessed of
such singular advantages, and situated at a convenient distance\footnotemark[68]
between the Gulf of Persia and the Mediterranean, was soon
frequented by the caravans which conveyed to the nations of
Europe a considerable part of the rich commodities of India.
Palmyra insensibly increased into an opulent and independent
city, and connecting the Roman and the Parthian monarchies by the
mutual benefits of commerce, was suffered to observe an humble
neutrality, till at length, after the victories of Trajan, the
little republic sunk into the bosom of Rome, and flourished more
than one hundred and fifty years in the subordinate though
honorable rank of a colony. It was during that peaceful period,
if we may judge from a few remaining inscriptions, that the
wealthy Palmyrenians constructed those temples, palaces, and
porticos of Grecian architecture, whose ruins, scattered over an
extent of several miles, have deserved the curiosity of our
travellers. The elevation of Odenathus and Zenobia appeared to
reflect new splendor on their country, and Palmyra, for a while,
stood forth the rival of Rome: but the competition was fatal, and
ages of prosperity were sacrificed to a moment of glory.\footnotemark[69]

\footnotetext[68]{It was five hundred and thirty-seven miles from
Seleucia, and two hundred and three from the nearest coast of
Syria, according to the reckoning of Pliny, who, in a few words,
(Hist. Natur. v. 21,) gives an excellent description of Palmyra.
* Note: Talmor, or Palmyra, was probably at a very early period
the connecting link between the commerce of Tyre and Babylon.
Heeren, Ideen, v. i. p. ii. p. 125. Tadmor was probably built by
Solomon as a commercial station. Hist. of Jews, v. p. 271—M.}

\footnotetext[69]{Some English travellers from Aleppo discovered the
ruins of Palmyra about the end of the last century. Our curiosity
has since been gratified in a more splendid manner by Messieurs
Wood and Dawkins. For the history of Palmyra, we may consult the
masterly dissertation of Dr. Halley in the Philosophical
Transactions: Lowthorp’s Abridgment, vol. iii. p. 518.}

In his march over the sandy desert between Emesa and Palmyra, the
emperor Aurelian was perpetually harassed by the Arabs; nor could
he always defend his army, and especially his baggage, from those
flying troops of active and daring robbers, who watched the
moment of surprise, and eluded the slow pursuit of the legions.
The siege of Palmyra was an object far more difficult and
important, and the emperor, who, with incessant vigor, pressed
the attacks in person, was himself wounded with a dart. “The
Roman people,” says Aurelian, in an original letter, “speak with
contempt of the war which I am waging against a woman. They are
ignorant both of the character and of the power of Zenobia. It is
impossible to enumerate her warlike preparations, of stones, of
arrows, and of every species of missile weapons. Every part of
the walls is provided with two or three \textit{balistæ} and artificial
fires are thrown from her military engines. The fear of
punishment has armed her with a desperate courage. Yet still I
trust in the protecting deities of Rome, who have hitherto been
favorable to all my undertakings.”\footnotemark[70] Doubtful, however, of the
protection of the gods, and of the event of the siege, Aurelian
judged it more prudent to offer terms of an advantageous
capitulation; to the queen, a splendid retreat; to the citizens,
their ancient privileges. His proposals were obstinately
rejected, and the refusal was accompanied with insult.

\footnotetext[70]{Vopiscus in Hist. August. p. 218.}

The firmness of Zenobia was supported by the hope, that in a very
short time famine would compel the Roman army to repass the
desert; and by the reasonable expectation that the kings of the
East, and particularly the Persian monarch, would arm in the
defence of their most natural ally. But fortune, and the
perseverance of Aurelian, overcame every obstacle. The death of
Sapor, which happened about this time,\footnotemark[71] distracted the councils
of Persia, and the inconsiderable succors that attempted to
relieve Palmyra were easily intercepted either by the arms or the
liberality of the emperor. From every part of Syria, a regular
succession of convoys safely arrived in the camp, which was
increased by the return of Probus with his victorious troops from
the conquest of Egypt. It was then that Zenobia resolved to fly.
She mounted the fleetest of her dromedaries,\footnotemark[72] and had already
reached the banks of the Euphrates, about sixty miles from
Palmyra, when she was overtaken by the pursuit of Aurelian’s
light horse, seized, and brought back a captive to the feet of
the emperor. Her capital soon afterwards surrendered, and was
treated with unexpected lenity. The arms, horses, and camels,
with an immense treasure of gold, silver, silk, and precious
stones, were all delivered to the conqueror, who, leaving only a
garrison of six hundred archers, returned to Emesa, and employed
some time in the distribution of rewards and punishments at the
end of so memorable a war, which restored to the obedience of
Rome those provinces that had renounced their allegiance since
the captivity of Valerian.

\footnotetext[71]{From a very doubtful chronology I have endeavored
to extract the most probable date.}

\footnotetext[72]{Hist. August. p. 218. Zosimus, l. i. p. 50. Though
the camel is a heavy beast of burden, the dromedary, which is
either of the same or of a kindred species, is used by the
natives of Asia and Africa on all occasions which require
celerity. The Arabs affirm, that he will run over as much ground
in one day as their fleetest horses can perform in eight or ten.
See Buffon, Hist. Naturelle, tom. xi. p. 222, and Shaw’s Travels
p. 167}

When the Syrian queen was brought into the presence of Aurelian,
he sternly asked her, How she had presumed to rise in arms
against the emperors of Rome! The answer of Zenobia was a prudent
mixture of respect and firmness. “Because I disdained to consider
as Roman emperors an Aureolus or a Gallienus. You alone I
acknowledge as my conqueror and my sovereign.”\footnotemark[73] But as female
fortitude is commonly artificial, so it is seldom steady or
consistent. The courage of Zenobia deserted her in the hour of
trial; she trembled at the angry clamors of the soldiers, who
called aloud for her immediate execution, forgot the generous
despair of Cleopatra, which she had proposed as her model, and
ignominiously purchased life by the sacrifice of her fame and her
friends. It was to their counsels, which governed the weakness of
her sex, that she imputed the guilt of her obstinate resistance;
it was on their heads that she directed the vengeance of the
cruel Aurelian. The fame of Longinus, who was included among the
numerous and perhaps innocent victims of her fear, will survive
that of the queen who betrayed, or the tyrant who condemned him.
Genius and learning were incapable of moving a fierce unlettered
soldier, but they had served to elevate and harmonize the soul of
Longinus. Without uttering a complaint, he calmly followed the
executioner, pitying his unhappy mistress, and bestowing comfort
on his afflicted friends.\footnotemark[74]

\footnotetext[73]{Pollio in Hist. August. p. 199.}

\footnotetext[74]{Vopiscus in Hist. August. p. 219. Zosimus, l. i. p.
51.}

Returning from the conquest of the East, Aurelian had already
crossed the Straits which divided Europe from Asia, when he was
provoked by the intelligence that the Palmyrenians had massacred
the governor and garrison which he had left among them, and again
erected the standard of revolt. Without a moment’s deliberation,
he once more turned his face towards Syria. Antioch was alarmed
by his rapid approach, and the helpless city of Palmyra felt the
irresistible weight of his resentment. We have a letter of
Aurelian himself, in which he acknowledges,\footnotemark[75] that old men,
women, children, and peasants, had been involved in that dreadful
execution, which should have been confined to armed rebellion;
and although his principal concern seems directed to the
reëstablishment of a temple of the Sun, he discovers some pity
for the remnant of the Palmyrenians, to whom he grants the
permission of rebuilding and inhabiting their city. But it is
easier to destroy than to restore. The seat of commerce, of arts,
and of Zenobia, gradually sunk into an obscure town, a trifling
fortress, and at length a miserable village. The present citizens
of Palmyra, consisting of thirty or forty families, have erected
their mud cottages within the spacious court of a magnificent
temple.

\footnotetext[75]{Hist. August. p. 219.}

Another and a last labor still awaited the indefatigable
Aurelian; to suppress a dangerous though obscure rebel, who,
during the revolt of Palmyra, had arisen on the banks of the
Nile. Firmus, the friend and ally, as he proudly styled himself,
of Odenathus and Zenobia, was no more than a wealthy merchant of
Egypt. In the course of his trade to India, he had formed very
intimate connections with the Saracens and the Blemmyes, whose
situation on either coast of the Red Sea gave them an easy
introduction into the Upper Egypt. The Egyptians he inflamed with
the hope of freedom, and, at the head of their furious multitude,
broke into the city of Alexandria, where he assumed the Imperial
purple, coined money, published edicts, and raised an army,
which, as he vainly boasted, he was capable of maintaining from
the sole profits of his paper trade. Such troops were a feeble
defence against the approach of Aurelian; and it seems almost
unnecessary to relate, that Firmus was routed, taken, tortured,
and put to death.\footnotemark[76] Aurelian might now congratulate the senate,
the people, and himself, that in little more than three years, he
had restored universal peace and order to the Roman world.

\footnotetext[76]{See Vopiscus in Hist. August. p. 220, 242. As an
instance of luxury, it is observed, that he had glass windows. He
was remarkable for his strength and appetite, his courage and
dexterity. From the letter of Aurelian, we may justly infer, that
Firmus was the last of the rebels, and consequently that Tetricus
was already suppressed.}

Since the foundation of Rome, no general had more nobly deserved
a triumph than Aurelian; nor was a triumph ever celebrated with
superior pride and magnificence.\footnotemark[77] The pomp was opened by twenty
elephants, four royal tigers, and above two hundred of the most
curious animals from every climate of the North, the East, and
the South. They were followed by sixteen hundred gladiators,
devoted to the cruel amusement of the amphitheatre. The wealth of
Asia, the arms and ensigns of so many conquered nations, and the
magnificent plate and wardrobe of the Syrian queen, were disposed
in exact symmetry or artful disorder. The ambassadors of the most
remote parts of the earth, of Æthiopia, Arabia, Persia,
Bactriana, India, and China, all remarkable by their rich or
singular dresses, displayed the fame and power of the Roman
emperor, who exposed likewise to the public view the presents
that he had received, and particularly a great number of crowns
of gold, the offerings of grateful cities.

The victories of Aurelian were attested by the long train of
captives who reluctantly attended his triumph, Goths, Vandals,
Sarmatians, Alemanni, Franks, Gauls, Syrians, and Egyptians. Each
people was distinguished by its peculiar inscription, and the
title of Amazons was bestowed on ten martial heroines of the
Gothic nation who had been taken in arms.\footnotemark[78] But every eye,
disregarding the crowd of captives, was fixed on the emperor
Tetricus and the queen of the East. The former, as well as his
son, whom he had created Augustus, was dressed in Gallic
trousers,\footnotemark[79] a saffron tunic, and a robe of purple. The beauteous
figure of Zenobia was confined by fetters of gold; a slave
supported the gold chain which encircled her neck, and she almost
fainted under the intolerable weight of jewels. She preceded on
foot the magnificent chariot, in which she once hoped to enter
the gates of Rome. It was followed by two other chariots, still
more sumptuous, of Odenathus and of the Persian monarch. The
triumphal car of Aurelian (it had formerly been used by a Gothic
king) was drawn, on this memorable occasion, either by four stags
or by four elephants.\footnotemark[80] The most illustrious of the senate, the
people, and the army, closed the solemn procession. Unfeigned
joy, wonder, and gratitude, swelled the acclamations of the
multitude; but the satisfaction of the senate was clouded by the
appearance of Tetricus; nor could they suppress a rising murmur,
that the haughty emperor should thus expose to public ignominy
the person of a Roman and a magistrate.\footnotemark[81]

\footnotetext[77]{See the triumph of Aurelian, described by Vopiscus.
He relates the particulars with his usual minuteness; and, on
this occasion, they happen to be interesting. Hist. August. p.
220.}

\footnotetext[78]{Among barbarous nations, women have often combated
by the side of their husbands. But it is almost impossible that a
society of Amazons should ever have existed either in the old or
new world. * Note: Klaproth’s theory on the origin of such
traditions is at least recommended by its ingenuity. The males of
a tribe having gone out on a marauding expedition, and having
been cut off to a man, the females may have endeavored, for a
time, to maintain their independence in their camp village, till
their children grew up. Travels, ch. xxx. Eng. Trans—M.}

\footnotetext[79]{The use of braccœ, breeches, or trousers, was
still considered in Italy as a Gallic and barbarian fashion. The
Romans, however, had made great advances towards it. To encircle
the legs and thighs with fasciœ, or bands, was understood, in
the time of Pompey and Horace, to be a proof of ill health or
effeminacy. In the age of Trajan, the custom was confined to the
rich and luxurious. It gradually was adopted by the meanest of
the people. See a very curious note of Casaubon, ad Sueton. in
August. c. 82.}

\footnotetext[80]{Most probably the former; the latter seen on the
medals of Aurelian, only denote (according to the learned
Cardinal Norris) an oriental victory.}

\footnotetext[81]{The expression of Calphurnius, (Eclog. i. 50)
Nullos decet captiva triumphos, as applied to Rome, contains a
very manifest allusion and censure.}

But however, in the treatment of his unfortunate rivals, Aurelian
might indulge his pride, he behaved towards them with a generous
clemency, which was seldom exercised by the ancient conquerors.
Princes who, without success, had defended their throne or
freedom, were frequently strangled in prison, as soon as the
triumphal pomp ascended the Capitol. These usurpers, whom their
defeat had convicted of the crime of treason, were permitted to
spend their lives in affluence and honorable repose.

The emperor presented Zenobia with an elegant villa at Tibur, or
Tivoli, about twenty miles from the capital; the Syrian queen
insensibly sunk into a Roman matron, her daughters married into
noble families, and her race was not yet extinct in the fifth
century.\footnotemark[82] Tetricus and his son were reinstated in their rank
and fortunes. They erected on the Cælian hill a magnificent
palace, and as soon as it was finished, invited Aurelian to
supper. On his entrance, he was agreeably surprised with a
picture which represented their singular history. They were
delineated offering to the emperor a civic crown and the sceptre
of Gaul, and again receiving at his hands the ornaments of the
senatorial dignity. The father was afterwards invested with the
government of Lucania,\footnotemark[83] and Aurelian, who soon admitted the
abdicated monarch to his friendship and conversation, familiarly
asked him, Whether it were not more desirable to administer a
province of Italy, than to reign beyond the Alps. The son long
continued a respectable member of the senate; nor was there any
one of the Roman nobility more esteemed by Aurelian, as well as
by his successors.\footnotemark[84]

\footnotetext[82]{Vopiscus in Hist. August. p. 199. Hieronym. in
Chron. Prosper in Chron. Baronius supposes that Zenobius, bishop
of Florence in the time of St. Ambrose, was of her family.}

\footnotetext[83]{Vopisc. in Hist. August. p. 222. Eutropius, ix. 13.
Victor Junior. But Pollio, in Hist. August. p. 196, says, that
Tetricus was made corrector of all Italy.}

\footnotetext[84]{Hist. August. p. 197.}

So long and so various was the pomp of Aurelian’s triumph, that
although it opened with the dawn of day, the slow majesty of the
procession ascended not the Capitol before the ninth hour; and it
was already dark when the emperor returned to the palace. The
festival was protracted by theatrical representations, the games
of the circus, the hunting of wild beasts, combats of gladiators,
and naval engagements. Liberal donatives were distributed to the
army and people, and several institutions, agreeable or
beneficial to the city, contributed to perpetuate the glory of
Aurelian. A considerable portion of his oriental spoils was
consecrated to the gods of Rome; the Capitol, and every other
temple, glittered with the offerings of his ostentatious piety;
and the temple of the Sun alone received above fifteen thousand
pounds of gold.\footnotemark[85] This last was a magnificent structure, erected
by the emperor on the side of the Quirinal hill, and dedicated,
soon after the triumph, to that deity whom Aurelian adored as the
parent of his life and fortunes. His mother had been an inferior
priestess in a chapel of the Sun; a peculiar devotion to the god
of Light was a sentiment which the fortunate peasant imbibed in
his infancy; and every step of his elevation, every victory of
his reign, fortified superstition by gratitude.\footnotemark[86]

\footnotetext[85]{Vopiscus in Hist. August. 222. Zosimus, l. i. p.
56. He placed in it the images of Belus and of the Sun, which he
had brought from Palmyra. It was dedicated in the fourth year of
his reign, (Euseb in Chron.,) but was most assuredly begun
immediately on his accession.}

\footnotetext[86]{See, in the Augustan History, p. 210, the omens of
his fortune. His devotion to the Sun appears in his letters, on
his medals, and is mentioned in the Cæsars of Julian. Commentaire
de Spanheim, p. 109.}

The arms of Aurelian had vanquished the foreign and domestic foes
of the republic. We are assured, that, by his salutary rigor,
crimes and factions, mischievous arts and pernicious connivance,
the luxurious growth of a feeble and oppressive government, were
eradicated throughout the Roman world.\footnotemark[87] But if we attentively
reflect how much swifter is the progress of corruption than its
cure, and if we remember that the years abandoned to public
disorders exceeded the months allotted to the martial reign of
Aurelian, we must confess that a few short intervals of peace
were insufficient for the arduous work of reformation. Even his
attempt to restore the integrity of the coin was opposed by a
formidable insurrection. The emperor’s vexation breaks out in one
of his private letters. “Surely,” says he, “the gods have decreed
that my life should be a perpetual warfare. A sedition within the
walls has just now given birth to a very serious civil war. The
workmen of the mint, at the instigation of Felicissimus, a slave
to whom I had intrusted an employment in the finances, have risen
in rebellion. They are at length suppressed; but seven thousand
of my soldiers have been slain in the contest, of those troops
whose ordinary station is in Dacia, and the camps along the
Danube.”\footnotemark[88] Other writers, who confirm the same fact, add
likewise, that it happened soon after Aurelian’s triumph; that
the decisive engagement was fought on the Cælian hill; that the
workmen of the mint had adulterated the coin; and that the
emperor restored the public credit, by delivering out good money
in exchange for the bad, which the people was commanded to bring
into the treasury.\footnotemark[89]

\footnotetext[87]{Vopiscus in Hist. August. p. 221.}

\footnotetext[88]{Hist. August. p. 222. Aurelian calls these soldiers
Hiberi Riporiences Castriani, and Dacisci.}

\footnotetext[89]{Zosimus, l. i. p. 56. Eutropius, ix. 14. Aurel
Victor.}

We might content ourselves with relating this extraordinary
transaction, but we cannot dissemble how much in its present form
it appears to us inconsistent and incredible. The debasement of
the coin is indeed well suited to the administration of
Gallienus; nor is it unlikely that the instruments of the
corruption might dread the inflexible justice of Aurelian. But
the guilt, as well as the profit, must have been confined to a
very few; nor is it easy to conceive by what arts they could arm
a people whom they had injured, against a monarch whom they had
betrayed. We might naturally expect that such miscreants should
have shared the public detestation with the informers and the
other ministers of oppression; and that the reformation of the
coin should have been an action equally popular with the
destruction of those obsolete accounts, which by the emperor’s
order were burnt in the forum of Trajan.\footnotemark[90] In an age when the
principles of commerce were so imperfectly understood, the most
desirable end might perhaps be effected by harsh and injudicious
means; but a temporary grievance of such a nature can scarcely
excite and support a serious civil war. The repetition of
intolerable taxes, imposed either on the land or on the
necessaries of life, may at last provoke those who will not, or
who cannot, relinquish their country. But the case is far
otherwise in every operation which, by whatsoever expedients,
restores the just value of money. The transient evil is soon
obliterated by the permanent benefit, the loss is divided among
multitudes; and if a few wealthy individuals experience a
sensible diminution of treasure, with their riches, they at the
same time lose the degree of weight and importance which they
derived from the possession of them. However Aurelian might
choose to disguise the real cause of the insurrection, his
reformation of the coin could furnish only a faint pretence to a
party already powerful and discontented. Rome, though deprived of
freedom, was distracted by faction. The people, towards whom the
emperor, himself a plebeian, always expressed a peculiar
fondness, lived in perpetual dissension with the senate, the
equestrian order, and the Prætorian guards.\footnotemark[91] Nothing less than
the firm though secret conspiracy of those orders, of the
authority of the first, the wealth of the second, and the arms of
the third, could have displayed a strength capable of contending
in battle with the veteran legions of the Danube, which, under
the conduct of a martial sovereign, had achieved the conquest of
the West and of the East.

\footnotetext[90]{Hist. August. p. 222. Aurel Victor.}

\footnotetext[91]{It already raged before Aurelian’s return from
Egypt. See Vipiscus, who quotes an original letter. Hist. August.
p. 244.}

Whatever was the cause or the object of this rebellion, imputed
with so little probability to the workmen of the mint, Aurelian
used his victory with unrelenting rigor.\footnotemark[92] He was naturally of a
severe disposition. A peasant and a soldier, his nerves yielded
not easily to the impressions of sympathy, and he could sustain
without emotion the sight of tortures and death. Trained from his
earliest youth in the exercise of arms, he set too small a value
on the life of a citizen, chastised by military execution the
slightest offences, and transferred the stern discipline of the
camp into the civil administration of the laws. His love of
justice often became a blind and furious passion; and whenever he
deemed his own or the public safety endangered, he disregarded
the rules of evidence, and the proportion of punishments. The
unprovoked rebellion with which the Romans rewarded his services,
exasperated his haughty spirit. The noblest families of the
capital were involved in the guilt or suspicion of this dark
conspiracy. A nasty spirit of revenge urged the bloody
prosecution, and it proved fatal to one of the nephews of the
emperor. The executioners (if we may use the expression of a
contemporary poet) were fatigued, the prisons were crowded, and
the unhappy senate lamented the death or absence of its most
illustrious members.\footnotemark[93] Nor was the pride of Aurelian less
offensive to that assembly than his cruelty. Ignorant or
impatient of the restraints of civil institutions, he disdained
to hold his power by any other title than that of the sword, and
governed by right of conquest an empire which he had saved and
subdued.\footnotemark[94]

\footnotetext[92]{Vopiscus in Hist. August p. 222. The two Victors.
Eutropius ix. 14. Zosimus (l. i. p. 43) mentions only three
senators, and placed their death before the eastern war.}

\footnotetext[93]{Nulla catenati feralis pompa senatus Carnificum
lassabit opus; nec carcere pleno Infelix raros numerabit curia
Patres. Calphurn. Eclog. i. 60.}

\footnotetext[94]{According to the younger Victor, he sometimes wore
the diadem, Deus and Dominus appear on his medals.}

It was observed by one of the most sagacious of the Roman
princes, that the talents of his predecessor Aurelian were better
suited to the command of an army, than to the government of an
empire.\footnotemark[95] Conscious of the character in which nature and
experience had enabled him to excel, he again took the field a
few months after his triumph. It was expedient to exercise the
restless temper of the legions in some foreign war, and the
Persian monarch, exulting in the shame of Valerian, still braved
with impunity the offended majesty of Rome. At the head of an
army, less formidable by its numbers than by its discipline and
valor, the emperor advanced as far as the Straits which divide
Europe from Asia. He there experienced that the most absolute
power is a weak defence against the effects of despair. He had
threatened one of his secretaries who was accused of extortion;
and it was known that he seldom threatened in vain. The last hope
which remained for the criminal was to involve some of the
principal officers of the army in his danger, or at least in his
fears. Artfully counterfeiting his master’s hand, he showed them,
in a long and bloody list, their own names devoted to death.
Without suspecting or examining the fraud, they resolved to
secure their lives by the murder of the emperor. On his march,
between Byzantium and Heraclea, Aurelian was suddenly attacked by
the conspirators, whose stations gave them a right to surround
his person, and after a short resistance, fell by the hand of
Mucapor, a general whom he had always loved and trusted. He died
regretted by the army, detested by the senate, but universally
acknowledged as a warlike and fortunate prince, the useful,
though severe reformer of a degenerate state.\footnotemark[96]

\footnotetext[95]{It was the observation of Dioclatian. See Vopiscus
in Hist. August. p. 224.}

\footnotetext[96]{Vopiscus in Hist. August. p. 221. Zosimus, l. i. p.
57. Eutrop ix. 15. The two Victors.}

