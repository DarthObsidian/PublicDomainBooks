\chapter{Reign Of Diocletian And His Three Associates.}
\section{Part \thesection.}

\textit{The Reign Of Diocletian And His Three Associates, Maximian,
Galerius, And Constantius. — General Reestablishment Of Order And
Tranquillity. — The Persian War, Victory, And Triumph. — The New Form
Of Administration. — Abdication And Retirement Of Diocletian And
Maximian.}
\vspace{\onelineskip}

As the reign of Diocletian was more illustrious than that of any
of his predecessors, so was his birth more abject and obscure.
The strong claims of merit and of violence had frequently
superseded the ideal prerogatives of nobility; but a distinct
line of separation was hitherto preserved between the free and
the servile part of mankind. The parents of Diocletian had been
slaves in the house of Anulinus, a Roman senator; nor was he
himself distinguished by any other name than that which he
derived from a small town in Dalmatia, from whence his mother
deduced her origin.\textsuperscript{1} It is, however, probable that his father
obtained the freedom of the family, and that he soon acquired an
office of scribe, which was commonly exercised by persons of his
condition.\textsuperscript{2} Favorable oracles, or rather the consciousness of
superior merit, prompted his aspiring son to pursue the
profession of arms and the hopes of fortune; and it would be
extremely curious to observe the gradation of arts and accidents
which enabled him in the end to fulfil those oracles, and to
display that merit to the world. Diocletian was successively
promoted to the government of Mæsia, the honors of the
consulship, and the important command of the guards of the
palace. He distinguished his abilities in the Persian war; and
after the death of Numerian, the slave, by the confession and
judgment of his rivals, was declared the most worthy of the
Imperial throne. The malice of religious zeal, whilst it arraigns
the savage fierceness of his colleague Maximian, has affected to
cast suspicions on the personal courage of the emperor
Diocletian.\textsuperscript{3} It would not be easy to persuade us of the
cowardice of a soldier of fortune, who acquired and preserved the
esteem of the legions as well as the favor of so many warlike
princes. Yet even calumny is sagacious enough to discover and to
attack the most vulnerable part. The valor of Diocletian was
never found inadequate to his duty, or to the occasion; but he
appears not to have possessed the daring and generous spirit of a
hero, who courts danger and fame, disdains artifice, and boldly
challenges the allegiance of his equals. His abilities were
useful rather than splendid; a vigorous mind, improved by the
experience and study of mankind; dexterity and application in
business; a judicious mixture of liberality and economy, of
mildness and rigor; profound dissimulation, under the disguise of
military frankness; steadiness to pursue his ends; flexibility to
vary his means; and, above all, the great art of submitting his
own passions, as well as those of others, to the interest of his
ambition, and of coloring his ambition with the most specious
pretences of justice and public utility. Like Augustus,
Diocletian may be considered as the founder of a new empire. Like
the adopted son of Cæsar, he was distinguished as a statesman
rather than as a warrior; nor did either of those princes employ
force, whenever their purpose could be effected by policy.

\pagenote[1]{Eutrop. ix. 19. Victor in Epitome. The town seems to
have been properly called Doclia, from a small tribe of
Illyrians, (see Cellarius, Geograph. Antiqua, tom. i. p. 393;)
and the original name of the fortunate slave was probably Docles;
he first lengthened it to the Grecian harmony of Diocles, and at
length to the Roman majesty of Diocletianus. He likewise assumed
the Patrician name of Valerius and it is usually given him by
Aurelius Victor.}

\pagenote[2]{See Dacier on the sixth satire of the second book of
Horace Cornel. Nepos, ’n Vit. Eumen. c. l.}

\pagenote[3]{Lactantius (or whoever was the author of the little
treatise De Mortibus Persecutorum) accuses Diocletian of timidity
in two places, c. 7. 8. In chap. 9 he says of him, “erat in omni
tumultu meticulosu et animi disjectus.”}

The victory of Diocletian was remarkable for its singular
mildness. A people accustomed to applaud the clemency of the
conqueror, if the usual punishments of death, exile, and
confiscation, were inflicted with any degree of temper and
equity, beheld, with the most pleasing astonishment, a civil war,
the flames of which were extinguished in the field of battle.
Diocletian received into his confidence Aristobulus, the
principal minister of the house of Carus, respected the lives,
the fortunes, and the dignity, of his adversaries, and even
continued in their respective stations the greater number of the
servants of Carinus.\textsuperscript{4} It is not improbable that motives of
prudence might assist the humanity of the artful Dalmatian; of
these servants, many had purchased his favor by secret treachery;
in others, he esteemed their grateful fidelity to an unfortunate
master. The discerning judgment of Aurelian, of Probus, and of
Carus, had filled the several departments of the state and army
with officers of approved merit, whose removal would have injured
the public service, without promoting the interest of his
successor. Such a conduct, however, displayed to the Roman world
the fairest prospect of the new reign, and the emperor affected
to confirm this favorable prepossession, by declaring, that,
among all the virtues of his predecessors, he was the most
ambitious of imitating the humane philosophy of Marcus Antoninus.\textsuperscript{5}

\pagenote[4]{In this encomium, Aurelius Victor seems to convey a
just, though indirect, censure of the cruelty of Constantius. It
appears from the Fasti, that Aristobulus remained præfect of the
city, and that he ended with Diocletian the consulship which he
had commenced with Carinus.}

\pagenote[5]{Aurelius Victor styles Diocletian, “Parentum potius
quam Dominum.” See Hist. August. p. 30.}

The first considerable action of his reign seemed to evince his
sincerity as well as his moderation. After the example of Marcus,
he gave himself a colleague in the person of Maximian, on whom he
bestowed at first the title of Cæsar, and afterwards that of
Augustus.\textsuperscript{6} But the motives of his conduct, as well as the object
of his choice, were of a very different nature from those of his
admired predecessor. By investing a luxurious youth with the
honors of the purple, Marcus had discharged a debt of private
gratitude, at the expense, indeed, of the happiness of the state.
By associating a friend and a fellow-soldier to the labors of
government, Diocletian, in a time of public danger, provided for
the defence both of the East and of the West. Maximian was born a
peasant, and, like Aurelian, in the territory of Sirmium.
Ignorant of letters,\textsuperscript{7} careless of laws, the rusticity of his
appearance and manners still betrayed in the most elevated
fortune the meanness of his extraction. War was the only art
which he professed. In a long course of service he had
distinguished himself on every frontier of the empire; and though
his military talents were formed to obey rather than to command,
though, perhaps, he never attained the skill of a consummate
general, he was capable, by his valor, constancy, and experience,
of executing the most arduous undertakings. Nor were the vices of
Maximian less useful to his benefactor. Insensible to pity, and
fearless of consequences, he was the ready instrument of every
act of cruelty which the policy of that artful prince might at
once suggest and disclaim. As soon as a bloody sacrifice had been
offered to prudence or to revenge, Diocletian, by his seasonable
intercession, saved the remaining few whom he had never designed
to punish, gently censured the severity of his stern colleague,
and enjoyed the comparison of a golden and an iron age, which was
universally applied to their opposite maxims of government.
Notwithstanding the difference of their characters, the two
emperors maintained, on the throne, that friendship which they
had contracted in a private station. The haughty, turbulent
spirit of Maximian, so fatal, afterwards, to himself and to the
public peace, was accustomed to respect the genius of Diocletian,
and confessed the ascendant of reason over brutal violence.\textsuperscript{8}
From a motive either of pride or superstition, the two emperors
assumed the titles, the one of Jovius, the other of Herculius.
Whilst the motion of the world (such was the language of their
venal orators) was maintained by the all-seeing wisdom of
Jupiter, the invincible arm of Hercules purged the earth from
monsters and tyrants.\textsuperscript{9}

\pagenote[6]{The question of the time when Maximian received the
honors of Cæsar and Augustus has divided modern critics, and
given occasion to a great deal of learned wrangling. I have
followed M. de Tillemont, (Histoire des Empereurs, tom. iv. p.
500-505,) who has weighed the several reasons and difficulties
with his scrupulous accuracy. * Note: Eckbel concurs in this
view, viii p. 15.—M.}

\pagenote[7]{In an oration delivered before him, (Panegyr. Vet.
ii. 8,) Mamertinus expresses a doubt, whether his hero, in
imitating the conduct of Hannibal and Scipio, had ever heard of
their names. From thence we may fairly infer, that Maximian was
more desirous of being considered as a soldier than as a man of
letters; and it is in this manner that we can often translate the
language of flattery into that of truth.}

\pagenote[8]{Lactantius de M. P. c. 8. Aurelius Victor. As among
the Panegyrics, we find orations pronounced in praise of
Maximian, and others which flatter his adversaries at his
expense, we derive some knowledge from the contrast.}

\pagenote[9]{See the second and third Panegyrics, particularly
iii. 3, 10, 14 but it would be tedious to copy the diffuse and
affected expressions of their false eloquence. With regard to the
titles, consult Aurel. Victor Lactantius de M. P. c. 52. Spanheim
de Usu Numismatum, \&c. xii 8.}

But even the omnipotence of Jovius and Herculius was insufficient
to sustain the weight of the public administration. The prudence
of Diocletian discovered that the empire, assailed on every side
by the barbarians, required on every side the presence of a great
army, and of an emperor. With this view, he resolved once more to
divide his unwieldy power, and with the inferior title of
\textit{Cæsars},\textsuperscript{901} to confer on two generals of approved merit an
unequal share of the sovereign authority.\textsuperscript{10} Galerius, surnamed
Armentarius, from his original profession of a herdsman, and
Constantius, who from his pale complexion had acquired the
denomination of Chlorus,\textsuperscript{11} were the two persons invested with
the second honors of the Imperial purple. In describing the
country, extraction, and manners of Herculius, we have already
delineated those of Galerius, who was often, and not improperly,
styled the younger Maximian, though, in many instances both of
virtue and ability, he appears to have possessed a manifest
superiority over the elder. The birth of Constantius was less
obscure than that of his colleagues. Eutropius, his father, was
one of the most considerable nobles of Dardania, and his mother
was the niece of the emperor Claudius.\textsuperscript{12} Although the youth of
Constantius had been spent in arms, he was endowed with a mild
and amiable disposition, and the popular voice had long since
acknowledged him worthy of the rank which he at last attained. To
strengthen the bonds of political, by those of domestic, union,
each of the emperors assumed the character of a father to one of
the Cæsars, Diocletian to Galerius, and Maximian to Constantius;
and each, obliging them to repudiate their former wives, bestowed
his daughter in marriage or his adopted son.\textsuperscript{13} These four
princes distributed among themselves the wide extent of the Roman
empire. The defence of Gaul, Spain,\textsuperscript{14} and Britain, was intrusted
to Constantius: Galerius was stationed on the banks of the
Danube, as the safeguard of the Illyrian provinces. Italy and
Africa were considered as the department of Maximian; and for his
peculiar portion, Diocletian reserved Thrace, Egypt, and the rich
countries of Asia. Every one was sovereign with his own
jurisdiction; but their united authority extended over the whole
monarchy, and each of them was prepared to assist his colleagues
with his counsels or presence. The Cæsars, in their exalted rank,
revered the majesty of the emperors, and the three younger
princes invariably acknowledged, by their gratitude and
obedience, the common parent of their fortunes. The suspicious
jealousy of power found not any place among them; and the
singular happiness of their union has been compared to a chorus
of music, whose harmony was regulated and maintained by the
skilful hand of the first artist.\textsuperscript{15}

\pagenote[901]{On the relative power of the Augusti and the
Cæsars, consult a dissertation at the end of Manso’s Leben
Constantius des Grossen—M.}

\pagenote[10]{Aurelius Victor. Victor in Epitome. Eutrop. ix. 22.
Lactant de M. P. c. 8. Hieronym. in Chron.}

\pagenote[11]{It is only among the modern Greeks that Tillemont
can discover his appellation of Chlorus. Any remarkable degree of
paleness seems inconsistent with the rubor mentioned in
Panegyric, v. 19.}

\pagenote[12]{Julian, the grandson of Constantius, boasts that
his family was derived from the warlike Mæsians. Misopogon, p.
348. The Dardanians dwelt on the edge of Mæsia.}

\pagenote[13]{Galerius married Valeria, the daughter of
Diocletian; if we speak with strictness, Theodora, the wife of
Constantius, was daughter only to the wife of Maximian. Spanheim,
Dissertat, xi. 2.}

\pagenote[14]{This division agrees with that of the four
præfectures; yet there is some reason to doubt whether Spain was
not a province of Maximian. See Tillemont, tom. iv. p. 517. *
Note: According to Aurelius Victor and other authorities, Thrace
belonged to the division of Galerius. See Tillemont, iv. 36. But
the laws of Diocletian are in general dated in Illyria or
Thrace.—M.}

\pagenote[15]{Julian in Cæsarib. p. 315. Spanheim’s notes to the
French translation, p. 122.}

This important measure was not carried into execution till about
six years after the association of Maximian, and that interval of
time had not been destitute of memorable incidents. But we have
preferred, for the sake of perspicuity, first to describe the
more perfect form of Diocletian’s government, and afterwards to
relate the actions of his reign, following rather the natural
order of the events, than the dates of a very doubtful
chronology.

The first exploit of Maximian, though it is mentioned in a few
words by our imperfect writers, deserves, from its singularity,
to be recorded in a history of human manners. He suppressed the
peasants of Gaul, who, under the appellation of Bagaudæ,\textsuperscript{16} had
risen in a general insurrection; very similar to those which in
the fourteenth century successively afflicted both France and
England.\textsuperscript{17} It should seem that very many of those institutions,
referred by an easy solution to the feudal system, are derived
from the Celtic barbarians. When Cæsar subdued the Gauls, that
great nation was already divided into three orders of men; the
clergy, the nobility, and the common people. The first governed
by superstition, the second by arms, but the third and last was
not of any weight or account in their public councils. It was
very natural for the plebeians, oppressed by debt, or
apprehensive of injuries, to implore the protection of some
powerful chief, who acquired over their persons and property the
same absolute right as, among the Greeks and Romans, a master
exercised over his slaves.\textsuperscript{18} The greatest part of the nation was
gradually reduced into a state of servitude; compelled to
perpetual labor on the estates of the Gallic nobles, and confined
to the soil, either by the real weight of fetters, or by the no
less cruel and forcible restraints of the laws. During the long
series of troubles which agitated Gaul, from the reign of
Gallienus to that of Diocletian, the condition of these servile
peasants was peculiarly miserable; and they experienced at once
the complicated tyranny of their masters, of the barbarians, of
the soldiers, and of the officers of the revenue.\textsuperscript{19}

\pagenote[16]{The general name of Bagaudæ (in the signification
of rebels) continued till the fifth century in Gaul. Some critics
derive it from a Celtic word Bagad, a tumultuous assembly.
Scaliger ad Euseb. Du Cange Glossar. (Compare S. Turner,
Anglo-Sax. History, i. 214.—M.)}

\pagenote[17]{Chronique de Froissart, vol. i. c. 182, ii. 73, 79.
The naivete of his story is lost in our best modern writers.}

\pagenote[18]{Cæsar de Bell. Gallic. vi. 13. Orgetorix, the
Helvetian, could arm for his defence a body of ten thousand
slaves.}

\pagenote[19]{Their oppression and misery are acknowledged by
Eumenius (Panegyr. vi. 8,) Gallias efferatas injuriis.}

Their patience was at last provoked into despair. On every side
they rose in multitudes, armed with rustic weapons, and with
irresistible fury. The ploughman became a foot soldier, the
shepherd mounted on horseback, the deserted villages and open
towns were abandoned to the flames, and the ravages of the
peasants equalled those of the fiercest barbarians.\textsuperscript{20} They
asserted the natural rights of men, but they asserted those
rights with the most savage cruelty. The Gallic nobles, justly
dreading their revenge, either took refuge in the fortified
cities, or fled from the wild scene of anarchy. The peasants
reigned without control; and two of their most daring leaders had
the folly and rashness to assume the Imperial ornaments.\textsuperscript{21} Their
power soon expired at the approach of the legions. The strength
of union and discipline obtained an easy victory over a
licentious and divided multitude.\textsuperscript{22} A severe retaliation was
inflicted on the peasants who were found in arms; the affrighted
remnant returned to their respective habitations, and their
unsuccessful effort for freedom served only to confirm their
slavery. So strong and uniform is the current of popular
passions, that we might almost venture, from very scanty
materials, to relate the particulars of this war; but we are not
disposed to believe that the principal leaders, Ælianus and
Amandus, were Christians,\textsuperscript{23} or to insinuate, that the rebellion,
as it happened in the time of Luther, was occasioned by the abuse
of those benevolent principles of Christianity, which inculcate
the natural freedom of mankind.

\pagenote[20]{Panegyr. Vet. ii. 4. Aurelius Victor.}

\pagenote[21]{Ælianus and Amandus. We have medals coined by them
Goltzius in Thes. R. A. p. 117, 121.}

\pagenote[22]{Levibus proeliis domuit. Eutrop. ix. 20.}

\pagenote[23]{The fact rests indeed on very slight authority, a
life of St. Babolinus, which is probably of the seventh century.
See Duchesne Scriptores Rer. Francicar. tom. i. p. 662.}

Maximian had no sooner recovered Gaul from the hands of the
peasants, than he lost Britain by the usurpation of Carausius.
Ever since the rash but successful enterprise of the Franks under
the reign of Probus, their daring countrymen had constructed
squadrons of light brigantines, in which they incessantly ravaged
the provinces adjacent to the ocean.\textsuperscript{24} To repel their desultory
incursions, it was found necessary to create a naval power; and
the judicious measure was prosecuted with prudence and vigor.
Gessoriacum, or Boulogne, in the straits of the British Channel,
was chosen by the emperor for the station of the Roman fleet; and
the command of it was intrusted to Carausius, a Menapian of the
meanest origin,\textsuperscript{25} but who had long signalized his skill as a
pilot, and his valor as a soldier. The integrity of the new
admiral corresponded not with his abilities. When the German
pirates sailed from their own harbors, he connived at their
passage, but he diligently intercepted their return, and
appropriated to his own use an ample share of the spoil which
they had acquired. The wealth of Carausius was, on this occasion,
very justly considered as an evidence of his guilt; and Maximian
had already given orders for his death. But the crafty Menapian
foresaw and prevented the severity of the emperor. By his
liberality he had attached to his fortunes the fleet which he
commanded, and secured the barbarians in his interest. From the
port of Boulogne he sailed over to Britain, persuaded the legion,
and the auxiliaries which guarded that island, to embrace his
party, and boldly assuming, with the Imperial purple, the title
of Augustus, defied the justice and the arms of his injured
sovereign.\textsuperscript{26}

\pagenote[24]{Aurelius Victor calls them Germans. Eutropius (ix.
21) gives them the name of Saxons. But Eutropius lived in the
ensuing century, and seems to use the language of his own times.}

\pagenote[25]{The three expressions of Eutropius, Aurelius
Victor, and Eumenius, “vilissime natus,” “Bataviæ alumnus,” and
“Menapiæ civis,” give us a very doubtful account of the birth of
Carausius. Dr. Stukely, however, (Hist. of Carausius, p. 62,)
chooses to make him a native of St. David’s and a prince of the
blood royal of Britain. The former idea he had found in Richard
of Cirencester, p. 44. * Note: The Menapians were settled between
the Scheldt and the Meuse, is the northern part of Brabant.
D’Anville, Geogr. Anc. i. 93.—G.}

\pagenote[26]{Panegyr. v. 12. Britain at this time was secure,
and slightly guarded.}

When Britain was thus dismembered from the empire, its importance
was sensibly felt, and its loss sincerely lamented. The Romans
celebrated, and perhaps magnified, the extent of that noble
island, provided on every side with convenient harbors; the
temperature of the climate, and the fertility of the soil, alike
adapted for the production of corn or of vines; the valuable
minerals with which it abounded; its rich pastures covered with
innumerable flocks, and its woods free from wild beasts or
venomous serpents. Above all, they regretted the large amount of
the revenue of Britain, whilst they confessed, that such a
province well deserved to become the seat of an independent
monarchy.\textsuperscript{27} During the space of seven years it was possessed by
Carausius; and fortune continued propitious to a rebellion
supported with courage and ability. The British emperor defended
the frontiers of his dominions against the Caledonians of the
North, invited, from the continent, a great number of skilful
artists, and displayed, on a variety of coins that are still
extant, his taste and opulence. Born on the confines of the
Franks, he courted the friendship of that formidable people, by
the flattering imitation of their dress and manners. The bravest
of their youth he enlisted among his land or sea forces; and, in
return for their useful alliance, he communicated to the
barbarians the dangerous knowledge of military and naval arts.
Carausius still preserved the possession of Boulogne and the
adjacent country. His fleets rode triumphant in the channel,
commanded the mouths of the Seine and of the Rhine, ravaged the
coasts of the ocean, and diffused beyond the columns of Hercules
the terror of his name. Under his command, Britain, destined in a
future age to obtain the empire of the sea, already assumed its
natural and respectable station of a maritime power.\textsuperscript{28}

\pagenote[27]{Panegyr. Vet v 11, vii. 9. The orator Eumenius
wished to exalt the glory of the hero (Constantius) with the
importance of the conquest. Notwithstanding our laudable
partiality for our native country, it is difficult to conceive,
that, in the beginning of the fourth century England deserved all
these commendations. A century and a half before, it hardly paid
its own establishment.}

\pagenote[28]{As a great number of medals of Carausius are still
preserved, he is become a very favorite object of antiquarian
curiosity, and every circumstance of his life and actions has
been investigated with sagacious accuracy. Dr. Stukely, in
particular, has devoted a large volume to the British emperor. I
have used his materials, and rejected most of his fanciful
conjectures.}

By seizing the fleet of Boulogne, Carausius had deprived his
master of the means of pursuit and revenge. And when, after a
vast expense of time and labor, a new armament was launched into
the water,\textsuperscript{29} the Imperial troops, unaccustomed to that element,
were easily baffled and defeated by the veteran sailors of the
usurper. This disappointed effort was soon productive of a treaty
of peace. Diocletian and his colleague, who justly dreaded the
enterprising spirit of Carausius, resigned to him the sovereignty
of Britain, and reluctantly admitted their perfidious servant to
a participation of the Imperial honors.\textsuperscript{30} But the adoption of
the two Cæsars restored new vigor to the Romans arms; and while
the Rhine was guarded by the presence of Maximian, his brave
associate Constantius assumed the conduct of the British war. His
first enterprise was against the important place of Boulogne. A
stupendous mole, raised across the entrance of the harbor,
intercepted all hopes of relief. The town surrendered after an
obstinate defence; and a considerable part of the naval strength
of Carausius fell into the hands of the besiegers. During the
three years which Constantius employed in preparing a fleet
adequate to the conquest of Britain, he secured the coast of
Gaul, invaded the country of the Franks, and deprived the usurper
of the assistance of those powerful allies.

\pagenote[29]{When Mamertinus pronounced his first panegyric, the
naval preparations of Maximian were completed; and the orator
presaged an assured victory. His silence in the second panegyric
might alone inform us that the expedition had not succeeded.}

\pagenote[30]{Aurelius Victor, Eutropius, and the medals, (Pax
Augg.) inform us of this temporary reconciliation; though I will
not presume (as Dr. Stukely has done, Medallic History of
Carausius, p. 86, \&c) to insert the identical articles of the
treaty.}

Before the preparations were finished, Constantius received the
intelligence of the tyrant’s death, and it was considered as a
sure presage of the approaching victory. The servants of
Carausius imitated the example of treason which he had given. He
was murdered by his first minister, Allectus, and the assassin
succeeded to his power and to his danger. But he possessed not
equal abilities either to exercise the one or to repel the other.

He beheld, with anxious terror, the opposite shores of the
continent already filled with arms, with troops, and with
vessels; for Constantius had very prudently divided his forces,
that he might likewise divide the attention and resistance of the
enemy. The attack was at length made by the principal squadron,
which, under the command of the præfect Asclepiodatus, an officer
of distinguished merit, had been assembled in the north of the
Seine. So imperfect in those times was the art of navigation,
that orators have celebrated the daring courage of the Romans,
who ventured to set sail with a side-wind, and on a stormy day.
The weather proved favorable to their enterprise. Under the cover
of a thick fog, they escaped the fleet of Allectus, which had
been stationed off the Isle of Wight to receive them, landed in
safety on some part of the western coast, and convinced the
Britons, that a superiority of naval strength will not always
protect their country from a foreign invasion. Asclepiodatus had
no sooner disembarked the imperial troops, then he set fire to
his ships; and, as the expedition proved fortunate, his heroic
conduct was universally admired. The usurper had posted himself
near London, to expect the formidable attack of Constantius, who
commanded in person the fleet of Boulogne; but the descent of a
new enemy required his immediate presence in the West. He
performed this long march in so precipitate a manner, that he
encountered the whole force of the præfect with a small body of
harassed and disheartened troops. The engagement was soon
terminated by the total defeat and death of Allectus; a single
battle, as it has often happened, decided the fate of this great
island; and when Constantius landed on the shores of Kent, he
found them covered with obedient subjects. Their acclamations
were loud and unanimous; and the virtues of the conqueror may
induce us to believe, that they sincerely rejoiced in a
revolution, which, after a separation of ten years, restored
Britain to the body of the Roman empire.\textsuperscript{31}

\pagenote[31]{With regard to the recovery of Britain, we obtain a
few hints from Aurelius Victor and Eutropius.}

\section{Part \thesection.}

Britain had none but domestic enemies to dread; and as long as
the governors preserved their fidelity, and the troops their
discipline, the incursions of the naked savages of Scotland or
Ireland could never materially affect the safety of the province.

The peace of the continent, and the defence of the principal
rivers which bounded the empire, were objects of far greater
difficulty and importance. The policy of Diocletian, which
inspired the councils of his associates, provided for the public
tranquility, by encouraging a spirit of dissension among the
barbarians, and by strengthening the fortifications of the Roman
limit. In the East he fixed a line of camps from Egypt to the
Persian dominions, and for every camp, he instituted an adequate
number of stationary troops, commanded by their respective
officers, and supplied with every kind of arms, from the new
arsenals which he had formed at Antioch, Emesa, and Damascus.\textsuperscript{32}
Nor was the precaution of the emperor less watchful against the
well-known valor of the barbarians of Europe. From the mouth of
the Rhine to that of the Danube, the ancient camps, towns, and
citidels, were diligently reëstablished, and, in the most exposed
places, new ones were skilfully constructed: the strictest
vigilance was introduced among the garrisons of the frontier, and
every expedient was practised that could render the long chain of
fortifications firm and impenetrable.\textsuperscript{33} A barrier so respectable
was seldom violated, and the barbarians often turned against each
other their disappointed rage. The Goths, the Vandals, the
Gepidæ, the Burgundians, the Alemanni, wasted each other’s
strength by destructive hostilities: and whosoever vanquished,
they vanquished the enemies of Rome. The subjects of Diocletian
enjoyed the bloody spectacle, and congratulated each other, that
the mischiefs of civil war were now experienced only by the
barbarians.\textsuperscript{34}

\pagenote[32]{John Malala, in Chron, Antiochen. tom. i. p. 408,
409.}

\pagenote[33]{Zosim. l. i. p. 3. That partial historian seems to
celebrate the vigilance of Diocletian with a design of exposing
the negligence of Constantine; we may, however, listen to an
orator: “Nam quid ego alarum et cohortium castra percenseam, toto
Rheni et Istri et Euphraus limite restituta.” Panegyr. Vet. iv.
18.}

\pagenote[34]{Ruunt omnes in sanguinem suum populi, quibus ron
contigilesse Romanis, obstinatæque feritatis poenas nunc sponte
persolvunt. Panegyr. Vet. iii. 16. Mamertinus illustrates the
fact by the example of almost all the nations in the world.}

Notwithstanding the policy of Diocletian, it was impossible to
maintain an equal and undisturbed tranquillity during a reign of
twenty years, and along a frontier of many hundred miles.
Sometimes the barbarians suspended their domestic animosities,
and the relaxed vigilance of the garrisons sometimes gave a
passage to their strength or dexterity. Whenever the provinces
were invaded, Diocletian conducted himself with that calm dignity
which he always affected or possessed; reserved his presence for
such occasions as were worthy of his interposition, never exposed
his person or reputation to any unnecessary danger, insured his
success by every means that prudence could suggest, and
displayed, with ostentation, the consequences of his victory. In
wars of a more difficult nature, and more doubtful event, he
employed the rough valor of Maximian; and that faithful soldier
was content to ascribe his own victories to the wise counsels and
auspicious influence of his benefactor. But after the adoption of
the two Cæsars, the emperors themselves, retiring to a less
laborious scene of action, devolved on their adopted sons the
defence of the Danube and of the Rhine. The vigilant Galerius was
never reduced to the necessity of vanquishing an army of
barbarians on the Roman territory. \textsuperscript{35} The brave and active
Constantius delivered Gaul from a very furious inroad of the
Alemanni; and his victories of Langres and Vindonissa appear to
have been actions of considerable danger and merit. As he
traversed the open country with a feeble guard, he was
encompassed on a sudden by the superior multitude of the enemy.
He retreated with difficulty towards Langres; but, in the general
consternation, the citizens refused to open their gates, and the
wounded prince was drawn up the wall by the means of a rope. But,
on the news of his distress, the Roman troops hastened from all
sides to his relief, and before the evening he had satisfied his
honor and revenge by the slaughter of six thousand Alemanni.\textsuperscript{36}
From the monuments of those times, the obscure traces of several
other victories over the barbarians of Sarmatia and Germany might
possibly be collected; but the tedious search would not be
rewarded either with amusement or with instruction.

\pagenote[35]{He complained, though not with the strictest truth,
“Jam fluxisse annos quindecim in quibus, in Illyrico, ad ripam
Danubii relegatus cum gentibus barbaris luctaret.” Lactant. de M.
P. c. 18.}

\pagenote[36]{In the Greek text of Eusebius, we read six
thousand, a number which I have preferred to the sixty thousand
of Jerome, Orosius Eutropius, and his Greek translator Pæanius.}

The conduct which the emperor Probus had adopted in the disposal
of the vanquished was imitated by Diocletian and his associates.
The captive barbarians, exchanging death for slavery, were
distributed among the provincials, and assigned to those
districts (in Gaul, the territories of Amiens, Beauvais, Cambray,
Treves, Langres, and Troyes, are particularly specified)\textsuperscript{37} which
had been depopulated by the calamities of war. They were usefully
employed as shepherds and husbandmen, but were denied the
exercise of arms, except when it was found expedient to enroll
them in the military service. Nor did the emperors refuse the
property of lands, with a less servile tenure, to such of the
barbarians as solicited the protection of Rome. They granted a
settlement to several colonies of the Carpi, the Bastarnæ, and
the Sarmatians; and, by a dangerous indulgence, permitted them in
some measure to retain their national manners and independence.\textsuperscript{38}
Among the provincials, it was a subject of flattering
exultation, that the barbarian, so lately an object of terror,
now cultivated their lands, drove their cattle to the neighboring
fair, and contributed by his labor to the public plenty. They
congratulated their masters on the powerful accession of subjects
and soldiers; but they forgot to observe, that multitudes of
secret enemies, insolent from favor, or desperate from
oppression, were introduced into the heart of the empire.\textsuperscript{39}

\pagenote[37]{Panegyr. Vet. vii. 21.}

\pagenote[38]{There was a settlement of the Sarmatians in the
neighborhood of Treves, which seems to have been deserted by
those lazy barbarians. Ausonius speaks of them in his Mosella:——
“Unde iter ingrediens nemorosa per avia solum, Et nulla humani
spectans vestigia cultus; ........ Arvaque Sauromatum nuper
metata colonis.”}

\pagenote[39]{There was a town of the Carpi in the Lower Mæsia.
See the rhetorical exultation of Eumenius.}

While the Cæsars exercised their valor on the banks of the Rhine
and Danube, the presence of the emperors was required on the
southern confines of the Roman world. From the Nile to Mount
Atlas, Africa was in arms. A confederacy of five Moorish nations
issued from their deserts to invade the peaceful provinces.\textsuperscript{40}
Julian had assumed the purple at Carthage.\textsuperscript{41} Achilleus at
Alexandria, and even the Blemmyes, renewed, or rather continued,
their incursions into the Upper Egypt. Scarcely any circumstances
have been preserved of the exploits of Maximian in the western
parts of Africa; but it appears, by the event, that the progress
of his arms was rapid and decisive, that he vanquished the
fiercest barbarians of Mauritania, and that he removed them from
the mountains, whose inaccessible strength had inspired their
inhabitants with a lawless confidence, and habituated them to a
life of rapine and violence.\textsuperscript{42} Diocletian, on his side, opened
the campaign in Egypt by the siege of Alexandria, cut off the
aqueducts which conveyed the waters of the Nile into every
quarter of that immense city,\textsuperscript{43} and rendering his camp
impregnable to the sallies of the besieged multitude, he pushed
his reiterated attacks with caution and vigor. After a siege of
eight months, Alexandria, wasted by the sword and by fire,
implored the clemency of the conqueror, but it experienced the
full extent of his severity. Many thousands of the citizens
perished in a promiscuous slaughter, and there were few obnoxious
persons in Egypt who escaped a sentence either of death or at
least of exile.\textsuperscript{44} The fate of Busiris and of Coptos was still
more melancholy than that of Alexandria: those proud cities, the
former distinguished by its antiquity, the latter enriched by the
passage of the Indian trade, were utterly destroyed by the arms
and by the severe order of Diocletian.\textsuperscript{45} The character of the
Egyptian nation, insensible to kindness, but extremely
susceptible of fear, could alone justify this excessive rigor.
The seditions of Alexandria had often affected the tranquillity
and subsistence of Rome itself. Since the usurpation of Firmus,
the province of Upper Egypt, incessantly relapsing into
rebellion, had embraced the alliance of the savages of Æthiopia.
The number of the Blemmyes, scattered between the Island of Meroe
and the Red Sea, was very inconsiderable, their disposition was
unwarlike, their weapons rude and inoffensive.\textsuperscript{46} Yet in the
public disorders, these barbarians, whom antiquity, shocked with
the deformity of their figure, had almost excluded from the human
species, presumed to rank themselves among the enemies of Rome.\textsuperscript{47}
Such had been the unworthy allies of the Egyptians; and while
the attention of the state was engaged in more serious wars,
their vexations inroads might again harass the repose of the
province. With a view of opposing to the Blemmyes a suitable
adversary, Diocletian persuaded the Nobatæ, or people of Nubia,
to remove from their ancient habitations in the deserts of Libya,
and resigned to them an extensive but unprofitable territory
above Syene and the cataracts of the Nile, with the stipulation,
that they should ever respect and guard the frontier of the
empire. The treaty long subsisted; and till the establishment of
Christianity introduced stricter notions of religious worship, it
was annually ratified by a solemn sacrifice in the isle of
Elephantine, in which the Romans, as well as the barbarians,
adored the same visible or invisible powers of the universe.\textsuperscript{48}

\pagenote[40]{Scaliger (Animadvers. ad Euseb. p. 243) decides, in
his usual manner, that the Quinque gentiani, or five African
nations, were the five great cities, the Pentapolis of the
inoffensive province of Cyrene.}

\pagenote[41]{After his defeat, Julian stabbed himself with a
dagger, and immediately leaped into the flames. Victor in
Epitome.}

\pagenote[42]{Tu ferocissimos Mauritaniæ populos inaccessis
montium jugis et naturali munitione fidentes, expugnasti,
recepisti, transtulisti. Panegyr Vet. vi. 8.}

\pagenote[43]{See the description of Alexandria, in Hirtius de
Bel. Alexandrin c. 5.}

\pagenote[44]{Eutrop. ix. 24. Orosius, vii. 25. John Malala in
Chron. Antioch. p. 409, 410. Yet Eumenius assures us, that Egypt
was pacified by the clemency of Diocletian.}

\pagenote[45]{Eusebius (in Chron.) places their destruction
several years sooner and at a time when Egypt itself was in a
state of rebellion against the Romans.}

\pagenote[46]{Strabo, l. xvii. p. 172. Pomponius Mela, l. i. c.
4. His words are curious: “Intra, si credere libet vix, homines
magisque semiferi Ægipanes, et Blemmyes, et Satyri.”}

\pagenote[47]{Ausus sese inserere fortunæ et provocare arma
Romana.}

\pagenote[48]{See Procopius de Bell. Persic. l. i. c. 19. Note:
Compare, on the epoch of the final extirpation of the rites of
Paganism from the Isle of Philæ, (Elephantine,) which subsisted
till the edict of Theodosius, in the sixth century, a
dissertation of M. Letronne, on certain Greek inscriptions. The
dissertation contains some very interesting observations on the
conduct and policy of Diocletian in Egypt. Mater pour l’Hist. du
Christianisme en Egypte, Nubie et Abyssinie, Paris 1817—M.}

At the same time that Diocletian chastised the past crimes of the
Egyptians, he provided for their future safety and happiness by
many wise regulations, which were confirmed and enforced under
the succeeding reigns.\textsuperscript{49} One very remarkable edict which he
published, instead of being condemned as the effect of jealous
tyranny, deserves to be applauded as an act of prudence and
humanity. He caused a diligent inquiry to be made “for all the
ancient books which treated of the admirable art of making gold
and silver, and without pity, committed them to the flames;
apprehensive, as we are assumed, lest the opulence of the
Egyptians should inspire them with confidence to rebel against
the empire.”\textsuperscript{50} But if Diocletian had been convinced of the
reality of that valuable art, far from extinguishing the memory,
he would have converted the operation of it to the benefit of the
public revenue. It is much more likely, that his good sense
discovered to him the folly of such magnificent pretensions, and
that he was desirous of preserving the reason and fortunes of his
subjects from the mischievous pursuit. It may be remarked, that
these ancient books, so liberally ascribed to Pythagoras, to
Solomon, or to Hermes, were the pious frauds of more recent
adepts. The Greeks were inattentive either to the use or to the
abuse of chemistry. In that immense register, where Pliny has
deposited the discoveries, the arts, and the errors of mankind,
there is not the least mention of the transmutation of metals;
and the persecution of Diocletian is the first authentic event in
the history of alchemy. The conquest of Egypt by the Arabs
diffused that vain science over the globe. Congenial to the
avarice of the human heart, it was studied in China as in Europe,
with equal eagerness, and with equal success. The darkness of the
middle ages insured a favorable reception to every tale of
wonder, and the revival of learning gave new vigor to hope, and
suggested more specious arts of deception. Philosophy, with the
aid of experience, has at length banished the study of alchemy;
and the present age, however desirous of riches, is content to
seek them by the humbler means of commerce and industry.\textsuperscript{51}

\pagenote[49]{He fixed the public allowance of corn, for the
people of Alexandria, at two millions of medimni; about four
hundred thousand quarters. Chron. Paschal. p. 276 Procop. Hist.
Arcan. c. 26.}

\pagenote[50]{John Antioch, in Excerp. Valesian. p. 834. Suidas
in Diocletian.}

\pagenote[51]{See a short history and confutation of Alchemy, in
the works of that philosophical compiler, La Mothe le Vayer, tom.
i. p. 32—353.}

The reduction of Egypt was immediately followed by the Persian
war. It was reserved for the reign of Diocletian to vanquish that
powerful nation, and to extort a confession from the successors
of Artaxerxes, of the superior majesty of the Roman empire.

We have observed, under the reign of Valerian, that Armenia was
subdued by the perfidy and the arms of the Persians, and that,
after the assassination of Chosroes, his son Tiridates, the
infant heir of the monarchy, was saved by the fidelity of his
friends, and educated under the protection of the emperors.
Tiridates derived from his exile such advantages as he could
never have obtained on the throne of Armenia; the early knowledge
of adversity, of mankind, and of the Roman discipline. He
signalized his youth by deeds of valor, and displayed a matchless
dexterity, as well as strength, in every martial exercise, and
even in the less honorable contests of the Olympian games.\textsuperscript{52}
Those qualities were more nobly exerted in the defence of his
benefactor Licinius.\textsuperscript{53} That officer, in the sedition which
occasioned the death of Probus, was exposed to the most imminent
danger, and the enraged soldiers were forcing their way into his
tent, when they were checked by the single arm of the Armenian
prince. The gratitude of Tiridates contributed soon afterwards to
his restoration. Licinius was in every station the friend and
companion of Galerius, and the merit of Galerius, long before he
was raised to the dignity of Cæsar, had been known and esteemed
by Diocletian. In the third year of that emperor’s reign
Tiridates was invested with the kingdom of Armenia. The justice
of the measure was not less evident than its expediency. It was
time to rescue from the usurpation of the Persian monarch an
important territory, which, since the reign of Nero, had been
always granted under the protection of the empire to a younger
branch of the house of Arsaces.\textsuperscript{54}

\pagenote[52]{See the education and strength of Tiridates in the
Armenian history of Moses of Chorene, l. ii. c. 76. He could
seize two wild bulls by the horns, and break them off with his
hands.}

\pagenote[53]{If we give credit to the younger Victor, who
supposes that in the year 323 Licinius was only sixty years of
age, he could scarcely be the same person as the patron of
Tiridates; but we know from much better authority, (Euseb. Hist.
Ecclesiast. l. x. c. 8,) that Licinius was at that time in the
last period of old age: sixteen years before, he is represented
with gray hairs, and as the contemporary of Galerius. See
Lactant. c. 32. Licinius was probably born about the year 250.}

\pagenote[54]{See the sixty-second and sixty-third books of Dion
Cassius.}

When Tiridates appeared on the frontiers of Armenia, he was
received with an unfeigned transport of joy and loyalty. During
twenty-six years, the country had experienced the real and
imaginary hardships of a foreign yoke. The Persian monarchs
adorned their new conquest with magnificent buildings; but those
monuments had been erected at the expense of the people, and were
abhorred as badges of slavery. The apprehension of a revolt had
inspired the most rigorous precautions: oppression had been
aggravated by insult, and the consciousness of the public hatred
had been productive of every measure that could render it still
more implacable. We have already remarked the intolerant spirit
of the Magian religion. The statues of the deified kings of
Armenia, and the sacred images of the sun and moon, were broke in
pieces by the zeal of the conqueror; and the perpetual fire of
Ormuzd was kindled and preserved upon an altar erected on the
summit of Mount Bagavan.\textsuperscript{55} It was natural, that a people
exasperated by so many injuries, should arm with zeal in the
cause of their independence, their religion, and their hereditary
sovereign. The torrent bore down every obstacle, and the Persian
garrisons retreated before its fury. The nobles of Armenia flew
to the standard of Tiridates, all alleging their past merit,
offering their future service, and soliciting from the new king
those honors and rewards from which they had been excluded with
disdain under the foreign government.\textsuperscript{56} The command of the army
was bestowed on Artavasdes, whose father had saved the infancy of
Tiridates, and whose family had been massacred for that generous
action. The brother of Artavasdes obtained the government of a
province. One of the first military dignities was conferred on
the satrap Otas, a man of singular temperance and fortitude, who
presented to the king his sister\textsuperscript{57} and a considerable treasure,
both of which, in a sequestered fortress, Otas had preserved from
violation. Among the Armenian nobles appeared an ally, whose
fortunes are too remarkable to pass unnoticed. His name was
Mamgo,\textsuperscript{571} his origin was Scythian, and the horde which
acknowledge his authority had encamped a very few years before on
the skirts of the Chinese empire,\textsuperscript{58} which at that time extended
as far as the neighborhood of Sogdiana.\textsuperscript{59} Having incurred the
displeasure of his master, Mamgo, with his followers, retired to
the banks of the Oxus, and implored the protection of Sapor. The
emperor of China claimed the fugitive, and alleged the rights of
sovereignty. The Persian monarch pleaded the laws of hospitality,
and with some difficulty avoided a war, by the promise that he
would banish Mamgo to the uttermost parts of the West, a
punishment, as he described it, not less dreadful than death
itself. Armenia was chosen for the place of exile, and a large
district was assigned to the Scythian horde, on which they might
feed their flocks and herds, and remove their encampment from one
place to another, according to the different seasons of the year.

They were employed to repel the invasion of Tiridates; but their
leader, after weighing the obligations and injuries which he had
received from the Persian monarch, resolved to abandon his party.

The Armenian prince, who was well acquainted with the merit as
well as power of Mamgo, treated him with distinguished respect;
and, by admitting him into his confidence, acquired a brave and
faithful servant, who contributed very effectually to his
restoration.\textsuperscript{60}

\pagenote[55]{Moses of Chorene. Hist. Armen. l. ii. c. 74. The
statues had been erected by Valarsaces, who reigned in Armenia
about 130 years before Christ, and was the first king of the
family of Arsaces, (see Moses, Hist. Armen. l. ii. 2, 3.) The
deification of the Arsacides is mentioned by Justin, (xli. 5,)
and by Ammianus Marcellinus, (xxiii. 6.)}

\pagenote[56]{The Armenian nobility was numerous and powerful.
Moses mentions many families which were distinguished under the
reign of Valarsaces, (l. ii. 7,) and which still subsisted in his
own time, about the middle of the fifth century. See the preface
of his Editors.}

\pagenote[57]{She was named Chosroiduchta, and had not the os
patulum like other women. (Hist. Armen. l. ii. c. 79.) I do not
understand the expression. * Note: Os patulum signifies merely a
large and widely opening mouth. Ovid (Metam. xv. 513) says,
speaking of the monster who attacked Hippolytus, patulo partem
maris evomit ore. Probably a wide mouth was a common defect among
the Armenian women.—G.}

\pagenote[571]{Mamgo (according to M. St. Martin, note to Le
Beau. ii. 213) belonged to the imperial race of Hon, who had
filled the throne of China for four hundred years. Dethroned by
the usurping race of Wei, Mamgo found a hospitable reception in
Persia in the reign of Ardeschir. The emperor of china having
demanded the surrender of the fugitive and his partisans, Sapor,
then king, threatened with war both by Rome and China, counselled
Mamgo to retire into Armenia. “I have expelled him from my
dominions, (he answered the Chinese ambassador;) I have banished
him to the extremity of the earth, where the sun sets; I have
dismissed him to certain death.” Compare Mem. sur l’Armenie, ii.
25.—M.}

\pagenote[58]{In the Armenian history, (l. ii. 78,) as well as in
the Geography, (p. 367,) China is called Zenia, or Zenastan. It
is characterized by the production of silk, by the opulence of
the natives, and by their love of peace, above all the other
nations of the earth. * Note: See St. Martin, Mem. sur l’Armenie,
i. 304.}

\pagenote[59]{Vou-ti, the first emperor of the seventh dynasty,
who then reigned in China, had political transactions with
Fergana, a province of Sogdiana, and is said to have received a
Roman embassy, (Histoire des Huns, tom. i. p. 38.) In those ages
the Chinese kept a garrison at Kashgar, and one of their
generals, about the time of Trajan, marched as far as the Caspian
Sea. With regard to the intercourse between China and the Western
countries, a curious memoir of M. de Guignes may be consulted, in
the Academie des Inscriptions, tom. xxii. p. 355. * Note: The
Chinese Annals mention, under the ninth year of Yan-hi, which
corresponds with the year 166 J. C., an embassy which arrived
from Tathsin, and was sent by a prince called An-thun, who can be
no other than Marcus Aurelius Antoninus, who then ruled over the
Romans. St. Martin, Mem. sur l’Armænic. ii. 30. See also
Klaproth, Tableaux Historiques de l’Asie, p. 69. The embassy came
by Jy-nan, Tonquin.—M.}

\pagenote[60]{See Hist. Armen. l. ii. c. 81.}

For a while, fortune appeared to favor the enterprising valor of
Tiridates. He not only expelled the enemies of his family and
country from the whole extent of Armenia, but in the prosecution
of his revenge he carried his arms, or at least his incursions,
into the heart of Assyria. The historian, who has preserved the
name of Tiridates from oblivion, celebrates, with a degree of
national enthusiasm, his personal prowess: and, in the true
spirit of eastern romance, describes the giants and the elephants
that fell beneath his invincible arm. It is from other
information that we discover the distracted state of the Persian
monarchy, to which the king of Armenia was indebted for some part
of his advantages. The throne was disputed by the ambition of
contending brothers; and Hormuz, after exerting without success
the strength of his own party, had recourse to the dangerous
assistance of the barbarians who inhabited the banks of the
Caspian Sea.\textsuperscript{61} The civil war was, however, soon terminated,
either by a victor or by a reconciliation; and Narses, who was
universally acknowledged as king of Persia, directed his whole
force against the foreign enemy. The contest then became too
unequal; nor was the valor of the hero able to withstand the
power of the monarch. Tiridates, a second time expelled from the
throne of Armenia, once more took refuge in the court of the
emperors.\textsuperscript{611} Narses soon reëstablished his authority over the
revolted province; and loudly complaining of the protection
afforded by the Romans to rebels and fugitives, aspired to the
conquest of the East.\textsuperscript{62}

\pagenote[61]{Ipsos Persas ipsumque Regem ascitis Saccis, et
Russis, et Gellis, petit frater Ormies. Panegyric. Vet. iii. 1.
The Saccæ were a nation of wandering Scythians, who encamped
towards the sources of the Oxus and the Jaxartes. The Gelli where
the inhabitants of Ghilan, along the Caspian Sea, and who so
long, under the name of Dilemines, infested the Persian monarchy.
See d’Herbelot, Bibliotheque}

\pagenote[611]{M St. Martin represents this differently. Le roi
de Perse * * * profits d’un voyage que Tiridate avoit fait a Rome
pour attaquer ce royaume. This reads like the evasion of the
national historians to disguise the fact discreditable to their
hero. See Mem. sur l’Armenie, i. 304.—M.}

\pagenote[62]{Moses of Chorene takes no notice of this second
revolution, which I have been obliged to collect from a passage
of Ammianus Marcellinus, (l. xxiii. c. 5.) Lactantius speaks of
the ambition of Narses: “Concitatus domesticis exemplis avi sui
Saporis ad occupandum orientem magnis copiis inhiabat.” De Mort.
Persecut. c. 9.}

Neither prudence nor honor could permit the emperors to forsake
the cause of the Armenian king, and it was resolved to exert the
force of the empire in the Persian war. Diocletian, with the calm
dignity which he constantly assumed, fixed his own station in the
city of Antioch, from whence he prepared and directed the
military operations.\textsuperscript{63} The conduct of the legions was intrusted
to the intrepid valor of Galerius, who, for that important
purpose, was removed from the banks of the Danube to those of the
Euphrates. The armies soon encountered each other in the plains
of Mesopotamia, and two battles were fought with various and
doubtful success; but the third engagement was of a more decisive
nature; and the Roman army received a total overthrow, which is
attributed to the rashness of Galerius, who, with an
inconsiderable body of troops, attacked the innumerable host of
the Persians.\textsuperscript{64} But the consideration of the country that was
the scene of action, may suggest another reason for his defeat.
The same ground on which Galerius was vanquished, had been
rendered memorable by the death of Crassus, and the slaughter of
ten legions. It was a plain of more than sixty miles, which
extended from the hills of Carrhæ to the Euphrates; a smooth and
barren surface of sandy desert, without a hillock, without a
tree, and without a spring of fresh water.\textsuperscript{65} The steady infantry
of the Romans, fainting with heat and thirst, could neither hope
for victory if they preserved their ranks, nor break their ranks
without exposing themselves to the most imminent danger. In this
situation they were gradually encompassed by the superior
numbers, harassed by the rapid evolutions, and destroyed by the
arrows of the barbarian cavalry.

The king of Armenia had signalized his valor in the battle, and
acquired personal glory by the public misfortune. He was pursued
as far as the Euphrates; his horse was wounded, and it appeared
impossible for him to escape the victorious enemy. In this
extremity Tiridates embraced the only refuge which appeared
before him: he dismounted and plunged into the stream. His armor
was heavy, the river very deep, and at those parts at least half
a mile in breadth;\textsuperscript{66} yet such was his strength and dexterity,
that he reached in safety the opposite bank.\textsuperscript{67} With regard to
the Roman general, we are ignorant of the circumstances of his
escape; but when he returned to Antioch, Diocletian received him,
not with the tenderness of a friend and colleague, but with the
indignation of an offended sovereign. The haughtiest of men,
clothed in his purple, but humbled by the sense of his fault and
misfortune, was obliged to follow the emperor’s chariot above a
mile on foot, and to exhibit, before the whole court, the
spectacle of his disgrace.\textsuperscript{68}

\pagenote[63]{We may readily believe, that Lactantius ascribes to
cowardice the conduct of Diocletian. Julian, in his oration,
says, that he remained with all the forces of the empire; a very
hyperbolical expression.}

\pagenote[64]{Our five abbreviators, Eutropius, Festus, the two
Victors, and Orosius, all relate the last and great battle; but
Orosius is the only one who speaks of the two former.}

\pagenote[65]{The nature of the country is finely described by
Plutarch, in the life of Crassus; and by Xenophon, in the first
book of the Anabasis}

\pagenote[66]{See Foster’s Dissertation in the second volume of
the translation of the Anabasis by Spelman; which I will venture
to recommend as one of the best versions extant.}

\pagenote[67]{Hist. Armen. l. ii. c. 76. I have transferred this
exploit of Tiridates from an imaginary defeat to the real one of
Galerius.}

\pagenote[68]{Ammian. Marcellin. l. xiv. The mile, in the hands
of Eutropoius, (ix. 24,) of Festus (c. 25,) and of Orosius, (vii
25), easily increased to several miles}

As soon as Diocletian had indulged his private resentment, and
asserted the majesty of supreme power, he yielded to the
submissive entreaties of the Cæsar, and permitted him to retrieve
his own honor, as well as that of the Roman arms. In the room of
the unwarlike troops of Asia, which had most probably served in
the first expedition, a second army was drawn from the veterans
and new levies of the Illyrian frontier, and a considerable body
of Gothic auxiliaries were taken into the Imperial pay.\textsuperscript{69} At the
head of a chosen army of twenty-five thousand men, Galerius again
passed the Euphrates; but, instead of exposing his legions in the
open plains of Mesopotamia he advanced through the mountains of
Armenia, where he found the inhabitants devoted to his cause, and
the country as favorable to the operations of infantry as it was
inconvenient for the motions of cavalry.\textsuperscript{70} Adversity had
confirmed the Roman discipline, while the barbarians, elated by
success, were become so negligent and remiss, that in the moment
when they least expected it, they were surprised by the active
conduct of Galerius, who, attended only by two horsemen, had with
his own eyes secretly examined the state and position of their
camp. A surprise, especially in the night time, was for the most
part fatal to a Persian army. “Their horses were tied, and
generally shackled, to prevent their running away; and if an
alarm happened, a Persian had his housing to fix, his horse to
bridle, and his corselet to put on, before he could mount.”\textsuperscript{71} On
this occasion, the impetuous attack of Galerius spread disorder
and dismay over the camp of the barbarians. A slight resistance
was followed by a dreadful carnage, and, in the general
confusion, the wounded monarch (for Narses commanded his armies
in person) fled towards the deserts of Media. His sumptuous
tents, and those of his satraps, afforded an immense booty to the
conqueror; and an incident is mentioned, which proves the rustic
but martial ignorance of the legions in the elegant superfluities
of life. A bag of shining leather, filled with pearls, fell into
the hands of a private soldier; he carefully preserved the bag,
but he threw away its contents, judging that whatever was of no
use could not possibly be of any value.\textsuperscript{72} The principal loss of
Narses was of a much more affecting nature. Several of his wives,
his sisters, and children, who had attended the army, were made
captives in the defeat. But though the character of Galerius had
in general very little affinity with that of Alexander, he
imitated, after his victory, the amiable behavior of the
Macedonian towards the family of Darius. The wives and children
of Narses were protected from violence and rapine, conveyed to a
place of safety, and treated with every mark of respect and
tenderness, that was due from a generous enemy to their age,
their sex, and their royal dignity.\textsuperscript{73}

\pagenote[69]{Aurelius Victor. Jornandes de Rebus Geticis, c.
21.}

\pagenote[70]{Aurelius Victor says, “Per Armeniam in hostes
contendit, quæ fermo sola, seu facilior vincendi via est.” He
followed the conduct of Trajan, and the idea of Julius Cæsar.}

\pagenote[71]{Xenophon’s Anabasis, l. iii. For that reason the
Persian cavalry encamped sixty stadia from the enemy.}

\pagenote[72]{The story is told by Ammianus, l. xxii. Instead of
saccum, some read scutum.}

\pagenote[73]{The Persians confessed the Roman superiority in
morals as well as in arms. Eutrop. ix. 24. But this respect and
gratitude of enemies is very seldom to be found in their own
accounts.}

\section{Part \thesection.}

While the East anxiously expected the decision of this great
contest, the emperor Diocletian, having assembled in Syria a
strong army of observation, displayed from a distance the
resources of the Roman power, and reserved himself for any future
emergency of the war. On the intelligence of the victory he
condescended to advance towards the frontier, with a view of
moderating, by his presence and counsels, the pride of Galerius.
The interview of the Roman princes at Nisibis was accompanied
with every expression of respect on one side, and of esteem on
the other. It was in that city that they soon afterwards gave
audience to the ambassador of the Great King.\textsuperscript{74} The power, or at
least the spirit, of Narses, had been broken by his last defeat;
and he considered an immediate peace as the only means that could
stop the progress of the Roman arms. He despatched Apharban, a
servant who possessed his favor and confidence, with a commission
to negotiate a treaty, or rather to receive whatever conditions
the conqueror should impose. Apharban opened the conference by
expressing his master’s gratitude for the generous treatment of
his family, and by soliciting the liberty of those illustrious
captives. He celebrated the valor of Galerius, without degrading
the reputation of Narses, and thought it no dishonor to confess
the superiority of the victorious Cæsar, over a monarch who had
surpassed in glory all the princes of his race. Notwithstanding
the justice of the Persian cause, he was empowered to submit the
present differences to the decision of the emperors themselves;
convinced as he was, that, in the midst of prosperity, they would
not be unmindful of the vicissitudes of fortune. Apharban
concluded his discourse in the style of eastern allegory, by
observing that the Roman and Persian monarchies were the two eyes
of the world, which would remain imperfect and mutilated if
either of them should be put out.

\pagenote[74]{The account of the negotiation is taken from the
fragments of Peter the Patrician, in the Excerpta Legationum,
published in the Byzantine Collection. Peter lived under
Justinian; but it is very evident, by the nature of his
materials, that they are drawn from the most authentic and
respectable writers.}

“It well becomes the Persians,” replied Galerius, with a
transport of fury, which seemed to convulse his whole frame, “it
well becomes the Persians to expatiate on the vicissitudes of
fortune, and calmly to read us lectures on the virtues of
moderation. Let them remember their own \textit{moderation} towards the
unhappy Valerian. They vanquished him by fraud, they treated him
with indignity. They detained him till the last moment of his
life in shameful captivity, and after his death they exposed his
body to perpetual ignominy.” Softening, however, his tone,
Galerius insinuated to the ambassador, that it had never been the
practice of the Romans to trample on a prostrate enemy; and that,
on this occasion, they should consult their own dignity rather
than the Persian merit. He dismissed Apharban with a hope that
Narses would soon be informed on what conditions he might obtain,
from the clemency of the emperors, a lasting peace, and the
restoration of his wives and children. In this conference we may
discover the fierce passions of Galerius, as well as his
deference to the superior wisdom and authority of Diocletian. The
ambition of the former grasped at the conquest of the East, and
had proposed to reduce Persia into the state of a province. The
prudence of the latter, who adhered to the moderate policy of
Augustus and the Antonines, embraced the favorable opportunity of
terminating a successful war by an honorable and advantageous
peace.\textsuperscript{75}

\pagenote[75]{Adeo victor (says Aurelius) ut ni Valerius, cujus
nutu omnis gerebantur, abnuisset, Romani fasces in provinciam
novam ferrentur Verum pars terrarum tamen nobis utilior quæsita.}

In pursuance of their promise, the emperors soon afterwards
appointed Sicorius Probus, one of their secretaries, to acquaint
the Persian court with their final resolution. As the minister of
peace, he was received with every mark of politeness and
friendship; but, under the pretence of allowing him the necessary
repose after so long a journey, the audience of Probus was
deferred from day to day; and he attended the slow motions of the
king, till at length he was admitted to his presence, near the
River Asprudus in Media. The secret motive of Narses, in this
delay, had been to collect such a military force as might enable
him, though sincerely desirous of peace, to negotiate with the
greater weight and dignity. Three persons only assisted at this
important conference, the minister Apharban, the præfect of the
guards, and an officer who had commanded on the Armenian
frontier.\textsuperscript{76} The first condition proposed by the ambassador is
not at present of a very intelligible nature; that the city of
Nisibis might be established for the place of mutual exchange,
or, as we should formerly have termed it, for the staple of
trade, between the two empires. There is no difficulty in
conceiving the intention of the Roman princes to improve their
revenue by some restraints upon commerce; but as Nisibis was
situated within their own dominions, and as they were masters
both of the imports and exports, it should seem that such
restraints were the objects of an internal law, rather than of a
foreign treaty. To render them more effectual, some stipulations
were probably required on the side of the king of Persia, which
appeared so very repugnant either to his interest or to his
dignity, that Narses could not be persuaded to subscribe them. As
this was the only article to which he refused his consent, it was
no longer insisted on; and the emperors either suffered the trade
to flow in its natural channels, or contented themselves with
such restrictions, as it depended on their own authority to
establish.

\pagenote[76]{He had been governor of Sumium, (Pot. Patricius in
Excerpt. Legat. p. 30.) This province seems to be mentioned by
Moses of Chorene, (Geograph. p. 360,) and lay to the east of
Mount Ararat. * Note: The Siounikh of the Armenian writers St.
Martin i. 142.—M.}

As soon as this difficulty was removed, a solemn peace was
concluded and ratified between the two nations. The conditions of
a treaty so glorious to the empire, and so necessary to Persia,
may deserve a more peculiar attention, as the history of Rome
presents very few transactions of a similar nature; most of her
wars having either been terminated by absolute conquest, or waged
against barbarians ignorant of the use of letters. I. The Aboras,
or, as it is called by Xenophon, the Araxes, was fixed as the
boundary between the two monarchies.\textsuperscript{77} That river, which rose
near the Tigris, was increased, a few miles below Nisibis, by the
little stream of the Mygdonius, passed under the walls of
Singara, and fell into the Euphrates at Circesium, a frontier
town, which, by the care of Diocletian, was very strongly
fortified.\textsuperscript{78} Mesopotomia, the object of so many wars, was ceded
to the empire; and the Persians, by this treaty, renounced all
pretensions to that great province. II. They relinquished to the
Romans five provinces beyond the Tigris.\textsuperscript{79} Their situation
formed a very useful barrier, and their natural strength was soon
improved by art and military skill. Four of these, to the north
of the river, were districts of obscure fame and inconsiderable
extent; Intiline, Zabdicene, Arzanene, and Moxoene;\textsuperscript{791} but on
the east of the Tigris, the empire acquired the large and
mountainous territory of Carduene, the ancient seat of the
Carduchians, who preserved for many ages their manly freedom in
the heart of the despotic monarchies of Asia. The ten thousand
Greeks traversed their country, after a painful march, or rather
engagement, of seven days; and it is confessed by their leader,
in his incomparable relation of the retreat, that they suffered
more from the arrows of the Carduchians, than from the power of
the Great King.\textsuperscript{80} Their posterity, the Curds, with very little
alteration either of name or manners,\textsuperscript{801} acknowledged the
nominal sovereignty of the Turkish sultan. III. It is almost
needless to observe, that Tiridates, the faithful ally of Rome,
was restored to the throne of his fathers, and that the rights of
the Imperial supremacy were fully asserted and secured. The
limits of Armenia were extended as far as the fortress of Sintha
in Media, and this increase of dominion was not so much an act of
liberality as of justice. Of the provinces already mentioned
beyond the Tigris, the four first had been dismembered by the
Parthians from the crown of Armenia;\textsuperscript{81} and when the Romans
acquired the possession of them, they stipulated, at the expense
of the usurpers, an ample compensation, which invested their ally
with the extensive and fertile country of Atropatene. Its
principal city, in the same situation perhaps as the modern
Tauris, was frequently honored by the residence of Tiridates; and
as it sometimes bore the name of Ecbatana, he imitated, in the
buildings and fortifications, the splendid capital of the Medes.\textsuperscript{82}
IV. The country of Iberia was barren, its inhabitants rude and
savage. But they were accustomed to the use of arms, and they
separated from the empire barbarians much fiercer and more
formidable than themselves. The narrow defiles of Mount Caucasus
were in their hands, and it was in their choice, either to admit
or to exclude the wandering tribes of Sarmatia, whenever a
rapacious spirit urged them to penetrate into the richer climes
of the South.\textsuperscript{83} The nomination of the kings of Iberia, which was
resigned by the Persian monarch to the emperors, contributed to
the strength and security of the Roman power in Asia.\textsuperscript{84} The East
enjoyed a profound tranquillity during forty years; and the
treaty between the rival monarchies was strictly observed till
the death of Tiridates; when a new generation, animated with
different views and different passions, succeeded to the
government of the world; and the grandson of Narses undertook a
long and memorable war against the princes of the house of
Constantine.

\pagenote[77]{By an error of the geographer Ptolemy, the position
of Singara is removed from the Aboras to the Tigris, which may
have produced the mistake of Peter, in assigning the latter river
for the boundary, instead of the former. The line of the Roman
frontier traversed, but never followed, the course of the Tigris.
* Note: There are here several errors. Gibbon has confounded the
streams, and the towns which they pass. The Aboras, or rather the
Chaboras, the Araxes of Xenophon, has its source above Ras-Ain or
Re-Saina, (Theodosiopolis,) about twenty-seven leagues from the
Tigris; it receives the waters of the Mygdonius, or Saocoras,
about thirty-three leagues below Nisibis. at a town now called Al
Nahraim; it does not pass under the walls of Singara; it is the
Saocoras that washes the walls of that town: the latter river has
its source near Nisibis. at five leagues from the Tigris. See
D’Anv. l’Euphrate et le Tigre, 46, 49, 50, and the map.—— To the
east of the Tigris is another less considerable river, named also
the Chaboras, which D’Anville calls the Centrites, Khabour,
Nicephorius, without quoting the authorities on which he gives
those names. Gibbon did not mean to speak of this river, which
does not pass by Singara, and does not fall into the Euphrates.
See Michaelis, Supp. ad Lex. Hebraica. 3d part, p. 664, 665.—G.}

\pagenote[78]{Procopius de Edificiis, l. ii. c. 6.}

\pagenote[79]{Three of the provinces, Zabdicene, Arzanene, and
Carduene, are allowed on all sides. But instead of the other two,
Peter (in Excerpt. Leg. p. 30) inserts Rehimene and Sophene. I
have preferred Ammianus, (l. xxv. 7,) because it might be proved
that Sophene was never in the hands of the Persians, either
before the reign of Diocletian, or after that of Jovian. For want
of correct maps, like those of M. d’Anville, almost all the
moderns, with Tillemont and Valesius at their head, have
imagined, that it was in respect to Persia, and not to Rome, that
the five provinces were situate beyond the Tigris.}

\pagenote[791]{See St. Martin, note on Le Beau, i. 380. He would
read, for Intiline, Ingeleme, the name of a small province of
Armenia, near the sources of the Tigris, mentioned by St.
Epiphanius, (Hæres, 60;) for the unknown name Arzacene, with
Gibbon, Arzanene. These provinces do not appear to have made an
integral part of the Roman empire; Roman garrisons replaced those
of Persia, but the sovereignty remained in the hands of the
feudatory princes of Armenia. A prince of Carduene, ally or
dependent on the empire, with the Roman name of Jovianus, occurs
in the reign of Julian.—M.}

\pagenote[80]{Xenophon’s Anabasis, l. iv. Their bows were three
cubits in length, their arrows two; they rolled down stones that
were each a wagon load. The Greeks found a great many villages in
that rude country.}

\pagenote[801]{I travelled through this country in 1810, and
should judge, from what I have read and seen of its inhabitants,
that they have remained unchanged in their appearance and
character for more than twenty centuries Malcolm, note to Hist.
of Persia, vol. i. p. 82.—M.}

\pagenote[81]{According to Eutropius, (vi. 9, as the text is
represented by the best Mss.,) the city of Tigranocerta was in
Arzanene. The names and situation of the other three may be
faintly traced.}

\pagenote[82]{Compare Herodotus, l. i. c. 97, with Moses
Choronens. Hist Armen. l. ii. c. 84, and the map of Armenia given
by his editors.}

\pagenote[83]{Hiberi, locorum potentes, Caspia via Sarmatam in
Armenios raptim effundunt. Tacit. Annal. vi. 34. See Strabon.
Geograph. l. xi. p. 764, edit. Casaub.}

\pagenote[84]{Peter Patricius (in Excerpt. Leg. p. 30) is the
only writer who mentions the Iberian article of the treaty.}

The arduous work of rescuing the distressed empire from tyrants
and barbarians had now been completely achieved by a succession
of Illyrian peasants. As soon as Diocletian entered into the
twentieth year of his reign, he celebrated that memorable æra, as
well as the success of his arms, by the pomp of a Roman triumph.\textsuperscript{85}
Maximian, the equal partner of his power, was his only
companion in the glory of that day. The two Cæsars had fought and
conquered, but the merit of their exploits was ascribed,
according to the rigor of ancient maxims, to the auspicious
influence of their fathers and emperors.\textsuperscript{86} The triumph of
Diocletian and Maximian was less magnificent, perhaps, than those
of Aurelian and Probus, but it was dignified by several
circumstances of superior fame and good fortune. Africa and
Britain, the Rhine, the Danube, and the Nile, furnished their
respective trophies; but the most distinguished ornament was of a
more singular nature, a Persian victory followed by an important
conquest. The representations of rivers, mountains, and
provinces, were carried before the Imperial car. The images of
the captive wives, the sisters, and the children of the Great
King, afforded a new and grateful spectacle to the vanity of the
people.\textsuperscript{87} In the eyes of posterity, this triumph is remarkable,
by a distinction of a less honorable kind. It was the last that
Rome ever beheld. Soon after this period, the emperors ceased to
vanquish, and Rome ceased to be the capital of the empire.

\pagenote[85]{Euseb. in Chron. Pagi ad annum. Till the discovery
of the treatise De Mortibus Persecutorum, it was not certain that
the triumph and the Vicennalia was celebrated at the same time.}

\pagenote[86]{At the time of the Vicennalia, Galerius seems to
have kept station on the Danube. See Lactant. de M. P. c. 38.}

\pagenote[87]{Eutropius (ix. 27) mentions them as a part of the
triumph. As the persons had been restored to Narses, nothing more
than their images could be exhibited.}

The spot on which Rome was founded had been consecrated by
ancient ceremonies and imaginary miracles. The presence of some
god, or the memory of some hero, seemed to animate every part of
the city, and the empire of the world had been promised to the
Capitol.\textsuperscript{88} The native Romans felt and confessed the power of
this agreeable illusion. It was derived from their ancestors, had
grown up with their earliest habits of life, and was protected,
in some measure, by the opinion of political utility. The form
and the seat of government were intimately blended together, nor
was it esteemed possible to transport the one without destroying
the other.\textsuperscript{89} But the sovereignty of the capital was gradually
annihilated in the extent of conquest; the provinces rose to the
same level, and the vanquished nations acquired the name and
privileges, without imbibing the partial affections, of Romans.
During a long period, however, the remains of the ancient
constitution, and the influence of custom, preserved the dignity
of Rome. The emperors, though perhaps of African or Illyrian
extraction, respected their adopted country, as the seat of their
power, and the centre of their extensive dominions. The
emergencies of war very frequently required their presence on the
frontiers; but Diocletian and Maximian were the first Roman
princes who fixed, in time of peace, their ordinary residence in
the provinces; and their conduct, however it might be suggested
by private motives, was justified by very specious considerations
of policy. The court of the emperor of the West was, for the most
part, established at Milan, whose situation, at the foot of the
Alps, appeared far more convenient than that of Rome, for the
important purpose of watching the motions of the barbarians of
Germany. Milan soon assumed the splendor of an Imperial city. The
houses are described as numerous and well built; the manners of
the people as polished and liberal. A circus, a theatre, a mint,
a palace, baths, which bore the name of their founder Maximian;
porticos adorned with statues, and a double circumference of
walls, contributed to the beauty of the new capital; nor did it
seem oppressed even by the proximity of Rome.\textsuperscript{90} To rival the
majesty of Rome was the ambition likewise of Diocletian, who
employed his leisure, and the wealth of the East, in the
embellishment of Nicomedia, a city placed on the verge of Europe
and Asia, almost at an equal distance between the Danube and the
Euphrates. By the taste of the monarch, and at the expense of the
people, Nicomedia acquired, in the space of a few years, a degree
of magnificence which might appear to have required the labor of
ages, and became inferior only to Rome, Alexandria, and Antioch,
in extent of populousness.\textsuperscript{91} The life of Diocletian and Maximian
was a life of action, and a considerable portion of it was spent
in camps, or in the long and frequent marches; but whenever the
public business allowed them any relaxation, they seemed to have
retired with pleasure to their favorite residences of Nicomedia
and Milan. Till Diocletian, in the twentieth year of his reign,
celebrated his Roman triumph, it is extremely doubtful whether he
ever visited the ancient capital of the empire. Even on that
memorable occasion his stay did not exceed two months. Disgusted
with the licentious familiarity of the people, he quitted Rome
with precipitation thirteen days before it was expected that he
should have appeared in the senate, invested with the ensigns of
the consular dignity.\textsuperscript{92}

\pagenote[88]{Livy gives us a speech of Camillus on that subject,
(v. 51—55,) full of eloquence and sensibility, in opposition to a
design of removing the seat of government from Rome to the
neighboring city of Veii.}

\pagenote[89]{Julius Cæsar was reproached with the intention of
removing the empire to Ilium or Alexandria. See Sueton. in Cæsar.
c. 79. According to the ingenious conjecture of Le Fevre and
Dacier, the ode of the third book of Horace was intended to
divert from the execution of a similar design.}

\pagenote[90]{See Aurelius Victor, who likewise mentions the
buildings erected by Maximian at Carthage, probably during the
Moorish war. We shall insert some verses of Ausonius de Clar.
Urb. v.—— Et Mediolani miræomnia: copia rerum; Innumeræ cultæque
domus; facunda virorum Ingenia, et mores læti: tum duplice muro
Amplificata loci species; populique voluptas Circus; et inclusi
moles cuneata Theatri; Templa, Palatinæque arces, opulensque
Moneta, Et regio Herculei celebris sub honore lavacri. Cunctaque
marmoreis ornata Peristyla signis; Moeniaque in valli formam
circumdata labro, Omnia quæ magnis operum velut æmula formis
Excellunt: nec juncta premit vicinia Romæ.}

\pagenote[91]{Lactant. de M. P. c. 17. Libanius, Orat. viii. p.
203.}

\pagenote[92]{Lactant. de M. P. c. 17. On a similar occasion,
Ammianus mentions the dicacitas plebis, as not very agreeable to
an Imperial ear. (See l. xvi. c. 10.)}

The dislike expressed by Diocletian towards Rome and Roman
freedom was not the effect of momentary caprice, but the result
of the most artful policy. That crafty prince had framed a new
system of Imperial government, which was afterwards completed by
the family of Constantine; and as the image of the old
constitution was religiously preserved in the senate, he resolved
to deprive that order of its small remains of power and
consideration. We may recollect, about eight years before the
elevation of Diocletian, the transient greatness, and the
ambitious hopes, of the Roman senate. As long as that enthusiasm
prevailed, many of the nobles imprudently displayed their zeal in
the cause of freedom; and after the successes of Probus had
withdrawn their countenance from the republican party, the
senators were unable to disguise their impotent resentment.

As the sovereign of Italy, Maximian was intrusted with the care
of extinguishing this troublesome, rather than dangerous spirit,
and the task was perfectly suited to his cruel temper. The most
illustrious members of the senate, whom Diocletian always
affected to esteem, were involved, by his colleague, in the
accusation of imaginary plots; and the possession of an elegant
villa, or a well-cultivated estate, was interpreted as a
convincing evidence of guilt.\textsuperscript{93} The camp of the Prætorians,
which had so long oppressed, began to protect, the majesty of
Rome; and as those haughty troops were conscious of the decline
of their power, they were naturally disposed to unite their
strength with the authority of the senate. By the prudent
measures of Diocletian, the numbers of the Prætorians were
insensibly reduced, their privileges abolished,\textsuperscript{94} and their
place supplied by two faithful legions of Illyricum, who, under
the new titles of Jovians and Herculians, were appointed to
perform the service of the Imperial guards.\textsuperscript{95} But the most fatal
though secret wound, which the senate received from the hands of
Diocletian and Maximian, was inflicted by the inevitable
operation of their absence. As long as the emperors resided at
Rome, that assembly might be oppressed, but it could scarcely be
neglected. The successors of Augustus exercised the power of
dictating whatever laws their wisdom or caprice might suggest;
but those laws were ratified by the sanction of the senate. The
model of ancient freedom was preserved in its deliberations and
decrees; and wise princes, who respected the prejudices of the
Roman people, were in some measure obliged to assume the language
and behavior suitable to the general and first magistrate of the
republic. In the armies and in the provinces, they displayed the
dignity of monarchs; and when they fixed their residence at a
distance from the capital, they forever laid aside the
dissimulation which Augustus had recommended to his successors.
In the exercise of the legislative as well as the executive
power, the sovereign advised with his ministers, instead of
consulting the great council of the nation. The name of the
senate was mentioned with honor till the last period of the
empire; the vanity of its members was still flattered with
honorary distinctions;\textsuperscript{96} but the assembly which had so long been
the source, and so long the instrument of power, was respectfully
suffered to sink into oblivion. The senate of Rome, losing all
connection with the Imperial court and the actual constitution,
was left a venerable but useless monument of antiquity on the
Capitoline hill.

\pagenote[93]{Lactantius accuses Maximian of destroying fictis
criminationibus lumina senatus, (De M. P. c. 8.) Aurelius Victor
speaks very doubtfully of the faith of Diocletian towards his
friends.}

\pagenote[94]{Truncatæ vires urbis, imminuto prætoriarum
cohortium atque in armis vulgi numero. Aurelius Victor.
Lactantius attributes to Galerius the prosecution of the same
plan, (c. 26.)}

\pagenote[95]{They were old corps stationed in Illyricum; and
according to the ancient establishment, they each consisted of
six thousand men. They had acquired much reputation by the use of
the plumbatæ, or darts loaded with lead. Each soldier carried
five of these, which he darted from a considerable distance, with
great strength and dexterity. See Vegetius, i. 17.}

\pagenote[96]{See the Theodosian Code, l. vi. tit. ii. with
Godefroy’s commentary.}

\section{Part \thesection.}

When the Roman princes had lost sight of the senate and of their
ancient capital, they easily forgot the origin and nature of
their legal power. The civil offices of consul, of proconsul, of
censor, and of tribune, by the union of which it had been formed,
betrayed to the people its republican extraction. Those modest
titles were laid aside;\textsuperscript{97} and if they still distinguished their
high station by the appellation of Emperor, or Imperator, that
word was understood in a new and more dignified sense, and no
longer denoted the general of the Roman armies, but the sovereign
of the Roman world. The name of Emperor, which was at first of a
military nature, was associated with another of a more servile
kind. The epithet of Dominus, or Lord, in its primitive
signification, was expressive not of the authority of a prince
over his subjects, or of a commander over his soldiers, but of
the despotic power of a master over his domestic slaves.\textsuperscript{98}
Viewing it in that odious light, it had been rejected with
abhorrence by the first Cæsars. Their resistance insensibly
became more feeble, and the name less odious; till at length the
style of \textit{our Lord and Emperor} was not only bestowed by
flattery, but was regularly admitted into the laws and public
monuments. Such lofty epithets were sufficient to elate and
satisfy the most excessive vanity; and if the successors of
Diocletian still declined the title of King, it seems to have
been the effect not so much of their moderation as of their
delicacy. Wherever the Latin tongue was in use, (and it was the
language of government throughout the empire,) the Imperial
title, as it was peculiar to themselves, conveyed a more
respectable idea than the name of king, which they must have
shared with a hundred barbarian chieftains; or which, at the
best, they could derive only from Romulus, or from Tarquin. But
the sentiments of the East were very different from those of the
West. From the earliest period of history, the sovereigns of Asia
had been celebrated in the Greek language by the title of
Basileus, or King; and since it was considered as the first
distinction among men, it was soon employed by the servile
provincials of the East, in their humble addresses to the Roman
throne.\textsuperscript{99} Even the attributes, or at least the titles, of the
DIVINITY, were usurped by Diocletian and Maximian, who
transmitted them to a succession of Christian emperors.\textsuperscript{100} Such
extravagant compliments, however, soon lose their impiety by
losing their meaning; and when the ear is once accustomed to the
sound, they are heard with indifference, as vague though
excessive professions of respect.

\pagenote[97]{See the 12th dissertation in Spanheim’s excellent
work de Usu Numismatum. From medals, inscriptions, and
historians, he examines every title separately, and traces it
from Augustus to the moment of its disappearing.}

\pagenote[98]{Pliny (in Panegyr. c. 3, 55, \&c.) speaks of Dominus
with execration, as synonymous to Tyrant, and opposite to Prince.
And the same Pliny regularly gives that title (in the tenth book
of the epistles) to his friend rather than master, the virtuous
Trajan. This strange contradiction puzzles the commentators, who
think, and the translators, who can write.}

\pagenote[99]{Synesius de Regno, edit. Petav. p. 15. I am
indebted for this quotation to the Abbé de la Bleterie.}

\pagenote[100]{Soe Vandale de Consecratione, p. 354, \&c. It was
customary for the emperors to mention (in the preamble of laws)
their numen, sacreo majesty, divine oracles, \&c. According to
Tillemont, Gregory Nazianzen complains most bitterly of the
profanation, especially when it was practised by an Arian
emperor. * Note: In the time of the republic, says Hegewisch,
when the consuls, the prætors, and the other magistrates appeared
in public, to perform the functions of their office, their
dignity was announced both by the symbols which use had
consecrated, and the brilliant cortege by which they were
accompanied. But this dignity belonged to the office, not to the
individual; this pomp belonged to the magistrate, not to the man.
* * The consul, followed, in the comitia, by all the senate, the
prætors, the quæstors, the ædiles, the lictors, the apparitors,
and the heralds, on reentering his house, was served only by
freedmen and by his slaves. The first emperors went no further.
Tiberius had, for his personal attendance, only a moderate number
of slaves, and a few freedmen. (Tacit. Ann. iv. 7.) But in
proportion as the republican forms disappeared, one after
another, the inclination of the emperors to environ themselves
with personal pomp, displayed itself more and more. ** The
magnificence and the ceremonial of the East were entirely
introduced by Diocletian, and were consecrated by Constantine to
the Imperial use. Thenceforth the palace, the court, the table,
all the personal attendance, distinguished the emperor from his
subjects, still more than his superior dignity. The organization
which Diocletian gave to his new court, attached less honor and
distinction to rank than to services performed towards the
members of the Imperial family. Hegewisch, Essai, Hist. sur les
Finances Romains. Few historians have characterized, in a more
philosophic manner, the influence of a new institution.—G.——It is
singular that the son of a slave reduced the haughty aristocracy
of Home to the offices of servitude.—M.}

From the time of Augustus to that of Diocletian, the Roman
princes, conversing in a familiar manner among their
fellow-citizens, were saluted only with the same respect that was
usually paid to senators and magistrates. Their principal
distinction was the Imperial or military robe of purple; whilst
the senatorial garment was marked by a broad, and the equestrian
by a narrow, band or stripe of the same honorable color. The
pride, or rather the policy, of Diocletian engaged that artful
prince to introduce the stately magnificence of the court of
Persia.\textsuperscript{101} He ventured to assume the diadem, an ornament
detested by the Romans as the odious ensign of royalty, and the
use of which had been considered as the most desperate act of the
madness of Caligula. It was no more than a broad white fillet set
with pearls, which encircled the emperor’s head. The sumptuous
robes of Diocletian and his successors were of silk and gold; and
it is remarked with indignation that even their shoes were
studded with the most precious gems. The access to their sacred
person was every day rendered more difficult by the institution
of new forms and ceremonies. The avenues of the palace were
strictly guarded by the various schools, as they began to be
called, of domestic officers. The interior apartments were
intrusted to the jealous vigilance of the eunuchs, the increase
of whose numbers and influence was the most infallible symptom of
the progress of despotism. When a subject was at length admitted
to the Imperial presence, he was obliged, whatever might be his
rank, to fall prostrate on the ground, and to adore, according to
the eastern fashion, the divinity of his lord and master.\textsuperscript{102}
Diocletian was a man of sense, who, in the course of private as
well as public life, had formed a just estimate both of himself
and of mankind; nor is it easy to conceive that in substituting
the manners of Persia to those of Rome he was seriously actuated
by so mean a principle as that of vanity. He flattered himself
that an ostentation of splendor and luxury would subdue the
imagination of the multitude; that the monarch would be less
exposed to the rude license of the people and the soldiers, as
his person was secluded from the public view; and that habits of
submission would insensibly be productive of sentiments of
veneration. Like the modesty affected by Augustus, the state
maintained by Diocletian was a theatrical representation; but it
must be confessed, that of the two comedies, the former was of a
much more liberal and manly character than the latter. It was the
aim of the one to disguise, and the object of the other to
display, the unbounded power which the emperors possessed over
the Roman world.

\pagenote[101]{See Spanheim de Usu Numismat. Dissert. xii.}

\pagenote[102]{Aurelius Victor. Eutropius, ix. 26. It appears by
the Panegyrists, that the Romans were soon reconciled to the name
and ceremony of adoration.}

Ostentation was the first principle of the new system instituted
by Diocletian. The second was division. He divided the empire,
the provinces, and every branch of the civil as well as military
administration. He multiplied the wheels of the machine of
government, and rendered its operations less rapid, but more
secure. Whatever advantages and whatever defects might attend
these innovations, they must be ascribed in a very great degree
to the first inventor; but as the new frame of policy was
gradually improved and completed by succeeding princes, it will
be more satisfactory to delay the consideration of it till the
season of its full maturity and perfection.\textsuperscript{103} Reserving,
therefore, for the reign of Constantine a more exact picture of
the new empire, we shall content ourselves with describing the
principal and decisive outline, as it was traced by the hand of
Diocletian. He had associated three colleagues in the exercise of
the supreme power; and as he was convinced that the abilities of
a single man were inadequate to the public defence, he considered
the joint administration of four princes not as a temporary
expedient, but as a fundamental law of the constitution. It was
his intention that the two elder princes should be distinguished
by the use of the diadem, and the title of \textit{Augusti;} that, as
affection or esteem might direct their choice, they should
regularly call to their assistance two subordinate colleagues;
and that the \textit{Cæsars}, rising in their turn to the first rank,
should supply an uninterrupted succession of emperors. The empire
was divided into four parts. The East and Italy were the most
honorable, the Danube and the Rhine the most laborious stations.
The former claimed the presence of the \textit{Augusti}, the latter were
intrusted to the administration of the \textit{Cæsars}. The strength of
the legions was in the hands of the four partners of sovereignty,
and the despair of successively vanquishing four formidable
rivals might intimidate the ambition of an aspiring general. In
their civil government the emperors were supposed to exercise the
undivided power of the monarch, and their edicts, inscribed with
their joint names, were received in all the provinces, as
promulgated by their mutual councils and authority.
Notwithstanding these precautions, the political union of the
Roman world was gradually dissolved, and a principle of division
was introduced, which, in the course of a few years, occasioned
the perpetual separation of the Eastern and Western Empires.

\pagenote[103]{The innovations introduced by Diocletian are
chiefly deduced, 1st, from some very strong passages in
Lactantius; and, 2dly, from the new and various offices which, in
the Theodosian code, appear already established in the beginning
of the reign of Constantine.}

The system of Diocletian was accompanied with another very
material disadvantage, which cannot even at present be totally
overlooked; a more expensive establishment, and consequently an
increase of taxes, and the oppression of the people. Instead of a
modest family of slaves and freedmen, such as had contented the
simple greatness of Augustus and Trajan, three or four
magnificent courts were established in the various parts of the
empire, and as many Roman \textit{kings} contended with each other and
with the Persian monarch for the vain superiority of pomp and
luxury. The number of ministers, of magistrates, of officers, and
of servants, who filled the different departments of the state,
was multiplied beyond the example of former times; and (if we may
borrow the warm expression of a contemporary) “when the
proportion of those who received exceeded the proportion of those
who contributed, the provinces were oppressed by the weight of
tributes.”\textsuperscript{104} From this period to the extinction of the empire,
it would be easy to deduce an uninterrupted series of clamors and
complaints. According to his religion and situation, each writer
chooses either Diocletian, or Constantine, or Valens, or
Theodosius, for the object of his invectives; but they
unanimously agree in representing the burden of the public
impositions, and particularly the land tax and capitation, as the
intolerable and increasing grievance of their own times. From
such a concurrence, an impartial historian, who is obliged to
extract truth from satire, as well as from panegyric, will be
inclined to divide the blame among the princes whom they accuse,
and to ascribe their exactions much less to their personal vices,
than to the uniform system of their administration.\textsuperscript{1041} The
emperor Diocletian was indeed the author of that system; but
during his reign, the growing evil was confined within the bounds
of modesty and discretion, and he deserves the reproach of
establishing pernicious precedents, rather than of exercising
actual oppression.\textsuperscript{105} It may be added, that his revenues were
managed with prudent economy; and that after all the current
expenses were discharged, there still remained in the Imperial
treasury an ample provision either for judicious liberality or
for any emergency of the state.

\pagenote[104]{Lactant. de M. P. c. 7.}

\pagenote[1041]{The most curious document which has come to light
since the publication of Gibbon’s History, is the edict of
Diocletian, published from an inscription found at Eskihissar,
(Stratoniccia,) by Col. Leake. This inscription was first copied
by Sherard, afterwards much more completely by Mr. Bankes. It is
confirmed and illustrated by a more imperfect copy of the same
edict, found in the Levant by a gentleman of Aix, and brought to
this country by M. Vescovali. This edict was issued in the name
of the four Cæsars, Diocletian, Maximian, Constantius, and
Galerius. It fixed a maximum of prices throughout the empire, for
all the necessaries and commodities of life. The preamble
insists, with great vehemence on the extortion and inhumanity of
the venders and merchants. Quis enim adeo obtunisi (obtusi)
pectores (is) et a sensu inhumanitatis extorris est qui ignorare
potest immo non senserit in venalibus rebus quævel in mercimoniis
aguntur vel diurna urbium conversatione tractantur, in tantum se
licen liam defusisse, ut effrænata libido rapien—rum copia nec
annorum ubertatibus mitigaretur. The edict, as Col. Leake clearly
shows, was issued A. C. 303. Among the articles of which the
maximum value is assessed, are oil, salt, honey, butchers’ meat,
poultry, game, fish, vegetables, fruit the wages of laborers and
artisans, schoolmasters and skins, boots and shoes, harness,
timber, corn, wine, and beer, (zythus.) The depreciation in the
value of money, or the rise in the price of commodities, had been
so great during the past century, that butchers’ meat, which, in
the second century of the empire, was in Rome about two denaril
the pound, was now fixed at a maximum of eight. Col. Leake
supposes the average price could not be less than four: at the
same time the maximum of the wages of the agricultural laborers
was twenty-five. The whole edict is, perhaps, the most gigantic
effort of a blind though well-intentioned despotism, to control
that which is, and ought to be, beyond the regulation of the
government. See an Edict of Diocletian, by Col. Leake, London,
1826. Col. Leake has not observed that this Edict is expressly
named in the treatise de Mort. Persecut. ch. vii. Idem cum variis
iniquitatibus immensam faceret caritatem, legem pretiis rerum
venalium statuere conatus.—M}

\pagenote[105]{Indicta lex nova quæ sane illorum temporum
modestia tolerabilis, in perniciem processit. Aurel. Victor., who
has treated the character of Diocletian with good sense, though
in bad Latin.}

It was in the twenty first year of his reign that Diocletian
executed his memorable resolution of abdicating the empire; an
action more naturally to have been expected from the elder or the
younger Antoninus, than from a prince who had never practised the
lessons of philosophy either in the attainment or in the use of
supreme power. Diocletian acquired the glory of giving to the
world the first example of a resignation,\textsuperscript{106} which has not been
very frequently imitated by succeeding monarchs. The parallel of
Charles the Fifth, however, will naturally offer itself to our
mind, not only since the eloquence of a modern historian has
rendered that name so familiar to an English reader, but from the
very striking resemblance between the characters of the two
emperors, whose political abilities were superior to their
military genius, and whose specious virtues were much less the
effect of nature than of art. The abdication of Charles appears
to have been hastened by the vicissitudes of fortune; and the
disappointment of his favorite schemes urged him to relinquish a
power which he found inadequate to his ambition. But the reign of
Diocletian had flowed with a tide of uninterrupted success; nor
was it till after he had vanquished all his enemies, and
accomplished all his designs, that he seems to have entertained
any serious thoughts of resigning the empire. Neither Charles nor
Diocletian were arrived at a very advanced period of life; since
the one was only fifty-five, and the other was no more than
fifty-nine years of age; but the active life of those princes,
their wars and journeys, the cares of royalty, and their
application to business, had already impaired their constitution,
and brought on the infirmities of a premature old age.\textsuperscript{107}

\pagenote[106]{Solus omnium post conditum Romanum Imperium, qui
extanto fastigio sponte ad privatæ vitæ statum civilitatemque
remearet, Eutrop. ix. 28.}

\pagenote[107]{The particulars of the journey and illness are
taken from Laclantius, c. 17, who may sometimes be admitted as an
evidence of public facts, though very seldom of private
anecdotes.}

Notwithstanding the severity of a very cold and rainy winter,
Diocletian left Italy soon after the ceremony of his triumph, and
began his progress towards the East round the circuit of the
Illyrian provinces. From the inclemency of the weather, and the
fatigue of the journey, he soon contracted a slow illness; and
though he made easy marches, and was generally carried in a close
litter, his disorder, before he arrived at Nicomedia, about the
end of the summer, was become very serious and alarming. During
the whole winter he was confined to his palace: his danger
inspired a general and unaffected concern; but the people could
only judge of the various alterations of his health, from the joy
or consternation which they discovered in the countenances and
behavior of his attendants. The rumor of his death was for some
time universally believed, and it was supposed to be concealed
with a view to prevent the troubles that might have happened
during the absence of the Cæsar Galerius. At length, however, on
the first of March, Diocletian once more appeared in public, but
so pale and emaciated, that he could scarcely have been
recognized by those to whom his person was the most familiar. It
was time to put an end to the painful struggle, which he had
sustained during more than a year, between the care of his health
and that of his dignity. The former required indulgence and
relaxation, the latter compelled him to direct, from the bed of
sickness, the administration of a great empire. He resolved to
pass the remainder of his days in honorable repose, to place his
glory beyond the reach of fortune, and to relinquish the theatre
of the world to his younger and more active associates.\textsuperscript{108}

\pagenote[108]{Aurelius Victor ascribes the abdication, which had
been so variously accounted for, to two causes: 1st, Diocletian’s
contempt of ambition; and 2dly, His apprehension of impending
troubles. One of the panegyrists (vi. 9) mentions the age and
infirmities of Diocletian as a very natural reason for his
retirement. * Note: Constantine (Orat. ad Sanct. c. 401) more
than insinuated that derangement of mind, connected with the
conflagration of the palace at Nicomedia by lightning, was the
cause of his abdication. But Heinichen. in a very sensible note
on this passage in Eusebius, while he admits that his long
illness might produce a temporary depression of spirits,
triumphantly appeals to the philosophical conduct of Diocletian
in his retreat, and the influence which he still retained on
public affairs.—M.}

The ceremony of his abdication was performed in a spacious plain,
about three miles from Nicomedia. The emperor ascended a lofty
throne, and in a speech, full of reason and dignity, declared his
intention, both to the people and to the soldiers who were
assembled on this extraordinary occasion. As soon as he had
divested himself of his purple, he withdrew from the gazing
multitude; and traversing the city in a covered chariot,
proceeded, without delay, to the favorite retirement which he had
chosen in his native country of Dalmatia. On the same day, which
was the first of May,\textsuperscript{109} Maximian, as it had been previously
concerted, made his resignation of the Imperial dignity at Milan.

Even in the splendor of the Roman triumph, Diocletian had
meditated his design of abdicating the government. As he wished
to secure the obedience of Maximian, he exacted from him either a
general assurance that he would submit his actions to the
authority of his benefactor, or a particular promise that he
would descend from the throne, whenever he should receive the
advice and the example. This engagement, though it was confirmed
by the solemnity of an oath before the altar of the Capitoline
Jupiter,\textsuperscript{110} would have proved a feeble restraint on the fierce
temper of Maximian, whose passion was the love of power, and who
neither desired present tranquility nor future reputation. But he
yielded, however reluctantly, to the ascendant which his wiser
colleague had acquired over him, and retired, immediately after
his abdication, to a villa in Lucania, where it was almost
impossible that such an impatient spirit could find any lasting
tranquility.

\pagenote[109]{The difficulties as well as mistakes attending the
dates both of the year and of the day of Diocletian’s abdication
are perfectly cleared up by Tillemont, Hist. des Empereurs, tom.
iv. p 525, note 19, and by Pagi ad annum.}

\pagenote[110]{See Panegyr. Veter. vi. 9. The oration was
pronounced after Maximian had resumed the purple.}

Diocletian, who, from a servile origin, had raised himself to the
throne, passed the nine last years of his life in a private
condition. Reason had dictated, and content seems to have
accompanied, his retreat, in which he enjoyed, for a long time,
the respect of those princes to whom he had resigned the
possession of the world.\textsuperscript{111} It is seldom that minds long
exercised in business have formed any habits of conversing with
themselves, and in the loss of power they principally regret the
want of occupation. The amusements of letters and of devotion,
which afford so many resources in solitude, were incapable of
fixing the attention of Diocletian; but he had preserved, or at
least he soon recovered, a taste for the most innocent as well as
natural pleasures, and his leisure hours were sufficiently
employed in building, planting, and gardening. His answer to
Maximian is deservedly celebrated. He was solicited by that
restless old man to reassume the reins of government, and the
Imperial purple. He rejected the temptation with a smile of pity,
calmly observing, that if he could show Maximian the cabbages
which he had planted with his own hands at Salona, he should no
longer be urged to relinquish the enjoyment of happiness for the
pursuit of power.\textsuperscript{112} In his conversations with his friends, he
frequently acknowledged, that of all arts, the most difficult was
the art of reigning; and he expressed himself on that favorite
topic with a degree of warmth which could be the result only of
experience. “How often,” was he accustomed to say, “is it the
interest of four or five ministers to combine together to deceive
their sovereign! Secluded from mankind by his exalted dignity,
the truth is concealed from his knowledge; he can see only with
their eyes, he hears nothing but their misrepresentations. He
confers the most important offices upon vice and weakness, and
disgraces the most virtuous and deserving among his subjects. By
such infamous arts,” added Diocletian, “the best and wisest
princes are sold to the venal corruption of their courtiers.”\textsuperscript{113}
A just estimate of greatness, and the assurance of immortal fame,
improve our relish for the pleasures of retirement; but the Roman
emperor had filled too important a character in the world, to
enjoy without alloy the comforts and security of a private
condition. It was impossible that he could remain ignorant of the
troubles which afflicted the empire after his abdication. It was
impossible that he could be indifferent to their consequences.
Fear, sorrow, and discontent, sometimes pursued him into the
solitude of Salona. His tenderness, or at least his pride, was
deeply wounded by the misfortunes of his wife and daughter; and
the last moments of Diocletian were imbittered by some affronts,
which Licinius and Constantine might have spared the father of so
many emperors, and the first author of their own fortune. A
report, though of a very doubtful nature, has reached our times,
that he prudently withdrew himself from their power by a
voluntary death.\textsuperscript{114}

\pagenote[111]{Eumenius pays him a very fine compliment: “At enim
divinum illum virum, qui primus imperium et participavit et
posuit, consilii et fact isui non poenitet; nec amisisse se putat
quod sponte transcripsit. Felix beatusque vere quem vestra,
tantorum principum, colunt privatum.” Panegyr. Vet. vii. 15.}

\pagenote[112]{We are obliged to the younger Victor for this
celebrated item. Eutropius mentions the thing in a more general
manner.}

\pagenote[113]{Hist. August. p. 223, 224. Vopiscus had learned
this conversation from his father.}

\pagenote[114]{The younger Victor slightly mentions the report.
But as Diocletian had disobliged a powerful and successful party,
his memory has been loaded with every crime and misfortune. It
has been affirmed that he died raving mad, that he was condemned
as a criminal by the Roman senate, \&c.}

Before we dismiss the consideration of the life and character of
Diocletian, we may, for a moment, direct our view to the place of
his retirement. Salona, a principal city of his native province
of Dalmatia, was near two hundred Roman miles (according to the
measurement of the public highways) from Aquileia and the
confines of Italy, and about two hundred and seventy from
Sirmium, the usual residence of the emperors whenever they
visited the Illyrian frontier.\textsuperscript{115} A miserable village still
preserves the name of Salona; but so late as the sixteenth
century, the remains of a theatre, and a confused prospect of
broken arches and marble columns, continued to attest its ancient
splendor.\textsuperscript{116} About six or seven miles from the city Diocletian
constructed a magnificent palace, and we may infer, from the
greatness of the work, how long he had meditated his design of
abdicating the empire. The choice of a spot which united all that
could contribute either to health or to luxury did not require
the partiality of a native. “The soil was dry and fertile, the
air is pure and wholesome, and, though extremely hot during the
summer months, this country seldom feels those sultry and noxious
winds to which the coasts of Istria and some parts of Italy are
exposed. The views from the palace are no less beautiful than the
soil and climate were inviting. Towards the west lies the fertile
shore that stretches along the Adriatic, in which a number of
small islands are scattered in such a manner as to give this part
of the sea the appearance of a great lake. On the north side lies
the bay, which led to the ancient city of Salona; and the country
beyond it, appearing in sight, forms a proper contrast to that
more extensive prospect of water, which the Adriatic presents
both to the south and to the east. Towards the north, the view is
terminated by high and irregular mountains, situated at a proper
distance, and in many places covered with villages, woods, and
vineyards.”\textsuperscript{117}

\pagenote[115]{See the Itiner. p. 269, 272, edit. Wessel.}

\pagenote[116]{The Abate Fortis, in his Viaggio in Dalmazia, p.
43, (printed at Venice in the year 1774, in two small volumes in
quarto,) quotes a Ms account of the antiquities of Salona,
composed by Giambattista Giustiniani about the middle of the
xvith century.}

\pagenote[117]{Adam’s Antiquities of Diocletian’s Palace at
Spalatro, p. 6. We may add a circumstance or two from the Abate
Fortis: the little stream of the Hyader, mentioned by Lucan,
produces most exquisite trout, which a sagacious writer, perhaps
a monk, supposes to have been one of the principal reasons that
determined Diocletian in the choice of his retirement. Fortis, p.
45. The same author (p. 38) observes, that a taste for
agriculture is reviving at Spalatro; and that an experimental
farm has lately been established near the city, by a society of
gentlemen.}

Though Constantine, from a very obvious prejudice, affects to
mention the palace of Diocletian with contempt,\textsuperscript{118} yet one of
their successors, who could only see it in a neglected and
mutilated state, celebrates its magnificence in terms of the
highest admiration.\textsuperscript{119} It covered an extent of ground consisting
of between nine and ten English acres. The form was quadrangular,
flanked with sixteen towers. Two of the sides were near six
hundred, and the other two near seven hundred feet in length. The
whole was constructed of a beautiful freestone, extracted from
the neighboring quarries of Trau, or Tragutium, and very little
inferior to marble itself. Four streets, intersecting each other
at right angles, divided the several parts of this great edifice,
and the approach to the principal apartment was from a very
stately entrance, which is still denominated the Golden Gate. The
approach was terminated by a \textit{peristylium} of granite columns, on
one side of which we discover the square temple of Æsculapius, on
the other the octagon temple of Jupiter. The latter of those
deities Diocletian revered as the patron of his fortunes, the
former as the protector of his health. By comparing the present
remains with the precepts of Vitruvius, the several parts of the
building, the baths, bedchamber, the \textit{atrium}, the \textit{basilica},
and the Cyzicene, Corinthian, and Egyptian halls have been
described with some degree of precision, or at least of
probability. Their forms were various, their proportions just;
but they all were attended with two imperfections, very repugnant
to our modern notions of taste and conveniency. These stately
rooms had neither windows nor chimneys. They were lighted from
the top, (for the building seems to have consisted of no more
than one story,) and they received their heat by the help of
pipes that were conveyed along the walls. The range of principal
apartments was protected towards the south-west by a portico five
hundred and seventeen feet long, which must have formed a very
noble and delightful walk, when the beauties of painting and
sculpture were added to those of the prospect.

\pagenote[118]{Constantin. Orat. ad Coetum Sanct. c. 25. In this
sermon, the emperor, or the bishop who composed it for him,
affects to relate the miserable end of all the persecutors of the
church.}

\pagenote[119]{Constantin. Porphyr. de Statu Imper. p. 86.}

Had this magnificent edifice remained in a solitary country, it
would have been exposed to the ravages of time; but it might,
perhaps, have escaped the rapacious industry of man. The village
of Aspalathus,\textsuperscript{120} and, long afterwards, the provincial town of
Spalatro, have grown out of its ruins. The Golden Gate now opens
into the market-place. St. John the Baptist has usurped the
honors of Æsculapius; and the temple of Jupiter, under the
protection of the Virgin, is converted into the cathedral church.

For this account of Diocletian’s palace we are principally
indebted to an ingenious artist of our own time and country, whom
a very liberal curiosity carried into the heart of Dalmatia.\textsuperscript{121}
But there is room to suspect that the elegance of his designs and
engraving has somewhat flattered the objects which it was their
purpose to represent. We are informed by a more recent and very
judicious traveller, that the awful ruins of Spalatro are not
less expressive of the decline of the art than of the greatness
of the Roman empire in the time of Diocletian.\textsuperscript{122} If such was
indeed the state of architecture, we must naturally believe that
painting and sculpture had experienced a still more sensible
decay. The practice of architecture is directed by a few general
and even mechanical rules. But sculpture, and, above all,
painting, propose to themselves the imitation not only of the
forms of nature, but of the characters and passions of the human
soul. In those sublime arts the dexterity of the hand is of
little avail, unless it is animated by fancy, and guided by the
most correct taste and observation.

\pagenote[120]{D’Anville, Geographie Ancienne, tom. i. p. 162.}

\pagenote[121]{Messieurs Adam and Clerisseau, attended by two
draughtsmen visited Spalatro in the month of July, 1757. The
magnificent work which their journey produced was published in
London seven years afterwards.}

\pagenote[122]{I shall quote the words of the Abate Fortis.
“E’bastevolmente agli amatori dell’ Architettura, e dell’
Antichita, l’opera del Signor Adams, che a donato molto a que’
superbi vestigi coll’abituale eleganza del suo toccalapis e del
bulino. In generale la rozzezza del scalpello, e’l cattivo gusto
del secolo vi gareggiano colla magnificenz del fabricato.” See
Viaggio in Dalmazia, p. 40.}

It is almost unnecessary to remark, that the civil distractions
of the empire, the license of the soldiers, the inroads of the
barbarians, and the progress of despotism, had proved very
unfavorable to genius, and even to learning. The succession of
Illyrian princes restored the empire without restoring the
sciences. Their military education was not calculated to inspire
them with the love of letters; and even the mind of Diocletian,
however active and capacious in business, was totally uninformed
by study or speculation. The professions of law and physic are of
such common use and certain profit that they will always secure a
sufficient number of practitioners endowed with a reasonable
degree of abilities and knowledge; but it does not appear that
the students in those two faculties appeal to any celebrated
masters who have flourished within that period. The voice of
poetry was silent. History was reduced to dry and confused
abridgments, alike destitute of amusement and instruction. A
languid and affected eloquence was still retained in the pay and
service of the emperors, who encouraged not any arts except those
which contributed to the gratification of their pride, or the
defence of their power.\textsuperscript{123}

\pagenote[123]{The orator Eumenius was secretary to the emperors
Maximian and Constantius, and Professor of Rhetoric in the
college of Autun. His salary was six hundred thousand sesterces,
which, according to the lowest computation of that age, must have
exceeded three thousand pounds a year. He generously requested
the permission of employing it in rebuilding the college. See his
Oration De Restaurandis Scholis; which, though not exempt from
vanity, may atone for his panegyrics.}

The declining age of learning and of mankind is marked, however,
by the rise and rapid progress of the new Platonists. The school
of Alexandria silenced those of Athens; and the ancient sects
enrolled themselves under the banners of the more fashionable
teachers, who recommended their system by the novelty of their
method, and the austerity of their manners. Several of these
masters, Ammonius, Plotinus, Amelius, and Porphyry,\textsuperscript{124} were men
of profound thought and intense application; but by mistaking the
true object of philosophy, their labors contributed much less to
improve than to corrupt the human understanding. The knowledge
that is suited to our situation and powers, the whole compass of
moral, natural, and mathematical science, was neglected by the
new Platonists; whilst they exhausted their strength in the
verbal disputes of metaphysics, attempted to explore the secrets
of the invisible world, and studied to reconcile Aristotle with
Plato, on subjects of which both these philosophers were as
ignorant as the rest of mankind. Consuming their reason in these
deep but unsubstantial meditations, their minds were exposed to
illusions of fancy. They flattered themselves that they possessed
the secret of disengaging the soul from its corporal prison;
claimed a familiar intercourse with demons and spirits; and, by a
very singular revolution, converted the study of philosophy into
that of magic. The ancient sages had derided the popular
superstition; after disguising its extravagance by the thin
pretence of allegory, the disciples of Plotinus and Porphyry
became its most zealous defenders. As they agreed with the
Christians in a few mysterious points of faith, they attacked the
remainder of their theological system with all the fury of civil
war. The new Platonists would scarcely deserve a place in the
history of science, but in that of the church the mention of them
will very frequently occur.

\pagenote[124]{Porphyry died about the time of Diocletian’s
abdication. The life of his master Plotinus, which he composed,
will give us the most complete idea of the genius of the sect,
and the manners of its professors. This very curious piece is
inserted in Fabricius Bibliotheca Græca tom. iv. p. 88—148.}

