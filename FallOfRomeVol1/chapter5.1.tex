\chapter{Sale Of The Empire To Didius Julianus.}
\section{Part \thesection.}

Public Sale Of The Empire To Didius Julianus By The Prætorian
Guards — Clodius Albinus In Britain, Pescennius Niger In Syria, And
Septimius Severus In Pannonia, Declare Against The Murderers Of
Pertinax — Civil Wars And Victory Of Severus Over His Three
Rivals — Relaxation Of Discipline — New Maxims Of Government.
\vspace{\onelineskip}

The power of the sword is more sensibly felt in an extensive
monarchy, than in a small community. It has been calculated by
the ablest politicians, that no state, without being soon
exhausted, can maintain above the hundredth part of its members
in arms and idleness. But although this relative proportion may
be uniform, the influence of the army over the rest of the
society will vary according to the degree of its positive
strength. The advantages of military science and discipline
cannot be exerted, unless a proper number of soldiers are united
into one body, and actuated by one soul. With a handful of men,
such a union would be ineffectual; with an unwieldy host, it
would be impracticable; and the powers of the machine would be
alike destroyed by the extreme minuteness or the excessive weight
of its springs. To illustrate this observation, we need only
reflect, that there is no superiority of natural strength,
artificial weapons, or acquired skill, which could enable one man
to keep in constant subjection one hundred of his
fellow-creatures: the tyrant of a single town, or a small
district, would soon discover that a hundred armed followers were
a weak defence against ten thousand peasants or citizens; but a
hundred thousand well-disciplined soldiers will command, with
despotic sway, ten millions of subjects; and a body of ten or
fifteen thousand guards will strike terror into the most numerous
populace that ever crowded the streets of an immense capital.

The Prætorian bands, whose licentious fury was the first symptom
and cause of the decline of the Roman empire, scarcely amounted
to the last-mentioned number.\footnotemark[1] They derived their institution
from Augustus. That crafty tyrant, sensible that laws might
color, but that arms alone could maintain, his usurped dominion,
had gradually formed this powerful body of guards, in constant
readiness to protect his person, to awe the senate, and either to
prevent or to crush the first motions of rebellion. He
distinguished these favored troops by a double pay and superior
privileges; but, as their formidable aspect would at once have
alarmed and irritated the Roman people, three cohorts only were
stationed in the capital, whilst the remainder was dispersed in
the adjacent towns of Italy.\footnotemark[2] But after fifty years of peace and
servitude, Tiberius ventured on a decisive measure, which forever
rivetted the fetters of his country. Under the fair pretences of
relieving Italy from the heavy burden of military quarters, and
of introducing a stricter discipline among the guards, he
assembled them at Rome, in a permanent camp,\footnotemark[3] which was
fortified with skilful care,\footnotemark[4] and placed on a commanding
situation.\footnotemark[5]

\footnotetext[1]{They were originally nine or ten thousand men, (for
Tacitus and son are not agreed upon the subject,) divided into as
many cohorts. Vitellius increased them to sixteen thousand, and
as far as we can learn from inscriptions, they never afterwards
sunk much below that number. See Lipsius de magnitudine Romana,
i. 4.}

\footnotetext[2]{Sueton. in August. c. 49.}

\footnotetext[3]{Tacit. Annal. iv. 2. Sueton. in Tiber. c. 37. Dion
Cassius, l. lvii. p. 867.}

\footnotetext[4]{In the civil war between Vitellius and Vespasian,
the Prætorian camp was attacked and defended with all the
machines used in the siege of the best fortified cities. Tacit.
Hist. iii. 84.}

\footnotetext[5]{Close to the walls of the city, on the broad summit
of the Quirinal and Viminal hills. See Nardini Roma Antica, p.
174. Donatus de Roma Antiqua, p. 46. * Note: Not on both these
hills: neither Donatus nor Nardini justify this position.
(Whitaker’s Review. p. 13.) At the northern extremity of this
hill (the Viminal) are some considerable remains of a walled
enclosure which bears all the appearance of a Roman camp, and
therefore is generally thought to correspond with the Castra
Prætoria. Cramer’s Italy 390.—M.}

Such formidable servants are always necessary, but often fatal to
the throne of despotism. By thus introducing the Prætorian guards
as it were into the palace and the senate, the emperors taught
them to perceive their own strength, and the weakness of the
civil government; to view the vices of their masters with
familiar contempt, and to lay aside that reverential awe, which
distance only, and mystery, can preserve towards an imaginary
power. In the luxurious idleness of an opulent city, their pride
was nourished by the sense of their irresistible weight; nor was
it possible to conceal from them, that the person of the
sovereign, the authority of the senate, the public treasure, and
the seat of empire, were all in their hands. To divert the
Prætorian bands from these dangerous reflections, the firmest and
best established princes were obliged to mix blandishments with
commands, rewards with punishments, to flatter their pride,
indulge their pleasures, connive at their irregularities, and to
purchase their precarious faith by a liberal donative; which,
since the elevation of Claudius, was enacted as a legal claim, on
the accession of every new emperor.\footnotemark[6]

\footnotetext[6]{Claudius, raised by the soldiers to the empire, was
the first who gave a donative. He gave quina dena, 120l. (Sueton.
in Claud. c. 10: ) when Marcus, with his colleague Lucius Versus,
took quiet possession of the throne, he gave vicena, 160l. to
each of the guards. Hist. August. p. 25, (Dion, l. lxxiii. p.
1231.) We may form some idea of the amount of these sums, by
Hadrian’s complaint that the promotion of a Cæsar had cost him
ter millies, two millions and a half sterling.}

The advocate of the guards endeavored to justify by arguments the
power which they asserted by arms; and to maintain that,
according to the purest principles of the constitution, \textit{their}
consent was essentially necessary in the appointment of an
emperor. The election of consuls, of generals, and of
magistrates, however it had been recently usurped by the senate,
was the ancient and undoubted right of the Roman people.\footnotemark[7] But
where was the Roman people to be found? Not surely amongst the
mixed multitude of slaves and strangers that filled the streets
of Rome; a servile populace, as devoid of spirit as destitute of
property. The defenders of the state, selected from the flower of
the Italian youth,\footnotemark[8] and trained in the exercise of arms and
virtue, were the genuine representatives of the people, and the
best entitled to elect the military chief of the republic. These
assertions, however defective in reason, became unanswerable when
the fierce Prætorians increased their weight, by throwing, like
the barbarian conqueror of Rome, their swords into the scale.\footnotemark[9]

\footnotetext[7]{Cicero de Legibus, iii. 3. The first book of Livy,
and the second of Dionysius of Halicarnassus, show the authority
of the people, even in the election of the kings.}

\footnotetext[8]{They were originally recruited in Latium, Etruria,
and the old colonies, (Tacit. Annal. iv. 5.) The emperor Otho
compliments their vanity with the flattering titles of Italiæ,
Alumni, Romana were juventus. Tacit. Hist. i. 84.}

\footnotetext[9]{In the siege of Rome by the Gauls. See Livy, v. 48.
Plutarch. in Camill. p. 143.}

The Prætorians had violated the sanctity of the throne by the
atrocious murder of Pertinax; they dishonored the majesty of it
by their subsequent conduct. The camp was without a leader, for
even the præfect Lætus, who had excited the tempest, prudently
declined the public indignation. Amidst the wild disorder,
Sulpicianus, the emperor’s father-in-law, and governor of the
city, who had been sent to the camp on the first alarm of mutiny,
was endeavoring to calm the fury of the multitude, when he was
silenced by the clamorous return of the murderers, bearing on a
lance the head of Pertinax. Though history has accustomed us to
observe every principle and every passion yielding to the
imperious dictates of ambition, it is scarcely credible that, in
these moments of horror, Sulpicianus should have aspired to
ascend a throne polluted with the recent blood of so near a
relation and so excellent a prince. He had already begun to use
the only effectual argument, and to treat for the Imperial
dignity; but the more prudent of the Prætorians, apprehensive
that, in this private contract, they should not obtain a just
price for so valuable a commodity, ran out upon the ramparts;
and, with a loud voice, proclaimed that the Roman world was to be
disposed of to the best bidder by public auction.\footnotemark[10]

\footnotetext[10]{Dion, L. lxxiii. p. 1234. Herodian, l. ii. p. 63.
Hist. August p. 60. Though the three historians agree that it was
in fact an auction, Herodian alone affirms that it was proclaimed
as such by the soldiers.}

This infamous offer, the most insolent excess of military
license, diffused a universal grief, shame, and indignation
throughout the city. It reached at length the ears of Didius
Julianus, a wealthy senator, who, regardless of the public
calamities, was indulging himself in the luxury of the table.\footnotemark[11]
His wife and his daughter, his freedmen and his parasites, easily
convinced him that he deserved the throne, and earnestly conjured
him to embrace so fortunate an opportunity. The vain old man
hastened to the Prætorian camp, where Sulpicianus was still in
treaty with the guards, and began to bid against him from the
foot of the rampart. The unworthy negotiation was transacted by
faithful emissaries, who passed alternately from one candidate to
the other, and acquainted each of them with the offers of his
rival. Sulpicianus had already promised a donative of five
thousand drachms (above one hundred and sixty pounds) to each
soldier; when Julian, eager for the prize, rose at once to the
sum of six thousand two hundred and fifty drachms, or upwards of
two hundred pounds sterling. The gates of the camp were instantly
thrown open to the purchaser; he was declared emperor, and
received an oath of allegiance from the soldiers, who retained
humanity enough to stipulate that he should pardon and forget the
competition of Sulpicianus.\footnotemark[111]

\footnotetext[11]{Spartianus softens the most odious parts of the
character and elevation of Julian.}

\footnotetext[111]{One of the principal causes of the preference of
Julianus by the soldiers, was the dexterty dexterity with which
he reminded them that Sulpicianus would not fail to revenge on
them the death of his son-in-law. (See Dion, p. 1234, 1234. c.
11. Herod. ii. 6.)—W.}

It was now incumbent on the Prætorians to fulfil the conditions
of the sale. They placed their new sovereign, whom they served
and despised, in the centre of their ranks, surrounded him on
every side with their shields, and conducted him in close order
of battle through the deserted streets of the city. The senate
was commanded to assemble; and those who had been the
distinguished friends of Pertinax, or the personal enemies of
Julian, found it necessary to affect a more than common share of
satisfaction at this happy revolution.\footnotemark[12] After Julian had filled
the senate house with armed soldiers, he expatiated on the
freedom of his election, his own eminent virtues, and his full
assurance of the affections of the senate. The obsequious
assembly congratulated their own and the public felicity; engaged
their allegiance, and conferred on him all the several branches
of the Imperial power.\footnotemark[13] From the senate Julian was conducted,
by the same military procession, to take possession of the
palace. The first objects that struck his eyes, were the
abandoned trunk of Pertinax, and the frugal entertainment
prepared for his supper. The one he viewed with indifference, the
other with contempt. A magnificent feast was prepared by his
order, and he amused himself, till a very late hour, with dice,
and the performances of Pylades, a celebrated dancer. Yet it was
observed, that after the crowd of flatterers dispersed, and left
him to darkness, solitude, and terrible reflection, he passed a
sleepless night; revolving most probably in his mind his own rash
folly, the fate of his virtuous predecessor, and the doubtful and
dangerous tenure of an empire which had not been acquired by
merit, but purchased by money.\footnotemark[14]

\footnotetext[12]{Dion Cassius, at that time prætor, had been a
personal enemy to Julian, i. lxxiii. p. 1235.}

\footnotetext[13]{Hist. August. p. 61. We learn from thence one
curious circumstance, that the new emperor, whatever had been his
birth, was immediately aggregated to the number of patrician
families. Note: A new fragment of Dion shows some shrewdness in
the character of Julian. When the senate voted him a golden
statue, he preferred one of brass, as more lasting. He “had
always observed,” he said, “that the statues of former emperors
were soon destroyed. Those of brass alone remained.” The
indignant historian adds that he was wrong. The virtue of
sovereigns alone preserves their images: the brazen statue of
Julian was broken to pieces at his death. Mai. Fragm. Vatican. p.
226.—M.}

\footnotetext[14]{Dion, l. lxxiii. p. 1235. Hist. August. p. 61. I
have endeavored to blend into one consistent story the seeming
contradictions of the two writers. * Note: The contradiction as
M. Guizot observed, is irreconcilable. He quotes both passages:
in one Julianus is represented as a miser, in the other as a
voluptuary. In the one he refuses to eat till the body of
Pertinax has been buried; in the other he gluts himself with
every luxury almost in the sight of his headless remains.—M.}

He had reason to tremble. On the throne of the world he found
himself without a friend, and even without an adherent. The
guards themselves were ashamed of the prince whom their avarice
had persuaded them to accept; nor was there a citizen who did not
consider his elevation with horror, as the last insult on the
Roman name. The nobility, whose conspicuous station, and ample
possessions, exacted the strictest caution, dissembled their
sentiments, and met the affected civility of the emperor with
smiles of complacency and professions of duty. But the people,
secure in their numbers and obscurity, gave a free vent to their
passions. The streets and public places of Rome resounded with
clamors and imprecations. The enraged multitude affronted the
person of Julian, rejected his liberality, and, conscious of the
impotence of their own resentment, they called aloud on the
legions of the frontiers to assert the violated majesty of the
Roman empire. The public discontent was soon diffused from the
centre to the frontiers of the empire. The armies of Britain, of
Syria, and of Illyricum, lamented the death of Pertinax, in whose
company, or under whose command, they had so often fought and
conquered. They received with surprise, with indignation, and
perhaps with envy, the extraordinary intelligence, that the
Prætorians had disposed of the empire by public auction; and they
sternly refused to ratify the ignominious bargain. Their
immediate and unanimous revolt was fatal to Julian, but it was
fatal at the same time to the public peace, as the generals of
the respective armies, Clodius Albinus, Pescennius Niger, and
Septimius Severus, were still more anxious to succeed than to
revenge the murdered Pertinax. Their forces were exactly
balanced. Each of them was at the head of three legions,\footnotemark[15] with
a numerous train of auxiliaries; and however different in their
characters, they were all soldiers of experience and capacity.

\footnotetext[15]{Dion, l. lxxiii. p. 1235.}

Clodius Albinus, governor of Britain, surpassed both his
competitors in the nobility of his extraction, which he derived
from some of the most illustrious names of the old republic.\footnotemark[16]
But the branch from which he claimed his descent was sunk into
mean circumstances, and transplanted into a remote province. It
is difficult to form a just idea of his true character. Under the
philosophic cloak of austerity, he stands accused of concealing
most of the vices which degrade human nature.\footnotemark[17] But his accusers
are those venal writers who adored the fortune of Severus, and
trampled on the ashes of an unsuccessful rival. Virtue, or the
appearances of virtue, recommended Albinus to the confidence and
good opinion of Marcus; and his preserving with the son the same
interest which he had acquired with the father, is a proof at
least that he was possessed of a very flexible disposition. The
favor of a tyrant does not always suppose a want of merit in the
object of it; he may, without intending it, reward a man of worth
and ability, or he may find such a man useful to his own service.
It does not appear that Albinus served the son of Marcus, either
as the minister of his cruelties, or even as the associate of his
pleasures. He was employed in a distant honorable command, when
he received a confidential letter from the emperor, acquainting
him of the treasonable designs of some discontented generals, and
authorizing him to declare himself the guardian and successor of
the throne, by assuming the title and ensigns of Cæsar.\footnotemark[18] The
governor of Britain wisely declined the dangerous honor, which
would have marked him for the jealousy, or involved him in the
approaching ruin, of Commodus. He courted power by nobler, or, at
least, by more specious arts. On a premature report of the death
of the emperor, he assembled his troops; and, in an eloquent
discourse, deplored the inevitable mischiefs of despotism,
described the happiness and glory which their ancestors had
enjoyed under the consular government, and declared his firm
resolution to reinstate the senate and people in their legal
authority. This popular harangue was answered by the loud
acclamations of the British legions, and received at Rome with a
secret murmur of applause. Safe in the possession of his little
world, and in the command of an army less distinguished indeed
for discipline than for numbers and valor,\footnotemark[19] Albinus braved the
menaces of Commodus, maintained towards Pertinax a stately
ambiguous reserve, and instantly declared against the usurpation
of Julian. The convulsions of the capital added new weight to his
sentiments, or rather to his professions of patriotism. A regard
to decency induced him to decline the lofty titles of Augustus
and Emperor; and he imitated perhaps the example of Galba, who,
on a similar occasion, had styled himself the Lieutenant of the
senate and people.\footnotemark[20]

\footnotetext[16]{The Posthumian and the Ce’onian; the former of whom
was raised to the consulship in the fifth year after its
institution.}

\footnotetext[17]{Spartianus, in his undigested collections, mixes up
all the virtues and all the vices that enter into the human
composition, and bestows them on the same object. Such, indeed
are many of the characters in the Augustan History.}

\footnotetext[18]{Hist. August. p. 80, 84.}

\footnotetext[19]{Pertinax, who governed Britain a few years before,
had been left for dead, in a mutiny of the soldiers. Hist.
August. p 54. Yet they loved and regretted him; admirantibus eam
virtutem cui irascebantur.}

\footnotetext[20]{Sueton. in Galb. c. 10.}

Personal merit alone had raised Pescennius Niger, from an obscure
birth and station, to the government of Syria; a lucrative and
important command, which in times of civil confusion gave him a
near prospect of the throne. Yet his parts seem to have been
better suited to the second than to the first rank; he was an
unequal rival, though he might have approved himself an excellent
lieutenant, to Severus, who afterwards displayed the greatness of
his mind by adopting several useful institutions from a
vanquished enemy.\footnotemark[21] In his government Niger acquired the esteem
of the soldiers and the love of the provincials. His rigid
discipline fortified the valor and confirmed the obedience of the
former, whilst the voluptuous Syrians were less delighted with
the mild firmness of his administration, than with the affability
of his manners, and the apparent pleasure with which he attended
their frequent and pompous festivals.\footnotemark[22] As soon as the
intelligence of the atrocious murder of Pertinax had reached
Antioch, the wishes of Asia invited Niger to assume the Imperial
purple and revenge his death. The legions of the eastern frontier
embraced his cause; the opulent but unarmed provinces, from the
frontiers of Æthiopia\footnotemark[23] to the Hadriatic, cheerfully submitted
to his power; and the kings beyond the Tigris and the Euphrates
congratulated his election, and offered him their homage and
services. The mind of Niger was not capable of receiving this
sudden tide of fortune: he flattered himself that his accession
would be undisturbed by competition and unstained by civil blood;
and whilst he enjoyed the vain pomp of triumph, he neglected to
secure the means of victory. Instead of entering into an
effectual negotiation with the powerful armies of the West, whose
resolution might decide, or at least must balance, the mighty
contest; instead of advancing without delay towards Rome and
Italy, where his presence was impatiently expected,\footnotemark[24] Niger
trifled away in the luxury of Antioch those irretrievable moments
which were diligently improved by the decisive activity of
Severus.\footnotemark[25]

\footnotetext[21]{Hist. August. p. 76.}

\footnotetext[22]{Herod. l. ii. p. 68. The Chronicle of John Malala,
of Antioch, shows the zealous attachment of his countrymen to
these festivals, which at once gratified their superstition, and
their love of pleasure.}

\footnotetext[23]{A king of Thebes, in Egypt, is mentioned, in the
Augustan History, as an ally, and, indeed, as a personal friend
of Niger. If Spartianus is not, as I strongly suspect, mistaken,
he has brought to light a dynasty of tributary princes totally
unknown to history.}

\footnotetext[24]{Dion, l. lxxiii. p. 1238. Herod. l. ii. p. 67. A
verse in every one’s mouth at that time, seems to express the
general opinion of the three rivals; Optimus est \textit{Niger},
[\textit{Fuscus}, which preserves the quantity.—M.] bonus \textit{Afer},
pessimus \textit{Albus}. Hist. August. p. 75.}

\footnotetext[25]{Herodian, l. ii. p. 71.}

The country of Pannonia and Dalmatia, which occupied the space
between the Danube and the Hadriatic, was one of the last and
most difficult conquests of the Romans. In the defence of
national freedom, two hundred thousand of these barbarians had
once appeared in the field, alarmed the declining age of
Augustus, and exercised the vigilant prudence of Tiberius at the
head of the collected force of the empire.\footnotemark[26] The Pannonians
yielded at length to the arms and institutions of Rome. Their
recent subjection, however, the neighborhood, and even the
mixture, of the unconquered tribes, and perhaps the climate,
adapted, as it has been observed, to the production of great
bodies and slow minds,\footnotemark[27] all contributed to preserve some
remains of their original ferocity, and under the tame and
uniform countenance of Roman provincials, the hardy features of
the natives were still to be discerned. Their warlike youth
afforded an inexhaustible supply of recruits to the legions
stationed on the banks of the Danube, and which, from a perpetual
warfare against the Germans and Sarmazans, were deservedly
esteemed the best troops in the service.

\footnotetext[26]{See an account of that memorable war in Velleius
Paterculus, is 110, \&c., who served in the army of Tiberius.}

\footnotetext[27]{Such is the reflection of Herodian, l. ii. p. 74.
Will the modern Austrians allow the influence?}

The Pannonian army was at this time commanded by Septimius
Severus, a native of Africa, who, in the gradual ascent of
private honors, had concealed his daring ambition, which was
never diverted from its steady course by the allurements of
pleasure, the apprehension of danger, or the feelings of
humanity.\footnotemark[28] On the first news of the murder of Pertinax, he
assembled his troops, painted in the most lively colors the
crime, the insolence, and the weakness of the Prætorian guards,
and animated the legions to arms and to revenge. He concluded
(and the peroration was thought extremely eloquent) with
promising every soldier about four hundred pounds; an honorable
donative, double in value to the infamous bribe with which Julian
had purchased the empire.\footnotemark[29] The acclamations of the army
immediately saluted Severus with the names of Augustus, Pertinax,
and Emperor; and he thus attained the lofty station to which he
was invited, by conscious merit and a long train of dreams and
omens, the fruitful offsprings either of his superstition or
policy.\footnotemark[30]

\footnotetext[28]{In the letter to Albinus, already mentioned,
Commodus accuses Severus, as one of the ambitious generals who
censured his conduct, and wished to occupy his place. Hist.
August. p. 80.}

\footnotetext[29]{Pannonia was too poor to supply such a sum. It was
probably promised in the camp, and paid at Rome, after the
victory. In fixing the sum, I have adopted the conjecture of
Casaubon. See Hist. August. p. 66. Comment. p. 115.}

\footnotetext[30]{Herodian, l. ii. p. 78. Severus was declared
emperor on the banks of the Danube, either at Carnuntum,
according to Spartianus, (Hist. August. p. 65,) or else at
Sabaria, according to Victor. Mr. Hume, in supposing that the
birth and dignity of Severus were too much inferior to the
Imperial crown, and that he marched into Italy as general only,
has not considered this transaction with his usual accuracy,
(Essay on the original contract.) * Note: Carnuntum, opposite to
the mouth of the Morava: its position is doubtful, either
Petronel or Haimburg. A little intermediate village seems to
indicate by its name (Altenburg) the site of an old town.
D’Anville Geogr. Anc. Sabaria, now Sarvar.—G. Compare note
37.—M.}

The new candidate for empire saw and improved the peculiar
advantage of his situation. His province extended to the Julian
Alps, which gave an easy access into Italy; and he remembered the
saying of Augustus, that a Pannonian army might in ten days
appear in sight of Rome.\footnotemark[31] By a celerity proportioned to the
greatness of the occasion, he might reasonably hope to revenge
Pertinax, punish Julian, and receive the homage of the senate and
people, as their lawful emperor, before his competitors,
separated from Italy by an immense tract of sea and land, were
apprised of his success, or even of his election. During the
whole expedition, he scarcely allowed himself any moments for
sleep or food; marching on foot, and in complete armor, at the
head of his columns, he insinuated himself into the confidence
and affection of his troops, pressed their diligence, revived
their spirits, animated their hopes, and was well satisfied to
share the hardships of the meanest soldier, whilst he kept in
view the infinite superiority of his reward.

\footnotetext[31]{Velleius Paterculus, l. ii. c. 3. We must reckon
the march from the nearest verge of Pannonia, and extend the
sight of the city as far as two hundred miles.}

The wretched Julian had expected, and thought himself prepared,
to dispute the empire with the governor of Syria; but in the
invincible and rapid approach of the Pannonian legions, he saw
his inevitable ruin. The hasty arrival of every messenger
increased his just apprehensions. He was successively informed,
that Severus had passed the Alps; that the Italian cities,
unwilling or unable to oppose his progress, had received him with
the warmest professions of joy and duty; that the important place
of Ravenna had surrendered without resistance, and that the
Hadriatic fleet was in the hands of the conqueror. The enemy was
now within two hundred and fifty miles of Rome; and every moment
diminished the narrow span of life and empire allotted to Julian.

He attempted, however, to prevent, or at least to protract, his
ruin. He implored the venal faith of the Prætorians, filled the
city with unavailing preparations for war, drew lines round the
suburbs, and even strengthened the fortifications of the palace;
as if those last intrenchments could be defended, without hope of
relief, against a victorious invader. Fear and shame prevented
the guards from deserting his standard; but they trembled at the
name of the Pannonian legions, commanded by an experienced
general, and accustomed to vanquish the barbarians on the frozen
Danube.\footnotemark[32] They quitted, with a sigh, the pleasures of the baths
and theatres, to put on arms, whose use they had almost
forgotten, and beneath the weight of which they were oppressed.
The unpractised elephants, whose uncouth appearance, it was
hoped, would strike terror into the army of the north, threw
their unskilful riders; and the awkward evolutions of the
marines, drawn from the fleet of Misenum, were an object of
ridicule to the populace; whilst the senate enjoyed, with secret
pleasure, the distress and weakness of the usurper.\footnotemark[33]

\footnotetext[32]{This is not a puerile figure of rhetoric, but an
allusion to a real fact recorded by Dion, l. lxxi. p. 1181. It
probably happened more than once.}

\footnotetext[33]{Dion, l. lxxiii. p. 1233. Herodian, l. ii. p. 81.
There is no surer proof of the military skill of the Romans, than
their first surmounting the idle terror, and afterwards
disdaining the dangerous use, of elephants in war. Note: These
elephants were kept for processions, perhaps for the games. Se
Herod. in loc.—M.}

Every motion of Julian betrayed his trembling perplexity. He
insisted that Severus should be declared a public enemy by the
senate. He entreated that the Pannonian general might be
associated to the empire. He sent public ambassadors of consular
rank to negotiate with his rival; he despatched private assassins
to take away his life. He designed that the Vestal virgins, and
all the colleges of priests, in their sacerdotal habits, and
bearing before them the sacred pledges of the Roman religion,
should advance in solemn procession to meet the Pannonian
legions; and, at the same time, he vainly tried to interrogate,
or to appease, the fates, by magic ceremonies and unlawful
sacrifices.\footnotemark[34]

\footnotetext[34]{Hist. August. p. 62, 63. * Note: Quæ ad speculum
dicunt fieri in quo pueri præligatis oculis, incantate...,
respicere dicuntur. * * * Tuncque puer vidisse dicitur et
adventun Severi et Juliani decessionem. This seems to have been a
practice somewhat similar to that of which our recent Egyptian
travellers relate such extraordinary circumstances. See also
Apulius, Orat. de Magia.—M.}

