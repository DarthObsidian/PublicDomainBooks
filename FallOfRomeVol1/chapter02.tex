\chapter{The Internal Prosperity In The Age Of The Antonines.}
\section{Part \thesection.}

\textit{Of The Union And Internal Prosperity Of The Roman Empire, In The
Age Of The Antonines.}
\vspace{\onelineskip}

It is not alone by the rapidity, or extent of conquest, that we
should estimate the greatness of Rome. The sovereign of the
Russian deserts commands a larger portion of the globe. In the
seventh summer after his passage of the Hellespont, Alexander
erected the Macedonian trophies on the banks of the Hyphasis.\textsuperscript{1}
Within less than a century, the irresistible Zingis, and the
Mogul princes of his race, spread their cruel devastations and
transient empire from the Sea of China, to the confines of Egypt
and Germany.\textsuperscript{2} But the firm edifice of Roman power was raised and
preserved by the wisdom of ages. The obedient provinces of Trajan
and the Antonines were united by laws, and adorned by arts. They
might occasionally suffer from the partial abuse of delegated
authority; but the general principle of government was wise,
simple, and beneficent. They enjoyed the religion of their
ancestors, whilst in civil honors and advantages they were
exalted, by just degrees, to an equality with their conquerors.

\pagenote[1]{They were erected about the midway between Lahor and
Delhi. The conquests of Alexander in Hindostan were confined to
the Punjab, a country watered by the five great streams of the
Indus. * Note: The Hyphasis is one of the five rivers which join
the Indus or the Sind, after having traversed the province of the
Pendj-ab—a name which in Persian, signifies \textit{five rivers}. * * *
G. The five rivers were, 1. The Hydaspes, now the Chelum, Behni,
or Bedusta, (\textit{Sanscrit}, Vitashà, Arrow-swift.) 2. The Acesines,
the Chenab, (\textit{Sanscrit}, Chandrabhágâ, Moon-gift.) 3. Hydraotes,
the Ravey, or Iraoty, (\textit{Sanscrit}, Irâvatî.) 4. Hyphasis, the
Beyah, (\textit{Sanscrit}, Vepâsà, Fetterless.) 5. The Satadru,
(\textit{Sanscrit}, the Hundred Streamed,) the Sutledj, known first to
the Greeks in the time of Ptolemy. Rennel. Vincent, Commerce of
Anc. book 2. Lassen, Pentapotam. Ind. Wilson’s Sanscrit Dict.,
and the valuable memoir of Lieut. Burnes, Journal of London
Geogr. Society, vol. iii. p. 2, with the travels of that very
able writer. Compare Gibbon’s own note, c. lxv. note 25.—M
substit. for G.}

\pagenote[2]{See M. de Guignes, Histoire des Huns, l. xv. xvi.
and xvii.}

I. The policy of the emperors and the senate, as far as it
concerned religion, was happily seconded by the reflections of
the enlightened, and by the habits of the superstitious, part of
their subjects. The various modes of worship, which prevailed in
the Roman world, were all considered by the people, as equally
true; by the philosopher, as equally false; and by the
magistrate, as equally useful. And thus toleration produced not
only mutual indulgence, but even religious concord.

The superstition of the people was not imbittered by any mixture
of theological rancor; nor was it confined by the chains of any
speculative system. The devout polytheist, though fondly attached
to his national rites, admitted with implicit faith the different
religions of the earth.\textsuperscript{3} Fear, gratitude, and curiosity, a dream
or an omen, a singular disorder, or a distant journey,
perpetually disposed him to multiply the articles of his belief,
and to enlarge the list of his protectors. The thin texture of
the Pagan mythology was interwoven with various but not
discordant materials. As soon as it was allowed that sages and
heroes, who had lived or who had died for the benefit of their
country, were exalted to a state of power and immortality, it was
universally confessed, that they deserved, if not the adoration,
at least the reverence, of all mankind. The deities of a thousand
groves and a thousand streams possessed, in peace, their local
and respective influence; nor could the Romans who deprecated the
wrath of the Tiber, deride the Egyptian who presented his
offering to the beneficent genius of the Nile. The visible powers
of nature, the planets, and the elements were the same throughout
the universe. The invisible governors of the moral world were
inevitably cast in a similar mould of fiction and allegory. Every
virtue, and even vice, acquired its divine representative; every
art and profession its patron, whose attributes, in the most
distant ages and countries, were uniformly derived from the
character of their peculiar votaries. A republic of gods of such
opposite tempers and interests required, in every system, the
moderating hand of a supreme magistrate, who, by the progress of
knowledge and flattery, was gradually invested with the sublime
perfections of an Eternal Parent, and an Omnipotent Monarch.\textsuperscript{4}
Such was the mild spirit of antiquity, that the nations were less
attentive to the difference, than to the resemblance, of their
religious worship. The Greek, the Roman, and the Barbarian, as
they met before their respective altars, easily persuaded
themselves, that under various names, and with various
ceremonies, they adored the same deities.\textsuperscript{5} The elegant mythology
of Homer gave a beautiful, and almost a regular form, to the
polytheism of the ancient world.

\pagenote[3]{There is not any writer who describes in so lively a
manner as Herodotus the true genius of polytheism. The best
commentary may be found in Mr. Hume’s Natural History of
Religion; and the best contrast in Bossuet’s Universal History.
Some obscure traces of an intolerant spirit appear in the conduct
of the Egyptians, (see Juvenal, Sat. xv.;) and the Christians, as
well as Jews, who lived under the Roman empire, formed a very
important exception; so important indeed, that the discussion
will require a distinct chapter of this work. * Note: M.
Constant, in his very learned and eloquent work, “Sur la
Religion,” with the two additional volumes, “Du Polytheisme
Romain,” has considered the whole history of polytheism in a tone
of philosophy, which, without subscribing to all his opinions, we
may be permitted to admire. “The boasted tolerance of polytheism
did not rest upon the respect due from society to the freedom of
individual opinion. The polytheistic nations, tolerant as they
were towards each other, as separate states, were not the less
ignorant of the eternal principle, the only basis of enlightened
toleration, that every one has a right to worship God in the
manner which seems to him the best. Citizens, on the contrary,
were bound to conform to the religion of the state; they had not
the liberty to adopt a foreign religion, though that religion
might be legally recognized in their own city, for the strangers
who were its votaries.” —Sur la Religion, v. 184. Du. Polyth.
Rom. ii. 308. At this time, the growing religious indifference,
and the general administration of the empire by Romans, who,
being strangers, would do no more than protect, not enlist
themselves in the cause of the local superstitions, had
introduced great laxity. But intolerance was clearly the theory
both of the Greek and Roman law. The subject is more fully
considered in another place.—M.}

\pagenote[4]{The rights, powers, and pretensions of the sovereign
of Olympus are very clearly described in the xvth book of the
Iliad; in the Greek original, I mean; for Mr. Pope, without
perceiving it, has improved the theology of Homer. * Note: There
is a curious coincidence between Gibbon’s expressions and those
of the newly-recovered “De Republica” of Cicero, though the
argument is rather the converse, lib. i. c. 36. “Sive hæc ad
utilitatem vitæ constitute sint a principibus rerum publicarum,
ut rex putaretur unus esse in cœlo, qui nutu, ut ait Homerus,
totum Olympum converteret, idemque et rex et patos haberetur
omnium.”—M.}

\pagenote[5]{See, for instance, Cæsar de Bell. Gall. vi. 17.
Within a century or two, the Gauls themselves applied to their
gods the names of Mercury, Mars, Apollo, \&c.}

The philosophers of Greece deduced their morals from the nature
of man, rather than from that of God. They meditated, however, on
the Divine Nature, as a very curious and important speculation;
and in the profound inquiry, they displayed the strength and
weakness of the human understanding.\textsuperscript{6} Of the four most
celebrated schools, the Stoics and the Platonists endeavored to
reconcile the jaring interests of reason and piety. They have
left us the most sublime proofs of the existence and perfections
of the first cause; but, as it was impossible for them to
conceive the creation of matter, the workman in the Stoic
philosophy was not sufficiently distinguished from the work;
whilst, on the contrary, the spiritual God of Plato and his
disciples resembled an idea, rather than a substance. The
opinions of the Academics and Epicureans were of a less religious
cast; but whilst the modest science of the former induced them to
doubt, the positive ignorance of the latter urged them to deny,
the providence of a Supreme Ruler. The spirit of inquiry,
prompted by emulation, and supported by freedom, had divided the
public teachers of philosophy into a variety of contending sects;
but the ingenious youth, who, from every part, resorted to
Athens, and the other seats of learning in the Roman empire, were
alike instructed in every school to reject and to despise the
religion of the multitude. How, indeed, was it possible that a
philosopher should accept, as divine truths, the idle tales of
the poets, and the incoherent traditions of antiquity; or that he
should adore, as gods, those imperfect beings whom he must have
despised, as men? Against such unworthy adversaries, Cicero
condescended to employ the arms of reason and eloquence; but the
satire of Lucian was a much more adequate, as well as more
efficacious, weapon. We may be well assured, that a writer,
conversant with the world, would never have ventured to expose
the gods of his country to public ridicule, had they not already
been the objects of secret contempt among the polished and
enlightened orders of society.\textsuperscript{7}

\pagenote[6]{The admirable work of Cicero de Natura Deorum is the
best clew we have to guide us through the dark and profound
abyss. He represents with candor, and confutes with subtlety, the
opinions of the philosophers.}

\pagenote[7]{I do not pretend to assert, that, in this
irreligious age, the natural terrors of superstition, dreams,
omens, apparitions, \&c., had lost their efficacy.}

Notwithstanding the fashionable irreligion which prevailed in the
age of the Antonines, both the interest of the priests and the
credulity of the people were sufficiently respected. In their
writings and conversation, the philosophers of antiquity asserted
the independent dignity of reason; but they resigned their
actions to the commands of law and of custom. Viewing, with a
smile of pity and indulgence, the various errors of the vulgar,
they diligently practised the ceremonies of their fathers,
devoutly frequented the temples of the gods; and sometimes
condescending to act a part on the theatre of superstition, they
concealed the sentiments of an atheist under the sacerdotal
robes. Reasoners of such a temper were scarcely inclined to
wrangle about their respective modes of faith, or of worship. It
was indifferent to them what shape the folly of the multitude
might choose to assume; and they approached with the same inward
contempt, and the same external reverence, the altars of the
Libyan, the Olympian, or the Capitoline Jupiter.\textsuperscript{8}

\pagenote[8]{Socrates, Epicurus, Cicero, and Plutarch always
inculcated a decent reverence for the religion of their own
country, and of mankind. The devotion of Epicurus was assiduous
and exemplary. Diogen. Lært. x. 10.}

It is not easy to conceive from what motives a spirit of
persecution could introduce itself into the Roman councils. The
magistrates could not be actuated by a blind, though honest
bigotry, since the magistrates were themselves philosophers; and
the schools of Athens had given laws to the senate. They could
not be impelled by ambition or avarice, as the temporal and
ecclesiastical powers were united in the same hands. The pontiffs
were chosen among the most illustrious of the senators; and the
office of Supreme Pontiff was constantly exercised by the
emperors themselves. They knew and valued the advantages of
religion, as it is connected with civil government. They
encouraged the public festivals which humanize the manners of the
people. They managed the arts of divination as a convenient
instrument of policy; and they respected, as the firmest bond of
society, the useful persuasion, that, either in this or in a
future life, the crime of perjury is most assuredly punished by
the avenging gods.\textsuperscript{9} But whilst they acknowledged the general
advantages of religion, they were convinced that the various
modes of worship contributed alike to the same salutary purposes;
and that, in every country, the form of superstition, which had
received the sanction of time and experience, was the best
adapted to the climate, and to its inhabitants. Avarice and taste
very frequently despoiled the vanquished nations of the elegant
statues of their gods, and the rich ornaments of their temples;\textsuperscript{10}
but, in the exercise of the religion which they derived from
their ancestors, they uniformly experienced the indulgence, and
even protection, of the Roman conquerors. The province of Gaul
seems, and indeed only seems, an exception to this universal
toleration. Under the specious pretext of abolishing human
sacrifices, the emperors Tiberius and Claudius suppressed the
dangerous power of the Druids:\textsuperscript{11} but the priests themselves,
their gods and their altars, subsisted in peaceful obscurity till
the final destruction of Paganism.\textsuperscript{12}

\pagenote[9]{Polybius, l. vi. c. 53, 54. Juvenal, Sat. xiii.
laments that in his time this apprehension had lost much of its
effect.}

\pagenote[10]{See the fate of Syracuse, Tarentum, Ambracia,
Corinth, \&c., the conduct of Verres, in Cicero, (Actio ii. Orat.
4,) and the usual practice of governors, in the viiith Satire of
Juvenal.}

\pagenote[11]{Seuton. in Claud.—Plin. Hist. Nat. xxx. 1.}

\pagenote[12]{Pelloutier, Histoire des Celtes, tom. vi. p.
230—252.}

Rome, the capital of a great monarchy, was incessantly filled
with subjects and strangers from every part of the world,\textsuperscript{13} who
all introduced and enjoyed the favorite superstitions of their
native country.\textsuperscript{14} Every city in the empire was justified in
maintaining the purity of its ancient ceremonies; and the Roman
senate, using the common privilege, sometimes interposed, to
check this inundation of foreign rites.\textsuperscript{141} The Egyptian
superstition, of all the most contemptible and abject, was
frequently prohibited: the temples of Serapis and Isis
demolished, and their worshippers banished from Rome and Italy.\textsuperscript{15}
But the zeal of fanaticism prevailed over the cold and feeble
efforts of policy. The exiles returned, the proselytes
multiplied, the temples were restored with increasing splendor,
and Isis and Serapis at length assumed their place among the
Roman Deities.\textsuperscript{151} \textsuperscript{16} Nor was this indulgence a departure from
the old maxims of government. In the purest ages of the
commonwealth, Cybele and Æsculapius had been invited by solemn
embassies;\textsuperscript{17} and it was customary to tempt the protectors of
besieged cities, by the promise of more distinguished honors than
they possessed in their native country.\textsuperscript{18} Rome gradually became
the common temple of her subjects; and the freedom of the city
was bestowed on all the gods of mankind.\textsuperscript{19}

\pagenote[13]{Seneca, Consolat. ad Helviam, p. 74. Edit., Lips.}

\pagenote[14]{Dionysius Halicarn. Antiquitat. Roman. l. ii. (vol.
i. p. 275, edit. Reiske.)}

\pagenote[141]{Yet the worship of foreign gods at Rome was only
guarantied to the natives of those countries from whence they
came. The Romans administered the priestly offices only to the
gods of their fathers. Gibbon, throughout the whole preceding
sketch of the opinions of the Romans and their subjects, has
shown through what causes they were free from religious hatred
and its consequences. But, on the other hand the internal state
of these religions, the infidelity and hypocrisy of the upper
orders, the indifference towards all religion, in even the better
part of the common people, during the last days of the republic,
and under the Cæsars, and the corrupting principles of the
philosophers, had exercised a very pernicious influence on the
manners, and even on the constitution.—W.}

\pagenote[15]{In the year of Rome 701, the temple of Isis and
Serapis was demolished by the order of the Senate, (Dion Cassius,
l. xl. p. 252,) and even by the hands of the consul, (Valerius
Maximus, l. 3.) After the death of Cæsar it was restored at the
public expense, (Dion. l. xlvii. p. 501.) When Augustus was in
Egypt, he revered the majesty of Serapis, (Dion, l. li. p. 647;)
but in the Pomærium of Rome, and a mile round it, he prohibited
the worship of the Egyptian gods, (Dion, l. liii. p. 679; l. liv.
p. 735.) They remained, however, very fashionable under his reign
(Ovid. de Art. Amand. l. i.) and that of his successor, till the
justice of Tiberius was provoked to some acts of severity. (See
Tacit. Annal. ii. 85. Joseph. Antiquit. l. xviii. c. 3.) * Note:
See, in the pictures from the walls of Pompeii, the
representation of an Isiac temple and worship. Vestiges of
Egyptian worship have been traced in Gaul, and, I am informed,
recently in Britain, in excavations at York.— M.}

\pagenote[151]{Gibbon here blends into one, two events, distant a
hundred and sixty-six years from each other. It was in the year
of Rome 535, that the senate having ordered the destruction of
the temples of Isis and Serapis, the workman would lend his hand;
and the consul, L. Paulus himself (Valer. Max. 1, 3) seized the
axe, to give the first blow. Gibbon attribute this circumstance
to the second demolition, which took place in the year 701 and
which he considers as the first.—W.}

\pagenote[16]{Tertullian in Apologetic. c. 6, p. 74. Edit.
Havercamp. I am inclined to attribute their establishment to the
devotion of the Flavian family.}

\pagenote[17]{See Livy, l. xi. [Suppl.] and xxix.}

\pagenote[18]{Macrob. Saturnalia, l. iii. c. 9. He gives us a
form of evocation.}

\pagenote[19]{Minutius Fælix in Octavio, p. 54. Arnobius, l. vi.
p. 115.}

II. The narrow policy of preserving, without any foreign mixture,
the pure blood of the ancient citizens, had checked the fortune,
and hastened the ruin, of Athens and Sparta. The aspiring genius
of Rome sacrificed vanity to ambition, and deemed it more
prudent, as well as honorable, to adopt virtue and merit for her
own wheresoever they were found, among slaves or strangers,
enemies or barbarians.\textsuperscript{20} During the most flourishing æra of the
Athenian commonwealth, the number of citizens gradually decreased
from about thirty\textsuperscript{21} to twenty-one thousand.\textsuperscript{22} If, on the
contrary, we study the growth of the Roman republic, we may
discover, that, notwithstanding the incessant demands of wars and
colonies, the citizens, who, in the first census of Servius
Tullius, amounted to no more than eighty-three thousand, were
multiplied, before the commencement of the social war, to the
number of four hundred and sixty-three thousand men, able to bear
arms in the service of their country.\textsuperscript{23} When the allies of Rome
claimed an equal share of honors and privileges, the senate
indeed preferred the chance of arms to an ignominious concession.
The Samnites and the Lucanians paid the severe penalty of their
rashness; but the rest of the Italian states, as they
successively returned to their duty, were admitted into the bosom
of the republic,\textsuperscript{24} and soon contributed to the ruin of public
freedom. Under a democratical government, the citizens exercise
the powers of sovereignty; and those powers will be first abused,
and afterwards lost, if they are committed to an unwieldy
multitude. But when the popular assemblies had been suppressed by
the administration of the emperors, the conquerors were
distinguished from the vanquished nations, only as the first and
most honorable order of subjects; and their increase, however
rapid, was no longer exposed to the same dangers. Yet the wisest
princes, who adopted the maxims of Augustus, guarded with the
strictest care the dignity of the Roman name, and diffused the
freedom of the city with a prudent liberality.\textsuperscript{25}

\pagenote[20]{Tacit. Annal. xi. 24. The Orbis Romanus of the
learned Spanheim is a complete history of the progressive
admission of Latium, Italy, and the provinces, to the freedom of
Rome. * Note: Democratic states, observes Denina, (delle Revoluz.
d’ Italia, l. ii. c. l.), are most jealous of communication the
privileges of citizenship; monarchies or oligarchies willingly
multiply the numbers of their free subjects. The most remarkable
accessions to the strength of Rome, by the aggregation of
conquered and foreign nations, took place under the regal and
patrician—we may add, the Imperial government.—M.}

\pagenote[21]{Herodotus, v. 97. It should seem, however, that he
followed a large and popular estimation.}

\pagenote[22]{Athenæus, Deipnosophist. l. vi. p. 272. Edit.
Casaubon. Meursius de Fortunâ Atticâ, c. 4. * Note: On the number
of citizens in Athens, compare Bœckh, Public Economy of Athens,
(English Tr.,) p. 45, et seq. Fynes Clinton, Essay in Fasti Hel
lenici, vol. i. 381.—M.}

\pagenote[23]{See a very accurate collection of the numbers of
each Lustrum in M. de Beaufort, Republique Romaine, l. iv. c. 4.
Note: All these questions are placed in an entirely new point of
view by Niebuhr, (Römische Geschichte, vol. i. p. 464.) He
rejects the census of Servius fullius as unhistoric, (vol. ii. p.
78, et seq.,) and he establishes the principle that the census
comprehended all the confederate cities which had the right of
Isopolity.—M.}

\pagenote[24]{Appian. de Bell. Civil. l. i. Velleius Paterculus,
l. ii. c. 15, 16, 17.}

\pagenote[25]{Mæcenas had advised him to declare, by one edict,
all his subjects citizens. But we may justly suspect that the
historian Dion was the author of a counsel so much adapted to the
practice of his own age, and so little to that of Augustus.}

\section{Part \thesection.}

Till the privileges of Romans had been progressively extended to
all the inhabitants of the empire, an important distinction was
preserved between Italy and the provinces. The former was
esteemed the centre of public unity, and the firm basis of the
constitution. Italy claimed the birth, or at least the residence,
of the emperors and the senate.\textsuperscript{26} The estates of the Italians
were exempt from taxes, their persons from the arbitrary
jurisdiction of governors. Their municipal corporations, formed
after the perfect model of the capital,\textsuperscript{261} were intrusted, under
the immediate eye of the supreme power, with the execution of the
laws. From the foot of the Alps to the extremity of Calabria, all
the natives of Italy were born citizens of Rome. Their partial
distinctions were obliterated, and they insensibly coalesced into
one great nation, united by language, manners, and civil
institutions, and equal to the weight of a powerful empire. The
republic gloried in her generous policy, and was frequently
rewarded by the merit and services of her adopted sons. Had she
always confined the distinction of Romans to the ancient families
within the walls of the city, that immortal name would have been
deprived of some of its noblest ornaments. Virgil was a native of
Mantua; Horace was inclined to doubt whether he should call
himself an Apulian or a Lucanian; it was in Padua that an
historian was found worthy to record the majestic series of Roman
victories. The patriot family of the Catos emerged from Tusculum;
and the little town of Arpinum claimed the double honor of
producing Marius and Cicero, the former of whom deserved, after
Romulus and Camillus, to be styled the Third Founder of Rome; and
the latter, after saving his country from the designs of
Catiline, enabled her to contend with Athens for the palm of
eloquence.\textsuperscript{27}

\pagenote[26]{The senators were obliged to have one third of
their own landed property in Italy. See Plin. l. vi. ep. 19. The
qualification was reduced by Marcus to one fourth. Since the
reign of Trajan, Italy had sunk nearer to the level of the
provinces.}

\pagenote[261]{It may be doubted whether the municipal government
of the cities was not the old Italian constitution rather than a
transcript from that of Rome. The free government of the cities,
observes Savigny, was the leading characteristic of Italy.
Geschichte des Römischen Rechts, i. p. G.—M.}

\pagenote[27]{The first part of the Verona Illustrata of the
Marquis Maffei gives the clearest and most comprehensive view of
the state of Italy under the Cæsars. * Note: Compare Denina,
Revol. d’ Italia, l. ii. c. 6, p. 100, 4 to edit.}

The provinces of the empire (as they have been described in the
preceding chapter) were destitute of any public force, or
constitutional freedom. In Etruria, in Greece,\textsuperscript{28} and in Gaul,\textsuperscript{29}
it was the first care of the senate to dissolve those dangerous
confederacies, which taught mankind that, as the Roman arms
prevailed by division, they might be resisted by union. Those
princes, whom the ostentation of gratitude or generosity
permitted for a while to hold a precarious sceptre, were
dismissed from their thrones, as soon as they had performed their
appointed task of fashioning to the yoke the vanquished nations.
The free states and cities which had embraced the cause of Rome
were rewarded with a nominal alliance, and insensibly sunk into
real servitude. The public authority was everywhere exercised by
the ministers of the senate and of the emperors, and that
authority was absolute, and without control.\textsuperscript{291} But the same
salutary maxims of government, which had secured the peace and
obedience of Italy were extended to the most distant conquests. A
nation of Romans was gradually formed in the provinces, by the
double expedient of introducing colonies, and of admitting the
most faithful and deserving of the provincials to the freedom of
Rome.

\pagenote[28]{See Pausanias, l. vii. The Romans condescended to
restore the names of those assemblies, when they could no longer
be dangerous.}

\pagenote[29]{They are frequently mentioned by Cæsar. The Abbé
Dubos attempts, with very little success, to prove that the
assemblies of Gaul were continued under the emperors. Histoire de
l’Etablissement de la Monarchie Francoise, l. i. c. 4.}

\pagenote[291]{This is, perhaps, rather overstated. Most cities
retained the choice of their municipal officers: some retained
valuable privileges; Athens, for instance, in form was still a
confederate city. (Tac. Ann. ii. 53.) These privileges, indeed,
depended entirely on the arbitrary will of the emperor, who
revoked or restored them according to his caprice. See Walther
Geschichte des Römischen Rechts, i. 324—an admirable summary of
the Roman constitutional history.—M.}

“Wheresoever the Roman conquers, he inhabits,” is a very just
observation of Seneca,\textsuperscript{30} confirmed by history and experience.
The natives of Italy, allured by pleasure or by interest,
hastened to enjoy the advantages of victory; and we may remark,
that, about forty years after the reduction of Asia, eighty
thousand Romans were massacred in one day, by the cruel orders of
Mithridates.\textsuperscript{31} These voluntary exiles were engaged, for the most
part, in the occupations of commerce, agriculture, and the farm
of the revenue. But after the legions were rendered permanent by
the emperors, the provinces were peopled by a race of soldiers;
and the veterans, whether they received the reward of their
service in land or in money, usually settled with their families
in the country, where they had honorably spent their youth.
Throughout the empire, but more particularly in the western
parts, the most fertile districts, and the most convenient
situations, were reserved for the establishment of colonies; some
of which were of a civil, and others of a military nature. In
their manners and internal policy, the colonies formed a perfect
representation of their great parent; and they were soon endeared
to the natives by the ties of friendship and alliance, they
effectually diffused a reverence for the Roman name, and a
desire, which was seldom disappointed, of sharing, in due time,
its honors and advantages.\textsuperscript{32} The municipal cities insensibly
equalled the rank and splendor of the colonies; and in the reign
of Hadrian, it was disputed which was the preferable condition,
of those societies which had issued from, or those which had been
received into, the bosom of Rome.\textsuperscript{33} The right of Latium, as it
was called,\textsuperscript{331} conferred on the cities to which it had been
granted, a more partial favor. The magistrates only, at the
expiration of their office, assumed the quality of Roman
citizens; but as those offices were annual, in a few years they
circulated round the principal families.\textsuperscript{34} Those of the
provincials who were permitted to bear arms in the legions;\textsuperscript{35}
those who exercised any civil employment; all, in a word, who
performed any public service, or displayed any personal talents,
were rewarded with a present, whose value was continually
diminished by the increasing liberality of the emperors. Yet
even, in the age of the Antonines, when the freedom of the city
had been bestowed on the greater number of their subjects, it was
still accompanied with very solid advantages. The bulk of the
people acquired, with that title, the benefit of the Roman laws,
particularly in the interesting articles of marriage, testaments,
and inheritances; and the road of fortune was open to those whose
pretensions were seconded by favor or merit. The grandsons of the
Gauls, who had besieged Julius Cæsar in Alesia, commanded
legions, governed provinces, and were admitted into the senate of
Rome.\textsuperscript{36} Their ambition, instead of disturbing the tranquillity
of the state, was intimately connected with its safety and
greatness.

\pagenote[30]{Seneca in Consolat. ad Helviam, c. 6.}

\pagenote[31]{Memnon apud Photium, (c. 33,) [c. 224, p. 231, ed
Bekker.] Valer. Maxim. ix. 2. Plutarch and Dion Cassius swell the
massacre to 150,000 citizens; but I should esteem the smaller
number to be more than sufficient.}

\pagenote[32]{Twenty-five colonies were settled in Spain, (see
Plin. Hist. Nat. iii. 3, 4; iv. 35;) and nine in Britain, of
which London, Colchester, Lincoln, Chester, Gloucester, and Bath
still remain considerable cities. (See Richard of Cirencester, p.
36, and Whittaker’s History of Manchester, l. i. c. 3.)}

\pagenote[33]{Aul. Gel. Noctes Atticæ, xvi 13. The Emperor
Hadrian expressed his surprise, that the cities of Utica, Gades,
and Italica, which already enjoyed the rights of \textit{Municipia},
should solicit the title of \textit{colonies}. Their example, however,
became fashionable, and the empire was filled with honorary
colonies. See Spanheim, de Usu Numismatum Dissertat. xiii.}

\pagenote[331]{The right of Latium conferred an exemption from
the government of the Roman præfect. Strabo states this
distinctly, l. iv. p. 295, edit. Cæsar’s. See also Walther, p.
233.—M}

\pagenote[34]{Spanheim, Orbis Roman. c. 8, p. 62.}

\pagenote[35]{Aristid. in Romæ Encomio. tom. i. p. 218, edit.
Jebb.}

\pagenote[36]{Tacit. Annal. xi. 23, 24. Hist. iv. 74.}

So sensible were the Romans of the influence of language over
national manners, that it was their most serious care to extend,
with the progress of their arms, the use of the Latin tongue.\textsuperscript{37}
The ancient dialects of Italy, the Sabine, the Etruscan, and the
Venetian, sunk into oblivion; but in the provinces, the east was
less docile than the west to the voice of its victorious
preceptors. This obvious difference marked the two portions of
the empire with a distinction of colors, which, though it was in
some degree concealed during the meridian splendor of prosperity,
became gradually more visible, as the shades of night descended
upon the Roman world. The western countries were civilized by the
same hands which subdued them. As soon as the barbarians were
reconciled to obedience, their minds were open to any new
impressions of knowledge and politeness. The language of Virgil
and Cicero, though with some inevitable mixture of corruption,
was so universally adopted in Africa, Spain, Gaul, Britain, and
Pannonia,\textsuperscript{38} that the faint traces of the Punic or Celtic idioms
were preserved only in the mountains, or among the peasants.\textsuperscript{39}
Education and study insensibly inspired the natives of those
countries with the sentiments of Romans; and Italy gave fashions,
as well as laws, to her Latin provincials. They solicited with
more ardor, and obtained with more facility, the freedom and
honors of the state; supported the national dignity in letters\textsuperscript{40}
and in arms; and at length, in the person of Trajan, produced an
emperor whom the Scipios would not have disowned for their
countryman. The situation of the Greeks was very different from
that of the barbarians. The former had been long since civilized
and corrupted. They had too much taste to relinquish their
language, and too much vanity to adopt any foreign institutions.
Still preserving the prejudices, after they had lost the virtues,
of their ancestors, they affected to despise the unpolished
manners of the Roman conquerors, whilst they were compelled to
respect their superior wisdom and power.\textsuperscript{41} Nor was the influence
of the Grecian language and sentiments confined to the narrow
limits of that once celebrated country. Their empire, by the
progress of colonies and conquest, had been diffused from the
Adriatic to the Euphrates and the Nile. Asia was covered with
Greek cities, and the long reign of the Macedonian kings had
introduced a silent revolution into Syria and Egypt. In their
pompous courts, those princes united the elegance of Athens with
the luxury of the East, and the example of the court was
imitated, at an humble distance, by the higher ranks of their
subjects. Such was the general division of the Roman empire into
the Latin and Greek languages. To these we may add a third
distinction for the body of the natives in Syria, and especially
in Egypt, the use of their ancient dialects, by secluding them
from the commerce of mankind, checked the improvements of those
barbarians.\textsuperscript{42} The slothful effeminacy of the former exposed them
to the contempt, the sullen ferociousness of the latter excited
the aversion, of the conquerors.\textsuperscript{43} Those nations had submitted
to the Roman power, but they seldom desired or deserved the
freedom of the city: and it was remarked, that more than two
hundred and thirty years elapsed after the ruin of the Ptolemies,
before an Egyptian was admitted into the senate of Rome.\textsuperscript{44}

\pagenote[37]{See Plin. Hist. Natur. iii. 5. Augustin. de
Civitate Dei, xix 7 Lipsius de Pronunciatione Linguæ Latinæ, c.
3.}

\pagenote[38]{Apuleius and Augustin will answer for Africa;
Strabo for Spain and Gaul; Tacitus, in the life of Agricola, for
Britain; and Velleius Paterculus, for Pannonia. To them we may
add the language of the Inscriptions. * Note: Mr. Hallam contests
this assertion as regards Britain. “Nor did the Romans ever
establish their language—I know not whether they wished to do
so—in this island, as we perceive by that stubborn British tongue
which has survived two conquests.” In his note, Mr. Hallam
examines the passage from Tacitus (Agric. xxi.) to which Gibbon
refers. It merely asserts the progress of Latin studies among the
higher orders. (Midd. Ages, iii. 314.) Probably it was a kind of
court language, and that of public affairs and prevailed in the
Roman colonies.—M.}

\pagenote[39]{The Celtic was preserved in the mountains of Wales,
Cornwall, and Armorica. We may observe, that Apuleius reproaches
an African youth, who lived among the populace, with the use of
the Punic; whilst he had almost forgot Greek, and neither could
nor would speak Latin, (Apolog. p. 596.) The greater part of St.
Austin’s congregations were strangers to the Punic.}

\pagenote[40]{Spain alone produced Columella, the Senecas, Lucan,
Martial, and Quintilian.}

\pagenote[41]{There is not, I believe, from Dionysius to Libanus,
a single Greek critic who mentions Virgil or Horace. They seem
ignorant that the Romans had any good writers.}

\pagenote[42]{The curious reader may see in Dupin, (Bibliotheque
Ecclesiastique, tom. xix. p. 1, c. 8,) how much the use of the
Syriac and Egyptian languages was still preserved.}

\pagenote[43]{See Juvenal, Sat. iii. and xv. Ammian. Marcellin.
xxii. 16.}

\pagenote[44]{Dion Cassius, l. lxxvii. p. 1275. The first
instance happened under the reign of Septimius Severus.}

It is a just though trite observation, that victorious Rome was
herself subdued by the arts of Greece. Those immortal writers who
still command the admiration of modern Europe, soon became the
favorite object of study and imitation in Italy and the western
provinces. But the elegant amusements of the Romans were not
suffered to interfere with their sound maxims of policy. Whilst
they acknowledged the charms of the Greek, they asserted the
dignity of the Latin tongue, and the exclusive use of the latter
was inflexibly maintained in the administration of civil as well
as military government.\textsuperscript{45} The two languages exercised at the
same time their separate jurisdiction throughout the empire: the
former, as the natural idiom of science; the latter, as the legal
dialect of public transactions. Those who united letters with
business were equally conversant with both; and it was almost
impossible, in any province, to find a Roman subject, of a
liberal education, who was at once a stranger to the Greek and to
the Latin language.

\pagenote[45]{See Valerius Maximus, l. ii. c. 2, n. 2. The
emperor Claudius disfranchised an eminent Grecian for not
understanding Latin. He was probably in some public office.
Suetonius in Claud. c. 16. * Note: Causes seem to have been
pleaded, even in the senate, in both languages. Val. Max. \textit{loc.
cit}. Dion. l. lvii. c. 15.—M}

It was by such institutions that the nations of the empire
insensibly melted away into the Roman name and people. But there
still remained, in the centre of every province and of every
family, an unhappy condition of men who endured the weight,
without sharing the benefits, of society. In the free states of
antiquity, the domestic slaves were exposed to the wanton rigor
of despotism. The perfect settlement of the Roman empire was
preceded by ages of violence and rapine. The slaves consisted,
for the most part, of barbarian captives,\textsuperscript{451} taken in thousands
by the chance of war, purchased at a vile price,\textsuperscript{46} accustomed to
a life of independence, and impatient to break and to revenge
their fetters. Against such internal enemies, whose desperate
insurrections had more than once reduced the republic to the
brink of destruction,\textsuperscript{47} the most
severe\textsuperscript{471} regulations, \textsuperscript{48} and
the most cruel treatment, seemed almost justified by the great
law of self-preservation. But when the principal nations of
Europe, Asia, and Africa were united under the laws of one
sovereign, the source of foreign supplies flowed with much less
abundance, and the Romans were reduced to the milder but more
tedious method of propagation.\textsuperscript{481} In their numerous families,
and particularly in their country estates, they encouraged the
marriage of their slaves.\textsuperscript{482} The sentiments of nature, the
habits of education, and the possession of a dependent species of
property, contributed to alleviate the hardships of servitude.\textsuperscript{49}
The existence of a slave became an object of greater value, and
though his happiness still depended on the temper and
circumstances of the master, the humanity of the latter, instead
of being restrained by fear, was encouraged by the sense of his
own interest. The progress of manners was accelerated by the
virtue or policy of the emperors; and by the edicts of Hadrian
and the Antonines, the protection of the laws was extended to the
most abject part of mankind. The jurisdiction of life and death
over the slaves, a power long exercised and often abused, was
taken out of private hands, and reserved to the magistrates
alone. The subterraneous prisons were abolished; and, upon a just
complaint of intolerable treatment, the injured slave obtained
either his deliverance, or a less cruel master.\textsuperscript{50}

\pagenote[451]{It was this which rendered the wars so sanguinary,
and the battles so obstinate. The immortal Robertson, in an
excellent discourse on the state of the world at the period of
the establishment of Christianity, has traced a picture of the
melancholy effects of slavery, in which we find all the depth of
his views and the strength of his mind. I shall oppose
successively some passages to the reflections of Gibbon. The
reader will see, not without interest, the truths which Gibbon
appears to have mistaken or voluntarily neglected, developed by
one of the best of modern historians. It is important to call
them to mind here, in order to establish the facts and their
consequences with accuracy. I shall more than once have occasion
to employ, for this purpose, the discourse of Robertson.
“Captives taken in war were, in all probability, the first
persons subjected to perpetual servitude; and, when the
necessities or luxury of mankind increased the demand for slaves,
every new war recruited their number, by reducing the vanquished
to that wretched condition. Hence proceeded the fierce and
desperate spirit with which wars were carried on among ancient
nations. While chains and slavery were the certain lot of the
conquered, battles were fought, and towns defended with a rage
and obstinacy which nothing but horror at such a fate could have
inspired; but, putting an end to the cruel institution of
slavery, Christianity extended its mild influences to the
practice of war, and that barbarous art, softened by its humane
spirit, ceased to be so destructive. Secure, in every event, of
personal liberty, the resistance of the vanquished became less
obstinate, and the triumph of the victor less cruel. Thus
humanity was introduced into the exercise of war, with which it
appears to be almost incompatible; and it is to the merciful
maxims of Christianity, much more than to any other cause, that
we must ascribe the little ferocity and bloodshed which accompany
modern victories.”—G.}

\pagenote[46]{In the camp of Lucullus, an ox sold for a drachma,
and a slave for four drachmæ, or about three shillings. Plutarch.
in Lucull. p. 580. * Note: Above 100,000 prisoners were taken in
the Jewish war.—G. Hist. of Jews, iii. 71. According to a
tradition preserved by S. Jerom, after the insurrection in the
time of Hadrian, they were sold as cheap as horse. Ibid. 124.
Compare Blair on Roman Slavery, p. 19.—M., and Dureau de la
blalle, Economie Politique des Romains, l. i. c. 15. But I cannot
think that this writer has made out his case as to the common
price of an agricultural slave being from 2000 to 2500 francs,
(80l. to 100l.) He has overlooked the passages which show the
ordinary prices, (i. e. Hor. Sat. ii. vii. 45,) and argued from
extraordinary and exceptional cases.—M. 1845.}

\pagenote[47]{Diodorus Siculus in Eclog. Hist. l. xxxiv. and
xxxvi. Florus, iii. 19, 20.}

\pagenote[471]{The following is the example: we shall see whether
the word “severe” is here in its place. “At the time in which L.
Domitius was prætor in Sicily, a slave killed a wild boar of
extraordinary size. The prætor, struck by the dexterity and
courage of the man, desired to see him. The poor wretch, highly
gratified with the distinction, came to present himself before
the prætor, in hopes, no doubt, of praise and reward; but
Domitius, on learning that he had only a javelin to attack and
kill the boar, ordered him to be instantly crucified, under the
barbarous pretext that the law prohibited the use of this weapon,
as of all others, to slaves.” Perhaps the cruelty of Domitius is
less astonishing than the indifference with which the Roman
orator relates this circumstance, which affects him so little
that he thus expresses himself: “Durum hoc fortasse videatur,
neque ego in ullam partem disputo.” “This may appear harsh, nor
do I give any opinion on the subject.” And it is the same orator
who exclaims in the same oration, “Facinus est cruciare civem
Romanum; scelus verberare; prope parricidium necare: quid dicam
in crucem tollere?” “It is a crime to imprison a Roman citizen;
wickedness to scourge; next to parricide to put to death, what
shall I call it to crucify?”\\
In general, this passage of Gibbon on slavery, is full, not only
of blamable indifference, but of an exaggeration of impartiality
which resembles dishonesty. He endeavors to extenuate all that is
appalling in the condition and treatment of the slaves; he would
make us consider those cruelties as possibly “justified by
necessity.” He then describes, with minute accuracy, the
slightest mitigations of their deplorable condition; he
attributes to the virtue or the policy of the emperors the
progressive amelioration in the lot of the slaves; and he passes
over in silence the most influential cause, that which, after
rendering the slaves less miserable, has contributed at length
entirely to enfranchise them from their sufferings and their
chains,—Christianity. It would be easy to accumulate the most
frightful, the most agonizing details, of the manner in which the
Romans treated their slaves; whole works have been devoted to the
description. I content myself with referring to them. Some
reflections of Robertson, taken from the discourse already
quoted, will make us feel that Gibbon, in tracing the mitigation
of the condition of the slaves, up to a period little later than
that which witnessed the establishment of Christianity in the
world, could not have avoided the acknowledgment of the influence
of that beneficent cause, if he had not already determined not to
speak of it.\\
“Upon establishing despotic government in the Roman empire,
domestic tyranny rose, in a short time, to an astonishing height.
In that rank soil, every vice, which power nourishes in the
great, or oppression engenders in the mean, thrived and grew up
apace. * * * It is not the authority of any single detached
precept in the gospel, but the spirit and genius of the Christian
religion, more powerful than any particular command, which hath
abolished the practice of slavery throughout the world. The
temper which Christianity inspired was mild and gentle; and the
doctrines it taught added such dignity and lustre to human
nature, as rescued it from the dishonorable servitude into which
it was sunk.”\\
It is in vain, then, that Gibbon pretends to attribute solely to
the desire of keeping up the number of slaves, the milder conduct
which the Romans began to adopt in their favor at the time of the
emperors. This cause had hitherto acted in an opposite direction;
how came it on a sudden to have a different influence? “The
masters,” he says, “encouraged the marriage of their slaves; * *
* the sentiments of nature, the habits of education, contributed
to alleviate the hardships of servitude.” The children of slaves
were the property of their master, who could dispose of or
alienate them like the rest of his property. Is it in such a
situation, with such notions, that the sentiments of nature
unfold themselves, or habits of education become mild and
peaceful? We must not attribute to causes inadequate or
altogether without force, effects which require to explain them a
reference to more influential causes; and even if these slighter
causes had in effect a manifest influence, we must not forget
that they are themselves the effect of a primary, a higher, and
more extensive cause, which, in giving to the mind and to the
character a more disinterested and more humane bias, disposed men
to second or themselves to advance, by their conduct, and by the
change of manners, the happy results which it tended to
produce.—G.\\
I have retained the whole of M. Guizot’s note, though, in his
zeal for the invaluable blessings of freedom and Christianity, he
has done Gibbon injustice. The condition of the slaves was
undoubtedly improved under the emperors. What a great authority
has said, “The condition of a slave is better under an arbitrary
than under a free government,” (Smith’s Wealth of Nations, iv.
7,) is, I believe, supported by the history of all ages and
nations. The protecting edicts of Hadrian and the Antonines are
historical facts, and can as little be attributed to the
influence of Christianity, as the milder language of heathen
writers, of Seneca, (particularly Ep. 47,) of Pliny, and of
Plutarch. The latter influence of Christianity is admitted by
Gibbon himself. The subject of Roman slavery has recently been
investigated with great diligence in a very modest but valuable
volume, by Wm. Blair, Esq., Edin. 1833. May we be permitted,
while on the subject, to refer to the most splendid passage
extant of Mr. Pitt’s eloquence, the description of the Roman
slave-dealer. on the shores of Britain, condemning the island to
irreclaimable barbarism, as a perpetual and prolific nursery of
slaves? Speeches, vol. ii. p. 80.\\
Gibbon, it should be added, was one of the first and most
consistent opponents of the African slave-trade. (See Hist. ch.
xxv. and Letters to Lor Sheffield, Misc. Works)—M.}

\pagenote[48]{See a remarkable instance of severity in Cicero in
Verrem, v. 3.}

\pagenote[481]{An active slave-trade, which was carried on in
many quarters, particularly the Euxine, the eastern provinces,
the coast of Africa, and British must be taken into the account.
Blair, 23—32.—M.}

\pagenote[482]{The Romans, as well in the first ages of the
republic as later, allowed to their slaves a kind of marriage,
(contubernium: ) notwithstanding this, luxury made a greater
number of slaves in demand. The increase in their population was
not sufficient, and recourse was had to the purchase of slaves,
which was made even in the provinces of the East subject to the
Romans. It is, moreover, known that slavery is a state little
favorable to population. (See Hume’s Essay, and Malthus on
population, i. 334.—G.) The testimony of Appian (B.C. l. i. c. 7)
is decisive in favor of the rapid multiplication of the
agricultural slaves; it is confirmed by the numbers engaged in
the servile wars. Compare also Blair, p. 119; likewise Columella
l. viii.—M.}

\pagenote[49]{See in Gruter, and the other collectors, a great
number of inscriptions addressed by slaves to their wives,
children, fellow-servants, masters, \&c. They are all most
probably of the Imperial age.}

\pagenote[50]{See the Augustan History, and a Dissertation of M.
de Burigny, in the xxxvth volume of the Academy of Inscriptions,
upon the Roman slaves.}

Hope, the best comfort of our imperfect condition, was not denied
to the Roman slave; and if he had any opportunity of rendering
himself either useful or agreeable, he might very naturally
expect that the diligence and fidelity of a few years would be
rewarded with the inestimable gift of freedom. The benevolence of
the master was so frequently prompted by the meaner suggestions
of vanity and avarice, that the laws found it more necessary to
restrain than to encourage a profuse and undistinguishing
liberality, which might degenerate into a very dangerous abuse.\textsuperscript{51}
It was a maxim of ancient jurisprudence, that a slave had not
any country of his own; he acquired with his liberty an admission
into the political society of which his patron was a member. The
consequences of this maxim would have prostituted the privileges
of the Roman city to a mean and promiscuous multitude. Some
seasonable exceptions were therefore provided; and the honorable
distinction was confined to such slaves only as, for just causes,
and with the approbation of the magistrate, should receive a
solemn and legal manumission. Even these chosen freedmen obtained
no more than the private rights of citizens, and were rigorously
excluded from civil or military honors. Whatever might be the
merit or fortune of their sons, \textit{they} likewise were esteemed
unworthy of a seat in the senate; nor were the traces of a
servile origin allowed to be completely obliterated till the
third or fourth generation.\textsuperscript{52} Without destroying the distinction
of ranks, a distant prospect of freedom and honors was presented,
even to those whom pride and prejudice almost disdained to number
among the human species.

\pagenote[51]{See another Dissertation of M. de Burigny, in the
xxxviith volume, on the Roman freedmen.}

\pagenote[52]{Spanheim, Orbis Roman. l. i. c. 16, p. 124, \&c.}
It was once proposed to discriminate the slaves by a peculiar habit;
but it was justly apprehended that there might be some danger in
acquainting them with their own numbers.\textsuperscript{53} Without interpreting,
in their utmost strictness, the liberal appellations of legions
and myriads,\textsuperscript{54} we may venture to pronounce, that the proportion
of slaves, who were valued as property, was more considerable
than that of servants, who can be computed only as an expense.\textsuperscript{55}
The youths of a promising genius were instructed in the arts and
sciences, and their price was ascertained by the degree of their
skill and talents.\textsuperscript{56} Almost every profession, either liberal\textsuperscript{57}
or mechanical, might be found in the household of an opulent
senator. The ministers of pomp and sensuality were multiplied
beyond the conception of modern luxury.\textsuperscript{58} It was more for the
interest of the merchant or manufacturer to purchase, than to
hire his workmen; and in the country, slaves were employed as the
cheapest and most laborious instruments of agriculture. To
confirm the general observation, and to display the multitude of
slaves, we might allege a variety of particular instances. It was
discovered, on a very melancholy occasion, that four hundred
slaves were maintained in a single palace of Rome.\textsuperscript{59} The same
number of four hundred belonged to an estate which an African
widow, of a very private condition, resigned to her son, whilst
she reserved for herself a much larger share of her property.\textsuperscript{60}
A freedman, under the name of Augustus, though his fortune had
suffered great losses in the civil wars, left behind him three
thousand six hundred yoke of oxen, two hundred and fifty thousand
head of smaller cattle, and what was almost included in the
description of cattle, four thousand one hundred and sixteen
slaves.\textsuperscript{61}

\pagenote[53]{Seneca de Clementia, l. i. c. 24. The original is
much stronger, “Quantum periculum immineret si servi nostri
numerare nos cœpissent.”}

\pagenote[54]{See Pliny (Hist. Natur. l. xxxiii.) and Athenæus
(Deipnosophist. l. vi. p. 272.) The latter boldly asserts, that
he knew very many Romans who possessed, not for use, but
ostentation, ten and even twenty thousand slaves.}

\pagenote[55]{In Paris there are not more than 43,000 domestics
of every sort, and not a twelfth part of the inhabitants.
Messange, Recherches sui la Population, p. 186.}

\pagenote[56]{A learned slave sold for many hundred pounds
sterling: Atticus always bred and taught them himself. Cornel.
Nepos in Vit. c. 13, [on the prices of slaves. Blair, 149.]—M.}

\pagenote[57]{Many of the Roman physicians were slaves. See Dr.
Middleton’s Dissertation and Defence.}

\pagenote[58]{Their ranks and offices are very copiously
enumerated by Pignorius de Servis.}

\pagenote[59]{Tacit. Annal. xiv. 43. They were all executed for
not preventing their master’s murder. * Note: The remarkable
speech of Cassius shows the proud feelings of the Roman
aristocracy on this subject.—M}

\pagenote[60]{Apuleius in Apolog. p. 548. edit. Delphin}

\pagenote[61]{Plin. Hist. Natur. l. xxxiii. 47.}

The number of subjects who acknowledged the laws of Rome, of
citizens, of provincials, and of slaves, cannot now be fixed with
such a degree of accuracy, as the importance of the object would
deserve. We are informed, that when the Emperor Claudius
exercised the office of censor, he took an account of six
millions nine hundred and forty-five thousand Roman citizens,
who, with the proportion of women and children, must have
amounted to about twenty millions of souls. The multitude of
subjects of an inferior rank was uncertain and fluctuating. But,
after weighing with attention every circumstance which could
influence the balance, it seems probable that there existed, in
the time of Claudius, about twice as many provincials as there
were citizens, of either sex, and of every age; and that the
slaves were at least equal in number to the free inhabitants of
the Roman world.\textsuperscript{611} The total amount of this imperfect
calculation would rise to about one hundred and twenty millions
of persons; a degree of population which possibly exceeds that of
modern Europe,\textsuperscript{62} and forms the most numerous society that has
ever been united under the same system of government.

\pagenote[611]{According to Robertson, there were twice as many slaves as free
citizens.—G. Mr. Blair (p. 15) estimates three slaves to one
freeman, between the conquest of Greece, B.C. 146, and the reign
of Alexander Severus, A. D. 222, 235. The proportion was probably
larger in Italy than in the provinces.—M. On the other hand,
Zumpt, in his Dissertation quoted below, (p. 86,) asserts it to
be a gross error in Gibbon to reckon the number of slaves equal
to that of the free population. The luxury and magnificence of
the great, (he observes,) at the commencement of the empire, must
not be taken as the groundwork of calculations for the whole
Roman world. “The agricultural laborer, and the artisan, in
Spain, Gaul, Britain, Syria, and Egypt, maintained himself, as in
the present day, by his own labor and that of his household,
without possessing a single slave.” The latter part of my note
was intended to suggest this consideration. Yet so completely was
slavery rooted in the social system, both in the east and the
west, that in the great diffusion of wealth at this time, every
one, I doubt not, who could afford a domestic slave, kept one;
and generally, the number of slaves was in proportion to the
wealth. I do not believe that the cultivation of the soil by
slaves was confined to Italy; the holders of large estates in the
provinces would probably, either from choice or necessity, adopt
the same mode of cultivation. The latifundia, says Pliny, had
ruined Italy, and had begun to ruin the provinces. Slaves were no
doubt employed in agricultural labor to a great extent in Sicily,
and were the estates of those six enormous landholders who were
said to have possessed the whole province of Africa, cultivated
altogether by free coloni? Whatever may have been the case in the
rural districts, in the towns and cities the household duties
were almost entirely discharged by slaves, and vast numbers
belonged to the public establishments. I do not, however, differ
so far from Zumpt, and from M. Dureau de la Malle, as to adopt
the higher and bolder estimate of Robertson and Mr. Blair, rather
than the more cautious suggestions of Gibbon. I would reduce
rather than increase the proportion of the slave population. The
very ingenious and elaborate calculations of the French writer,
by which he deduces the amount of the population from the produce
and consumption of corn in Italy, appear to me neither precise
nor satisfactory bases for such complicated political arithmetic.
I am least satisfied with his views as to the population of the
city of Rome; but this point will be more fitly reserved for a
note on the thirty-first chapter of Gibbon. The work, however, of
M. Dureau de la Malle is very curious and full on some of the
minuter points of Roman statistics.—M. 1845.}

\pagenote[62]{Compute twenty millions in France, twenty-two in
Germany, four in Hungary, ten in Italy with its islands, eight in
Great Britain and Ireland, eight in Spain and Portugal, ten or
twelve in the European Russia, six in Poland, six in Greece and
Turkey, four in Sweden, three in Denmark and Norway, four in the
Low Countries. The whole would amount to one hundred and five, or
one hundred and seven millions. See Voltaire, de l’Histoire
Generale. * Note: The present population of Europe is estimated
at 227,700,000. Malts Bran, Geogr. Trans edit. 1832 See details
in the different volumes Another authority, (Almanach de Gotha,)
quoted in a recent English publication, gives the following
details:—\\
France, 32,897,521 Germany, (including Hungary, Prussian and
Austrian Poland,) 56,136,\\213 Italy, 20,548,616 Great Britain and
Ireland, 24,062,947 Spain and Portugal, 13,953,959. 3,144,000
Russia, including Poland, 44,220,600 Cracow, 128,480 Turkey,
(including Pachalic of Dschesair,) 9,545,300 Greece, 637,700
Ionian Islands, 208,100 Sweden and Norway, 3,914,\\963 Denmark,
2,012,998 Belgium, 3,533,538 Holland, 2,444,550 Switzerland,
985,000. Total, 219,344,116\\
Since the publication of my first annotated edition of Gibbon,
the subject of the population of the Roman empire has been
investigated by two writers of great industry and learning; Mons.
Dureau de la Malle, in his Economie Politique des Romains, liv.
ii. c. 1. to 8, and M. Zumpt, in a dissertation printed in the
Transactions of the Berlin Academy, 1840. M. Dureau de la Malle
confines his inquiry almost entirely to the city of Rome, and
Roman Italy. Zumpt examines at greater length the axiom, which he
supposes to have been assumed by Gibbon as unquestionable, “that
Italy and the Roman world was never so populous as in the time of
the Antonines.” Though this probably was Gibbon’s opinion, he has
not stated it so peremptorily as asserted by Mr. Zumpt. It had
before been expressly laid down by Hume, and his statement was
controverted by Wallace and by Malthus. Gibbon says (p. 84) that
there is no reason to believe the country (of Italy) less
populous in the age of the Antonines, than in that of Romulus;
and Zumpt acknowledges that we have no satisfactory knowledge of
the state of Italy at that early age. Zumpt, in my opinion with
some reason, takes the period just before the first Punic war, as
that in which Roman Italy (all south of the Rubicon) was most
populous. From that time, the numbers began to diminish, at first
from the enormous waste of life out of the free population in the
foreign, and afterwards in the civil wars; from the cultivation
of the soil by slaves; towards the close of the republic, from
the repugnance to marriage, which resisted alike the dread of
legal punishment and the offer of legal immunity and privilege;
and from the depravity of manners, which interfered with the
procreation, the birth, and the rearing of children. The
arguments and the authorities of Zumpt are equally conclusive as
to the decline of population in Greece. Still the details, which
he himself adduces as to the prosperity and populousness of Asia
Minor, and the whole of the Roman East, with the advancement of
the European provinces, especially Gaul, Spain, and Britain, in
civilization, and therefore in populousness, (for I have no
confidence in the vast numbers sometimes assigned to the
barbarous inhabitants of these countries,) may, I think, fairly
compensate for any deduction to be made from Gibbon’s general
estimate on account of Greece and Italy. Gibbon himself
acknowledges his own estimate to be vague and conjectural; and I
may venture to recommend the dissertation of Zumpt as deserving
respectful consideration.—M 1815.}

\section{Part \thesection.}

Domestic peace and union were the natural consequences of the
moderate and comprehensive policy embraced by the Romans. If we
turn our eyes towards the monarchies of Asia, we shall behold
despotism in the centre, and weakness in the extremities; the
collection of the revenue, or the administration of justice,
enforced by the presence of an army; hostile barbarians
established in the heart of the country, hereditary satraps
usurping the dominion of the provinces, and subjects inclined to
rebellion, though incapable of freedom. But the obedience of the
Roman world was uniform, voluntary, and permanent. The vanquished
nations, blended into one great people, resigned the hope, nay,
even the wish, of resuming their independence, and scarcely
considered their own existence as distinct from the existence of
Rome. The established authority of the emperors pervaded without
an effort the wide extent of their dominions, and was exercised
with the same facility on the banks of the Thames, or of the
Nile, as on those of the Tyber. The legions were destined to
serve against the public enemy, and the civil magistrate seldom
required the aid of a military force.\textsuperscript{63} In this state of general
security, the leisure, as well as opulence, both of the prince
and people, were devoted to improve and to adorn the Roman
empire.

\pagenote[63]{Joseph. de Bell. Judaico, l. ii. c. 16. The oration
of Agrippa, or rather of the historian, is a fine picture of the
Roman empire.}

Among the innumerable monuments of architecture constructed by
the Romans, how many have escaped the notice of history, how few
have resisted the ravages of time and barbarism! And yet, even
the majestic ruins that are still scattered over Italy and the
provinces, would be sufficient to prove that those countries were
once the seat of a polite and powerful empire. Their greatness
alone, or their beauty, might deserve our attention: but they are
rendered more interesting, by two important circumstances, which
connect the agreeable history of the arts with the more useful
history of human manners. Many of those works were erected at
private expense, and almost all were intended for public benefit.

It is natural to suppose that the greatest number, as well as the
most considerable of the Roman edifices, were raised by the
emperors, who possessed so unbounded a command both of men and
money. Augustus was accustomed to boast that he had found his
capital of brick, and that he had left it of marble.\textsuperscript{64} The
strict economy of Vespasian was the source of his magnificence.
The works of Trajan bear the stamp of his genius. The public
monuments with which Hadrian adorned every province of the
empire, were executed not only by his orders, but under his
immediate inspection. He was himself an artist; and he loved the
arts, as they conduced to the glory of the monarch. They were
encouraged by the Antonines, as they contributed to the happiness
of the people. But if the emperors were the first, they were not
the only architects of their dominions. Their example was
universally imitated by their principal subjects, who were not
afraid of declaring to the world that they had spirit to
conceive, and wealth to accomplish, the noblest undertakings.
Scarcely had the proud structure of the Coliseum been dedicated
at Rome, before the edifices, of a smaller scale indeed, but of
the same design and materials, were erected for the use, and at
the expense, of the cities of Capua and Verona.\textsuperscript{65} The
inscription of the stupendous bridge of Alcantara attests that it
was thrown over the Tagus by the contribution of a few Lusitanian
communities. When Pliny was intrusted with the government of
Bithynia and Pontus, provinces by no means the richest or most
considerable of the empire, he found the cities within his
jurisdiction striving with each other in every useful and
ornamental work, that might deserve the curiosity of strangers,
or the gratitude of their citizens. It was the duty of the
proconsul to supply their deficiencies, to direct their taste,
and sometimes to moderate their emulation.\textsuperscript{66} The opulent
senators of Rome and the provinces esteemed it an honor, and
almost an obligation, to adorn the splendor of their age and
country; and the influence of fashion very frequently supplied
the want of taste or generosity. Among a crowd of these private
benefactors, we may select Herodes Atticus, an Athenian citizen,
who lived in the age of the Antonines. Whatever might be the
motive of his conduct, his magnificence would have been worthy of
the greatest kings.

\pagenote[64]{Sueton. in August. c. 28. Augustus built in Rome
the temple and forum of Mars the Avenger; the temple of Jupiter
Tonans in the Capitol; that of Apollo Palatine, with public
libraries; the portico and basilica of Caius and Lucius; the
porticos of Livia and Octavia; and the theatre of Marcellus. The
example of the sovereign was imitated by his ministers and
generals; and his friend Agrippa left behind him the immortal
monument of the Pantheon.}

\pagenote[65]{See Maffei, Veroni Illustrata, l. iv. p. 68.}

\pagenote[66]{Footnote 66: See the xth book of Pliny’s Epistles.
He mentions the following works carried on at the expense of the
cities. At Nicomedia, a new forum, an aqueduct, and a canal, left
unfinished by a king; at Nice, a gymnasium, and a theatre, which
had already cost near ninety thousand pounds; baths at Prusa and
Claudiopolis, and an aqueduct of sixteen miles in length for the
use of Sinope.}

The family of Herod, at least after it had been favored by
fortune, was lineally descended from Cimon and Miltiades, Theseus
and Cecrops, Æacus and Jupiter. But the posterity of so many gods
and heroes was fallen into the most abject state. His grandfather
had suffered by the hands of justice, and Julius Atticus, his
father, must have ended his life in poverty and contempt, had he
not discovered an immense treasure buried under an old house, the
last remains of his patrimony. According to the rigor of the law,
the emperor might have asserted his claim, and the prudent
Atticus prevented, by a frank confession, the officiousness of
informers. But the equitable Nerva, who then filled the throne,
refused to accept any part of it, and commanded him to use,
without scruple, the present of fortune. The cautious Athenian
still insisted, that the treasure was too considerable for a
subject, and that he knew not how to \textit{use it. Abuse it then},
replied the monarch, with a good-natured peevishness; for it is
your own.\textsuperscript{67} Many will be of opinion, that Atticus literally
obeyed the emperor’s last instructions; since he expended the
greatest part of his fortune, which was much increased by an
advantageous marriage, in the service of the public. He had
obtained for his son Herod the prefecture of the free cities of
Asia; and the young magistrate, observing that the town of Troas
was indifferently supplied with water, obtained from the
munificence of Hadrian three hundred myriads of drachms, (about a
hundred thousand pounds,) for the construction of a new aqueduct.
But in the execution of the work, the charge amounted to more
than double the estimate, and the officers of the revenue began
to murmur, till the generous Atticus silenced their complaints,
by requesting that he might be permitted to take upon himself the
whole additional expense.\textsuperscript{68}

\pagenote[67]{Hadrian afterwards made a very equitable
regulation, which divided all treasure-trove between the right of
property and that of discovery. Hist. August. p. 9.}

\pagenote[68]{Philostrat. in Vit. Sophist. l. ii. p. 548.}

The ablest preceptors of Greece and Asia had been invited by
liberal rewards to direct the education of young Herod. Their
pupil soon became a celebrated orator, according to the useless
rhetoric of that age, which, confining itself to the schools,
disdained to visit either the Forum or the Senate.

He was honored with the consulship at Rome: but the greatest part
of his life was spent in a philosophic retirement at Athens, and
his adjacent villas; perpetually surrounded by sophists, who
acknowledged, without reluctance, the superiority of a rich and
generous rival.\textsuperscript{69} The monuments of his genius have perished;
some considerable ruins still preserve the fame of his taste and
munificence: modern travellers have measured the remains of the
stadium which he constructed at Athens. It was six hundred feet
in length, built entirely of white marble, capable of admitting
the whole body of the people, and finished in four years, whilst
Herod was president of the Athenian games. To the memory of his
wife Regilla he dedicated a theatre, scarcely to be paralleled in
the empire: no wood except cedar, very curiously carved, was
employed in any part of the building. The Odeum,\textsuperscript{691} designed by
Pericles for musical performances, and the rehearsal of new
tragedies, had been a trophy of the victory of the arts over
barbaric greatness; as the timbers employed in the construction
consisted chiefly of the masts of the Persian vessels.
Notwithstanding the repairs bestowed on that ancient edifice by a
king of Cappadocia, it was again fallen to decay. Herod restored
its ancient beauty and magnificence. Nor was the liberality of
that illustrious citizen confined to the walls of Athens. The
most splendid ornaments bestowed on the temple of Neptune in the
Isthmus, a theatre at Corinth, a stadium at Delphi, a bath at
Thermopylæ, and an aqueduct at Canusium in Italy, were
insufficient to exhaust his treasures. The people of Epirus,
Thessaly, Eubœa, Bœotia, and Peloponnesus, experienced his
favors; and many inscriptions of the cities of Greece and Asia
gratefully style Herodes Atticus their patron and benefactor.\textsuperscript{70}

\pagenote[69]{Aulus Gellius, in Noct. Attic. i. 2, ix. 2, xviii.
10, xix. 12. Phil ostrat. p. 564.}

\pagenote[691]{The Odeum served for the rehearsal of new comedies
as well as tragedies; they were read or repeated, before
representation, without music or decorations, \&c. No piece could
be represented in the theatre if it had not been previously
approved by judges for this purpose. The king of Cappadocia who
restored the Odeum, which had been burnt by Sylla, was
Araobarzanes. See Martini, Dissertation on the Odeons of the
Ancients, Leipsic. 1767, p. 10—91.—W.}

\pagenote[70]{See Philostrat. l. ii. p. 548, 560. Pausanias, l.
i. and vii. 10. The life of Herodes, in the xxxth volume of the
Memoirs of the Academy of Inscriptions.}

In the commonwealths of Athens and Rome, the modest simplicity of
private houses announced the equal condition of freedom; whilst
the sovereignty of the people was represented in the majestic
edifices designed to the public use;\textsuperscript{71} nor was this republican
spirit totally extinguished by the introduction of wealth and
monarchy. It was in works of national honor and benefit, that the
most virtuous of the emperors affected to display their
magnificence. The golden palace of Nero excited a just
indignation, but the vast extent of ground which had been usurped
by his selfish luxury was more nobly filled under the succeeding
reigns by the Coliseum, the baths of Titus, the Claudian portico,
and the temples dedicated to the goddess of Peace, and to the
genius of Rome.\textsuperscript{72} These monuments of architecture, the property
of the Roman people, were adorned with the most beautiful
productions of Grecian painting and sculpture; and in the temple
of Peace, a very curious library was open to the curiosity of the
learned.\textsuperscript{721} At a small distance from thence was situated the
Forum of Trajan. It was surrounded by a lofty portico, in the
form of a quadrangle, into which four triumphal arches opened a
noble and spacious entrance: in the centre arose a column of
marble, whose height, of one hundred and ten feet, denoted the
elevation of the hill that had been cut away. This column, which
still subsists in its ancient beauty, exhibited an exact
representation of the Dacian victories of its founder. The
veteran soldier contemplated the story of his own campaigns, and
by an easy illusion of national vanity, the peaceful citizen
associated himself to the honors of the triumph. All the other
quarters of the capital, and all the provinces of the empire,
were embellished by the same liberal spirit of public
magnificence, and were filled with amphitheatres, theatres,
temples, porticoes, triumphal arches, baths and aqueducts, all
variously conducive to the health, the devotion, and the
pleasures of the meanest citizen. The last mentioned of those
edifices deserve our peculiar attention. The boldness of the
enterprise, the solidity of the execution, and the uses to which
they were subservient, rank the aqueducts among the noblest
monuments of Roman genius and power. The aqueducts of the capital
claim a just preeminence; but the curious traveller, who, without
the light of history, should examine those of Spoleto, of Metz,
or of Segovia, would very naturally conclude that those
provincial towns had formerly been the residence of some potent
monarch. The solitudes of Asia and Africa were once covered with
flourishing cities, whose populousness, and even whose existence,
was derived from such artificial supplies of a perennial stream
of fresh water.\textsuperscript{73}

\pagenote[71]{It is particularly remarked of Athens by
Dicæarchus, de Statu Græciæ, p. 8, inter Geographos Minores,
edit. Hudson.}

\pagenote[72]{Donatus de Roma Vetere, l. iii. c. 4, 5, 6. Nardini
Roma Antica, l. iii. 11, 12, 13, and a Ms. description of ancient
Rome, by Bernardus Oricellarius, or Rucellai, of which I obtained
a copy from the library of the Canon Ricardi at Florence. Two
celebrated pictures of Timanthes and of Protogenes are mentioned
by Pliny, as in the Temple of Peace; and the Laocoon was found in
the baths of Titus.}

\pagenote[721]{The Emperor Vespasian, who had caused the Temple
of Peace to be built, transported to it the greatest part of the
pictures, statues, and other works of art which had escaped the
civil tumults. It was there that every day the artists and the
learned of Rome assembled; and it is on the site of this temple
that a multitude of antiques have been dug up. See notes of
Reimar on Dion Cassius, lxvi. c. 15, p. 1083.—W.}

\pagenote[73]{Montfaucon l’Antiquite Expliquee, tom. iv. p. 2, l.
i. c. 9. Fabretti has composed a very learned treatise on the
aqueducts of Rome.}

We have computed the inhabitants, and contemplated the public
works, of the Roman empire. The observation of the number and
greatness of its cities will serve to confirm the former, and to
multiply the latter. It may not be unpleasing to collect a few
scattered instances relative to that subject without forgetting,
however, that from the vanity of nations and the poverty of
language, the vague appellation of city has been indifferently
bestowed on Rome and upon Laurentum.

I. \textit{Ancient} Italy is said to have contained eleven hundred and
ninety-seven cities; and for whatsoever æra of antiquity the
expression might be intended,\textsuperscript{74} there is not any reason to
believe the country less populous in the age of the Antonines,
than in that of Romulus. The petty states of Latium were
contained within the metropolis of the empire, by whose superior
influence they had been attracted.\textsuperscript{741} Those parts of Italy which
have so long languished under the lazy tyranny of priests and
viceroys, had been afflicted only by the more tolerable
calamities of war; and the first symptoms of decay which \textit{they}
experienced, were amply compensated by the rapid improvements of
the Cisalpine Gaul. The splendor of Verona may be traced in its
remains: yet Verona was less celebrated than Aquileia or Padua,
Milan or Ravenna. II. The spirit of improvement had passed the
Alps, and been felt even in the woods of Britain, which were
gradually cleared away to open a free space for convenient and
elegant habitations. York was the seat of government; London was
already enriched by commerce; and Bath was celebrated for the
salutary effects of its medicinal waters. Gaul could boast of her
twelve hundred cities;\textsuperscript{75} and though, in the northern parts, many
of them, without excepting Paris itself, were little more than
the rude and imperfect townships of a rising people, the southern
provinces imitated the wealth and elegance of Italy.\textsuperscript{76} Many were
the cities of Gaul, Marseilles, Arles, Nismes, Narbonne,
Thoulouse, Bourdeaux, Autun, Vienna, Lyons, Langres, and Treves,
whose ancient condition might sustain an equal, and perhaps
advantageous comparison with their present state. With regard to
Spain, that country flourished as a province, and has declined as
a kingdom. Exhausted by the abuse of her strength, by America,
and by superstition, her pride might possibly be confounded, if
we required such a list of three hundred and sixty cities, as
Pliny has exhibited under the reign of Vespasian.\textsuperscript{77} III. Three
hundred African cities had once acknowledged the authority of
Carthage,\textsuperscript{78} nor is it likely that their numbers diminished under
the administration of the emperors: Carthage itself rose with new
splendor from its ashes; and that capital, as well as Capua and
Corinth, soon recovered all the advantages which can be separated
from independent sovereignty. IV. The provinces of the East
present the contrast of Roman magnificence with Turkish
barbarism. The ruins of antiquity scattered over uncultivated
fields, and ascribed, by ignorance, to the power of magic,
scarcely afford a shelter to the oppressed peasant or wandering
Arab. Under the reign of the Cæsars, the proper Asia alone
contained five hundred populous cities,\textsuperscript{79} enriched with all the
gifts of nature, and adorned with all the refinements of art.
Eleven cities of Asia had once disputed the honor of dedicating a
temple of Tiberius, and their respective merits were examined by
the senate.\textsuperscript{80} Four of them were immediately rejected as unequal
to the burden; and among these was Laodicea, whose splendor is
still displayed in its ruins.\textsuperscript{81} Laodicea collected a very
considerable revenue from its flocks of sheep, celebrated for the
fineness of their wool, and had received, a little before the
contest, a legacy of above four hundred thousand pounds by the
testament of a generous citizen.\textsuperscript{82} If such was the poverty of
Laodicea, what must have been the wealth of those cities, whose
claim appeared preferable, and particularly of Pergamus, of
Smyrna, and of Ephesus, who so long disputed with each other the
titular primacy of Asia?\textsuperscript{83} The capitals of Syria and Egypt held
a still superior rank in the empire; Antioch and Alexandria
looked down with disdain on a crowd of dependent cities,\textsuperscript{84} and
yielded, with reluctance, to the majesty of Rome itself.

\pagenote[74]{Ælian. Hist. Var. lib. ix. c. 16. He lived in the
time of Alexander Severus. See Fabricius, Biblioth. Græca, l. iv.
c. 21.}

\pagenote[741]{This may in some degree account for the difficulty
started by Livy, as to the incredibly numerous armies raised by
the small states around Rome where, in his time, a scanty stock
of free soldiers among a larger population of Roman slaves broke
the solitude. Vix seminario exiguo militum relicto servitia
Romana ab solitudine vindicant, Liv. vi. vii. Compare Appian Bel
Civ. i. 7.—M. subst. for G.}

\pagenote[75]{Joseph. de Bell. Jud. ii. 16. The number, however,
is mentioned, and should be received with a degree of latitude.
Note: Without doubt no reliance can be placed on this passage of
Josephus. The historian makes Agrippa give advice to the Jews, as
to the power of the Romans; and the speech is full of declamation
which can furnish no conclusions to history. While enumerating
the nations subject to the Romans, he speaks of the Gauls as
submitting to 1200 soldiers, (which is false, as there were eight
legions in Gaul, Tac. iv. 5,) while there are nearly twelve
hundred cities.—G. Josephus (infra) places these eight legions on
the Rhine, as Tacitus does.—M.}

\pagenote[76]{Plin. Hist. Natur. iii. 5.}

\pagenote[77]{Plin. Hist. Natur. iii. 3, 4, iv. 35. The list
seems authentic and accurate; the division of the provinces, and
the different condition of the cities, are minutely
distinguished.}

\pagenote[78]{Strabon. Geograph. l. xvii. p. 1189.}

\pagenote[79]{Joseph. de Bell. Jud. ii. 16. Philostrat. in Vit.
Sophist. l. ii. p. 548, edit. Olear.}

\pagenote[80]{Tacit. Annal. iv. 55. I have taken some pains in
consulting and comparing modern travellers, with regard to the
fate of those eleven cities of Asia. Seven or eight are totally
destroyed: Hypæpe, Tralles, Laodicea, Hium, Halicarnassus,
Miletus, Ephesus, and we may add Sardes. Of the remaining three,
Pergamus is a straggling village of two or three thousand
inhabitants; Magnesia, under the name of Guzelhissar, a town of
some consequence; and Smyrna, a great city, peopled by a hundred
thousand souls. But even at Smyrna, while the Franks have
maintained a commerce, the Turks have ruined the arts.}

\pagenote[81]{See a very exact and pleasing description of the
ruins of Laodicea, in Chandler’s Travels through Asia Minor, p.
225, \&c.}

\pagenote[82]{Strabo, l. xii. p. 866. He had studied at Tralles.}

\pagenote[83]{See a Dissertation of M. de Boze, Mem. de
l’Academie, tom. xviii. Aristides pronounced an oration, which is
still extant, to recommend concord to the rival cities.}

\pagenote[84]{The inhabitants of Egypt, exclusive of Alexandria,
amounted to seven millions and a half, (Joseph. de Bell. Jud. ii.
16.) Under the military government of the Mamelukes, Syria was
supposed to contain sixty thousand villages, (Histoire de Timur
Bec, l. v. c. 20.)}

\section{Part \thesection.}

All these cities were connected with each other, and with the
capital, by the public highways, which, issuing from the Forum of
Rome, traversed Italy, pervaded the provinces, and were
terminated only by the frontiers of the empire. If we carefully
trace the distance from the wall of Antoninus to Rome, and from
thence to Jerusalem, it will be found that the great chain of
communication, from the north-west to the south-east point of the
empire, was drawn out to the length of four thousand and eighty
Roman miles.\textsuperscript{85} The public roads were accurately divided by
mile-stones, and ran in a direct line from one city to another,
with very little respect for the obstacles either of nature or
private property. Mountains were perforated, and bold arches
thrown over the broadest and most rapid streams.\textsuperscript{86} The middle
part of the road was raised into a terrace which commanded the
adjacent country, consisted of several strata of sand, gravel,
and cement, and was paved with large stones, or, in some places
near the capital, with granite.\textsuperscript{87} Such was the solid
construction of the Roman highways, whose firmness has not
entirely yielded to the effort of fifteen centuries. They united
the subjects of the most distant provinces by an easy and
familiar intercourse; but their primary object had been to
facilitate the marches of the legions; nor was any country
considered as completely subdued, till it had been rendered, in
all its parts, pervious to the arms and authority of the
conqueror. The advantage of receiving the earliest intelligence,
and of conveying their orders with celerity, induced the emperors
to establish, throughout their extensive dominions, the regular
institution of posts.\textsuperscript{88} Houses were everywhere erected at the
distance only of five or six miles; each of them was constantly
provided with forty horses, and by the help of these relays, it
was easy to travel a hundred miles in a day along the Roman
roads.\textsuperscript{89} \textsuperscript{891} The use of posts was allowed to those who claimed
it by an Imperial mandate; but though originally intended for the
public service, it was sometimes indulged to the business or
conveniency of private citizens.\textsuperscript{90} Nor was the communication of
the Roman empire less free and open by sea than it was by land.
The provinces surrounded and enclosed the Mediterranean: and
Italy, in the shape of an immense promontory, advanced into the
midst of that great lake. The coasts of Italy are, in general,
destitute of safe harbors; but human industry had corrected the
deficiencies of nature; and the artificial port of Ostia, in
particular, situate at the mouth of the Tyber, and formed by the
emperor Claudius, was a useful monument of Roman greatness.\textsuperscript{91}
From this port, which was only sixteen miles from the capital, a
favorable breeze frequently carried vessels in seven days to the
columns of Hercules, and in nine or ten, to Alexandria in Egypt.\textsuperscript{92}

\pagenote[85]{The following Itinerary may serve to convey some
idea of the direction of the road, and of the distance between
the principal towns. I. From the wall of Antoninus to York, 222
Roman miles. II. London, 227. III. Rhutupiæ or Sandwich, 67. IV.
The navigation to Boulogne, 45. V. Rheims, 174. VI. Lyons, 330.
VII. Milan, 324. VIII. Rome, 426. IX. Brundusium, 360. X. The
navigation to Dyrrachium, 40. XI. Byzantium, 711. XII. Ancyra,
283. XIII. Tarsus, 301. XIV. Antioch, 141. XV. Tyre, 252. XVI.
Jerusalem, 168. In all 4080 Roman, or 3740 English miles. See the
Itineraries published by Wesseling, his annotations; Gale and
Stukeley for Britain, and M. d’Anville for Gaul and Italy.}

\pagenote[86]{Montfaucon, l’Antiquite Expliquee, (tom. 4, p. 2,
l. i. c. 5,) has described the bridges of Narni, Alcantara,
Nismes, \&c.}

\pagenote[87]{Bergier, Histoire des grands Chemins de l’Empire
Romain, l. ii. c. l. l—28.}

\pagenote[88]{Procopius in Hist. Arcana, c. 30. Bergier, Hist.
des grands Chemins, l. iv. Codex Theodosian. l. viii. tit. v.
vol. ii. p. 506—563 with Godefroy’s learned commentary.}

\pagenote[89]{In the time of Theodosius, Cæsarius, a magistrate
of high rank, went post from Antioch to Constantinople. He began
his journey at night, was in Cappadocia (165 miles from Antioch)
the ensuing evening, and arrived at Constantinople the sixth day
about noon. The whole distance was 725 Roman, or 665 English
miles. See Libanius, Orat. xxii., and the Itineria, p. 572—581.
Note: A courier is mentioned in Walpole’s Travels, ii. 335, who
was to travel from Aleppo to Constantinople, more than 700 miles,
in eight days, an unusually short journey.—M.}

\pagenote[891]{Posts for the conveyance of intelligence were
established by Augustus. Suet. Aug. 49. The couriers travelled
with amazing speed. Blair on Roman Slavery, note, p. 261. It is
probable that the posts, from the time of Augustus, were confined
to the public service, and supplied by impressment Nerva, as it
appears from a coin of his reign, made an important change; “he
established posts upon all the public roads of Italy, and made
the service chargeable upon his own exchequer. Hadrian,
perceiving the advantage of this improvement, extended it to all
the provinces of the empire.” Cardwell on Coins, p. 220.—M.}

\pagenote[90]{Pliny, though a favorite and a minister, made an
apology for granting post-horses to his wife on the most urgent
business. Epist. x. 121, 122.}

\pagenote[91]{Bergier, Hist. des grands Chemins, l. iv. c. 49.}

\pagenote[92]{Plin. Hist. Natur. xix. i. [In Proœm.] * Note:
Pliny says Puteoli, which seems to have been the usual landing
place from the East. See the voyages of St. Paul, Acts xxviii.
13, and of Josephus, Vita, c. 3—M.}

Whatever evils either reason or declamation have imputed to
extensive empire, the power of Rome was attended with some
beneficial consequences to mankind; and the same freedom of
intercourse which extended the vices, diffused likewise the
improvements, of social life. In the more remote ages of
antiquity, the world was unequally divided. The East was in the
immemorial possession of arts and luxury; whilst the West was
inhabited by rude and warlike barbarians, who either disdained
agriculture, or to whom it was totally unknown. Under the
protection of an established government, the productions of
happier climates, and the industry of more civilized nations,
were gradually introduced into the western countries of Europe;
and the natives were encouraged, by an open and profitable
commerce, to multiply the former, as well as to improve the
latter. It would be almost impossible to enumerate all the
articles, either of the animal or the vegetable reign, which were
successively imported into Europe from Asia and Egypt:\textsuperscript{93} but it
will not be unworthy of the dignity, and much less of the
utility, of an historical work, slightly to touch on a few of the
principal heads. 1. Almost all the flowers, the herbs, and the
fruits, that grow in our European gardens, are of foreign
extraction, which, in many cases, is betrayed even by their
names: the apple was a native of Italy, and when the Romans had
tasted the richer flavor of the apricot, the peach, the
pomegranate, the citron, and the orange, they contented
themselves with applying to all these new fruits the common
denomination of apple, discriminating them from each other by the
additional epithet of their country. 2. In the time of Homer, the
vine grew wild in the island of Sicily, and most probably in the
adjacent continent; but it was not improved by the skill, nor did
it afford a liquor grateful to the taste, of the savage
inhabitants.\textsuperscript{94} A thousand years afterwards, Italy could boast,
that of the fourscore most generous and celebrated wines, more
than two thirds were produced from her soil.\textsuperscript{95} The blessing was
soon communicated to the Narbonnese province of Gaul; but so
intense was the cold to the north of the Cevennes, that, in the
time of Strabo, it was thought impossible to ripen the grapes in
those parts of Gaul.\textsuperscript{96} This difficulty, however, was gradually
vanquished; and there is some reason to believe, that the
vineyards of Burgundy are as old as the age of the Antonines.\textsuperscript{97}
3. The olive, in the western world, followed the progress of
peace, of which it was considered as the symbol. Two centuries
after the foundation of Rome, both Italy and Africa were
strangers to that useful plant: it was naturalized in those
countries; and at length carried into the heart of Spain and
Gaul. The timid errors of the ancients, that it required a
certain degree of heat, and could only flourish in the
neighborhood of the sea, were insensibly exploded by industry and
experience. 4. The cultivation of flax was transported from Egypt
to Gaul, and enriched the whole country, however it might
impoverish the particular lands on which it was sown.\textsuperscript{99} 5. The
use of artificial grasses became familiar to the farmers both of
Italy and the provinces, particularly the Lucerne, which derived
its name and origin from Media.\pagenote[100] The assured supply of
wholesome and plentiful food for the cattle during winter,
multiplied the number of the docks and herds, which in their turn
contributed to the fertility of the soil. To all these
improvements may be added an assiduous attention to mines and
fisheries, which, by employing a multitude of laborious hands,
serve to increase the pleasures of the rich and the subsistence
of the poor. The elegant treatise of Columella describes the
advanced state of the Spanish husbandry under the reign of
Tiberius; and it may be observed, that those famines, which so
frequently afflicted the infant republic, were seldom or never
experienced by the extensive empire of Rome. The accidental
scarcity, in any single province, was immediately relieved by the
plenty of its more fortunate neighbors.

\pagenote[93]{It is not improbable that the Greeks and Phœnicians
introduced some new arts and productions into the neighborhood of
Marseilles and Gades.}

\pagenote[94]{See Homer, Odyss. l. ix. v. 358.}

\pagenote[95]{Plin. Hist. Natur. l. xiv.}

\pagenote[96]{Strab. Geograph. l. iv. p. 269. The intense cold of
a Gallic winter was almost proverbial among the ancients. * Note:
Strabo only says that the grape does not ripen. Attempts had been
made in the time of Augustus to naturalize the vine in the north
of Gaul; but the cold was too great. Diod. Sic. edit. Rhodom. p.
304.—W. Diodorus (lib. v. 26) gives a curious picture of the
Italian traders bartering, with the savages of Gaul, a cask of
wine for a slave.—M. —It appears from the newly discovered
treatise of Cicero de Republica, that there was a law of the
republic prohibiting the culture of the vine and olive beyond the
Alps, in order to keep up the value of those in Italy. Nos
justissimi homines, qui transalpinas gentes oleam et vitem serere
non sinimus, quo pluris sint nostra oliveta nostræque vineæ. Lib.
iii. 9. The restrictive law of Domitian was veiled under the
decent pretext of encouraging the cultivation of grain. Suet.
Dom. vii. It was repealed by Probus Vopis Strobus, 18.—M.}

\pagenote[97]{In the beginning of the fourth century, the orator
Eumenius (Panegyr. Veter. viii. 6, edit. Delphin.) speaks of the
vines in the territory of Autun, which were decayed through age,
and the first plantation of which was totally unknown. The Pagus
Arebrignus is supposed by M. d’Anville to be the district of
Beaune, celebrated, even at present for one of the first growths
of Burgundy. * Note: This is proved by a passage of Pliny the
Elder, where he speaks of a certain kind of grape (vitis picata.
vinum picatum) which grows naturally to the district of Vienne,
and had recently been transplanted into the country of the
Arverni, (Auvergne,) of the Helvii, (the Vivarias.) and the
Burgundy and Franche Compte. Pliny wrote A.D. 77. Hist. Nat. xiv.
1.— W.}

\pagenote[99]{Plin. Hist. Natur. l. xix.}

\pagenote[100]{See the agreeable Essays on Agriculture by Mr.
Harte, in which he has collected all that the ancients and
moderns have said of Lucerne.}

Agriculture is the foundation of manufactures; since the
productions of nature are the materials of art. Under the Roman
empire, the labor of an industrious and ingenious people was
variously, but incessantly, employed in the service of the rich.
In their dress, their table, their houses, and their furniture,
the favorites of fortune united every refinement of conveniency,
of elegance, and of splendor, whatever could soothe their pride
or gratify their sensuality. Such refinements, under the odious
name of luxury, have been severely arraigned by the moralists of
every age; and it might perhaps be more conducive to the virtue,
as well as happiness, of mankind, if all possessed the
necessaries, and none the superfluities, of life. But in the
present imperfect condition of society, luxury, though it may
proceed from vice or folly, seems to be the only means that can
correct the unequal distribution of property. The diligent
mechanic, and the skilful artist, who have obtained no share in
the division of the earth, receive a voluntary tax from the
possessors of land; and the latter are prompted, by a sense of
interest, to improve those estates, with whose produce they may
purchase additional pleasures. This operation, the particular
effects of which are felt in every society, acted with much more
diffusive energy in the Roman world. The provinces would soon
have been exhausted of their wealth, if the manufactures and
commerce of luxury had not insensibly restored to the industrious
subjects the sums which were exacted from them by the arms and
authority of Rome. As long as the circulation was confined within
the bounds of the empire, it impressed the political machine with
a new degree of activity, and its consequences, sometimes
beneficial, could never become pernicious.

But it is no easy task to confine luxury within the limits of an
empire. The most remote countries of the ancient world were
ransacked to supply the pomp and delicacy of Rome. The forests of
Scythia afforded some valuable furs. Amber was brought over land
from the shores of the Baltic to the Danube; and the barbarians
were astonished at the price which they received in exchange for
so useless a commodity.\textsuperscript{101} There was a considerable demand for
Babylonian carpets, and other manufactures of the East; but the
most important and unpopular branch of foreign trade was carried
on with Arabia and India. Every year, about the time of the
summer solstice, a fleet of a hundred and twenty vessels sailed
from Myos-hormos, a port of Egypt, on the Red Sea. By the
periodical assistance of the monsoons, they traversed the ocean
in about forty days. The coast of Malabar, or the island of
Ceylon,\textsuperscript{102} was the usual term of their navigation, and it was in
those markets that the merchants from the more remote countries
of Asia expected their arrival. The return of the fleet of Egypt
was fixed to the months of December or January; and as soon as
their rich cargo had been transported on the backs of camels,
from the Red Sea to the Nile, and had descended that river as far
as Alexandria, it was poured, without delay, into the capital of
the empire.\textsuperscript{103} The objects of oriental traffic were splendid and
trifling; silk, a pound of which was esteemed not inferior in
value to a pound of gold;\textsuperscript{104} precious stones, among which the
pearl claimed the first rank after the diamond;\textsuperscript{105} and a variety
of aromatics, that were consumed in religious worship and the
pomp of funerals. The labor and risk of the voyage was rewarded
with almost incredible profit; but the profit was made upon Roman
subjects, and a few individuals were enriched at the expense of
the public. As the natives of Arabia and India were contented
with the productions and manufactures of their own country,
silver, on the side of the Romans, was the principal, if not the
only\textsuperscript{1051} instrument of commerce. It was a complaint worthy of
the gravity of the senate, that, in the purchase of female
ornaments, the wealth of the state was irrecoverably given away
to foreign and hostile nations.\textsuperscript{106} The annual loss is computed,
by a writer of an inquisitive but censorious temper, at upwards
of eight hundred thousand pounds sterling.\textsuperscript{107} Such was the style
of discontent, brooding over the dark prospect of approaching
poverty. And yet, if we compare the proportion between gold and
silver, as it stood in the time of Pliny, and as it was fixed in
the reign of Constantine, we shall discover within that period a
very considerable increase.\textsuperscript{108} There is not the least reason to
suppose that gold was become more scarce; it is therefore evident
that silver was grown more common; that whatever might be the
amount of the Indian and Arabian exports, they were far from
exhausting the wealth of the Roman world; and that the produce of
the mines abundantly supplied the demands of commerce.

\pagenote[101]{Tacit. Germania, c. 45. Plin. Hist. Nat. xxxvii.
13. The latter observed, with some humor, that even fashion had
not yet found out the use of amber. Nero sent a Roman knight to
purchase great quantities on the spot where it was produced, the
coast of modern Prussia.}

\pagenote[102]{Called Taprobana by the Romans, and Serindib by
the Arabs. It was discovered under the reign of Claudius, and
gradually became the principal mart of the East.}

\pagenote[103]{Plin. Hist. Natur. l. vi. Strabo, l. xvii.}

\pagenote[104]{Hist. August. p. 224. A silk garment was
considered as an ornament to a woman, but as a disgrace to a
man.}

\pagenote[105]{The two great pearl fisheries were the same as at
present, Ormuz and Cape Comorin. As well as we can compare
ancient with modern geography, Rome was supplied with diamonds
from the mine of Jumelpur, in Bengal, which is described in the
Voyages de Tavernier, tom. ii. p. 281.}

\pagenote[1051]{Certainly not the only one. The Indians were not
so contented with regard to foreign productions. Arrian has a
long list of European wares, which they received in exchange for
their own; Italian and other wines, brass, tin, lead, coral,
chrysolith, storax, glass, dresses of one or many colors, zones,
\&c. See Periplus Maris Erythræi in Hudson, Geogr. Min. i. p.
27.—W. The German translator observes that Gibbon has confined
the use of aromatics to religious worship and funerals. His error
seems the omission of other spices, of which the Romans must have
consumed great quantities in their cookery. Wenck, however,
admits that silver was the chief article of exchange.—M. In 1787,
a peasant (near Nellore in the Carnatic) struck, in digging, on
the remains of a Hindu temple; he found, also, a pot which
contained Roman coins and medals of the second century, mostly
Trajans, Adrians, and Faustinas, all of gold, many of them fresh
and beautiful, others defaced or perforated, as if they had been
worn as ornaments. (Asiatic Researches, ii. 19.)—M.}

\pagenote[106]{Tacit. Annal. iii. 53. In a speech of Tiberius.}

\pagenote[107]{Plin. Hist. Natur. xii. 18. In another place he
computes half that sum; Quingenties H. S. for India exclusive of
Arabia.}

\pagenote[108]{The proportion, which was 1 to 10, and 12 1/2,
rose to 14 2/5, the legal regulation of Constantine. See
Arbuthnot’s Tables of ancient Coins, c. 5.}

Notwithstanding the propensity of mankind to exalt the past, and
to depreciate the present, the tranquil and prosperous state of
the empire was warmly felt, and honestly confessed, by the
provincials as well as Romans. “They acknowledged that the true
principles of social life, laws, agriculture, and science, which
had been first invented by the wisdom of Athens, were now firmly
established by the power of Rome, under whose auspicious
influence the fiercest barbarians were united by an equal
government and common language. They affirm, that with the
improvement of arts, the human species were visibly multiplied.
They celebrate the increasing splendor of the cities, the
beautiful face of the country, cultivated and adorned like an
immense garden; and the long festival of peace which was enjoyed
by so many nations, forgetful of the ancient animosities, and
delivered from the apprehension of future danger.”\textsuperscript{109} Whatever
suspicions may be suggested by the air of rhetoric and
declamation, which seems to prevail in these passages, the
substance of them is perfectly agreeable to historic truth.

\pagenote[109]{Among many other passages, see Pliny, (Hist.
Natur. iii. 5.) Aristides, (de Urbe Roma,) and Tertullian, (de
Anima, c. 30.)}

It was scarcely possible that the eyes of contemporaries should
discover in the public felicity the latent causes of decay and
corruption. This long peace, and the uniform government of the
Romans, introduced a slow and secret poison into the vitals of
the empire. The minds of men were gradually reduced to the same
level, the fire of genius was extinguished, and even the military
spirit evaporated. The natives of Europe were brave and robust.
Spain, Gaul, Britain, and Illyricum supplied the legions with
excellent soldiers, and constituted the real strength of the
monarchy. Their personal valor remained, but they no longer
possessed that public courage which is nourished by the love of
independence, the sense of national honor, the presence of
danger, and the habit of command. They received laws and
governors from the will of their sovereign, and trusted for their
defence to a mercenary army. The posterity of their boldest
leaders was contented with the rank of citizens and subjects. The
most aspiring spirits resorted to the court or standard of the
emperors; and the deserted provinces, deprived of political
strength or union, insensibly sunk into the languid indifference
of private life.

The love of letters, almost inseparable from peace and
refinement, was fashionable among the subjects of Hadrian and the
Antonines, who were themselves men of learning and curiosity. It
was diffused over the whole extent of their empire; the most
northern tribes of Britons had acquired a taste for rhetoric;
Homer as well as Virgil were transcribed and studied on the banks
of the Rhine and Danube; and the most liberal rewards sought out
the faintest glimmerings of literary merit.\textsuperscript{110} The sciences of
physic and astronomy were successfully cultivated by the Greeks;
the observations of Ptolemy and the writings of Galen are studied
by those who have improved their discoveries and corrected their
errors; but if we except the inimitable Lucian, this age of
indolence passed away without having produced a single writer of
original genius, or who excelled in the arts of elegant
composition.\textsuperscript{1101} The authority of Plato and Aristotle, of Zeno
and Epicurus, still reigned in the schools; and their systems,
transmitted with blind deference from one generation of disciples
to another, precluded every generous attempt to exercise the
powers, or enlarge the limits, of the human mind. The beauties of
the poets and orators, instead of kindling a fire like their own,
inspired only cold and servile imitations: or if any ventured to
deviate from those models, they deviated at the same time from
good sense and propriety. On the revival of letters, the youthful
vigor of the imagination, after a long repose, national
emulation, a new religion, new languages, and a new world, called
forth the genius of Europe. But the provincials of Rome, trained
by a uniform artificial foreign education, were engaged in a very
unequal competition with those bold ancients, who, by expressing
their genuine feelings in their native tongue, had already
occupied every place of honor. The name of Poet was almost
forgotten; that of Orator was usurped by the sophists. A cloud of
critics, of compilers, of commentators, darkened the face of
learning, and the decline of genius was soon followed by the
corruption of taste.

\pagenote[110]{Herodes Atticus gave the sophist Polemo above
eight thousand pounds for three declamations. See Philostrat. l.
i. p. 538. The Antonines founded a school at Athens, in which
professors of grammar, rhetoric, politics, and the four great
sects of philosophy were maintained at the public expense for the
instruction of youth. The salary of a philosopher was ten
thousand drachmæ, between three and four hundred pounds a year.
Similar establishments were formed in the other great cities of
the empire. See Lucian in Eunuch. tom. ii. p. 352, edit. Reitz.
Philostrat. l. ii. p. 566. Hist. August. p. 21. Dion Cassius, l.
lxxi. p. 1195. Juvenal himself, in a morose satire, which in
every line betrays his own disappointment and envy, is obliged,
however, to say,—“—O Juvenes, circumspicit et stimulat vos.
Materiamque sibi Ducis indulgentia quærit.”—Satir. vii. 20. Note:
Vespasian first gave a salary to professors: he assigned to each
professor of rhetoric, Greek and Roman, centena sestertia.
(Sueton. in Vesp. 18). Hadrian and the Antonines, though still
liberal, were less profuse.—G. from W. Suetonius wrote annua
centena L. 807, 5, 10.—M.}

\pagenote[1101]{This judgment is rather severe: besides the
physicians, astronomers, and grammarians, among whom there were
some very distinguished men, there were still, under Hadrian,
Suetonius, Florus, Plutarch; under the Antonines, Arrian,
Pausanias, Appian, Marcus Aurelius himself, Sextus Empiricus, \&c.
Jurisprudence gained much by the labors of Salvius Julianus,
Julius Celsus, Sex. Pomponius, Caius, and others.—G. from W. Yet
where, among these, is the writer of original genius, unless,
perhaps Plutarch? or even of a style really elegant?— M.}

The sublime Longinus, who, in somewhat a later period, and in the
court of a Syrian queen, preserved the spirit of ancient Athens,
observes and laments this degeneracy of his contemporaries, which
debased their sentiments, enervated their courage, and depressed
their talents. “In the same manner,” says he, “as some children
always remain pygmies, whose infant limbs have been too closely
confined, thus our tender minds, fettered by the prejudices and
habits of a just servitude, are unable to expand themselves, or
to attain that well-proportioned greatness which we admire in the
ancients; who, living under a popular government, wrote with the
same freedom as they acted.”\textsuperscript{111} This diminutive stature of
mankind, if we pursue the metaphor, was daily sinking below the
old standard, and the Roman world was indeed peopled by a race of
pygmies; when the fierce giants of the north broke in, and mended
the puny breed. They restored a manly spirit of freedom; and
after the revolution of ten centuries, freedom became the happy
parent of taste and science.

\pagenote[111]{Longin. de Sublim. c. 44, p. 229, edit. Toll.
Here, too, we may say of Longinus, “his own example strengthens
all his laws.” Instead of proposing his sentiments with a manly
boldness, he insinuates them with the most guarded caution; puts
them into the mouth of a friend, and as far as we can collect
from a corrupted text, makes a show of refuting them himself.}

