\section{Part \thesection.}
\thispagestyle{simple}

The death of Claudius had revived the fainting spirit of the
Goths. The troops which guarded the passes of Mount Hæmus, and
the banks of the Danube, had been drawn away by the apprehension
of a civil war; and it seems probable that the remaining body of
the Gothic and Vandalic tribes embraced the favorable
opportunity, abandoned their settlements of the Ukraine,
traversed the rivers, and swelled with new multitudes the
destroying host of their countrymen. Their united numbers were at
length encountered by Aurelian, and the bloody and doubtful
conflict ended only with the approach of night.\footnotemark[20] Exhausted by
so many calamities, which they had mutually endured and inflicted
during a twenty years’ war, the Goths and the Romans consented to
a lasting and beneficial treaty. It was earnestly solicited by
the barbarians, and cheerfully ratified by the legions, to whose
suffrage the prudence of Aurelian referred the decision of that
important question. The Gothic nation engaged to supply the
armies of Rome with a body of two thousand auxiliaries,
consisting entirely of cavalry, and stipulated in return an
undisturbed retreat, with a regular market as far as the Danube,
provided by the emperor’s care, but at their own expense. The
treaty was observed with such religious fidelity, that when a
party of five hundred men straggled from the camp in quest of
plunder, the king or general of the barbarians commanded that the
guilty leader should be apprehended and shot to death with darts,
as a victim devoted to the sanctity of their engagements.\footnotemark[201] It
is, however, not unlikely, that the precaution of Aurelian, who
had exacted as hostages the sons and daughters of the Gothic
chiefs, contributed something to this pacific temper. The youths
he trained in the exercise of arms, and near his own person: to
the damsels he gave a liberal and Roman education, and by
bestowing them in marriage on some of his principal officers,
gradually introduced between the two nations the closest and most
endearing connections.\footnotemark[21]

\footnotetext[20]{Zosimus, l. i. p. 45.}

\footnotetext[201]{The five hundred stragglers were all slain.—M.}

\footnotetext[21]{Dexipphus (ap. Excerpta Legat. p. 12) relates the
whole transaction under the name of Vandals. Aurelian married one
of the Gothic ladies to his general Bonosus, who was able to
drink with the Goths and discover their secrets. Hist. August. p.
247.}

But the most important condition of peace was understood rather
than expressed in the treaty. Aurelian withdrew the Roman forces
from Dacia, and tacitly relinquished that great province to the
Goths and Vandals.\footnotemark[22] His manly judgment convinced him of the
solid advantages, and taught him to despise the seeming disgrace,
of thus contracting the frontiers of the monarchy. The Dacian
subjects, removed from those distant possessions which they were
unable to cultivate or defend, added strength and populousness to
the southern side of the Danube. A fertile territory, which the
repetition of barbarous inroads had changed into a desert, was
yielded to their industry, and a new province of Dacia still
preserved the memory of Trajan’s conquests. The old country of
that name detained, however, a considerable number of its
inhabitants, who dreaded exile more than a Gothic master.\footnotemark[23]
These degenerate Romans continued to serve the empire, whose
allegiance they had renounced, by introducing among their
conquerors the first notions of agriculture, the useful arts, and
the conveniences of civilized life. An intercourse of commerce
and language was gradually established between the opposite banks
of the Danube; and after Dacia became an independent state, it
often proved the firmest barrier of the empire against the
invasions of the savages of the North. A sense of interest
attached these more settled barbarians to the alliance of Rome,
and a permanent interest very frequently ripens into sincere and
useful friendship. This various colony, which filled the ancient
province, and was insensibly blended into one great people, still
acknowledged the superior renown and authority of the Gothic
tribe, and claimed the fancied honor of a Scandinavian origin. At
the same time, the lucky though accidental resemblance of the
name of Getæ,\footnotemark[231] infused among the credulous Goths a vain
persuasion, that in a remote age, their own ancestors, already
seated in the Dacian provinces, had received the instructions of
Zamolxis, and checked the victorious arms of Sesostris and
Darius.\footnotemark[24]

\footnotetext[22]{Hist. August. p. 222. Eutrop. ix. 15. Sextus Rufus,
c. 9. de Mortibus Persecutorum, c. 9.}

\footnotetext[23]{The Walachians still preserve many traces of the
Latin language and have boasted, in every age, of their Roman
descent. They are surrounded by, but not mixed with, the
barbarians. See a Memoir of M. d’Anville on ancient Dacia, in the
Academy of Inscriptions, tom. xxx.}

\footnotetext[231]{The connection between the Getæ and the Goths is
still in my opinion incorrectly maintained by some learned
writers—M.}

\footnotetext[24]{See the first chapter of Jornandes. The Vandals,
however, (c. 22,) maintained a short independence between the
Rivers Marisia and Crissia, (Maros and Keres,) which fell into
the Teiss.}

While the vigorous and moderate conduct of Aurelian restored the
Illyrian frontier, the nation of the Alemanni\footnotemark[25] violated the
conditions of peace, which either Gallienus had purchased, or
Claudius had imposed, and, inflamed by their impatient youth,
suddenly flew to arms. Forty thousand horse appeared in the
field,\footnotemark[26] and the numbers of the infantry doubled those of the
cavalry.\footnotemark[27] The first objects of their avarice were a few cities
of the Rhætian frontier; but their hopes soon rising with
success, the rapid march of the Alemanni traced a line of
devastation from the Danube to the Po.\footnotemark[28]

\footnotetext[25]{Dexippus, p. 7—12. Zosimus, l. i. p. 43. Vopiscus
in Aurelian in Hist. August. However these historians differ in
names, (Alemanni Juthungi, and Marcomanni,) it is evident that
they mean the same people, and the same war; but it requires some
care to conciliate and explain them.}

\footnotetext[26]{Cantoclarus, with his usual accuracy, chooses to
translate three hundred thousand: his version is equally
repugnant to sense and to grammar.}

\footnotetext[27]{We may remark, as an instance of bad taste, that
Dexippus applies to the light infantry of the Alemanni the
technical terms proper only to the Grecian phalanx.}

\footnotetext[28]{In Dexippus, we at present read Rhodanus: M. de
Valois very judiciously alters the word to Eridanus.}

The emperor was almost at the same time informed of the
irruption, and of the retreat, of the barbarians. Collecting an
active body of troops, he marched with silence and celerity along
the skirts of the Hercynian forest; and the Alemanni, laden with
the spoils of Italy, arrived at the Danube, without suspecting,
that on the opposite bank, and in an advantageous post, a Roman
army lay concealed and prepared to intercept their return.
Aurelian indulged the fatal security of the barbarians, and
permitted about half their forces to pass the river without
disturbance and without precaution. Their situation and
astonishment gave him an easy victory; his skilful conduct
improved the advantage. Disposing the legions in a semicircular
form, he advanced the two horns of the crescent across the
Danube, and wheeling them on a sudden towards the centre,
enclosed the rear of the German host. The dismayed barbarians, on
whatsoever side they cast their eyes, beheld, with despair, a
wasted country, a deep and rapid stream, a victorious and
implacable enemy.

Reduced to this distressed condition, the Alemanni no longer
disdained to sue for peace. Aurelian received their ambassadors
at the head of his camp, and with every circumstance of martial
pomp that could display the greatness and discipline of Rome. The
legions stood to their arms in well-ordered ranks and awful
silence. The principal commanders, distinguished by the ensigns
of their rank, appeared on horseback on either side of the
Imperial throne. Behind the throne the consecrated images of the
emperor, and his predecessors,\footnotemark[29] the golden eagles, and the
various titles of the legions, engraved in letters of gold, were
exalted in the air on lofty pikes covered with silver. When
Aurelian assumed his seat, his manly grace and majestic figure\footnotemark[30]
taught the barbarians to revere the person as well as the purple
of their conqueror. The ambassadors fell prostrate on the ground
in silence. They were commanded to rise, and permitted to speak.
By the assistance of interpreters they extenuated their perfidy,
magnified their exploits, expatiated on the vicissitudes of
fortune and the advantages of peace, and, with an ill-timed
confidence, demanded a large subsidy, as the price of the
alliance which they offered to the Romans. The answer of the
emperor was stern and imperious. He treated their offer with
contempt, and their demand with indignation, reproached the
barbarians, that they were as ignorant of the arts of war as of
the laws of peace, and finally dismissed them with the choice
only of submitting to this unconditional mercy, or awaiting the
utmost severity of his resentment.\footnotemark[31] Aurelian had resigned a
distant province to the Goths; but it was dangerous to trust or
to pardon these perfidious barbarians, whose formidable power
kept Italy itself in perpetual alarms.

\footnotetext[29]{The emperor Claudius was certainly of the number;
but we are ignorant how far this mark of respect was extended; if
to Cæsar and Augustus, it must have produced a very awful
spectacle; a long line of the masters of the world.}

\footnotetext[30]{Vopiscus in Hist. August. p. 210.}

\footnotetext[31]{Dexippus gives them a subtle and prolix oration,
worthy of a Grecian sophist.}

Immediately after this conference, it should seem that some
unexpected emergency required the emperor’s presence in Pannonia.

He devolved on his lieutenants the care of finishing the
destruction of the Alemanni, either by the sword, or by the surer
operation of famine. But an active despair has often triumphed
over the indolent assurance of success. The barbarians, finding
it impossible to traverse the Danube and the Roman camp, broke
through the posts in their rear, which were more feebly or less
carefully guarded; and with incredible diligence, but by a
different road, returned towards the mountains of Italy.\footnotemark[32]
Aurelian, who considered the war as totally extinguished,
received the mortifying intelligence of the escape of the
Alemanni, and of the ravage which they already committed in the
territory of Milan. The legions were commanded to follow, with as
much expedition as those heavy bodies were capable of exerting,
the rapid flight of an enemy whose infantry and cavalry moved
with almost equal swiftness. A few days afterwards, the emperor
himself marched to the relief of Italy, at the head of a chosen
body of auxiliaries, (among whom were the hostages and cavalry of
the Vandals,) and of all the Prætorian guards who had served in
the wars on the Danube.\footnotemark[33]

\footnotetext[32]{Hist. August. p. 215.}

\footnotetext[33]{Dexippus, p. 12.}

As the light troops of the Alemanni had spread themselves from
the Alps to the Apennine, the incessant vigilance of Aurelian and
his officers was exercised in the discovery, the attack, and the
pursuit of the numerous detachments. Notwithstanding this
desultory war, three considerable battles are mentioned, in which
the principal force of both armies was obstinately engaged.\footnotemark[34]
The success was various. In the first, fought near Placentia, the
Romans received so severe a blow, that, according to the
expression of a writer extremely partial to Aurelian, the
immediate dissolution of the empire was apprehended.\footnotemark[35] The
crafty barbarians, who had lined the woods, suddenly attacked the
legions in the dusk of the evening, and, it is most probable,
after the fatigue and disorder of a long march.

The fury of their charge was irresistible; but, at length, after
a dreadful slaughter, the patient firmness of the emperor rallied
his troops, and restored, in some degree, the honor of his arms.
The second battle was fought near Fano in Umbria; on the spot
which, five hundred years before, had been fatal to the brother
of Hannibal.\footnotemark[36] Thus far the successful Germans had advanced
along the Æmilian and Flaminian way, with a design of sacking the
defenceless mistress of the world. But Aurelian, who, watchful
for the safety of Rome, still hung on their rear, found in this
place the decisive moment of giving them a total and
irretrievable defeat.\footnotemark[37] The flying remnant of their host was
exterminated in a third and last battle near Pavia; and Italy was
delivered from the inroads of the Alemanni.

\footnotetext[34]{Victor Junior in Aurelian.}

\footnotetext[35]{Vopiscus in Hist. August. p. 216.}

\footnotetext[36]{The little river, or rather torrent, of, Metaurus,
near Fano, has been immortalized, by finding such an historian as
Livy, and such a poet as Horace.}

\footnotetext[37]{It is recorded by an inscription found at Pesaro.
See Gruter cclxxvi. 3.}

Fear has been the original parent of superstition, and every new
calamity urges trembling mortals to deprecate the wrath of their
invisible enemies. Though the best hope of the republic was in
the valor and conduct of Aurelian, yet such was the public
consternation, when the barbarians were hourly expected at the
gates of Rome, that, by a decree of the senate the Sibylline
books were consulted. Even the emperor himself, from a motive
either of religion or of policy, recommended this salutary
measure, chided the tardiness of the senate,\footnotemark[38] and offered to
supply whatever expense, whatever animals, whatever captives of
any nation, the gods should require. Notwithstanding this liberal
offer, it does not appear, that any human victims expiated with
their blood the sins of the Roman people. The Sibylline books
enjoined ceremonies of a more harmless nature, processions of
priests in white robes, attended by a chorus of youths and
virgins; lustrations of the city and adjacent country; and
sacrifices, whose powerful influence disabled the barbarians from
passing the mystic ground on which they had been celebrated.
However puerile in themselves, these superstitious arts were
subservient to the success of the war; and if, in the decisive
battle of Fano, the Alemanni fancied they saw an army of spectres
combating on the side of Aurelian, he received a real and
effectual aid from this imaginary reënforcement.\footnotemark[39]

\footnotetext[38]{One should imagine, he said, that you were
assembled in a Christian church, not in the temple of all the
gods.}

\footnotetext[39]{Vopiscus, in Hist. August. p. 215, 216, gives a
long account of these ceremonies from the Registers of the
senate.}

But whatever confidence might be placed in ideal ramparts, the
experience of the past, and the dread of the future, induced the
Romans to construct fortifications of a grosser and more
substantial kind. The seven hills of Rome had been surrounded by
the successors of Romulus with an ancient wall of more than
thirteen miles.\footnotemark[40] The vast enclosure may seem disproportioned to
the strength and numbers of the infant-state. But it was
necessary to secure an ample extent of pasture and arable land
against the frequent and sudden incursions of the tribes of
Latium, the perpetual enemies of the republic. With the progress
of Roman greatness, the city and its inhabitants gradually
increased, filled up the vacant space, pierced through the
useless walls, covered the field of Mars, and, on every side,
followed the public highways in long and beautiful suburbs.\footnotemark[41]
The extent of the new walls, erected by Aurelian, and finished in
the reign of Probus, was magnified by popular estimation to near
fifty,\footnotemark[42] but is reduced by accurate measurement to about
twenty-one miles. \footnotemark[43] It was a great but a melancholy labor, since
the defence of the capital betrayed the decline of monarchy. The
Romans of a more prosperous age, who trusted to the arms of the
legions the safety of the frontier camps,\footnotemark[44] were very far from
entertaining a suspicion that it would ever become necessary to
fortify the seat of empire against the inroads of the barbarians.\footnotemark[45]

\footnotetext[40]{Plin. Hist. Natur. iii. 5. To confirm our idea, we
may observe, that for a long time Mount Cælius was a grove of
oaks, and Mount Viminal was overrun with osiers; that, in the
fourth century, the Aventine was a vacant and solitary
retirement; that, till the time of Augustus, the Esquiline was an
unwholesome burying-ground; and that the numerous inequalities,
remarked by the ancients in the Quirinal, sufficiently prove that
it was not covered with buildings. Of the seven hills, the
Capitoline and Palatine only, with the adjacent valleys, were the
primitive habitations of the Roman people. But this subject would
require a dissertation.}

\footnotetext[41]{Exspatiantia tecta multas addidere urbes, is the
expression of Pliny.}

\footnotetext[42]{Hist. August. p. 222. Both Lipsius and Isaac
Vossius have eagerly embraced this measure.}

\footnotetext[43]{See Nardini, Roman Antica, l. i. c. 8. * Note: But
compare Gibbon, ch. xli. note 77.—M.}

\footnotetext[44]{Tacit. Hist. iv. 23.}

\footnotetext[45]{For Aurelian’s walls, see Vopiscus in Hist. August.
p. 216, 222. Zosimus, l. i. p. 43. Eutropius, ix. 15. Aurel.
Victor in Aurelian Victor Junior in Aurelian. Euseb. Hieronym. et
Idatius in Chronic}

The victory of Claudius over the Goths, and the success of
Aurelian against the Alemanni, had already restored to the arms
of Rome their ancient superiority over the barbarous nations of
the North. To chastise domestic tyrants, and to reunite the
dismembered parts of the empire, was a task reserved for the
second of those warlike emperors. Though he was acknowledged by
the senate and people, the frontiers of Italy, Africa, Illyricum,
and Thrace, confined the limits of his reign. Gaul, Spain, and
Britain, Egypt, Syria, and Asia Minor, were still possessed by
two rebels, who alone, out of so numerous a list, had hitherto
escaped the dangers of their situation; and to complete the
ignominy of Rome, these rival thrones had been usurped by women.

A rapid succession of monarchs had arisen and fallen in the
provinces of Gaul. The rigid virtues of Posthumus served only to
hasten his destruction. After suppressing a competitor, who had
assumed the purple at Mentz, he refused to gratify his troops
with the plunder of the rebellious city; and in the seventh year
of his reign, became the victim of their disappointed avarice.\footnotemark[46]
The death of Victorinus, his friend and associate, was occasioned
by a less worthy cause. The shining accomplishments\footnotemark[47] of that
prince were stained by a licentious passion, which he indulged in
acts of violence, with too little regard to the laws of society,
or even to those of love.\footnotemark[48] He was slain at Cologne, by a
conspiracy of jealous husbands, whose revenge would have appeared
more justifiable, had they spared the innocence of his son. After
the murder of so many valiant princes, it is somewhat remarkable,
that a female for a long time controlled the fierce legions of
Gaul, and still more singular, that she was the mother of the
unfortunate Victorinus. The arts and treasures of Victoria
enabled her successively to place Marius and Tetricus on the
throne, and to reign with a manly vigor under the name of those
dependent emperors. Money of copper, of silver, and of gold, was
coined in her name; she assumed the titles of Augusta and Mother
of the Camps: her power ended only with her life; but her life
was perhaps shortened by the ingratitude of Tetricus.\footnotemark[49]

\footnotetext[46]{His competitor was Lollianus, or Ælianus, if,
indeed, these names mean the same person. See Tillemont, tom.
iii. p. 1177. Note: The medals which bear the name of Lollianus
are considered forgeries except one in the museum of the Prince
of Waldeck there are many extent bearing the name of Lælianus,
which appears to have been that of the competitor of Posthumus.
Eckhel. Doct. Num. t. vi. 149—G.}

\footnotetext[47]{The character of this prince by Julius Aterianus
(ap. Hist. August. p. 187) is worth transcribing, as it seems
fair and impartial Victorino qui Post Junium Posthumium Gallias
rexit neminem existemo præferendum; non in virtute Trajanum; non
Antoninum in clementia; non in gravitate Nervam; non in
gubernando ærario Vespasianum; non in Censura totius vitæ ac
severitate militari Pertinacem vel Severum. Sed omnia hæc libido
et cupiditas voluptatis mulierriæ sic perdidit, ut nemo audeat
virtutes ejus in literas mittere quem constat omnium judicio
meruisse puniri.}

\footnotetext[48]{He ravished the wife of Attitianus, an actuary, or
army agent, Hist. August. p. 186. Aurel. Victor in Aurelian.}

\footnotetext[49]{Pollio assigns her an article among the thirty
tyrants. Hist. August. p. 200.}

When, at the instigation of his ambitious patroness, Tetricus
assumed the ensigns of royalty, he was governor of the peaceful
province of Aquitaine, an employment suited to his character and
education. He reigned four or five years over Gaul, Spain, and
Britain, the slave and sovereign of a licentious army, whom he
dreaded, and by whom he was despised. The valor and fortune of
Aurelian at length opened the prospect of a deliverance. He
ventured to disclose his melancholy situation, and conjured the
emperor to hasten to the relief of his unhappy rival. Had this
secret correspondence reached the ears of the soldiers, it would
most probably have cost Tetricus his life; nor could he resign
the sceptre of the West without committing an act of treason
against himself. He affected the appearances of a civil war, led
his forces into the field, against Aurelian, posted them in the
most disadvantageous manner, betrayed his own counsels to his
enemy, and with a few chosen friends deserted in the beginning of
the action. The rebel legions, though disordered and dismayed by
the unexpected treachery of their chief, defended themselves with
desperate valor, till they were cut in pieces almost to a man, in
this bloody and memorable battle, which was fought near Chalons
in Champagne.\footnotemark[50] The retreat of the irregular auxiliaries, Franks
and Batavians,\footnotemark[51] whom the conqueror soon compelled or persuaded
to repass the Rhine, restored the general tranquillity, and the
power of Aurelian was acknowledged from the wall of Antoninus to
the columns of Hercules.

\footnotetext[50]{Pollio in Hist. August. p. 196. Vopiscus in Hist.
August. p. 220. The two Victors, in the lives of Gallienus and
Aurelian. Eutrop. ix. 13. Euseb. in Chron. Of all these writers,
only the two last (but with strong probability) place the fall of
Tetricus before that of Zenobia. M. de Boze (in the Academy of
Inscriptions, tom. xxx.) does not wish, and Tillemont (tom. iii.
p. 1189) does not dare to follow them. I have been fairer than
the one, and bolder than the other.}

\footnotetext[51]{Victor Junior in Aurelian. Eumenius mentions
Batavicœ; some critics, without any reason, would fain alter the
word to Bagandicœ.}

As early as the reign of Claudius, the city
of Autun, alone and unassisted, had ventured to declare against
the legions of Gaul. After a siege of seven months, they stormed
and plundered that unfortunate city, already wasted by famine.\footnotemark[52]
Lyons, on the contrary, had resisted with obstinate disaffection
the arms of Aurelian. We read of the punishment of Lyons,\footnotemark[53] but
there is not any mention of the rewards of Autun. Such, indeed,
is the policy of civil war: severely to remember injuries, and to
forget the most important services. Revenge is profitable,
gratitude is expensive.

\footnotetext[52]{Eumen. in Vet. Panegyr. iv. 8.}

\footnotetext[53]{Vopiscus in Hist. August. p. 246. Autun was not
restored till the reign of Diocletian. See Eumenius de
restaurandis scholis.}

Aurelian had no sooner secured the person and provinces of
Tetricus, than he turned his arms against Zenobia, the celebrated
queen of Palmyra and the East. Modern Europe has produced several
illustrious women who have sustained with glory the weight of
empire; nor is our own age destitute of such distinguished
characters. But if we except the doubtful achievements of
Semiramis, Zenobia is perhaps the only female whose superior
genius broke through the servile indolence imposed on her sex by
the climate and manners of Asia.\footnotemark[54] She claimed her descent from
the Macedonian kings of Egypt,\footnotemark[541] equalled in beauty her
ancestor Cleopatra, and far surpassed that princess in chastity\footnotemark[55]
and valor. Zenobia was esteemed the most lovely as well as the
most heroic of her sex. She was of a dark complexion (for in
speaking of a lady these trifles become important). Her teeth
were of a pearly whiteness, and her large black eyes sparkled
with uncommon fire, tempered by the most attractive sweetness.
Her voice was strong and harmonious. Her manly understanding was
strengthened and adorned by study. She was not ignorant of the
Latin tongue, but possessed in equal perfection the Greek, the
Syriac, and the Egyptian languages. She had drawn up for her own
use an epitome of oriental history, and familiarly compared the
beauties of Homer and Plato under the tuition of the sublime
Longinus.

\footnotetext[54]{Almost everything that is said of the manners of
Odenathus and Zenobia is taken from their lives in the Augustan
History, by Trebeljus Pollio; see p. 192, 198.}

\footnotetext[541]{According to some Christian writers, Zenobia was a
Jewess. (Jost Geschichte der Israel. iv. 16. Hist. of Jews, iii.
175.)—M.}

\footnotetext[55]{She never admitted her husband’s embraces but for
the sake of posterity. If her hopes were baffled, in the ensuing
month she reiterated the experiment.}

This accomplished woman gave her hand to Odenathus,\footnotemark[551] who, from
a private station, raised himself to the dominion of the East.
She soon became the friend and companion of a hero. In the
intervals of war, Odenathus passionately delighted in the
exercise of hunting; he pursued with ardor the wild beasts of the
desert, lions, panthers, and bears; and the ardor of Zenobia in
that dangerous amusement was not inferior to his own. She had
inured her constitution to fatigue, disdained the use of a
covered carriage, generally appeared on horseback in a military
habit, and sometimes marched several miles on foot at the head of
the troops. The success of Odenathus was in a great measure
ascribed to her incomparable prudence and fortitude. Their
splendid victories over the Great King, whom they twice pursued
as far as the gates of Ctesiphon, laid the foundations of their
united fame and power. The armies which they commanded, and the
provinces which they had saved, acknowledged not any other
sovereigns than their invincible chiefs. The senate and people of
Rome revered a stranger who had avenged their captive emperor,
and even the insensible son of Valerian accepted Odenathus for
his legitimate colleague.

\footnotetext[551]{According to Zosimus, Odenathus was of a noble
family in Palmyra and according to Procopius, he was prince of
the Saracens, who inhabit the ranks of the Euphrates. Echhel.
Doct. Num. vii. 489.—G.}

