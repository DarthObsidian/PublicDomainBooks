\chapter{Reigns Of Tacitus, Probus, Carus And His Sons.}
\section{Part \thesection.}

\textit{Conduct Of The Army And Senate After The Death Of Aurelian.
— Reigns Of Tacitus, Probus, Carus, And His Sons.}
\vspace{\onelineskip}

Such was the unhappy condition of the Roman emperors, that,
whatever might be their conduct, their fate was commonly the
same. A life of pleasure or virtue, of severity or mildness, of
indolence or glory, alike led to an untimely grave; and almost
every reign is closed by the same disgusting repetition of
treason and murder. The death of Aurelian, however, is remarkable
by its extraordinary consequences. The legions admired, lamented,
and revenged their victorious chief. The artifice of his
perfidious secretary was discovered and punished.

The deluded conspirators attended the funeral of their injured
sovereign, with sincere or well-feigned contrition, and submitted
to the unanimous resolution of the military order, which was
signified by the following epistle: “The brave and fortunate
armies to the senate and people of Rome.—The crime of one man,
and the error of many, have deprived us of the late emperor
Aurelian. May it please you, venerable lords and fathers! to
place him in the number of the gods, and to appoint a successor
whom your judgment shall declare worthy of the Imperial purple!
None of those whose guilt or misfortune have contributed to our
loss, shall ever reign over us.”\footnotemark[1] The Roman senators heard,
without surprise, that another emperor had been assassinated in
his camp; they secretly rejoiced in the fall of Aurelian; but the
modest and dutiful address of the legions, when it was
communicated in full assembly by the consul, diffused the most
pleasing astonishment. Such honors as fear and perhaps esteem
could extort, they liberally poured forth on the memory of their
deceased sovereign. Such acknowledgments as gratitude could
inspire, they returned to the faithful armies of the republic,
who entertained so just a sense of the legal authority of the
senate in the choice of an emperor. Yet, notwithstanding this
flattering appeal, the most prudent of the assembly declined
exposing their safety and dignity to the caprice of an armed
multitude. The strength of the legions was, indeed, a pledge of
their sincerity, since those who may command are seldom reduced
to the necessity of dissembling; but could it naturally be
expected, that a hasty repentance would correct the inveterate
habits of fourscore years? Should the soldiers relapse into their
accustomed seditions, their insolence might disgrace the majesty
of the senate, and prove fatal to the object of its choice.
Motives like these dictated a decree, by which the election of a
new emperor was referred to the suffrage of the military order.

\footnotetext[1]{Vopiscus in Hist. August. p. 222. Aurelius Victor
mentions a formal deputation from the troops to the senate.}

The contention that ensued is one of the best attested, but most
improbable events in the history of mankind.\footnotemark[2] The troops, as if
satiated with the exercise of power, again conjured the senate to
invest one of its own body with the Imperial purple. The senate
still persisted in its refusal; the army in its request. The
reciprocal offer was pressed and rejected at least three times,
and, whilst the obstinate modesty of either party was resolved to
receive a master from the hands of the other, eight months
insensibly elapsed; an amazing period of tranquil anarchy, during
which the Roman world remained without a sovereign, without a
usurper, and without a sedition. 201 The generals and magistrates
appointed by Aurelian continued to execute their ordinary
functions; and it is observed, that a proconsul of Asia was the
only considerable person removed from his office in the whole
course of the interregnum.

\footnotetext[2]{Vopiscus, our principal authority, wrote at Rome,
sixteen years only after the death of Aurelian; and, besides the
recent notoriety of the facts, constantly draws his materials
from the Journals of the Senate, and the original papers of the
Ulpian library. Zosimus and Zonaras appear as ignorant of this
transaction as they were in general of the Roman constitution.}

\footnotetext[201]{The interregnum could not be more than seven
months; Aurelian was assassinated in the middle of March, the
year of Rome 1028. Tacitus was elected the 25th September in the
same year.—G.}

An event somewhat similar, but much less authentic, is supposed
to have happened after the death of Romulus, who, in his life and
character, bore some affinity with Aurelian. The throne was
vacant during twelve months, till the election of a Sabine
philosopher, and the public peace was guarded in the same manner,
by the union of the several orders of the state. But, in the time
of Numa and Romulus, the arms of the people were controlled by
the authority of the Patricians; and the balance of freedom was
easily preserved in a small and virtuous community.\footnotemark[3] The decline
of the Roman state, far different from its infancy, was attended
with every circumstance that could banish from an interregnum the
prospect of obedience and harmony: an immense and tumultuous
capital, a wide extent of empire, the servile equality of
despotism, an army of four hundred thousand mercenaries, and the
experience of frequent revolutions. Yet, notwithstanding all
these temptations, the discipline and memory of Aurelian still
restrained the seditious temper of the troops, as well as the
fatal ambition of their leaders. The flower of the legions
maintained their stations on the banks of the Bosphorus, and the
Imperial standard awed the less powerful camps of Rome and of the
provinces. A generous though transient enthusiasm seemed to
animate the military order; and we may hope that a few real
patriots cultivated the returning friendship of the army and the
senate as the only expedient capable of restoring the republic to
its ancient beauty and vigor.

\footnotetext[3]{Liv. i. 17 Dionys. Halicarn. l. ii. p. 115. Plutarch
in Numa, p. 60. The first of these writers relates the story like
an orator, the second like a lawyer, and the third like a
moralist, and none of them probably without some intermixture of
fable.}

On the twenty-fifth of September, near eight months after the
murder of Aurelian, the consul convoked an assembly of the
senate, and reported the doubtful and dangerous situation of the
empire. He slightly insinuated, that the precarious loyalty of
the soldiers depended on the chance of every hour, and of every
accident; but he represented, with the most convincing eloquence,
the various dangers that might attend any further delay in the
choice of an emperor. Intelligence, he said, was already
received, that the Germans had passed the Rhine, and occupied
some of the strongest and most opulent cities of Gaul. The
ambition of the Persian king kept the East in perpetual alarms;
Egypt, Africa, and Illyricum, were exposed to foreign and
domestic arms, and the levity of Syria would prefer even a female
sceptre to the sanctity of the Roman laws. The consul, then
addressing himself to Tacitus, the first of the senators,\footnotemark[4]
required his opinion on the important subject of a proper
candidate for the vacant throne.

\footnotetext[4]{Vopiscus (in Hist. August p. 227) calls him “primæ
sententia consularis;” and soon afterwards Princeps senatus. It
is natural to suppose, that the monarchs of Rome, disdaining that
humble title, resigned it to the most ancient of the senators.}

If we can prefer personal merit to accidental greatness, we shall
esteem the birth of Tacitus more truly noble than that of kings.
He claimed his descent from the philosophic historian whose
writings will instruct the last generations of mankind.\footnotemark[5] The
senator Tacitus was then seventy-five years of age.\footnotemark[6] The long
period of his innocent life was adorned with wealth and honors.
He had twice been invested with the consular dignity,\footnotemark[7] and
enjoyed with elegance and sobriety his ample patrimony of between
two and three millions sterling.\footnotemark[8] The experience of so many
princes, whom he had esteemed or endured, from the vain follies
of Elagabalus to the useful rigor of Aurelian, taught him to form
a just estimate of the duties, the dangers, and the temptations
of their sublime station. From the assiduous study of his
immortal ancestor, he derived the knowledge of the Roman
constitution, and of human nature.\footnotemark[9] The voice of the people had
already named Tacitus as the citizen the most worthy of empire.
The ungrateful rumor reached his ears, and induced him to seek
the retirement of one of his villas in Campania. He had passed
two months in the delightful privacy of Baiæ, when he reluctantly
obeyed the summons of the consul to resume his honorable place in
the senate, and to assist the republic with his counsels on this
important occasion.

\footnotetext[5]{The only objection to this genealogy is, that the
historian was named Cornelius, the emperor, Claudius. But under
the lower empire, surnames were extremely various and uncertain.}

\footnotetext[6]{Zonaras, l. xii. p. 637. The Alexandrian Chronicle,
by an obvious mistake, transfers that age to Aurelian.}

\footnotetext[7]{In the year 273, he was ordinary consul. But he must
have been Suffectus many years before, and most probably under
Valerian.}

\footnotetext[8]{Bis millies octingenties. Vopiscus in Hist. August
p. 229. This sum, according to the old standard, was equivalent
to eight hundred and forty thousand Roman pounds of silver, each
of the value of three pounds sterling. But in the age of Tacitus,
the coin had lost much of its weight and purity.}

\footnotetext[9]{After his accession, he gave orders that ten copies
of the historian should be annually transcribed and placed in the
public libraries. The Roman libraries have long since perished,
and the most valuable part of Tacitus was preserved in a single
Ms., and discovered in a monastery of Westphalia. See Bayle,
Dictionnaire, Art. Tacite, and Lipsius ad Annal. ii. 9.}

He arose to speak, when from every quarter of the house, he was
saluted with the names of Augustus and emperor. “Tacitus
Augustus, the gods preserve thee! we choose thee for our
sovereign; to thy care we intrust the republic and the world.
Accept the empire from the authority of the senate. It is due to
thy rank, to thy conduct, to thy manners.” As soon as the tumult
of acclamations subsided, Tacitus attempted to decline the
dangerous honor, and to express his wonder, that they should
elect his age and infirmities to succeed the martial vigor of
Aurelian. “Are these limbs, conscript fathers! fitted to sustain
the weight of armor, or to practise the exercises of the camp?
The variety of climates, and the hardships of a military life,
would soon oppress a feeble constitution, which subsists only by
the most tender management. My exhausted strength scarcely
enables me to discharge the duty of a senator; how insufficient
would it prove to the arduous labors of war and government! Can
you hope, that the legions will respect a weak old man, whose
days have been spent in the shade of peace and retirement? Can
you desire that I should ever find reason to regret the favorable
opinion of the senate?”\footnotemark[10]

\footnotetext[10]{Vopiscus in Hist. August. p. 227.}

The reluctance of Tacitus (and it might possibly be sincere) was
encountered by the affectionate obstinacy of the senate. Five
hundred voices repeated at once, in eloquent confusion, that the
greatest of the Roman princes, Numa, Trajan, Hadrian, and the
Antonines, had ascended the throne in a very advanced season of
life; that the mind, not the body, a sovereign, not a soldier,
was the object of their choice; and that they expected from him
no more than to guide by his wisdom the valor of the legions.
These pressing though tumultuary instances were seconded by a
more regular oration of Metius Falconius, the next on the
consular bench to Tacitus himself. He reminded the assembly of
the evils which Rome had endured from the vices of headstrong and
capricious youths, congratulated them on the election of a
virtuous and experienced senator, and, with a manly, though
perhaps a selfish, freedom, exhorted Tacitus to remember the
reasons of his elevation, and to seek a successor, not in his own
family, but in the republic. The speech of Falconius was enforced
by a general acclamation. The emperor elect submitted to the
authority of his country, and received the voluntary homage of
his equals. The judgment of the senate was confirmed by the
consent of the Roman people and of the Prætorian guards.\footnotemark[11]

\footnotetext[11]{Hist. August. p. 228. Tacitus addressed the
Prætorians by the appellation of sanctissimi milites, and the
people by that of sacratissim. Quirites.}

The administration of Tacitus was not unworthy of his life and
principles. A grateful servant of the senate, he considered that
national council as the author, and himself as the subject, of
the laws.\footnotemark[12] He studied to heal the wounds which Imperial pride,
civil discord, and military violence, had inflicted on the
constitution, and to restore, at least, the image of the ancient
republic, as it had been preserved by the policy of Augustus, and
the virtues of Trajan and the Antonines. It may not be useless to
recapitulate some of the most important prerogatives which the
senate appeared to have regained by the election of Tacitus.\footnotemark[13]
1. To invest one of their body, under the title of emperor, with
the general command of the armies, and the government of the
frontier provinces. 2. To determine the list, or, as it was then
styled, the College of Consuls. They were twelve in number, who,
in successive pairs, each, during the space of two months, filled
the year, and represented the dignity of that ancient office. The
authority of the senate, in the nomination of the consuls, was
exercised with such independent freedom, that no regard was paid
to an irregular request of the emperor in favor of his brother
Florianus. “The senate,” exclaimed Tacitus, with the honest
transport of a patriot, “understand the character of a prince
whom they have chosen.” 3. To appoint the proconsuls and
presidents of the provinces, and to confer on all the magistrates
their civil jurisdiction. 4. To receive appeals through the
intermediate office of the præfect of the city from all the
tribunals of the empire. 5. To give force and validity, by their
decrees, to such as they should approve of the emperor’s edicts.
6. To these several branches of authority we may add some
inspection over the finances, since, even in the stern reign of
Aurelian, it was in their power to divert a part of the revenue
from the public service.\footnotemark[14]

\footnotetext[12]{In his manumissions he never exceeded the number of
a hundred, as limited by the Caninian law, which was enacted
under Augustus, and at length repealed by Justinian. See Casaubon
ad locum Vopisci.}

\footnotetext[13]{See the lives of Tacitus, Florianus, and Probus, in
the Augustan History; we may be well assured, that whatever the
soldier gave the senator had already given.}

\footnotetext[14]{Vopiscus in Hist. August. p. 216. The passage is
perfectly clear, both Casaubon and Salmasius wish to correct it.}

Circular epistles were sent, without delay, to all the principal
cities of the empire, Treves, Milan, Aquileia, Thessalonica,
Corinth, Athens, Antioch, Alexandria, and Carthage, to claim
their obedience, and to inform them of the happy revolution,
which had restored the Roman senate to its ancient dignity. Two
of these epistles are still extant. We likewise possess two very
singular fragments of the private correspondence of the senators
on this occasion. They discover the most excessive joy, and the
most unbounded hopes. “Cast away your indolence,” it is thus that
one of the senators addresses his friend, “emerge from your
retirements of Baiæ and Puteoli. Give yourself to the city, to
the senate. Rome flourishes, the whole republic flourishes.
Thanks to the Roman army, to an army truly Roman; at length we
have recovered our just authority, the end of all our desires. We
hear appeals, we appoint proconsuls, we create emperors; perhaps
too we may restrain them—to the wise a word is sufficient.”\footnotemark[15]
These lofty expectations were, however, soon disappointed; nor,
indeed, was it possible that the armies and the provinces should
long obey the luxurious and unwarlike nobles of Rome. On the
slightest touch, the unsupported fabric of their pride and power
fell to the ground. The expiring senate displayed a sudden
lustre, blazed for a moment, and was extinguished forever.

\footnotetext[15]{Vopiscus in Hist. August. p. 230, 232, 233. The
senators celebrated the happy restoration with hecatombs and
public rejoicings.}

All that had yet passed at Rome was no more than a theatrical
representation, unless it was ratified by the more substantial
power of the legions. Leaving the senators to enjoy their dream
of freedom and ambition, Tacitus proceeded to the Thracian camp,
and was there, by the Prætorian præfect, presented to the
assembled troops, as the prince whom they themselves had
demanded, and whom the senate had bestowed. As soon as the
præfect was silent, the emperor addressed himself to the soldiers
with eloquence and propriety. He gratified their avarice by a
liberal distribution of treasure, under the names of pay and
donative. He engaged their esteem by a spirited declaration, that
although his age might disable him from the performance of
military exploits, his counsels should never be unworthy of a
Roman general, the successor of the brave Aurelian.\footnotemark[16]

\footnotetext[16]{Hist. August. p. 228.}

Whilst the deceased emperor was making preparations for a second
expedition into the East, he had negotiated with the Alani,\footnotemark[161] a
Scythian people, who pitched their tents in the neighborhood of
the Lake Mæotis. Those barbarians, allured by presents and
subsidies, had promised to invade Persia with a numerous body of
light cavalry. They were faithful to their engagements; but when
they arrived on the Roman frontier, Aurelian was already dead,
the design of the Persian war was at least suspended, and the
generals, who, during the interregnum, exercised a doubtful
authority, were unprepared either to receive or to oppose them.
Provoked by such treatment, which they considered as trifling and
perfidious, the Alani had recourse to their own valor for their
payment and revenge; and as they moved with the usual swiftness
of Tartars, they had soon spread themselves over the provinces of
Pontus, Cappadocia, Cilicia, and Galatia. The legions, who from
the opposite shores of the Bosphorus could almost distinguish the
flames of the cities and villages, impatiently urged their
general to lead them against the invaders. The conduct of Tacitus
was suitable to his age and station. He convinced the barbarians
of the faith, as well as the power, of the empire. Great numbers
of the Alani, appeased by the punctual discharge of the
engagements which Aurelian had contracted with them, relinquished
their booty and captives, and quietly retreated to their own
deserts, beyond the Phasis. Against the remainder, who refused
peace, the Roman emperor waged, in person, a successful war.
Seconded by an army of brave and experienced veterans, in a few
weeks he delivered the provinces of Asia from the terror of the
Scythian invasion.\footnotemark[17]

\footnotetext[161]{On the Alani, see ch. xxvi. note 55.—M.}

\footnotetext[17]{Vopiscus in Hist. August. p. 230. Zosimus, l. i. p.
57. Zonaras, l. xii. p. 637. Two passages in the life of Probus
(p. 236, 238) convince me, that these Scythian invaders of Pontus
were Alani. If we may believe Zosimus, (l. i. p. 58,) Florianus
pursued them as far as the Cimmerian Bosphorus. But he had
scarcely time for so long and difficult an expedition.}

But the glory and life of Tacitus were of short duration.
Transported, in the depth of winter, from the soft retirement of
Campania to the foot of Mount Caucasus, he sunk under the
unaccustomed hardships of a military life. The fatigues of the
body were aggravated by the cares of the mind. For a while, the
angry and selfish passions of the soldiers had been suspended by
the enthusiasm of public virtue. They soon broke out with
redoubled violence, and raged in the camp, and even in the tent
of the aged emperor. His mild and amiable character served only
to inspire contempt, and he was incessantly tormented with
factions which he could not assuage, and by demands which it was
impossible to satisfy. Whatever flattering expectations he had
conceived of reconciling the public disorders, Tacitus soon was
convinced that the licentiousness of the army disdained the
feeble restraint of laws, and his last hour was hastened by
anguish and disappointment. It may be doubtful whether the
soldiers imbrued their hands in the blood of this innocent
prince.\footnotemark[18] It is certain that their insolence was the cause of
his death. He expired at Tyana in Cappadocia, after a reign of
only six months and about twenty days.\footnotemark[19]

\footnotetext[18]{Eutropius and Aurelius Victor only say that he
died; Victor Junior adds, that it was of a fever. Zosimus and
Zonaras affirm, that he was killed by the soldiers. Vopiscus
mentions both accounts, and seems to hesitate. Yet surely these
jarring opinions are easily reconciled.}

\footnotetext[19]{According to the two Victors, he reigned exactly
two hundred days.}

The eyes of Tacitus were scarcely closed, before his brother
Florianus showed himself unworthy to reign, by the hasty
usurpation of the purple, without expecting the approbation of
the senate. The reverence for the Roman constitution, which yet
influenced the camp and the provinces, was sufficiently strong to
dispose them to censure, but not to provoke them to oppose, the
precipitate ambition of Florianus. The discontent would have
evaporated in idle murmurs, had not the general of the East, the
heroic Probus, boldly declared himself the avenger of the senate.

The contest, however, was still unequal; nor could the most able
leader, at the head of the effeminate troops of Egypt and Syria,
encounter, with any hopes of victory, the legions of Europe,
whose irresistible strength appeared to support the brother of
Tacitus. But the fortune and activity of Probus triumphed over
every obstacle. The hardy veterans of his rival, accustomed to
cold climates, sickened and consumed away in the sultry heats of
Cilicia, where the summer proved remarkably unwholesome. Their
numbers were diminished by frequent desertion; the passes of the
mountains were feebly defended; Tarsus opened its gates; and the
soldiers of Florianus, when they had permitted him to enjoy the
Imperial title about three months, delivered the empire from
civil war by the easy sacrifice of a prince whom they despised.\footnotemark[20]

\footnotetext[20]{Hist. August, p. 231. Zosimus, l. i. p. 58, 59.
Zonaras, l. xii. p. 637. Aurelius Victor says, that Probus
assumed the empire in Illyricum; an opinion which (though adopted
by a very learned man) would throw that period of history into
inextricable confusion.}

The perpetual revolutions of the throne had so perfectly erased
every notion of hereditary title, that the family of an
unfortunate emperor was incapable of exciting the jealousy of his
successors. The children of Tacitus and Florianus were permitted
to descend into a private station, and to mingle with the general
mass of the people. Their poverty indeed became an additional
safeguard to their innocence. When Tacitus was elected by the
senate, he resigned his ample patrimony to the public service;\footnotemark[21]
an act of generosity specious in appearance, but which evidently
disclosed his intention of transmitting the empire to his
descendants. The only consolation of their fallen state was the
remembrance of transient greatness, and a distant hope, the child
of a flattering prophecy, that at the end of a thousand years, a
monarch of the race of Tacitus should arise, the protector of the
senate, the restorer of Rome, and the conqueror of the whole
earth.\footnotemark[22]

\footnotetext[21]{Hist. August. p. 229}

\footnotetext[22]{He was to send judges to the Parthians, Persians,
and Sarmatians, a president to Taprobani, and a proconsul to the
Roman island, (supposed by Casaubon and Salmasius to mean
Britain.) Such a history as mine (says Vopiscus with proper
modesty) will not subsist a thousand years, to expose or justify
the prediction.}

The peasants of Illyricum, who had already given Claudius and
Aurelian to the sinking empire, had an equal right to glory in
the elevation of Probus.\footnotemark[23] Above twenty years before, the
emperor Valerian, with his usual penetration, had discovered the
rising merit of the young soldier, on whom he conferred the rank
of tribune, long before the age prescribed by the military
regulations. The tribune soon justified his choice, by a victory
over a great body of Sarmatians, in which he saved the life of a
near relation of Valerian; and deserved to receive from the
emperor’s hand the collars, bracelets, spears, and banners, the
mural and the civic crown, and all the honorable rewards reserved
by ancient Rome for successful valor. The third, and afterwards
the tenth, legion were intrusted to the command of Probus, who,
in every step of his promotion, showed himself superior to the
station which he filled. Africa and Pontus, the Rhine, the
Danube, the Euphrates, and the Nile, by turns afforded him the
most splendid occasions of displaying his personal prowess and
his conduct in war. Aurelian was indebted for the honest courage
with which he often checked the cruelty of his master. Tacitus,
who desired by the abilities of his generals to supply his own
deficiency of military talents, named him commander-in-chief of
all the eastern provinces, with five times the usual salary, the
promise of the consulship, and the hope of a triumph. When Probus
ascended the Imperial throne, he was about forty-four years of
age;\footnotemark[24] in the full possession of his fame, of the love of the
army, and of a mature vigor of mind and body.

\footnotetext[23]{For the private life of Probus, see Vopiscus in
Hist. August p. 234—237}

\footnotetext[24]{According to the Alexandrian chronicle, he was
fifty at the time of his death.}

His acknowledged merit, and the success of his arms against
Florianus, left him without an enemy or a competitor. Yet, if we
may credit his own professions, very far from being desirous of
the empire, he had accepted it with the most sincere reluctance.
“But it is no longer in my power,” says Probus, in a private
letter, “to lay down a title so full of envy and of danger. I
must continue to personate the character which the soldiers have
imposed upon me.”\footnotemark[25] His dutiful address to the senate displayed
the sentiments, or at least the language, of a Roman patriot:
“When you elected one of your order, conscript fathers! to
succeed the emperor Aurelian, you acted in a manner suitable to
your justice and wisdom. For you are the legal sovereigns of the
world, and the power which you derive from your ancestors will
descend to your posterity. Happy would it have been, if
Florianus, instead of usurping the purple of his brother, like a
private inheritance, had expected what your majesty might
determine, either in his favor, or in that of any other person.
The prudent soldiers have punished his rashness. To me they have
offered the title of Augustus. But I submit to your clemency my
pretensions and my merits.”\footnotemark[26] When this respectful epistle was
read by the consul, the senators were unable to disguise their
satisfaction, that Probus should condescend thus numbly to
solicit a sceptre which he already possessed. They celebrated
with the warmest gratitude his virtues, his exploits, and above
all his moderation. A decree immediately passed, without a
dissenting voice, to ratify the election of the eastern armies,
and to confer on their chief all the several branches of the
Imperial dignity: the names of Cæsar and Augustus, the title of
Father of his country, the right of making in the same day three
motions in the senate,\footnotemark[27] the office of Pontifex Maximus, the
tribunitian power, and the proconsular command; a mode of
investiture, which, though it seemed to multiply the authority of
the emperor, expressed the constitution of the ancient republic.
The reign of Probus corresponded with this fair beginning. The
senate was permitted to direct the civil administration of the
empire. Their faithful general asserted the honor of the Roman
arms, and often laid at their feet crowns of gold and barbaric
trophies, the fruits of his numerous victories.\footnotemark[28] Yet, whilst he
gratified their vanity, he must secretly have despised their
indolence and weakness. Though it was every moment in their power
to repeal the disgraceful edict of Gallienus, the proud
successors of the Scipios patiently acquiesced in their exclusion
from all military employments. They soon experienced, that those
who refuse the sword must renounce the sceptre.

\footnotetext[25]{This letter was addressed to the Prætorian præfect,
whom (on condition of his good behavior) he promised to continue
in his great office. See Hist. August. p. 237.}

\footnotetext[26]{Vopiscus in Hist. August. p. 237. The date of the
letter is assuredly faulty. Instead of Nen. Februar. we may read
Non August.}

\footnotetext[27]{Hist. August. p. 238. It is odd that the senate
should treat Probus less favorably than Marcus Antoninus. That
prince had received, even before the death of Pius, Jus quintoe
relationis. See Capitolin. in Hist. August. p. 24.}

\footnotetext[28]{See the dutiful letter of Probus to the senate,
after his German victories. Hist. August. p. 239.}

