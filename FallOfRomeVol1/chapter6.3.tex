\section{Part \thesection.}
\thispagestyle{simple}

To this temple, as to the common centre of religious worship, the
Imperial fanatic attempted to remove the Ancilia, the Palladium,\footnotemark[54]
and all the sacred pledges of the faith of Numa. A crowd of
inferior deities attended in various stations the majesty of the
god of Emesa; but his court was still imperfect, till a female of
distinguished rank was admitted to his bed. Pallas had been first
chosen for his consort; but as it was dreaded lest her warlike
terrors might affright the soft delicacy of a Syrian deity, the
Moon, adored by the Africans under the name of Astarte, was
deemed a more suitable companion for the Sun. Her image, with the
rich offerings of her temple as a marriage portion, was
transported with solemn pomp from Carthage to Rome, and the day
of these mystic nuptials was a general festival in the capital
and throughout the empire.\footnotemark[55]

\footnotetext[54]{He broke into the sanctuary of Vesta, and carried
away a statue, which he supposed to be the palladium; but the
vestals boasted that, by a pious fraud, they had imposed a
counterfeit image on the profane intruder. Hist. August., p.
103.}

\footnotetext[55]{Dion, l. lxxix. p. 1360. Herodian, l. v. p. 193.
The subjects of the empire were obliged to make liberal presents
to the new married couple; and whatever they had promised during
the life of Elagabalus was carefully exacted under the
administration of Mamæa.}

A rational voluptuary adheres with invariable respect to the
temperate dictates of nature, and improves the gratifications of
sense by social intercourse, endearing connections, and the soft
coloring of taste and the imagination. But Elagabalus, (I speak
of the emperor of that name,) corrupted by his youth, his
country, and his fortune, abandoned himself to the grossest
pleasures with ungoverned fury, and soon found disgust and
satiety in the midst of his enjoyments. The inflammatory powers
of art were summoned to his aid: the confused multitude of women,
of wines, and of dishes, and the studied variety of attitude and
sauces, served to revive his languid appetites. New terms and new
inventions in these sciences, the only ones cultivated and
patronized by the monarch,\footnotemark[56] signalized his reign, and
transmitted his infamy to succeeding times. A capricious
prodigality supplied the want of taste and elegance; and whilst
Elagabalus lavished away the treasures of his people in the
wildest extravagance, his own voice and that of his flatterers
applauded a spirit of magnificence unknown to the tameness of his
predecessors. To confound the order of seasons and climates,\footnotemark[57]
to sport with the passions and prejudices of his subjects, and to
subvert every law of nature and decency, were in the number of
his most delicious amusements. A long train of concubines, and a
rapid succession of wives, among whom was a vestal virgin,
ravished by force from her sacred asylum,\footnotemark[58] were insufficient to
satisfy the impotence of his passions. The master of the Roman
world affected to copy the dress and manners of the female sex,
preferred the distaff to the sceptre, and dishonored the
principal dignities of the empire by distributing them among his
numerous lovers; one of whom was publicly invested with the title
and authority of the emperor’s, or, as he more properly styled
himself, of the empress’s husband.\footnotemark[59]

\footnotetext[56]{The invention of a new sauce was liberally
rewarded; but if it was not relished, the inventor was confined
to eat of nothing else till he had discovered another more
agreeable to the Imperial palate Hist. August. p. 111.}

\footnotetext[57]{He never would eat sea-fish except at a great
distance from the sea; he then would distribute vast quantities
of the rarest sorts, brought at an immense expense, to the
peasants of the inland country. Hist. August. p. 109.}

\footnotetext[58]{Dion, l. lxxix. p. 1358. Herodian, l. v. p. 192.}

\footnotetext[59]{Hierocles enjoyed that honor; but he would have
been supplanted by one Zoticus, had he not contrived, by a
potion, to enervate the powers of his rival, who, being found on
trial unequal to his reputation, was driven with ignominy from
the palace. Dion, l. lxxix. p. 1363, 1364. A dancer was made
præfect of the city, a charioteer præfect of the watch, a barber
præfect of the provisions. These three ministers, with many
inferior officers, were all recommended enormitate membrorum.
Hist. August. p. 105.}

It may seem probable, the vices and follies of Elagabalus have
been adorned by fancy, and blackened by prejudice.\footnotemark[60] Yet,
confining ourselves to the public scenes displayed before the
Roman people, and attested by grave and contemporary historians,
their inexpressible infamy surpasses that of any other age or
country. The license of an eastern monarch is secluded from the
eye of curiosity by the inaccessible walls of his seraglio. The
sentiments of honor and gallantry have introduced a refinement of
pleasure, a regard for decency, and a respect for the public
opinion, into the modern courts of Europe;\footnotemark[601] but the corrupt
and opulent nobles of Rome gratified every vice that could be
collected from the mighty conflux of nations and manners. Secure
of impunity, careless of censure, they lived without restraint in
the patient and humble society of their slaves and parasites. The
emperor, in his turn, viewing every rank of his subjects with the
same contemptuous indifference, asserted without control his
sovereign privilege of lust and luxury.

\footnotetext[60]{Even the credulous compiler of his life, in the
Augustan History (p. 111) is inclined to suspect that his vices
may have been exaggerated.}

\footnotetext[601]{Wenck has justly observed that Gibbon should have
reckoned the influence of Christianity in this great change. In
the most savage times, and the most corrupt courts, since the
introduction of Christianity there have been no Neros or
Domitians, no Commodus or Elagabalus.—M.}

The most worthless of mankind are not afraid to condemn in others
the same disorders which they allow in themselves; and can
readily discover some nice difference of age, character, or
station, to justify the partial distinction. The licentious
soldiers, who had raised to the throne the dissolute son of
Caracalla, blushed at their ignominious choice, and turned with
disgust from that monster, to contemplate with pleasure the
opening virtues of his cousin Alexander, the son of Mamæa. The
crafty Mæsa, sensible that her grandson Elagabalus must
inevitably destroy himself by his own vices, had provided another
and surer support of her family. Embracing a favorable moment of
fondness and devotion, she had persuaded the young emperor to
adopt Alexander, and to invest him with the title of Cæsar, that
his own divine occupations might be no longer interrupted by the
care of the earth. In the second rank that amiable prince soon
acquired the affections of the public, and excited the tyrant’s
jealousy, who resolved to terminate the dangerous competition,
either by corrupting the manners, or by taking away the life, of
his rival. His arts proved unsuccessful; his vain designs were
constantly discovered by his own loquacious folly, and
disappointed by those virtuous and faithful servants whom the
prudence of Mamæa had placed about the person of her son. In a
hasty sally of passion, Elagabalus resolved to execute by force
what he had been unable to compass by fraud, and by a despotic
sentence degraded his cousin from the rank and honors of Cæsar.
The message was received in the senate with silence, and in the
camp with fury. The Prætorian guards swore to protect Alexander,
and to revenge the dishonored majesty of the throne. The tears
and promises of the trembling Elagabalus, who only begged them to
spare his life, and to leave him in the possession of his beloved
Hierocles, diverted their just indignation; and they contented
themselves with empowering their præfects to watch over the
safety of Alexander, and the conduct of the emperor. 61

\footnotetext[61]{Dion, l. lxxix. p. 1365. Herodian, l. v. p.
195—201. Hist. August. p. 105. The last of the three historians
seems to have followed the best authors in his account of the
revolution.}

It was impossible that such a reconciliation should last, or that
even the mean soul of Elagabalus could hold an empire on such
humiliating terms of dependence. He soon attempted, by a
dangerous experiment, to try the temper of the soldiers. The
report of the death of Alexander, and the natural suspicion that
he had been murdered, inflamed their passions into fury, and the
tempest of the camp could only be appeased by the presence and
authority of the popular youth. Provoked at this new instance of
their affection for his cousin, and their contempt for his
person, the emperor ventured to punish some of the leaders of the
mutiny. His unseasonable severity proved instantly fatal to his
minions, his mother, and himself. Elagabalus was massacred by the
indignant Prætorians, his mutilated corpse dragged through the
streets of the city, and thrown into the Tiber. His memory was
branded with eternal infamy by the senate; the justice of whose
decree has been ratified by posterity.\footnotemark[62]

\footnotetext[62]{The æra of the death of Elagabalus, and of the
accession of Alexander, has employed the learning and ingenuity
of Pagi, Tillemont, Valsecchi, Vignoli, and Torre, bishop of
Adria. The question is most assuredly intricate; but I still
adhere to the authority of Dion, the truth of whose calculations
is undeniable, and the purity of whose text is justified by the
agreement of Xiphilin, Zonaras, and Cedrenus. Elagabalus reigned
three years nine months and four days, from his victory over
Macrinus, and was killed March 10, 222. But what shall we reply
to the medals, undoubtedly genuine, which reckon the fifth year
of his tribunitian power? We shall reply, with the learned
Valsecchi, that the usurpation of Macrinus was annihilated, and
that the son of Caracalla dated his reign from his father’s
death? After resolving this great difficulty, the smaller knots
of this question may be easily untied, or cut asunder. Note: This
opinion of Valsecchi has been triumphantly contested by Eckhel,
who has shown the impossibility of reconciling it with the medals
of Elagabalus, and has given the most satisfactory explanation of
the five tribunates of that emperor. He ascended the throne and
received the tribunitian power the 16th of May, in the year of
Rome 971; and on the 1st January of the next year, 972, he began
a new tribunate, according to the custom established by preceding
emperors. During the years 972, 973, 974, he enjoyed the
tribunate, and commenced his fifth in the year 975, during which
he was killed on the 10th March. Eckhel de Doct. Num. viii. 430
\&c.—G.}

In the room of Elagabalus, his cousin Alexander was raised to the
throne by the Prætorian guards. His relation to the family of
Severus, whose name he assumed, was the same as that of his
predecessor; his virtue and his danger had already endeared him
to the Romans, and the eager liberality of the senate conferred
upon him, in one day, the various titles and powers of the
Imperial dignity.\footnotemark[63] But as Alexander was a modest and dutiful
youth, of only seventeen years of age, the reins of government
were in the hands of two women, of his mother, Mamæa, and of
Mæsa, his grandmother. After the death of the latter, who
survived but a short time the elevation of Alexander, Mamæa
remained the sole regent of her son and of the empire.

\footnotetext[63]{Hist. August. p. 114. By this unusual
precipitation, the senate meant to confound the hopes of
pretenders, and prevent the factions of the armies.}

In every age and country, the wiser, or at least the stronger, of
the two sexes, has usurped the powers of the state, and confined
the other to the cares and pleasures of domestic life. In
hereditary monarchies, however, and especially in those of modern
Europe, the gallant spirit of chivalry, and the law of
succession, have accustomed us to allow a singular exception; and
a woman is often acknowledged the absolute sovereign of a great
kingdom, in which she would be deemed incapable of exercising the
smallest employment, civil or military. But as the Roman emperors
were still considered as the generals and magistrates of the
republic, their wives and mothers, although distinguished by the
name of Augusta, were never associated to their personal honors;
and a female reign would have appeared an inexpiable prodigy in
the eyes of those primitive Romans, who married without love, or
loved without delicacy and respect.\footnotemark[64] The haughty Agrippina
aspired, indeed, to share the honors of the empire which she had
conferred on her son; but her mad ambition, detested by every
citizen who felt for the dignity of Rome, was disappointed by the
artful firmness of Seneca and Burrhus.\footnotemark[65] The good sense, or the
indifference, of succeeding princes, restrained them from
offending the prejudices of their subjects; and it was reserved
for the profligate Elagabalus to discharge the acts of the senate
with the name of his mother Soæmias, who was placed by the side
of the consuls, and subscribed, as a regular member, the decrees
of the legislative assembly. Her more prudent sister, Mamæa,
declined the useless and odious prerogative, and a solemn law was
enacted, excluding women forever from the senate, and devoting to
the infernal gods the head of the wretch by whom this sanction
should be violated.\footnotemark[66] The substance, not the pageantry, of power
was the object of Mamæa’s manly ambition. She maintained an
absolute and lasting empire over the mind of her son, and in his
affection the mother could not brook a rival. Alexander, with her
consent, married the daughter of a patrician; but his respect for
his father-in-law, and love for the empress, were inconsistent
with the tenderness of interest of Mamæa. The patrician was
executed on the ready accusation of treason, and the wife of
Alexander driven with ignominy from the palace, and banished into
Africa.\footnotemark[67]

\footnotetext[64]{Metellus Numidicus, the censor, acknowledged to the
Roman people, in a public oration, that had kind nature allowed
us to exist without the help of women, we should be delivered
from a very troublesome companion; and he could recommend
matrimony only as the sacrifice of private pleasure to public
duty. Aulus Gellius, i. 6.}

\footnotetext[65]{Tacit. Annal. xiii. 5.}

\footnotetext[66]{Hist. August. p. 102, 107.}

\footnotetext[67]{Dion, l. lxxx. p. 1369. Herodian, l. vi. p. 206.
Hist. August. p. 131. Herodian represents the patrician as
innocent. The Augustian History, on the authority of Dexippus,
condemns him, as guilty of a conspiracy against the life of
Alexander. It is impossible to pronounce between them; but Dion
is an irreproachable witness of the jealousy and cruelty of Mamæa
towards the young empress, whose hard fate Alexander lamented,
but durst not oppose.}

Notwithstanding this act of jealous cruelty, as well as some
instances of avarice, with which Mamæa is charged, the general
tenor of her administration was equally for the benefit of her
son and of the empire. With the approbation of the senate, she
chose sixteen of the wisest and most virtuous senators as a
perpetual council of state, before whom every public business of
moment was debated and determined. The celebrated Ulpian, equally
distinguished by his knowledge of, and his respect for, the laws
of Rome, was at their head; and the prudent firmness of this
aristocracy restored order and authority to the government. As
soon as they had purged the city from foreign superstition and
luxury, the remains of the capricious tyranny of Elagabalus, they
applied themselves to remove his worthless creatures from every
department of the public administration, and to supply their
places with men of virtue and ability. Learning, and the love of
justice, became the only recommendations for civil offices;
valor, and the love of discipline, the only qualifications for
military employments.\footnotemark[68]

\footnotetext[68]{Herodian, l. vi. p. 203. Hist. August. p. 119. The
latter insinuates, that when any law was to be passed, the
council was assisted by a number of able lawyers and experienced
senators, whose opinions were separately given, and taken down in
writing.}

But the most important care of Mamæa and her wise counsellors,
was to form the character of the young emperor, on whose personal
qualities the happiness or misery of the Roman world must
ultimately depend. The fortunate soil assisted, and even
prevented, the hand of cultivation. An excellent understanding
soon convinced Alexander of the advantages of virtue, the
pleasure of knowledge, and the necessity of labor. A natural
mildness and moderation of temper preserved him from the assaults
of passion, and the allurements of vice. His unalterable regard
for his mother, and his esteem for the wise Ulpian, guarded his
unexperienced youth from the poison of flattery. 581

\footnotetext[581]{Alexander received into his chapel all the
religions which prevailed in the empire; he admitted Jesus
Christ, Abraham, Orpheus, Apollonius of Tyana, \&c. It was almost
certain that his mother Mamæa had instructed him in the morality
of Christianity. Historians in general agree in calling her a
Christian; there is reason to believe that she had begun to have
a taste for the principles of Christianity. (See Tillemont,
Alexander Severus) Gibbon has not noticed this circumstance; he
appears to have wished to lower the character of this empress; he
has throughout followed the narrative of Herodian, who, by the
acknowledgment of Capitolinus himself, detested Alexander.
Without believing the exaggerated praises of Lampridius, he ought
not to have followed the unjust severity of Herodian, and, above
all, not to have forgotten to say that the virtuous Alexander
Severus had insured to the Jews the preservation of their
privileges, and permitted the exercise of Christianity. Hist.
Aug. p. 121. The Christians had established their worship in a
public place, of which the victuallers (cauponarii) claimed, not
the property, but possession by custom. Alexander answered, that
it was better that the place should be used for the service of
God, in any form, than for victuallers.—G. I have scrupled to
omit this note, as it contains some points worthy of notice; but
it is very unjust to Gibbon, who mentions almost all the
circumstances, which he is accused of omitting, in another, and,
according to his plan, a better place, and, perhaps, in stronger
terms than M. Guizot. See Chap. xvi.— M.}

The simple journal of his ordinary occupations exhibits a
pleasing picture of an accomplished emperor,\footnotemark[69] and, with some
allowance for the difference of manners, might well deserve the
imitation of modern princes. Alexander rose early: the first
moments of the day were consecrated to private devotion, and his
domestic chapel was filled with the images of those heroes, who,
by improving or reforming human life, had deserved the grateful
reverence of posterity. But as he deemed the service of mankind
the most acceptable worship of the gods, the greatest part of his
morning hours was employed in his council, where he discussed
public affairs, and determined private causes, with a patience
and discretion above his years. The dryness of business was
relieved by the charms of literature; and a portion of time was
always set apart for his favorite studies of poetry, history, and
philosophy. The works of Virgil and Horace, the republics of
Plato and Cicero, formed his taste, enlarged his understanding,
and gave him the noblest ideas of man and government. The
exercises of the body succeeded to those of the mind; and
Alexander, who was tall, active, and robust, surpassed most of
his equals in the gymnastic arts. Refreshed by the use of the
bath and a slight dinner, he resumed, with new vigor, the
business of the day; and, till the hour of supper, the principal
meal of the Romans, he was attended by his secretaries, with whom
he read and answered the multitude of letters, memorials, and
petitions, that must have been addressed to the master of the
greatest part of the world. His table was served with the most
frugal simplicity, and whenever he was at liberty to consult his
own inclination, the company consisted of a few select friends,
men of learning and virtue, amongst whom Ulpian was constantly
invited. Their conversation was familiar and instructive; and the
pauses were occasionally enlivened by the recital of some
pleasing composition, which supplied the place of the dancers,
comedians, and even gladiators, so frequently summoned to the
tables of the rich and luxurious Romans.\footnotemark[70] The dress of
Alexander was plain and modest, his demeanor courteous and
affable: at the proper hours his palace was open to all his
subjects, but the voice of a crier was heard, as in the
Eleusinian mysteries, pronouncing the same salutary admonition:
“Let none enter these holy walls, unless he is conscious of a
pure and innocent mind.”\footnotemark[71]

\footnotetext[69]{See his life in the Augustan History. The
undistinguishing compiler has buried these interesting anecdotes
under a load of trivial unmeaning circumstances.}

\footnotetext[70]{See the 13th Satire of Juvenal.}

\footnotetext[71]{Hist. August. p. 119.}

Such a uniform tenor of life, which left not a moment for vice or
folly, is a better proof of the wisdom and justice of Alexander’s
government, than all the trifling details preserved in the
compilation of Lampridius. Since the accession of Commodus, the
Roman world had experienced, during the term of forty years, the
successive and various vices of four tyrants. From the death of
Elagabalus, it enjoyed an auspicious calm of thirteen years.\footnotemark[711]
The provinces, relieved from the oppressive taxes invented by
Caracalla and his pretended son, flourished in peace and
prosperity, under the administration of magistrates who were
convinced by experience that to deserve the love of the subjects
was their best and only method of obtaining the favor of their
sovereign. While some gentle restraints were imposed on the
innocent luxury of the Roman people, the price of provisions and
the interest of money, were reduced by the paternal care of
Alexander, whose prudent liberality, without distressing the
industrious, supplied the wants and amusements of the populace.
The dignity, the freedom, the authority of the senate was
restored; and every virtuous senator might approach the person of
the emperor without a fear and without a blush.

\footnotetext[711]{Wenck observes that Gibbon, enchanted with the
virtue of Alexander has heightened, particularly in this
sentence, its effect on the state of the world. His own account,
which follows, of the insurrections and foreign wars, is not in
harmony with this beautiful picture.—M.}

The name of Antoninus, ennobled by the virtues of Pius and
Marcus, had been communicated by adoption to the dissolute Verus,
and by descent to the cruel Commodus. It became the honorable
appellation of the sons of Severus, was bestowed on young
Diadumenianus, and at length prostituted to the infamy of the
high priest of Emesa. Alexander, though pressed by the studied,
and, perhaps, sincere importunity of the senate, nobly refused
the borrowed lustre of a name; whilst in his whole conduct he
labored to restore the glories and felicity of the age of the
genuine Antonines.\footnotemark[72]

\footnotetext[72]{See, in the Hist. August. p. 116, 117, the whole
contest between Alexander and the senate, extracted from the
journals of that assembly. It happened on the sixth of March,
probably of the year 223, when the Romans had enjoyed, almost a
twelvemonth, the blessings of his reign. Before the appellation
of Antoninus was offered him as a title of honor, the senate
waited to see whether Alexander would not assume it as a family
name.}

In the civil administration of Alexander, wisdom was enforced by
power, and the people, sensible of the public felicity, repaid
their benefactor with their love and gratitude. There still
remained a greater, a more necessary, but a more difficult
enterprise; the reformation of the military order, whose interest
and temper, confirmed by long impunity, rendered them impatient
of the restraints of discipline, and careless of the blessings of
public tranquillity. In the execution of his design, the emperor
affected to display his love, and to conceal his fear of the
army. The most rigid economy in every other branch of the
administration supplied a fund of gold and silver for the
ordinary pay and the extraordinary rewards of the troops. In
their marches he relaxed the severe obligation of carrying
seventeen days’ provision on their shoulders. Ample magazines
were formed along the public roads, and as soon as they entered
the enemy’s country, a numerous train of mules and camels waited
on their haughty laziness. As Alexander despaired of correcting
the luxury of his soldiers, he attempted, at least, to direct it
to objects of martial pomp and ornament, fine horses, splendid
armor, and shields enriched with silver and gold. He shared
whatever fatigues he was obliged to impose, visited, in person,
the sick and wounded, preserved an exact register of their
services and his own gratitude, and expressed on every occasion,
the warmest regard for a body of men, whose welfare, as he
affected to declare, was so closely connected with that of the
state.\footnotemark[73] By the most gentle arts he labored to inspire the
fierce multitude with a sense of duty, and to restore at least a
faint image of that discipline to which the Romans owed their
empire over so many other nations, as warlike and more powerful
than themselves. But his prudence was vain, his courage fatal,
and the attempt towards a reformation served only to inflame the
ills it was meant to cure.

\footnotetext[73]{It was a favorite saying of the emperor’s Se
milites magis servare, quam seipsum, quod salus publica in his
esset. Hist. Aug. p. 130.}

The Prætorian guards were attached to the youth of Alexander.
They loved him as a tender pupil, whom they had saved from a
tyrant’s fury, and placed on the Imperial throne. That amiable
prince was sensible of the obligation; but as his gratitude was
restrained within the limits of reason and justice, they soon
were more dissatisfied with the virtues of Alexander, than they
had ever been with the vices of Elagabalus. Their præfect, the
wise Ulpian, was the friend of the laws and of the people; he was
considered as the enemy of the soldiers, and to his pernicious
counsels every scheme of reformation was imputed. Some trifling
accident blew up their discontent into a furious mutiny; and the
civil war raged, during three days, in Rome, whilst the life of
that excellent minister was defended by the grateful people.
Terrified, at length, by the sight of some houses in flames, and
by the threats of a general conflagration, the people yielded
with a sigh, and left the virtuous but unfortunate Ulpian to his
fate. He was pursued into the Imperial palace, and massacred at
the feet of his master, who vainly strove to cover him with the
purple, and to obtain his pardon from the inexorable soldiers.\footnotemark[731]
Such was the deplorable weakness of government, that the
emperor was unable to revenge his murdered friend and his
insulted dignity, without stooping to the arts of patience and
dissimulation. Epagathus, the principal leader of the mutiny, was
removed from Rome, by the honorable employment of præfect of
Egypt: from that high rank he was gently degraded to the
government of Crete; and when at length, his popularity among the
guards was effaced by time and absence, Alexander ventured to
inflict the tardy but deserved punishment of his crimes.\footnotemark[74] Under
the reign of a just and virtuous prince, the tyranny of the army
threatened with instant death his most faithful ministers, who
were suspected of an intention to correct their intolerable
disorders. The historian Dion Cassius had commanded the Pannonian
legions with the spirit of ancient discipline. Their brethren of
Rome, embracing the common cause of military license, demanded
the head of the reformer. Alexander, however, instead of yielding
to their seditious clamors, showed a just sense of his merit and
services, by appointing him his colleague in the consulship, and
defraying from his own treasury the expense of that vain dignity:
but as was justly apprehended, that if the soldiers beheld him
with the ensigns of his office, they would revenge the insult in
his blood, the nominal first magistrate of the state retired, by
the emperor’s advice, from the city, and spent the greatest part
of his consulship at his villas in Campania.\footnotemark[75] \footnotemark[751]

\footnotetext[731]{Gibbon has confounded two events altogether
different— the quarrel of the people with the Prætorians, which
lasted three days, and the assassination of Ulpian by the latter.
Dion relates first the death of Ulpian, afterwards, reverting
back according to a manner which is usual with him, he says that
during the life of Ulpian, there had been a war of three days
between the Prætorians and the people. But Ulpian was not the
cause. Dion says, on the contrary, that it was occasioned by some
unimportant circumstance; whilst he assigns a weighty reason for
the murder of Ulpian, the judgment by which that Prætorian
præfect had condemned his predecessors, Chrestus and Flavian, to
death, whom the soldiers wished to revenge. Zosimus (l. 1, c.
xi.) attributes this sentence to Mamæra; but, even then, the
troops might have imputed it to Ulpian, who had reaped all the
advantage and was otherwise odious to them.—W.}

\footnotetext[74]{Though the author of the life of Alexander (Hist.
August. p. 182) mentions the sedition raised against Ulpian by
the soldiers, he conceals the catastrophe, as it might discover a
weakness in the administration of his hero. From this designed
omission, we may judge of the weight and candor of that author.}

\footnotetext[75]{For an account of Ulpian’s fate and his own danger,
see the mutilated conclusion of Dion’s History, l. lxxx. p.
1371.}

\footnotetext[751]{Dion possessed no estates in Campania, and was not
rich. He only says that the emperor advised him to reside, during
his consulate, in some place out of Rome; that he returned to
Rome after the end of his consulate, and had an interview with
the emperor in Campania. He asked and obtained leave to pass the
rest of his life in his native city, (Nice, in Bithynia:) it was
there that he finished his history, which closes with his second
consulship.—W.}

