\section{Part \thesection.}
\thispagestyle{simple}

Severus, who dreaded neither his arms nor his enchantments,
guarded himself from the only danger of secret conspiracy, by the
faithful attendance of six hundred chosen men, who never quitted
his person or their cuirasses, either by night or by day, during
the whole march. Advancing with a steady and rapid course, he
passed, without difficulty, the defiles of the Apennine, received
into his party the troops and ambassadors sent to retard his
progress, and made a short halt at Interamnia, about seventy
miles from Rome. His victory was already secure, but the despair
of the Prætorians might have rendered it bloody; and Severus had
the laudable ambition of ascending the throne without drawing the
sword.\footnotemark[35] His emissaries, dispersed in the capital, assured the
guards, that provided they would abandon their worthless prince,
and the perpetrators of the murder of Pertinax, to the justice of
the conqueror, he would no longer consider that melancholy event
as the act of the whole body. The faithless Prætorians, whose
resistance was supported only by sullen obstinacy, gladly
complied with the easy conditions, seized the greatest part of
the assassins, and signified to the senate, that they no longer
defended the cause of Julian. That assembly, convoked by the
consul, unanimously acknowledged Severus as lawful emperor,
decreed divine honors to Pertinax, and pronounced a sentence of
deposition and death against his unfortunate successor. Julian
was conducted into a private apartment of the baths of the
palace, and beheaded as a common criminal, after having
purchased, with an immense treasure, an anxious and precarious
reign of only sixty-six days.\footnotemark[36] The almost incredible expedition
of Severus, who, in so short a space of time, conducted a
numerous army from the banks of the Danube to those of the Tyber,
proves at once the plenty of provisions produced by agriculture
and commerce, the goodness of the roads, the discipline of the
legions, and the indolent, subdued temper of the provinces.\footnotemark[37]

\footnotetext[35]{Victor and Eutropius, viii. 17, mention a combat
near the Milvian bridge, the Ponte Molle, unknown to the better
and more ancient writers.}

\footnotetext[36]{Dion, l. lxxiii. p. 1240. Herodian, l. ii. p. 83.
Hist. August. p. 63.}

\footnotetext[37]{From these sixty-six days, we must first deduct
sixteen, as Pertinax was murdered on the 28th of March, and
Severus most probably elected on the 13th of April, (see Hist.
August. p. 65, and Tillemont, Hist. des Empereurs, tom. iii. p.
393, note 7.) We cannot allow less than ten days after his
election, to put a numerous army in motion. Forty days remain for
this rapid march; and as we may compute about eight hundred miles
from Rome to the neighborhood of Vienna, the army of Severus
marched twenty miles every day, without halt or intermission.}

The first cares of Severus were bestowed on two measures, the one
dictated by policy, the other by decency; the revenge, and the
honors, due to the memory of Pertinax. Before the new emperor
entered Rome, he issued his commands to the Prætorian guards,
directing them to wait his arrival on a large plain near the
city, without arms, but in the habits of ceremony, in which they
were accustomed to attend their sovereign. He was obeyed by those
haughty troops, whose contrition was the effect of their just
terrors. A chosen part of the Illyrian army encompassed them with
levelled spears. Incapable of flight or resistance, they expected
their fate in silent consternation. Severus mounted the tribunal,
sternly reproached them with perfidy and cowardice, dismissed
them with ignominy from the trust which they had betrayed,
despoiled them of their splendid ornaments, and banished them, on
pain of death, to the distance of a hundred miles from the
capital. During the transaction, another detachment had been sent
to seize their arms, occupy their camp, and prevent the hasty
consequences of their despair.\footnotemark[38]

\footnotetext[38]{Dion, l. lxxiv. p. 1241. Herodian, l. ii. p. 84.}

The funeral and consecration of Pertinax was next solemnized with
every circumstance of sad magnificence.\footnotemark[39] The senate, with a
melancholy pleasure, performed the last rites to that excellent
prince, whom they had loved, and still regretted. The concern of
his successor was probably less sincere; he esteemed the virtues
of Pertinax, but those virtues would forever have confined his
ambition to a private station. Severus pronounced his funeral
oration with studied eloquence, inward satisfaction, and
well-acted sorrow; and by this pious regard to his memory,
convinced the credulous multitude, that \textit{he alone} was worthy to
supply his place. Sensible, however, that arms, not ceremonies,
must assert his claim to the empire, he left Rome at the end of
thirty days, and without suffering himself to be elated by this
easy victory, prepared to encounter his more formidable rivals.

\footnotetext[39]{Dion, (l. lxxiv. p. 1244,) who assisted at the
ceremony as a senator, gives a most pompous description of it.}

The uncommon abilities and fortune of Severus have induced an
elegant historian to compare him with the first and greatest of
the Cæsars.\footnotemark[40] The parallel is, at least, imperfect. Where shall
we find, in the character of Severus, the commanding superiority
of soul, the generous clemency, and the various genius, which
could reconcile and unite the love of pleasure, the thirst of
knowledge, and the fire of ambition?\footnotemark[41] In one instance only,
they may be compared, with some degree of propriety, in the
celerity of their motions, and their civil victories. In less
than four years,\footnotemark[42] Severus subdued the riches of the East, and
the valor of the West. He vanquished two competitors of
reputation and ability, and defeated numerous armies, provided
with weapons and discipline equal to his own. In that age, the
art of fortification, and the principles of tactics, were well
understood by all the Roman generals; and the constant
superiority of Severus was that of an artist, who uses the same
instruments with more skill and industry than his rivals. I shall
not, however, enter into a minute narrative of these military
operations; but as the two civil wars against Niger and against
Albinus were almost the same in their conduct, event, and
consequences, I shall collect into one point of view the most
striking circumstances, tending to develop the character of the
conqueror and the state of the empire.

\footnotetext[40]{Herodian, l. iii. p. 112}

\footnotetext[41]{Though it is not, most assuredly, the intention of
Lucan to exalt the character of Cæsar, yet the idea he gives of
that hero, in the tenth book of the Pharsalia, where he describes
him, at the same time, making love to Cleopatra, sustaining a
siege against the power of Egypt, and conversing with the sages
of the country, is, in reality, the noblest panegyric. * Note:
Lord Byron wrote, no doubt, from a reminiscence of that
passage—“It is possible to be a very great man, and to be still
very inferior to Julius Cæsar, the most complete character, so
Lord Bacon thought, of all antiquity. Nature seems incapable of
such extraordinary combinations as composed his versatile
capacity, which was the wonder even of the Romans themselves. The
first general; the only triumphant politician; inferior to none
in point of eloquence; comparable to any in the attainments of
wisdom, in an age made up of the greatest commanders, statesmen,
orators, and philosophers, that ever appeared in the world; an
author who composed a perfect specimen of military annals in his
travelling carriage; at one time in a controversy with Cato, at
another writing a treatise on punuing, and collecting a set of
good sayings; fighting and making love at the same moment, and
willing to abandon both his empire and his mistress for a sight
of the fountains of the Nile. Such did Julius Cæsar appear to his
contemporaries, and to those of the subsequent ages who were the
most inclined to deplore and execrate his fatal genius.” Note 47
to Canto iv. of Childe Harold.—M.}

\footnotetext[42]{Reckoning from his election, April 13, 193, to the
death of Albinus, February 19, 197. See Tillemont’s Chronology.}

Falsehood and insincerity, unsuitable as they seem to the dignity
of public transactions, offend us with a less degrading idea of
meanness, than when they are found in the intercourse of private
life. In the latter, they discover a want of courage; in the
other, only a defect of power: and, as it is impossible for the
most able statesmen to subdue millions of followers and enemies
by their own personal strength, the world, under the name of
policy, seems to have granted them a very liberal indulgence of
craft and dissimulation. Yet the arts of Severus cannot be
justified by the most ample privileges of state reason. He
promised only to betray, he flattered only to ruin; and however
he might occasionally bind himself by oaths and treaties, his
conscience, obsequious to his interest, always released him from
the inconvenient obligation.\footnotemark[43]

\footnotetext[43]{Herodian, l. ii. p. 85.}

If his two competitors, reconciled by their common danger, had
advanced upon him without delay, perhaps Severus would have sunk
under their united effort. Had they even attacked him, at the
same time, with separate views and separate armies, the contest
might have been long and doubtful. But they fell, singly and
successively, an easy prey to the arts as well as arms of their
subtle enemy, lulled into security by the moderation of his
professions, and overwhelmed by the rapidity of his action. He
first marched against Niger, whose reputation and power he the
most dreaded: but he declined any hostile declarations,
suppressed the name of his antagonist, and only signified to the
senate and people his intention of regulating the eastern
provinces. In private, he spoke of Niger, his old friend and
intended successor,\footnotemark[44] with the most affectionate regard, and
highly applauded his generous design of revenging the murder of
Pertinax. To punish the vile usurper of the throne, was the duty
of every Roman general. To persevere in arms, and to resist a
lawful emperor, acknowledged by the senate, would alone render
him criminal.\footnotemark[45] The sons of Niger had fallen into his hands
among the children of the provincial governors, detained at Rome
as pledges for the loyalty of their parents.\footnotemark[46] As long as the
power of Niger inspired terror, or even respect, they were
educated with the most tender care, with the children of Severus
himself; but they were soon involved in their father’s ruin, and
removed first by exile, and afterwards by death, from the eye of
public compassion.\footnotemark[47]

\footnotetext[44]{Whilst Severus was very dangerously ill, it was
industriously given out, that he intended to appoint Niger and
Albinus his successors. As he could not be sincere with respect
to both, he might not be so with regard to either. Yet Severus
carried his hypocrisy so far, as to profess that intention in the
memoirs of his own life.}

\footnotetext[45]{Hist. August. p. 65.}

\footnotetext[46]{This practice, invented by Commodus, proved very
useful to Severus. He found at Rome the children of many of the
principal adherents of his rivals; and he employed them more than
once to intimidate, or seduce, the parents.}

\footnotetext[47]{Herodian, l. iii. p. 95. Hist. August. p. 67, 68.}

Whilst Severus was engaged in his eastern war, he had reason to
apprehend that the governor of Britain might pass the sea and the
Alps, occupy the vacant seat of empire, and oppose his return
with the authority of the senate and the forces of the West. The
ambiguous conduct of Albinus, in not assuming the Imperial title,
left room for negotiation. Forgetting, at once, his professions
of patriotism, and the jealousy of sovereign power, he accepted
the precarious rank of Cæsar, as a reward for his fatal
neutrality. Till the first contest was decided, Severus treated
the man, whom he had doomed to destruction, with every mark of
esteem and regard. Even in the letter, in which he announced his
victory over Niger, he styles Albinus the brother of his soul and
empire, sends him the affectionate salutations of his wife Julia,
and his young family, and entreats him to preserve the armies and
the republic faithful to their common interest. The messengers
charged with this letter were instructed to accost the Cæsar with
respect, to desire a private audience, and to plunge their
daggers into his heart.\footnotemark[48] The conspiracy was discovered, and the
too credulous Albinus, at length, passed over to the continent,
and prepared for an unequal contest with his rival, who rushed
upon him at the head of a veteran and victorious army.

\footnotetext[48]{Hist. August. p. 84. Spartianus has inserted this
curious letter at full length.}

The military labors of Severus seem inadequate to the importance
of his conquests. Two engagements,\footnotemark[481] the one near the
Hellespont, the other in the narrow defiles of Cilicia, decided
the fate of his Syrian competitor; and the troops of Europe
asserted their usual ascendant over the effeminate natives of
Asia.\footnotemark[49] The battle of Lyons, where one hundred and fifty
thousand Romans\footnotemark[50] were engaged, was equally fatal to Albinus.
The valor of the British army maintained, indeed, a sharp and
doubtful contest, with the hardy discipline of the Illyrian
legions. The fame and person of Severus appeared, during a few
moments, irrecoverably lost, till that warlike prince rallied his
fainting troops, and led them on to a decisive victory.\footnotemark[51] The
war was finished by that memorable day.\footnotemark[511]

\footnotetext[481]{There were three actions; one near Cyzicus, on the
Hellespont, one near Nice, in Bithynia, the third near the Issus,
in Cilicia, where Alexander conquered Darius. (Dion, lxiv. c. 6.
Herodian, iii. 2, 4.)—W Herodian represents the second battle as
of less importance than Dion—M.}

\footnotetext[49]{Consult the third book of Herodian, and the
seventy-fourth book of Dion Cassius.}

\footnotetext[50]{Dion, l. lxxv. p. 1260.}

\footnotetext[51]{Dion, l. lxxv. p. 1261. Herodian, l. iii. p. 110.
Hist. August. p. 68. The battle was fought in the plain of
Trevoux, three or four leagues from Lyons. See Tillemont, tom.
iii. p. 406, note 18.}

\footnotetext[511]{According to Herodian, it was his lieutenant Lætus
who led back the troops to the battle, and gained the day, which
Severus had almost lost. Dion also attributes to Lætus a great
share in the victory. Severus afterwards put him to death, either
from fear or jealousy.—W. and G. Wenck and M. Guizot have not
given the real statement of Herodian or of Dion. According to the
former, Lætus appeared with his own army entire, which he was
suspected of having designedly kept disengaged when the battle
was still doudtful, or rather after the rout of severus. Dion
says that he did not move till Severus had won the victory.—M.}

The civil wars of modern Europe have been distinguished, not only
by the fierce animosity, but likewise by the obstinate
perseverance, of the contending factions. They have generally
been justified by some principle, or, at least, colored by some
pretext, of religion, freedom, or loyalty. The leaders were
nobles of independent property and hereditary influence. The
troops fought like men interested in the decision of the quarrel;
and as military spirit and party zeal were strongly diffused
throughout the whole community, a vanquished chief was
immediately supplied with new adherents, eager to shed their
blood in the same cause. But the Romans, after the fall of the
republic, combated only for the choice of masters. Under the
standard of a popular candidate for empire, a few enlisted from
affection, some from fear, many from interest, none from
principle. The legions, uninflamed by party zeal, were allured
into civil war by liberal donatives, and still more liberal
promises. A defeat, by disabling the chief from the performance
of his engagements, dissolved the mercenary allegiance of his
followers, and left them to consult their own safety by a timely
desertion of an unsuccessful cause. It was of little moment to
the provinces, under whose name they were oppressed or governed;
they were driven by the impulsion of the present power, and as
soon as that power yielded to a superior force, they hastened to
implore the clemency of the conqueror, who, as he had an immense
debt to discharge, was obliged to sacrifice the most guilty
countries to the avarice of his soldiers. In the vast extent of
the Roman empire, there were few fortified cities capable of
protecting a routed army; nor was there any person, or family, or
order of men, whose natural interest, unsupported by the powers
of government, was capable of restoring the cause of a sinking
party.\footnotemark[52]

\footnotetext[52]{Montesquieu, Considerations sur la Grandeur et la
Decadence des Romains, c. xiii.}

Yet, in the contest between Niger and Severus, a single city
deserves an honorable exception. As Byzantium was one of the
greatest passages from Europe into Asia, it had been provided
with a strong garrison, and a fleet of five hundred vessels was
anchored in the harbor.\footnotemark[53] The impetuosity of Severus
disappointed this prudent scheme of defence; he left to his
generals the siege of Byzantium, forced the less guarded passage
of the Hellespont, and, impatient of a meaner enemy, pressed
forward to encounter his rival. Byzantium, attacked by a numerous
and increasing army, and afterwards by the whole naval power of
the empire, sustained a siege of three years, and remained
faithful to the name and memory of Niger. The citizens and
soldiers (we know not from what cause) were animated with equal
fury; several of the principal officers of Niger, who despaired
of, or who disdained, a pardon, had thrown themselves into this
last refuge: the fortifications were esteemed impregnable, and,
in the defence of the place, a celebrated engineer displayed all
the mechanic powers known to the ancients.\footnotemark[54] Byzantium, at
length, surrendered to famine. The magistrates and soldiers were
put to the sword, the walls demolished, the privileges
suppressed, and the destined capital of the East subsisted only
as an open village, subject to the insulting jurisdiction of
Perinthus. The historian Dion, who had admired the flourishing,
and lamented the desolate, state of Byzantium, accused the
revenge of Severus, for depriving the Roman people of the
strongest bulwark against the barbarians of Pontus and Asia\footnotemark[55]
The truth of this observation was but too well justified in the
succeeding age, when the Gothic fleets covered the Euxine, and
passed through the undefined Bosphorus into the centre of the
Mediterranean.

\footnotetext[53]{Most of these, as may be supposed, were small open
vessels; some, however, were galleys of two, and a few of three
ranks of oars.}

\footnotetext[54]{The engineer’s name was Priscus. His skill saved his
life, and he was taken into the service of the conqueror. For the
particular facts of the siege, consult Dion Cassius (l. lxxv. p.
1251) and Herodian, (l. iii. p. 95;) for the theory of it, the
fanciful chevalier de Folard may be looked into. See Polybe, tom.
i. p. 76.}

\footnotetext[55]{Notwithstanding the authority of Spartianus, and
some modern Greeks, we may be assured, from Dion and Herodian,
that Byzantium, many years after the death of Severus, lay in
ruins. There is no contradiction between the relation of Dion and
that of Spartianus and the modern Greeks. Dion does not say that
Severus destroyed Byzantium, but that he deprived it of its
franchises and privileges, stripped the inhabitants of their
property, razed the fortifications, and subjected the city to the
jurisdiction of Perinthus. Therefore, when Spartian, Suidas,
Cedrenus, say that Severus and his son Antoninus restored to
Byzantium its rights and franchises, ordered temples to be built,
\&c., this is easily reconciled with the relation of Dion. Perhaps
the latter mentioned it in some of the fragments of his history
which have been lost. As to Herodian, his expressions are
evidently exaggerated, and he has been guilty of so many
inaccuracies in the history of Severus, that we have a right to
suppose one in this passage.—G. from W Wenck and M. Guizot have
omitted to cite Zosimus, who mentions a particular portico built
by Severus, and called, apparently, by his name. Zosim. Hist. ii.
c. xxx. p. 151, 153, edit Heyne.—M.}

Both Niger and Albinus were discovered and put to death in their
flight from the field of battle. Their fate excited neither
surprise nor compassion. They had staked their lives against the
chance of empire, and suffered what they would have inflicted;
nor did Severus claim the arrogant superiority of suffering his
rivals to live in a private station. But his unforgiving temper,
stimulated by avarice, indulged a spirit of revenge, where there
was no room for apprehension. The most considerable of the
provincials, who, without any dislike to the fortunate candidate,
had obeyed the governor under whose authority they were
accidentally placed, were punished by death, exile, and
especially by the confiscation of their estates. Many cities of
the East were stripped of their ancient honors, and obliged to
pay, into the treasury of Severus, four times the amount of the
sums contributed by them for the service of Niger.\footnotemark[56]

\footnotetext[56]{Dion, l. lxxiv. p. 1250.}

Till the final decision of the war, the cruelty of Severus was,
in some measure, restrained by the uncertainty of the event, and
his pretended reverence for the senate. The head of Albinus,
accompanied with a menacing letter, announced to the Romans that
he was resolved to spare none of the adherents of his unfortunate
competitors. He was irritated by the just auspicion that he had
never possessed the affections of the senate, and he concealed
his old malevolence under the recent discovery of some
treasonable correspondences. Thirty-five senators, however,
accused of having favored the party of Albinus, he freely
pardoned, and, by his subsequent behavior, endeavored to convince
them, that he had forgotten, as well as forgiven, their supposed
offences. But, at the same time, he condemned forty-one\footnotemark[57] other
senators, whose names history has recorded; their wives,
children, and clients attended them in death,\footnotemark[571] and the noblest
provincials of Spain and Gaul were involved in the same ruin.\footnotemark[572]
Such rigid justice—for so he termed it—was, in the opinion of
Severus, the only conduct capable of insuring peace to the people
or stability to the prince; and he condescended slightly to
lament, that to be mild, it was necessary that he should first be
cruel.\footnotemark[58]

\footnotetext[57]{Dion, (l. lxxv. p. 1264;) only twenty-nine senators
are mentioned by him, but forty-one are named in the Augustan
History, p. 69, among whom were six of the name of Pescennius.
Herodian (l. iii. p. 115) speaks in general of the cruelties of
Severus.}

\footnotetext[571]{Wenck denies that there is any authority for this
massacre of the wives of the senators. He adds, that only the
children and relatives of Niger and Albinus were put to death.
This is true of the family of Albinus, whose bodies were thrown
into the Rhone; those of Niger, according to Lampridius, were
sent into exile, but afterwards put to death. Among the partisans
of Albinus who were put to death were many women of rank, multæ
fœminæ illustres. Lamprid. in Sever.—M.}

\footnotetext[572]{A new fragment of Dion describes the state of Rome
during this contest. All pretended to be on the side of Severus;
but their secret sentiments were often betrayed by a change of
countenance on the arrival of some sudden report. Some were
detected by overacting their loyalty, Mai. Fragm. Vatican. p. 227
Severus told the senate he would rather have their hearts than
their votes.—Ibid.—M.}

\footnotetext[58]{Aurelius Victor.}

The true interest of an absolute monarch generally coincides with
that of his people. Their numbers, their wealth, their order, and
their security, are the best and only foundations of his real
greatness; and were he totally devoid of virtue, prudence might
supply its place, and would dictate the same rule of conduct.
Severus considered the Roman empire as his property, and had no
sooner secured the possession, than he bestowed his care on the
cultivation and improvement of so valuable an acquisition.
Salutary laws, executed with inflexible firmness, soon corrected
most of the abuses with which, since the death of Marcus, every
part of the government had been infected. In the administration
of justice, the judgments of the emperor were characterized by
attention, discernment, and impartiality; and whenever he
deviated from the strict line of equity, it was generally in
favor of the poor and oppressed; not so much indeed from any
sense of humanity, as from the natural propensity of a despot to
humble the pride of greatness, and to sink all his subjects to
the same common level of absolute dependence. His expensive taste
for building, magnificent shows, and above all a constant and
liberal distribution of corn and provisions, were the surest
means of captivating the affection of the Roman people.\footnotemark[59] The
misfortunes of civil discord were obliterated. The calm of peace
and prosperity was once more experienced in the provinces; and
many cities, restored by the munificence of Severus, assumed the
title of his colonies, and attested by public monuments their
gratitude and felicity.\footnotemark[60] The fame of the Roman arms was revived
by that warlike and successful emperor,\footnotemark[61] and he boasted, with a
just pride, that, having received the empire oppressed with
foreign and domestic wars, he left it established in profound,
universal, and honorable peace.\footnotemark[62]

\footnotetext[59]{Dion, l. lxxvi. p. 1272. Hist. August. p. 67.
Severus celebrated the secular games with extraordinary
magnificence, and he left in the public granaries a provision of
corn for seven years, at the rate of 75,000 modii, or about 2500
quarters per day. I am persuaded that the granaries of Severus
were supplied for a long term, but I am not less persuaded, that
policy on one hand, and admiration on the other, magnified the
hoard far beyond its true contents.}

\footnotetext[60]{See Spanheim’s treatise of ancient medals, the
inscriptions, and our learned travellers Spon and Wheeler, Shaw,
Pocock, \&c, who, in Africa, Greece, and Asia, have found more
monuments of Severus than of any other Roman emperor whatsoever.}

\footnotetext[61]{He carried his victorious arms to Seleucia and
Ctesiphon, the capitals of the Parthian monarchy. I shall have
occasion to mention this war in its proper place.}

\footnotetext[62]{Etiam in Britannis, was his own just and emphatic
expression Hist. August. 73.}

Although the wounds of civil war appeared completely healed, its
mortal poison still lurked in the vitals of the constitution.
Severus possessed a considerable share of vigor and ability; but
the daring soul of the first Cæsar, or the deep policy of
Augustus, were scarcely equal to the task of curbing the
insolence of the victorious legions. By gratitude, by misguided
policy, by seeming necessity, Severus was reduced to relax the
nerves of discipline.\footnotemark[63] The vanity of his soldiers was flattered
with the honor of wearing gold rings; their ease was indulged in
the permission of living with their wives in the idleness of
quarters. He increased their pay beyond the example of former
times, and taught them to expect, and soon to claim,
extraordinary donatives on every public occasion of danger or
festivity. Elated by success, enervated by luxury, and raised
above the level of subjects by their dangerous privileges,\footnotemark[64]
they soon became incapable of military fatigue, oppressive to the
country, and impatient of a just subordination. Their officers
asserted the superiority of rank by a more profuse and elegant
luxury. There is still extant a letter of Severus, lamenting the
licentious stage of the army,\footnotemark[641] and exhorting one of his
generals to begin the necessary reformation from the tribunes
themselves; since, as he justly observes, the officer who has
forfeited the esteem, will never command the obedience, of his
soldiers.\footnotemark[65] Had the emperor pursued the train of reflection, he
would have discovered, that the primary cause of this general
corruption might be ascribed, not indeed to the example, but to
the pernicious indulgence, however, of the commander-in-chief.

\footnotetext[63]{Herodian, l. iii. p. 115. Hist. August. p. 68.}

\footnotetext[64]{Upon the insolence and privileges of the soldier,
the 16th satire, falsely ascribed to Juvenal, may be consulted;
the style and circumstances of it would induce me to believe,
that it was composed under the reign of Severus, or that of his
son.}

\footnotetext[641]{Not of the army, but of the troops in Gaul. The
contents of this letter seem to prove that Severus was really
anxious to restore discipline Herodian is the only historian who
accuses him of being the first cause of its relaxation.—G. from W
Spartian mentions his increase of the pays.—M.}

\footnotetext[65]{Hist. August. p. 73.}

The Prætorians, who murdered their emperor and sold the empire,
had received the just punishment of their treason; but the
necessary, though dangerous, institution of guards was soon
restored on a new model by Severus, and increased to four times
the ancient number.\footnotemark[66] Formerly these troops had been recruited
in Italy; and as the adjacent provinces gradually imbibed the
softer manners of Rome, the levies were extended to Macedonia,
Noricum, and Spain. In the room of these elegant troops, better
adapted to the pomp of courts than to the uses of war, it was
established by Severus, that from all the legions of the
frontiers, the soldiers most distinguished for strength, valor,
and fidelity, should be occasionally draughted; and promoted, as
an honor and reward, into the more eligible service of the
guards.\footnotemark[67] By this new institution, the Italian youth were
diverted from the exercise of arms, and the capital was terrified
by the strange aspect and manners of a multitude of barbarians.
But Severus flattered himself, that the legions would consider
these chosen Prætorians as the representatives of the whole
military order; and that the present aid of fifty thousand men,
superior in arms and appointments to any force that could be
brought into the field against them, would forever crush the
hopes of rebellion, and secure the empire to himself and his
posterity.

\footnotetext[66]{Herodian, l. iii. p. 131.}

\footnotetext[67]{Dion, l. lxxiv. p. 1243.}

The command of these favored and formidable troops soon became
the first office of the empire. As the government degenerated
into military despotism, the Prætorian Præfect, who in his origin
had been a simple captain of the guards,\footnotemark[671] was placed not only
at the head of the army, but of the finances, and even of the
law. In every department of administration, he represented the
person, and exercised the authority, of the emperor. The first
præfect who enjoyed and abused this immense power was Plautianus,
the favorite minister of Severus. His reign lasted above ten
years, till the marriage of his daughter with the eldest son of
the emperor, which seemed to assure his fortune, proved the
occasion of his ruin.\footnotemark[68] The animosities of the palace, by
irritating the ambition and alarming the fears of Plautianus,\footnotemark[681]
threatened to produce a revolution, and obliged the emperor, who
still loved him, to consent with reluctance to his death.\footnotemark[69]
After the fall of Plautianus, an eminent lawyer, the celebrated
Papinian, was appointed to execute the motley office of Prætorian
Præfect.

\footnotetext[671]{The Prætorian Præfect had never been a simple
captain of the guards; from the first creation of this office,
under Augustus, it possessed great power. That emperor,
therefore, decreed that there should be always two Prætorian
Præfects, who could only be taken from the equestrian order
Tiberius first departed from the former clause of this edict;
Alexander Severus violated the second by naming senators
præfects. It appears that it was under Commodus that the
Prætorian Præfects obtained the province of civil jurisdiction.
It extended only to Italy, with the exception of Rome and its
district, which was governed by the Præfectus urbi. As to the
control of the finances, and the levying of taxes, it was not
intrusted to them till after the great change that Constantine I.
made in the organization of the empire at least, I know no
passage which assigns it to them before that time; and
Drakenborch, who has treated this question in his Dissertation de
official præfectorum prætorio, vi., does not quote one.—W.}

\footnotetext[68]{One of his most daring and wanton acts of power,
was the castration of a hundred free Romans, some of them married
men, and even fathers of families; merely that his daughter, on
her marriage with the young emperor, might be attended by a train
of eunuchs worthy of an eastern queen. Dion, l. lxxvi. p. 1271.}

\footnotetext[681]{Plautianus was compatriot, relative, and the old
friend, of Severus; he had so completely shut up all access to
the emperor, that the latter was ignorant how far he abused his
powers: at length, being informed of it, he began to limit his
authority. The marriage of Plautilla with Caracalla was
unfortunate; and the prince who had been forced to consent to it,
menaced the father and the daughter with death when he should
come to the throne. It was feared, after that, that Plautianus
would avail himself of the power which he still possessed,
against the Imperial family; and Severus caused him to be
assassinated in his presence, upon the pretext of a conspiracy,
which Dion considers fictitious.—W. This note is not, perhaps,
very necessary and does not contain the whole facts. Dion
considers the conspiracy the invention of Caracalla, by whose
command, almost by whose hand, Plautianus was slain in the
presence of Severus.—M.}

\footnotetext[69]{Dion, l. lxxvi. p. 1274. Herodian, l. iii. p. 122,
129. The grammarian of Alexander seems, as is not unusual, much
better acquainted with this mysterious transaction, and more
assured of the guilt of Plautianus than the Roman senator
ventures to be.}

Till the reign of Severus, the virtue and even the good sense of
the emperors had been distinguished by their zeal or affected
reverence for the senate, and by a tender regard to the nice
frame of civil policy instituted by Augustus. But the youth of
Severus had been trained in the implicit obedience of camps, and
his riper years spent in the despotism of military command. His
haughty and inflexible spirit could not discover, or would not
acknowledge, the advantage of preserving an intermediate power,
however imaginary, between the emperor and the army. He disdained
to profess himself the servant of an assembly that detested his
person and trembled at his frown; he issued his commands, where
his requests would have proved as effectual; assumed the conduct
and style of a sovereign and a conqueror, and exercised, without
disguise, the whole legislative, as well as the executive power.

The victory over the senate was easy and inglorious. Every eye
and every passion were directed to the supreme magistrate, who
possessed the arms and treasure of the state; whilst the senate,
neither elected by the people, nor guarded by military force, nor
animated by public spirit, rested its declining authority on the
frail and crumbling basis of ancient opinion. The fine theory of
a republic insensibly vanished, and made way for the more natural
and substantial feelings of monarchy. As the freedom and honors
of Rome were successively communicated to the provinces, in which
the old government had been either unknown, or was remembered
with abhorrence, the tradition of republican maxims was gradually
obliterated. The Greek historians of the age of the Antonines\footnotemark[70]
observe, with a malicious pleasure, that although the sovereign
of Rome, in compliance with an obsolete prejudice, abstained from
the name of king, he possessed the full measure of regal power.
In the reign of Severus, the senate was filled with polished and
eloquent slaves from the eastern provinces, who justified
personal flattery by speculative principles of servitude. These
new advocates of prerogative were heard with pleasure by the
court, and with patience by the people, when they inculcated the
duty of passive obedience, and descanted on the inevitable
mischiefs of freedom. The lawyers and historians concurred in
teaching, that the Imperial authority was held, not by the
delegated commission, but by the irrevocable resignation of the
senate; that the emperor was freed from the restraint of civil
laws, could command by his arbitrary will the lives and fortunes
of his subjects, and might dispose of the empire as of his
private patrimony.\footnotemark[71] The most eminent of the civil lawyers, and
particularly Papinian, Paulus, and Ulpian, flourished under the
house of Severus; and the Roman jurisprudence, having closely
united itself with the system of monarchy, was supposed to have
attained its full majority and perfection.

\footnotetext[70]{Appian in Proœm.}

\footnotetext[71]{Dion Cassius seems to have written with no other
view than to form these opinions into an historical system. The
Pandea’s will how how assiduously the lawyers, on their side,
laboree in the cause of prerogative.}

The contemporaries of Severus in the enjoyment of the peace and
glory of his reign, forgave the cruelties by which it had been
introduced. Posterity, who experienced the fatal effects of his
maxims and example, justly considered him as the principal author
of the decline of the Roman empire.

