\section{Part \thesection.}
\thispagestyle{simple}

Domestic peace and union were the natural consequences of the
moderate and comprehensive policy embraced by the Romans. If we
turn our eyes towards the monarchies of Asia, we shall behold
despotism in the centre, and weakness in the extremities; the
collection of the revenue, or the administration of justice,
enforced by the presence of an army; hostile barbarians
established in the heart of the country, hereditary satraps
usurping the dominion of the provinces, and subjects inclined to
rebellion, though incapable of freedom. But the obedience of the
Roman world was uniform, voluntary, and permanent. The vanquished
nations, blended into one great people, resigned the hope, nay,
even the wish, of resuming their independence, and scarcely
considered their own existence as distinct from the existence of
Rome. The established authority of the emperors pervaded without
an effort the wide extent of their dominions, and was exercised
with the same facility on the banks of the Thames, or of the
Nile, as on those of the Tyber. The legions were destined to
serve against the public enemy, and the civil magistrate seldom
required the aid of a military force.\footnotemark[63] In this state of general
security, the leisure, as well as opulence, both of the prince
and people, were devoted to improve and to adorn the Roman
empire.

\footnotetext[63]{Joseph. de Bell. Judaico, l. ii. c. 16. The oration
of Agrippa, or rather of the historian, is a fine picture of the
Roman empire.}

Among the innumerable monuments of architecture constructed by
the Romans, how many have escaped the notice of history, how few
have resisted the ravages of time and barbarism! And yet, even
the majestic ruins that are still scattered over Italy and the
provinces, would be sufficient to prove that those countries were
once the seat of a polite and powerful empire. Their greatness
alone, or their beauty, might deserve our attention: but they are
rendered more interesting, by two important circumstances, which
connect the agreeable history of the arts with the more useful
history of human manners. Many of those works were erected at
private expense, and almost all were intended for public benefit.

It is natural to suppose that the greatest number, as well as the
most considerable of the Roman edifices, were raised by the
emperors, who possessed so unbounded a command both of men and
money. Augustus was accustomed to boast that he had found his
capital of brick, and that he had left it of marble.\footnotemark[64] The
strict economy of Vespasian was the source of his magnificence.
The works of Trajan bear the stamp of his genius. The public
monuments with which Hadrian adorned every province of the
empire, were executed not only by his orders, but under his
immediate inspection. He was himself an artist; and he loved the
arts, as they conduced to the glory of the monarch. They were
encouraged by the Antonines, as they contributed to the happiness
of the people. But if the emperors were the first, they were not
the only architects of their dominions. Their example was
universally imitated by their principal subjects, who were not
afraid of declaring to the world that they had spirit to
conceive, and wealth to accomplish, the noblest undertakings.
Scarcely had the proud structure of the Coliseum been dedicated
at Rome, before the edifices, of a smaller scale indeed, but of
the same design and materials, were erected for the use, and at
the expense, of the cities of Capua and Verona.\footnotemark[65] The
inscription of the stupendous bridge of Alcantara attests that it
was thrown over the Tagus by the contribution of a few Lusitanian
communities. When Pliny was intrusted with the government of
Bithynia and Pontus, provinces by no means the richest or most
considerable of the empire, he found the cities within his
jurisdiction striving with each other in every useful and
ornamental work, that might deserve the curiosity of strangers,
or the gratitude of their citizens. It was the duty of the
proconsul to supply their deficiencies, to direct their taste,
and sometimes to moderate their emulation.\footnotemark[66] The opulent
senators of Rome and the provinces esteemed it an honor, and
almost an obligation, to adorn the splendor of their age and
country; and the influence of fashion very frequently supplied
the want of taste or generosity. Among a crowd of these private
benefactors, we may select Herodes Atticus, an Athenian citizen,
who lived in the age of the Antonines. Whatever might be the
motive of his conduct, his magnificence would have been worthy of
the greatest kings.

\footnotetext[64]{Sueton. in August. c. 28. Augustus built in Rome
the temple and forum of Mars the Avenger; the temple of Jupiter
Tonans in the Capitol; that of Apollo Palatine, with public
libraries; the portico and basilica of Caius and Lucius; the
porticos of Livia and Octavia; and the theatre of Marcellus. The
example of the sovereign was imitated by his ministers and
generals; and his friend Agrippa left behind him the immortal
monument of the Pantheon.}

\footnotetext[65]{See Maffei, Veroni Illustrata, l. iv. p. 68.}

\footnotetext[66]{Footnote 66: See the xth book of Pliny’s Epistles.
He mentions the following works carried on at the expense of the
cities. At Nicomedia, a new forum, an aqueduct, and a canal, left
unfinished by a king; at Nice, a gymnasium, and a theatre, which
had already cost near ninety thousand pounds; baths at Prusa and
Claudiopolis, and an aqueduct of sixteen miles in length for the
use of Sinope.}

The family of Herod, at least after it had been favored by
fortune, was lineally descended from Cimon and Miltiades, Theseus
and Cecrops, Æacus and Jupiter. But the posterity of so many gods
and heroes was fallen into the most abject state. His grandfather
had suffered by the hands of justice, and Julius Atticus, his
father, must have ended his life in poverty and contempt, had he
not discovered an immense treasure buried under an old house, the
last remains of his patrimony. According to the rigor of the law,
the emperor might have asserted his claim, and the prudent
Atticus prevented, by a frank confession, the officiousness of
informers. But the equitable Nerva, who then filled the throne,
refused to accept any part of it, and commanded him to use,
without scruple, the present of fortune. The cautious Athenian
still insisted, that the treasure was too considerable for a
subject, and that he knew not how to \textit{use it. Abuse it then},
replied the monarch, with a good-natured peevishness; for it is
your own.\footnotemark[67] Many will be of opinion, that Atticus literally
obeyed the emperor’s last instructions; since he expended the
greatest part of his fortune, which was much increased by an
advantageous marriage, in the service of the public. He had
obtained for his son Herod the prefecture of the free cities of
Asia; and the young magistrate, observing that the town of Troas
was indifferently supplied with water, obtained from the
munificence of Hadrian three hundred myriads of drachms, (about a
hundred thousand pounds,) for the construction of a new aqueduct.
But in the execution of the work, the charge amounted to more
than double the estimate, and the officers of the revenue began
to murmur, till the generous Atticus silenced their complaints,
by requesting that he might be permitted to take upon himself the
whole additional expense.\footnotemark[68]

\footnotetext[67]{Hadrian afterwards made a very equitable
regulation, which divided all treasure-trove between the right of
property and that of discovery. Hist. August. p. 9.}

\footnotetext[68]{Philostrat. in Vit. Sophist. l. ii. p. 548.}

The ablest preceptors of Greece and Asia had been invited by
liberal rewards to direct the education of young Herod. Their
pupil soon became a celebrated orator, according to the useless
rhetoric of that age, which, confining itself to the schools,
disdained to visit either the Forum or the Senate.

He was honored with the consulship at Rome: but the greatest part
of his life was spent in a philosophic retirement at Athens, and
his adjacent villas; perpetually surrounded by sophists, who
acknowledged, without reluctance, the superiority of a rich and
generous rival.\footnotemark[69] The monuments of his genius have perished;
some considerable ruins still preserve the fame of his taste and
munificence: modern travellers have measured the remains of the
stadium which he constructed at Athens. It was six hundred feet
in length, built entirely of white marble, capable of admitting
the whole body of the people, and finished in four years, whilst
Herod was president of the Athenian games. To the memory of his
wife Regilla he dedicated a theatre, scarcely to be paralleled in
the empire: no wood except cedar, very curiously carved, was
employed in any part of the building. The Odeum, 691 designed by
Pericles for musical performances, and the rehearsal of new
tragedies, had been a trophy of the victory of the arts over
barbaric greatness; as the timbers employed in the construction
consisted chiefly of the masts of the Persian vessels.
Notwithstanding the repairs bestowed on that ancient edifice by a
king of Cappadocia, it was again fallen to decay. Herod restored
its ancient beauty and magnificence. Nor was the liberality of
that illustrious citizen confined to the walls of Athens. The
most splendid ornaments bestowed on the temple of Neptune in the
Isthmus, a theatre at Corinth, a stadium at Delphi, a bath at
Thermopylæ, and an aqueduct at Canusium in Italy, were
insufficient to exhaust his treasures. The people of Epirus,
Thessaly, Eubœa, Bœotia, and Peloponnesus, experienced his
favors; and many inscriptions of the cities of Greece and Asia
gratefully style Herodes Atticus their patron and benefactor.\footnotemark[70]

\footnotetext[69]{Aulus Gellius, in Noct. Attic. i. 2, ix. 2, xviii.
10, xix. 12. Phil ostrat. p. 564.}

\footnotetext[691]{The Odeum served for the rehearsal of new comedies
as well as tragedies; they were read or repeated, before
representation, without music or decorations, \&c. No piece could
be represented in the theatre if it had not been previously
approved by judges for this purpose. The king of Cappadocia who
restored the Odeum, which had been burnt by Sylla, was
Araobarzanes. See Martini, Dissertation on the Odeons of the
Ancients, Leipsic. 1767, p. 10—91.—W.}

\footnotetext[70]{See Philostrat. l. ii. p. 548, 560. Pausanias, l.
i. and vii. 10. The life of Herodes, in the xxxth volume of the
Memoirs of the Academy of Inscriptions.}

In the commonwealths of Athens and Rome, the modest simplicity of
private houses announced the equal condition of freedom; whilst
the sovereignty of the people was represented in the majestic
edifices designed to the public use;\footnotemark[71] nor was this republican
spirit totally extinguished by the introduction of wealth and
monarchy. It was in works of national honor and benefit, that the
most virtuous of the emperors affected to display their
magnificence. The golden palace of Nero excited a just
indignation, but the vast extent of ground which had been usurped
by his selfish luxury was more nobly filled under the succeeding
reigns by the Coliseum, the baths of Titus, the Claudian portico,
and the temples dedicated to the goddess of Peace, and to the
genius of Rome.\footnotemark[72] These monuments of architecture, the property
of the Roman people, were adorned with the most beautiful
productions of Grecian painting and sculpture; and in the temple
of Peace, a very curious library was open to the curiosity of the
learned.\footnotemark[721] At a small distance from thence was situated the
Forum of Trajan. It was surrounded by a lofty portico, in the
form of a quadrangle, into which four triumphal arches opened a
noble and spacious entrance: in the centre arose a column of
marble, whose height, of one hundred and ten feet, denoted the
elevation of the hill that had been cut away. This column, which
still subsists in its ancient beauty, exhibited an exact
representation of the Dacian victories of its founder. The
veteran soldier contemplated the story of his own campaigns, and
by an easy illusion of national vanity, the peaceful citizen
associated himself to the honors of the triumph. All the other
quarters of the capital, and all the provinces of the empire,
were embellished by the same liberal spirit of public
magnificence, and were filled with amphitheatres, theatres,
temples, porticoes, triumphal arches, baths and aqueducts, all
variously conducive to the health, the devotion, and the
pleasures of the meanest citizen. The last mentioned of those
edifices deserve our peculiar attention. The boldness of the
enterprise, the solidity of the execution, and the uses to which
they were subservient, rank the aqueducts among the noblest
monuments of Roman genius and power. The aqueducts of the capital
claim a just preeminence; but the curious traveller, who, without
the light of history, should examine those of Spoleto, of Metz,
or of Segovia, would very naturally conclude that those
provincial towns had formerly been the residence of some potent
monarch. The solitudes of Asia and Africa were once covered with
flourishing cities, whose populousness, and even whose existence,
was derived from such artificial supplies of a perennial stream
of fresh water.\footnotemark[73]

\footnotetext[71]{It is particularly remarked of Athens by
Dicæarchus, de Statu Græciæ, p. 8, inter Geographos Minores,
edit. Hudson.}

\footnotetext[72]{Donatus de Roma Vetere, l. iii. c. 4, 5, 6. Nardini
Roma Antica, l. iii. 11, 12, 13, and a Ms. description of ancient
Rome, by Bernardus Oricellarius, or Rucellai, of which I obtained
a copy from the library of the Canon Ricardi at Florence. Two
celebrated pictures of Timanthes and of Protogenes are mentioned
by Pliny, as in the Temple of Peace; and the Laocoon was found in
the baths of Titus.}

\footnotetext[721]{The Emperor Vespasian, who had caused the Temple
of Peace to be built, transported to it the greatest part of the
pictures, statues, and other works of art which had escaped the
civil tumults. It was there that every day the artists and the
learned of Rome assembled; and it is on the site of this temple
that a multitude of antiques have been dug up. See notes of
Reimar on Dion Cassius, lxvi. c. 15, p. 1083.—W.}

\footnotetext[73]{Montfaucon l’Antiquite Expliquee, tom. iv. p. 2, l.
i. c. 9. Fabretti has composed a very learned treatise on the
aqueducts of Rome.}

We have computed the inhabitants, and contemplated the public
works, of the Roman empire. The observation of the number and
greatness of its cities will serve to confirm the former, and to
multiply the latter. It may not be unpleasing to collect a few
scattered instances relative to that subject without forgetting,
however, that from the vanity of nations and the poverty of
language, the vague appellation of city has been indifferently
bestowed on Rome and upon Laurentum.

I. \textit{Ancient} Italy is said to have contained eleven hundred and
ninety-seven cities; and for whatsoever æra of antiquity the
expression might be intended,\footnotemark[74] there is not any reason to
believe the country less populous in the age of the Antonines,
than in that of Romulus. The petty states of Latium were
contained within the metropolis of the empire, by whose superior
influence they had been attracted.\footnotemark[741] Those parts of Italy which
have so long languished under the lazy tyranny of priests and
viceroys, had been afflicted only by the more tolerable
calamities of war; and the first symptoms of decay which \textit{they}
experienced, were amply compensated by the rapid improvements of
the Cisalpine Gaul. The splendor of Verona may be traced in its
remains: yet Verona was less celebrated than Aquileia or Padua,
Milan or Ravenna. II. The spirit of improvement had passed the
Alps, and been felt even in the woods of Britain, which were
gradually cleared away to open a free space for convenient and
elegant habitations. York was the seat of government; London was
already enriched by commerce; and Bath was celebrated for the
salutary effects of its medicinal waters. Gaul could boast of her
twelve hundred cities;\footnotemark[75] and though, in the northern parts, many
of them, without excepting Paris itself, were little more than
the rude and imperfect townships of a rising people, the southern
provinces imitated the wealth and elegance of Italy.\footnotemark[76] Many were
the cities of Gaul, Marseilles, Arles, Nismes, Narbonne,
Thoulouse, Bourdeaux, Autun, Vienna, Lyons, Langres, and Treves,
whose ancient condition might sustain an equal, and perhaps
advantageous comparison with their present state. With regard to
Spain, that country flourished as a province, and has declined as
a kingdom. Exhausted by the abuse of her strength, by America,
and by superstition, her pride might possibly be confounded, if
we required such a list of three hundred and sixty cities, as
Pliny has exhibited under the reign of Vespasian.\footnotemark[77] III. Three
hundred African cities had once acknowledged the authority of
Carthage,\footnotemark[78] nor is it likely that their numbers diminished under
the administration of the emperors: Carthage itself rose with new
splendor from its ashes; and that capital, as well as Capua and
Corinth, soon recovered all the advantages which can be separated
from independent sovereignty. IV. The provinces of the East
present the contrast of Roman magnificence with Turkish
barbarism. The ruins of antiquity scattered over uncultivated
fields, and ascribed, by ignorance, to the power of magic,
scarcely afford a shelter to the oppressed peasant or wandering
Arab. Under the reign of the Cæsars, the proper Asia alone
contained five hundred populous cities,\footnotemark[79] enriched with all the
gifts of nature, and adorned with all the refinements of art.
Eleven cities of Asia had once disputed the honor of dedicating a
temple of Tiberius, and their respective merits were examined by
the senate.\footnotemark[80] Four of them were immediately rejected as unequal
to the burden; and among these was Laodicea, whose splendor is
still displayed in its ruins.\footnotemark[81] Laodicea collected a very
considerable revenue from its flocks of sheep, celebrated for the
fineness of their wool, and had received, a little before the
contest, a legacy of above four hundred thousand pounds by the
testament of a generous citizen.\footnotemark[82] If such was the poverty of
Laodicea, what must have been the wealth of those cities, whose
claim appeared preferable, and particularly of Pergamus, of
Smyrna, and of Ephesus, who so long disputed with each other the
titular primacy of Asia?\footnotemark[83] The capitals of Syria and Egypt held
a still superior rank in the empire; Antioch and Alexandria
looked down with disdain on a crowd of dependent cities,\footnotemark[84] and
yielded, with reluctance, to the majesty of Rome itself.

\footnotetext[74]{Ælian. Hist. Var. lib. ix. c. 16. He lived in the
time of Alexander Severus. See Fabricius, Biblioth. Græca, l. iv.
c. 21.}

\footnotetext[741]{This may in some degree account for the difficulty
started by Livy, as to the incredibly numerous armies raised by
the small states around Rome where, in his time, a scanty stock
of free soldiers among a larger population of Roman slaves broke
the solitude. Vix seminario exiguo militum relicto servitia
Romana ab solitudine vindicant, Liv. vi. vii. Compare Appian Bel
Civ. i. 7.—M. subst. for G.}

\footnotetext[75]{Joseph. de Bell. Jud. ii. 16. The number, however,
is mentioned, and should be received with a degree of latitude.
Note: Without doubt no reliance can be placed on this passage of
Josephus. The historian makes Agrippa give advice to the Jews, as
to the power of the Romans; and the speech is full of declamation
which can furnish no conclusions to history. While enumerating
the nations subject to the Romans, he speaks of the Gauls as
submitting to 1200 soldiers, (which is false, as there were eight
legions in Gaul, Tac. iv. 5,) while there are nearly twelve
hundred cities.—G. Josephus (infra) places these eight legions on
the Rhine, as Tacitus does.—M.}

\footnotetext[76]{Plin. Hist. Natur. iii. 5.}

\footnotetext[77]{Plin. Hist. Natur. iii. 3, 4, iv. 35. The list
seems authentic and accurate; the division of the provinces, and
the different condition of the cities, are minutely
distinguished.}

\footnotetext[78]{Strabon. Geograph. l. xvii. p. 1189.}

\footnotetext[79]{Joseph. de Bell. Jud. ii. 16. Philostrat. in Vit.
Sophist. l. ii. p. 548, edit. Olear.}

\footnotetext[80]{Tacit. Annal. iv. 55. I have taken some pains in
consulting and comparing modern travellers, with regard to the
fate of those eleven cities of Asia. Seven or eight are totally
destroyed: Hypæpe, Tralles, Laodicea, Hium, Halicarnassus,
Miletus, Ephesus, and we may add Sardes. Of the remaining three,
Pergamus is a straggling village of two or three thousand
inhabitants; Magnesia, under the name of Guzelhissar, a town of
some consequence; and Smyrna, a great city, peopled by a hundred
thousand souls. But even at Smyrna, while the Franks have
maintained a commerce, the Turks have ruined the arts.}

\footnotetext[81]{See a very exact and pleasing description of the
ruins of Laodicea, in Chandler’s Travels through Asia Minor, p.
225, \&c.}

\footnotetext[82]{Strabo, l. xii. p. 866. He had studied at Tralles.}

\footnotetext[83]{See a Dissertation of M. de Boze, Mem. de
l’Academie, tom. xviii. Aristides pronounced an oration, which is
still extant, to recommend concord to the rival cities.}

\footnotetext[84]{The inhabitants of Egypt, exclusive of Alexandria,
amounted to seven millions and a half, (Joseph. de Bell. Jud. ii.
16.) Under the military government of the Mamelukes, Syria was
supposed to contain sixty thousand villages, (Histoire de Timur
Bec, l. v. c. 20.)}

