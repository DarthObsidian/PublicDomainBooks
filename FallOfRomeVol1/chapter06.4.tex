\section{Part \thesection.}
\thispagestyle{simple}

The lenity of the emperor confirmed the insolence of the troops;
the legions imitated the example of the guards, and defended
their prerogative of licentiousness with the same furious
obstinacy. The administration of Alexander was an unavailing
struggle against the corruption of his age. In llyricum, in
Mauritania, in Armenia, in Mesopotamia, in Germany, fresh
mutinies perpetually broke out; his officers were murdered, his
authority was insulted, and his life at last sacrificed to the
fierce discontents of the army.\footnotemark[76] One particular fact well
deserves to be recorded, as it illustrates the manners of the
troops, and exhibits a singular instance of their return to a
sense of duty and obedience. Whilst the emperor lay at Antioch,
in his Persian expedition, the particulars of which we shall
hereafter relate, the punishment of some soldiers, who had been
discovered in the baths of women, excited a sedition in the
legion to which they belonged. Alexander ascended his tribunal,
and with a modest firmness represented to the armed multitude the
absolute necessity, as well as his inflexible resolution, of
correcting the vices introduced by his impure predecessor, and of
maintaining the discipline, which could not be relaxed without
the ruin of the Roman name and empire. Their clamors interrupted
his mild expostulation. “Reserve your shout,” said the undaunted
emperor, “till you take the field against the Persians, the
Germans, and the Sarmatians. Be silent in the presence of your
sovereign and benefactor, who bestows upon you the corn, the
clothing, and the money of the provinces. Be silent, or I shall
no longer style you solders, but \textit{citizens},\footnotemark[77] if those indeed
who disclaim the laws of Rome deserve to be ranked among the
meanest of the people.” His menaces inflamed the fury of the
legion, and their brandished arms already threatened his person.
“Your courage,” resumed the intrepid Alexander, “would be more
nobly displayed in the field of battle; \textit{me} you may destroy, you
cannot intimidate; and the severe justice of the republic would
punish your crime and revenge my death.” The legion still
persisted in clamorous sedition, when the emperor pronounced,
with a loud voice, the decisive sentence, “\textit{Citizens!} lay down
your arms, and depart in peace to your respective habitations.”
The tempest was instantly appeased: the soldiers, filled with
grief and shame, silently confessed the justice of their
punishment, and the power of discipline, yielded up their arms
and military ensigns, and retired in confusion, not to their
camp, but to the several inns of the city. Alexander enjoyed,
during thirty days, the edifying spectacle of their repentance;
nor did he restore them to their former rank in the army, till he
had punished with death those tribunes whose connivance had
occasioned the mutiny. The grateful legion served the emperor
whilst living, and revenged him when dead.\footnotemark[78]

\footnotetext[76]{Annot. Reimar. ad Dion Cassius, l. lxxx. p. 1369.}

\footnotetext[77]{Julius Cæsar had appeased a sedition with the same
word, Quirites; which, thus opposed to soldiers, was used in a
sense of contempt, and reduced the offenders to the less
honorable condition of mere citizens. Tacit. Annal. i. 43.}

\footnotetext[78]{Hist. August. p. 132.}

The resolutions of the multitude generally depend on a moment;
and the caprice of passion might equally determine the seditious
legion to lay down their arms at the emperor’s feet, or to plunge
them into his breast. Perhaps, if this singular transaction had
been investigated by the penetration of a philosopher, we should
discover the secret causes which on that occasion authorized the
boldness of the prince, and commanded the obedience of the
troops; and perhaps, if it had been related by a judicious
historian, we should find this action, worthy of Cæsar himself,
reduced nearer to the level of probability and the common
standard of the character of Alexander Severus. The abilities of
that amiable prince seem to have been inadequate to the
difficulties of his situation, the firmness of his conduct
inferior to the purity of his intentions. His virtues, as well as
the vices of Elagabalus, contracted a tincture of weakness and
effeminacy from the soft climate of Syria, of which he was a
native; though he blushed at his foreign origin, and listened
with a vain complacency to the flattering genealogists, who
derived his race from the ancient stock of Roman nobility.\footnotemark[79] The
pride and avarice of his mother cast a shade on the glories of
his reign; and by exacting from his riper years the same dutiful
obedience which she had justly claimed from his unexperienced
youth, Mamæa exposed to public ridicule both her son’s character
and her own.\footnotemark[80] The fatigues of the Persian war irritated the
military discontent; the unsuccessful event\footnotemark[801] degraded the
reputation of the emperor as a general, and even as a soldier.
Every cause prepared, and every circumstance hastened, a
revolution, which distracted the Roman empire with a long series
of intestine calamities.

\footnotetext[79]{From the Metelli. Hist. August. p. 119. The choice
was judicious. In one short period of twelve years, the Metelli
could reckon seven consulships and five triumphs. See Velleius
Paterculus, ii. 11, and the Fasti.}

\footnotetext[80]{The life of Alexander, in the Augustan History, is
the mere idea of a perfect prince, an awkward imitation of the
Cyropædia. The account of his reign, as given by Herodian, is
rational and moderate, consistent with the general history of the
age; and, in some of the most invidious particulars, confirmed by
the decisive fragments of Dion. Yet from a very paltry prejudice,
the greater number of our modern writers abuse Herodian, and copy
the Augustan History. See Mess de Tillemont and Wotton. From the
opposite prejudice, the emperor Julian (in Cæsarib. p. 315)
dwells with a visible satisfaction on the effeminate weakness of
the Syrian, and the ridiculous avarice of his mother.}

\footnotetext[801]{Historians are divided as to the success of the
campaign against the Persians; Herodian alone speaks of defeat.
Lampridius, Eutropius, Victor, and others, say that it was very
glorious to Alexander; that he beat Artaxerxes in a great battle,
and repelled him from the frontiers of the empire. This much is
certain, that Alexander, on his return to Rome, (Lamp. Hist. Aug.
c. 56, 133, 134,) received the honors of a triumph, and that he
said, in his oration to the people. Quirites, vicimus Persas,
milites divites reduximus, vobis congiarium pollicemur, cras
ludos circenses Persicos donabimus. Alexander, says Eckhel, had
too much modesty and wisdom to permit himself to receive honors
which ought only to be the reward of victory, if he had not
deserved them; he would have contented himself with dissembling
his losses. Eckhel, Doct. Num. vet. vii. 276. The medals
represent him as in triumph; one, among others, displays him
crowned by Victory between two rivers, the Euphrates and the
Tigris. P. M. TR. P. xii. Cos. iii. PP. Imperator paludatus D.
hastam. S. parazonium, stat inter duos fluvios humi jacentes, et
ab accedente retro Victoria coronatur. Æ. max. mod. (Mus. Reg.
Gall.) Although Gibbon treats this question more in detail when
he speaks of the Persian monarchy, I have thought fit to place
here what contradicts his opinion.—G}

The dissolute tyranny of Commodus, the civil wars occasioned by
his death, and the new maxims of policy introduced by the house
of Severus, had all contributed to increase the dangerous power
of the army, and to obliterate the faint image of laws and
liberty that was still impressed on the minds of the Romans. The
internal change, which undermined the foundations of the empire,
we have endeavored to explain with some degree of order and
perspicuity. The personal characters of the emperors, their
victories, laws, follies, and fortunes, can interest us no
farther than as they are connected with the general history of
the Decline and Fall of the monarchy. Our constant attention to
that great object will not suffer us to overlook a most important
edict of Antoninus Caracalla, which communicated to all the free
inhabitants of the empire the name and privileges of Roman
citizens. His unbounded liberality flowed not, however, from the
sentiments of a generous mind; it was the sordid result of
avarice, and will naturally be illustrated by some observations
on the finances of that state, from the victorious ages of the
commonwealth to the reign of Alexander Severus.

The siege of Veii in Tuscany, the first considerable enterprise
of the Romans, was protracted to the tenth year, much less by the
strength of the place than by the unskilfulness of the besiegers.
The unaccustomed hardships of so many winter campaigns, at the
distance of near twenty miles from home,\footnotemark[81] required more than
common encouragements; and the senate wisely prevented the
clamors of the people, by the institution of a regular pay for
the soldiers, which was levied by a general tribute, assessed
according to an equitable proportion on the property of the
citizens.\footnotemark[82] During more than two hundred years after the
conquest of Veii, the victories of the republic added less to the
wealth than to the power of Rome. The states of Italy paid their
tribute in military service only, and the vast force, both by sea
and land, which was exerted in the Punic wars, was maintained at
the expense of the Romans themselves. That high-spirited people
(such is often the generous enthusiasm of freedom) cheerfully
submitted to the most excessive but voluntary burdens, in the
just confidence that they should speedily enjoy the rich harvest
of their labors. Their expectations were not disappointed. In the
course of a few years, the riches of Syracuse, of Carthage, of
Macedonia, and of Asia, were brought in triumph to Rome. The
treasures of Perseus alone amounted to near two millions
sterling, and the Roman people, the sovereign of so many nations,
was forever delivered from the weight of taxes.\footnotemark[83] The increasing
revenue of the provinces was found sufficient to defray the
ordinary establishment of war and government, and the superfluous
mass of gold and silver was deposited in the temple of Saturn,
and reserved for any unforeseen emergency of the state.\footnotemark[84]

\footnotetext[81]{According to the more accurate Dionysius, the city
itself was only a hundred stadia, or twelve miles and a half,
from Rome, though some out-posts might be advanced farther on the
side of Etruria. Nardini, in a professed treatise, has combated
the popular opinion and the authority of two popes, and has
removed Veii from Civita Castellana, to a little spot called
Isola, in the midway between Rome and the Lake Bracianno. * Note:
See the interesting account of the site and ruins of Veii in Sir
W Gell’s topography of Rome and its Vicinity. v. ii. p. 303.—M.}

\footnotetext[82]{See the 4th and 5th books of Livy. In the Roman
census, property, power, and taxation were commensurate with each
other.}

\footnotetext[83]{Plin. Hist. Natur. l. xxxiii. c. 3. Cicero de
Offic. ii. 22. Plutarch, P. Æmil. p. 275.}

\footnotetext[84]{See a fine description of this accumulated wealth
of ages in Phars. l. iii. v. 155, \&c.}

History has never, perhaps, suffered a greater or more
irreparable injury than in the loss of the curious register\footnotemark[841]
bequeathed by Augustus to the senate, in which that experienced
prince so accurately balanced the revenues and expenses of the
Roman empire.\footnotemark[85] Deprived of this clear and comprehensive
estimate, we are reduced to collect a few imperfect hints from
such of the ancients as have accidentally turned aside from the
splendid to the more useful parts of history. We are informed
that, by the conquests of Pompey, the tributes of Asia were
raised from fifty to one hundred and thirty-five millions of
drachms; or about four millions and a half sterling.\footnotemark[86] \footnotemark[861] Under
the last and most indolent of the Ptolemies, the revenue of Egypt
is said to have amounted to twelve thousand five hundred talents;
a sum equivalent to more than two millions and a half of our
money, but which was afterwards considerably improved by the more
exact economy of the Romans, and the increase of the trade of
Æthiopia and India.\footnotemark[87] Gaul was enriched by rapine, as Egypt was
by commerce, and the tributes of those two great provinces have
been compared as nearly equal to each other in value.\footnotemark[88] The ten
thousand Euboic or Phœnician talents, about four millions
sterling,\footnotemark[89] which vanquished Carthage was condemned to pay
within the term of fifty years, were a slight acknowledgment of
the superiority of Rome,\footnotemark[90] and cannot bear the least proportion
with the taxes afterwards raised both on the lands and on the
persons of the inhabitants, when the fertile coast of Africa was
reduced into a province.\footnotemark[91]

\footnotetext[841]{See Rationarium imperii. Compare besides Tacitus,
Suet. Aug. c. ult. Dion, p. 832. Other emperors kept and
published similar registers. See a dissertation of Dr. Wolle, de
Rationario imperii Rom. Leipsig, 1773. The last book of Appian
also contained the statistics of the Roman empire, but it is
lost.—W.}

\footnotetext[85]{Tacit. in Annal. i. ll. It seems to have existed in
the time of Appian.}

\footnotetext[86]{Plutarch, in Pompeio, p. 642.}

\footnotetext[861]{Wenck contests the accuracy of Gibbon’s version of
Plutarch, and supposes that Pompey only raised the revenue from
50,000,000 to 85,000,000 of drachms; but the text of Plutarch
seems clearly to mean that his conquests added 85,000,000 to the
ordinary revenue. Wenck adds, “Plutarch says in another part,
that Antony made Asia pay, at one time, 200,000 talents, that is
to say, 38,875,000 L. sterling.” But Appian explains this by
saying that it was the revenue of ten years, which brings the
annual revenue, at the time of Antony, to 3,875,000 L.
sterling.—M.}

\footnotetext[87]{Strabo, l. xvii. p. 798.}

\footnotetext[88]{Velleius Paterculus, l. ii. c. 39. He seems to give
the preference to the revenue of Gaul.}

\footnotetext[89]{The Euboic, the Phœnician, and the Alexandrian
talents were double in weight to the Attic. See Hooper on ancient
weights and measures, p. iv. c. 5. It is very probable that the
same talent was carried from Tyre to Carthage.}

\footnotetext[90]{Polyb. l. xv. c. 2.}

\footnotetext[91]{Appian in Punicis, p. 84.}

Spain, by a very singular fatality, was the Peru and Mexico of
the old world. The discovery of the rich western continent by the
Phœnicians, and the oppression of the simple natives, who were
compelled to labor in their own mines for the benefit of
strangers, form an exact type of the more recent history of
Spanish America.\footnotemark[92] The Phœnicians were acquainted only with the
sea-coast of Spain; avarice, as well as ambition, carried the
arms of Rome and Carthage into the heart of the country, and
almost every part of the soil was found pregnant with copper,
silver, and gold.\footnotemark[921] Mention is made of a mine near Carthagena
which yielded every day twenty-five thousand drachmns of silver,
or about three hundred thousand pounds a year.\footnotemark[93] Twenty thousand
pound weight of gold was annually received from the provinces of
Asturia, Gallicia, and Lusitania.\footnotemark[94]

\footnotetext[92]{Diodorus Siculus, l. 5. Oadiz was built by the
Phœnicians a little more than a thousand years before Christ. See
Vell. Pa ter. i.2.}

\footnotetext[921]{Compare Heeren’s Researches vol. i. part ii. p.}

\footnotetext[93]{Strabo, l. iii. p. 148.}

\footnotetext[94]{Plin. Hist. Natur. l. xxxiii. c. 3. He mentions
likewise a silver mine in Dalmatia, that yielded every day fifty
pounds to the state.}

We want both leisure and materials to
pursue this curious inquiry through the many potent states that
were annihilated in the Roman empire. Some notion, however, may
be formed of the revenue of the provinces where considerable
wealth had been deposited by nature, or collected by man, if we
observe the severe attention that was directed to the abodes of
solitude and sterility. Augustus once received a petition from
the inhabitants of Gyarus, humbly praying that they might be
relieved from one third of their excessive impositions. Their
whole tax amounted indeed to no more than one hundred and fifty
drachms, or about five pounds: but Gyarus was a little island, or
rather a rock, of the Ægean Sea, destitute of fresh water and
every necessary of life, and inhabited only by a few wretched
fishermen.\footnotemark[95]

\footnotetext[95]{Strabo, l. x. p. 485. Tacit. Annal. iu. 69, and iv.
30. See Tournefort (Voyages au Levant, Lettre viii.) a very
lively picture of the actual misery of Gyarus.}

From the faint glimmerings of such doubtful and scattered lights,
we should be inclined to believe, 1st, That (with every fair
allowance for the differences of times and circumstances) the
general income of the Roman provinces could seldom amount to less
than fifteen or twenty millions of our money;\footnotemark[96] and, 2dly, That
so ample a revenue must have been fully adequate to all the
expenses of the moderate government instituted by Augustus, whose
court was the modest family of a private senator, and whose
military establishment was calculated for the defence of the
frontiers, without any aspiring views of conquest, or any serious
apprehension of a foreign invasion.

\footnotetext[96]{Lipsius de magnitudine Romana (l. ii. c. 3)
computes the revenue at one hundred and fifty millions of gold
crowns; but his whole book, though learned and ingenious, betrays
a very heated imagination. Note: If Justus Lipsius has
exaggerated the revenue of the Roman empire Gibbon, on the other
hand, has underrated it. He fixes it at fifteen or twenty
millions of our money. But if we take only, on a moderate
calculation, the taxes in the provinces which he has already
cited, they will amount, considering the augmentations made by
Augustus, to nearly that sum. There remain also the provinces of
Italy, of Rhætia, of Noricum, Pannonia, and Greece, \&c., \&c. Let
us pay attention, besides, to the prodigious expenditure of some
emperors, (Suet. Vesp. 16;) we shall see that such a revenue
could not be sufficient. The authors of the Universal History,
part xii., assign forty millions sterling as the sum to about
which the public revenue might amount.—G. from W.}

Notwithstanding the seeming probability of both these
conclusions, the latter of them at least is positively disowned
by the language and conduct of Augustus. It is not easy to
determine whether, on this occasion, he acted as the common
father of the Roman world, or as the oppressor of liberty;
whether he wished to relieve the provinces, or to impoverish the
senate and the equestrian order. But no sooner had he assumed the
reins of government, than he frequently intimated the
insufficiency of the tributes, and the necessity of throwing an
equitable proportion of the public burden upon Rome and Italy.\footnotemark[961]
In the prosecution of this unpopular design, he advanced,
however, by cautious and well-weighed steps. The introduction of
customs was followed by the establishment of an excise, and the
scheme of taxation was completed by an artful assessment on the
real and personal property of the Roman citizens, who had been
exempted from any kind of contribution above a century and a
half.

\footnotetext[961]{It is not astonishing that Augustus held this
language. The senate declared also under Nero, that the state
could not exist without the imposts as well augmented as founded
by Augustus. Tac. Ann. xiii. 50. After the abolition of the
different tributes paid by Italy, an abolition which took place
A. U. 646, 694, and 695, the state derived no revenues from that
great country, but the twentieth part of the manumissions,
(vicesima manumissionum,) and Ciero laments this in many places,
particularly in his epistles to ii. 15.—G. from W.}

I. In a great empire like that of Rome, a natural balance of
money must have gradually established itself. It has been already
observed, that as the wealth of the provinces was attracted to
the capital by the strong hand of conquest and power, so a
considerable part of it was restored to the industrious provinces
by the gentle influence of commerce and arts. In the reign of
Augustus and his successors, duties were imposed on every kind of
merchandise, which through a thousand channels flowed to the
great centre of opulence and luxury; and in whatsoever manner the
law was expressed, it was the Roman purchaser, and not the
provincial merchant, who paid the tax.\footnotemark[97] The rate of the customs
varied from the eighth to the fortieth part of the value of the
commodity; and we have a right to suppose that the variation was
directed by the unalterable maxims of policy; that a higher duty
was fixed on the articles of luxury than on those of necessity,
and that the productions raised or manufactured by the labor of
the subjects of the empire were treated with more indulgence than
was shown to the pernicious, or at least the unpopular, commerce
of Arabia and India.\footnotemark[98] There is still extant a long but
imperfect catalogue of eastern commodities, which about the time
of Alexander Severus were subject to the payment of duties;
cinnamon, myrrh, pepper, ginger, and the whole tribe of
aromatics; a great variety of precious stones, among which the
diamond was the most remarkable for its price, and the emerald
for its beauty;\footnotemark[99] Parthian and Babylonian leather, cottons,
silks, both raw and manufactured, ebony ivory, and eunuchs.\footnotemark[100]
We may observe that the use and value of those effeminate slaves
gradually rose with the decline of the empire.

\footnotetext[97]{Tacit. Annal. xiii. 31. * Note: The customs
(portoria) existed in the times of the ancient kings of Rome.
They were suppressed in Italy, A. U. 694, by the Prætor, Cecilius
Matellus Nepos. Augustus only reestablished them. See note
above.—W.}

\footnotetext[98]{See Pliny, (Hist. Natur. l. vi. c. 23, lxii. c. 18.)
His observation that the Indian commodities were sold at Rome at
a hundred times their original price, may give us some notion of
the produce of the customs, since that original price amounted to
more than eight hundred thousand pounds.}

\footnotetext[99]{The ancients were unacquainted with the art of
cutting diamonds.}

\footnotetext[100]{M. Bouchaud, in his treatise de l’Impot chez les
Romains, has transcribed this catalogue from the Digest, and
attempts to illustrate it by a very prolix commentary. * Note: In
the Pandects, l. 39, t. 14, de Publican. Compare Cicero in
Verrem. c. 72—74.—W.}

II. The excise, introduced by Augustus after the civil wars, was
extremely moderate, but it was general. It seldom exceeded one
\textit{per cent}.; but it comprehended whatever was sold in the markets
or by public auction, from the most considerable purchases of
lands and houses, to those minute objects which can only derive a
value from their infinite multitude and daily consumption. Such a
tax, as it affects the body of the people, has ever been the
occasion of clamor and discontent. An emperor well acquainted
with the wants and resources of the state was obliged to declare,
by a public edict, that the support of the army depended in a
great measure on the produce of the excise.\footnotemark[101]

\footnotetext[101]{Tacit. Annal. i. 78. Two years afterwards, the
reduction of the poor kingdom of Cappadocia gave Tiberius a
pretence for diminishing the excise of one half, but the relief
was of very short duration.}

III. When Augustus resolved to establish a permanent military
force for the defence of his government against foreign and
domestic enemies, he instituted a peculiar treasury for the pay
of the soldiers, the rewards of the veterans, and the
extra-ordinary expenses of war. The ample revenue of the excise,
though peculiarly appropriated to those uses, was found
inadequate. To supply the deficiency, the emperor suggested a new
tax of five per cent. on all legacies and inheritances. But the
nobles of Rome were more tenacious of property than of freedom.
Their indignant murmurs were received by Augustus with his usual
temper. He candidly referred the whole business to the senate,
and exhorted them to provide for the public service by some other
expedient of a less odious nature. They were divided and
perplexed. He insinuated to them, that their obstinacy would
oblige him to \textit{propose} a general land tax and capitation. They
acquiesced in silence.\footnotemark[102] The new imposition on legacies and
inheritances was, however, mitigated by some restrictions. It did
not take place unless the object was of a certain value, most
probably of fifty or a hundred pieces of gold;\footnotemark[103] nor could it
be exacted from the nearest of kin on the father’s side.\footnotemark[104] When
the rights of nature and poverty were thus secured, it seemed
reasonable, that a stranger, or a distant relation, who acquired
an unexpected accession of fortune, should cheerfully resign a
twentieth part of it, for the benefit of the state. \footnotemark[105]

\footnotetext[102]{Dion Cassius, l. lv. p. 794, l. lvi. p. 825. Note:
Dion neither mentions this proposition nor the capitation. He
only says that the emperor imposed a tax upon landed property,
and sent every where men employed to make a survey, without
fixing how much, and for how much each was to pay. The senators
then preferred giving the tax on legacies and inheritances.—W.}

\footnotetext[103]{The sum is only fixed by conjecture.}

\footnotetext[104]{As the Roman law subsisted for many ages, the
Cognati, or relations on the mother’s side, were not called to
the succession. This harsh institution was gradually undermined
by humanity, and finally abolished by Justinian.}

\footnotetext[105]{Plin. Panegyric. c. 37.}

Such a tax, plentiful as it must prove in every wealthy
community, was most happily suited to the situation of the
Romans, who could frame their arbitrary wills, according to the
dictates of reason or caprice, without any restraint from the
modern fetters of entails and settlements. From various causes,
the partiality of paternal affection often lost its influence
over the stern patriots of the commonwealth, and the dissolute
nobles of the empire; and if the father bequeathed to his son the
fourth part of his estate, he removed all ground of legal
complaint.\footnotemark[106] But a rich childish old man was a domestic tyrant,
and his power increased with his years and infirmities. A servile
crowd, in which he frequently reckoned prætors and consuls,
courted his smiles, pampered his avarice, applauded his follies,
served his passions, and waited with impatience for his death.
The arts of attendance and flattery were formed into a most
lucrative science; those who professed it acquired a peculiar
appellation; and the whole city, according to the lively
descriptions of satire, was divided between two parties, the
hunters and their game.\footnotemark[107] Yet, while so many unjust and
extravagant wills were every day dictated by cunning and
subscribed by folly, a few were the result of rational esteem and
virtuous gratitude. Cicero, who had so often defended the lives
and fortunes of his fellow-citizens, was rewarded with legacies
to the amount of a hundred and seventy thousand pounds;\footnotemark[108] nor
do the friends of the younger Pliny seem to have been less
generous to that amiable orator.\footnotemark[109] Whatever was the motive of
the testator, the treasury claimed, without distinction, the
twentieth part of his estate: and in the course of two or three
generations, the whole property of the subject must have
gradually passed through the coffers of the state.

\footnotetext[106]{See Heineccius in the Antiquit. Juris Romani, l.
ii.}

\footnotetext[107]{Horat. l. ii. Sat. v. Potron. c. 116, \&c. Plin. l.
ii. Epist. 20.}

\footnotetext[108]{Cicero in Philip. ii. c. 16.}

\footnotetext[109]{See his epistles. Every such will gave him an
occasion of displaying his reverence to the dead, and his justice
to the living. He reconciled both in his behavior to a son who
had been disinherited by his mother, (v.l.)}

In the first and golden years of the reign of Nero, that prince,
from a desire of popularity, and perhaps from a blind impulse of
benevolence, conceived a wish of abolishing the oppression of the
customs and excise. The wisest senators applauded his
magnanimity: but they diverted him from the execution of a design
which would have dissolved the strength and resources of the
republic.\footnotemark[110] Had it indeed been possible to realize this dream
of fancy, such princes as Trajan and the Antonines would surely
have embraced with ardor the glorious opportunity of conferring
so signal an obligation on mankind. Satisfied, however, with
alleviating the public burden, they attempted not to remove it.
The mildness and precision of their laws ascertained the rule and
measure of taxation, and protected the subject of every rank
against arbitrary interpretations, antiquated claims, and the
insolent vexation of the farmers of the revenue.\footnotemark[111] For it is
somewhat singular, that, in every age, the best and wisest of the
Roman governors persevered in this pernicious method of
collecting the principal branches at least of the excise and
customs.\footnotemark[112]

\footnotetext[110]{Tacit. Annal. xiii. 50. Esprit des Loix, l. xii.
c. 19.}

\footnotetext[111]{See Pliny’s Panegyric, the Augustan History, and
Burman de Vectigal. passim.}

\footnotetext[112]{The tributes (properly so called) were not farmed;
since the good princes often remitted many millions of arrears.}

The sentiments, and, indeed, the situation, of Caracalla were
very different from those of the Antonines. Inattentive, or
rather averse, to the welfare of his people, he found himself
under the necessity of gratifying the insatiate avarice which he
had excited in the army. Of the several impositions introduced by
Augustus, the twentieth on inheritances and legacies was the most
fruitful, as well as the most comprehensive. As its influence was
not confined to Rome or Italy, the produce continually increased
with the gradual extension of the Roman City. The new citizens,
though charged, on equal terms,\footnotemark[113] with the payment of new
taxes, which had not affected them as subjects, derived an ample
compensation from the rank they obtained, the privileges they
acquired, and the fair prospect of honors and fortune that was
thrown open to their ambition. But the favor which implied a
distinction was lost in the prodigality of Caracalla, and the
reluctant provincials were compelled to assume the vain title,
and the real obligations, of Roman citizens.\footnotemark[1131] Nor was the
rapacious son of Severus contented with such a measure of
taxation as had appeared sufficient to his moderate predecessors.
Instead of a twentieth, he exacted a tenth of all legacies and
inheritances; and during his reign (for the ancient proportion
was restored after his death) he crushed alike every part of the
empire under the weight of his iron sceptre.\footnotemark[114]

\footnotetext[113]{The situation of the new citizens is minutely
described by Pliny, (Panegyric, c. 37, 38, 39). Trajan published
a law very much in their favor.}

\footnotetext[1131]{Gibbon has adopted the opinion of Spanheim and of
Burman, which attributes to Caracalla this edict, which gave the
right of the city to all the inhabitants of the provinces. This
opinion may be disputed. Several passages of Spartianus, of
Aurelius Victor, and of Aristides, attribute this edict to Marc.
Aurelius. See a learned essay, entitled Joh. P. Mahneri Comm. de
Marc. Aur. Antonino Constitutionis de Civitate Universo Orbi
Romano data auctore. Halæ, 1772, 8vo. It appears that Marc.
Aurelius made some modifications of this edict, which released
the provincials from some of the charges imposed by the right of
the city, and deprived them of some of the advantages which it
conferred. Caracalla annulled these modifications.—W.}

\footnotetext[114]{Dion, l. lxxvii. p. 1295.}

When all the provincials became liable to the peculiar
impositions of Roman citizens, they seemed to acquire a legal
exemption from the tributes which they had paid in their former
condition of subjects. Such were not the maxims of government
adopted by Caracalla and his pretended son. The old as well as
the new taxes were, at the same time, levied in the provinces. It
was reserved for the virtue of Alexander to relieve them in a
great measure from this intolerable grievance, by reducing the
tributes to a thirteenth part of the sum exacted at the time of
his accession.\footnotemark[115] It is impossible to conjecture the motive that
engaged him to spare so trifling a remnant of the public evil;
but the noxious weed, which had not been totally eradicated,
again sprang up with the most luxuriant growth, and in the
succeeding age darkened the Roman world with its deadly shade. In
the course of this history, we shall be too often summoned to
explain the land tax, the capitation, and the heavy contributions
of corn, wine, oil, and meat, which were exacted from the
provinces for the use of the court, the army, and the capital.

\footnotetext[115]{He who paid ten aurei, the usual tribute, was
charged with no more than the third part of an aureus, and
proportional pieces of gold were coined by Alexander’s order.
Hist. August. p. 127, with the commentary of Salmasius.}

As long as Rome and Italy were respected as the centre of
government, a national spirit was preserved by the ancient, and
insensibly imbibed by the adopted, citizens. The principal
commands of the army were filled by men who had received a
liberal education, were well instructed in the advantages of laws
and letters, and who had risen, by equal steps, through the
regular succession of civil and military honors.\footnotemark[116] To their
influence and example we may partly ascribe the modest obedience
of the legions during the two first centuries of the Imperial
history.

\footnotetext[116]{See the lives of Agricola, Vespasian, Trajan,
Severus, and his three competitors; and indeed of all the eminent
men of those times.}

But when the last enclosure of the Roman constitution was
trampled down by Caracalla, the separation of professions
gradually succeeded to the distinction of ranks. The more
polished citizens of the internal provinces were alone qualified
to act as lawyers and magistrates. The rougher trade of arms was
abandoned to the peasants and barbarians of the frontiers, who
knew no country but their camp, no science but that of war, no
civil laws, and scarcely those of military discipline. With
bloody hands, savage manners, and desperate resolutions, they
sometimes guarded, but much oftener subverted, the throne of the
emperors.

