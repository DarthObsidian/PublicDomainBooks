\section{Part \thesection.}
\thispagestyle{simple}

The Germans respected only those duties which they imposed on
themselves. The most obscure soldier resisted with disdain the
authority of the magistrates. “The noblest youths blushed not to
be numbered among the faithful companions of some renowned chief,
to whom they devoted their arms and service. A noble emulation
prevailed among the companions to obtain the first place in the
esteem of their chief; amongst the chiefs, to acquire the
greatest number of valiant companions. To be ever surrounded by a
band of select youths was the pride and strength of the chiefs,
their ornament in peace, their defence in war. The glory of such
distinguished heroes diffused itself beyond the narrow limits of
their own tribe. Presents and embassies solicited their
friendship, and the fame of their arms often insured victory to
the party which they espoused. In the hour of danger it was
shameful for the chief to be surpassed in valor by his
companions; shameful for the companions not to equal the valor of
their chief. To survive his fall in battle was indelible infamy.
To protect his person, and to adorn his glory with the trophies
of their own exploits, were the most sacred of their duties. The
chiefs combated for victory, the companions for the chief. The
noblest warriors, whenever their native country was sunk into the
laziness of peace, maintained their numerous bands in some
distant scene of action, to exercise their restless spirit, and
to acquire renown by voluntary dangers. Gifts worthy of
soldiers—the warlike steed, the bloody and ever victorious
lance—were the rewards which the companions claimed from the
liberality of their chief. The rude plenty of his hospitable
board was the only pay that \textit{he} could bestow, or \textit{they} would
accept. War, rapine, and the free-will offerings of his friends,
supplied the materials of this munificence.”\footnotemark[53] This institution,
however it might accidentally weaken the several republics,
invigorated the general character of the Germans, and even
ripened amongst them all the virtues of which barbarians are
susceptible; the faith and valor, the hospitality and the
courtesy, so conspicuous long afterwards in the ages of chivalry.

The honorable gifts, bestowed by the chief on his brave
companions, have been supposed, by an ingenious writer, to
contain the first rudiments of the fiefs, distributed after the
conquest of the Roman provinces, by the barbarian lords among
their vassals, with a similar duty of homage and military
service.\footnotemark[54] These conditions are, however, very repugnant to the
maxims of the ancient Germans, who delighted in mutual presents,
but without either imposing, or accepting, the weight of
obligations.\footnotemark[55]

\footnotetext[53]{Tacit. Germ. 13, 14.}

\footnotetext[54]{Esprit des Loix, l. xxx. c. 3. The brilliant
imagination of Montesquieu is corrected, however, by the dry,
cold reason of the Abbé de Mably. Observations sur l’Histoire de
France, tom. i. p. 356.}

\footnotetext[55]{Gaudent muneribus, sed nec data imputant, nec
acceptis obligautur. Tacit. Germ. c. 21.}

“In the days of chivalry, or more properly of romance, all the
men were brave and all the women were chaste;” and
notwithstanding the latter of these virtues is acquired and
preserved with much more difficulty than the former, it is
ascribed, almost without exception, to the wives of the ancient
Germans. Polygamy was not in use, except among the princes, and
among them only for the sake of multiplying their alliances.
Divorces were prohibited by manners rather than by laws.
Adulteries were punished as rare and inexpiable crimes; nor was
seduction justified by example and fashion.\footnotemark[56] We may easily
discover that Tacitus indulges an honest pleasure in the contrast
of barbarian virtue with the dissolute conduct of the Roman
ladies; yet there are some striking circumstances that give an
air of truth, or at least probability, to the conjugal faith and
chastity of the Germans.

\footnotetext[56]{The adulteress was whipped through the village.
Neither wealth nor beauty could inspire compassion, or procure
her a second husband. 18, 19.}

Although the progress of civilization has undoubtedly contributed
to assuage the fiercer passions of human nature, it seems to have
been less favorable to the virtue of chastity, whose most
dangerous enemy is the softness of the mind. The refinements of
life corrupt while they polish the intercourse of the sexes. The
gross appetite of love becomes most dangerous when it is
elevated, or rather, indeed, disguised by sentimental passion.
The elegance of dress, of motion, and of manners, gives a lustre
to beauty, and inflames the senses through the imagination.
Luxurious entertainments, midnight dances, and licentious
spectacles, present at once temptation and opportunity to female
frailty.\footnotemark[57] From such dangers the unpolished wives of the
barbarians were secured by poverty, solitude, and the painful
cares of a domestic life. The German huts, open, on every side,
to the eye of indiscretion or jealousy, were a better safeguard
of conjugal fidelity than the walls, the bolts, and the eunuchs
of a Persian harem. To this reason another may be added of a more
honorable nature. The Germans treated their women with esteem and
confidence, consulted them on every occasion of importance, and
fondly believed, that in their breasts resided a sanctity and
wisdom more than human. Some of the interpreters of fate, such as
Velleda, in the Batavian war, governed, in the name of the deity,
the fiercest nations of Germany.\footnotemark[58] The rest of the sex, without
being adored as goddesses, were respected as the free and equal
companions of soldiers; associated even by the marriage ceremony
to a life of toil, of danger, and of glory.\footnotemark[59] In their great
invasions, the camps of the barbarians were filled with a
multitude of women, who remained firm and undaunted amidst the
sound of arms, the various forms of destruction, and the
honorable wounds of their sons and husbands.\footnotemark[60] Fainting armies
of Germans have, more than once, been driven back upon the enemy
by the generous despair of the women, who dreaded death much less
than servitude. If the day was irrecoverably lost, they well knew
how to deliver themselves and their children, with their own
hands, from an insulting victor.\footnotemark[61] Heroines of such a cast may
claim our admiration; but they were most assuredly neither lovely
nor very susceptible of love. Whilst they affected to emulate the
stern virtues of \textit{man}, they must have resigned that attractive
softness in which principally consist the charm and weakness of
\textit{woman}. Conscious pride taught the German females to suppress
every tender emotion that stood in competition with honor, and
the first honor of the sex has ever been that of chastity. The
sentiments and conduct of these high-spirited matrons may, at
once, be considered as a cause, as an effect, and as a proof of
the general character of the nation. Female courage, however it
may be raised by fanaticism, or confirmed by habit, can be only a
faint and imperfect imitation of the manly valor that
distinguishes the age or country in which it may be found.

\footnotetext[57]{Ovid employs two hundred lines in the research of
places the most favorable to love. Above all, he considers the
theatre as the best adapted to collect the beauties of Rome, and
to melt them into tenderness and sensuality,}

\footnotetext[58]{Tacit. Germ. iv. 61, 65.}

\footnotetext[59]{The marriage present was a yoke of oxen, horses,
and arms. See Germ. c. 18. Tacitus is somewhat too florid on the
subject.}

\footnotetext[60]{The change of exigere into exugere is a most
excellent correction.}

\footnotetext[61]{Tacit. Germ. c. 7. Plutarch in Mario. Before the
wives of the Teutones destroyed themselves and their children,
they had offered to surrender, on condition that they should be
received as the slaves of the vestal virgins.}

The religious system of the Germans (if the wild opinions of
savages can deserve that name) was dictated by their wants, their
fears, and their ignorance.\footnotemark[62] They adored the great visible
objects and agents of nature, the Sun and the Moon, the Fire and
the Earth; together with those imaginary deities, who were
supposed to preside over the most important occupations of human
life. They were persuaded, that, by some ridiculous arts of
divination, they could discover the will of the superior beings,
and that human sacrifices were the most precious and acceptable
offering to their altars. Some applause has been hastily bestowed
on the sublime notion, entertained by that people, of the Deity,
whom they neither confined within the walls of the temple, nor
represented by any human figure; but when we recollect, that the
Germans were unskilled in architecture, and totally unacquainted
with the art of sculpture, we shall readily assign the true
reason of a scruple, which arose not so much from a superiority
of reason, as from a want of ingenuity. The only temples in
Germany were dark and ancient groves, consecrated by the
reverence of succeeding generations. Their secret gloom, the
imagined residence of an invisible power, by presenting no
distinct object of fear or worship, impressed the mind with a
still deeper sense of religious horror;\footnotemark[63] and the priests, rude
and illiterate as they were, had been taught by experience the
use of every artifice that could preserve and fortify impressions
so well suited to their own interest.

\footnotetext[62]{Tacitus has employed a few lines, and Cluverius one
hundred and twenty-four pages, on this obscure subject. The
former discovers in Germany the gods of Greece and Rome. The
latter is positive, that, under the emblems of the sun, the moon,
and the fire, his pious ancestors worshipped the Trinity in
unity}

\footnotetext[63]{The sacred wood, described with such sublime horror
by Lucan, was in the neighborhood of Marseilles; but there were
many of the same kind in Germany. * Note: The ancient Germans had
shapeless idols, and, when they began to build more settled
habitations, they raised also temples, such as that to the
goddess Teufana, who presided over divination. See Adelung, Hist.
of Ane Germans, p 296—G}

The same ignorance, which renders barbarians incapable of
conceiving or embracing the useful restraints of laws, exposes
them naked and unarmed to the blind terrors of superstition. The
German priests, improving this favorable temper of their
countrymen, had assumed a jurisdiction even in temporal concerns,
which the magistrate could not venture to exercise; and the
haughty warrior patiently submitted to the lash of correction,
when it was inflicted, not by any human power, but by the
immediate order of the god of war.\footnotemark[64] The defects of civil policy
were sometimes supplied by the interposition of ecclesiastical
authority. The latter was constantly exerted to maintain silence
and decency in the popular assemblies; and was sometimes extended
to a more enlarged concern for the national welfare. A solemn
procession was occasionally celebrated in the present countries
of Mecklenburgh and Pomerania. The unknown symbol of the \textit{Earth},
covered with a thick veil, was placed on a carriage drawn by
cows; and in this manner the goddess, whose common residence was
in the Isles of Rugen, visited several adjacent tribes of her
worshippers. During her progress the sound of war was hushed,
quarrels were suspended, arms laid aside, and the restless
Germans had an opportunity of tasting the blessings of peace and
harmony.\footnotemark[65] The \textit{truce of God}, so often and so ineffectually
proclaimed by the clergy of the eleventh century, was an obvious
imitation of this ancient custom.\footnotemark[66]

\footnotetext[64]{Tacit. Germania, c. 7.}

\footnotetext[65]{Tacit. Germania, c. 40.}

\footnotetext[66]{See Dr. Robertson’s History of Charles V. vol. i.
note 10.}

But the influence of religion was far more powerful to inflame,
than to moderate, the fierce passions of the Germans. Interest
and fanaticism often prompted its ministers to sanctify the most
daring and the most unjust enterprises, by the approbation of
Heaven, and full assurances of success. The consecrated
standards, long revered in the groves of superstition, were
placed in the front of the battle;\footnotemark[67] and the hostile army was
devoted with dire execrations to the gods of war and of thunder.\footnotemark[68]
In the faith of soldiers (and such were the Germans) cowardice
is the most unpardonable of sins. A brave man was the worthy
favorite of their martial deities; the wretch who had lost his
shield was alike banished from the religious and civil assemblies
of his countrymen. Some tribes of the north seem to have embraced
the doctrine of transmigration,\footnotemark[69] others imagined a gross
paradise of immortal drunkenness.\footnotemark[70] All agreed that a life spent
in arms, and a glorious death in battle, were the best
preparations for a happy futurity, either in this or in another
world.

\footnotetext[67]{Tacit. Germania, c. 7. These standards were only
the heads of wild beasts.}

\footnotetext[68]{See an instance of this custom, Tacit. Annal. xiii.
57.}

\footnotetext[69]{Cæsar Diodorus, and Lucan, seem to ascribe this
doctrine to the Gauls, but M. Pelloutier (Histoire des Celtes, l.
iii. c. 18) labors to reduce their expressions to a more orthodox
sense.}

\footnotetext[70]{Concerning this gross but alluring doctrine of the
Edda, see Fable xx. in the curious version of that book,
published by M. Mallet, in his Introduction to the History of
Denmark.}

The immortality so vainly promised by the priests, was, in some
degree, conferred by the bards. That singular order of men has
most deservedly attracted the notice of all who have attempted to
investigate the antiquities of the Celts, the Scandinavians, and
the Germans. Their genius and character, as well as the reverence
paid to that important office, have been sufficiently
illustrated. But we cannot so easily express, or even conceive,
the enthusiasm of arms and glory which they kindled in the breast
of their audience. Among a polished people a taste for poetry is
rather an amusement of the fancy than a passion of the soul. And
yet, when in calm retirement we peruse the combats described by
Homer or Tasso, we are insensibly seduced by the fiction, and
feel a momentary glow of martial ardor. But how faint, how cold
is the sensation which a peaceful mind can receive from solitary
study! It was in the hour of battle, or in the feast of victory,
that the bards celebrated the glory of the heroes of ancient
days, the ancestors of those warlike chieftains, who listened
with transport to their artless but animated strains. The view of
arms and of danger heightened the effect of the military song;
and the passions which it tended to excite, the desire of fame,
and the contempt of death, were the habitual sentiments of a
German mind.\footnotemark[71] \footnotemark[711]

\footnotetext[71]{See Tacit. Germ. c. 3. Diod. Sicul. l. v. Strabo,
l. iv. p. 197. The classical reader may remember the rank of
Demodocus in the Phæacian court, and the ardor infused by Tyrtæus
into the fainting Spartans. Yet there is little probability that
the Greeks and the Germans were the same people. Much learned
trifling might be spared, if our antiquarians would condescend to
reflect, that similar manners will naturally be produced by
similar situations.}

\footnotetext[711]{Besides these battle songs, the Germans sang at
their festival banquets, (Tac. Ann. i. 65,) and around the bodies
of their slain heroes. King Theodoric, of the tribe of the Goths,
killed in a battle against Attila, was honored by songs while he
was borne from the field of battle. Jornandes, c. 41. The same
honor was paid to the remains of Attila. Ibid. c. 49. According
to some historians, the Germans had songs also at their weddings;
but this appears to me inconsistent with their customs, in which
marriage was no more than the purchase of a wife. Besides, there
is but one instance of this, that of the Gothic king, Ataulph,
who sang himself the nuptial hymn when he espoused Placidia,
sister of the emperors Arcadius and Honorius, (Olympiodor. p. 8.)
But this marriage was celebrated according to the Roman rites, of
which the nuptial songs formed a part. Adelung, p. 382.—G.
Charlemagne is said to have collected the national songs of the
ancient Germans. Eginhard, Vit. Car. Mag.—M.}

Such was the situation, and such were the manners of the ancient
Germans. Their climate, their want of learning, of arts, and of
laws, their notions of honor, of gallantry, and of religion,
their sense of freedom, impatience of peace, and thirst of
enterprise, all contributed to form a people of military heroes.
And yet we find, that during more than two hundred and fifty
years that elapsed from the defeat of Varus to the reign of
Decius, these formidable barbarians made few considerable
attempts, and not any material impression on the luxurious, and
enslaved provinces of the empire. Their progress was checked by
their want of arms and discipline, and their fury was diverted by
the intestine divisions of ancient Germany. I. It has been
observed, with ingenuity, and not without truth, that the command
of iron soon gives a nation the command of gold. But the rude
tribes of Germany, alike destitute of both those valuable metals,
were reduced slowly to acquire, by their unassisted strength, the
possession of the one as well as the other. The face of a German
army displayed their poverty of iron. Swords, and the longer kind
of lances, they could seldom use. Their \textit{frameæ} (as they called
them in their own language) were long spears headed with a sharp
but narrow iron point, and which, as occasion required, they
either darted from a distance, or pushed in close onset. With
this spear, and with a shield, their cavalry was contented. A
multitude of darts, scattered\footnotemark[72] with incredible force, were an
additional resource of the infantry. Their military dress, when
they wore any, was nothing more than a loose mantle. A variety of
colors was the only ornament of their wooden or osier shields.
Few of the chiefs were distinguished by cuirasses, scarcely any
by helmets. Though the horses of Germany were neither beautiful,
swift, nor practised in the skilful evolutions of the Roman
manege, several of the nations obtained renown by their cavalry;
but, in general, the principal strength of the Germans consisted
in their infantry,\footnotemark[73] which was drawn up in several deep columns,
according to the distinction of tribes and families. Impatient of
fatigue and delay, these half-armed warriors rushed to battle
with dissonant shouts and disordered ranks; and sometimes, by the
effort of native valor, prevailed over the constrained and more
artificial bravery of the Roman mercenaries. But as the
barbarians poured forth their whole souls on the first onset,
they knew not how to rally or to retire. A repulse was a sure
defeat; and a defeat was most commonly total destruction. When we
recollect the complete armor of the Roman soldiers, their
discipline, exercises, evolutions, fortified camps, and military
engines, it appears a just matter of surprise, how the naked and
unassisted valor of the barbarians could dare to encounter, in
the field, the strength of the legions, and the various troops of
the auxiliaries, which seconded their operations. The contest was
too unequal, till the introduction of luxury had enervated the
vigor, and a spirit of disobedience and sedition had relaxed the
discipline, of the Roman armies. The introduction of barbarian
auxiliaries into those armies, was a measure attended with very
obvious dangers, as it might gradually instruct the Germans in
the arts of war and of policy. Although they were admitted in
small numbers and with the strictest precaution, the example of
Civilis was proper to convince the Romans, that the danger was
not imaginary, and that their precautions were not always
sufficient.\footnotemark[74] During the civil wars that followed the death of
Nero, that artful and intrepid Batavian, whom his enemies
condescended to compare with Hannibal and Sertorius,\footnotemark[75] formed a
great design of freedom and ambition. Eight Batavian cohorts
renowned in the wars of Britain and Italy, repaired to his
standard. He introduced an army of Germans into Gaul, prevailed
on the powerful cities of Treves and Langres to embrace his
cause, defeated the legions, destroyed their fortified camps, and
employed against the Romans the military knowledge which he had
acquired in their service. When at length, after an obstinate
struggle, he yielded to the power of the empire, Civilis secured
himself and his country by an honorable treaty. The Batavians
still continued to occupy the islands of the Rhine,\footnotemark[76] the
allies, not the servants, of the Roman monarchy.

\footnotetext[72]{Missilia spargunt, Tacit. Germ. c. 6. Either that
historian used a vague expression, or he meant that they were
thrown at random.}

\footnotetext[73]{It was their principal distinction from the
Sarmatians, who generally fought on horseback.}

\footnotetext[74]{The relation of this enterprise occupies a great
part of the fourth and fifth books of the History of Tacitus, and
is more remarkable for its eloquence than perspicuity. Sir Henry
Saville has observed several inaccuracies.}

\footnotetext[75]{Tacit. Hist. iv. 13. Like them he had lost an eye.}

\footnotetext[76]{It was contained between the two branches of the
old Rhine, as they subsisted before the face of the country was
changed by art and nature. See Cluver German. Antiq. l. iii. c.
30, 37.}

II. The strength of ancient Germany appears formidable, when we
consider the effects that might have been produced by its united
effort. The wide extent of country might very possibly contain a
million of warriors, as all who were of age to bear arms were of
a temper to use them. But this fierce multitude, incapable of
concerting or executing any plan of national greatness, was
agitated by various and often hostile intentions. Germany was
divided into more than forty independent states; and, even in
each state, the union of the several tribes was extremely loose
and precarious. The barbarians were easily provoked; they knew
not how to forgive an injury, much less an insult; their
resentments were bloody and implacable. The casual disputes that
so frequently happened in their tumultuous parties of hunting or
drinking were sufficient to inflame the minds of whole nations;
the private feuds of any considerable chieftains diffused itself
among their followers and allies. To chastise the insolent, or to
plunder the defenceless, were alike causes of war. The most
formidable states of Germany affected to encompass their
territories with a wide frontier of solitude and devastation. The
awful distance preserved by their neighbors attested the terror
of their arms, and in some measure defended them from the danger
of unexpected incursions.\footnotemark[77]

\footnotetext[77]{Cæsar de Bell. Gal. l. vi. 23.}

“The Bructeri\footnotemark[771] (it is Tacitus who now speaks) were totally
exterminated by the neighboring tribes,\footnotemark[78] provoked by their
insolence, allured by the hopes of spoil, and perhaps inspired by
the tutelar deities of the empire. Above sixty thousand
barbarians were destroyed; not by the Roman arms, but in our
sight, and for our entertainment. May the nations, enemies of
Rome, ever preserve this enmity to each other! We have now
attained the utmost verge of prosperity,\footnotemark[79] and have nothing left
to demand of fortune, except the discord of the barbarians.”\footnotemark[80]
—These sentiments, less worthy of the humanity than of the
patriotism of Tacitus, express the invariable maxims of the
policy of his countrymen. They deemed it a much safer expedient
to divide than to combat the barbarians, from whose defeat they
could derive neither honor nor advantage. The money and
negotiations of Rome insinuated themselves into the heart of
Germany; and every art of seduction was used with dignity, to
conciliate those nations whom their proximity to the Rhine or
Danube might render the most useful friends as well as the most
troublesome enemies. Chiefs of renown and power were flattered by
the most trifling presents, which they received either as marks
of distinction, or as the instruments of luxury. In civil
dissensions the weaker faction endeavored to strengthen its
interest by entering into secret connections with the governors
of the frontier provinces. Every quarrel among the Germans was
fomented by the intrigues of Rome; and every plan of union and
public good was defeated by the stronger bias of private jealousy
and interest.\footnotemark[81]

\footnotetext[771]{The Bructeri were a non-Suevian tribe, who dwelt
below the duchies of Oldenburgh, and Lauenburgh, on the borders
of the Lippe, and in the Hartz Mountains. It was among them that
the priestess Velleda obtained her renown.—G.}

\footnotetext[78]{They are mentioned, however, in the ivth and vth
centuries by Nazarius, Ammianus, Claudian, \&c., as a tribe of
Franks. See Cluver. Germ. Antiq. l. iii. c. 13.}

\footnotetext[79]{Urgentibus is the common reading; but good sense,
Lipsius, and some Mss. declare for Vergentibus.}

\footnotetext[80]{Tacit Germania, c. 33. The pious Abbé de la
Bleterie is very angry with Tacitus, talks of the devil, who was
a murderer from the beginning, \&c., \&c.}

\footnotetext[81]{Many traces of this policy may be discovered in
Tacitus and Dion: and many more may be inferred from the
principles of human nature.}

The general conspiracy which terrified the Romans under the reign
of Marcus Antoninus, comprehended almost all the nations of
Germany, and even Sarmatia, from the mouth of the Rhine to that
of the Danube.\footnotemark[82] It is impossible for us to determine whether
this hasty confederation was formed by necessity, by reason, or
by passion; but we may rest assured, that the barbarians were
neither allured by the indolence, nor provoked by the ambition,
of the Roman monarch. This dangerous invasion required all the
firmness and vigilance of Marcus. He fixed generals of ability in
the several stations of attack, and assumed in person the conduct
of the most important province on the Upper Danube. After a long
and doubtful conflict, the spirit of the barbarians was subdued.
The Quadi and the Marcomanni,\footnotemark[83] who had taken the lead in the
war, were the most severely punished in its catastrophe. They
were commanded to retire five miles\footnotemark[84] from their own banks of
the Danube, and to deliver up the flower of the youth, who were
immediately sent into Britain, a remote island, where they might
be secure as hostages, and useful as soldiers.\footnotemark[85] On the frequent
rebellions of the Quadi and Marcomanni, the irritated emperor
resolved to reduce their country into the form of a province. His
designs were disappointed by death. This formidable league,
however, the only one that appears in the two first centuries of
the Imperial history, was entirely dissipated, without leaving
any traces behind in Germany.

\footnotetext[82]{Hist. Aug. p. 31. Ammian. Marcellin. l. xxxi. c. 5.
Aurel. Victor. The emperor Marcus was reduced to sell the rich
furniture of the palace, and to enlist slaves and robbers.}

\footnotetext[83]{The Marcomanni, a colony, who, from the banks of
the Rhine occupied Bohemia and Moravia, had once erected a great
and formidable monarchy under their king Maroboduus. See Strabo,
l. vii. [p. 290.] Vell. Pat. ii. 108. Tacit. Annal. ii. 63. *
Note: The Mark-manæn, the March-men or borderers. There seems
little doubt that this was an appellation, rather than a proper
name of a part of the great Suevian or Teutonic race.—M.}

\footnotetext[84]{Mr. Wotton (History of Rome, p. 166) increases the
prohibition to ten times the distance. His reasoning is specious,
but not conclusive. Five miles were sufficient for a fortified
barrier.}

\footnotetext[85]{Dion, l. lxxi. and lxxii.}

In the course of this introductory chapter, we have confined
ourselves to the general outlines of the manners of Germany,
without attempting to describe or to distinguish the various
tribes which filled that great country in the time of Cæsar, of
Tacitus, or of Ptolemy. As the ancient, or as new tribes
successively present themselves in the series of this history, we
shall concisely mention their origin, their situation, and their
particular character. Modern nations are fixed and permanent
societies, connected among themselves by laws and government,
bound to their native soil by art and agriculture. The German
tribes were voluntary and fluctuating associations of soldiers,
almost of savages. The same territory often changed its
inhabitants in the tide of conquest and emigration. The same
communities, uniting in a plan of defence or invasion, bestowed a
new title on their new confederacy. The dissolution of an ancient
confederacy restored to the independent tribes their peculiar but
long-forgotten appellation. A victorious state often communicated
its own name to a vanquished people. Sometimes crowds of
volunteers flocked from all parts to the standard of a favorite
leader; his camp became their country, and some circumstance of
the enterprise soon gave a common denomination to the mixed
multitude. The distinctions of the ferocious invaders were
perpetually varied by themselves, and confounded by the
astonished subjects of the Roman empire.\footnotemark[86]

\footnotetext[86]{See an excellent dissertation on the origin and
migrations of nations, in the Memoires de l’Academie des
Inscriptions, tom. xviii. p. 48—71. It is seldom that the
antiquarian and the philosopher are so happily blended.}

Wars, and the administration of public affairs, are the principal
subjects of history; but the number of persons interested in
these busy scenes is very different, according to the different
condition of mankind. In great monarchies, millions of obedient
subjects pursue their useful occupations in peace and obscurity.
The attention of the writer, as well as of the reader, is solely
confined to a court, a capital, a regular army, and the districts
which happen to be the occasional scene of military operations.
But a state of freedom and barbarism, the season of civil
commotions, or the situation of petty republics,\footnotemark[87] raises almost
every member of the community into action, and consequently into
notice. The irregular divisions, and the restless motions, of the
people of Germany, dazzle our imagination, and seem to multiply
their numbers. The profuse enumeration of kings, of warriors, of
armies and nations, inclines us to forget that the same objects
are continually repeated under a variety of appellations, and
that the most splendid appellations have been frequently lavished
on the most inconsiderable objects.

\footnotetext[87]{Should we suspect that Athens contained only 21,000
citizens, and Sparta no more than 39,000? See Hume and Wallace on
the number of mankind in ancient and modern times. * Note: This
number, though too positively stated, is probably not far wrong,
as an average estimate. On the subject of Athenian population,
see St. Croix, Acad. des Inscrip. xlviii. Bœckh, Public Economy
of Athens, i. 47. Eng Trans, Fynes Clinton, Fasti Hellenici, vol.
i. p. 381. The latter author estimates the citizens of Sparta at
33,000—M.}

