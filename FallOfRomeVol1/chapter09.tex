\chapter{State Of Germany Until The Barbarians.}
\section{Part \thesection.}

\textit{The State Of Germany Till The Invasion Of The Barbarians In The
Time Of The Emperor Decius.}
\vspace{\onelineskip}

The government and religion of Persia have deserved some notice,
from their connection with the decline and fall of the Roman
empire. We shall occasionally mention the Scythian or Sarmatian
tribes,\textsuperscript{1001} which, with their arms and horses, their flocks and
herds, their wives and families, wandered over the immense plains
which spread themselves from the Caspian Sea to the Vistula, from
the confines of Persia to those of Germany. But the warlike
Germans, who first resisted, then invaded, and at length
overturned the Western monarchy of Rome, will occupy a much more
important place in this history, and possess a stronger, and, if
we may use the expression, a more domestic, claim to our
attention and regard. The most civilized nations of modern Europe
issued from the woods of Germany; and in the rude institutions of
those barbarians we may still distinguish the original principles
of our present laws and manners. In their primitive state of
simplicity and independence, the Germans were surveyed by the
discerning eye, and delineated by the masterly pencil, of
Tacitus,\textsuperscript{1002} the first of historians who applied the science of
philosophy to the study of facts. The expressive conciseness of
his descriptions has served to exercise the diligence of
innumerable antiquarians, and to excite the genius and
penetration of the philosophic historians of our own times. The
subject, however various and important, has already been so
frequently, so ably, and so successfully discussed, that it is
now grown familiar to the reader, and difficult to the writer. We
shall therefore content ourselves with observing, and indeed with
repeating, some of the most important circumstances of climate,
of manners, and of institutions, which rendered the wild
barbarians of Germany such formidable enemies to the Roman power.

\pagenote[1001]{The Scythians, even according to the ancients,
are not Sarmatians. It may be doubted whether Gibbon intended to
confound them.—M. ——The Greeks, after having divided the world
into Greeks and barbarians. divided the barbarians into four
great classes, the Celts, the Scythians, the Indians, and the
Ethiopians. They called Celts all the inhabitants of Gaul.
Scythia extended from the Baltic Sea to the Lake Aral: the people
enclosed in the angle to the north-east, between Celtica and
Scythia, were called Celto-Scythians, and the Sarmatians were
placed in the southern part of that angle. But these names of
Celts, of Scythians, of Celto-Scythians, and Sarmatians, were
invented, says Schlozer, by the profound cosmographical ignorance
of the Greeks, and have no real ground; they are purely
geographical divisions, without any relation to the true
affiliation of the different races. Thus all the inhabitants of
Gaul are called Celts by most of the ancient writers; yet Gaul
contained three totally distinct nations, the Belgæ, the
Aquitani, and the Gauls, properly so called. Hi omnes lingua
institutis, legibusque inter se differunt. Cæsar. Com. c. i. It
is thus the Turks call all Europeans Franks. Schlozer, Allgemeine
Nordische Geschichte, p. 289. 1771. Bayer (de Origine et priscis
Sedibus Scytharum, in Opusc. p. 64) says, Primus eorum, de quibus
constat, Ephorus, in quarto historiarum libro, orbem terrarum
inter Scythas, Indos, Æthiopas et Celtas divisit. Fragmentum ejus
loci Cosmas Indicopleustes in topographia Christiana, f. 148,
conservavit. Video igitur Ephorum, cum locorum positus per certa
capita distribuere et explicare constitueret, insigniorum nomina
gentium vastioribus spatiis adhibuisse, nulla mala fraude et
successu infelici. Nam Ephoro quoquomodo dicta pro exploratis
habebant Græci plerique et Romani: ita gliscebat error
posteritate. Igitur tot tamque diversæ stirpis gentes non modo
intra communem quandam regionem definitæ, unum omnes Scytharum
nomen his auctoribus subierunt, sed etiam ab illa regionis
adpellatione in eandem nationem sunt conflatæ. Sic Cimmeriorum
res cum Scythicis, Scytharum cum Sarmaticis, Russicis, Hunnicis,
Tataricis commiscentur.—G.}

\pagenote[1002]{The Germania of Tacitus has been a fruitful
source of hypothesis to the ingenuity of modern writers, who have
endeavored to account for the form of the work and the views of
the author. According to Luden, (Geschichte des T. V. i. 432, and
note,) it contains the unfinished and disarranged for a larger
work. An anonymous writer, supposed by Luden to be M. Becker,
conceives that it was intended as an episode in his larger
history. According to M. Guizot, “Tacite a peint les Germains
comme Montaigne et Rousseau les sauvages, dans un acces d’humeur
contre sa patrie: son livre est une satire des mœurs Romaines,
l’eloquente boutade d’un patriote philosophe qui veut voir la
vertu la, ou il ne rencontre pas la mollesse honteuse et la
depravation savante d’une vielle societe.” Hist. de la
Civilisation Moderne, i. 258.—M.}

Ancient Germany, excluding from its independent limits the
province westward of the Rhine, which had submitted to the Roman
yoke, extended itself over a third part of Europe.\textsuperscript{1} Almost the
whole of modern Germany, Denmark, Norway, Sweden, Finland,
Livonia, Prussia, and the greater part of Poland, were peopled by
the various tribes of one great nation, whose complexion,
manners, and language denoted a common origin, and preserved a
striking resemblance. On the west, ancient Germany was divided by
the Rhine from the Gallic, and on the south, by the Danube, from
the Illyrian, provinces of the empire. A ridge of hills, rising
from the Danube, and called the Carpathian Mountains, covered
Germany on the side of Dacia or Hungary. The eastern frontier was
faintly marked by the mutual fears of the Germans and the
Sarmatians, and was often confounded by the mixture of warring
and confederating tribes of the two nations. In the remote
darkness of the north, the ancients imperfectly descried a frozen
ocean that lay beyond the Baltic Sea, and beyond the Peninsula,
or islands\textsuperscript{1001a} of Scandinavia.

\pagenote[1]{Germany was not of such vast extent. It is from
Cæsar, and more particularly from Ptolemy, (says Gatterer,) that
we can know what was the state of ancient Germany before the wars
with the Romans had changed the positions of the tribes. Germany,
as changed by these wars, has been described by Strabo, Pliny,
and Tacitus. Germany, properly so called, was bounded on the west
by the Rhine, on the east by the Vistula, on the north by the
southern point of Norway, by Sweden, and Esthonia. On the south,
the Maine and the mountains to the north of Bohemia formed the
limits. Before the time of Cæsar, the country between the Maine
and the Danube was partly occupied by the Helvetians and other
Gauls, partly by the Hercynian forest but, from the time of Cæsar
to the great migration, these boundaries were advanced as far as
the Danube, or, what is the same thing, to the Suabian Alps,
although the Hercynian forest still occupied, from north to
south, a space of nine days’ journey on both banks of the Danube.
“Gatterer, Versuch einer all-gemeinen Welt-Geschichte,” p. 424,
edit. de 1792. This vast country was far from being inhabited by
a single nation divided into different tribes of the same origin.
We may reckon three principal races, very distinct in their
language, their origin, and their customs. 1. To the east, the
Slaves or Vandals. 2. To the west, the Cimmerians or Cimbri. 3.
Between the Slaves and Cimbrians, the Germans, properly so
called, the Suevi of Tacitus. The South was inhabited, before
Julius Cæsar, by nations of Gaulish origin, afterwards by the
Suevi.—G. On the position of these nations, the German
antiquaries differ. I. The Slaves, or Sclavonians, or Wendish
tribes, according to Schlozer, were originally settled in parts
of Germany unknown to the Romans, Mecklenburgh, Pomerania,
Brandenburgh, Upper Saxony; and Lusatia. According to Gatterer,
they remained to the east of the Theiss, the Niemen, and the
Vistula, till the third century. The Slaves, according to
Procopius and Jornandes, formed three great divisions. 1. The
Venedi or Vandals, who took the latter name, (the Wenden,) having
expelled the Vandals, properly so called, (a Suevian race, the
conquerors of Africa,) from the country between the Memel and the
Vistula. 2. The Antes, who inhabited between the Dneister and the
Dnieper. 3. The Sclavonians, properly so called, in the north of
Dacia. During the great migration, these races advanced into
Germany as far as the Saal and the Elbe. The Sclavonian language
is the stem from which have issued the Russian, the Polish, the
Bohemian, and the dialects of Lusatia, of some parts of the duchy
of Luneburgh, of Carniola, Carinthia, and Styria, \&c.; those of
Croatia, Bosnia, and Bulgaria. Schlozer, Nordische Geschichte, p.
323, 335. II. The Cimbric race. Adelung calls by this name all
who were not Suevi. This race had passed the Rhine, before the
time of Cæsar, occupied Belgium, and are the Belgæ of Cæsar and
Pliny. The Cimbrians also occupied the Isle of Jutland. The Cymri
of Wales and of Britain are of this race. Many tribes on the
right bank of the Rhine, the Guthini in Jutland, the Usipeti in
Westphalia, the Sigambri in the duchy of Berg, were German
Cimbrians. III. The Suevi, known in very early times by the
Romans, for they are mentioned by L. Corn. Sisenna, who lived 123
years before Christ, (Nonius v. Lancea.) This race, the real
Germans, extended to the Vistula, and from the Baltic to the
Hercynian forest. The name of Suevi was sometimes confined to a
single tribe, as by Cæsar to the Catti. The name of the Suevi has
been preserved in Suabia. These three were the principal races
which inhabited Germany; they moved from east to west, and are
the parent stem of the modern natives. But northern Europe,
according to Schlozer, was not peopled by them alone; other
races, of different origin, and speaking different languages,
have inhabited and left descendants in these countries. The
German tribes called themselves, from very remote times, by the
generic name of Teutons, (Teuten, Deutschen,) which Tacitus
derives from that of one of their gods, Tuisco. It appears more
probable that it means merely men, people. Many savage nations
have given themselves no other name. Thus the Laplanders call
themselves Almag, people; the Samoiedes Nilletz, Nissetsch, men,
\&c. As to the name of Germans, (Germani,) Cæsar found it in use
in Gaul, and adopted it as a word already known to the Romans.
Many of the learned (from a passage of Tacitus, de Mor Germ. c.
2) have supposed that it was only applied to the Teutons after
Cæsar’s time; but Adelung has triumphantly refuted this opinion.
The name of Germans is found in the Fasti Capitolini. See Gruter,
Iscrip. 2899, in which the consul Marcellus, in the year of Rome
531, is said to have defeated the Gauls, the Insubrians, and the
Germans, commanded by Virdomar. See Adelung, Ælt. Geschichte der
Deutsch, p. 102.—Compressed from G.}

\pagenote[1001a]{The modern philosophers of Sweden seem agreed
that the waters of the Baltic gradually sink in a regular
proportion, which they have ventured to estimate at half an inch
every year. Twenty centuries ago the flat country of Scandinavia
must have been covered by the sea; while the high lands rose
above the waters, as so many islands of various forms and
dimensions. Such, indeed, is the notion given us by Mela, Pliny,
and Tacitus, of the vast countries round the Baltic. See in the
Bibliotheque Raisonnee, tom. xl. and xlv. a large abstract of
Dalin’s History of Sweden, composed in the Swedish language. *
Note: Modern geologists have rejected this theory of the
depression of the Baltic, as inconsistent with recent
observation. The considerable changes which have taken place on
its shores, Mr. Lyell, from actual observation now decidedly
attributes to the regular and uniform elevation of the
land.—Lyell’s Geology, b. ii. c. 17—M.}

Some ingenious writers\textsuperscript{2} have suspected that Europe was much
colder formerly than it is at present; and the most ancient
descriptions of the climate of Germany tend exceedingly to
confirm their theory. The general complaints of intense frost and
eternal winter are perhaps little to be regarded, since we have
no method of reducing to the accurate standard of the
thermometer, the feelings, or the expressions, of an orator born
in the happier regions of Greece or Asia. But I shall select two
remarkable circumstances of a less equivocal nature. 1. The great
rivers which covered the Roman provinces, the Rhine and the
Danube, were frequently frozen over, and capable of supporting
the most enormous weights. The barbarians, who often chose that
severe season for their inroads, transported, without
apprehension or danger, their numerous armies, their cavalry, and
their heavy wagons, over a vast and solid bridge of ice.\textsuperscript{3} Modern
ages have not presented an instance of a like phenomenon. 2. The
reindeer, that useful animal, from whom the savage of the North
derives the best comforts of his dreary life, is of a
constitution that supports, and even requires, the most intense
cold. He is found on the rock of Spitzberg, within ten degrees of
the Pole; he seems to delight in the snows of Lapland and
Siberia: but at present he cannot subsist, much less multiply, in
any country to the south of the Baltic.\textsuperscript{4} In the time of Cæsar
the reindeer, as well as the elk and the wild bull, was a native
of the Hercynian forest, which then overshadowed a great part of
Germany and Poland.\textsuperscript{5} The modern improvements sufficiently
explain the causes of the diminution of the cold. These immense
woods have been gradually cleared, which intercepted from the
earth the rays of the sun.\textsuperscript{6} The morasses have been drained, and,
in proportion as the soil has been cultivated, the air has become
more temperate. Canada, at this day, is an exact picture of
ancient Germany. Although situated in the same parallel with the
finest provinces of France and England, that country experiences
the most rigorous cold. The reindeer are very numerous, the
ground is covered with deep and lasting snow, and the great river
of St. Lawrence is regularly frozen, in a season when the waters
of the Seine and the Thames are usually free from ice.\textsuperscript{7}

\pagenote[2]{In particular, Mr. Hume, the Abbé du Bos, and M.
Pelloutier. Hist. des Celtes, tom. i.}

\pagenote[3]{Diodorus Siculus, l. v. p. 340, edit. Wessel.
Herodian, l. vi. p. 221. Jornandes, c. 55. On the banks of the
Danube, the wine, when brought to table, was frequently frozen
into great lumps, frusta vini. Ovid. Epist. ex Ponto, l. iv. 7,
9, 10. Virgil. Georgic. l. iii. 355. The fact is confirmed by a
soldier and a philosopher, who had experienced the intense cold
of Thrace. See Xenophon, Anabasis, l. vii. p. 560, edit.
Hutchinson. Note: The Danube is constantly frozen over. At Pesth
the bridge is usually taken up, and the traffic and communication
between the two banks carried on over the ice. The Rhine is
likewise in many parts passable at least two years out of five.
Winter campaigns are so unusual, in modern warfare, that I
recollect but one instance of an army crossing either river on
the ice. In the thirty years’ war, (1635,) Jan van Werth, an
Imperialist partisan, crossed the Rhine from Heidelberg on the
ice with 5000 men, and surprised Spiers. Pichegru’s memorable
campaign, (1794-5,) when the freezing of the Meuse and Waal
opened Holland to his conquests, and his cavalry and artillery
attacked the ships frozen in, on the Zuyder Zee, was in a winter
of unprecedented severity.—M. 1845.}

\pagenote[4]{Buffon, Histoire Naturelle, tom. xii. p. 79, 116.}

\pagenote[5]{Cæsar de Bell. Gallic. vi. 23, \&c. The most
inquisitive of the Germans were ignorant of its utmost limits,
although some of them had travelled in it more than sixty days’
journey. * Note: The passage of Cæsar, “parvis renonum tegumentis
utuntur,” is obscure, observes Luden, (Geschichte des Teutschen
Volkes,) and insufficient to prove the reindeer to have existed
in Germany. It is supported however, by a fragment of Sallust.
Germani intectum rhenonibus corpus tegunt.—M. It has been
suggested to me that Cæsar (as old Gesner supposed) meant the
reindeer in the following description. Est bos cervi figura cujus
a media fronte inter aures unum cornu existit, excelsius magisque
directum (divaricatum, qu?) his quæ nobis nota sunt cornibus. At
ejus summo, sicut palmæ, rami quam late diffunduntur. Bell.
vi.—M. 1845.}

\pagenote[6]{Cluverius (Germania Antiqua, l. iii. c. 47)
investigates the small and scattered remains of the Hercynian
wood.}

\pagenote[7]{Charlevoix, Histoire du Canada.}

It is difficult to ascertain, and easy to exaggerate, the
influence of the climate of ancient Germany over the minds and
bodies of the natives. Many writers have supposed, and most have
allowed, though, as it should seem, without any adequate proof,
that the rigorous cold of the North was favorable to long life
and generative vigor, that the women were more fruitful, and the
human species more prolific, than in warmer or more temperate
climates.\textsuperscript{8} We may assert, with greater confidence, that the keen
air of Germany formed the large and masculine limbs of the
natives, who were, in general, of a more lofty stature than the
people of the South,\textsuperscript{9} gave them a kind of strength better
adapted to violent exertions than to patient labor, and inspired
them with constitutional bravery, which is the result of nerves
and spirits. The severity of a winter campaign, that chilled the
courage of the Roman troops, was scarcely felt by these hardy
children of the North,\textsuperscript{10} who, in their turn, were unable to
resist the summer heats, and dissolved away in languor and
sickness under the beams of an Italian sun.\textsuperscript{11}

\pagenote[8]{Olaus Rudbeck asserts that the Swedish women often
bear ten or twelve children, and not uncommonly twenty or thirty;
but the authority of Rudbeck is much to be suspected.}

\pagenote[9]{In hos artus, in hæc corpora, quæ miramur,
excrescunt. Tæit Germania, 3, 20. Cluver. l. i. c. 14.}

\pagenote[10]{Plutarch. in Mario. The Cimbri, by way of
amusement, often did down mountains of snow on their broad
shields.}

\pagenote[11]{The Romans made war in all climates, and by their
excellent discipline were in a great measure preserved in health
and vigor. It may be remarked, that man is the only animal which
can live and multiply in every country from the equator to the
poles. The hog seems to approach the nearest to our species in
that privilege.}

\section{Part \thesection.}

There is not anywhere upon the globe a large tract of country,
which we have discovered destitute of inhabitants, or whose first
population can be fixed with any degree of historical certainty.
And yet, as the most philosophic minds can seldom refrain from
investigating the infancy of great nations, our curiosity
consumes itself in toilsome and disappointed efforts. When
Tacitus considered the purity of the German blood, and the
forbidding aspect of the country, he was disposed to pronounce
those barbarians \textit{Indigenæ}, or natives of the soil. We may allow
with safety, and perhaps with truth, that ancient Germany was not
originally peopled by any foreign colonies already formed into a
political society;\textsuperscript{12} but that the name and nation received their
existence from the gradual union of some wandering savages of the
Hercynian woods. To assert those savages to have been the
spontaneous production of the earth which they inhabited would be
a rash inference, condemned by religion, and unwarranted by
reason.

\pagenote[12]{Facit. Germ. c. 3. The emigration of the Gauls
followed the course of the Danube, and discharged itself on
Greece and Asia. Tacitus could discover only one inconsiderable
tribe that retained any traces of a Gallic origin. * Note: The
Gothini, who must not be confounded with the Gothi, a Suevian
tribe. In the time of Cæsar many other tribes of Gaulish origin
dwelt along the course of the Danube, who could not long resist
the attacks of the Suevi. The Helvetians, who dwelt on the
borders of the Black Forest, between the Maine and the Danube,
had been expelled long before the time of Cæsar. He mentions also
the Volci Tectosagi, who came from Languedoc and settled round
the Black Forest. The Boii, who had penetrated into that forest,
and also have left traces of their name in Bohemia, were subdued
in the first century by the Marcomanni. The Boii settled in
Noricum, were mingled afterwards with the Lombards, and received
the name of Boio Arii (Bavaria) or Boiovarii: var, in some German
dialects, appearing to mean remains, descendants. Compare Malte
B-m, Geography, vol. i. p. 410, edit 1832—M.}

Such rational doubt is but ill suited with the genius of popular
vanity. Among the nations who have adopted the Mosaic history of
the world, the ark of Noah has been of the same use, as was
formerly to the Greeks and Romans the siege of Troy. On a narrow
basis of acknowledged truth, an immense but rude superstructure
of fable has been erected; and the wild Irishman,\textsuperscript{13} as well as
the wild Tartar,\textsuperscript{14} could point out the individual son of Japhet,
from whose loins his ancestors were lineally descended. The last
century abounded with antiquarians of profound learning and easy
faith, who, by the dim light of legends and traditions, of
conjectures and etymologies, conducted the great grandchildren of
Noah from the Tower of Babel to the extremities of the globe. Of
these judicious critics, one of the most entertaining was Olaus
Rudbeck, professor in the university of Upsal.\textsuperscript{15} Whatever is
celebrated either in history or fable this zealous patriot
ascribes to his country. From Sweden (which formed so
considerable a part of ancient Germany) the Greeks themselves
derived their alphabetical characters, their astronomy, and their
religion. Of that delightful region (for such it appeared to the
eyes of a native) the Atlantis of Plato, the country of the
Hyperboreans, the gardens of the Hesperides, the Fortunate
Islands, and even the Elysian Fields, were all but faint and
imperfect transcripts. A clime so profusely favored by Nature
could not long remain desert after the flood. The learned Rudbeck
allows the family of Noah a few years to multiply from eight to
about twenty thousand persons. He then disperses them into small
colonies to replenish the earth, and to propagate the human
species. The German or Swedish detachment (which marched, if I am
not mistaken, under the command of Askenaz, the son of Gomer, the
son of Japhet) distinguished itself by a more than common
diligence in the prosecution of this great work. The northern
hive cast its swarms over the greatest part of Europe, Africa,
and Asia; and (to use the author’s metaphor) the blood circulated
from the extremities to the heart.

\pagenote[13]{According to Dr. Keating, (History of Ireland, p.
13, 14,) the giant Portholanus, who was the son of Seara, the son
of Esra, the son of Sru, the son of Framant, the son of
Fathaclan, the son of Magog, the son of Japhet, the son of Noah,
landed on the coast of Munster the 14th day of May, in the year
of the world one thousand nine hundred and seventy-eight. Though
he succeeded in his great enterprise, the loose behavior of his
wife rendered his domestic life very unhappy, and provoked him to
such a degree, that he killed—her favorite greyhound. This, as
the learned historian very properly observes, was the first
instance of female falsehood and infidelity ever known in
Ireland.}

\pagenote[14]{Genealogical History of the Tartars, by Abulghazi
Bahadur Khan.}

\pagenote[15]{His work, entitled Atlantica, is uncommonly scarce.
Bayle has given two most curious extracts from it. Republique des
Lettres Janvier et Fevrier, 1685.}

But all this well-labored system of German antiquities is
annihilated by a single fact, too well attested to admit of any
doubt, and of too decisive a nature to leave room for any reply.
The Germans, in the age of Tacitus, were unacquainted with the
use of letters;\textsuperscript{16} and the use of letters is the principal
circumstance that distinguishes a civilized people from a herd of
savages incapable of knowledge or reflection. Without that
artificial help, the human memory soon dissipates or corrupts the
ideas intrusted to her charge; and the nobler faculties of the
mind, no longer supplied with models or with materials, gradually
forget their powers; the judgment becomes feeble and lethargic,
the imagination languid or irregular. Fully to apprehend this
important truth, let us attempt, in an improved society, to
calculate the immense distance between the man of learning and
the \textit{illiterate} peasant. The former, by reading and reflection,
multiplies his own experience, and lives in distant ages and
remote countries; whilst the latter, rooted to a single spot, and
confined to a few years of existence, surpasses but very little
his fellow-laborer, the ox, in the exercise of his mental
faculties. The same, and even a greater, difference will be found
between nations than between individuals; and we may safely
pronounce, that without some species of writing, no people has
ever preserved the faithful annals of their history, ever made
any considerable progress in the abstract sciences, or ever
possessed, in any tolerable degree of perfection, the useful and
agreeable arts of life.

\pagenote[16]{Tacit. Germ. ii. 19. Literarum secreta viri pariter
ac fœminæ ignorant. We may rest contented with this decisive
authority, without entering into the obscure disputes concerning
the antiquity of the Runic characters. The learned Celsius, a
Swede, a scholar, and a philosopher, was of opinion, that they
were nothing more than the Roman letters, with the curves changed
into straight lines for the ease of engraving. See Pelloutier,
Histoire des Celtes, l. ii. c. 11. Dictionnaire Diplomatique,
tom. i. p. 223. We may add, that the oldest Runic inscriptions
are supposed to be of the third century, and the most ancient
writer who mentions the Runic characters is Venan tius
Frotunatus, (Carm. vii. 18,) who lived towards the end of the
sixth century. Barbara fraxineis pingatur Runa tabellis. * Note:
The obscure subject of the Runic characters has exercised the
industry and ingenuity of the modern scholars of the north. There
are three distinct theories; one, maintained by Schlozer,
(Nordische Geschichte, p. 481, \&c.,) who considers their sixteen
letters to be a corruption of the Roman alphabet, post-Christian
in their date, and Schlozer would attribute their introduction
into the north to the Alemanni. The second, that of Frederick
Schlegel, (Vorlesungen uber alte und neue Literatur,) supposes
that these characters were left on the coasts of the
Mediterranean and Northern Seas by the Phœnicians, preserved by
the priestly castes, and employed for purposes of magic. Their
common origin from the Phœnician would account for heir
similarity to the Roman letters. The last, to which we incline,
claims much higher and more venerable antiquity for the Runic,
and supposes them to have been the original characters of the
Indo-Teutonic tribes, brought from the East, and preserved among
the different races of that stock. See Ueber Deutsche Runen von
W. C. Grimm, 1821. A Memoir by Dr. Legis. Fundgruben des alten
Nordens. Foreign Quarterly Review vol. ix. p. 438.—M.}

Of these arts, the ancient Germans were wretchedly destitute.\textsuperscript{1601}
They passed their lives in a state of ignorance and poverty,
which it has pleased some declaimers to dignify with the
appellation of virtuous simplicity. Modern Germany is said to
contain about two thousand three hundred walled towns.\textsuperscript{17} In a
much wider extent of country, the geographer Ptolemy could
discover no more than ninety places which he decorates with the
name of cities;\textsuperscript{18} though, according to our ideas, they would but
ill deserve that splendid title. We can only suppose them to have
been rude fortifications, constructed in the centre of the woods,
and designed to secure the women, children, and cattle, whilst
the warriors of the tribe marched out to repel a sudden invasion.\textsuperscript{19}
But Tacitus asserts, as a well-known fact, that the Germans,
in his time, had \textit{no} cities;\textsuperscript{20} and that they affected to
despise the works of Roman industry, as places of confinement
rather than of security.\textsuperscript{21} Their edifices were not even
contiguous, or formed into regular villas;\textsuperscript{22} each barbarian
fixed his independent dwelling on the spot to which a plain, a
wood, or a stream of fresh water, had induced him to give the
preference. Neither stone, nor brick, nor tiles, were employed in
these slight habitations.\textsuperscript{23} They were indeed no more than low
huts, of a circular figure, built of rough timber, thatched with
straw, and pierced at the top to leave a free passage for the
smoke. In the most inclement winter, the hardy German was
satisfied with a scanty garment made of the skin of some animal.
The nations who dwelt towards the North clothed themselves in
furs; and the women manufactured for their own use a coarse kind
of linen.\textsuperscript{24} The game of various sorts, with which the forests of
Germany were plentifully stocked, supplied its inhabitants with
food and exercise.\textsuperscript{25} Their monstrous herds of cattle, less
remarkable indeed for their beauty than for their utility,\textsuperscript{26}
formed the principal object of their wealth. A small quantity of
corn was the only produce exacted from the earth; the use of
orchards or artificial meadows was unknown to the Germans; nor
can we expect any improvements in agriculture from a people,
whose prosperity every year experienced a general change by a new
division of the arable lands, and who, in that strange operation,
avoided disputes, by suffering a great part of their territory to
lie waste and without tillage.\textsuperscript{27}

\pagenote[1601]{Luden (the author of the Geschichte des Teutschen
Volkes) has surpassed most writers in his patriotic enthusiasm
for the virtues and noble manners of his ancestors. Even the cold
of the climate, and the want of vines and fruit trees, as well as
the barbarism of the inhabitants, are calumnies of the luxurious
Italians. M. Guizot, on the other side, (in his Histoire de la
Civilisation, vol. i. p. 272, \&c.,) has drawn a curious parallel
between the Germans of Tacitus and the North American
Indians.—M.}

\pagenote[17]{Recherches Philosophiques sur les Americains, tom.
iii. p. 228. The author of that very curious work is, if I am not
misinformed, a German by birth. (De Pauw.)}

\pagenote[18]{The Alexandrian Geographer is often criticized by
the accurate Cluverius.}

\pagenote[19]{See Cæsar, and the learned Mr. Whitaker in his
History of Manchester, vol. i.}

\pagenote[20]{Tacit. Germ. 15.}

\pagenote[21]{When the Germans commanded the Ubii of Cologne to
cast off the Roman yoke, and with their new freedom to resume
their ancient manners, they insisted on the immediate demolition
of the walls of the colony. “Postulamus a vobis, muros coloniæ,
munimenta servitii, detrahatis; etiam fera animalia, si clausa
teneas, virtutis obliviscuntur.” Tacit. Hist. iv. 64.}

\pagenote[22]{The straggling villages of Silesia are several
miles in length. See Cluver. l. i. c. 13.}

\pagenote[23]{One hundred and forty years after Tacitus, a few
more regular structures were erected near the Rhine and Danube.
Herodian, l. vii. p. 234.}

\pagenote[24]{Tacit. Germ. 17.}

\pagenote[25]{Tacit. Germ. 5.}

\pagenote[26]{Cæsar de Bell. Gall. vi. 21.}

\pagenote[27]{Tacit. Germ. 26. Cæsar, vi. 22.}

Gold, silver, and iron, were extremely scarce in Germany. Its
barbarous inhabitants wanted both skill and patience to
investigate those rich veins of silver, which have so liberally
rewarded the attention of the princes of Brunswick and Saxony.
Sweden, which now supplies Europe with iron, was equally ignorant
of its own riches; and the appearance of the arms of the Germans
furnished a sufficient proof how little iron they were able to
bestow on what they must have deemed the noblest use of that
metal. The various transactions of peace and war had introduced
some Roman coins (chiefly silver) among the borderers of the
Rhine and Danube; but the more distant tribes were absolutely
unacquainted with the use of money, carried on their confined
traffic by the exchange of commodities, and prized their rude
earthen vessels as of equal value with the silver vases, the
presents of Rome to their princes and ambassadors.\textsuperscript{28} To a mind
capable of reflection, such leading facts convey more
instruction, than a tedious detail of subordinate circumstances.
The value of money has been settled by general consent to express
our wants and our property, as letters were invented to express
our ideas; and both these institutions, by giving a more active
energy to the powers and passions of human nature, have
contributed to multiply the objects they were designed to
represent. The use of gold and silver is in a great measure
factitious; but it would be impossible to enumerate the important
and various services which agriculture, and all the arts, have
received from iron, when tempered and fashioned by the operation
of fire and the dexterous hand of man. Money, in a word, is the
most universal incitement, iron the most powerful instrument, of
human industry; and it is very difficult to conceive by what
means a people, neither actuated by the one, nor seconded by the
other, could emerge from the grossest barbarism.\textsuperscript{29}

\pagenote[28]{Tacit. Germ. 6.}

\pagenote[29]{It is said that the Mexicans and Peruvians, without
the use of either money or iron, had made a very great progress
in the arts. Those arts, and the monuments they produced, have
been strangely magnified. See Recherches sur les Americains, tom.
ii. p. 153, \&c}

If we contemplate a savage nation in any part of the globe, a
supine indolence and a carelessness of futurity will be found to
constitute their general character. In a civilized state every
faculty of man is expanded and exercised; and the great chain of
mutual dependence connects and embraces the several members of
society. The most numerous portion of it is employed in constant
and useful labor. The select few, placed by fortune above that
necessity, can, however, fill up their time by the pursuits of
interest or glory, by the improvement of their estate or of their
understanding, by the duties, the pleasures, and even the follies
of social life. The Germans were not possessed of these varied
resources. The care of the house and family, the management of
the land and cattle, were delegated to the old and the infirm, to
women and slaves. The lazy warrior, destitute of every art that
might employ his leisure hours, consumed his days and nights in
the animal gratifications of sleep and food. And yet, by a
wonderful diversity of nature, (according to the remark of a
writer who had pierced into its darkest recesses,) the same
barbarians are by turns the most indolent and the most restless
of mankind. They delight in sloth, they detest tranquility.\textsuperscript{30}
The languid soul, oppressed with its own weight, anxiously
required some new and powerful sensation; and war and danger were
the only amusements adequate to its fierce temper. The sound that
summoned the German to arms was grateful to his ear. It roused
him from his uncomfortable lethargy, gave him an active pursuit,
and, by strong exercise of the body, and violent emotions of the
mind, restored him to a more lively sense of his existence. In
the dull intervals of peace, these barbarians were immoderately
addicted to deep gaming and excessive drinking; both of which, by
different means, the one by inflaming their passions, the other
by extinguishing their reason, alike relieved them from the pain
of thinking. They gloried in passing whole days and nights at
table; and the blood of friends and relations often stained their
numerous and drunken assemblies.\textsuperscript{31} Their debts of honor (for in
that light they have transmitted to us those of play) they
discharged with the most romantic fidelity. The desperate
gamester, who had staked his person and liberty on a last throw
of the dice, patiently submitted to the decision of fortune, and
suffered himself to be bound, chastised, and sold into remote
slavery, by his weaker but more lucky antagonist.\textsuperscript{32}

\pagenote[30]{Tacit. Germ. 15.}

\pagenote[31]{Tacit. Germ. 22, 23.}

\pagenote[32]{Id. 24. The Germans might borrow the arts of play
from the Romans, but the passion is wonderfully inherent in the
human species.}

Strong beer, a liquor extracted with very little art from wheat
or barley, and \textit{corrupted} (as it is strongly expressed by
Tacitus) into a certain semblance of wine, was sufficient for the
gross purposes of German debauchery. But those who had tasted the
rich wines of Italy, and afterwards of Gaul, sighed for that more
delicious species of intoxication. They attempted not, however,
(as has since been executed with so much success,) to naturalize
the vine on the banks of the Rhine and Danube; nor did they
endeavor to procure by industry the materials of an advantageous
commerce. To solicit by labor what might be ravished by arms, was
esteemed unworthy of the German spirit.\textsuperscript{33} The intemperate thirst
of strong liquors often urged the barbarians to invade the
provinces on which art or nature had bestowed those much envied
presents. The Tuscan who betrayed his country to the Celtic
nations, attracted them into Italy by the prospect of the rich
fruits and delicious wines, the productions of a happier climate.\textsuperscript{34}
And in the same manner the German auxiliaries, invited into
France during the civil wars of the sixteenth century, were
allured by the promise of plenteous quarters in the provinces of
Champaigne and Burgundy.\textsuperscript{35} Drunkenness, the most illiberal, but
not the most dangerous of \textit{our} vices, was sometimes capable, in
a less civilized state of mankind, of occasioning a battle, a
war, or a revolution.

\pagenote[33]{Tacit. Germ. 14.}

\pagenote[34]{Plutarch. in Camillo. T. Liv. v. 33.}

\pagenote[35]{Dubos. Hist. de la Monarchie Francoise, tom. i. p.
193.}

The climate of ancient Germany has been modified, and the soil
fertilized, by the labor of ten centuries from the time of
Charlemagne. The same extent of ground which at present
maintains, in ease and plenty, a million of husbandmen and
artificers, was unable to supply a hundred thousand lazy warriors
with the simple necessaries of life.\textsuperscript{36} The Germans abandoned
their immense forests to the exercise of hunting, employed in
pasturage the most considerable part of their lands, bestowed on
the small remainder a rude and careless cultivation, and then
accused the scantiness and sterility of a country that refused to
maintain the multitude of its inhabitants. When the return of
famine severely admonished them of the importance of the arts,
the national distress was sometimes alleviated by the emigration
of a third, perhaps, or a fourth part of their youth.\textsuperscript{37} The
possession and the enjoyment of property are the pledges which
bind a civilized people to an improved country. But the Germans,
who carried with them what they most valued, their arms, their
cattle, and their women, cheerfully abandoned the vast silence of
their woods for the unbounded hopes of plunder and conquest. The
innumerable swarms that issued, or seemed to issue, from the
great storehouse of nations, were multiplied by the fears of the
vanquished, and by the credulity of succeeding ages. And from
facts thus exaggerated, an opinion was gradually established, and
has been supported by writers of distinguished reputation, that,
in the age of Cæsar and Tacitus, the inhabitants of the North
were far more numerous than they are in our days.\textsuperscript{38} A more
serious inquiry into the causes of population seems to have
convinced modern philosophers of the falsehood, and indeed the
impossibility, of the supposition. To the names of Mariana and of
Machiavel,\textsuperscript{39} we can oppose the equal names of Robertson and
Hume.\textsuperscript{40}

\pagenote[36]{The Helvetian nation, which issued from a country
called Switzerland, contained, of every age and sex, 368,000
persons, (Cæsar de Bell. Gal. i. 29.) At present, the number of
people in the Pays de Vaud (a small district on the banks of the
Leman Lake, much more distinguished for politeness than for
industry) amounts to 112,591. See an excellent tract of M. Muret,
in the Memoires de la Societe de Born.}

\pagenote[37]{Paul Diaconus, c. 1, 2, 3. Machiavel, Davila, and
the rest of Paul’s followers, represent these emigrations too
much as regular and concerted measures.}

\pagenote[38]{Sir William Temple and Montesquieu have indulged,
on this subject, the usual liveliness of their fancy.}

\pagenote[39]{Machiavel, Hist. di Firenze, l. i. Mariana, Hist.
Hispan. l. v. c. 1}

\pagenote[40]{Robertson’s Charles V. Hume’s Political Essays.
Note: It is a wise observation of Malthus, that these nations
“were not populous in proportion to the land they occupied, but
to the food they produced.” They were prolific from their pure
morals and constitutions, but their institutions were not
calculated to produce food for those whom they brought into
being.—M—1845.}

A warlike nation like the Germans, without either cities,
letters, arts, or money, found some compensation for this savage
state in the enjoyment of liberty. Their poverty secured their
freedom, since our desires and our possessions are the strongest
fetters of despotism. “Among the Suiones (says Tacitus) riches
are held in honor. They are \textit{therefore} subject to an absolute
monarch, who, instead of intrusting his people with the free use
of arms, as is practised in the rest of Germany, commits them to
the safe custody, not of a citizen, or even of a freedman, but of
a slave. The neighbors of the Suiones, the Sitones, are sunk even
below servitude; they obey a woman.”\textsuperscript{41} In the mention of these
exceptions, the great historian sufficiently acknowledges the
general theory of government. We are only at a loss to conceive
by what means riches and despotism could penetrate into a remote
corner of the North, and extinguish the generous flame that
blazed with such fierceness on the frontier of the Roman
provinces, or how the ancestors of those Danes and Norwegians, so
distinguished in latter ages by their unconquered spirit, could
thus tamely resign the great character of German liberty.\textsuperscript{42} Some
tribes, however, on the coast of the Baltic, acknowledged the
authority of kings, though without relinquishing the rights of
men,\textsuperscript{43} but in the far greater part of Germany, the form of
government was a democracy, tempered, indeed, and controlled, not
so much by general and positive laws, as by the occasional
ascendant of birth or valor, of eloquence or superstition.\textsuperscript{44}

\pagenote[41]{Tacit. German. 44, 45. Freinshemius (who dedicated
his supplement to Livy to Christina of Sweden) thinks proper to
be very angry with the Roman who expressed so very little
reverence for Northern queens. Note: The Suiones and the Sitones
are the ancient inhabitants of Scandinavia, their name may be
traced in that of Sweden; they did not belong to the race of the
Suevi, but that of the non-Suevi or Cimbri, whom the Suevi, in
very remote times, drove back part to the west, part to the
north; they were afterwards mingled with Suevian tribes, among
others the Goths, who have traces of their name and power in the
isle of Gothland.—G}

\pagenote[42]{May we not suspect that superstition was the parent
of despotism? The descendants of Odin, (whose race was not
extinct till the year 1060) are said to have reigned in Sweden
above a thousand years. The temple of Upsal was the ancient seat
of religion and empire. In the year 1153 I find a singular law,
prohibiting the use and profession of arms to any except the
king’s guards. Is it not probable that it was colored by the
pretence of reviving an old institution? See Dalin’s History of
Sweden in the Bibliotheque Raisonneo tom. xl. and xlv.}

\pagenote[43]{Tacit. Germ. c. 43.}

\pagenote[44]{Id. c. 11, 12, 13, \& c.}

Civil governments, in their first institution, are voluntary
associations for mutual defence. To obtain the desired end, it is
absolutely necessary that each individual should conceive himself
obliged to submit his private opinions and actions to the
judgment of the greater number of his associates. The German
tribes were contented with this rude but liberal outline of
political society. As soon as a youth, born of free parents, had
attained the age of manhood, he was introduced into the general
council of his countrymen, solemnly invested with a shield and
spear, and adopted as an equal and worthy member of the military
commonwealth. The assembly of the warriors of the tribe was
convened at stated seasons, or on sudden emergencies. The trial
of public offences, the election of magistrates, and the great
business of peace and war, were determined by its independent
voice. Sometimes indeed, these important questions were
previously considered and prepared in a more select council of
the principal chieftains.\textsuperscript{45} The magistrates might deliberate and
persuade, the people only could resolve and execute; and the
resolutions of the Germans were for the most part hasty and
violent. Barbarians accustomed to place their freedom in
gratifying the present passion, and their courage in overlooking
all future consequences, turned away with indignant contempt from
the remonstrances of justice and policy, and it was the practice
to signify by a hollow murmur their dislike of such timid
counsels. But whenever a more popular orator proposed to
vindicate the meanest citizen from either foreign or domestic
injury, whenever he called upon his fellow-countrymen to assert
the national honor, or to pursue some enterprise full of danger
and glory, a loud clashing of shields and spears expressed the
eager applause of the assembly. For the Germans always met in
arms, and it was constantly to be dreaded, lest an irregular
multitude, inflamed with faction and strong liquors, should use
those arms to enforce, as well as to declare, their furious
resolves. We may recollect how often the diets of Poland have
been polluted with blood, and the more numerous party has been
compelled to yield to the more violent and seditious.\textsuperscript{46}

\pagenote[45]{Grotius changes an expression of Tacitus,
pertractantur into Prætractantur. The correction is equally just
and ingenious.}

\pagenote[46]{Even in our ancient parliament, the barons often
carried a question, not so much by the number of votes, as by
that of their armed followers.}

A general of the tribe was elected on occasions of danger; and,
if the danger was pressing and extensive, several tribes
concurred in the choice of the same general. The bravest warrior
was named to lead his countrymen into the field, by his example
rather than by his commands. But this power, however limited, was
still invidious. It expired with the war, and in time of peace
the German tribes acknowledged not any supreme chief.\textsuperscript{47}
\textit{Princes} were, however, appointed, in the general assembly, to
administer justice, or rather to compose differences,\textsuperscript{48} in their
respective districts. In the choice of these magistrates, as much
regard was shown to birth as to merit.\textsuperscript{49} To each was assigned,
by the public, a guard, and a council of a hundred persons, and
the first of the princes appears to have enjoyed a preeminence of
rank and honor which sometimes tempted the Romans to compliment
him with the regal title.\textsuperscript{50}

\pagenote[47]{Cæsar de Bell. Gal. vi. 23.}

\pagenote[48]{Minuunt controversias, is a very happy expression
of Cæsar’s.}

\pagenote[49]{Reges ex nobilitate, duces ex virtute sumunt. Tacit
Germ. 7}

\pagenote[50]{Cluver. Germ. Ant. l. i. c. 38.}

The comparative view of the powers of the magistrates, in two
remarkable instances, is alone sufficient to represent the whole
system of German manners. The disposal of the landed property
within their district was absolutely vested in their hands, and
they distributed it every year according to a new division.\textsuperscript{51} At
the same time they were not authorized to punish with death, to
imprison, or even to strike a private citizen.\textsuperscript{52} A people thus
jealous of their persons, and careless of their possessions, must
have been totally destitute of industry and the arts, but
animated with a high sense of honor and independence.

\pagenote[51]{Cæsar, vi. 22. Tacit Germ. 26.}

\pagenote[52]{Tacit. Germ. 7.}

\section{Part \thesection.}

The Germans respected only those duties which they imposed on
themselves. The most obscure soldier resisted with disdain the
authority of the magistrates. “The noblest youths blushed not to
be numbered among the faithful companions of some renowned chief,
to whom they devoted their arms and service. A noble emulation
prevailed among the companions to obtain the first place in the
esteem of their chief; amongst the chiefs, to acquire the
greatest number of valiant companions. To be ever surrounded by a
band of select youths was the pride and strength of the chiefs,
their ornament in peace, their defence in war. The glory of such
distinguished heroes diffused itself beyond the narrow limits of
their own tribe. Presents and embassies solicited their
friendship, and the fame of their arms often insured victory to
the party which they espoused. In the hour of danger it was
shameful for the chief to be surpassed in valor by his
companions; shameful for the companions not to equal the valor of
their chief. To survive his fall in battle was indelible infamy.
To protect his person, and to adorn his glory with the trophies
of their own exploits, were the most sacred of their duties. The
chiefs combated for victory, the companions for the chief. The
noblest warriors, whenever their native country was sunk into the
laziness of peace, maintained their numerous bands in some
distant scene of action, to exercise their restless spirit, and
to acquire renown by voluntary dangers. Gifts worthy of
soldiers—the warlike steed, the bloody and ever victorious
lance—were the rewards which the companions claimed from the
liberality of their chief. The rude plenty of his hospitable
board was the only pay that \textit{he} could bestow, or \textit{they} would
accept. War, rapine, and the free-will offerings of his friends,
supplied the materials of this munificence.”\textsuperscript{53} This institution,
however it might accidentally weaken the several republics,
invigorated the general character of the Germans, and even
ripened amongst them all the virtues of which barbarians are
susceptible; the faith and valor, the hospitality and the
courtesy, so conspicuous long afterwards in the ages of chivalry.

The honorable gifts, bestowed by the chief on his brave
companions, have been supposed, by an ingenious writer, to
contain the first rudiments of the fiefs, distributed after the
conquest of the Roman provinces, by the barbarian lords among
their vassals, with a similar duty of homage and military
service.\textsuperscript{54} These conditions are, however, very repugnant to the
maxims of the ancient Germans, who delighted in mutual presents,
but without either imposing, or accepting, the weight of
obligations.\textsuperscript{55}

\pagenote[53]{Tacit. Germ. 13, 14.}

\pagenote[54]{Esprit des Loix, l. xxx. c. 3. The brilliant
imagination of Montesquieu is corrected, however, by the dry,
cold reason of the Abbé de Mably. Observations sur l’Histoire de
France, tom. i. p. 356.}

\pagenote[55]{Gaudent muneribus, sed nec data imputant, nec
acceptis obligautur. Tacit. Germ. c. 21.}

“In the days of chivalry, or more properly of romance, all the
men were brave and all the women were chaste;” and
notwithstanding the latter of these virtues is acquired and
preserved with much more difficulty than the former, it is
ascribed, almost without exception, to the wives of the ancient
Germans. Polygamy was not in use, except among the princes, and
among them only for the sake of multiplying their alliances.
Divorces were prohibited by manners rather than by laws.
Adulteries were punished as rare and inexpiable crimes; nor was
seduction justified by example and fashion.\textsuperscript{56} We may easily
discover that Tacitus indulges an honest pleasure in the contrast
of barbarian virtue with the dissolute conduct of the Roman
ladies; yet there are some striking circumstances that give an
air of truth, or at least probability, to the conjugal faith and
chastity of the Germans.

\pagenote[56]{The adulteress was whipped through the village.
Neither wealth nor beauty could inspire compassion, or procure
her a second husband. 18, 19.}

Although the progress of civilization has undoubtedly contributed
to assuage the fiercer passions of human nature, it seems to have
been less favorable to the virtue of chastity, whose most
dangerous enemy is the softness of the mind. The refinements of
life corrupt while they polish the intercourse of the sexes. The
gross appetite of love becomes most dangerous when it is
elevated, or rather, indeed, disguised by sentimental passion.
The elegance of dress, of motion, and of manners, gives a lustre
to beauty, and inflames the senses through the imagination.
Luxurious entertainments, midnight dances, and licentious
spectacles, present at once temptation and opportunity to female
frailty.\textsuperscript{57} From such dangers the unpolished wives of the
barbarians were secured by poverty, solitude, and the painful
cares of a domestic life. The German huts, open, on every side,
to the eye of indiscretion or jealousy, were a better safeguard
of conjugal fidelity than the walls, the bolts, and the eunuchs
of a Persian harem. To this reason another may be added of a more
honorable nature. The Germans treated their women with esteem and
confidence, consulted them on every occasion of importance, and
fondly believed, that in their breasts resided a sanctity and
wisdom more than human. Some of the interpreters of fate, such as
Velleda, in the Batavian war, governed, in the name of the deity,
the fiercest nations of Germany.\textsuperscript{58} The rest of the sex, without
being adored as goddesses, were respected as the free and equal
companions of soldiers; associated even by the marriage ceremony
to a life of toil, of danger, and of glory.\textsuperscript{59} In their great
invasions, the camps of the barbarians were filled with a
multitude of women, who remained firm and undaunted amidst the
sound of arms, the various forms of destruction, and the
honorable wounds of their sons and husbands.\textsuperscript{60} Fainting armies
of Germans have, more than once, been driven back upon the enemy
by the generous despair of the women, who dreaded death much less
than servitude. If the day was irrecoverably lost, they well knew
how to deliver themselves and their children, with their own
hands, from an insulting victor.\textsuperscript{61} Heroines of such a cast may
claim our admiration; but they were most assuredly neither lovely
nor very susceptible of love. Whilst they affected to emulate the
stern virtues of \textit{man}, they must have resigned that attractive
softness in which principally consist the charm and weakness of
\textit{woman}. Conscious pride taught the German females to suppress
every tender emotion that stood in competition with honor, and
the first honor of the sex has ever been that of chastity. The
sentiments and conduct of these high-spirited matrons may, at
once, be considered as a cause, as an effect, and as a proof of
the general character of the nation. Female courage, however it
may be raised by fanaticism, or confirmed by habit, can be only a
faint and imperfect imitation of the manly valor that
distinguishes the age or country in which it may be found.

\pagenote[57]{Ovid employs two hundred lines in the research of
places the most favorable to love. Above all, he considers the
theatre as the best adapted to collect the beauties of Rome, and
to melt them into tenderness and sensuality,}

\pagenote[58]{Tacit. Germ. iv. 61, 65.}

\pagenote[59]{The marriage present was a yoke of oxen, horses,
and arms. See Germ. c. 18. Tacitus is somewhat too florid on the
subject.}

\pagenote[60]{The change of exigere into exugere is a most
excellent correction.}

\pagenote[61]{Tacit. Germ. c. 7. Plutarch in Mario. Before the
wives of the Teutones destroyed themselves and their children,
they had offered to surrender, on condition that they should be
received as the slaves of the vestal virgins.}

The religious system of the Germans (if the wild opinions of
savages can deserve that name) was dictated by their wants, their
fears, and their ignorance.\textsuperscript{62} They adored the great visible
objects and agents of nature, the Sun and the Moon, the Fire and
the Earth; together with those imaginary deities, who were
supposed to preside over the most important occupations of human
life. They were persuaded, that, by some ridiculous arts of
divination, they could discover the will of the superior beings,
and that human sacrifices were the most precious and acceptable
offering to their altars. Some applause has been hastily bestowed
on the sublime notion, entertained by that people, of the Deity,
whom they neither confined within the walls of the temple, nor
represented by any human figure; but when we recollect, that the
Germans were unskilled in architecture, and totally unacquainted
with the art of sculpture, we shall readily assign the true
reason of a scruple, which arose not so much from a superiority
of reason, as from a want of ingenuity. The only temples in
Germany were dark and ancient groves, consecrated by the
reverence of succeeding generations. Their secret gloom, the
imagined residence of an invisible power, by presenting no
distinct object of fear or worship, impressed the mind with a
still deeper sense of religious horror;\textsuperscript{63} and the priests, rude
and illiterate as they were, had been taught by experience the
use of every artifice that could preserve and fortify impressions
so well suited to their own interest.

\pagenote[62]{Tacitus has employed a few lines, and Cluverius one
hundred and twenty-four pages, on this obscure subject. The
former discovers in Germany the gods of Greece and Rome. The
latter is positive, that, under the emblems of the sun, the moon,
and the fire, his pious ancestors worshipped the Trinity in
unity}

\pagenote[63]{The sacred wood, described with such sublime horror
by Lucan, was in the neighborhood of Marseilles; but there were
many of the same kind in Germany. * Note: The ancient Germans had
shapeless idols, and, when they began to build more settled
habitations, they raised also temples, such as that to the
goddess Teufana, who presided over divination. See Adelung, Hist.
of Ane Germans, p 296—G}

The same ignorance, which renders barbarians incapable of
conceiving or embracing the useful restraints of laws, exposes
them naked and unarmed to the blind terrors of superstition. The
German priests, improving this favorable temper of their
countrymen, had assumed a jurisdiction even in temporal concerns,
which the magistrate could not venture to exercise; and the
haughty warrior patiently submitted to the lash of correction,
when it was inflicted, not by any human power, but by the
immediate order of the god of war.\textsuperscript{64} The defects of civil policy
were sometimes supplied by the interposition of ecclesiastical
authority. The latter was constantly exerted to maintain silence
and decency in the popular assemblies; and was sometimes extended
to a more enlarged concern for the national welfare. A solemn
procession was occasionally celebrated in the present countries
of Mecklenburgh and Pomerania. The unknown symbol of the \textit{Earth},
covered with a thick veil, was placed on a carriage drawn by
cows; and in this manner the goddess, whose common residence was
in the Isles of Rugen, visited several adjacent tribes of her
worshippers. During her progress the sound of war was hushed,
quarrels were suspended, arms laid aside, and the restless
Germans had an opportunity of tasting the blessings of peace and
harmony.\textsuperscript{65} The \textit{truce of God}, so often and so ineffectually
proclaimed by the clergy of the eleventh century, was an obvious
imitation of this ancient custom.\textsuperscript{66}

\pagenote[64]{Tacit. Germania, c. 7.}

\pagenote[65]{Tacit. Germania, c. 40.}

\pagenote[66]{See Dr. Robertson’s History of Charles V. vol. i.
note 10.}

But the influence of religion was far more powerful to inflame,
than to moderate, the fierce passions of the Germans. Interest
and fanaticism often prompted its ministers to sanctify the most
daring and the most unjust enterprises, by the approbation of
Heaven, and full assurances of success. The consecrated
standards, long revered in the groves of superstition, were
placed in the front of the battle;\textsuperscript{67} and the hostile army was
devoted with dire execrations to the gods of war and of thunder.\textsuperscript{68}
In the faith of soldiers (and such were the Germans) cowardice
is the most unpardonable of sins. A brave man was the worthy
favorite of their martial deities; the wretch who had lost his
shield was alike banished from the religious and civil assemblies
of his countrymen. Some tribes of the north seem to have embraced
the doctrine of transmigration,\textsuperscript{69} others imagined a gross
paradise of immortal drunkenness.\textsuperscript{70} All agreed that a life spent
in arms, and a glorious death in battle, were the best
preparations for a happy futurity, either in this or in another
world.

\pagenote[67]{Tacit. Germania, c. 7. These standards were only
the heads of wild beasts.}

\pagenote[68]{See an instance of this custom, Tacit. Annal. xiii.
57.}

\pagenote[69]{Cæsar Diodorus, and Lucan, seem to ascribe this
doctrine to the Gauls, but M. Pelloutier (Histoire des Celtes, l.
iii. c. 18) labors to reduce their expressions to a more orthodox
sense.}

\pagenote[70]{Concerning this gross but alluring doctrine of the
Edda, see Fable xx. in the curious version of that book,
published by M. Mallet, in his Introduction to the History of
Denmark.}

The immortality so vainly promised by the priests, was, in some
degree, conferred by the bards. That singular order of men has
most deservedly attracted the notice of all who have attempted to
investigate the antiquities of the Celts, the Scandinavians, and
the Germans. Their genius and character, as well as the reverence
paid to that important office, have been sufficiently
illustrated. But we cannot so easily express, or even conceive,
the enthusiasm of arms and glory which they kindled in the breast
of their audience. Among a polished people a taste for poetry is
rather an amusement of the fancy than a passion of the soul. And
yet, when in calm retirement we peruse the combats described by
Homer or Tasso, we are insensibly seduced by the fiction, and
feel a momentary glow of martial ardor. But how faint, how cold
is the sensation which a peaceful mind can receive from solitary
study! It was in the hour of battle, or in the feast of victory,
that the bards celebrated the glory of the heroes of ancient
days, the ancestors of those warlike chieftains, who listened
with transport to their artless but animated strains. The view of
arms and of danger heightened the effect of the military song;
and the passions which it tended to excite, the desire of fame,
and the contempt of death, were the habitual sentiments of a
German mind.\textsuperscript{71} \textsuperscript{711}

\pagenote[71]{See Tacit. Germ. c. 3. Diod. Sicul. l. v. Strabo,
l. iv. p. 197. The classical reader may remember the rank of
Demodocus in the Phæacian court, and the ardor infused by Tyrtæus
into the fainting Spartans. Yet there is little probability that
the Greeks and the Germans were the same people. Much learned
trifling might be spared, if our antiquarians would condescend to
reflect, that similar manners will naturally be produced by
similar situations.}

\pagenote[711]{Besides these battle songs, the Germans sang at
their festival banquets, (Tac. Ann. i. 65,) and around the bodies
of their slain heroes. King Theodoric, of the tribe of the Goths,
killed in a battle against Attila, was honored by songs while he
was borne from the field of battle. Jornandes, c. 41. The same
honor was paid to the remains of Attila. Ibid. c. 49. According
to some historians, the Germans had songs also at their weddings;
but this appears to me inconsistent with their customs, in which
marriage was no more than the purchase of a wife. Besides, there
is but one instance of this, that of the Gothic king, Ataulph,
who sang himself the nuptial hymn when he espoused Placidia,
sister of the emperors Arcadius and Honorius, (Olympiodor. p. 8.)
But this marriage was celebrated according to the Roman rites, of
which the nuptial songs formed a part. Adelung, p. 382.—G.
Charlemagne is said to have collected the national songs of the
ancient Germans. Eginhard, Vit. Car. Mag.—M.}

Such was the situation, and such were the manners of the ancient
Germans. Their climate, their want of learning, of arts, and of
laws, their notions of honor, of gallantry, and of religion,
their sense of freedom, impatience of peace, and thirst of
enterprise, all contributed to form a people of military heroes.
And yet we find, that during more than two hundred and fifty
years that elapsed from the defeat of Varus to the reign of
Decius, these formidable barbarians made few considerable
attempts, and not any material impression on the luxurious, and
enslaved provinces of the empire. Their progress was checked by
their want of arms and discipline, and their fury was diverted by
the intestine divisions of ancient Germany. I. It has been
observed, with ingenuity, and not without truth, that the command
of iron soon gives a nation the command of gold. But the rude
tribes of Germany, alike destitute of both those valuable metals,
were reduced slowly to acquire, by their unassisted strength, the
possession of the one as well as the other. The face of a German
army displayed their poverty of iron. Swords, and the longer kind
of lances, they could seldom use. Their \textit{frameæ} (as they called
them in their own language) were long spears headed with a sharp
but narrow iron point, and which, as occasion required, they
either darted from a distance, or pushed in close onset. With
this spear, and with a shield, their cavalry was contented. A
multitude of darts, scattered\textsuperscript{72} with incredible force, were an
additional resource of the infantry. Their military dress, when
they wore any, was nothing more than a loose mantle. A variety of
colors was the only ornament of their wooden or osier shields.
Few of the chiefs were distinguished by cuirasses, scarcely any
by helmets. Though the horses of Germany were neither beautiful,
swift, nor practised in the skilful evolutions of the Roman
manege, several of the nations obtained renown by their cavalry;
but, in general, the principal strength of the Germans consisted
in their infantry,\textsuperscript{73} which was drawn up in several deep columns,
according to the distinction of tribes and families. Impatient of
fatigue and delay, these half-armed warriors rushed to battle
with dissonant shouts and disordered ranks; and sometimes, by the
effort of native valor, prevailed over the constrained and more
artificial bravery of the Roman mercenaries. But as the
barbarians poured forth their whole souls on the first onset,
they knew not how to rally or to retire. A repulse was a sure
defeat; and a defeat was most commonly total destruction. When we
recollect the complete armor of the Roman soldiers, their
discipline, exercises, evolutions, fortified camps, and military
engines, it appears a just matter of surprise, how the naked and
unassisted valor of the barbarians could dare to encounter, in
the field, the strength of the legions, and the various troops of
the auxiliaries, which seconded their operations. The contest was
too unequal, till the introduction of luxury had enervated the
vigor, and a spirit of disobedience and sedition had relaxed the
discipline, of the Roman armies. The introduction of barbarian
auxiliaries into those armies, was a measure attended with very
obvious dangers, as it might gradually instruct the Germans in
the arts of war and of policy. Although they were admitted in
small numbers and with the strictest precaution, the example of
Civilis was proper to convince the Romans, that the danger was
not imaginary, and that their precautions were not always
sufficient.\textsuperscript{74} During the civil wars that followed the death of
Nero, that artful and intrepid Batavian, whom his enemies
condescended to compare with Hannibal and Sertorius,\textsuperscript{75} formed a
great design of freedom and ambition. Eight Batavian cohorts
renowned in the wars of Britain and Italy, repaired to his
standard. He introduced an army of Germans into Gaul, prevailed
on the powerful cities of Treves and Langres to embrace his
cause, defeated the legions, destroyed their fortified camps, and
employed against the Romans the military knowledge which he had
acquired in their service. When at length, after an obstinate
struggle, he yielded to the power of the empire, Civilis secured
himself and his country by an honorable treaty. The Batavians
still continued to occupy the islands of the Rhine,\textsuperscript{76} the
allies, not the servants, of the Roman monarchy.

\pagenote[72]{Missilia spargunt, Tacit. Germ. c. 6. Either that
historian used a vague expression, or he meant that they were
thrown at random.}

\pagenote[73]{It was their principal distinction from the
Sarmatians, who generally fought on horseback.}

\pagenote[74]{The relation of this enterprise occupies a great
part of the fourth and fifth books of the History of Tacitus, and
is more remarkable for its eloquence than perspicuity. Sir Henry
Saville has observed several inaccuracies.}

\pagenote[75]{Tacit. Hist. iv. 13. Like them he had lost an eye.}

\pagenote[76]{It was contained between the two branches of the
old Rhine, as they subsisted before the face of the country was
changed by art and nature. See Cluver German. Antiq. l. iii. c.
30, 37.}

II. The strength of ancient Germany appears formidable, when we
consider the effects that might have been produced by its united
effort. The wide extent of country might very possibly contain a
million of warriors, as all who were of age to bear arms were of
a temper to use them. But this fierce multitude, incapable of
concerting or executing any plan of national greatness, was
agitated by various and often hostile intentions. Germany was
divided into more than forty independent states; and, even in
each state, the union of the several tribes was extremely loose
and precarious. The barbarians were easily provoked; they knew
not how to forgive an injury, much less an insult; their
resentments were bloody and implacable. The casual disputes that
so frequently happened in their tumultuous parties of hunting or
drinking were sufficient to inflame the minds of whole nations;
the private feuds of any considerable chieftains diffused itself
among their followers and allies. To chastise the insolent, or to
plunder the defenceless, were alike causes of war. The most
formidable states of Germany affected to encompass their
territories with a wide frontier of solitude and devastation. The
awful distance preserved by their neighbors attested the terror
of their arms, and in some measure defended them from the danger
of unexpected incursions.\textsuperscript{77}

\pagenote[77]{Cæsar de Bell. Gal. l. vi. 23.}

“The Bructeri\textsuperscript{771} (it is Tacitus who now speaks) were totally
exterminated by the neighboring tribes,\textsuperscript{78} provoked by their
insolence, allured by the hopes of spoil, and perhaps inspired by
the tutelar deities of the empire. Above sixty thousand
barbarians were destroyed; not by the Roman arms, but in our
sight, and for our entertainment. May the nations, enemies of
Rome, ever preserve this enmity to each other! We have now
attained the utmost verge of prosperity,\textsuperscript{79} and have nothing left
to demand of fortune, except the discord of the barbarians.”\textsuperscript{80}
—These sentiments, less worthy of the humanity than of the
patriotism of Tacitus, express the invariable maxims of the
policy of his countrymen. They deemed it a much safer expedient
to divide than to combat the barbarians, from whose defeat they
could derive neither honor nor advantage. The money and
negotiations of Rome insinuated themselves into the heart of
Germany; and every art of seduction was used with dignity, to
conciliate those nations whom their proximity to the Rhine or
Danube might render the most useful friends as well as the most
troublesome enemies. Chiefs of renown and power were flattered by
the most trifling presents, which they received either as marks
of distinction, or as the instruments of luxury. In civil
dissensions the weaker faction endeavored to strengthen its
interest by entering into secret connections with the governors
of the frontier provinces. Every quarrel among the Germans was
fomented by the intrigues of Rome; and every plan of union and
public good was defeated by the stronger bias of private jealousy
and interest.\textsuperscript{81}

\pagenote[771]{The Bructeri were a non-Suevian tribe, who dwelt
below the duchies of Oldenburgh, and Lauenburgh, on the borders
of the Lippe, and in the Hartz Mountains. It was among them that
the priestess Velleda obtained her renown.—G.}

\pagenote[78]{They are mentioned, however, in the ivth and vth
centuries by Nazarius, Ammianus, Claudian, \&c., as a tribe of
Franks. See Cluver. Germ. Antiq. l. iii. c. 13.}

\pagenote[79]{Urgentibus is the common reading; but good sense,
Lipsius, and some Mss. declare for Vergentibus.}

\pagenote[80]{Tacit Germania, c. 33. The pious Abbé de la
Bleterie is very angry with Tacitus, talks of the devil, who was
a murderer from the beginning, \&c., \&c.}

\pagenote[81]{Many traces of this policy may be discovered in
Tacitus and Dion: and many more may be inferred from the
principles of human nature.}

The general conspiracy which terrified the Romans under the reign
of Marcus Antoninus, comprehended almost all the nations of
Germany, and even Sarmatia, from the mouth of the Rhine to that
of the Danube.\textsuperscript{82} It is impossible for us to determine whether
this hasty confederation was formed by necessity, by reason, or
by passion; but we may rest assured, that the barbarians were
neither allured by the indolence, nor provoked by the ambition,
of the Roman monarch. This dangerous invasion required all the
firmness and vigilance of Marcus. He fixed generals of ability in
the several stations of attack, and assumed in person the conduct
of the most important province on the Upper Danube. After a long
and doubtful conflict, the spirit of the barbarians was subdued.
The Quadi and the Marcomanni,\textsuperscript{83} who had taken the lead in the
war, were the most severely punished in its catastrophe. They
were commanded to retire five miles\textsuperscript{84} from their own banks of
the Danube, and to deliver up the flower of the youth, who were
immediately sent into Britain, a remote island, where they might
be secure as hostages, and useful as soldiers.\textsuperscript{85} On the frequent
rebellions of the Quadi and Marcomanni, the irritated emperor
resolved to reduce their country into the form of a province. His
designs were disappointed by death. This formidable league,
however, the only one that appears in the two first centuries of
the Imperial history, was entirely dissipated, without leaving
any traces behind in Germany.

\pagenote[82]{Hist. Aug. p. 31. Ammian. Marcellin. l. xxxi. c. 5.
Aurel. Victor. The emperor Marcus was reduced to sell the rich
furniture of the palace, and to enlist slaves and robbers.}

\pagenote[83]{The Marcomanni, a colony, who, from the banks of
the Rhine occupied Bohemia and Moravia, had once erected a great
and formidable monarchy under their king Maroboduus. See Strabo,
l. vii. [p. 290.] Vell. Pat. ii. 108. Tacit. Annal. ii. 63. *
Note: The Mark-manæn, the March-men or borderers. There seems
little doubt that this was an appellation, rather than a proper
name of a part of the great Suevian or Teutonic race.—M.}

\pagenote[84]{Mr. Wotton (History of Rome, p. 166) increases the
prohibition to ten times the distance. His reasoning is specious,
but not conclusive. Five miles were sufficient for a fortified
barrier.}

\pagenote[85]{Dion, l. lxxi. and lxxii.}

In the course of this introductory chapter, we have confined
ourselves to the general outlines of the manners of Germany,
without attempting to describe or to distinguish the various
tribes which filled that great country in the time of Cæsar, of
Tacitus, or of Ptolemy. As the ancient, or as new tribes
successively present themselves in the series of this history, we
shall concisely mention their origin, their situation, and their
particular character. Modern nations are fixed and permanent
societies, connected among themselves by laws and government,
bound to their native soil by art and agriculture. The German
tribes were voluntary and fluctuating associations of soldiers,
almost of savages. The same territory often changed its
inhabitants in the tide of conquest and emigration. The same
communities, uniting in a plan of defence or invasion, bestowed a
new title on their new confederacy. The dissolution of an ancient
confederacy restored to the independent tribes their peculiar but
long-forgotten appellation. A victorious state often communicated
its own name to a vanquished people. Sometimes crowds of
volunteers flocked from all parts to the standard of a favorite
leader; his camp became their country, and some circumstance of
the enterprise soon gave a common denomination to the mixed
multitude. The distinctions of the ferocious invaders were
perpetually varied by themselves, and confounded by the
astonished subjects of the Roman empire.\textsuperscript{86}

\pagenote[86]{See an excellent dissertation on the origin and
migrations of nations, in the Memoires de l’Academie des
Inscriptions, tom. xviii. p. 48—71. It is seldom that the
antiquarian and the philosopher are so happily blended.}

Wars, and the administration of public affairs, are the principal
subjects of history; but the number of persons interested in
these busy scenes is very different, according to the different
condition of mankind. In great monarchies, millions of obedient
subjects pursue their useful occupations in peace and obscurity.
The attention of the writer, as well as of the reader, is solely
confined to a court, a capital, a regular army, and the districts
which happen to be the occasional scene of military operations.
But a state of freedom and barbarism, the season of civil
commotions, or the situation of petty republics,\textsuperscript{87} raises almost
every member of the community into action, and consequently into
notice. The irregular divisions, and the restless motions, of the
people of Germany, dazzle our imagination, and seem to multiply
their numbers. The profuse enumeration of kings, of warriors, of
armies and nations, inclines us to forget that the same objects
are continually repeated under a variety of appellations, and
that the most splendid appellations have been frequently lavished
on the most inconsiderable objects.

\pagenote[87]{Should we suspect that Athens contained only 21,000
citizens, and Sparta no more than 39,000? See Hume and Wallace on
the number of mankind in ancient and modern times. * Note: This
number, though too positively stated, is probably not far wrong,
as an average estimate. On the subject of Athenian population,
see St. Croix, Acad. des Inscrip. xlviii. Bœckh, Public Economy
of Athens, i. 47. Eng Trans, Fynes Clinton, Fasti Hellenici, vol.
i. p. 381. The latter author estimates the citizens of Sparta at
33,000—M.}

