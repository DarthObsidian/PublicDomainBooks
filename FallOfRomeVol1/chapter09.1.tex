\chapter{State Of Germany Until The Barbarians.}
\section{Part \thesection.}

\textit{The State Of Germany Till The Invasion Of The Barbarians In The
Time Of The Emperor Decius.}
\vspace{\onelineskip}

The government and religion of Persia have deserved some notice,
from their connection with the decline and fall of the Roman
empire. We shall occasionally mention the Scythian or Sarmatian
tribes,\footnotemark[1001] which, with their arms and horses, their flocks and
herds, their wives and families, wandered over the immense plains
which spread themselves from the Caspian Sea to the Vistula, from
the confines of Persia to those of Germany. But the warlike
Germans, who first resisted, then invaded, and at length
overturned the Western monarchy of Rome, will occupy a much more
important place in this history, and possess a stronger, and, if
we may use the expression, a more domestic, claim to our
attention and regard. The most civilized nations of modern Europe
issued from the woods of Germany; and in the rude institutions of
those barbarians we may still distinguish the original principles
of our present laws and manners. In their primitive state of
simplicity and independence, the Germans were surveyed by the
discerning eye, and delineated by the masterly pencil, of
Tacitus,\footnotemark[1002] the first of historians who applied the science of
philosophy to the study of facts. The expressive conciseness of
his descriptions has served to exercise the diligence of
innumerable antiquarians, and to excite the genius and
penetration of the philosophic historians of our own times. The
subject, however various and important, has already been so
frequently, so ably, and so successfully discussed, that it is
now grown familiar to the reader, and difficult to the writer. We
shall therefore content ourselves with observing, and indeed with
repeating, some of the most important circumstances of climate,
of manners, and of institutions, which rendered the wild
barbarians of Germany such formidable enemies to the Roman power.

\footnotetext[1001]{The Scythians, even according to the ancients,
are not Sarmatians. It may be doubted whether Gibbon intended to
confound them.—M. ——The Greeks, after having divided the world
into Greeks and barbarians. divided the barbarians into four
great classes, the Celts, the Scythians, the Indians, and the
Ethiopians. They called Celts all the inhabitants of Gaul.
Scythia extended from the Baltic Sea to the Lake Aral: the people
enclosed in the angle to the north-east, between Celtica and
Scythia, were called Celto-Scythians, and the Sarmatians were
placed in the southern part of that angle. But these names of
Celts, of Scythians, of Celto-Scythians, and Sarmatians, were
invented, says Schlozer, by the profound cosmographical ignorance
of the Greeks, and have no real ground; they are purely
geographical divisions, without any relation to the true
affiliation of the different races. Thus all the inhabitants of
Gaul are called Celts by most of the ancient writers; yet Gaul
contained three totally distinct nations, the Belgæ, the
Aquitani, and the Gauls, properly so called. Hi omnes lingua
institutis, legibusque inter se differunt. Cæsar. Com. c. i. It
is thus the Turks call all Europeans Franks. Schlozer, Allgemeine
Nordische Geschichte, p. 289. 1771. Bayer (de Origine et priscis
Sedibus Scytharum, in Opusc. p. 64) says, Primus eorum, de quibus
constat, Ephorus, in quarto historiarum libro, orbem terrarum
inter Scythas, Indos, Æthiopas et Celtas divisit. Fragmentum ejus
loci Cosmas Indicopleustes in topographia Christiana, f. 148,
conservavit. Video igitur Ephorum, cum locorum positus per certa
capita distribuere et explicare constitueret, insigniorum nomina
gentium vastioribus spatiis adhibuisse, nulla mala fraude et
successu infelici. Nam Ephoro quoquomodo dicta pro exploratis
habebant Græci plerique et Romani: ita gliscebat error
posteritate. Igitur tot tamque diversæ stirpis gentes non modo
intra communem quandam regionem definitæ, unum omnes Scytharum
nomen his auctoribus subierunt, sed etiam ab illa regionis
adpellatione in eandem nationem sunt conflatæ. Sic Cimmeriorum
res cum Scythicis, Scytharum cum Sarmaticis, Russicis, Hunnicis,
Tataricis commiscentur.—G.}

\footnotetext[1002]{The Germania of Tacitus has been a fruitful
source of hypothesis to the ingenuity of modern writers, who have
endeavored to account for the form of the work and the views of
the author. According to Luden, (Geschichte des T. V. i. 432, and
note,) it contains the unfinished and disarranged for a larger
work. An anonymous writer, supposed by Luden to be M. Becker,
conceives that it was intended as an episode in his larger
history. According to M. Guizot, “Tacite a peint les Germains
comme Montaigne et Rousseau les sauvages, dans un acces d’humeur
contre sa patrie: son livre est une satire des mœurs Romaines,
l’eloquente boutade d’un patriote philosophe qui veut voir la
vertu la, ou il ne rencontre pas la mollesse honteuse et la
depravation savante d’une vielle societe.” Hist. de la
Civilisation Moderne, i. 258.—M.}

Ancient Germany, excluding from its independent limits the
province westward of the Rhine, which had submitted to the Roman
yoke, extended itself over a third part of Europe.\footnotemark[1] Almost the
whole of modern Germany, Denmark, Norway, Sweden, Finland,
Livonia, Prussia, and the greater part of Poland, were peopled by
the various tribes of one great nation, whose complexion,
manners, and language denoted a common origin, and preserved a
striking resemblance. On the west, ancient Germany was divided by
the Rhine from the Gallic, and on the south, by the Danube, from
the Illyrian, provinces of the empire. A ridge of hills, rising
from the Danube, and called the Carpathian Mountains, covered
Germany on the side of Dacia or Hungary. The eastern frontier was
faintly marked by the mutual fears of the Germans and the
Sarmatians, and was often confounded by the mixture of warring
and confederating tribes of the two nations. In the remote
darkness of the north, the ancients imperfectly descried a frozen
ocean that lay beyond the Baltic Sea, and beyond the Peninsula,
or islands 1001a of Scandinavia.

\footnotetext[1]{Germany was not of such vast extent. It is from
Cæsar, and more particularly from Ptolemy, (says Gatterer,) that
we can know what was the state of ancient Germany before the wars
with the Romans had changed the positions of the tribes. Germany,
as changed by these wars, has been described by Strabo, Pliny,
and Tacitus. Germany, properly so called, was bounded on the west
by the Rhine, on the east by the Vistula, on the north by the
southern point of Norway, by Sweden, and Esthonia. On the south,
the Maine and the mountains to the north of Bohemia formed the
limits. Before the time of Cæsar, the country between the Maine
and the Danube was partly occupied by the Helvetians and other
Gauls, partly by the Hercynian forest but, from the time of Cæsar
to the great migration, these boundaries were advanced as far as
the Danube, or, what is the same thing, to the Suabian Alps,
although the Hercynian forest still occupied, from north to
south, a space of nine days’ journey on both banks of the Danube.
“Gatterer, Versuch einer all-gemeinen Welt-Geschichte,” p. 424,
edit. de 1792. This vast country was far from being inhabited by
a single nation divided into different tribes of the same origin.
We may reckon three principal races, very distinct in their
language, their origin, and their customs. 1. To the east, the
Slaves or Vandals. 2. To the west, the Cimmerians or Cimbri. 3.
Between the Slaves and Cimbrians, the Germans, properly so
called, the Suevi of Tacitus. The South was inhabited, before
Julius Cæsar, by nations of Gaulish origin, afterwards by the
Suevi.—G. On the position of these nations, the German
antiquaries differ. I. The Slaves, or Sclavonians, or Wendish
tribes, according to Schlozer, were originally settled in parts
of Germany unknown to the Romans, Mecklenburgh, Pomerania,
Brandenburgh, Upper Saxony; and Lusatia. According to Gatterer,
they remained to the east of the Theiss, the Niemen, and the
Vistula, till the third century. The Slaves, according to
Procopius and Jornandes, formed three great divisions. 1. The
Venedi or Vandals, who took the latter name, (the Wenden,) having
expelled the Vandals, properly so called, (a Suevian race, the
conquerors of Africa,) from the country between the Memel and the
Vistula. 2. The Antes, who inhabited between the Dneister and the
Dnieper. 3. The Sclavonians, properly so called, in the north of
Dacia. During the great migration, these races advanced into
Germany as far as the Saal and the Elbe. The Sclavonian language
is the stem from which have issued the Russian, the Polish, the
Bohemian, and the dialects of Lusatia, of some parts of the duchy
of Luneburgh, of Carniola, Carinthia, and Styria, \&c.; those of
Croatia, Bosnia, and Bulgaria. Schlozer, Nordische Geschichte, p.
323, 335. II. The Cimbric race. Adelung calls by this name all
who were not Suevi. This race had passed the Rhine, before the
time of Cæsar, occupied Belgium, and are the Belgæ of Cæsar and
Pliny. The Cimbrians also occupied the Isle of Jutland. The Cymri
of Wales and of Britain are of this race. Many tribes on the
right bank of the Rhine, the Guthini in Jutland, the Usipeti in
Westphalia, the Sigambri in the duchy of Berg, were German
Cimbrians. III. The Suevi, known in very early times by the
Romans, for they are mentioned by L. Corn. Sisenna, who lived 123
years before Christ, (Nonius v. Lancea.) This race, the real
Germans, extended to the Vistula, and from the Baltic to the
Hercynian forest. The name of Suevi was sometimes confined to a
single tribe, as by Cæsar to the Catti. The name of the Suevi has
been preserved in Suabia. These three were the principal races
which inhabited Germany; they moved from east to west, and are
the parent stem of the modern natives. But northern Europe,
according to Schlozer, was not peopled by them alone; other
races, of different origin, and speaking different languages,
have inhabited and left descendants in these countries. The
German tribes called themselves, from very remote times, by the
generic name of Teutons, (Teuten, Deutschen,) which Tacitus
derives from that of one of their gods, Tuisco. It appears more
probable that it means merely men, people. Many savage nations
have given themselves no other name. Thus the Laplanders call
themselves Almag, people; the Samoiedes Nilletz, Nissetsch, men,
\&c. As to the name of Germans, (Germani,) Cæsar found it in use
in Gaul, and adopted it as a word already known to the Romans.
Many of the learned (from a passage of Tacitus, de Mor Germ. c.
2) have supposed that it was only applied to the Teutons after
Cæsar’s time; but Adelung has triumphantly refuted this opinion.
The name of Germans is found in the Fasti Capitolini. See Gruter,
Iscrip. 2899, in which the consul Marcellus, in the year of Rome
531, is said to have defeated the Gauls, the Insubrians, and the
Germans, commanded by Virdomar. See Adelung, Ælt. Geschichte der
Deutsch, p. 102.—Compressed from G.}

\footnotetext[1001]{The modern philosophers of Sweden seem agreed
that the waters of the Baltic gradually sink in a regular
proportion, which they have ventured to estimate at half an inch
every year. Twenty centuries ago the flat country of Scandinavia
must have been covered by the sea; while the high lands rose
above the waters, as so many islands of various forms and
dimensions. Such, indeed, is the notion given us by Mela, Pliny,
and Tacitus, of the vast countries round the Baltic. See in the
Bibliotheque Raisonnee, tom. xl. and xlv. a large abstract of
Dalin’s History of Sweden, composed in the Swedish language. *
Note: Modern geologists have rejected this theory of the
depression of the Baltic, as inconsistent with recent
observation. The considerable changes which have taken place on
its shores, Mr. Lyell, from actual observation now decidedly
attributes to the regular and uniform elevation of the
land.—Lyell’s Geology, b. ii. c. 17—M.}

Some ingenious writers\footnotemark[2] have suspected that Europe was much
colder formerly than it is at present; and the most ancient
descriptions of the climate of Germany tend exceedingly to
confirm their theory. The general complaints of intense frost and
eternal winter are perhaps little to be regarded, since we have
no method of reducing to the accurate standard of the
thermometer, the feelings, or the expressions, of an orator born
in the happier regions of Greece or Asia. But I shall select two
remarkable circumstances of a less equivocal nature. 1. The great
rivers which covered the Roman provinces, the Rhine and the
Danube, were frequently frozen over, and capable of supporting
the most enormous weights. The barbarians, who often chose that
severe season for their inroads, transported, without
apprehension or danger, their numerous armies, their cavalry, and
their heavy wagons, over a vast and solid bridge of ice.\footnotemark[3] Modern
ages have not presented an instance of a like phenomenon. 2. The
reindeer, that useful animal, from whom the savage of the North
derives the best comforts of his dreary life, is of a
constitution that supports, and even requires, the most intense
cold. He is found on the rock of Spitzberg, within ten degrees of
the Pole; he seems to delight in the snows of Lapland and
Siberia: but at present he cannot subsist, much less multiply, in
any country to the south of the Baltic.\footnotemark[4] In the time of Cæsar
the reindeer, as well as the elk and the wild bull, was a native
of the Hercynian forest, which then overshadowed a great part of
Germany and Poland.\footnotemark[5] The modern improvements sufficiently
explain the causes of the diminution of the cold. These immense
woods have been gradually cleared, which intercepted from the
earth the rays of the sun.\footnotemark[6] The morasses have been drained, and,
in proportion as the soil has been cultivated, the air has become
more temperate. Canada, at this day, is an exact picture of
ancient Germany. Although situated in the same parallel with the
finest provinces of France and England, that country experiences
the most rigorous cold. The reindeer are very numerous, the
ground is covered with deep and lasting snow, and the great river
of St. Lawrence is regularly frozen, in a season when the waters
of the Seine and the Thames are usually free from ice.\footnotemark[7]

\footnotetext[2]{In particular, Mr. Hume, the Abbé du Bos, and M.
Pelloutier. Hist. des Celtes, tom. i.}

\footnotetext[3]{Diodorus Siculus, l. v. p. 340, edit. Wessel.
Herodian, l. vi. p. 221. Jornandes, c. 55. On the banks of the
Danube, the wine, when brought to table, was frequently frozen
into great lumps, frusta vini. Ovid. Epist. ex Ponto, l. iv. 7,
9, 10. Virgil. Georgic. l. iii. 355. The fact is confirmed by a
soldier and a philosopher, who had experienced the intense cold
of Thrace. See Xenophon, Anabasis, l. vii. p. 560, edit.
Hutchinson. Note: The Danube is constantly frozen over. At Pesth
the bridge is usually taken up, and the traffic and communication
between the two banks carried on over the ice. The Rhine is
likewise in many parts passable at least two years out of five.
Winter campaigns are so unusual, in modern warfare, that I
recollect but one instance of an army crossing either river on
the ice. In the thirty years’ war, (1635,) Jan van Werth, an
Imperialist partisan, crossed the Rhine from Heidelberg on the
ice with 5000 men, and surprised Spiers. Pichegru’s memorable
campaign, (1794-5,) when the freezing of the Meuse and Waal
opened Holland to his conquests, and his cavalry and artillery
attacked the ships frozen in, on the Zuyder Zee, was in a winter
of unprecedented severity.—M. 1845.}

\footnotetext[4]{Buffon, Histoire Naturelle, tom. xii. p. 79, 116.}

\footnotetext[5]{Cæsar de Bell. Gallic. vi. 23, \&c. The most
inquisitive of the Germans were ignorant of its utmost limits,
although some of them had travelled in it more than sixty days’
journey. * Note: The passage of Cæsar, “parvis renonum tegumentis
utuntur,” is obscure, observes Luden, (Geschichte des Teutschen
Volkes,) and insufficient to prove the reindeer to have existed
in Germany. It is supported however, by a fragment of Sallust.
Germani intectum rhenonibus corpus tegunt.—M. It has been
suggested to me that Cæsar (as old Gesner supposed) meant the
reindeer in the following description. Est bos cervi figura cujus
a media fronte inter aures unum cornu existit, excelsius magisque
directum (divaricatum, qu?) his quæ nobis nota sunt cornibus. At
ejus summo, sicut palmæ, rami quam late diffunduntur. Bell.
vi.—M. 1845.}

\footnotetext[6]{Cluverius (Germania Antiqua, l. iii. c. 47)
investigates the small and scattered remains of the Hercynian
wood.}

\footnotetext[7]{Charlevoix, Histoire du Canada.}

It is difficult to ascertain, and easy to exaggerate, the
influence of the climate of ancient Germany over the minds and
bodies of the natives. Many writers have supposed, and most have
allowed, though, as it should seem, without any adequate proof,
that the rigorous cold of the North was favorable to long life
and generative vigor, that the women were more fruitful, and the
human species more prolific, than in warmer or more temperate
climates.\footnotemark[8] We may assert, with greater confidence, that the keen
air of Germany formed the large and masculine limbs of the
natives, who were, in general, of a more lofty stature than the
people of the South,\footnotemark[9] gave them a kind of strength better
adapted to violent exertions than to patient labor, and inspired
them with constitutional bravery, which is the result of nerves
and spirits. The severity of a winter campaign, that chilled the
courage of the Roman troops, was scarcely felt by these hardy
children of the North,\footnotemark[10] who, in their turn, were unable to
resist the summer heats, and dissolved away in languor and
sickness under the beams of an Italian sun.\footnotemark[11]

\footnotetext[8]{Olaus Rudbeck asserts that the Swedish women often
bear ten or twelve children, and not uncommonly twenty or thirty;
but the authority of Rudbeck is much to be suspected.}

\footnotetext[9]{In hos artus, in hæc corpora, quæ miramur,
excrescunt. Tæit Germania, 3, 20. Cluver. l. i. c. 14.}

\footnotetext[10]{Plutarch. in Mario. The Cimbri, by way of
amusement, often did down mountains of snow on their broad
shields.}

\footnotetext[11]{The Romans made war in all climates, and by their
excellent discipline were in a great measure preserved in health
and vigor. It may be remarked, that man is the only animal which
can live and multiply in every country from the equator to the
poles. The hog seems to approach the nearest to our species in
that privilege.}

