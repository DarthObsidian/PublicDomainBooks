\section{Part \thesection.}
\thispagestyle{simple}

The Romans had long experienced the daring valor of the people of
Lower Germany. The union of their strength threatened Gaul with a
more formidable invasion, and required the presence of Gallienus,
the heir and colleague of Imperial power.\footnotemark[75] Whilst that prince,
and his infant son Salonius, displayed, in the court of Treves,
the majesty of the empire, its armies were ably conducted by
their general, Posthumus, who, though he afterwards betrayed the
family of Valerian, was ever faithful to the great interests of
the monarchy. The treacherous language of panegyrics and medals
darkly announces a long series of victories. Trophies and titles
attest (if such evidence can attest) the fame of Posthumus, who
is repeatedly styled the Conqueror of the Germans, and the Savior
of Gaul.\footnotemark[76]

\footnotetext[75]{Zosimus, l. i. p. 27.}

\footnotetext[76]{M. de Brequigny (in the Memoires de l’Academie,
tom. xxx.) has given us a very curious life of Posthumus. A
series of the Augustan History from Medals and Inscriptions has
been more than once planned, and is still much wanted. * Note: M.
Eckhel, Keeper of the Cabinet of Medals, and Professor of
Antiquities at Vienna, lately deceased, has supplied this want by
his excellent work, Doctrina veterum Nummorum, conscripta a Jos.
Eckhel, 8 vol. in 4to Vindobona, 1797.—G. Captain Smyth has
likewise printed (privately) a valuable Descriptive Catologue of
a series of Large Brass Medals of this period Bedford, 1834.—M.
1845.}

But a single fact, the only one indeed of which we have any
distinct knowledge, erases, in a great measure, these monuments
of vanity and adulation. The Rhine, though dignified with the
title of Safeguard of the provinces, was an imperfect barrier
against the daring spirit of enterprise with which the Franks
were actuated. Their rapid devastations stretched from the river
to the foot of the Pyrenees; nor were they stopped by those
mountains. Spain, which had never dreaded, was unable to resist,
the inroads of the Germans. During twelve years, the greatest
part of the reign of Gallienus, that opulent country was the
theatre of unequal and destructive hostilities. Tarragona, the
flourishing capital of a peaceful province, was sacked and almost
destroyed;\footnotemark[77] and so late as the days of Orosius, who wrote in
the fifth century, wretched cottages, scattered amidst the ruins
of magnificent cities, still recorded the rage of the barbarians.\footnotemark[78]
When the exhausted country no longer supplied a variety of
plunder, the Franks seized on some vessels in the ports of Spain,\footnotemark[79]
and transported themselves into Mauritania. The distant
province was astonished with the fury of these barbarians, who
seemed to fall from a new world, as their name, manners, and
complexion, were equally unknown on the coast of Africa.\footnotemark[80]

\footnotetext[77]{Aurel. Victor, c. 33. Instead of Pœne direpto,
both the sense and the expression require deleto; though indeed,
for different reasons, it is alike difficult to correct the text
of the best, and of the worst, writers.}

\footnotetext[78]{In the time of Ausonius (the end of the fourth
century) Ilerda or Lerida was in a very ruinous state, (Auson.
Epist. xxv. 58,) which probably was the consequence of this
invasion.}

\footnotetext[79]{Valesius is therefore mistaken in supposing that
the Franks had invaded Spain by sea.}

\footnotetext[80]{Aurel. Victor. Eutrop. ix. 6.}

II. In that part of Upper Saxony, beyond the Elbe, which is at
present called the Marquisate of Lusace, there existed, in
ancient times, a sacred wood, the awful seat of the superstition
of the Suevi. None were permitted to enter the holy precincts,
without confessing, by their servile bonds and suppliant posture,
the immediate presence of the sovereign Deity.\footnotemark[81] Patriotism
contributed, as well as devotion, to consecrate the Sonnenwald,
or wood of the Semnones.\footnotemark[82] It was universally believed, that the
nation had received its first existence on that sacred spot. At
stated periods, the numerous tribes who gloried in the Suevic
blood, resorted thither by their ambassadors; and the memory of
their common extraction was perpetrated by barbaric rites and
human sacrifices. The wide-extended name of Suevi filled the
interior countries of Germany, from the banks of the Oder to
those of the Danube. They were distinguished from the other
Germans by their peculiar mode of dressing their long hair, which
they gathered into a rude knot on the crown of the head; and they
delighted in an ornament that showed their ranks more lofty and
terrible in the eyes of the enemy.\footnotemark[83] Jealous as the Germans were
of military renown, they all confessed the superior valor of the
Suevi; and the tribes of the Usipetes and Tencteri, who, with a
vast army, encountered the dictator Cæsar, declared that they
esteemed it not a disgrace to have fled before a people to whose
arms the immortal gods themselves were unequal.\footnotemark[84]

\footnotetext[81]{Tacit.Germania, 38.}

\footnotetext[82]{Cluver. Germ. Antiq. iii. 25.}

\footnotetext[83]{Sic Suevi a ceteris Germanis, sic Suerorum ingenui
a servis separantur. A proud separation!}

\footnotetext[84]{Cæsar in Bello Gallico, iv. 7.}

In the reign of the emperor Caracalla, an innumerable swarm of
Suevi appeared on the banks of the Main, and in the neighborhood
of the Roman provinces, in quest either of food, of plunder, or
of glory.\footnotemark[85] The hasty army of volunteers gradually coalesced
into a great and permanent nation, and, as it was composed from
so many different tribes, assumed the name of Alemanni, \footnotemark[851] or
\textit{Allmen}, to denote at once their various lineage and their
common bravery.\footnotemark[86] The latter was soon felt by the Romans in many
a hostile inroad. The Alemanni fought chiefly on horseback; but
their cavalry was rendered still more formidable by a mixture of
light infantry, selected from the bravest and most active of the
youth, whom frequent exercise had inured to accompany the
horsemen in the longest march, the most rapid charge, or the most
precipitate retreat.\footnotemark[87]

\footnotetext[85]{Victor in Caracal. Dion Cassius, lxvii. p. 1350.}

\footnotetext[851]{The nation of the Alemanni was not originally
formed by the Suavi properly so called; these have always
preserved their own name. Shortly afterwards they made (A. D.
357) an irruption into Rhætia, and it was not long after that
they were reunited with the Alemanni. Still they have always been
a distinct people; at the present day, the people who inhabit the
north-west of the Black Forest call themselves Schwaben,
Suabians, Sueves, while those who inhabit near the Rhine, in
Ortenau, the Brisgaw, the Margraviate of Baden, do not consider
themselves Suabians, and are by origin Alemanni. The Teucteri and
the Usipetæ, inhabitants of the interior and of the north of
Westphalia, formed, says Gatterer, the nucleus of the Alemannic
nation; they occupied the country where the name of the Alemanni
first appears, as conquered in 213, by Caracalla. They were well
trained to fight on horseback, (according to Tacitus, Germ. c.
32;) and Aurelius Victor gives the same praise to the Alemanni:
finally, they never made part of the Frankish league. The
Alemanni became subsequently a centre round which gathered a
multitude of German tribes, See Eumen. Panegyr. c. 2. Amm. Marc.
xviii. 2, xxix. 4.—G. ——The question whether the Suevi was a
generic name comprehending the clans which peopled central
Germany, is rather hastily decided by M. Guizot Mr. Greenwood,
who has studied the modern German writers on their own origin,
supposes the Suevi, Alemanni, and Marcomanni, one people, under
different appellations. History of Germany, vol i.—M.}

\footnotetext[86]{This etymology (far different from those which
amuse the fancy of the learned) is preserved by Asinius
Quadratus, an original historian, quoted by Agathias, i. c. 5.}

\footnotetext[87]{The Suevi engaged Cæsar in this manner, and the
manœuvre deserved the approbation of the conqueror, (in Bello
Gallico, i. 48.)}

This warlike people of Germans had been astonished by the immense
preparations of Alexander Severus; they were dismayed by the arms
of his successor, a barbarian equal in valor and fierceness to
themselves. But still hovering on the frontiers of the empire,
they increased the general disorder that ensued after the death
of Decius. They inflicted severe wounds on the rich provinces of
Gaul; they were the first who removed the veil that covered the
feeble majesty of Italy. A numerous body of the Alemanni
penetrated across the Danube and through the Rhætian Alps into
the plains of Lombardy, advanced as far as Ravenna, and displayed
the victorious banners of barbarians almost in sight of Rome.\footnotemark[88]

\footnotetext[88]{Hist. August. p. 215, 216. Dexippus in the
Excerpts. Legationam, p. 8. Hieronym. Chron. Orosius, vii. 22.}

The insult and the danger rekindled in the senate some sparks of
their ancient virtue. Both the emperors were engaged in far
distant wars, Valerian in the East, and Gallienus on the Rhine.
All the hopes and resources of the Romans were in themselves. In
this emergency, the senators resumed the defence of the republic,
drew out the Prætorian guards, who had been left to garrison the
capital, and filled up their numbers, by enlisting into the
public service the stoutest and most willing of the Plebeians.
The Alemanni, astonished with the sudden appearance of an army
more numerous than their own, retired into Germany, laden with
spoil; and their retreat was esteemed as a victory by the
unwarlike Romans.\footnotemark[89]

\footnotetext[89]{Zosimus, l. i. p. 34.}

When Gallienus received the intelligence that his capital was
delivered from the barbarians, he was much less delighted than
alarmed with the courage of the senate, since it might one day
prompt them to rescue the public from domestic tyranny as well as
from foreign invasion. His timid ingratitude was published to his
subjects, in an edict which prohibited the senators from
exercising any military employment, and even from approaching the
camps of the legions. But his fears were groundless. The rich and
luxurious nobles, sinking into their natural character, accepted,
as a favor, this disgraceful exemption from military service; and
as long as they were indulged in the enjoyment of their baths,
their theatres, and their villas, they cheerfully resigned the
more dangerous cares of empire to the rough hands of peasants and
soldiers.\footnotemark[90]

\footnotetext[90]{Aurel. Victor, in Gallieno et Probo. His complaints
breathe as uncommon spirit of freedom.}

Another invasion of the Alemanni, of a more formidable aspect,
but more glorious event, is mentioned by a writer of the lower
empire. Three hundred thousand are said to have been vanquished,
in a battle near Milan, by Gallienus in person, at the head of
only ten thousand Romans.\footnotemark[91] We may, however, with great
probability, ascribe this incredible victory either to the
credulity of the historian, or to some exaggerated exploits of
one of the emperor’s lieutenants. It was by arms of a very
different nature, that Gallienus endeavored to protect Italy from
the fury of the Germans. He espoused Pipa, the daughter of a king
of the Marcomanni, a Suevic tribe, which was often confounded
with the Alemanni in their wars and conquests.\footnotemark[92] To the father,
as the price of his alliance, he granted an ample settlement in
Pannonia. The native charms of unpolished beauty seem to have
fixed the daughter in the affections of the inconstant emperor,
and the bands of policy were more firmly connected by those of
love. But the haughty prejudice of Rome still refused the name of
marriage to the profane mixture of a citizen and a barbarian; and
has stigmatized the German princess with the opprobrious title of
concubine of Gallienus.\footnotemark[93]

\footnotetext[91]{Zonaras, l. xii. p. 631.}

\footnotetext[92]{One of the Victors calls him king of the
Marcomanni; the other of the Germans.}

\footnotetext[93]{See Tillemont, Hist. des Empereurs, tom. iii. p.
398, \&c.}

III. We have already traced the emigration of the Goths from
Scandinavia, or at least from Prussia, to the mouth of the
Borysthenes, and have followed their victorious arms from the
Borysthenes to the Danube. Under the reigns of Valerian and
Gallienus, the frontier of the last-mentioned river was
perpetually infested by the inroads of Germans and Sarmatians;
but it was defended by the Romans with more than usual firmness
and success. The provinces that were the seat of war, recruited
the armies of Rome with an inexhaustible supply of hardy
soldiers; and more than one of these Illyrian peasants attained
the station, and displayed the abilities, of a general. Though
flying parties of the barbarians, who incessantly hovered on the
banks of the Danube, penetrated sometimes to the confines of
Italy and Macedonia, their progress was commonly checked, or
their return intercepted, by the Imperial lieutenants.\footnotemark[94] But the
great stream of the Gothic hostilities was diverted into a very
different channel. The Goths, in their new settlement of the
Ukraine, soon became masters of the northern coast of the Euxine:
to the south of that inland sea were situated the soft and
wealthy provinces of Asia Minor, which possessed all that could
attract, and nothing that could resist, a barbarian conqueror.

\footnotetext[94]{See the lives of Claudius, Aurelian, and Probus, in
the Augustan History.}

The banks of the Borysthenes are only sixty miles distant from
the narrow entrance\footnotemark[95] of the peninsula of Crim Tartary, known to
the ancients under the name of Chersonesus Taurica.\footnotemark[96] On that
inhospitable shore, Euripides, embellishing with exquisite art
the tales of antiquity, has placed the scene of one of his most
affecting tragedies.\footnotemark[97] The bloody sacrifices of Diana, the
arrival of Orestes and Pylades, and the triumph of virtue and
religion over savage fierceness, serve to represent an historical
truth, that the Tauri, the original inhabitants of the peninsula,
were, in some degree, reclaimed from their brutal manners by a
gradual intercourse with the Grecian colonies, which settled
along the maritime coast. The little kingdom of Bosphorus, whose
capital was situated on the Straits, through which the Mæotis
communicates itself to the Euxine, was composed of degenerate
Greeks and half-civilized barbarians. It subsisted, as an
independent state, from the time of the Peloponnesian war,\footnotemark[98] was
at last swallowed up by the ambition of Mithridates,\footnotemark[99] and, with
the rest of his dominions, sunk under the weight of the Roman
arms. From the reign of Augustus,\footnotemark[100] the kings of Bosphorus were
the humble, but not useless, allies of the empire. By presents,
by arms, and by a slight fortification drawn across the Isthmus,
they effectually guarded, against the roving plunderers of
Sarmatia, the access of a country which, from its peculiar
situation and convenient harbors, commanded the Euxine Sea and
Asia Minor.\footnotemark[101] As long as the sceptre was possessed by a lineal
succession of kings, they acquitted themselves of their important
charge with vigilance and success. Domestic factions, and the
fears, or private interest, of obscure usurpers, who seized on
the vacant throne, admitted the Goths into the heart of
Bosphorus. With the acquisition of a superfluous waste of fertile
soil, the conquerors obtained the command of a naval force,
sufficient to transport their armies to the coast of Asia.\footnotemark[102]
These ships used in the navigation of the Euxine were of a very
singular construction. They were slight flat-bottomed barks
framed of timber only, without the least mixture of iron, and
occasionally covered with a shelving roof, on the appearance of a
tempest.\footnotemark[103] In these floating houses, the Goths carelessly
trusted themselves to the mercy of an unknown sea, under the
conduct of sailors pressed into the service, and whose skill and
fidelity were equally suspicious. But the hopes of plunder had
banished every idea of danger, and a natural fearlessness of
temper supplied in their minds the more rational confidence,
which is the just result of knowledge and experience. Warriors of
such a daring spirit must have often murmured against the
cowardice of their guides, who required the strongest assurances
of a settled calm before they would venture to embark; and would
scarcely ever be tempted to lose sight of the land. Such, at
least, is the practice of the modern Turks;\footnotemark[104] and they are
probably not inferior, in the art of navigation, to the ancient
inhabitants of Bosphorus.

\footnotetext[95]{It is about half a league in breadth. Genealogical
History of the Tartars, p 598.}

\footnotetext[96]{M. de Peyssonel, who had been French Consul at
Caffa, in his Observations sur les Peuples Barbares, que ont
habite les bords du Danube}

\footnotetext[97]{Eeripides in Iphigenia in Taurid.}

\footnotetext[98]{Strabo, l. vii. p. 309. The first kings of
Bosphorus were the allies of Athens.}

\footnotetext[99]{Appian in Mithridat.}

\footnotetext[100]{It was reduced by the arms of Agrippa. Orosius,
vi. 21. Eu tropius, vii. 9. The Romans once advanced within three
days’ march of the Tanais. Tacit. Annal. xii. 17.}

\footnotetext[101]{See the Toxaris of Lucian, if we credit the
sincerity and the virtues of the Scythian, who relates a great
war of his nation against the kings of Bosphorus.}

\footnotetext[102]{Zosimus, l. i. p. 28.}

\footnotetext[103]{Strabo, l. xi. Tacit. Hist. iii. 47. They were
called Camarœ.}

\footnotetext[104]{See a very natural picture of the Euxine
navigation, in the xvith letter of Tournefort.}

The fleet of the Goths, leaving the coast of Circassia on the
left hand, first appeared before Pityus,\footnotemark[105] the utmost limits of
the Roman provinces; a city provided with a convenient port, and
fortified with a strong wall. Here they met with a resistance
more obstinate than they had reason to expect from the feeble
garrison of a distant fortress. They were repulsed; and their
disappointment seemed to diminish the terror of the Gothic name.
As long as Successianus, an officer of superior rank and merit,
defended that frontier, all their efforts were ineffectual; but
as soon as he was removed by Valerian to a more honorable but
less important station, they resumed the attack of Pityus; and by
the destruction of that city, obliterated the memory of their
former disgrace.\footnotemark[106]

\footnotetext[105]{Arrian places the frontier garrison at Dioscurias,
or Sebastopolis, forty-four miles to the east of Pityus. The
garrison of Phasis consisted in his time of only four hundred
foot. See the Periplus of the Euxine. * Note: Pityus is
Pitchinda, according to D’Anville, ii. 115.—G. Rather Boukoun.—M.
Dioscurias is Iskuriah.—G.}

\footnotetext[106]{Zosimus, l. i. p. 30.}

Circling round the eastern extremity of the Euxine Sea, the
navigation from Pityus to Trebizond is about three hundred miles.\footnotemark[107]
The course of the Goths carried them in sight of the country
of Colchis, so famous by the expedition of the Argonauts; and
they even attempted, though without success, to pillage a rich
temple at the mouth of the River Phasis. Trebizond, celebrated in
the retreat of the ten thousand as an ancient colony of Greeks,\footnotemark[108]
derived its wealth and splendor from the magnificence of the
emperor Hadrian, who had constructed an artificial port on a
coast left destitute by nature of secure harbors.\footnotemark[109] The city
was large and populous; a double enclosure of walls seemed to
defy the fury of the Goths, and the usual garrison had been
strengthened by a reënforcement of ten thousand men. But there
are not any advantages capable of supplying the absence of
discipline and vigilance. The numerous garrison of Trebizond,
dissolved in riot and luxury, disdained to guard their
impregnable fortifications. The Goths soon discovered the supine
negligence of the besieged, erected a lofty pile of fascines,
ascended the walls in the silence of the night, and entered the
defenceless city sword in hand. A general massacre of the people
ensued, whilst the affrighted soldiers escaped through the
opposite gates of the town. The most holy temples, and the most
splendid edifices, were involved in a common destruction. The
booty that fell into the hands of the Goths was immense: the
wealth of the adjacent countries had been deposited in Trebizond,
as in a secure place of refuge. The number of captives was
incredible, as the victorious barbarians ranged without
opposition through the extensive province of Pontus.\footnotemark[110] The rich
spoils of Trebizond filled a great fleet of ships that had been
found in the port. The robust youth of the sea-coast were chained
to the oar; and the Goths, satisfied with the success of their
first naval expedition, returned in triumph to their new
establishment in the kingdom of Bosphorus.\footnotemark[111]

\footnotetext[107]{Arrian (in Periplo Maris Euxine, p. 130) calls the
distance 2610 stadia.}

\footnotetext[108]{Xenophon, Anabasis, l. iv. p. 348, edit.
Hutchinson. Note: Fallmerayer (Geschichte des Kaiserthums von
Trapezunt, p. 6, \&c) assigns a very ancient date to the first
(Pelasgic) foundation of Trapezun (Trebizond)—M.}

\footnotetext[109]{Arrian, p. 129. The general observation is
Tournefort’s.}

\footnotetext[110]{See an epistle of Gregory Thaumaturgus, bishop of
Neo-Cæoarea, quoted by Mascou, v. 37.}

\footnotetext[111]{Zosimus, l. i. p. 32, 33.}

The second expedition of the Goths was undertaken with greater
powers of men and ships; but they steered a different course,
and, disdaining the exhausted provinces of Pontus, followed the
western coast of the Euxine, passed before the wide mouths of the
Borysthenes, the Niester, and the Danube, and increasing their
fleet by the capture of a great number of fishing barks, they
approached the narrow outlet through which the Euxine Sea pours
its waters into the Mediterranean, and divides the continents of
Europe and Asia. The garrison of Chalcedon was encamped near the
temple of Jupiter Urius, on a promontory that commanded the
entrance of the Strait; and so inconsiderable were the dreaded
invasions of the barbarians that this body of troops surpassed in
number the Gothic army. But it was in numbers alone that they
surpassed it. They deserted with precipitation their advantageous
post, and abandoned the town of Chalcedon, most plentifully
stored with arms and money, to the discretion of the conquerors.
Whilst they hesitated whether they should prefer the sea or land,
Europe or Asia, for the scene of their hostilities, a perfidious
fugitive pointed out Nicomedia,\footnotemark[1111] once the capital of the
kings of Bithynia, as a rich and easy conquest. He guided the
march, which was only sixty miles from the camp of Chalcedon,\footnotemark[112]
directed the resistless attack, and partook of the booty; for the
Goths had learned sufficient policy to reward the traitor whom
they detested. Nice, Prusa, Apamæa, Cius,\footnotemark[1121] cities that had
sometimes rivalled, or imitated, the splendor of Nicomedia, were
involved in the same calamity, which, in a few weeks, raged
without control through the whole province of Bithynia. Three
hundred years of peace, enjoyed by the soft inhabitants of Asia,
had abolished the exercise of arms, and removed the apprehension
of danger. The ancient walls were suffered to moulder away, and
all the revenue of the most opulent cities was reserved for the
construction of baths, temples, and theatres.\footnotemark[113]

\footnotetext[1111]{It has preserved its name, joined to the
preposition of place in that of Nikmid. D’Anv. Geog. Anc. ii.
28.—G.}

\footnotetext[112]{Itiner. Hierosolym. p. 572. Wesseling.}

\footnotetext[1121]{Now Isnik, Bursa, Mondania Ghio or Kemlik D’Anv.
ii. 23.—G.}

\footnotetext[113]{Zosimus, l.. p. 32, 33.}

When the city of Cyzicus withstood the utmost effort of
Mithridates,\footnotemark[114] it was distinguished by wise laws, a naval power
of two hundred galleys, and three arsenals, of arms, of military
engines, and of corn.\footnotemark[115] It was still the seat of wealth and
luxury; but of its ancient strength, nothing remained except the
situation, in a little island of the Propontis, connected with
the continent of Asia only by two bridges. From the recent sack
of Prusa, the Goths advanced within eighteen miles\footnotemark[116] of the
city, which they had devoted to destruction; but the ruin of
Cyzicus was delayed by a fortunate accident. The season was
rainy, and the Lake Apolloniates, the reservoir of all the
springs of Mount Olympus, rose to an uncommon height. The little
river of Rhyndacus, which issues from the lake, swelled into a
broad and rapid stream, and stopped the progress of the Goths.
Their retreat to the maritime city of Heraclea, where the fleet
had probably been stationed, was attended by a long train of
wagons, laden with the spoils of Bithynia, and was marked by the
flames of Nico and Nicomedia, which they wantonly burnt.\footnotemark[117] Some
obscure hints are mentioned of a doubtful combat that secured
their retreat.\footnotemark[118] But even a complete victory would have been of
little moment, as the approach of the autumnal equinox summoned
them to hasten their return. To navigate the Euxine before the
month of May, or after that of September, is esteemed by the
modern Turks the most unquestionable instance of rashness and
folly.\footnotemark[119]

\footnotetext[114]{He besieged the place with 400 galleys, 150,000
foot, and a numerous cavalry. See Plutarch in Lucul. Appian in
Mithridat Cicero pro Lege Manilia, c. 8.}

\footnotetext[115]{Strabo, l. xii. p. 573.}

\footnotetext[116]{Pocock’s Description of the East, l. ii. c. 23,
24.}

\footnotetext[117]{Zosimus, l. i. p. 33.}

\footnotetext[118]{Syncellus tells an unintelligible story of Prince
Odenathus, who defeated the Goths, and who was killed by Prince
Odenathus.}

\footnotetext[119]{Footnote 119: Voyages de Chardin, tom. i. p. 45. He
sailed with the Turks from Constantinople to Caffa.}

When we are informed that the third fleet, equipped by the Goths
in the ports of Bosphorus, consisted of five hundred sails of
ships,\footnotemark[120] our ready imagination instantly computes and
multiplies the formidable armament; but, as we are assured by the
judicious Strabo,\footnotemark[121] that the piratical vessels used by the
barbarians of Pontus and the Lesser Scythia, were not capable of
containing more than twenty-five or thirty men we may safely
affirm, that fifteen thousand warriors, at the most, embarked in
this great expedition. Impatient of the limits of the Euxine,
they steered their destructive course from the Cimmerian to the
Thracian Bosphorus. When they had almost gained the middle of the
Straits, they were suddenly driven back to the entrance of them;
till a favorable wind, springing up the next day, carried them in
a few hours into the placid sea, or rather lake, of the
Propontis. Their landing on the little island of Cyzicus was
attended with the ruin of that ancient and noble city. From
thence issuing again through the narrow passage of the
Hellespont, they pursued their winding navigation amidst the
numerous islands scattered over the Archipelago, or the Ægean
Sea. The assistance of captives and deserters must have been very
necessary to pilot their vessels, and to direct their various
incursions, as well on the coast of Greece as on that of Asia. At
length the Gothic fleet anchored in the port of Piræus, five
miles distant from Athens,\footnotemark[122] which had attempted to make some
preparations for a vigorous defence. Cleodamus, one of the
engineers employed by the emperor’s orders to fortify the
maritime cities against the Goths, had already begun to repair
the ancient walls, fallen to decay since the time of Scylla. The
efforts of his skill were ineffectual, and the barbarians became
masters of the native seat of the muses and the arts. But while
the conquerors abandoned themselves to the license of plunder and
intemperance, their fleet, that lay with a slender guard in the
harbor of Piræus, was unexpectedly attacked by the brave
Dexippus, who, flying with the engineer Cleodamus from the sack
of Athens, collected a hasty band of volunteers, peasants as well
as soldiers, and in some measure avenged the calamities of his
country.\footnotemark[123]

\footnotetext[120]{Syncellus (p. 382) speaks of this expedition, as
undertaken by the Heruli.}

\footnotetext[121]{Strabo, l. xi. p. 495.}

\footnotetext[122]{Plin. Hist. Natur. iii. 7.}

\footnotetext[123]{Hist. August. p. 181. Victor, c. 33. Orosius, vii.
42. Zosimus, l. i. p. 35. Zonaras, l. xii. 635. Syncellus, p.
382. It is not without some attention, that we can explain and
conciliate their imperfect hints. We can still discover some
traces of the partiality of Dexippus, in the relation of his own
and his countrymen’s exploits. * Note: According to a new
fragment of Dexippus, published by Mai, the 2000 men took up a
strong position in a mountainous and woods district, and kept up
a harassing warfare. He expresses a hope of being speedily joined
by the Imperial fleet. Dexippus in rov. Byzantinorum Collect a
Niebuhr, p. 26, 8—M.}

But this exploit, whatever lustre it might shed on the declining
age of Athens, served rather to irritate than to subdue the
undaunted spirit of the northern invaders. A general
conflagration blazed out at the same time in every district of
Greece. Thebes and Argos, Corinth and Sparta, which had formerly
waged such memorable wars against each other, were now unable to
bring an army into the field, or even to defend their ruined
fortifications. The rage of war, both by land and by sea, spread
from the eastern point of Sunium to the western coast of Epirus.
The Goths had already advanced within sight of Italy, when the
approach of such imminent danger awakened the indolent Gallienus
from his dream of pleasure. The emperor appeared in arms; and his
presence seems to have checked the ardor, and to have divided the
strength, of the enemy. Naulobatus, a chief of the Heruli,
accepted an honorable capitulation, entered with a large body of
his countrymen into the service of Rome, and was invested with
the ornaments of the consular dignity, which had never before
been profaned by the hands of a barbarian.\footnotemark[124] Great numbers of
the Goths, disgusted with the perils and hardships of a tedious
voyage, broke into Mæsia, with a design of forcing their way over
the Danube to their settlements in the Ukraine. The wild attempt
would have proved inevitable destruction, if the discord of the
Roman generals had not opened to the barbarians the means of an
escape.\footnotemark[125] The small remainder of this destroying host returned
on board their vessels; and measuring back their way through the
Hellespont and the Bosphorus, ravaged in their passage the shores
of Troy, whose fame, immortalized by Homer, will probably survive
the memory of the Gothic conquests. As soon as they found
themselves in safety within the basin of the Euxine, they landed
at Anchialus in Thrace, near the foot of Mount Hæmus; and, after
all their toils, indulged themselves in the use of those pleasant
and salutary hot baths. What remained of the voyage was a short
and easy navigation.\footnotemark[126] Such was the various fate of this third
and greatest of their naval enterprises. It may seem difficult to
conceive how the original body of fifteen thousand warriors could
sustain the losses and divisions of so bold an adventure. But as
their numbers were gradually wasted by the sword, by shipwrecks,
and by the influence of a warm climate, they were perpetually
renewed by troops of banditti and deserters, who flocked to the
standard of plunder, and by a crowd of fugitive slaves, often of
German or Sarmatian extraction, who eagerly seized the glorious
opportunity of freedom and revenge. In these expeditions, the
Gothic nation claimed a superior share of honor and danger; but
the tribes that fought under the Gothic banners are sometimes
distinguished and sometimes confounded in the imperfect histories
of that age; and as the barbarian fleets seemed to issue from the
mouth of the Tanais, the vague but familiar appellation of
Scythians was frequently bestowed on the mixed multitude.\footnotemark[127]

\footnotetext[124]{Syncellus, p. 382. This body of Heruli was for a
long time faithful and famous.}

\footnotetext[125]{Claudius, who commanded on the Danube, thought
with propriety and acted with spirit. His colleague was jealous
of his fame Hist. August. p. 181.}

\footnotetext[126]{Jornandes, c. 20.}

\footnotetext[127]{Zosimus and the Greeks (as the author of the
Philopatris) give the name of Scythians to those whom Jornandes,
and the Latin writers, constantly represent as Goths.}

