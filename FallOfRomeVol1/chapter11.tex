\chapter{Reign Of Claudius, Defeat Of The Goths.}
\section{Part \thesection.}

\textit{Reign Of Claudius. — Defeat Of The Goths. — Victories, Triumph, And
Death Of Aurelian.}
\vspace{\onelineskip}

Under the deplorable reigns of Valerian and Gallienus, the empire
was oppressed and almost destroyed by the soldiers, the tyrants,
and the barbarians. It was saved by a series of great princes,
who derived their obscure origin from the martial provinces of
Illyricum. Within a period of about thirty years, Claudius,
Aurelian, Probus, Diocletian and his colleagues, triumphed over
the foreign and domestic enemies of the state, reëstablished,
with the military discipline, the strength of the frontiers, and
deserved the glorious title of Restorers of the Roman world.

The removal of an effeminate tyrant made way for a succession of
heroes. The indignation of the people imputed all their
calamities to Gallienus, and the far greater part were, indeed,
the consequence of his dissolute manners and careless
administration. He was even destitute of a sense of honor, which
so frequently supplies the absence of public virtue; and as long
as he was permitted to enjoy the possession of Italy, a victory
of the barbarians, the loss of a province, or the rebellion of a
general, seldom disturbed the tranquil course of his pleasures.
At length, a considerable army, stationed on the Upper Danube,
invested with the Imperial purple their leader Aureolus; who,
disdaining a confined and barren reign over the mountains of
Rhætia, passed the Alps, occupied Milan, threatened Rome, and
challenged Gallienus to dispute in the field the sovereignty of
Italy. The emperor, provoked by the insult, and alarmed by the
instant danger, suddenly exerted that latent vigor which
sometimes broke through the indolence of his temper. Forcing
himself from the luxury of the palace, he appeared in arms at the
head of his legions, and advanced beyond the Po to encounter his
competitor. The corrupted name of Pontirolo\textsuperscript{1} still preserves the
memory of a bridge over the Adda, which, during the action, must
have proved an object of the utmost importance to both armies.
The Rhætian usurper, after receiving a total defeat and a
dangerous wound, retired into Milan. The siege of that great city
was immediately formed; the walls were battered with every engine
in use among the ancients; and Aureolus, doubtful of his internal
strength, and hopeless of foreign succors already anticipated the
fatal consequences of unsuccessful rebellion.

\pagenote[1]{Pons Aureoli, thirteen miles from Bergamo, and
thirty-two from Milan. See Cluver. Italia, Antiq. tom. i. p. 245.
Near this place, in the year 1703, the obstinate battle of
Cassano was fought between the French and Austrians. The
excellent relation of the Chevalier de Folard, who was present,
gives a very distinct idea of the ground. See Polybe de Folard,
tom. iii. p. 233-248.}

His last resource was an attempt to seduce the loyalty of the
besiegers. He scattered libels through the camp, inviting the
troops to desert an unworthy master, who sacrificed the public
happiness to his luxury, and the lives of his most valuable
subjects to the slightest suspicions. The arts of Aureolus
diffused fears and discontent among the principal officers of his
rival. A conspiracy was formed by Heraclianus, the Prætorian
præfect, by Marcian, a general of rank and reputation, and by
Cecrops, who commanded a numerous body of Dalmatian guards. The
death of Gallienus was resolved; and notwithstanding their desire
of first terminating the siege of Milan, the extreme danger which
accompanied every moment’s delay obliged them to hasten the
execution of their daring purpose. At a late hour of the night,
but while the emperor still protracted the pleasures of the
table, an alarm was suddenly given, that Aureolus, at the head of
all his forces, had made a desperate sally from the town;
Gallienus, who was never deficient in personal bravery, started
from his silken couch, and without allowing himself time either
to put on his armor, or to assemble his guards, he mounted on
horseback, and rode full speed towards the supposed place of the
attack. Encompassed by his declared or concealed enemies, he
soon, amidst the nocturnal tumult, received a mortal dart from an
uncertain hand. Before he expired, a patriotic sentiment rising
in the mind of Gallienus, induced him to name a deserving
successor; and it was his last request, that the Imperial
ornaments should be delivered to Claudius, who then commanded a
detached army in the neighborhood of Pavia. The report at least
was diligently propagated, and the order cheerfully obeyed by the
conspirators, who had already agreed to place Claudius on the
throne. On the first news of the emperor’s death, the troops
expressed some suspicion and resentment, till the one was
removed, and the other assuaged, by a donative of twenty pieces
of gold to each soldier. They then ratified the election, and
acknowledged the merit of their new sovereign.\textsuperscript{2}

\pagenote[2]{On the death of Gallienus, see Trebellius Pollio in
Hist. August. p. 181. Zosimus, l. i. p. 37. Zonaras, l. xii. p.
634. Eutrop. ix. ll. Aurelius Victor in Epitom. Victor in Cæsar.
I have compared and blended them all, but have chiefly followed
Aurelius Victor, who seems to have had the best memoirs.}

The obscurity which covered the origin of Claudius, though it was
afterwards embellished by some flattering fictions,\textsuperscript{3}
sufficiently betrays the meanness of his birth. We can only
discover that he was a native of one of the provinces bordering
on the Danube; that his youth was spent in arms, and that his
modest valor attracted the favor and confidence of Decius. The
senate and people already considered him as an excellent officer,
equal to the most important trusts; and censured the inattention
of Valerian, who suffered him to remain in the subordinate
station of a tribune. But it was not long before that emperor
distinguished the merit of Claudius, by declaring him general and
chief of the Illyrian frontier, with the command of all the
troops in Thrace, Mæsia, Dacia, Pannonia, and Dalmatia, the
appointments of the præfect of Egypt, the establishment of the
proconsul of Africa, and the sure prospect of the consulship. By
his victories over the Goths, he deserved from the senate the
honor of a statue, and excited the jealous apprehensions of
Gallienus. It was impossible that a soldier could esteem so
dissolute a sovereign, nor is it easy to conceal a just contempt.
Some unguarded expressions which dropped from Claudius were
officiously transmitted to the royal ear. The emperor’s answer to
an officer of confidence describes in very lively colors his own
character, and that of the times. “There is not any thing capable
of giving me more serious concern, than the intelligence
contained in your last despatch;\textsuperscript{4} that some malicious
suggestions have indisposed towards us the mind of our friend and
\textit{parent} Claudius. As you regard your allegiance, use every means
to appease his resentment, but conduct your negotiation with
secrecy; let it not reach the knowledge of the Dacian troops;
they are already provoked, and it might inflame their fury. I
myself have sent him some presents: be it your care that he
accept them with pleasure. Above all, let him not suspect that I
am made acquainted with his imprudence. The fear of my anger
might urge him to desperate counsels.”\textsuperscript{5} The presents which
accompanied this humble epistle, in which the monarch solicited a
reconciliation with his discontented subject, consisted of a
considerable sum of money, a splendid wardrobe, and a valuable
service of silver and gold plate. By such arts Gallienus softened
the indignation and dispelled the fears of his Illyrian general;
and during the remainder of that reign, the formidable sword of
Claudius was always drawn in the cause of a master whom he
despised. At last, indeed, he received from the conspirators the
bloody purple of Gallienus: but he had been absent from their
camp and counsels; and however he might applaud the deed, we may
candidly presume that he was innocent of the knowledge of it.\textsuperscript{6}
When Claudius ascended the throne, he was about fifty-four years
of age.

\pagenote[3]{Some supposed him, oddly enough, to be a bastard of
the younger Gordian. Others took advantage of the province of
Dardania, to deduce his origin from Dardanus, and the ancient
kings of Troy.}

\pagenote[4]{Notoria, a periodical and official despatch which
the emperor received from the frumentarii, or agents dispersed
through the provinces. Of these we may speak hereafter.}

\pagenote[5]{Hist. August. p. 208. Gallienus describes the plate,
vestments, etc., like a man who loved and understood those
splendid trifles.}

\pagenote[6]{Julian (Orat. i. p. 6) affirms that Claudius
acquired the empire in a just and even holy manner. But we may
distrust the partiality of a kinsman.}

The siege of Milan was still continued, and Aureolus soon
discovered that the success of his artifices had only raised up a
more determined adversary. He attempted to negotiate with
Claudius a treaty of alliance and partition. “Tell him,” replied
the intrepid emperor, “that such proposals should have been made
to Gallienus; \textit{he}, perhaps, might have listened to them with
patience, and accepted a colleague as despicable as himself.”\textsuperscript{7}
This stern refusal, and a last unsuccessful effort, obliged
Aureolus to yield the city and himself to the discretion of the
conqueror. The judgment of the army pronounced him worthy of
death; and Claudius, after a feeble resistance, consented to the
execution of the sentence. Nor was the zeal of the senate less
ardent in the cause of their new sovereign. They ratified,
perhaps with a sincere transport of zeal, the election of
Claudius; and, as his predecessor had shown himself the personal
enemy of their order, they exercised, under the name of justice,
a severe revenge against his friends and family. The senate was
permitted to discharge the ungrateful office of punishment, and
the emperor reserved for himself the pleasure and merit of
obtaining by his intercession a general act of indemnity.\textsuperscript{8}

\pagenote[7]{Hist. August. p. 203. There are some trifling
differences concerning the circumstances of the last defeat and
death of Aureolus}

\pagenote[8]{Aurelius Victor in Gallien. The people loudly prayed
for the damnation of Gallienus. The senate decreed that his
relations and servants should be thrown down headlong from the
Gemonian stairs. An obnoxious officer of the revenue had his eyes
torn out whilst under examination. Note: The expression is
curious, “terram matrem deosque inferos impias uti Gallieno
darent.”—M.}

Such ostentatious clemency discovers less of the real character
of Claudius, than a trifling circumstance in which he seems to
have consulted only the dictates of his heart. The frequent
rebellions of the provinces had involved almost every person in
the guilt of treason, almost every estate in the case of
confiscation; and Gallienus often displayed his liberality by
distributing among his officers the property of his subjects. On
the accession of Claudius, an old woman threw herself at his
feet, and complained that a general of the late emperor had
obtained an arbitrary grant of her patrimony. This general was
Claudius himself, who had not entirely escaped the contagion of
the times. The emperor blushed at the reproach, but deserved the
confidence which she had reposed in his equity. The confession of
his fault was accompanied with immediate and ample restitution.\textsuperscript{9}

\pagenote[9]{Zonaras, l. xii. p. 137.}

In the arduous task which Claudius had undertaken, of restoring
the empire to its ancient splendor, it was first necessary to
revive among his troops a sense of order and obedience. With the
authority of a veteran commander, he represented to them that the
relaxation of discipline had introduced a long train of
disorders, the effects of which were at length experienced by the
soldiers themselves; that a people ruined by oppression, and
indolent from despair, could no longer supply a numerous army
with the means of luxury, or even of subsistence; that the danger
of each individual had increased with the despotism of the
military order, since princes who tremble on the throne will
guard their safety by the instant sacrifice of every obnoxious
subject. The emperor expiated on the mischiefs of a lawless
caprice, which the soldiers could only gratify at the expense of
their own blood; as their seditious elections had so frequently
been followed by civil wars, which consumed the flower of the
legions either in the field of battle, or in the cruel abuse of
victory. He painted in the most lively colors the exhausted state
of the treasury, the desolation of the provinces, the disgrace of
the Roman name, and the insolent triumph of rapacious barbarians.
It was against those barbarians, he declared, that he intended to
point the first effort of their arms. Tetricus might reign for a
while over the West, and even Zenobia might preserve the dominion
of the East.\textsuperscript{10} These usurpers were his personal adversaries; nor
could he think of indulging any private resentment till he had
saved an empire, whose impending ruin would, unless it was timely
prevented, crush both the army and the people.

\pagenote[10]{Zonaras on this occasion mentions Posthumus but the
registers of the senate (Hist. August. p. 203) prove that
Tetricus was already emperor of the western provinces.}

The various nations of Germany and Sarmatia, who fought under the
Gothic standard, had already collected an armament more
formidable than any which had yet issued from the Euxine. On the
banks of the Niester, one of the great rivers that discharge
themselves into that sea, they constructed a fleet of two
thousand, or even of six thousand vessels;\textsuperscript{11} numbers which,
however incredible they may seem, would have been insufficient to
transport their pretended army of three hundred and twenty
thousand barbarians. Whatever might be the real strength of the
Goths, the vigor and success of the expedition were not adequate
to the greatness of the preparations. In their passage through
the Bosphorus, the unskilful pilots were overpowered by the
violence of the current; and while the multitude of their ships
were crowded in a narrow channel, many were dashed against each
other, or against the shore. The barbarians made several descents
on the coasts both of Europe and Asia; but the open country was
already plundered, and they were repulsed with shame and loss
from the fortified cities which they assaulted. A spirit of
discouragement and division arose in the fleet, and some of their
chiefs sailed away towards the islands of Crete and Cyprus; but
the main body, pursuing a more steady course, anchored at length
near the foot of Mount Athos, and assaulted the city of
Thessalonica, the wealthy capital of all the Macedonian
provinces. Their attacks, in which they displayed a fierce but
artless bravery, were soon interrupted by the rapid approach of
Claudius, hastening to a scene of action that deserved the
presence of a warlike prince at the head of the remaining powers
of the empire. Impatient for battle, the Goths immediately broke
up their camp, relinquished the siege of Thessalonica, left their
navy at the foot of Mount Athos, traversed the hills of
Macedonia, and pressed forwards to engage the last defence of
Italy.

\pagenote[11]{The Augustan History mentions the smaller, Zonaras
the larger number; the lively fancy of Montesquieu induced him to
prefer the latter.}

We still posses an original letter addressed by Claudius to the
senate and people on this memorable occasion. “Conscript
fathers,” says the emperor, “know that three hundred and twenty
thousand Goths have invaded the Roman territory. If I vanquish
them, your gratitude will reward my services. Should I fall,
remember that I am the successor of Gallienus. The whole republic
is fatigued and exhausted. We shall fight after Valerian, after
Ingenuus, Regillianus, Lollianus, Posthumus, Celsus, and a
thousand others, whom a just contempt for Gallienus provoked into
rebellion. We are in want of darts, of spears, and of shields.
The strength of the empire, Gaul, and Spain, are usurped by
Tetricus, and we blush to acknowledge that the archers of the
East serve under the banners of Zenobia. Whatever we shall
perform will be sufficiently great.”\textsuperscript{12} The melancholy firmness
of this epistle announces a hero careless of his fate, conscious
of his danger, but still deriving a well-grounded hope from the
resources of his own mind.

\pagenote[12]{Trebell. Pollio in Hist. August. p. 204.}

The event surpassed his own expectations and those of the world.
By the most signal victories he delivered the empire from this
host of barbarians, and was distinguished by posterity under the
glorious appellation of the Gothic Claudius. The imperfect
historians of an irregular war\textsuperscript{13} do not enable us to describe
the order and circumstances of his exploits; but, if we could be
indulged in the allusion, we might distribute into three acts
this memorable tragedy. I. The decisive battle was fought near
Naissus, a city of Dardania. The legions at first gave way,
oppressed by numbers, and dismayed by misfortunes. Their ruin was
inevitable, had not the abilities of their emperor prepared a
seasonable relief. A large detachment, rising out of the secret
and difficult passes of the mountains, which, by his order, they
had occupied, suddenly assailed the rear of the victorious Goths.

The favorable instant was improved by the activity of Claudius.
He revived the courage of his troops, restored their ranks, and
pressed the barbarians on every side. Fifty thousand men are
reported to have been slain in the battle of Naissus. Several
large bodies of barbarians, covering their retreat with a movable
fortification of wagons, retired, or rather escaped, from the
field of slaughter.

II. We may presume that some insurmountable difficulty, the
fatigue, perhaps, or the disobedience, of the conquerors,
prevented Claudius from completing in one day the destruction of
the Goths. The war was diffused over the province of Mæsia,
Thrace, and Macedonia, and its operations drawn out into a
variety of marches, surprises, and tumultuary engagements, as
well by sea as by land. When the Romans suffered any loss, it was
commonly occasioned by their own cowardice or rashness; but the
superior talents of the emperor, his perfect knowledge of the
country, and his judicious choice of measures as well as
officers, assured on most occasions the success of his arms. The
immense booty, the fruit of so many victories, consisted for the
greater part of cattle and slaves. A select body of the Gothic
youth was received among the Imperial troops; the remainder was
sold into servitude; and so considerable was the number of female
captives that every soldier obtained to his share two or three
women. A circumstance from which we may conclude, that the
invaders entertained some designs of settlement as well as of
plunder; since even in a naval expedition, they were accompanied
by their families.

III. The loss of their fleet, which was either taken or sunk, had
intercepted the retreat of the Goths. A vast circle of Roman
posts, distributed with skill, supported with firmness, and
gradually closing towards a common centre, forced the barbarians
into the most inaccessible parts of Mount Hæmus, where they found
a safe refuge, but a very scanty subsistence. During the course
of a rigorous winter in which they were besieged by the emperor’s
troops, famine and pestilence, desertion and the sword,
continually diminished the imprisoned multitude. On the return of
spring, nothing appeared in arms except a hardy and desperate
band, the remnant of that mighty host which had embarked at the
mouth of the Niester.

\pagenote[13]{Hist. August. in Claud. Aurelian. et Prob. Zosimus,
l. i. p. 38-42. Zonaras, l. xii. p. 638. Aurel. Victor in Epitom.
Victor Junior in Cæsar. Eutrop. ix ll. Euseb. in Chron.}

The pestilence which swept away such numbers of the barbarians,
at length proved fatal to their conqueror. After a short but
glorious reign of two years, Claudius expired at Sirmium, amidst
the tears and acclamations of his subjects. In his last illness,
he convened the principal officers of the state and army, and in
their presence recommended Aurelian,\textsuperscript{14} one of his generals, as
the most deserving of the throne, and the best qualified to
execute the great design which he himself had been permitted only
to undertake. The virtues of Claudius, his valor, affability,
justice, and temperance, his love of fame and of his country,
place him in that short list of emperors who added lustre to the
Roman purple. Those virtues, however, were celebrated with
peculiar zeal and complacency by the courtly writers of the age
of Constantine, who was the great-grandson of Crispus, the elder
brother of Claudius. The voice of flattery was soon taught to
repeat, that gods, who so hastily had snatched Claudius from the
earth, rewarded his merit and piety by the perpetual
establishment of the empire in his family.\textsuperscript{15}

\pagenote[14]{According to Zonaras, (l. xii. p. 638,) Claudius,
before his death, invested him with the purple; but this singular
fact is rather contradicted than confirmed by other writers.}

\pagenote[15]{See the Life of Claudius by Pollio, and the
Orations of Mamertinus, Eumenius, and Julian. See likewise the
Cæsars of Julian p. 318. In Julian it was not adulation, but
superstition and vanity.}

Notwithstanding these oracles, the greatness of the Flavian
family (a name which it had pleased them to assume) was deferred
above twenty years, and the elevation of Claudius occasioned the
immediate ruin of his brother Quintilius, who possessed not
sufficient moderation or courage to descend into the private
station to which the patriotism of the late emperor had condemned
him. Without delay or reflection, he assumed the purple at
Aquileia, where he commanded a considerable force; and though his
reign lasted only seventeen days,\textsuperscript{151} he had time to obtain the
sanction of the senate, and to experience a mutiny of the troops.

As soon as he was informed that the great army of the Danube had
invested the well-known valor of Aurelian with Imperial power, he
sunk under the fame and merit of his rival; and ordering his
veins to be opened, prudently withdrew himself from the unequal
contest.\textsuperscript{16}

\pagenote[151]{Such is the narrative of the greater part of the
older historians; but the number and the variety of his medals
seem to require more time, and give probability to the report of
Zosimus, who makes him reign some months.—G.}

\pagenote[16]{Zosimus, l. i. p. 42. Pollio (Hist. August. p. 107)
allows him virtues, and says, that, like Pertinax, he was killed
by the licentious soldiers. According to Dexippus, he died of a
disease.}

The general design of this work will not permit us minutely to
relate the actions of every emperor after he ascended the throne,
much less to deduce the various fortunes of his private life. We
shall only observe, that the father of Aurelian was a peasant of
the territory of Sirmium, who occupied a small farm, the property
of Aurelius, a rich senator. His warlike son enlisted in the
troops as a common soldier, successively rose to the rank of a
centurion, a tribune, the præfect of a legion, the inspector of
the camp, the general, or, as it was then called, the duke, of a
frontier; and at length, during the Gothic war, exercised the
important office of commander-in-chief of the cavalry. In every
station he distinguished himself by matchless valor,\textsuperscript{17} rigid
discipline, and successful conduct. He was invested with the
consulship by the emperor Valerian, who styles him, in the
pompous language of that age, the deliverer of Illyricum, the
restorer of Gaul, and the rival of the Scipios. At the
recommendation of Valerian, a senator of the highest rank and
merit, Ulpius Crinitus, whose blood was derived from the same
source as that of Trajan, adopted the Pannonian peasant, gave him
his daughter in marriage, and relieved with his ample fortune the
honorable poverty which Aurelian had preserved inviolate.\textsuperscript{18}

\pagenote[17]{Theoclius (as quoted in the Augustan History, p.
211) affirms that in one day he killed with his own hand
forty-eight Sarmatians, and in several subsequent engagements
nine hundred and fifty. This heroic valor was admired by the
soldiers, and celebrated in their rude songs, the burden of which
was, mille, mile, mille, occidit.}

\pagenote[18]{Acholius (ap. Hist. August. p. 213) describes the
ceremony of the adoption, as it was performed at Byzantium, in
the presence of the emperor and his great officers.}

The reign of Aurelian lasted only four years and about nine
months; but every instant of that short period was filled by some
memorable achievement. He put an end to the Gothic war, chastised
the Germans who invaded Italy, recovered Gaul, Spain, and Britain
out of the hands of Tetricus, and destroyed the proud monarchy
which Zenobia had erected in the East on the ruins of the
afflicted empire.

It was the rigid attention of Aurelian, even to the minutest
articles of discipline, which bestowed such uninterrupted success
on his arms. His military regulations are contained in a very
concise epistle to one of his inferior officers, who is commanded
to enforce them, as he wishes to become a tribune, or as he is
desirous to live. Gaming, drinking, and the arts of divination,
were severely prohibited. Aurelian expected that his soldiers
should be modest, frugal, and laborious; that their armor should
be constantly kept bright, their weapons sharp, their clothing
and horses ready for immediate service; that they should live in
their quarters with chastity and sobriety, without damaging the
cornfields, without stealing even a sheep, a fowl, or a bunch of
grapes, without exacting from their landlords either salt, or
oil, or wood. “The public allowance,” continues the emperor, “is
sufficient for their support; their wealth should be collected
from the spoils of the enemy, not from the tears of the
provincials.”\textsuperscript{19} A single instance will serve to display the
rigor, and even cruelty, of Aurelian. One of the soldiers had
seduced the wife of his host. The guilty wretch was fastened to
two trees forcibly drawn towards each other, and his limbs were
torn asunder by their sudden separation. A few such examples
impressed a salutary consternation. The punishments of Aurelian
were terrible; but he had seldom occasion to punish more than
once the same offence. His own conduct gave a sanction to his
laws, and the seditious legions dreaded a chief who had learned
to obey, and who was worthy to command.

\pagenote[19]{Hist. August, p. 211 This laconic epistle is truly
the work of a soldier; it abounds with military phrases and
words, some of which cannot be understood without difficulty.
Ferramenta samiata is well explained by Salmasius. The former of
the words means all weapons of offence, and is contrasted with
Arma, defensive armor The latter signifies keen and well
sharpened.}

\section{Part \thesection.}

The death of Claudius had revived the fainting spirit of the
Goths. The troops which guarded the passes of Mount Hæmus, and
the banks of the Danube, had been drawn away by the apprehension
of a civil war; and it seems probable that the remaining body of
the Gothic and Vandalic tribes embraced the favorable
opportunity, abandoned their settlements of the Ukraine,
traversed the rivers, and swelled with new multitudes the
destroying host of their countrymen. Their united numbers were at
length encountered by Aurelian, and the bloody and doubtful
conflict ended only with the approach of night.\textsuperscript{20} Exhausted by
so many calamities, which they had mutually endured and inflicted
during a twenty years’ war, the Goths and the Romans consented to
a lasting and beneficial treaty. It was earnestly solicited by
the barbarians, and cheerfully ratified by the legions, to whose
suffrage the prudence of Aurelian referred the decision of that
important question. The Gothic nation engaged to supply the
armies of Rome with a body of two thousand auxiliaries,
consisting entirely of cavalry, and stipulated in return an
undisturbed retreat, with a regular market as far as the Danube,
provided by the emperor’s care, but at their own expense. The
treaty was observed with such religious fidelity, that when a
party of five hundred men straggled from the camp in quest of
plunder, the king or general of the barbarians commanded that the
guilty leader should be apprehended and shot to death with darts,
as a victim devoted to the sanctity of their engagements.\textsuperscript{201} It
is, however, not unlikely, that the precaution of Aurelian, who
had exacted as hostages the sons and daughters of the Gothic
chiefs, contributed something to this pacific temper. The youths
he trained in the exercise of arms, and near his own person: to
the damsels he gave a liberal and Roman education, and by
bestowing them in marriage on some of his principal officers,
gradually introduced between the two nations the closest and most
endearing connections.\textsuperscript{21}

\pagenote[20]{Zosimus, l. i. p. 45.}

\pagenote[201]{The five hundred stragglers were all slain.—M.}

\pagenote[21]{Dexipphus (ap. Excerpta Legat. p. 12) relates the
whole transaction under the name of Vandals. Aurelian married one
of the Gothic ladies to his general Bonosus, who was able to
drink with the Goths and discover their secrets. Hist. August. p.
247.}

But the most important condition of peace was understood rather
than expressed in the treaty. Aurelian withdrew the Roman forces
from Dacia, and tacitly relinquished that great province to the
Goths and Vandals.\textsuperscript{22} His manly judgment convinced him of the
solid advantages, and taught him to despise the seeming disgrace,
of thus contracting the frontiers of the monarchy. The Dacian
subjects, removed from those distant possessions which they were
unable to cultivate or defend, added strength and populousness to
the southern side of the Danube. A fertile territory, which the
repetition of barbarous inroads had changed into a desert, was
yielded to their industry, and a new province of Dacia still
preserved the memory of Trajan’s conquests. The old country of
that name detained, however, a considerable number of its
inhabitants, who dreaded exile more than a Gothic master.\textsuperscript{23}
These degenerate Romans continued to serve the empire, whose
allegiance they had renounced, by introducing among their
conquerors the first notions of agriculture, the useful arts, and
the conveniences of civilized life. An intercourse of commerce
and language was gradually established between the opposite banks
of the Danube; and after Dacia became an independent state, it
often proved the firmest barrier of the empire against the
invasions of the savages of the North. A sense of interest
attached these more settled barbarians to the alliance of Rome,
and a permanent interest very frequently ripens into sincere and
useful friendship. This various colony, which filled the ancient
province, and was insensibly blended into one great people, still
acknowledged the superior renown and authority of the Gothic
tribe, and claimed the fancied honor of a Scandinavian origin. At
the same time, the lucky though accidental resemblance of the
name of Getæ,\textsuperscript{231} infused among the credulous Goths a vain
persuasion, that in a remote age, their own ancestors, already
seated in the Dacian provinces, had received the instructions of
Zamolxis, and checked the victorious arms of Sesostris and
Darius.\textsuperscript{24}

\pagenote[22]{Hist. August. p. 222. Eutrop. ix. 15. Sextus Rufus,
c. 9. de Mortibus Persecutorum, c. 9.}

\pagenote[23]{The Walachians still preserve many traces of the
Latin language and have boasted, in every age, of their Roman
descent. They are surrounded by, but not mixed with, the
barbarians. See a Memoir of M. d’Anville on ancient Dacia, in the
Academy of Inscriptions, tom. xxx.}

\pagenote[231]{The connection between the Getæ and the Goths is
still in my opinion incorrectly maintained by some learned
writers—M.}

\pagenote[24]{See the first chapter of Jornandes. The Vandals,
however, (c. 22,) maintained a short independence between the
Rivers Marisia and Crissia, (Maros and Keres,) which fell into
the Teiss.}

While the vigorous and moderate conduct of Aurelian restored the
Illyrian frontier, the nation of the Alemanni\textsuperscript{25} violated the
conditions of peace, which either Gallienus had purchased, or
Claudius had imposed, and, inflamed by their impatient youth,
suddenly flew to arms. Forty thousand horse appeared in the
field,\textsuperscript{26} and the numbers of the infantry doubled those of the
cavalry.\textsuperscript{27} The first objects of their avarice were a few cities
of the Rhætian frontier; but their hopes soon rising with
success, the rapid march of the Alemanni traced a line of
devastation from the Danube to the Po.\textsuperscript{28}

\pagenote[25]{Dexippus, p. 7—12. Zosimus, l. i. p. 43. Vopiscus
in Aurelian in Hist. August. However these historians differ in
names, (Alemanni Juthungi, and Marcomanni,) it is evident that
they mean the same people, and the same war; but it requires some
care to conciliate and explain them.}

\pagenote[26]{Cantoclarus, with his usual accuracy, chooses to
translate three hundred thousand: his version is equally
repugnant to sense and to grammar.}

\pagenote[27]{We may remark, as an instance of bad taste, that
Dexippus applies to the light infantry of the Alemanni the
technical terms proper only to the Grecian phalanx.}

\pagenote[28]{In Dexippus, we at present read Rhodanus: M. de
Valois very judiciously alters the word to Eridanus.}

The emperor was almost at the same time informed of the
irruption, and of the retreat, of the barbarians. Collecting an
active body of troops, he marched with silence and celerity along
the skirts of the Hercynian forest; and the Alemanni, laden with
the spoils of Italy, arrived at the Danube, without suspecting,
that on the opposite bank, and in an advantageous post, a Roman
army lay concealed and prepared to intercept their return.
Aurelian indulged the fatal security of the barbarians, and
permitted about half their forces to pass the river without
disturbance and without precaution. Their situation and
astonishment gave him an easy victory; his skilful conduct
improved the advantage. Disposing the legions in a semicircular
form, he advanced the two horns of the crescent across the
Danube, and wheeling them on a sudden towards the centre,
enclosed the rear of the German host. The dismayed barbarians, on
whatsoever side they cast their eyes, beheld, with despair, a
wasted country, a deep and rapid stream, a victorious and
implacable enemy.

Reduced to this distressed condition, the Alemanni no longer
disdained to sue for peace. Aurelian received their ambassadors
at the head of his camp, and with every circumstance of martial
pomp that could display the greatness and discipline of Rome. The
legions stood to their arms in well-ordered ranks and awful
silence. The principal commanders, distinguished by the ensigns
of their rank, appeared on horseback on either side of the
Imperial throne. Behind the throne the consecrated images of the
emperor, and his predecessors,\textsuperscript{29} the golden eagles, and the
various titles of the legions, engraved in letters of gold, were
exalted in the air on lofty pikes covered with silver. When
Aurelian assumed his seat, his manly grace and majestic figure\textsuperscript{30}
taught the barbarians to revere the person as well as the purple
of their conqueror. The ambassadors fell prostrate on the ground
in silence. They were commanded to rise, and permitted to speak.
By the assistance of interpreters they extenuated their perfidy,
magnified their exploits, expatiated on the vicissitudes of
fortune and the advantages of peace, and, with an ill-timed
confidence, demanded a large subsidy, as the price of the
alliance which they offered to the Romans. The answer of the
emperor was stern and imperious. He treated their offer with
contempt, and their demand with indignation, reproached the
barbarians, that they were as ignorant of the arts of war as of
the laws of peace, and finally dismissed them with the choice
only of submitting to this unconditional mercy, or awaiting the
utmost severity of his resentment.\textsuperscript{31} Aurelian had resigned a
distant province to the Goths; but it was dangerous to trust or
to pardon these perfidious barbarians, whose formidable power
kept Italy itself in perpetual alarms.

\pagenote[29]{The emperor Claudius was certainly of the number;
but we are ignorant how far this mark of respect was extended; if
to Cæsar and Augustus, it must have produced a very awful
spectacle; a long line of the masters of the world.}

\pagenote[30]{Vopiscus in Hist. August. p. 210.}

\pagenote[31]{Dexippus gives them a subtle and prolix oration,
worthy of a Grecian sophist.}

Immediately after this conference, it should seem that some
unexpected emergency required the emperor’s presence in Pannonia.

He devolved on his lieutenants the care of finishing the
destruction of the Alemanni, either by the sword, or by the surer
operation of famine. But an active despair has often triumphed
over the indolent assurance of success. The barbarians, finding
it impossible to traverse the Danube and the Roman camp, broke
through the posts in their rear, which were more feebly or less
carefully guarded; and with incredible diligence, but by a
different road, returned towards the mountains of Italy.\textsuperscript{32}
Aurelian, who considered the war as totally extinguished,
received the mortifying intelligence of the escape of the
Alemanni, and of the ravage which they already committed in the
territory of Milan. The legions were commanded to follow, with as
much expedition as those heavy bodies were capable of exerting,
the rapid flight of an enemy whose infantry and cavalry moved
with almost equal swiftness. A few days afterwards, the emperor
himself marched to the relief of Italy, at the head of a chosen
body of auxiliaries, (among whom were the hostages and cavalry of
the Vandals,) and of all the Prætorian guards who had served in
the wars on the Danube.\textsuperscript{33}

\pagenote[32]{Hist. August. p. 215.}

\pagenote[33]{Dexippus, p. 12.}

As the light troops of the Alemanni had spread themselves from
the Alps to the Apennine, the incessant vigilance of Aurelian and
his officers was exercised in the discovery, the attack, and the
pursuit of the numerous detachments. Notwithstanding this
desultory war, three considerable battles are mentioned, in which
the principal force of both armies was obstinately engaged.\textsuperscript{34}
The success was various. In the first, fought near Placentia, the
Romans received so severe a blow, that, according to the
expression of a writer extremely partial to Aurelian, the
immediate dissolution of the empire was apprehended.\textsuperscript{35} The
crafty barbarians, who had lined the woods, suddenly attacked the
legions in the dusk of the evening, and, it is most probable,
after the fatigue and disorder of a long march.

The fury of their charge was irresistible; but, at length, after
a dreadful slaughter, the patient firmness of the emperor rallied
his troops, and restored, in some degree, the honor of his arms.
The second battle was fought near Fano in Umbria; on the spot
which, five hundred years before, had been fatal to the brother
of Hannibal.\textsuperscript{36} Thus far the successful Germans had advanced
along the Æmilian and Flaminian way, with a design of sacking the
defenceless mistress of the world. But Aurelian, who, watchful
for the safety of Rome, still hung on their rear, found in this
place the decisive moment of giving them a total and
irretrievable defeat.\textsuperscript{37} The flying remnant of their host was
exterminated in a third and last battle near Pavia; and Italy was
delivered from the inroads of the Alemanni.

\pagenote[34]{Victor Junior in Aurelian.}

\pagenote[35]{Vopiscus in Hist. August. p. 216.}

\pagenote[36]{The little river, or rather torrent, of, Metaurus,
near Fano, has been immortalized, by finding such an historian as
Livy, and such a poet as Horace.}

\pagenote[37]{It is recorded by an inscription found at Pesaro.
See Gruter cclxxvi. 3.}

Fear has been the original parent of superstition, and every new
calamity urges trembling mortals to deprecate the wrath of their
invisible enemies. Though the best hope of the republic was in
the valor and conduct of Aurelian, yet such was the public
consternation, when the barbarians were hourly expected at the
gates of Rome, that, by a decree of the senate the Sibylline
books were consulted. Even the emperor himself, from a motive
either of religion or of policy, recommended this salutary
measure, chided the tardiness of the senate,\textsuperscript{38} and offered to
supply whatever expense, whatever animals, whatever captives of
any nation, the gods should require. Notwithstanding this liberal
offer, it does not appear, that any human victims expiated with
their blood the sins of the Roman people. The Sibylline books
enjoined ceremonies of a more harmless nature, processions of
priests in white robes, attended by a chorus of youths and
virgins; lustrations of the city and adjacent country; and
sacrifices, whose powerful influence disabled the barbarians from
passing the mystic ground on which they had been celebrated.
However puerile in themselves, these superstitious arts were
subservient to the success of the war; and if, in the decisive
battle of Fano, the Alemanni fancied they saw an army of spectres
combating on the side of Aurelian, he received a real and
effectual aid from this imaginary reënforcement.\textsuperscript{39}

\pagenote[38]{One should imagine, he said, that you were
assembled in a Christian church, not in the temple of all the
gods.}

\pagenote[39]{Vopiscus, in Hist. August. p. 215, 216, gives a
long account of these ceremonies from the Registers of the
senate.}

But whatever confidence might be placed in ideal ramparts, the
experience of the past, and the dread of the future, induced the
Romans to construct fortifications of a grosser and more
substantial kind. The seven hills of Rome had been surrounded by
the successors of Romulus with an ancient wall of more than
thirteen miles.\textsuperscript{40} The vast enclosure may seem disproportioned to
the strength and numbers of the infant-state. But it was
necessary to secure an ample extent of pasture and arable land
against the frequent and sudden incursions of the tribes of
Latium, the perpetual enemies of the republic. With the progress
of Roman greatness, the city and its inhabitants gradually
increased, filled up the vacant space, pierced through the
useless walls, covered the field of Mars, and, on every side,
followed the public highways in long and beautiful suburbs.\textsuperscript{41}
The extent of the new walls, erected by Aurelian, and finished in
the reign of Probus, was magnified by popular estimation to near
fifty,\textsuperscript{42} but is reduced by accurate measurement to about
twenty-one miles. \textsuperscript{43} It was a great but a melancholy labor, since
the defence of the capital betrayed the decline of monarchy. The
Romans of a more prosperous age, who trusted to the arms of the
legions the safety of the frontier camps,\textsuperscript{44} were very far from
entertaining a suspicion that it would ever become necessary to
fortify the seat of empire against the inroads of the barbarians.\textsuperscript{45}

\pagenote[40]{Plin. Hist. Natur. iii. 5. To confirm our idea, we
may observe, that for a long time Mount Cælius was a grove of
oaks, and Mount Viminal was overrun with osiers; that, in the
fourth century, the Aventine was a vacant and solitary
retirement; that, till the time of Augustus, the Esquiline was an
unwholesome burying-ground; and that the numerous inequalities,
remarked by the ancients in the Quirinal, sufficiently prove that
it was not covered with buildings. Of the seven hills, the
Capitoline and Palatine only, with the adjacent valleys, were the
primitive habitations of the Roman people. But this subject would
require a dissertation.}

\pagenote[41]{Exspatiantia tecta multas addidere urbes, is the
expression of Pliny.}

\pagenote[42]{Hist. August. p. 222. Both Lipsius and Isaac
Vossius have eagerly embraced this measure.}

\pagenote[43]{See Nardini, Roman Antica, l. i. c. 8. * Note: But
compare Gibbon, ch. xli. note 77.—M.}

\pagenote[44]{Tacit. Hist. iv. 23.}

\pagenote[45]{For Aurelian’s walls, see Vopiscus in Hist. August.
p. 216, 222. Zosimus, l. i. p. 43. Eutropius, ix. 15. Aurel.
Victor in Aurelian Victor Junior in Aurelian. Euseb. Hieronym. et
Idatius in Chronic}

The victory of Claudius over the Goths, and the success of
Aurelian against the Alemanni, had already restored to the arms
of Rome their ancient superiority over the barbarous nations of
the North. To chastise domestic tyrants, and to reunite the
dismembered parts of the empire, was a task reserved for the
second of those warlike emperors. Though he was acknowledged by
the senate and people, the frontiers of Italy, Africa, Illyricum,
and Thrace, confined the limits of his reign. Gaul, Spain, and
Britain, Egypt, Syria, and Asia Minor, were still possessed by
two rebels, who alone, out of so numerous a list, had hitherto
escaped the dangers of their situation; and to complete the
ignominy of Rome, these rival thrones had been usurped by women.

A rapid succession of monarchs had arisen and fallen in the
provinces of Gaul. The rigid virtues of Posthumus served only to
hasten his destruction. After suppressing a competitor, who had
assumed the purple at Mentz, he refused to gratify his troops
with the plunder of the rebellious city; and in the seventh year
of his reign, became the victim of their disappointed avarice.\textsuperscript{46}
The death of Victorinus, his friend and associate, was occasioned
by a less worthy cause. The shining accomplishments\textsuperscript{47} of that
prince were stained by a licentious passion, which he indulged in
acts of violence, with too little regard to the laws of society,
or even to those of love.\textsuperscript{48} He was slain at Cologne, by a
conspiracy of jealous husbands, whose revenge would have appeared
more justifiable, had they spared the innocence of his son. After
the murder of so many valiant princes, it is somewhat remarkable,
that a female for a long time controlled the fierce legions of
Gaul, and still more singular, that she was the mother of the
unfortunate Victorinus. The arts and treasures of Victoria
enabled her successively to place Marius and Tetricus on the
throne, and to reign with a manly vigor under the name of those
dependent emperors. Money of copper, of silver, and of gold, was
coined in her name; she assumed the titles of Augusta and Mother
of the Camps: her power ended only with her life; but her life
was perhaps shortened by the ingratitude of Tetricus.\textsuperscript{49}

\pagenote[46]{His competitor was Lollianus, or Ælianus, if,
indeed, these names mean the same person. See Tillemont, tom.
iii. p. 1177. Note: The medals which bear the name of Lollianus
are considered forgeries except one in the museum of the Prince
of Waldeck there are many extent bearing the name of Lælianus,
which appears to have been that of the competitor of Posthumus.
Eckhel. Doct. Num. t. vi. 149—G.}

\pagenote[47]{The character of this prince by Julius Aterianus
(ap. Hist. August. p. 187) is worth transcribing, as it seems
fair and impartial Victorino qui Post Junium Posthumium Gallias
rexit neminem existemo præferendum; non in virtute Trajanum; non
Antoninum in clementia; non in gravitate Nervam; non in
gubernando ærario Vespasianum; non in Censura totius vitæ ac
severitate militari Pertinacem vel Severum. Sed omnia hæc libido
et cupiditas voluptatis mulierriæ sic perdidit, ut nemo audeat
virtutes ejus in literas mittere quem constat omnium judicio
meruisse puniri.}

\pagenote[48]{He ravished the wife of Attitianus, an actuary, or
army agent, Hist. August. p. 186. Aurel. Victor in Aurelian.}

\pagenote[49]{Pollio assigns her an article among the thirty
tyrants. Hist. August. p. 200.}

When, at the instigation of his ambitious patroness, Tetricus
assumed the ensigns of royalty, he was governor of the peaceful
province of Aquitaine, an employment suited to his character and
education. He reigned four or five years over Gaul, Spain, and
Britain, the slave and sovereign of a licentious army, whom he
dreaded, and by whom he was despised. The valor and fortune of
Aurelian at length opened the prospect of a deliverance. He
ventured to disclose his melancholy situation, and conjured the
emperor to hasten to the relief of his unhappy rival. Had this
secret correspondence reached the ears of the soldiers, it would
most probably have cost Tetricus his life; nor could he resign
the sceptre of the West without committing an act of treason
against himself. He affected the appearances of a civil war, led
his forces into the field, against Aurelian, posted them in the
most disadvantageous manner, betrayed his own counsels to his
enemy, and with a few chosen friends deserted in the beginning of
the action. The rebel legions, though disordered and dismayed by
the unexpected treachery of their chief, defended themselves with
desperate valor, till they were cut in pieces almost to a man, in
this bloody and memorable battle, which was fought near Chalons
in Champagne.\textsuperscript{50} The retreat of the irregular auxiliaries, Franks
and Batavians,\textsuperscript{51} whom the conqueror soon compelled or persuaded
to repass the Rhine, restored the general tranquillity, and the
power of Aurelian was acknowledged from the wall of Antoninus to
the columns of Hercules.

\pagenote[50]{Pollio in Hist. August. p. 196. Vopiscus in Hist.
August. p. 220. The two Victors, in the lives of Gallienus and
Aurelian. Eutrop. ix. 13. Euseb. in Chron. Of all these writers,
only the two last (but with strong probability) place the fall of
Tetricus before that of Zenobia. M. de Boze (in the Academy of
Inscriptions, tom. xxx.) does not wish, and Tillemont (tom. iii.
p. 1189) does not dare to follow them. I have been fairer than
the one, and bolder than the other.}

\pagenote[51]{Victor Junior in Aurelian. Eumenius mentions
Batavicœ; some critics, without any reason, would fain alter the
word to Bagandicœ.}

As early as the reign of Claudius, the city
of Autun, alone and unassisted, had ventured to declare against
the legions of Gaul. After a siege of seven months, they stormed
and plundered that unfortunate city, already wasted by famine.\textsuperscript{52}
Lyons, on the contrary, had resisted with obstinate disaffection
the arms of Aurelian. We read of the punishment of Lyons,\textsuperscript{53} but
there is not any mention of the rewards of Autun. Such, indeed,
is the policy of civil war: severely to remember injuries, and to
forget the most important services. Revenge is profitable,
gratitude is expensive.

\pagenote[52]{Eumen. in Vet. Panegyr. iv. 8.}

\pagenote[53]{Vopiscus in Hist. August. p. 246. Autun was not
restored till the reign of Diocletian. See Eumenius de
restaurandis scholis.}

Aurelian had no sooner secured the person and provinces of
Tetricus, than he turned his arms against Zenobia, the celebrated
queen of Palmyra and the East. Modern Europe has produced several
illustrious women who have sustained with glory the weight of
empire; nor is our own age destitute of such distinguished
characters. But if we except the doubtful achievements of
Semiramis, Zenobia is perhaps the only female whose superior
genius broke through the servile indolence imposed on her sex by
the climate and manners of Asia.\textsuperscript{54} She claimed her descent from
the Macedonian kings of Egypt,\textsuperscript{541} equalled in beauty her
ancestor Cleopatra, and far surpassed that princess in chastity\textsuperscript{55}
and valor. Zenobia was esteemed the most lovely as well as the
most heroic of her sex. She was of a dark complexion (for in
speaking of a lady these trifles become important). Her teeth
were of a pearly whiteness, and her large black eyes sparkled
with uncommon fire, tempered by the most attractive sweetness.
Her voice was strong and harmonious. Her manly understanding was
strengthened and adorned by study. She was not ignorant of the
Latin tongue, but possessed in equal perfection the Greek, the
Syriac, and the Egyptian languages. She had drawn up for her own
use an epitome of oriental history, and familiarly compared the
beauties of Homer and Plato under the tuition of the sublime
Longinus.

\pagenote[54]{Almost everything that is said of the manners of
Odenathus and Zenobia is taken from their lives in the Augustan
History, by Trebeljus Pollio; see p. 192, 198.}

\pagenote[541]{According to some Christian writers, Zenobia was a
Jewess. (Jost Geschichte der Israel. iv. 16. Hist. of Jews, iii.
175.)—M.}

\pagenote[55]{She never admitted her husband’s embraces but for
the sake of posterity. If her hopes were baffled, in the ensuing
month she reiterated the experiment.}

This accomplished woman gave her hand to Odenathus,\textsuperscript{551} who, from
a private station, raised himself to the dominion of the East.
She soon became the friend and companion of a hero. In the
intervals of war, Odenathus passionately delighted in the
exercise of hunting; he pursued with ardor the wild beasts of the
desert, lions, panthers, and bears; and the ardor of Zenobia in
that dangerous amusement was not inferior to his own. She had
inured her constitution to fatigue, disdained the use of a
covered carriage, generally appeared on horseback in a military
habit, and sometimes marched several miles on foot at the head of
the troops. The success of Odenathus was in a great measure
ascribed to her incomparable prudence and fortitude. Their
splendid victories over the Great King, whom they twice pursued
as far as the gates of Ctesiphon, laid the foundations of their
united fame and power. The armies which they commanded, and the
provinces which they had saved, acknowledged not any other
sovereigns than their invincible chiefs. The senate and people of
Rome revered a stranger who had avenged their captive emperor,
and even the insensible son of Valerian accepted Odenathus for
his legitimate colleague.

\pagenote[551]{According to Zosimus, Odenathus was of a noble
family in Palmyra and according to Procopius, he was prince of
the Saracens, who inhabit the ranks of the Euphrates. Echhel.
Doct. Num. vii. 489.—G.}

\section{Part \thesection.}

After a successful expedition against the Gothic plunderers of
Asia, the Palmyrenian prince returned to the city of Emesa in
Syria. Invincible in war, he was there cut off by domestic
treason, and his favorite amusement of hunting was the cause, or
at least the occasion, of his death.\textsuperscript{56} His nephew Mæonius
presumed to dart his javelin before that of his uncle; and though
admonished of his error, repeated the same insolence. As a
monarch, and as a sportsman, Odenathus was provoked, took away
his horse, a mark of ignominy among the barbarians, and chastised
the rash youth by a short confinement. The offence was soon
forgot, but the punishment was remembered; and Mæonius, with a
few daring associates, assassinated his uncle in the midst of a
great entertainment. Herod, the son of Odenathus, though not of
Zenobia, a young man of a soft and effeminate temper,\textsuperscript{57} was
killed with his father. But Mæonius obtained only the pleasure of
revenge by this bloody deed. He had scarcely time to assume the
title of Augustus, before he was sacrificed by Zenobia to the
memory of her husband.\textsuperscript{58}

\pagenote[56]{Hist. August. p. 192, 193. Zosimus, l. i. p. 36.
Zonaras, l. xii p. 633. The last is clear and probable, the
others confused and inconsistent. The text of Syncellus, if not
corrupt, is absolute nonsense.}

\pagenote[57]{Odenathus and Zenobia often sent him, from the
spoils of the enemy, presents of gems and toys, which he received
with infinite delight.}

\pagenote[58]{Some very unjust suspicions have been cast on
Zenobia, as if she was accessory to her husband’s death.}

With the assistance of his most faithful friends, she immediately
filled the vacant throne, and governed with manly counsels
Palmyra, Syria, and the East, above five years. By the death of
Odenathus, that authority was at an end which the senate had
granted him only as a personal distinction; but his martial
widow, disdaining both the senate and Gallienus, obliged one of
the Roman generals, who was sent against her, to retreat into
Europe, with the loss of his army and his reputation.\textsuperscript{59} Instead
of the little passions which so frequently perplex a female
reign, the steady administration of Zenobia was guided by the
most judicious maxims of policy. If it was expedient to pardon,
she could calm her resentment; if it was necessary to punish, she
could impose silence on the voice of pity. Her strict economy was
accused of avarice; yet on every proper occasion she appeared
magnificent and liberal. The neighboring states of Arabia,
Armenia, and Persia, dreaded her enmity, and solicited her
alliance. To the dominions of Odenathus, which extended from the
Euphrates to the frontiers of Bithynia, his widow added the
inheritance of her ancestors, the populous and fertile kingdom of
Egypt.\textsuperscript{60} The emperor Claudius acknowledged her merit, and was
content, that, while \textit{he} pursued the Gothic war, \textit{she} should
assert the dignity of the empire in the East. The conduct,
however, of Zenobia was attended with some ambiguity; not is it
unlikely that she had conceived the design of erecting an
independent and hostile monarchy. She blended with the popular
manners of Roman princes the stately pomp of the courts of Asia,
and exacted from her subjects the same adoration that was paid to
the successor of Cyrus. She bestowed on her three sons\textsuperscript{61} a Latin
education, and often showed them to the troops adorned with the
Imperial purple. For herself she reserved the diadem, with the
splendid but doubtful title of Queen of the East.

\pagenote[59]{Hist. August. p. 180, 181.}

\pagenote[60]{[See, in Hist. August. p. 198, Aurelian’s testimony
to her merit; and for the conquest of Egypt, Zosimus, l. i. p.
39, 40.] This seems very doubtful. Claudius, during all his
reign, is represented as emperor on the medals of Alexandria,
which are very numerous. If Zenobia possessed any power in Egypt,
it could only have been at the beginning of the reign of
Aurelian. The same circumstance throws great improbability on her
conquests in Galatia. Perhaps Zenobia administered Egypt in the
name of Claudius, and emboldened by the death of that prince,
subjected it to her own power.—G.}

\pagenote[61]{Timolaus, Herennianus, and Vaballathus. It is
supposed that the two former were already dead before the war. On
the last, Aurelian bestowed a small province of Armenia, with the
title of King; several of his medals are still extant. See
Tillemont, tom. 3, p. 1190.}

When Aurelian passed over into Asia, against an adversary whose
sex alone could render her an object of contempt, his presence
restored obedience to the province of Bithynia, already shaken by
the arms and intrigues of Zenobia.\textsuperscript{62} Advancing at the head of
his legions, he accepted the submission of Ancyra, and was
admitted into Tyana, after an obstinate siege, by the help of a
perfidious citizen. The generous though fierce temper of Aurelian
abandoned the traitor to the rage of the soldiers; a
superstitious reverence induced him to treat with lenity the
countrymen of Apollonius the philosopher.\textsuperscript{63} Antioch was deserted
on his approach, till the emperor, by his salutary edicts,
recalled the fugitives, and granted a general pardon to all who,
from necessity rather than choice, had been engaged in the
service of the Palmyrenian Queen. The unexpected mildness of such
a conduct reconciled the minds of the Syrians, and as far as the
gates of Emesa, the wishes of the people seconded the terror of
his arms.\textsuperscript{64}

\pagenote[62]{Zosimus, l. i. p. 44.}

\pagenote[63]{Vopiscus (in Hist. August. p. 217) gives us an
authentic letter and a doubtful vision, of Aurelian. Apollonius
of Tyana was born about the same time as Jesus Christ. His life
(that of the former) is related in so fabulous a manner by his
disciples, that we are at a loss to discover whether he was a
sage, an impostor, or a fanatic.}

\pagenote[64]{Zosimus, l. i. p. 46.}

Zenobia would have ill deserved her reputation, had she
indolently permitted the emperor of the West to approach within a
hundred miles of her capital. The fate of the East was decided in
two great battles; so similar in almost every circumstance, that
we can scarcely distinguish them from each other, except by
observing that the first was fought near Antioch,\textsuperscript{65} and the
second near Emesa.\textsuperscript{66} In both the queen of Palmyra animated the
armies by her presence, and devolved the execution of her orders
on Zabdas, who had already signalized his military talents by the
conquest of Egypt. The numerous forces of Zenobia consisted for
the most part of light archers, and of heavy cavalry clothed in
complete steel. The Moorish and Illyrian horse of Aurelian were
unable to sustain the ponderous charge of their antagonists. They
fled in real or affected disorder, engaged the Palmyrenians in a
laborious pursuit, harassed them by a desultory combat, and at
length discomfited this impenetrable but unwieldy body of
cavalry. The light infantry, in the mean time, when they had
exhausted their quivers, remaining without protection against a
closer onset, exposed their naked sides to the swords of the
legions. Aurelian had chosen these veteran troops, who were
usually stationed on the Upper Danube, and whose valor had been
severely tried in the Alemannic war.\textsuperscript{67} After the defeat of
Emesa, Zenobia found it impossible to collect a third army. As
far as the frontier of Egypt, the nations subject to her empire
had joined the standard of the conqueror, who detached Probus,
the bravest of his generals, to possess himself of the Egyptian
provinces. Palmyra was the last resource of the widow of
Odenathus. She retired within the walls of her capital, made
every preparation for a vigorous resistance, and declared, with
the intrepidity of a heroine, that the last moment of her reign
and of her life should be the same.

\pagenote[65]{At a place called Immæ. Eutropius, Sextus Rufus,
and Jerome, mention only this first battle.}

\pagenote[66]{Vopiscus (in Hist. August. p. 217) mentions only
the second.}

\pagenote[67]{Zosimus, l. i. p. 44—48. His account of the two
battles is clear and circumstantial.}

Amid the barren deserts of Arabia, a few cultivated spots rise
like islands out of the sandy ocean. Even the name of Tadmor, or
Palmyra, by its signification in the Syriac as well as in the
Latin language, denoted the multitude of palm-trees which
afforded shade and verdure to that temperate region. The air was
pure, and the soil, watered by some invaluable springs, was
capable of producing fruits as well as corn. A place possessed of
such singular advantages, and situated at a convenient distance\textsuperscript{68}
between the Gulf of Persia and the Mediterranean, was soon
frequented by the caravans which conveyed to the nations of
Europe a considerable part of the rich commodities of India.
Palmyra insensibly increased into an opulent and independent
city, and connecting the Roman and the Parthian monarchies by the
mutual benefits of commerce, was suffered to observe an humble
neutrality, till at length, after the victories of Trajan, the
little republic sunk into the bosom of Rome, and flourished more
than one hundred and fifty years in the subordinate though
honorable rank of a colony. It was during that peaceful period,
if we may judge from a few remaining inscriptions, that the
wealthy Palmyrenians constructed those temples, palaces, and
porticos of Grecian architecture, whose ruins, scattered over an
extent of several miles, have deserved the curiosity of our
travellers. The elevation of Odenathus and Zenobia appeared to
reflect new splendor on their country, and Palmyra, for a while,
stood forth the rival of Rome: but the competition was fatal, and
ages of prosperity were sacrificed to a moment of glory.\textsuperscript{69}

\pagenote[68]{It was five hundred and thirty-seven miles from
Seleucia, and two hundred and three from the nearest coast of
Syria, according to the reckoning of Pliny, who, in a few words,
(Hist. Natur. v. 21,) gives an excellent description of Palmyra.
* Note: Talmor, or Palmyra, was probably at a very early period
the connecting link between the commerce of Tyre and Babylon.
Heeren, Ideen, v. i. p. ii. p. 125. Tadmor was probably built by
Solomon as a commercial station. Hist. of Jews, v. p. 271—M.}

\pagenote[69]{Some English travellers from Aleppo discovered the
ruins of Palmyra about the end of the last century. Our curiosity
has since been gratified in a more splendid manner by Messieurs
Wood and Dawkins. For the history of Palmyra, we may consult the
masterly dissertation of Dr. Halley in the Philosophical
Transactions: Lowthorp’s Abridgment, vol. iii. p. 518.}

In his march over the sandy desert between Emesa and Palmyra, the
emperor Aurelian was perpetually harassed by the Arabs; nor could
he always defend his army, and especially his baggage, from those
flying troops of active and daring robbers, who watched the
moment of surprise, and eluded the slow pursuit of the legions.
The siege of Palmyra was an object far more difficult and
important, and the emperor, who, with incessant vigor, pressed
the attacks in person, was himself wounded with a dart. “The
Roman people,” says Aurelian, in an original letter, “speak with
contempt of the war which I am waging against a woman. They are
ignorant both of the character and of the power of Zenobia. It is
impossible to enumerate her warlike preparations, of stones, of
arrows, and of every species of missile weapons. Every part of
the walls is provided with two or three \textit{balistæ} and artificial
fires are thrown from her military engines. The fear of
punishment has armed her with a desperate courage. Yet still I
trust in the protecting deities of Rome, who have hitherto been
favorable to all my undertakings.”\textsuperscript{70} Doubtful, however, of the
protection of the gods, and of the event of the siege, Aurelian
judged it more prudent to offer terms of an advantageous
capitulation; to the queen, a splendid retreat; to the citizens,
their ancient privileges. His proposals were obstinately
rejected, and the refusal was accompanied with insult.

\pagenote[70]{Vopiscus in Hist. August. p. 218.}

The firmness of Zenobia was supported by the hope, that in a very
short time famine would compel the Roman army to repass the
desert; and by the reasonable expectation that the kings of the
East, and particularly the Persian monarch, would arm in the
defence of their most natural ally. But fortune, and the
perseverance of Aurelian, overcame every obstacle. The death of
Sapor, which happened about this time,\textsuperscript{71} distracted the councils
of Persia, and the inconsiderable succors that attempted to
relieve Palmyra were easily intercepted either by the arms or the
liberality of the emperor. From every part of Syria, a regular
succession of convoys safely arrived in the camp, which was
increased by the return of Probus with his victorious troops from
the conquest of Egypt. It was then that Zenobia resolved to fly.
She mounted the fleetest of her dromedaries,\textsuperscript{72} and had already
reached the banks of the Euphrates, about sixty miles from
Palmyra, when she was overtaken by the pursuit of Aurelian’s
light horse, seized, and brought back a captive to the feet of
the emperor. Her capital soon afterwards surrendered, and was
treated with unexpected lenity. The arms, horses, and camels,
with an immense treasure of gold, silver, silk, and precious
stones, were all delivered to the conqueror, who, leaving only a
garrison of six hundred archers, returned to Emesa, and employed
some time in the distribution of rewards and punishments at the
end of so memorable a war, which restored to the obedience of
Rome those provinces that had renounced their allegiance since
the captivity of Valerian.

\pagenote[71]{From a very doubtful chronology I have endeavored
to extract the most probable date.}

\pagenote[72]{Hist. August. p. 218. Zosimus, l. i. p. 50. Though
the camel is a heavy beast of burden, the dromedary, which is
either of the same or of a kindred species, is used by the
natives of Asia and Africa on all occasions which require
celerity. The Arabs affirm, that he will run over as much ground
in one day as their fleetest horses can perform in eight or ten.
See Buffon, Hist. Naturelle, tom. xi. p. 222, and Shaw’s Travels
p. 167}

When the Syrian queen was brought into the presence of Aurelian,
he sternly asked her, How she had presumed to rise in arms
against the emperors of Rome! The answer of Zenobia was a prudent
mixture of respect and firmness. “Because I disdained to consider
as Roman emperors an Aureolus or a Gallienus. You alone I
acknowledge as my conqueror and my sovereign.”\textsuperscript{73} But as female
fortitude is commonly artificial, so it is seldom steady or
consistent. The courage of Zenobia deserted her in the hour of
trial; she trembled at the angry clamors of the soldiers, who
called aloud for her immediate execution, forgot the generous
despair of Cleopatra, which she had proposed as her model, and
ignominiously purchased life by the sacrifice of her fame and her
friends. It was to their counsels, which governed the weakness of
her sex, that she imputed the guilt of her obstinate resistance;
it was on their heads that she directed the vengeance of the
cruel Aurelian. The fame of Longinus, who was included among the
numerous and perhaps innocent victims of her fear, will survive
that of the queen who betrayed, or the tyrant who condemned him.
Genius and learning were incapable of moving a fierce unlettered
soldier, but they had served to elevate and harmonize the soul of
Longinus. Without uttering a complaint, he calmly followed the
executioner, pitying his unhappy mistress, and bestowing comfort
on his afflicted friends.\textsuperscript{74}

\pagenote[73]{Pollio in Hist. August. p. 199.}

\pagenote[74]{Vopiscus in Hist. August. p. 219. Zosimus, l. i. p.
51.}

Returning from the conquest of the East, Aurelian had already
crossed the Straits which divided Europe from Asia, when he was
provoked by the intelligence that the Palmyrenians had massacred
the governor and garrison which he had left among them, and again
erected the standard of revolt. Without a moment’s deliberation,
he once more turned his face towards Syria. Antioch was alarmed
by his rapid approach, and the helpless city of Palmyra felt the
irresistible weight of his resentment. We have a letter of
Aurelian himself, in which he acknowledges,\textsuperscript{75} that old men,
women, children, and peasants, had been involved in that dreadful
execution, which should have been confined to armed rebellion;
and although his principal concern seems directed to the
reëstablishment of a temple of the Sun, he discovers some pity
for the remnant of the Palmyrenians, to whom he grants the
permission of rebuilding and inhabiting their city. But it is
easier to destroy than to restore. The seat of commerce, of arts,
and of Zenobia, gradually sunk into an obscure town, a trifling
fortress, and at length a miserable village. The present citizens
of Palmyra, consisting of thirty or forty families, have erected
their mud cottages within the spacious court of a magnificent
temple.

\pagenote[75]{Hist. August. p. 219.}

Another and a last labor still awaited the indefatigable
Aurelian; to suppress a dangerous though obscure rebel, who,
during the revolt of Palmyra, had arisen on the banks of the
Nile. Firmus, the friend and ally, as he proudly styled himself,
of Odenathus and Zenobia, was no more than a wealthy merchant of
Egypt. In the course of his trade to India, he had formed very
intimate connections with the Saracens and the Blemmyes, whose
situation on either coast of the Red Sea gave them an easy
introduction into the Upper Egypt. The Egyptians he inflamed with
the hope of freedom, and, at the head of their furious multitude,
broke into the city of Alexandria, where he assumed the Imperial
purple, coined money, published edicts, and raised an army,
which, as he vainly boasted, he was capable of maintaining from
the sole profits of his paper trade. Such troops were a feeble
defence against the approach of Aurelian; and it seems almost
unnecessary to relate, that Firmus was routed, taken, tortured,
and put to death.\textsuperscript{76} Aurelian might now congratulate the senate,
the people, and himself, that in little more than three years, he
had restored universal peace and order to the Roman world.

\pagenote[76]{See Vopiscus in Hist. August. p. 220, 242. As an
instance of luxury, it is observed, that he had glass windows. He
was remarkable for his strength and appetite, his courage and
dexterity. From the letter of Aurelian, we may justly infer, that
Firmus was the last of the rebels, and consequently that Tetricus
was already suppressed.}

Since the foundation of Rome, no general had more nobly deserved
a triumph than Aurelian; nor was a triumph ever celebrated with
superior pride and magnificence.\textsuperscript{77} The pomp was opened by twenty
elephants, four royal tigers, and above two hundred of the most
curious animals from every climate of the North, the East, and
the South. They were followed by sixteen hundred gladiators,
devoted to the cruel amusement of the amphitheatre. The wealth of
Asia, the arms and ensigns of so many conquered nations, and the
magnificent plate and wardrobe of the Syrian queen, were disposed
in exact symmetry or artful disorder. The ambassadors of the most
remote parts of the earth, of Æthiopia, Arabia, Persia,
Bactriana, India, and China, all remarkable by their rich or
singular dresses, displayed the fame and power of the Roman
emperor, who exposed likewise to the public view the presents
that he had received, and particularly a great number of crowns
of gold, the offerings of grateful cities.

The victories of Aurelian were attested by the long train of
captives who reluctantly attended his triumph, Goths, Vandals,
Sarmatians, Alemanni, Franks, Gauls, Syrians, and Egyptians. Each
people was distinguished by its peculiar inscription, and the
title of Amazons was bestowed on ten martial heroines of the
Gothic nation who had been taken in arms.\textsuperscript{78} But every eye,
disregarding the crowd of captives, was fixed on the emperor
Tetricus and the queen of the East. The former, as well as his
son, whom he had created Augustus, was dressed in Gallic
trousers,\textsuperscript{79} a saffron tunic, and a robe of purple. The beauteous
figure of Zenobia was confined by fetters of gold; a slave
supported the gold chain which encircled her neck, and she almost
fainted under the intolerable weight of jewels. She preceded on
foot the magnificent chariot, in which she once hoped to enter
the gates of Rome. It was followed by two other chariots, still
more sumptuous, of Odenathus and of the Persian monarch. The
triumphal car of Aurelian (it had formerly been used by a Gothic
king) was drawn, on this memorable occasion, either by four stags
or by four elephants.\textsuperscript{80} The most illustrious of the senate, the
people, and the army, closed the solemn procession. Unfeigned
joy, wonder, and gratitude, swelled the acclamations of the
multitude; but the satisfaction of the senate was clouded by the
appearance of Tetricus; nor could they suppress a rising murmur,
that the haughty emperor should thus expose to public ignominy
the person of a Roman and a magistrate.\textsuperscript{81}

\pagenote[77]{See the triumph of Aurelian, described by Vopiscus.
He relates the particulars with his usual minuteness; and, on
this occasion, they happen to be interesting. Hist. August. p.
220.}

\pagenote[78]{Among barbarous nations, women have often combated
by the side of their husbands. But it is almost impossible that a
society of Amazons should ever have existed either in the old or
new world. * Note: Klaproth’s theory on the origin of such
traditions is at least recommended by its ingenuity. The males of
a tribe having gone out on a marauding expedition, and having
been cut off to a man, the females may have endeavored, for a
time, to maintain their independence in their camp village, till
their children grew up. Travels, ch. xxx. Eng. Trans—M.}

\pagenote[79]{The use of braccœ, breeches, or trousers, was
still considered in Italy as a Gallic and barbarian fashion. The
Romans, however, had made great advances towards it. To encircle
the legs and thighs with fasciœ, or bands, was understood, in
the time of Pompey and Horace, to be a proof of ill health or
effeminacy. In the age of Trajan, the custom was confined to the
rich and luxurious. It gradually was adopted by the meanest of
the people. See a very curious note of Casaubon, ad Sueton. in
August. c. 82.}

\pagenote[80]{Most probably the former; the latter seen on the
medals of Aurelian, only denote (according to the learned
Cardinal Norris) an oriental victory.}

\pagenote[81]{The expression of Calphurnius, (Eclog. i. 50)
Nullos decet captiva triumphos, as applied to Rome, contains a
very manifest allusion and censure.}

But however, in the treatment of his unfortunate rivals, Aurelian
might indulge his pride, he behaved towards them with a generous
clemency, which was seldom exercised by the ancient conquerors.
Princes who, without success, had defended their throne or
freedom, were frequently strangled in prison, as soon as the
triumphal pomp ascended the Capitol. These usurpers, whom their
defeat had convicted of the crime of treason, were permitted to
spend their lives in affluence and honorable repose.

The emperor presented Zenobia with an elegant villa at Tibur, or
Tivoli, about twenty miles from the capital; the Syrian queen
insensibly sunk into a Roman matron, her daughters married into
noble families, and her race was not yet extinct in the fifth
century.\textsuperscript{82} Tetricus and his son were reinstated in their rank
and fortunes. They erected on the Cælian hill a magnificent
palace, and as soon as it was finished, invited Aurelian to
supper. On his entrance, he was agreeably surprised with a
picture which represented their singular history. They were
delineated offering to the emperor a civic crown and the sceptre
of Gaul, and again receiving at his hands the ornaments of the
senatorial dignity. The father was afterwards invested with the
government of Lucania,\textsuperscript{83} and Aurelian, who soon admitted the
abdicated monarch to his friendship and conversation, familiarly
asked him, Whether it were not more desirable to administer a
province of Italy, than to reign beyond the Alps. The son long
continued a respectable member of the senate; nor was there any
one of the Roman nobility more esteemed by Aurelian, as well as
by his successors.\textsuperscript{84}

\pagenote[82]{Vopiscus in Hist. August. p. 199. Hieronym. in
Chron. Prosper in Chron. Baronius supposes that Zenobius, bishop
of Florence in the time of St. Ambrose, was of her family.}

\pagenote[83]{Vopisc. in Hist. August. p. 222. Eutropius, ix. 13.
Victor Junior. But Pollio, in Hist. August. p. 196, says, that
Tetricus was made corrector of all Italy.}

\pagenote[84]{Hist. August. p. 197.}

So long and so various was the pomp of Aurelian’s triumph, that
although it opened with the dawn of day, the slow majesty of the
procession ascended not the Capitol before the ninth hour; and it
was already dark when the emperor returned to the palace. The
festival was protracted by theatrical representations, the games
of the circus, the hunting of wild beasts, combats of gladiators,
and naval engagements. Liberal donatives were distributed to the
army and people, and several institutions, agreeable or
beneficial to the city, contributed to perpetuate the glory of
Aurelian. A considerable portion of his oriental spoils was
consecrated to the gods of Rome; the Capitol, and every other
temple, glittered with the offerings of his ostentatious piety;
and the temple of the Sun alone received above fifteen thousand
pounds of gold.\textsuperscript{85} This last was a magnificent structure, erected
by the emperor on the side of the Quirinal hill, and dedicated,
soon after the triumph, to that deity whom Aurelian adored as the
parent of his life and fortunes. His mother had been an inferior
priestess in a chapel of the Sun; a peculiar devotion to the god
of Light was a sentiment which the fortunate peasant imbibed in
his infancy; and every step of his elevation, every victory of
his reign, fortified superstition by gratitude.\textsuperscript{86}

\pagenote[85]{Vopiscus in Hist. August. 222. Zosimus, l. i. p.
56. He placed in it the images of Belus and of the Sun, which he
had brought from Palmyra. It was dedicated in the fourth year of
his reign, (Euseb in Chron.,) but was most assuredly begun
immediately on his accession.}

\pagenote[86]{See, in the Augustan History, p. 210, the omens of
his fortune. His devotion to the Sun appears in his letters, on
his medals, and is mentioned in the Cæsars of Julian. Commentaire
de Spanheim, p. 109.}

The arms of Aurelian had vanquished the foreign and domestic foes
of the republic. We are assured, that, by his salutary rigor,
crimes and factions, mischievous arts and pernicious connivance,
the luxurious growth of a feeble and oppressive government, were
eradicated throughout the Roman world.\textsuperscript{87} But if we attentively
reflect how much swifter is the progress of corruption than its
cure, and if we remember that the years abandoned to public
disorders exceeded the months allotted to the martial reign of
Aurelian, we must confess that a few short intervals of peace
were insufficient for the arduous work of reformation. Even his
attempt to restore the integrity of the coin was opposed by a
formidable insurrection. The emperor’s vexation breaks out in one
of his private letters. “Surely,” says he, “the gods have decreed
that my life should be a perpetual warfare. A sedition within the
walls has just now given birth to a very serious civil war. The
workmen of the mint, at the instigation of Felicissimus, a slave
to whom I had intrusted an employment in the finances, have risen
in rebellion. They are at length suppressed; but seven thousand
of my soldiers have been slain in the contest, of those troops
whose ordinary station is in Dacia, and the camps along the
Danube.”\textsuperscript{88} Other writers, who confirm the same fact, add
likewise, that it happened soon after Aurelian’s triumph; that
the decisive engagement was fought on the Cælian hill; that the
workmen of the mint had adulterated the coin; and that the
emperor restored the public credit, by delivering out good money
in exchange for the bad, which the people was commanded to bring
into the treasury.\textsuperscript{89}

\pagenote[87]{Vopiscus in Hist. August. p. 221.}

\pagenote[88]{Hist. August. p. 222. Aurelian calls these soldiers
Hiberi Riporiences Castriani, and Dacisci.}

\pagenote[89]{Zosimus, l. i. p. 56. Eutropius, ix. 14. Aurel
Victor.}

We might content ourselves with relating this extraordinary
transaction, but we cannot dissemble how much in its present form
it appears to us inconsistent and incredible. The debasement of
the coin is indeed well suited to the administration of
Gallienus; nor is it unlikely that the instruments of the
corruption might dread the inflexible justice of Aurelian. But
the guilt, as well as the profit, must have been confined to a
very few; nor is it easy to conceive by what arts they could arm
a people whom they had injured, against a monarch whom they had
betrayed. We might naturally expect that such miscreants should
have shared the public detestation with the informers and the
other ministers of oppression; and that the reformation of the
coin should have been an action equally popular with the
destruction of those obsolete accounts, which by the emperor’s
order were burnt in the forum of Trajan.\textsuperscript{90} In an age when the
principles of commerce were so imperfectly understood, the most
desirable end might perhaps be effected by harsh and injudicious
means; but a temporary grievance of such a nature can scarcely
excite and support a serious civil war. The repetition of
intolerable taxes, imposed either on the land or on the
necessaries of life, may at last provoke those who will not, or
who cannot, relinquish their country. But the case is far
otherwise in every operation which, by whatsoever expedients,
restores the just value of money. The transient evil is soon
obliterated by the permanent benefit, the loss is divided among
multitudes; and if a few wealthy individuals experience a
sensible diminution of treasure, with their riches, they at the
same time lose the degree of weight and importance which they
derived from the possession of them. However Aurelian might
choose to disguise the real cause of the insurrection, his
reformation of the coin could furnish only a faint pretence to a
party already powerful and discontented. Rome, though deprived of
freedom, was distracted by faction. The people, towards whom the
emperor, himself a plebeian, always expressed a peculiar
fondness, lived in perpetual dissension with the senate, the
equestrian order, and the Prætorian guards.\textsuperscript{91} Nothing less than
the firm though secret conspiracy of those orders, of the
authority of the first, the wealth of the second, and the arms of
the third, could have displayed a strength capable of contending
in battle with the veteran legions of the Danube, which, under
the conduct of a martial sovereign, had achieved the conquest of
the West and of the East.

\pagenote[90]{Hist. August. p. 222. Aurel Victor.}

\pagenote[91]{It already raged before Aurelian’s return from
Egypt. See Vipiscus, who quotes an original letter. Hist. August.
p. 244.}

Whatever was the cause or the object of this rebellion, imputed
with so little probability to the workmen of the mint, Aurelian
used his victory with unrelenting rigor.\textsuperscript{92} He was naturally of a
severe disposition. A peasant and a soldier, his nerves yielded
not easily to the impressions of sympathy, and he could sustain
without emotion the sight of tortures and death. Trained from his
earliest youth in the exercise of arms, he set too small a value
on the life of a citizen, chastised by military execution the
slightest offences, and transferred the stern discipline of the
camp into the civil administration of the laws. His love of
justice often became a blind and furious passion; and whenever he
deemed his own or the public safety endangered, he disregarded
the rules of evidence, and the proportion of punishments. The
unprovoked rebellion with which the Romans rewarded his services,
exasperated his haughty spirit. The noblest families of the
capital were involved in the guilt or suspicion of this dark
conspiracy. A nasty spirit of revenge urged the bloody
prosecution, and it proved fatal to one of the nephews of the
emperor. The executioners (if we may use the expression of a
contemporary poet) were fatigued, the prisons were crowded, and
the unhappy senate lamented the death or absence of its most
illustrious members.\textsuperscript{93} Nor was the pride of Aurelian less
offensive to that assembly than his cruelty. Ignorant or
impatient of the restraints of civil institutions, he disdained
to hold his power by any other title than that of the sword, and
governed by right of conquest an empire which he had saved and
subdued.\textsuperscript{94}

\pagenote[92]{Vopiscus in Hist. August p. 222. The two Victors.
Eutropius ix. 14. Zosimus (l. i. p. 43) mentions only three
senators, and placed their death before the eastern war.}

\pagenote[93]{Nulla catenati feralis pompa senatus Carnificum
lassabit opus; nec carcere pleno Infelix raros numerabit curia
Patres. Calphurn. Eclog. i. 60.}

\pagenote[94]{According to the younger Victor, he sometimes wore
the diadem, Deus and Dominus appear on his medals.}

It was observed by one of the most sagacious of the Roman
princes, that the talents of his predecessor Aurelian were better
suited to the command of an army, than to the government of an
empire.\textsuperscript{95} Conscious of the character in which nature and
experience had enabled him to excel, he again took the field a
few months after his triumph. It was expedient to exercise the
restless temper of the legions in some foreign war, and the
Persian monarch, exulting in the shame of Valerian, still braved
with impunity the offended majesty of Rome. At the head of an
army, less formidable by its numbers than by its discipline and
valor, the emperor advanced as far as the Straits which divide
Europe from Asia. He there experienced that the most absolute
power is a weak defence against the effects of despair. He had
threatened one of his secretaries who was accused of extortion;
and it was known that he seldom threatened in vain. The last hope
which remained for the criminal was to involve some of the
principal officers of the army in his danger, or at least in his
fears. Artfully counterfeiting his master’s hand, he showed them,
in a long and bloody list, their own names devoted to death.
Without suspecting or examining the fraud, they resolved to
secure their lives by the murder of the emperor. On his march,
between Byzantium and Heraclea, Aurelian was suddenly attacked by
the conspirators, whose stations gave them a right to surround
his person, and after a short resistance, fell by the hand of
Mucapor, a general whom he had always loved and trusted. He died
regretted by the army, detested by the senate, but universally
acknowledged as a warlike and fortunate prince, the useful,
though severe reformer of a degenerate state.\textsuperscript{96}

\pagenote[95]{It was the observation of Dioclatian. See Vopiscus
in Hist. August. p. 224.}

\pagenote[96]{Vopiscus in Hist. August. p. 221. Zosimus, l. i. p.
57. Eutrop ix. 15. The two Victors.}

