\chapter{Reign Of Diocletian And His Three Associates.}
\section{Part \thesection.}

\textit{The Reign Of Diocletian And His Three Associates, Maximian,
Galerius, And Constantius. — General Reestablishment Of Order And
Tranquillity. — The Persian War, Victory, And Triumph. — The New Form
Of Administration. — Abdication And Retirement Of Diocletian And
Maximian.}
\vspace{\onelineskip}

As the reign of Diocletian was more illustrious than that of any
of his predecessors, so was his birth more abject and obscure.
The strong claims of merit and of violence had frequently
superseded the ideal prerogatives of nobility; but a distinct
line of separation was hitherto preserved between the free and
the servile part of mankind. The parents of Diocletian had been
slaves in the house of Anulinus, a Roman senator; nor was he
himself distinguished by any other name than that which he
derived from a small town in Dalmatia, from whence his mother
deduced her origin.\footnotemark[1] It is, however, probable that his father
obtained the freedom of the family, and that he soon acquired an
office of scribe, which was commonly exercised by persons of his
condition.\footnotemark[2] Favorable oracles, or rather the consciousness of
superior merit, prompted his aspiring son to pursue the
profession of arms and the hopes of fortune; and it would be
extremely curious to observe the gradation of arts and accidents
which enabled him in the end to fulfil those oracles, and to
display that merit to the world. Diocletian was successively
promoted to the government of Mæsia, the honors of the
consulship, and the important command of the guards of the
palace. He distinguished his abilities in the Persian war; and
after the death of Numerian, the slave, by the confession and
judgment of his rivals, was declared the most worthy of the
Imperial throne. The malice of religious zeal, whilst it arraigns
the savage fierceness of his colleague Maximian, has affected to
cast suspicions on the personal courage of the emperor
Diocletian.\footnotemark[3] It would not be easy to persuade us of the
cowardice of a soldier of fortune, who acquired and preserved the
esteem of the legions as well as the favor of so many warlike
princes. Yet even calumny is sagacious enough to discover and to
attack the most vulnerable part. The valor of Diocletian was
never found inadequate to his duty, or to the occasion; but he
appears not to have possessed the daring and generous spirit of a
hero, who courts danger and fame, disdains artifice, and boldly
challenges the allegiance of his equals. His abilities were
useful rather than splendid; a vigorous mind, improved by the
experience and study of mankind; dexterity and application in
business; a judicious mixture of liberality and economy, of
mildness and rigor; profound dissimulation, under the disguise of
military frankness; steadiness to pursue his ends; flexibility to
vary his means; and, above all, the great art of submitting his
own passions, as well as those of others, to the interest of his
ambition, and of coloring his ambition with the most specious
pretences of justice and public utility. Like Augustus,
Diocletian may be considered as the founder of a new empire. Like
the adopted son of Cæsar, he was distinguished as a statesman
rather than as a warrior; nor did either of those princes employ
force, whenever their purpose could be effected by policy.

\footnotetext[1]{Eutrop. ix. 19. Victor in Epitome. The town seems to
have been properly called Doclia, from a small tribe of
Illyrians, (see Cellarius, Geograph. Antiqua, tom. i. p. 393;)
and the original name of the fortunate slave was probably Docles;
he first lengthened it to the Grecian harmony of Diocles, and at
length to the Roman majesty of Diocletianus. He likewise assumed
the Patrician name of Valerius and it is usually given him by
Aurelius Victor.}

\footnotetext[2]{See Dacier on the sixth satire of the second book of
Horace Cornel. Nepos, ’n Vit. Eumen. c. l.}

\footnotetext[3]{Lactantius (or whoever was the author of the little
treatise De Mortibus Persecutorum) accuses Diocletian of timidity
in two places, c. 7. 8. In chap. 9 he says of him, “erat in omni
tumultu meticulosu et animi disjectus.”}

The victory of Diocletian was remarkable for its singular
mildness. A people accustomed to applaud the clemency of the
conqueror, if the usual punishments of death, exile, and
confiscation, were inflicted with any degree of temper and
equity, beheld, with the most pleasing astonishment, a civil war,
the flames of which were extinguished in the field of battle.
Diocletian received into his confidence Aristobulus, the
principal minister of the house of Carus, respected the lives,
the fortunes, and the dignity, of his adversaries, and even
continued in their respective stations the greater number of the
servants of Carinus.\footnotemark[4] It is not improbable that motives of
prudence might assist the humanity of the artful Dalmatian; of
these servants, many had purchased his favor by secret treachery;
in others, he esteemed their grateful fidelity to an unfortunate
master. The discerning judgment of Aurelian, of Probus, and of
Carus, had filled the several departments of the state and army
with officers of approved merit, whose removal would have injured
the public service, without promoting the interest of his
successor. Such a conduct, however, displayed to the Roman world
the fairest prospect of the new reign, and the emperor affected
to confirm this favorable prepossession, by declaring, that,
among all the virtues of his predecessors, he was the most
ambitious of imitating the humane philosophy of Marcus Antoninus.\footnotemark[5]

\footnotetext[4]{In this encomium, Aurelius Victor seems to convey a
just, though indirect, censure of the cruelty of Constantius. It
appears from the Fasti, that Aristobulus remained præfect of the
city, and that he ended with Diocletian the consulship which he
had commenced with Carinus.}

\footnotetext[5]{Aurelius Victor styles Diocletian, “Parentum potius
quam Dominum.” See Hist. August. p. 30.}

The first considerable action of his reign seemed to evince his
sincerity as well as his moderation. After the example of Marcus,
he gave himself a colleague in the person of Maximian, on whom he
bestowed at first the title of Cæsar, and afterwards that of
Augustus.\footnotemark[6] But the motives of his conduct, as well as the object
of his choice, were of a very different nature from those of his
admired predecessor. By investing a luxurious youth with the
honors of the purple, Marcus had discharged a debt of private
gratitude, at the expense, indeed, of the happiness of the state.
By associating a friend and a fellow-soldier to the labors of
government, Diocletian, in a time of public danger, provided for
the defence both of the East and of the West. Maximian was born a
peasant, and, like Aurelian, in the territory of Sirmium.
Ignorant of letters,\footnotemark[7] careless of laws, the rusticity of his
appearance and manners still betrayed in the most elevated
fortune the meanness of his extraction. War was the only art
which he professed. In a long course of service he had
distinguished himself on every frontier of the empire; and though
his military talents were formed to obey rather than to command,
though, perhaps, he never attained the skill of a consummate
general, he was capable, by his valor, constancy, and experience,
of executing the most arduous undertakings. Nor were the vices of
Maximian less useful to his benefactor. Insensible to pity, and
fearless of consequences, he was the ready instrument of every
act of cruelty which the policy of that artful prince might at
once suggest and disclaim. As soon as a bloody sacrifice had been
offered to prudence or to revenge, Diocletian, by his seasonable
intercession, saved the remaining few whom he had never designed
to punish, gently censured the severity of his stern colleague,
and enjoyed the comparison of a golden and an iron age, which was
universally applied to their opposite maxims of government.
Notwithstanding the difference of their characters, the two
emperors maintained, on the throne, that friendship which they
had contracted in a private station. The haughty, turbulent
spirit of Maximian, so fatal, afterwards, to himself and to the
public peace, was accustomed to respect the genius of Diocletian,
and confessed the ascendant of reason over brutal violence.\footnotemark[8]
From a motive either of pride or superstition, the two emperors
assumed the titles, the one of Jovius, the other of Herculius.
Whilst the motion of the world (such was the language of their
venal orators) was maintained by the all-seeing wisdom of
Jupiter, the invincible arm of Hercules purged the earth from
monsters and tyrants.\footnotemark[9]

\footnotetext[6]{The question of the time when Maximian received the
honors of Cæsar and Augustus has divided modern critics, and
given occasion to a great deal of learned wrangling. I have
followed M. de Tillemont, (Histoire des Empereurs, tom. iv. p.
500-505,) who has weighed the several reasons and difficulties
with his scrupulous accuracy. * Note: Eckbel concurs in this
view, viii p. 15.—M.}

\footnotetext[7]{In an oration delivered before him, (Panegyr. Vet.
ii. 8,) Mamertinus expresses a doubt, whether his hero, in
imitating the conduct of Hannibal and Scipio, had ever heard of
their names. From thence we may fairly infer, that Maximian was
more desirous of being considered as a soldier than as a man of
letters; and it is in this manner that we can often translate the
language of flattery into that of truth.}

\footnotetext[8]{Lactantius de M. P. c. 8. Aurelius Victor. As among
the Panegyrics, we find orations pronounced in praise of
Maximian, and others which flatter his adversaries at his
expense, we derive some knowledge from the contrast.}

\footnotetext[9]{See the second and third Panegyrics, particularly
iii. 3, 10, 14 but it would be tedious to copy the diffuse and
affected expressions of their false eloquence. With regard to the
titles, consult Aurel. Victor Lactantius de M. P. c. 52. Spanheim
de Usu Numismatum, \&c. xii 8.}

But even the omnipotence of Jovius and Herculius was insufficient
to sustain the weight of the public administration. The prudence
of Diocletian discovered that the empire, assailed on every side
by the barbarians, required on every side the presence of a great
army, and of an emperor. With this view, he resolved once more to
divide his unwieldy power, and with the inferior title of
\textit{Cæsars},\footnotemark[901] to confer on two generals of approved merit an
unequal share of the sovereign authority.\footnotemark[10] Galerius, surnamed
Armentarius, from his original profession of a herdsman, and
Constantius, who from his pale complexion had acquired the
denomination of Chlorus,\footnotemark[11] were the two persons invested with
the second honors of the Imperial purple. In describing the
country, extraction, and manners of Herculius, we have already
delineated those of Galerius, who was often, and not improperly,
styled the younger Maximian, though, in many instances both of
virtue and ability, he appears to have possessed a manifest
superiority over the elder. The birth of Constantius was less
obscure than that of his colleagues. Eutropius, his father, was
one of the most considerable nobles of Dardania, and his mother
was the niece of the emperor Claudius.\footnotemark[12] Although the youth of
Constantius had been spent in arms, he was endowed with a mild
and amiable disposition, and the popular voice had long since
acknowledged him worthy of the rank which he at last attained. To
strengthen the bonds of political, by those of domestic, union,
each of the emperors assumed the character of a father to one of
the Cæsars, Diocletian to Galerius, and Maximian to Constantius;
and each, obliging them to repudiate their former wives, bestowed
his daughter in marriage or his adopted son.\footnotemark[13] These four
princes distributed among themselves the wide extent of the Roman
empire. The defence of Gaul, Spain,\footnotemark[14] and Britain, was intrusted
to Constantius: Galerius was stationed on the banks of the
Danube, as the safeguard of the Illyrian provinces. Italy and
Africa were considered as the department of Maximian; and for his
peculiar portion, Diocletian reserved Thrace, Egypt, and the rich
countries of Asia. Every one was sovereign with his own
jurisdiction; but their united authority extended over the whole
monarchy, and each of them was prepared to assist his colleagues
with his counsels or presence. The Cæsars, in their exalted rank,
revered the majesty of the emperors, and the three younger
princes invariably acknowledged, by their gratitude and
obedience, the common parent of their fortunes. The suspicious
jealousy of power found not any place among them; and the
singular happiness of their union has been compared to a chorus
of music, whose harmony was regulated and maintained by the
skilful hand of the first artist.\footnotemark[15]

\footnotetext[901]{On the relative power of the Augusti and the
Cæsars, consult a dissertation at the end of Manso’s Leben
Constantius des Grossen—M.}

\footnotetext[10]{Aurelius Victor. Victor in Epitome. Eutrop. ix. 22.
Lactant de M. P. c. 8. Hieronym. in Chron.}

\footnotetext[11]{It is only among the modern Greeks that Tillemont
can discover his appellation of Chlorus. Any remarkable degree of
paleness seems inconsistent with the rubor mentioned in
Panegyric, v. 19.}

\footnotetext[12]{Julian, the grandson of Constantius, boasts that
his family was derived from the warlike Mæsians. Misopogon, p.
348. The Dardanians dwelt on the edge of Mæsia.}

\footnotetext[13]{Galerius married Valeria, the daughter of
Diocletian; if we speak with strictness, Theodora, the wife of
Constantius, was daughter only to the wife of Maximian. Spanheim,
Dissertat, xi. 2.}

\footnotetext[14]{This division agrees with that of the four
præfectures; yet there is some reason to doubt whether Spain was
not a province of Maximian. See Tillemont, tom. iv. p. 517. *
Note: According to Aurelius Victor and other authorities, Thrace
belonged to the division of Galerius. See Tillemont, iv. 36. But
the laws of Diocletian are in general dated in Illyria or
Thrace.—M.}

\footnotetext[15]{Julian in Cæsarib. p. 315. Spanheim’s notes to the
French translation, p. 122.}

This important measure was not carried into execution till about
six years after the association of Maximian, and that interval of
time had not been destitute of memorable incidents. But we have
preferred, for the sake of perspicuity, first to describe the
more perfect form of Diocletian’s government, and afterwards to
relate the actions of his reign, following rather the natural
order of the events, than the dates of a very doubtful
chronology.

The first exploit of Maximian, though it is mentioned in a few
words by our imperfect writers, deserves, from its singularity,
to be recorded in a history of human manners. He suppressed the
peasants of Gaul, who, under the appellation of Bagaudæ,\footnotemark[16] had
risen in a general insurrection; very similar to those which in
the fourteenth century successively afflicted both France and
England.\footnotemark[17] It should seem that very many of those institutions,
referred by an easy solution to the feudal system, are derived
from the Celtic barbarians. When Cæsar subdued the Gauls, that
great nation was already divided into three orders of men; the
clergy, the nobility, and the common people. The first governed
by superstition, the second by arms, but the third and last was
not of any weight or account in their public councils. It was
very natural for the plebeians, oppressed by debt, or
apprehensive of injuries, to implore the protection of some
powerful chief, who acquired over their persons and property the
same absolute right as, among the Greeks and Romans, a master
exercised over his slaves.\footnotemark[18] The greatest part of the nation was
gradually reduced into a state of servitude; compelled to
perpetual labor on the estates of the Gallic nobles, and confined
to the soil, either by the real weight of fetters, or by the no
less cruel and forcible restraints of the laws. During the long
series of troubles which agitated Gaul, from the reign of
Gallienus to that of Diocletian, the condition of these servile
peasants was peculiarly miserable; and they experienced at once
the complicated tyranny of their masters, of the barbarians, of
the soldiers, and of the officers of the revenue.\footnotemark[19]

\footnotetext[16]{The general name of Bagaudæ (in the signification
of rebels) continued till the fifth century in Gaul. Some critics
derive it from a Celtic word Bagad, a tumultuous assembly.
Scaliger ad Euseb. Du Cange Glossar. (Compare S. Turner,
Anglo-Sax. History, i. 214.—M.)}

\footnotetext[17]{Chronique de Froissart, vol. i. c. 182, ii. 73, 79.
The naivete of his story is lost in our best modern writers.}

\footnotetext[18]{Cæsar de Bell. Gallic. vi. 13. Orgetorix, the
Helvetian, could arm for his defence a body of ten thousand
slaves.}

\footnotetext[19]{Their oppression and misery are acknowledged by
Eumenius (Panegyr. vi. 8,) Gallias efferatas injuriis.}

Their patience was at last provoked into despair. On every side
they rose in multitudes, armed with rustic weapons, and with
irresistible fury. The ploughman became a foot soldier, the
shepherd mounted on horseback, the deserted villages and open
towns were abandoned to the flames, and the ravages of the
peasants equalled those of the fiercest barbarians.\footnotemark[20] They
asserted the natural rights of men, but they asserted those
rights with the most savage cruelty. The Gallic nobles, justly
dreading their revenge, either took refuge in the fortified
cities, or fled from the wild scene of anarchy. The peasants
reigned without control; and two of their most daring leaders had
the folly and rashness to assume the Imperial ornaments.\footnotemark[21] Their
power soon expired at the approach of the legions. The strength
of union and discipline obtained an easy victory over a
licentious and divided multitude.\footnotemark[22] A severe retaliation was
inflicted on the peasants who were found in arms; the affrighted
remnant returned to their respective habitations, and their
unsuccessful effort for freedom served only to confirm their
slavery. So strong and uniform is the current of popular
passions, that we might almost venture, from very scanty
materials, to relate the particulars of this war; but we are not
disposed to believe that the principal leaders, Ælianus and
Amandus, were Christians,\footnotemark[23] or to insinuate, that the rebellion,
as it happened in the time of Luther, was occasioned by the abuse
of those benevolent principles of Christianity, which inculcate
the natural freedom of mankind.

\footnotetext[20]{Panegyr. Vet. ii. 4. Aurelius Victor.}

\footnotetext[21]{Ælianus and Amandus. We have medals coined by them
Goltzius in Thes. R. A. p. 117, 121.}

\footnotetext[22]{Levibus proeliis domuit. Eutrop. ix. 20.}

\footnotetext[23]{The fact rests indeed on very slight authority, a
life of St. Babolinus, which is probably of the seventh century.
See Duchesne Scriptores Rer. Francicar. tom. i. p. 662.}

Maximian had no sooner recovered Gaul from the hands of the
peasants, than he lost Britain by the usurpation of Carausius.
Ever since the rash but successful enterprise of the Franks under
the reign of Probus, their daring countrymen had constructed
squadrons of light brigantines, in which they incessantly ravaged
the provinces adjacent to the ocean.\footnotemark[24] To repel their desultory
incursions, it was found necessary to create a naval power; and
the judicious measure was prosecuted with prudence and vigor.
Gessoriacum, or Boulogne, in the straits of the British Channel,
was chosen by the emperor for the station of the Roman fleet; and
the command of it was intrusted to Carausius, a Menapian of the
meanest origin,\footnotemark[25] but who had long signalized his skill as a
pilot, and his valor as a soldier. The integrity of the new
admiral corresponded not with his abilities. When the German
pirates sailed from their own harbors, he connived at their
passage, but he diligently intercepted their return, and
appropriated to his own use an ample share of the spoil which
they had acquired. The wealth of Carausius was, on this occasion,
very justly considered as an evidence of his guilt; and Maximian
had already given orders for his death. But the crafty Menapian
foresaw and prevented the severity of the emperor. By his
liberality he had attached to his fortunes the fleet which he
commanded, and secured the barbarians in his interest. From the
port of Boulogne he sailed over to Britain, persuaded the legion,
and the auxiliaries which guarded that island, to embrace his
party, and boldly assuming, with the Imperial purple, the title
of Augustus, defied the justice and the arms of his injured
sovereign.\footnotemark[26]

\footnotetext[24]{Aurelius Victor calls them Germans. Eutropius (ix.
21) gives them the name of Saxons. But Eutropius lived in the
ensuing century, and seems to use the language of his own times.}

\footnotetext[25]{The three expressions of Eutropius, Aurelius
Victor, and Eumenius, “vilissime natus,” “Bataviæ alumnus,” and
“Menapiæ civis,” give us a very doubtful account of the birth of
Carausius. Dr. Stukely, however, (Hist. of Carausius, p. 62,)
chooses to make him a native of St. David’s and a prince of the
blood royal of Britain. The former idea he had found in Richard
of Cirencester, p. 44. * Note: The Menapians were settled between
the Scheldt and the Meuse, is the northern part of Brabant.
D’Anville, Geogr. Anc. i. 93.—G.}

\footnotetext[26]{Panegyr. v. 12. Britain at this time was secure,
and slightly guarded.}

When Britain was thus dismembered from the empire, its importance
was sensibly felt, and its loss sincerely lamented. The Romans
celebrated, and perhaps magnified, the extent of that noble
island, provided on every side with convenient harbors; the
temperature of the climate, and the fertility of the soil, alike
adapted for the production of corn or of vines; the valuable
minerals with which it abounded; its rich pastures covered with
innumerable flocks, and its woods free from wild beasts or
venomous serpents. Above all, they regretted the large amount of
the revenue of Britain, whilst they confessed, that such a
province well deserved to become the seat of an independent
monarchy.\footnotemark[27] During the space of seven years it was possessed by
Carausius; and fortune continued propitious to a rebellion
supported with courage and ability. The British emperor defended
the frontiers of his dominions against the Caledonians of the
North, invited, from the continent, a great number of skilful
artists, and displayed, on a variety of coins that are still
extant, his taste and opulence. Born on the confines of the
Franks, he courted the friendship of that formidable people, by
the flattering imitation of their dress and manners. The bravest
of their youth he enlisted among his land or sea forces; and, in
return for their useful alliance, he communicated to the
barbarians the dangerous knowledge of military and naval arts.
Carausius still preserved the possession of Boulogne and the
adjacent country. His fleets rode triumphant in the channel,
commanded the mouths of the Seine and of the Rhine, ravaged the
coasts of the ocean, and diffused beyond the columns of Hercules
the terror of his name. Under his command, Britain, destined in a
future age to obtain the empire of the sea, already assumed its
natural and respectable station of a maritime power.\footnotemark[28]

\footnotetext[27]{Panegyr. Vet v 11, vii. 9. The orator Eumenius
wished to exalt the glory of the hero (Constantius) with the
importance of the conquest. Notwithstanding our laudable
partiality for our native country, it is difficult to conceive,
that, in the beginning of the fourth century England deserved all
these commendations. A century and a half before, it hardly paid
its own establishment.}

\footnotetext[28]{As a great number of medals of Carausius are still
preserved, he is become a very favorite object of antiquarian
curiosity, and every circumstance of his life and actions has
been investigated with sagacious accuracy. Dr. Stukely, in
particular, has devoted a large volume to the British emperor. I
have used his materials, and rejected most of his fanciful
conjectures.}

By seizing the fleet of Boulogne, Carausius had deprived his
master of the means of pursuit and revenge. And when, after a
vast expense of time and labor, a new armament was launched into
the water,\footnotemark[29] the Imperial troops, unaccustomed to that element,
were easily baffled and defeated by the veteran sailors of the
usurper. This disappointed effort was soon productive of a treaty
of peace. Diocletian and his colleague, who justly dreaded the
enterprising spirit of Carausius, resigned to him the sovereignty
of Britain, and reluctantly admitted their perfidious servant to
a participation of the Imperial honors.\footnotemark[30] But the adoption of
the two Cæsars restored new vigor to the Romans arms; and while
the Rhine was guarded by the presence of Maximian, his brave
associate Constantius assumed the conduct of the British war. His
first enterprise was against the important place of Boulogne. A
stupendous mole, raised across the entrance of the harbor,
intercepted all hopes of relief. The town surrendered after an
obstinate defence; and a considerable part of the naval strength
of Carausius fell into the hands of the besiegers. During the
three years which Constantius employed in preparing a fleet
adequate to the conquest of Britain, he secured the coast of
Gaul, invaded the country of the Franks, and deprived the usurper
of the assistance of those powerful allies.

\footnotetext[29]{When Mamertinus pronounced his first panegyric, the
naval preparations of Maximian were completed; and the orator
presaged an assured victory. His silence in the second panegyric
might alone inform us that the expedition had not succeeded.}

\footnotetext[30]{Aurelius Victor, Eutropius, and the medals, (Pax
Augg.) inform us of this temporary reconciliation; though I will
not presume (as Dr. Stukely has done, Medallic History of
Carausius, p. 86, \&c) to insert the identical articles of the
treaty.}

Before the preparations were finished, Constantius received the
intelligence of the tyrant’s death, and it was considered as a
sure presage of the approaching victory. The servants of
Carausius imitated the example of treason which he had given. He
was murdered by his first minister, Allectus, and the assassin
succeeded to his power and to his danger. But he possessed not
equal abilities either to exercise the one or to repel the other.

He beheld, with anxious terror, the opposite shores of the
continent already filled with arms, with troops, and with
vessels; for Constantius had very prudently divided his forces,
that he might likewise divide the attention and resistance of the
enemy. The attack was at length made by the principal squadron,
which, under the command of the præfect Asclepiodatus, an officer
of distinguished merit, had been assembled in the north of the
Seine. So imperfect in those times was the art of navigation,
that orators have celebrated the daring courage of the Romans,
who ventured to set sail with a side-wind, and on a stormy day.
The weather proved favorable to their enterprise. Under the cover
of a thick fog, they escaped the fleet of Allectus, which had
been stationed off the Isle of Wight to receive them, landed in
safety on some part of the western coast, and convinced the
Britons, that a superiority of naval strength will not always
protect their country from a foreign invasion. Asclepiodatus had
no sooner disembarked the imperial troops, then he set fire to
his ships; and, as the expedition proved fortunate, his heroic
conduct was universally admired. The usurper had posted himself
near London, to expect the formidable attack of Constantius, who
commanded in person the fleet of Boulogne; but the descent of a
new enemy required his immediate presence in the West. He
performed this long march in so precipitate a manner, that he
encountered the whole force of the præfect with a small body of
harassed and disheartened troops. The engagement was soon
terminated by the total defeat and death of Allectus; a single
battle, as it has often happened, decided the fate of this great
island; and when Constantius landed on the shores of Kent, he
found them covered with obedient subjects. Their acclamations
were loud and unanimous; and the virtues of the conqueror may
induce us to believe, that they sincerely rejoiced in a
revolution, which, after a separation of ten years, restored
Britain to the body of the Roman empire.\footnotemark[31]

\footnotetext[31]{With regard to the recovery of Britain, we obtain a
few hints from Aurelius Victor and Eutropius.}

