\section{Part \thesection.}
\thispagestyle{simple}

The tender respect of Augustus for a free constitution which he
had destroyed, can only be explained by an attentive
consideration of the character of that subtle tyrant. A cool
head, an unfeeling heart, and a cowardly disposition, prompted
him at the age of nineteen to assume the mask of hypocrisy, which
he never afterwards laid aside. With the same hand, and probably
with the same temper, he signed the proscription of Cicero, and
the pardon of Cinna. His virtues, and even his vices, were
artificial; and according to the various dictates of his
interest, he was at first the enemy, and at last the father, of
the Roman world.\footnotemark[26] When he framed the artful system of the
Imperial authority, his moderation was inspired by his fears. He
wished to deceive the people by an image of civil liberty, and
the armies by an image of civil government.

\footnotetext[26]{As Octavianus advanced to the banquet of the
Cæsars, his color changed like that of the chameleon; pale at
first, then red, afterwards black, he at last assumed the mild
livery of Venus and the Graces, (Cæsars, p. 309.) This image,
employed by Julian in his ingenious fiction, is just and elegant;
but when he considers this change of character as real and
ascribes it to the power of philosophy, he does too much honor to
philosophy and to Octavianus.}

I. The death of Cæsar was ever before his eyes. He had lavished
wealth and honors on his adherents; but the most favored friends
of his uncle were in the number of the conspirators. The fidelity
of the legions might defend his authority against open rebellion;
but their vigilance could not secure his person from the dagger
of a determined republican; and the Romans, who revered the
memory of Brutus,\footnotemark[27] would applaud the imitation of his virtue.
Cæsar had provoked his fate, as much as by the ostentation of his
power, as by his power itself. The consul or the tribune might
have reigned in peace. The title of king had armed the Romans
against his life. Augustus was sensible that mankind is governed
by names; nor was he deceived in his expectation, that the senate
and people would submit to slavery, provided they were
respectfully assured that they still enjoyed their ancient
freedom. A feeble senate and enervated people cheerfully
acquiesced in the pleasing illusion, as long as it was supported
by the virtue, or even by the prudence, of the successors of
Augustus. It was a motive of self-preservation, not a principle
of liberty, that animated the conspirators against Caligula,
Nero, and Domitian. They attacked the person of the tyrant,
without aiming their blow at the authority of the emperor.

\footnotetext[27]{Two centuries after the establishment of monarchy,
the emperor Marcus Antoninus recommends the character of Brutus
as a perfect model of Roman virtue. * Note: In a very ingenious
essay, Gibbon has ventured to call in question the preeminent
virtue of Brutus. Misc Works, iv. 95.—M.}

There appears, indeed, \textit{one} memorable occasion, in which the
senate, after seventy years of patience, made an ineffectual
attempt to re-assume its long-forgotten rights. When the throne
was vacant by the murder of Caligula, the consuls convoked that
assembly in the Capitol, condemned the memory of the Cæsars, gave
the watchword \textit{liberty} to the few cohorts who faintly adhered to
their standard, and during eight-and-forty hours acted as the
independent chiefs of a free commonwealth. But while they
deliberated, the prætorian guards had resolved. The stupid
Claudius, brother of Germanicus, was already in their camp,
invested with the Imperial purple, and prepared to support his
election by arms. The dream of liberty was at an end; and the
senate awoke to all the horrors of inevitable servitude. Deserted
by the people, and threatened by a military force, that feeble
assembly was compelled to ratify the choice of the prætorians,
and to embrace the benefit of an amnesty, which Claudius had the
prudence to offer, and the generosity to observe.\footnotemark[28]

\footnotetext[28]{It is much to be regretted that we have lost the
part of Tacitus which treated of that transaction. We are forced
to content ourselves with the popular rumors of Josephus, and the
imperfect hints of Dion and Suetonius.}

II. The insolence of the armies inspired Augustus with fears of a
still more alarming nature. The despair of the citizens could
only attempt, what the power of the soldiers was, at any time,
able to execute. How precarious was his own authority over men
whom he had taught to violate every social duty! He had heard
their seditious clamors; he dreaded their calmer moments of
reflection. One revolution had been purchased by immense rewards;
but a second revolution might double those rewards. The troops
professed the fondest attachment to the house of Cæsar; but the
attachments of the multitude are capricious and inconstant.
Augustus summoned to his aid whatever remained in those fierce
minds of Roman prejudices; enforced the rigor of discipline by
the sanction of law; and, interposing the majesty of the senate
between the emperor and the army, boldly claimed their
allegiance, as the first magistrate of the republic.

During a long period of two hundred and twenty years from the
establishment of this artful system to the death of Commodus, the
dangers inherent to a military government were, in a great
measure, suspended. The soldiers were seldom roused to that fatal
sense of their own strength, and of the weakness of the civil
authority, which was, before and afterwards, productive of such
dreadful calamities. Caligula and Domitian were assassinated in
their palace by their own domestics:\footnotemark[281] the convulsions which
agitated Rome on the death of the former, were confined to the
walls of the city. But Nero involved the whole empire in his
ruin. In the space of eighteen months, four princes perished by
the sword; and the Roman world was shaken by the fury of the
contending armies. Excepting only this short, though violent
eruption of military license, the two centuries from Augustus\footnotemark[29]
to Commodus passed away unstained with civil blood, and
undisturbed by revolutions. The emperor was elected by \textit{the
authority of the senate}, and \textit{the consent of the soldiers}.\footnotemark[30]
The legions respected their oath of fidelity; and it requires a
minute inspection of the Roman annals to discover three
inconsiderable rebellions, which were all suppressed in a few
months, and without even the hazard of a battle.\footnotemark[31]

\footnotetext[281]{Caligula perished by a conspiracy formed by the
officers of the prætorian troops, and Domitian would not,
perhaps, have been assassinated without the participation of the
two chiefs of that guard in his death.—W.}

\footnotetext[29]{Augustus restored the ancient severity of
discipline. After the civil wars, he dropped the endearing name
of Fellow-Soldiers, and called them only Soldiers, (Sueton. in
August. c. 25.) See the use Tiberius made of the Senate in the
mutiny of the Pannonian legions, (Tacit. Annal. i.)}

\footnotetext[30]{These words seem to have been the constitutional
language. See Tacit. Annal. xiii. 4. * Note: This panegyric on
the soldiery is rather too liberal. Claudius was obliged to
purchase their consent to his coronation: the presents which he
made, and those which the prætorians received on other occasions,
considerably embarrassed the finances. Moreover, this formidable
guard favored, in general, the cruelties of the tyrants. The
distant revolts were more frequent than Gibbon thinks: already,
under Tiberius, the legions of Germany would have seditiously
constrained Germanicus to assume the Imperial purple. On the
revolt of Claudius Civilis, under Vespasian, the legions of Gaul
murdered their general, and offered their assistance to the Gauls
who were in insurrection. Julius Sabinus made himself be
proclaimed emperor, \&c. The wars, the merit, and the severe
discipline of Trajan, Hadrian, and the two Antonines,
established, for some time, a greater degree of subordination.—W}

\footnotetext[31]{The first was Camillus Scribonianus, who took up
arms in Dalmatia against Claudius, and was deserted by his own
troops in five days, the second, L. Antonius, in Germany, who
rebelled against Domitian; and the third, Avidius Cassius, in the
reign of M. Antoninus. The two last reigned but a few months, and
were cut off by their own adherents. We may observe, that both
Camillus and Cassius colored their ambition with the design of
restoring the republic; a task, said Cassius peculiarly reserved
for his name and family.}

In elective monarchies, the vacancy of the throne is a moment big
with danger and mischief. The Roman emperors, desirous to spare
the legions that interval of suspense, and the temptation of an
irregular choice, invested their designed successor with so large
a share of present power, as should enable him, after their
decease, to assume the remainder, without suffering the empire to
perceive the change of masters. Thus Augustus, after all his
fairer prospects had been snatched from him by untimely deaths,
rested his last hopes on Tiberius, obtained for his adopted son
the censorial and tribunitian powers, and dictated a law, by
which the future prince was invested with an authority equal to
his own, over the provinces and the armies.\footnotemark[32] Thus Vespasian
subdued the generous mind of his eldest son. Titus was adored by
the eastern legions, which, under his command, had recently
achieved the conquest of Judæa. His power was dreaded, and, as
his virtues were clouded by the intemperance of youth, his
designs were suspected. Instead of listening to such unworthy
suspicions, the prudent monarch associated Titus to the full
powers of the Imperial dignity; and the grateful son ever
approved himself the humble and faithful minister of so indulgent
a father.\footnotemark[33]

\footnotetext[32]{Velleius Paterculus, l. ii. c. 121. Sueton. in
Tiber. c. 26.}

\footnotetext[33]{Sueton. in Tit. c. 6. Plin. in Præfat. Hist.
Natur.}

The good sense of Vespasian engaged him indeed to embrace every
measure that might confirm his recent and precarious elevation.
The military oath, and the fidelity of the troops, had been
consecrated, by the habits of a hundred years, to the name and
family of the Cæsars; and although that family had been continued
only by the fictitious rite of adoption, the Romans still
revered, in the person of Nero, the grandson of Germanicus, and
the lineal successor of Augustus. It was not without reluctance
and remorse, that the prætorian guards had been persuaded to
abandon the cause of the tyrant.\footnotemark[34] The rapid downfall of Galba,
Otho, and Vitellus, taught the armies to consider the emperors as
the creatures of \textit{their} will, and the instruments of \textit{their}
license. The birth of Vespasian was mean: his grandfather had
been a private soldier, his father a petty officer of the
revenue;\footnotemark[35] his own merit had raised him, in an advanced age, to
the empire; but his merit was rather useful than shining, and his
virtues were disgraced by a strict and even sordid parsimony.
Such a prince consulted his true interest by the association of a
son, whose more splendid and amiable character might turn the
public attention from the obscure origin, to the future glories,
of the Flavian house. Under the mild administration of Titus, the
Roman world enjoyed a transient felicity, and his beloved memory
served to protect, above fifteen years, the vices of his brother
Domitian.

\footnotetext[34]{This idea is frequently and strongly inculcated by
Tacitus. See Hist. i. 5, 16, ii. 76.}

\footnotetext[35]{The emperor Vespasian, with his usual good sense,
laughed at the genealogists, who deduced his family from Flavius,
the founder of Reate, (his native country,) and one of the
companions of Hercules Suet in Vespasian, c. 12.}

Nerva had scarcely accepted the purple from the assassins of
Domitian, before he discovered that his feeble age was unable to
stem the torrent of public disorders, which had multiplied under
the long tyranny of his predecessor. His mild disposition was
respected by the good; but the degenerate Romans required a more
vigorous character, whose justice should strike terror into the
guilty. Though he had several relations, he fixed his choice on a
stranger. He adopted Trajan, then about forty years of age, and
who commanded a powerful army in the Lower Germany; and
immediately, by a decree of the senate, declared him his
colleague and successor in the empire.\footnotemark[36] It is sincerely to be
lamented, that whilst we are fatigued with the disgustful
relation of Nero’s crimes and follies, we are reduced to collect
the actions of Trajan from the glimmerings of an abridgment, or
the doubtful light of a panegyric. There remains, however, one
panegyric far removed beyond the suspicion of flattery. Above two
hundred and fifty years after the death of Trajan, the senate, in
pouring out the customary acclamations on the accession of a new
emperor, wished that he might surpass the felicity of Augustus,
and the virtue of Trajan.\footnotemark[37]

\footnotetext[36]{Dion, l. lxviii. p. 1121. Plin. Secund. in
Panegyric.}

\footnotetext[37]{Felicior Augusto, Melior Trajano. Eutrop. viii. 5.}

We may readily believe, that the father of his country hesitated
whether he ought to intrust the various and doubtful character of
his kinsman Hadrian with sovereign power. In his last moments the
arts of the empress Plotina either fixed the irresolution of
Trajan, or boldly supposed a fictitious adoption;\footnotemark[38] the truth of
which could not be safely disputed, and Hadrian was peaceably
acknowledged as his lawful successor. Under his reign, as has
been already mentioned, the empire flourished in peace and
prosperity. He encouraged the arts, reformed the laws, asserted
military discipline, and visited all his provinces in person. His
vast and active genius was equally suited to the most enlarged
views, and the minute details of civil policy. But the ruling
passions of his soul were curiosity and vanity. As they
prevailed, and as they were attracted by different objects,
Hadrian was, by turns, an excellent prince, a ridiculous sophist,
and a jealous tyrant. The general tenor of his conduct deserved
praise for its equity and moderation. Yet in the first days of
his reign, he put to death four consular senators, his personal
enemies, and men who had been judged worthy of empire; and the
tediousness of a painful illness rendered him, at last, peevish
and cruel. The senate doubted whether they should pronounce him a
god or a tyrant; and the honors decreed to his memory were
granted to the prayers of the pious Antoninus.\footnotemark[39]

\footnotetext[38]{Dion (l. lxix. p. 1249) affirms the whole to have
been a fiction, on the authority of his father, who, being
governor of the province where Trajan died, had very good
opportunities of sifting this mysterious transaction. Yet Dodwell
(Prælect. Camden. xvii.) has maintained that Hadrian was called
to the certain hope of the empire, during the lifetime of
Trajan.}

\footnotetext[39]{Dion, (l. lxx. p. 1171.) Aurel. Victor.}

The caprice of Hadrian influenced his choice of a successor.

After revolving in his mind several men of distinguished merit,
whom he esteemed and hated, he adopted Ælius Verus a gay and
voluptuous nobleman, recommended by uncommon beauty to the lover
of Antinous.\footnotemark[40] But whilst Hadrian was delighting himself with
his own applause, and the acclamations of the soldiers, whose
consent had been secured by an immense donative, the new Cæsar\footnotemark[41]
was ravished from his embraces by an untimely death. He left only
one son. Hadrian commended the boy to the gratitude of the
Antonines. He was adopted by Pius; and, on the accession of
Marcus, was invested with an equal share of sovereign power.
Among the many vices of this younger Verus, he possessed one
virtue; a dutiful reverence for his wiser colleague, to whom he
willingly abandoned the ruder cares of empire. The philosophic
emperor dissembled his follies, lamented his early death, and
cast a decent veil over his memory.

\footnotetext[40]{The deification of Antinous, his medals, his
statues, temples, city, oracles, and constellation, are well
known, and still dishonor the memory of Hadrian. Yet we may
remark, that of the first fifteen emperors, Claudius was the only
one whose taste in love was entirely correct. For the honors of
Antinous, see Spanheim, Commentaire sui les Cæsars de Julien, p.
80.}

\footnotetext[41]{Hist. August. p. 13. Aurelius Victor in Epitom.}

As soon as Hadrian’s passion was either gratified or
disappointed, he resolved to deserve the thanks of posterity, by
placing the most exalted merit on the Roman throne. His
discerning eye easily discovered a senator about fifty years of
age, blameless in all the offices of life; and a youth of about
seventeen, whose riper years opened a fair prospect of every
virtue: the elder of these was declared the son and successor of
Hadrian, on condition, however, that he himself should
immediately adopt the younger. The two Antonines (for it is of
them that we are now speaking,) governed the Roman world
forty-two years, with the same invariable spirit of wisdom and
virtue. Although Pius had two sons,\footnotemark[42] he preferred the welfare
of Rome to the interest of his family, gave his daughter
Faustina, in marriage to young Marcus, obtained from the senate
the tribunitian and proconsular powers, and, with a noble
disdain, or rather ignorance of jealousy, associated him to all
the labors of government. Marcus, on the other hand, revered the
character of his benefactor, loved him as a parent, obeyed him as
his sovereign,\footnotemark[43] and, after he was no more, regulated his own
administration by the example and maxims of his predecessor.
Their united reigns are possibly the only period of history in
which the happiness of a great people was the sole object of
government.

\footnotetext[42]{Without the help of medals and inscriptions, we
should be ignorant of this fact, so honorable to the memory of
Pius. Note: Gibbon attributes to Antoninus Pius a merit which he
either did not possess, or was not in a situation to display.

1. He was adopted only on the condition that he would adopt, in
his turn, Marcus Aurelius and L. Verus.

2. His two sons died children, and one of them, M. Galerius,
alone, appears to have survived, for a few years, his father’s
coronation. Gibbon is also mistaken when he says (note 42) that
“without the help of medals and inscriptions, we should be
ignorant that Antoninus had two sons.” Capitolinus says
expressly, (c. 1,) Filii mares duo, duæ-fœminæ; we only owe
their names to the medals. Pagi. Cont. Baron, i. 33, edit
Paris.—W.}

\footnotetext[43]{During the twenty-three years of Pius’s reign,
Marcus was only two nights absent from the palace, and even those
were at different times. Hist. August. p. 25.}

Titus Antoninus Pius has been justly denominated a second Numa.
The same love of religion, justice, and peace, was the
distinguishing characteristic of both princes. But the situation
of the latter opened a much larger field for the exercise of
those virtues. Numa could only prevent a few neighboring villages
from plundering each other’s harvests. Antoninus diffused order
and tranquillity over the greatest part of the earth. His reign
is marked by the rare advantage of furnishing very few materials
for history; which is, indeed, little more than the register of
the crimes, follies, and misfortunes of mankind. In private life,
he was an amiable, as well as a good man. The native simplicity
of his virtue was a stranger to vanity or affectation. He enjoyed
with moderation the conveniences of his fortune, and the innocent
pleasures of society;\footnotemark[44] and the benevolence of his soul
displayed itself in a cheerful serenity of temper.

\footnotetext[44]{He was fond of the theatre, and not insensible to
the charms of the fair sex. Marcus Antoninus, i. 16. Hist.
August. p. 20, 21. Julian in Cæsar.}

The virtue of Marcus Aurelius Antoninus was of severer and more
laborious kind.\footnotemark[45] It was the well-earned harvest of many a
learned conference, of many a patient lecture, and many a
midnight lucubration. At the age of twelve years he embraced the
rigid system of the Stoics, which taught him to submit his body
to his mind, his passions to his reason; to consider virtue as
the only good, vice as the only evil, all things external as
things indifferent.\footnotemark[46] His meditations, composed in the tumult of
the camp, are still extant; and he even condescended to give
lessons of philosophy, in a more public manner than was perhaps
consistent with the modesty of sage, or the dignity of an
emperor.\footnotemark[47] But his life was the noblest commentary on the
precepts of Zeno. He was severe to himself, indulgent to the
imperfections of others, just and beneficent to all mankind. He
regretted that Avidius Cassius, who excited a rebellion in Syria,
had disappointed him, by a voluntary death,\footnotemark[471] of the pleasure
of converting an enemy into a friend;; and he justified the
sincerity of that sentiment, by moderating the zeal of the senate
against the adherents of the traitor.\footnotemark[48] War he detested, as the
disgrace and calamity of human nature;\footnotemark[481] but when the necessity
of a just defence called upon him to take up arms, he readily
exposed his person to eight winter campaigns, on the frozen banks
of the Danube, the severity of which was at last fatal to the
weakness of his constitution. His memory was revered by a
grateful posterity, and above a century after his death, many
persons preserved the image of Marcus Antoninus among those of
their household gods.\footnotemark[49]

\footnotetext[45]{The enemies of Marcus charged him with hypocrisy,
and with a want of that simplicity which distinguished Pius and
even Verus. (Hist. August. 6, 34.) This suspicions, unjust as it
was, may serve to account for the superior applause bestowed upon
personal qualifications, in preference to the social virtues.
Even Marcus Antoninus has been called a hypocrite; but the
wildest scepticism never insinuated that Cæsar might probably be
a coward, or Tully a fool. Wit and valor are qualifications more
easily ascertained than humanity or the love of justice.}

\footnotetext[46]{Tacitus has characterized, in a few words, the
principles of the portico: Doctores sapientiæ secutus est, qui
sola bona quæ honesta, main tantum quæ turpia; potentiam,
nobilitatem, æteraque extra... bonis neque malis adnumerant.
Tacit. Hist. iv. 5.}

\footnotetext[47]{Before he went on the second expedition against the
Germans, he read lectures of philosophy to the Roman people,
during three days. He had already done the same in the cities of
Greece and Asia. Hist. August. in Cassio, c. 3.}

\footnotetext[471]{Cassius was murdered by his own partisans. Vulcat.
Gallic. in Cassio, c. 7. Dion, lxxi. c. 27.—W.}

\footnotetext[48]{Dion, l. lxxi. p. 1190. Hist. August. in Avid.
Cassio. Note: See one of the newly discovered passages of Dion
Cassius. Marcus wrote to the senate, who urged the execution of
the partisans of Cassius, in these words: “I entreat and beseech
you to preserve my reign unstained by senatorial blood. None of
your order must perish either by your desire or mine.” Mai.
Fragm. Vatican. ii. p. 224.—M.}

\footnotetext[481]{Marcus would not accept the services of any of the
barbarian allies who crowded to his standard in the war against
Avidius Cassius. “Barbarians,” he said, with wise but vain
sagacity, “must not become acquainted with the dissensions of the
Roman people.” Mai. Fragm Vatican l. 224.—M.}

\footnotetext[49]{Hist. August. in Marc. Antonin. c. 18.}

If a man were called to fix the period in the history of the
world, during which the condition of the human race was most
happy and prosperous, he would, without hesitation, name that
which elapsed from the death of Domitian to the accession of
Commodus. The vast extent of the Roman empire was governed by
absolute power, under the guidance of virtue and wisdom. The
armies were restrained by the firm but gentle hand of four
successive emperors, whose characters and authority commanded
involuntary respect. The forms of the civil administration were
carefully preserved by Nerva, Trajan, Hadrian, and the Antonines,
who delighted in the image of liberty, and were pleased with
considering themselves as the accountable ministers of the laws.
Such princes deserved the honor of restoring the republic, had
the Romans of their days been capable of enjoying a rational
freedom.

The labors of these monarchs were overpaid by the immense reward
that inseparably waited on their success; by the honest pride of
virtue, and by the exquisite delight of beholding the general
happiness of which they were the authors. A just but melancholy
reflection imbittered, however, the noblest of human enjoyments.
They must often have recollected the instability of a happiness
which depended on the character of single man. The fatal moment
was perhaps approaching, when some licentious youth, or some
jealous tyrant, would abuse, to the destruction, that absolute
power, which they had exerted for the benefit of their people.
The ideal restraints of the senate and the laws might serve to
display the virtues, but could never correct the vices, of the
emperor. The military force was a blind and irresistible
instrument of oppression; and the corruption of Roman manners
would always supply flatterers eager to applaud, and ministers
prepared to serve, the fear or the avarice, the lust or the
cruelty, of their master. These gloomy apprehensions had been
already justified by the experience of the Romans. The annals of
the emperors exhibit a strong and various picture of human
nature, which we should vainly seek among the mixed and doubtful
characters of modern history. In the conduct of those monarchs we
may trace the utmost lines of vice and virtue; the most exalted
perfection, and the meanest degeneracy of our own species. The
golden age of Trajan and the Antonines had been preceded by an
age of iron. It is almost superfluous to enumerate the unworthy
successors of Augustus. Their unparalleled vices, and the
splendid theatre on which they were acted, have saved them from
oblivion. The dark, unrelenting Tiberius, the furious Caligula,
the feeble Claudius, the profligate and cruel Nero, the beastly
Vitellius,\footnotemark[50] and the timid, inhuman Domitian, are condemned to
everlasting infamy. During fourscore years (excepting only the
short and doubtful respite of Vespasian’s reign)\footnotemark[51] Rome groaned
beneath an unremitting tyranny, which exterminated the ancient
families of the republic, and was fatal to almost every virtue
and every talent that arose in that unhappy period.

\footnotetext[50]{Vitellius consumed in mere eating at least six
millions of our money in about seven months. It is not easy to
express his vices with dignity, or even decency. Tacitus fairly
calls him a hog, but it is by substituting for a coarse word a
very fine image. “At Vitellius, umbraculis hortorum abditus, ut
ignava animalia, quibus si cibum suggeras, jacent torpentque,
præterita, instantia, futura, pari oblivione dimiserat. Atque
illum nemore Aricino desidem et marcentum,” \&c. Tacit. Hist. iii.
36, ii. 95. Sueton. in Vitell. c. 13. Dion. Cassius, l xv. p.
1062.}

\footnotetext[51]{The execution of Helvidius Priscus, and of the
virtuous Eponina, disgraced the reign of Vespasian.}

Under the reign of these monsters, the slavery of the Romans was
accompanied with two peculiar circumstances, the one occasioned
by their former liberty, the other by their extensive conquests,
which rendered their condition more completely wretched than that
of the victims of tyranny in any other age or country. From these
causes were derived, 1. The exquisite sensibility of the
sufferers; and, 2. The impossibility of escaping from the hand of
the oppressor.

I. When Persia was governed by the descendants of Sefi, a race of
princes whose wanton cruelty often stained their divan, their
table, and their bed, with the blood of their favorites, there is
a saying recorded of a young nobleman, that he never departed
from the sultan’s presence, without satisfying himself whether
his head was still on his shoulders. The experience of every day
might almost justify the scepticism of Rustan.\footnotemark[52] Yet the fatal
sword, suspended above him by a single thread, seems not to have
disturbed the slumbers, or interrupted the tranquillity, of the
Persian. The monarch’s frown, he well knew, could level him with
the dust; but the stroke of lightning or apoplexy might be
equally fatal; and it was the part of a wise man to forget the
inevitable calamities of human life in the enjoyment of the
fleeting hour. He was dignified with the appellation of the
king’s slave; had, perhaps, been purchased from obscure parents,
in a country which he had never known; and was trained up from
his infancy in the severe discipline of the seraglio.\footnotemark[53] His
name, his wealth, his honors, were the gift of a master, who
might, without injustice, resume what he had bestowed. Rustan’s
knowledge, if he possessed any, could only serve to confirm his
habits by prejudices. His language afforded not words for any
form of government, except absolute monarchy. The history of the
East informed him, that such had ever been the condition of
mankind.\footnotemark[54] The Koran, and the interpreters of that divine book,
inculcated to him, that the sultan was the descendant of the
prophet, and the vicegerent of heaven; that patience was the
first virtue of a Mussulman, and unlimited obedience the great
duty of a subject.

\footnotetext[52]{Voyage de Chardin en Perse, vol. iii. p. 293.}

\footnotetext[53]{The practice of raising slaves to the great offices
of state is still more common among the Turks than among the
Persians. The miserable countries of Georgia and Circassia supply
rulers to the greatest part of the East.}

\footnotetext[54]{Chardin says, that European travellers have
diffused among the Persians some ideas of the freedom and
mildness of our governments. They have done them a very ill
office.}

The minds of the Romans were very differently prepared for
slavery. Oppressed beneath the weight of their own corruption and
of military violence, they for a long while preserved the
sentiments, or at least the ideas, of their free-born ancestors.
The education of Helvidius and Thrasea, of Tacitus and Pliny, was
the same as that of Cato and Cicero. From Grecian philosophy,
they had imbibed the justest and most liberal notions of the
dignity of human nature, and the origin of civil society. The
history of their own country had taught them to revere a free, a
virtuous, and a victorious commonwealth; to abhor the successful
crimes of Cæsar and Augustus; and inwardly to despise those
tyrants whom they adored with the most abject flattery. As
magistrates and senators they were admitted into the great
council, which had once dictated laws to the earth, whose
authority was so often prostituted to the vilest purposes of
tyranny. Tiberius, and those emperors who adopted his maxims,
attempted to disguise their murders by the formalities of
justice, and perhaps enjoyed a secret pleasure in rendering the
senate their accomplice as well as their victim. By this
assembly, the last of the Romans were condemned for imaginary
crimes and real virtues. Their infamous accusers assumed the
language of independent patriots, who arraigned a dangerous
citizen before the tribunal of his country; and the public
service was rewarded by riches and honors.\footnotemark[55] The servile judges
professed to assert the majesty of the commonwealth, violated in
the person of its first magistrate,\footnotemark[56] whose clemency they most
applauded when they trembled the most at his inexorable and
impending cruelty.\footnotemark[57] The tyrant beheld their baseness with just
contempt, and encountered their secret sentiments of detestation
with sincere and avowed hatred for the whole body of the senate.

\footnotetext[55]{They alleged the example of Scipio and Cato,
(Tacit. Annal. iii. 66.) Marcellus Epirus and Crispus Vibius had
acquired two millions and a half under Nero. Their wealth, which
aggravated their crimes, protected them under Vespasian. See
Tacit. Hist. iv. 43. Dialog. de Orator. c. 8. For one accusation,
Regulus, the just object of Pliny’s satire, received from the
senate the consular ornaments, and a present of sixty thousand
pounds.}

\footnotetext[56]{The crime of majesty was formerly a treasonable
offence against the Roman people. As tribunes of the people,
Augustus and Tiberius applied tit to their own persons, and
extended it to an infinite latitude. Note: It was Tiberius, not
Augustus, who first took in this sense the words crimen læsæ
majestatis. Bachii Trajanus, 27. —W.}

\footnotetext[57]{After the virtuous and unfortunate widow of
Germanicus had been put to death, Tiberius received the thanks of
the senate for his clemency. she had not been publicly strangled;
nor was the body drawn with a hook to the Gemoniæ, where those of
common male factors were exposed. See Tacit. Annal. vi. 25.
Sueton. in Tiberio c. 53.}

II. The division of Europe into a number of independent states,
connected, however, with each other by the general resemblance of
religion, language, and manners, is productive of the most
beneficial consequences to the liberty of mankind. A modern
tyrant, who should find no resistance either in his own breast,
or in his people, would soon experience a gentle restraint from
the example of his equals, the dread of present censure, the
advice of his allies, and the apprehension of his enemies. The
object of his displeasure, escaping from the narrow limits of his
dominions, would easily obtain, in a happier climate, a secure
refuge, a new fortune adequate to his merit, the freedom of
complaint, and perhaps the means of revenge. But the empire of
the Romans filled the world, and when the empire fell into the
hands of a single person, the world became a safe and dreary
prison for his enemies. The slave of Imperial despotism, whether
he was condemned to drag his gilded chain in rome and the senate,
or to were out a life of exile on the barren rock of Seriphus, or
the frozen bank of the Danube, expected his fate in silent
despair.\footnotemark[58] To resist was fatal, and it was impossible to fly. On
every side he was encompassed with a vast extent of sea and land,
which he could never hope to traverse without being discovered,
seized, and restored to his irritated master. Beyond the
frontiers, his anxious view could discover nothing, except the
ocean, inhospitable deserts, hostile tribes of barbarians, of
fierce manners and unknown language, or dependent kings, who
would gladly purchase the emperor’s protection by the sacrifice
of an obnoxious fugitive.\footnotemark[59] “Wherever you are,” said Cicero to
the exiled Marcellus, “remember that you are equally within the
power of the conqueror.”\footnotemark[60]

\footnotetext[58]{Seriphus was a small rocky island in the Ægean Sea,
the inhabitants of which were despised for their ignorance and
obscurity. The place of Ovid’s exile is well known, by his just,
but unmanly lamentations. It should seem, that he only received
an order to leave rome in so many days, and to transport himself
to Tomi. Guards and jailers were unnecessary.}

\footnotetext[59]{Under Tiberius, a Roman knight attempted to fly to
the Parthians. He was stopped in the straits of Sicily; but so
little danger did there appear in the example, that the most
jealous of tyrants disdained to punish it. Tacit. Annal. vi. 14.}

\footnotetext[60]{Cicero ad Familiares, iv. 7.}

