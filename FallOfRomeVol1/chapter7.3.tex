\section{Part \thesection.}
\thispagestyle{simple}

On his return from the East to Rome, Philip, desirous of
obliterating the memory of his crimes, and of captivating the
affections of the people, solemnized the secular games with
infinite pomp and magnificence. Since their institution or
revival by Augustus,\footnotemark[56] they had been celebrated by Claudius, by
Domitian, and by Severus, and were now renewed the fifth time, on
the accomplishment of the full period of a thousand years from
the foundation of Rome. Every circumstance of the secular games
was skillfully adapted to inspire the superstitious mind with
deep and solemn reverence. The long interval between them\footnotemark[57]
exceeded the term of human life; and as none of the spectators
had already seen them, none could flatter themselves with the
expectation of beholding them a second time. The mystic
sacrifices were performed, during three nights, on the banks of
the Tyber; and the Campus Martius resounded with music and
dances, and was illuminated with innumerable lamps and torches.
Slaves and strangers were excluded from any participation in
these national ceremonies. A chorus of twenty-seven youths, and
as many virgins, of noble families, and whose parents were both
alive, implored the propitious gods in favor of the present, and
for the hope of the rising generation; requesting, in religious
hymns, that according to the faith of their ancient oracles, they
would still maintain the virtue, the felicity, and the empire of
the Roman people. The magnificence of Philip’s shows and
entertainments dazzled the eyes of the multitude. The devout were
employed in the rites of superstition, whilst the reflecting few
revolved in their anxious minds the past history and the future
fate of the empire.\footnotemark[58]

\footnotetext[56]{The account of the last supposed celebration,
though in an enlightened period of history, was so very doubtful
and obscure, that the alternative seems not doubtful. When the
popish jubilees, the copy of the secular games, were invented by
Boniface VII., the crafty pope pretended that he only revived an
ancient institution. See M. le Chais, Lettres sur les Jubiles.}

\footnotetext[57]{Either of a hundred or a hundred and ten years.
Varro and Livy adopted the former opinion, but the infallible
authority of the Sybil consecrated the latter, (Censorinus de Die
Natal. c. 17.) The emperors Claudius and Philip, however, did not
treat the oracle with implicit respect.}

\footnotetext[58]{The idea of the secular games is best understood
from the poem of Horace, and the description of Zosimus, 1. l.
ii. p. 167, \&c.}

Since Romulus, with a small band of shepherds
and outlaws, fortified himself on the hills near the Tyber, ten
centuries had already elapsed.\footnotemark[59] During the four first ages, the
Romans, in the laborious school of poverty, had acquired the
virtues of war and government: by the vigorous exertion of those
virtues, and by the assistance of fortune, they had obtained, in
the course of the three succeeding centuries, an absolute empire
over many countries of Europe, Asia, and Africa. The last three
hundred years had been consumed in apparent prosperity and
internal decline. The nation of soldiers, magistrates, and
legislators, who composed the thirty-five tribes of the Roman
people, were dissolved into the common mass of mankind, and
confounded with the millions of servile provincials, who had
received the name, without adopting the spirit, of Romans. A
mercenary army, levied among the subjects and barbarians of the
frontier, was the only order of men who preserved and abused
their independence. By their tumultuary election, a Syrian, a
Goth, or an Arab, was exalted to the throne of Rome, and invested
with despotic power over the conquests and over the country of
the Scipios.

\footnotetext[59]{The received calculation of Varro assigns to the
foundation of Rome an æra that corresponds with the 754th year
before Christ. But so little is the chronology of Rome to be
depended on, in the more early ages, that Sir Isaac Newton has
brought the same event as low as the year 627 (Compare Niebuhr
vol. i. p. 271.—M.)}

The limits of the Roman empire still extended from the Western
Ocean to the Tigris, and from Mount Atlas to the Rhine and the
Danube. To the undiscerning eye of the vulgar, Philip appeared a
monarch no less powerful than Hadrian or Augustus had formerly
been. The form was still the same, but the animating health and
vigor were fled. The industry of the people was discouraged and
exhausted by a long series of oppression. The discipline of the
legions, which alone, after the extinction of every other virtue,
had propped the greatness of the state, was corrupted by the
ambition, or relaxed by the weakness, of the emperors. The
strength of the frontiers, which had always consisted in arms
rather than in fortifications, was insensibly undermined; and the
fairest provinces were left exposed to the rapaciousness or
ambition of the barbarians, who soon discovered the decline of
the Roman empire.

