\section{Part \thesection.}
\thispagestyle{simple}

The execution of so many innocent citizens was bewailed by the
secret tears of their friends and families. The death of
Papinian, the Prætorian Præfect, was lamented as a public
calamity.\footnotemark[282] During the last seven years of Severus, he had
exercised the most important offices of the state, and, by his
salutary influence, guided the emperor’s steps in the paths of
justice and moderation. In full assurance of his virtue and
abilities, Severus, on his death-bed, had conjured him to watch
over the prosperity and union of the Imperial family.\footnotemark[29] The
honest labors of Papinian served only to inflame the hatred which
Caracalla had already conceived against his father’s minister.
After the murder of Geta, the Præfect was commanded to exert the
powers of his skill and eloquence in a studied apology for that
atrocious deed. The philosophic Seneca had condescended to
compose a similar epistle to the senate, in the name of the son
and assassin of Agrippina.\footnotemark[30] “That it was easier to commit than
to justify a parricide,” was the glorious reply of Papinian;\footnotemark[31]
who did not hesitate between the loss of life and that of honor.
Such intrepid virtue, which had escaped pure and unsullied from
the intrigues of courts, the habits of business, and the arts of
his profession, reflects more lustre on the memory of Papinian,
than all his great employments, his numerous writings, and the
superior reputation as a lawyer, which he has preserved through
every age of the Roman jurisprudence.\footnotemark[32]

\footnotetext[282]{Papinian was no longer Prætorian Præfect.
Caracalla had deprived him of that office immediately after the
death of Severus. Such is the statement of Dion; and the
testimony of Spartian, who gives Papinian the Prætorian
præfecture till his death, is of little weight opposed to that of
a senator then living at Rome.—W.}

\footnotetext[29]{It is said that Papinian was himself a relation of
the empress Julia.}

\footnotetext[30]{Tacit. Annal. xiv. 2.}

\footnotetext[31]{Hist. August. p. 88.}

\footnotetext[32]{With regard to Papinian, see Heineccius’s Historia
Juris Roma ni, l. 330, \&c.}

It had hitherto been the peculiar felicity of the Romans, and in
the worst of times the consolation, that the virtue of the
emperors was active, and their vice indolent. Augustus, Trajan,
Hadrian, and Marcus visited their extensive dominions in person,
and their progress was marked by acts of wisdom and beneficence.
The tyranny of Tiberius, Nero, and Domitian, who resided almost
constantly at Rome, or in the adjacent was confined to the
senatorial and equestrian orders.\footnotemark[33] But Caracalla was the common
enemy of mankind. He left the capital (and he never returned to
it) about a year after the murder of Geta. The rest of his reign
was spent in the several provinces of the empire, particularly
those of the East, and every province was by turns the scene of
his rapine and cruelty. The senators, compelled by fear to attend
his capricious motions, were obliged to provide daily
entertainments at an immense expense, which he abandoned with
contempt to his guards; and to erect, in every city, magnificent
palaces and theatres, which he either disdained to visit, or
ordered immediately thrown down. The most wealthy families were
ruined by partial fines and confiscations, and the great body of
his subjects oppressed by ingenious and aggravated taxes.\footnotemark[34] In
the midst of peace, and upon the slightest provocation, he issued
his commands, at Alexandria, in Egypt for a general massacre.
From a secure post in the temple of Serapis, he viewed and
directed the slaughter of many thousand citizens, as well as
strangers, without distinguishing the number or the crime of the
sufferers; since as he coolly informed the senate, \textit{all} the
Alexandrians, those who had perished, and those who had escaped,
were alike guilty.\footnotemark[35]

\footnotetext[33]{Tiberius and Domitian never moved from the
neighborhood of Rome. Nero made a short journey into Greece. “Et
laudatorum Principum usus ex æquo, quamvis procul agentibus. Sævi
proximis ingruunt.” Tacit. Hist. iv. 74.}

\footnotetext[34]{Dion, l. lxxvii. p. 1294.}

\footnotetext[35]{Dion, l. lxxvii. p. 1307. Herodian, l. iv. p. 158.
The former represents it as a cruel massacre, the latter as a
perfidious one too. It seems probable that the Alexandrians has
irritated the tyrant by their railleries, and perhaps by their
tumults. * Note: After these massacres, Caracalla also deprived
the Alexandrians of their spectacles and public feasts; he
divided the city into two parts by a wall with towers at
intervals, to prevent the peaceful communications of the
citizens. Thus was treated the unhappy Alexandria, says Dion, by
the savage beast of Ausonia. This, in fact, was the epithet which
the oracle had applied to him; it is said, indeed, that he was
much pleased with the name and often boasted of it. Dion, lxxvii.
p. 1307.—G.}

The wise instructions of Severus never made any lasting
impression on the mind of his son, who, although not destitute of
imagination and eloquence, was equally devoid of judgment and
humanity.\footnotemark[36] One dangerous maxim, worthy of a tyrant, was
remembered and abused by Caracalla. “To secure the affections of
the army, and to esteem the rest of his subjects as of little
moment.”\footnotemark[37] But the liberality of the father had been restrained
by prudence, and his indulgence to the troops was tempered by
firmness and authority. The careless profusion of the son was the
policy of one reign, and the inevitable ruin both of the army and
of the empire. The vigor of the soldiers, instead of being
confirmed by the severe discipline of camps, melted away in the
luxury of cities. The excessive increase of their pay and
donatives\footnotemark[38] exhausted the state to enrich the military order,
whose modesty in peace, and service in war, is best secured by an
honorable poverty. The demeanor of Caracalla was haughty and full
of pride; but with the troops he forgot even the proper dignity
of his rank, encouraged their insolent familiarity, and,
neglecting the essential duties of a general, affected to imitate
the dress and manners of a common soldier.

\footnotetext[36]{Dion, l. lxxvii. p. 1296.}

\footnotetext[37]{Dion, l. lxxvi. p. 1284. Mr. Wotton (Hist. of Rome,
p. 330) suspects that this maxim was invented by Caracalla
himself, and attributed to his father.}

\footnotetext[38]{Dion (l. lxxviii. p. 1343) informs us that the
extraordinary gifts of Caracalla to the army amounted annually to
seventy millions of drachmæ (about two millions three hundred and
fifty thousand pounds.) There is another passage in Dion,
concerning the military pay, infinitely curious, were it not
obscure, imperfect, and probably corrupt. The best sense seems to
be, that the Prætorian guards received twelve hundred and fifty
drachmæ, (forty pounds a year,) (Dion, l. lxxvii. p. 1307.) Under
the reign of Augustus, they were paid at the rate of two drachmæ,
or denarii, per day, 720 a year, (Tacit. Annal. i. 17.) Domitian,
who increased the soldiers’ pay one fourth, must have raised the
Prætorians to 960 drachmæ, (Gronoviue de Pecunia Veteri, l. iii.
c. 2.) These successive augmentations ruined the empire; for,
with the soldiers’ pay, their numbers too were increased. We have
seen the Prætorians alone increased from 10,000 to 50,000 men.
Note: Valois and Reimar have explained in a very simple and
probable manner this passage of Dion, which Gibbon seems to me
not to have understood. He ordered that the soldiers should
receive, as the reward of their services the Prætorians 1250
drachms, the other 5000 drachms. Valois thinks that the numbers
have been transposed, and that Caracalla added 5000 drachms to
the donations made to the Prætorians, 1250 to those of the
legionaries. The Prætorians, in fact, always received more than
the others. The error of Gibbon arose from his considering that
this referred to the annual pay of the soldiers, while it relates
to the sum they received as a reward for their services on their
discharge: donatives means recompense for service. Augustus had
settled that the Prætorians, after sixteen campaigns, should
receive 5000 drachms: the legionaries received only 3000 after
twenty years. Caracalla added 5000 drachms to the donative of the
Prætorians, 1250 to that of the legionaries. Gibbon appears to
have been mistaken both in confounding this donative on discharge
with the annual pay, and in not paying attention to the remark of
Valois on the transposition of the numbers in the text.—G}

It was impossible that such a character, and such conduct as that
of Caracalla, could inspire either love or esteem; but as long as
his vices were beneficial to the armies, he was secure from the
danger of rebellion. A secret conspiracy, provoked by his own
jealousy, was fatal to the tyrant. The Prætorian præfecture was
divided between two ministers. The military department was
intrusted to Adventus, an experienced rather than able soldier;
and the civil affairs were transacted by Opilius Macrinus, who,
by his dexterity in business, had raised himself, with a fair
character, to that high office. But his favor varied with the
caprice of the emperor, and his life might depend on the
slightest suspicion, or the most casual circumstance. Malice or
fanaticism had suggested to an African, deeply skilled in the
knowledge of futurity, a very dangerous prediction, that Macrinus
and his son were destined to reign over the empire. The report
was soon diffused through the province; and when the man was sent
in chains to Rome, he still asserted, in the presence of the
præfect of the city, the faith of his prophecy. That magistrate,
who had received the most pressing instructions to inform himself
of the \textit{successors} of Caracalla, immediately communicated the
examination of the African to the Imperial court, which at that
time resided in Syria. But, notwithstanding the diligence of the
public messengers, a friend of Macrinus found means to apprise
him of the approaching danger. The emperor received the letters
from Rome; and as he was then engaged in the conduct of a chariot
race, he delivered them unopened to the Prætorian Præfect,
directing him to despatch the ordinary affairs, and to report the
more important business that might be contained in them. Macrinus
read his fate, and resolved to prevent it. He inflamed the
discontents of some inferior officers, and employed the hand of
Martialis, a desperate soldier, who had been refused the rank of
centurion. The devotion of Caracalla prompted him to make a
pilgrimage from Edessa to the celebrated temple of the Moon at
Carrhæ.\footnotemark[381] He was attended by a body of cavalry: but having
stopped on the road for some necessary occasion, his guards
preserved a respectful distance, and Martialis, approaching his
person under a presence of duty, stabbed him with a dagger. The
bold assassin was instantly killed by a Scythian archer of the
Imperial guard. Such was the end of a monster whose life
disgraced human nature, and whose reign accused the patience of
the Romans.\footnotemark[39] The grateful soldiers forgot his vices, remembered
only his partial liberality, and obliged the senate to prostitute
their own dignity and that of religion, by granting him a place
among the gods. Whilst he was upon earth, Alexander the Great was
the only hero whom this god deemed worthy his admiration. He
assumed the name and ensigns of Alexander, formed a Macedonian
phalanx of guards, persecuted the disciples of Aristotle, and
displayed, with a puerile enthusiasm, the only sentiment by which
he discovered any regard for virtue or glory. We can easily
conceive, that after the battle of Narva, and the conquest of
Poland, Charles XII. (though he still wanted the more elegant
accomplishments of the son of Philip) might boast of having
rivalled his valor and magnanimity; but in no one action of his
life did Caracalla express the faintest resemblance of the
Macedonian hero, except in the murder of a great number of his
own and of his father’s friends.\footnotemark[40]

\footnotetext[381]{Carrhæ, now Harran, between Edessan and Nisibis,
famous for the defeat of Crassus—the Haran from whence Abraham
set out for the land of Canaan. This city has always been
remarkable for its attachment to Sabaism—G}

\footnotetext[39]{Dion, l. lxxviii. p. 1312. Herodian, l. iv. p.
168.}

\footnotetext[40]{The fondness of Caracalla for the name and ensigns
of Alexander is still preserved on the medals of that emperor.
See Spanheim, de Usu Numismatum, Dissertat. xii. Herodian (l. iv.
p. 154) had seen very ridiculous pictures, in which a figure was
drawn with one side of the face like Alexander, and the other
like Caracalla.}

After the extinction of the house of Severus, the Roman world
remained three days without a master. The choice of the army (for
the authority of a distant and feeble senate was little regarded)
hung in anxious suspense, as no candidate presented himself whose
distinguished birth and merit could engage their attachment and
unite their suffrages. The decisive weight of the Prætorian
guards elevated the hopes of their præfects, and these powerful
ministers began to assert their \textit{legal} claim to fill the vacancy
of the Imperial throne. Adventus, however, the senior præfect,
conscious of his age and infirmities, of his small reputation,
and his smaller abilities, resigned the dangerous honor to the
crafty ambition of his colleague Macrinus, whose well-dissembled
grief removed all suspicion of his being accessary to his
master’s death.\footnotemark[41] The troops neither loved nor esteemed his
character. They cast their eyes around in search of a competitor,
and at last yielded with reluctance to his promises of unbounded
liberality and indulgence. A short time after his accession, he
conferred on his son Diadumenianus, at the age of only ten years,
the Imperial title, and the popular name of Antoninus. The
beautiful figure of the youth, assisted by an additional
donative, for which the ceremony furnished a pretext, might
attract, it was hoped, the favor of the army, and secure the
doubtful throne of Macrinus.

\footnotetext[41]{Herodian, l. iv. p. 169. Hist. August. p. 94.}

The authority of the new sovereign had been ratified by the
cheerful submission of the senate and provinces. They exulted in
their unexpected deliverance from a hated tyrant, and it seemed
of little consequence to examine into the virtues of the
successor of Caracalla. But as soon as the first transports of
joy and surprise had subsided, they began to scrutinize the
merits of Macrinus with a critical severity, and to arraign the
nasty choice of the army. It had hitherto been considered as a
fundamental maxim of the constitution, that the emperor must be
always chosen in the senate, and the sovereign power, no longer
exercised by the whole body, was always delegated to one of its
members. But Macrinus was not a senator.\footnotemark[42] The sudden elevation
of the Prætorian præfects betrayed the meanness of their origin;
and the equestrian order was still in possession of that great
office, which commanded with arbitrary sway the lives and
fortunes of the senate. A murmur of indignation was heard, that a
man, whose obscure\footnotemark[43] extraction had never been illustrated by
any signal service, should dare to invest himself with the
purple, instead of bestowing it on some distinguished senator,
equal in birth and dignity to the splendor of the Imperial
station. As soon as the character of Macrinus was surveyed by the
sharp eye of discontent, some vices, and many defects, were
easily discovered. The choice of his ministers was in many
instances justly censured, and the dissatisfied people, with
their usual candor, accused at once his indolent tameness and his
excessive severity.\footnotemark[44]

\footnotetext[42]{Dion, l. lxxxviii. p. 1350. Elagabalus reproached
his predecessor with daring to seat himself on the throne;
though, as Prætorian præfect, he could not have been admitted
into the senate after the voice of the crier had cleared the
house. The personal favor of Plautianus and Sejanus had broke
through the established rule. They rose, indeed, from the
equestrian order; but they preserved the præfecture, with the
rank of senator and even with the annulship.}

\footnotetext[43]{He was a native of Cæsarea, in Numidia, and began
his fortune by serving in the household of Plautian, from whose
ruin he narrowly escaped. His enemies asserted that he was born a
slave, and had exercised, among other infamous professions, that
of Gladiator. The fashion of aspersing the birth and condition of
an adversary seems to have lasted from the time of the Greek
orators to the learned grammarians of the last age.}

\footnotetext[44]{Both Dion and Herodian speak of the virtues and
vices of Macrinus with candor and impartiality; but the author of
his life, in the Augustan History, seems to have implicitly
copied some of the venal writers, employed by Elagabalus, to
blacken the memory of his predecessor.}

His rash ambition had climbed a height where it was difficult to
stand with firmness, and impossible to fall without instant
destruction. Trained in the arts of courts and the forms of civil
business, he trembled in the presence of the fierce and
undisciplined multitude, over whom he had assumed the command;
his military talents were despised, and his personal courage
suspected; a whisper that circulated in the camp, disclosed the
fatal secret of the conspiracy against the late emperor,
aggravated the guilt of murder by the baseness of hypocrisy, and
heightened contempt by detestation. To alienate the soldiers, and
to provoke inevitable ruin, the character of a reformer was only
wanting; and such was the peculiar hardship of his fate, that
Macrinus was compelled to exercise that invidious office. The
prodigality of Caracalla had left behind it a long train of ruin
and disorder; and if that worthless tyrant had been capable of
reflecting on the sure consequences of his own conduct, he would
perhaps have enjoyed the dark prospect of the distress and
calamities which he bequeathed to his successors.

In the management of this necessary reformation, Macrinus
proceeded with a cautious prudence, which would have restored
health and vigor to the Roman army in an easy and almost
imperceptible manner. To the soldiers already engaged in the
service, he was constrained to leave the dangerous privileges and
extravagant pay given by Caracalla; but the new recruits were
received on the more moderate though liberal establishment of
Severus, and gradually formed to modesty and obedience.\footnotemark[45] One
fatal error destroyed the salutary effects of this judicious
plan. The numerous army, assembled in the East by the late
emperor, instead of being immediately dispersed by Macrinus
through the several provinces, was suffered to remain united in
Syria, during the winter that followed his elevation. In the
luxurious idleness of their quarters, the troops viewed their
strength and numbers, communicated their complaints, and revolved
in their minds the advantages of another revolution. The
veterans, instead of being flattered by the advantageous
distinction, were alarmed by the first steps of the emperor,
which they considered as the presage of his future intentions.
The recruits, with sullen reluctance, entered on a service, whose
labors were increased while its rewards were diminished by a
covetous and unwarlike sovereign. The murmurs of the army swelled
with impunity into seditious clamors; and the partial mutinies
betrayed a spirit of discontent and disaffection that waited only
for the slightest occasion to break out on every side into a
general rebellion. To minds thus disposed, the occasion soon
presented itself.

\footnotetext[45]{Dion, l. lxxxiii. p. 1336. The sense of the author
is as the intention of the emperor; but Mr. Wotton has mistaken
both, by understanding the distinction, not of veterans and
recruits, but of old and new legions. History of Rome, p. 347.}

The empress Julia had experienced all the vicissitudes of
fortune. From an humble station she had been raised to greatness,
only to taste the superior bitterness of an exalted rank. She was
doomed to weep over the death of one of her sons, and over the
life of the other. The cruel fate of Caracalla, though her good
sense must have long taught her to expect it, awakened the
feelings of a mother and of an empress. Notwithstanding the
respectful civility expressed by the usurper towards the widow of
Severus, she descended with a painful struggle into the condition
of a subject, and soon withdrew herself, by a voluntary death,
from the anxious and humiliating dependence.\footnotemark[46] \footnotemark[461] Julia Mæsa,
her sister, was ordered to leave the court and Antioch. She
retired to Emesa with an immense fortune, the fruit of twenty
years’ favor accompanied by her two daughters, Soæmias and Mamæ,
each of whom was a widow, and each had an only son. Bassianus,\footnotemark[462]
for that was the name of the son of Soæmias, was consecrated
to the honorable ministry of high priest of the Sun; and this
holy vocation, embraced either from prudence or superstition,
contributed to raise the Syrian youth to the empire of Rome. A
numerous body of troops was stationed at Emesa; and as the severe
discipline of Macrinus had constrained them to pass the winter
encamped, they were eager to revenge the cruelty of such
unaccustomed hardships. The soldiers, who resorted in crowds to
the temple of the Sun, beheld with veneration and delight the
elegant dress and figure of the young pontiff; they recognized,
or they thought that they recognized, the features of Caracalla,
whose memory they now adored. The artful Mæsa saw and cherished
their rising partiality, and readily sacrificing her daughter’s
reputation to the fortune of her grandson, she insinuated that
Bassianus was the natural son of their murdered sovereign. The
sums distributed by her emissaries with a lavish hand silenced
every objection, and the profusion sufficiently proved the
affinity, or at least the resemblance, of Bassianus with the
great original. The young Antoninus (for he had assumed and
polluted that respectable name) was declared emperor by the
troops of Emesa, asserted his hereditary right, and called aloud
on the armies to follow the standard of a young and liberal
prince, who had taken up arms to revenge his father’s death and
the oppression of the military order.\footnotemark[47]

\footnotetext[46]{Dion, l. lxxviii. p. 1330. The abridgment of
Xiphilin, though less particular, is in this place clearer than
the original.}

\footnotetext[461]{As soon as this princess heard of the death of
Caracalla, she wished to starve herself to death: the respect
shown to her by Macrinus, in making no change in her attendants
or her court, induced her to prolong her life. But it appears, as
far as the mutilated text of Dion and the imperfect epitome of
Xiphilin permit us to judge, that she conceived projects of
ambition, and endeavored to raise herself to the empire. She
wished to tread in the steps of Semiramis and Nitocris, whose
country bordered on her own. Macrinus sent her an order
immediately to leave Antioch, and to retire wherever she chose.
She returned to her former purpose, and starved herself to
death.—G.}

\footnotetext[462]{He inherited this name from his great-grandfather
of the mother’s side, Bassianus, father of Julia Mæsa, his
grandmother, and of Julia Domna, wife of Severus. Victor (in his
epitome) is perhaps the only historian who has given the key to
this genealogy, when speaking of Caracalla. His Bassianus ex avi
materni nomine dictus. Caracalla, Elagabalus, and Alexander
Seyerus, bore successively this name.—G.}

\footnotetext[47]{According to Lampridius, (Hist. August. p. 135,)
Alexander Severus lived twenty-nine years three months and seven
days. As he was killed March 19, 235, he was born December 12,
205 and was consequently about this time thirteen years old, as
his elder cousin might be about seventeen. This computation suits
much better the history of the young princes than that of
Herodian, (l. v. p. 181,) who represents them as three years
younger; whilst, by an opposite error of chronology, he lengthens
the reign of Elagabalus two years beyond its real duration. For
the particulars of the conspiracy, see Dion, l. lxxviii. p. 1339.
Herodian, l. v. p. 184.}

Whilst a conspiracy of women and eunuchs was concerted with
prudence, and conducted with rapid vigor, Macrinus, who, by a
decisive motion, might have crushed his infant enemy, floated
between the opposite extremes of terror and security, which alike
fixed him inactive at Antioch. A spirit of rebellion diffused
itself through all the camps and garrisons of Syria, successive
detachments murdered their officers,\footnotemark[48] and joined the party of
the rebels; and the tardy restitution of military pay and
privileges was imputed to the acknowledged weakness of Macrinus.
At length he marched out of Antioch, to meet the increasing and
zealous army of the young pretender. His own troops seemed to
take the field with faintness and reluctance; but, in the heat of
the battle,\footnotemark[49] the Prætorian guards, almost by an involuntary
impulse, asserted the superiority of their valor and discipline.
The rebel ranks were broken; when the mother and grandmother of
the Syrian prince, who, according to their eastern custom, had
attended the army, threw themselves from their covered chariots,
and, by exciting the compassion of the soldiers, endeavored to
animate their drooping courage. Antoninus himself, who, in the
rest of his life, never acted like a man, in this important
crisis of his fate, approved himself a hero, mounted his horse,
and, at the head of his rallied troops, charged sword in hand
among the thickest of the enemy; whilst the eunuch Gannys, 491
whose occupations had been confined to female cares and the soft
luxury of Asia, displayed the talents of an able and experienced
general. The battle still raged with doubtful violence, and
Macrinus might have obtained the victory, had he not betrayed his
own cause by a shameful and precipitate flight. His cowardice
served only to protract his life a few days, and to stamp
deserved ignominy on his misfortunes. It is scarcely necessary to
add, that his son Diadumenianus was involved in the same fate.

As soon as the stubborn Prætorians could be convinced that they
fought for a prince who had basely deserted them, they
surrendered to the conqueror: the contending parties of the Roman
army, mingling tears of joy and tenderness, united under the
banners of the imagined son of Caracalla, and the East
acknowledged with pleasure the first emperor of Asiatic
extraction.

\footnotetext[48]{By a most dangerous proclamation of the pretended
Antoninus, every soldier who brought in his officer’s head became
entitled to his private estate, as well as to his military
commission.}

\footnotetext[49]{Dion, l. lxxviii. p. 1345. Herodian, l. v. p. 186.
The battle was fought near the village of Immæ, about
two-and-twenty miles from Antioch.}

\footnotetext[491]{Gannys was not a eunuch. Dion, p. 1355.—W}

The letters of Macrinus had condescended to inform the senate of
the slight disturbance occasioned by an impostor in Syria, and a
decree immediately passed, declaring the rebel and his family
public enemies; with a promise of pardon, however, to such of his
deluded adherents as should merit it by an immediate return to
their duty. During the twenty days that elapsed from the
declaration of the victory of Antoninus (for in so short an
interval was the fate of the Roman world decided,) the capital
and the provinces, more especially those of the East, were
distracted with hopes and fears, agitated with tumult, and
stained with a useless effusion of civil blood, since whosoever
of the rivals prevailed in Syria must reign over the empire. The
specious letters in which the young conqueror announced his
victory to the obedient senate were filled with professions of
virtue and moderation; the shining examples of Marcus and
Augustus, he should ever consider as the great rule of his
administration; and he affected to dwell with pride on the
striking resemblance of his own age and fortunes with those of
Augustus, who in the earliest youth had revenged, by a successful
war, the murder of his father. By adopting the style of Marcus
Aurelius Antoninus, son of Antoninus and grandson of Severus, he
tacitly asserted his hereditary claim to the empire; but, by
assuming the tribunitian and proconsular powers before they had
been conferred on him by a decree of the senate, he offended the
delicacy of Roman prejudice. This new and injudicious violation
of the constitution was probably dictated either by the ignorance
of his Syrian courtiers, or the fierce disdain of his military
followers.\footnotemark[50]

\footnotetext[50]{Dion, l. lxxix. p. 1353.}

As the attention of the new emperor was diverted by the most
trifling amusements, he wasted many months in his luxurious
progress from Syria to Italy, passed at Nicomedia his first
winter after his victory, and deferred till the ensuing summer
his triumphal entry into the capital. A faithful picture,
however, which preceded his arrival, and was placed by his
immediate order over the altar of Victory in the senate house,
conveyed to the Romans the just but unworthy resemblance of his
person and manners. He was drawn in his sacerdotal robes of silk
and gold, after the loose flowing fashion of the Medes and
Phœnicians; his head was covered with a lofty tiara, his numerous
collars and bracelets were adorned with gems of an inestimable
value. His eyebrows were tinged with black, and his cheeks
painted with an artificial red and white.\footnotemark[51] The grave senators
confessed with a sigh, that, after having long experienced the
stern tyranny of their own countrymen, Rome was at length humbled
beneath the effeminate luxury of Oriental despotism.

\footnotetext[51]{Dion, l. lxxix. p. 1363. Herodian, l. v. p. 189.}

The Sun was worshipped at Emesa, under the name of Elagabalus,\footnotemark[52]
and under the form of a black conical stone, which, as it was
universally believed, had fallen from heaven on that sacred
place. To this protecting deity, Antoninus, not without some
reason, ascribed his elevation to the throne. The display of
superstitious gratitude was the only serious business of his
reign. The triumph of the god of Emesa over all the religions of
the earth, was the great object of his zeal and vanity; and the
appellation of Elagabalus (for he presumed as pontiff and
favorite to adopt that sacred name) was dearer to him than all
the titles of Imperial greatness. In a solemn procession through
the streets of Rome, the way was strewed with gold dust; the
black stone, set in precious gems, was placed on a chariot drawn
by six milk-white horses richly caparisoned. The pious emperor
held the reins, and, supported by his ministers, moved slowly
backwards, that he might perpetually enjoy the felicity of the
divine presence. In a magnificent temple raised on the Palatine
Mount, the sacrifices of the god Elagabalus were celebrated with
every circumstance of cost and solemnity. The richest wines, the
most extraordinary victims, and the rarest aromatics, were
profusely consumed on his altar. Around the altar, a chorus of
Syrian damsels performed their lascivious dances to the sound of
barbarian music, whilst the gravest personages of the state and
army, clothed in long Phœnician tunics, officiated in the meanest
functions, with affected zeal and secret indignation.\footnotemark[53]

\footnotetext[52]{This name is derived by the learned from two Syrian
words, Ela a God, and Gabal, to form, the forming or plastic god,
a proper, and even happy epithet for the sun. Wotton’s History of
Rome, p. 378 Note: The name of Elagabalus has been disfigured in
various ways. Herodian calls him; Lampridius, and the more modern
writers, make him Heliogabalus. Dion calls him Elegabalus; but
Elegabalus was the true name, as it appears on the medals.
(Eckhel. de Doct. num. vet. t. vii. p. 250.) As to its etymology,
that which Gibbon adduces is given by Bochart, Chan. ii. 5; but
Salmasius, on better grounds. (not. in Lamprid. in Elagab.,)
derives the name of Elagabalus from the idol of that god,
represented by Herodian and the medals in the form of a mountain,
(gibel in Hebrew,) or great stone cut to a point, with marks
which represent the sun. As it was not permitted, at Hierapolis,
in Syria, to make statues of the sun and moon, because, it was
said, they are themselves sufficiently visible, the sun was
represented at Emesa in the form of a great stone, which, as it
appeared, had fallen from heaven. Spanheim, Cæsar. notes, p.
46.—G. The name of Elagabalus, in “nummis rarius legetur.”
Rasche, Lex. Univ. Ref. Numm. Rasche quotes two.—M}

\footnotetext[53]{Herodian, l. v. p. 190.}

