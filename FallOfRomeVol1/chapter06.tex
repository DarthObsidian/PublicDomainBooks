\chapter{Death Of Severus, Tyranny Of Caracalla, Usurpation Of Marcinus.}
\section{Part \thesection.}

\textit{The Death Of Severus. — Tyranny Of Caracalla. — Usurpation Of
Macrinus. — Follies Of Elagabalus. — Virtues Of Alexander
Severus. — Licentiousness Of The Army. — General State Of The Roman
Finances.}
\vspace{\onelineskip}

The ascent to greatness, however steep and dangerous, may
entertain an active spirit with the consciousness and exercise of
its own powers: but the possession of a throne could never yet
afford a lasting satisfaction to an ambitious mind. This
melancholy truth was felt and acknowledged by Severus. Fortune
and merit had, from an humble station, elevated him to the first
place among mankind. “He had been all things,” as he said
himself, “and all was of little value.”\textsuperscript{1} Distracted with the
care, not of acquiring, but of preserving an empire, oppressed
with age and infirmities, careless of fame,\textsuperscript{2} and satiated with
power, all his prospects of life were closed. The desire of
perpetuating the greatness of his family was the only remaining
wish of his ambition and paternal tenderness.

\pagenote[1]{Hist. August. p. 71. “Omnia fui, et nihil expedit.”}

\pagenote[2]{Dion Cassius, l. lxxvi. p. 1284.}

Like most of the Africans, Severus was passionately addicted to
the vain studies of magic and divination, deeply versed in the
interpretation of dreams and omens, and perfectly acquainted with
the science of judicial astrology; which, in almost every age
except the present, has maintained its dominion over the mind of
man. He had lost his first wife, while he was governor of the
Lionnese Gaul.\textsuperscript{3} In the choice of a second, he sought only to
connect himself with some favorite of fortune; and as soon as he
had discovered that the young lady of Emesa in Syria had \textit{a royal
nativity}, he solicited and obtained her hand.\textsuperscript{4} Julia Domna (for
that was her name) deserved all that the stars could promise her.

She possessed, even in advanced age, the attractions of beauty,\textsuperscript{5}
and united to a lively imagination a firmness of mind, and
strength of judgment, seldom bestowed on her sex. Her amiable
qualities never made any deep impression on the dark and jealous
temper of her husband; but in her son’s reign, she administered
the principal affairs of the empire, with a prudence that
supported his authority, and with a moderation that sometimes
corrected his wild extravagancies.\textsuperscript{6} Julia applied herself to
letters and philosophy, with some success, and with the most
splendid reputation. She was the patroness of every art, and the
friend of every man of genius.\textsuperscript{7} The grateful flattery of the
learned has celebrated her virtues; but, if we may credit the
scandal of ancient history, chastity was very far from being the
most conspicuous virtue of the empress Julia.\textsuperscript{8}

\pagenote[3]{About the year 186. M. de Tillemont is miserably
embarrassed with a passage of Dion, in which the empress
Faustina, who died in the year 175, is introduced as having
contributed to the marriage of Severus and Julia, (l. lxxiv. p.
1243.) The learned compiler forgot that Dion is relating not a
real fact, but a dream of Severus; and dreams are circumscribed
to no limits of time or space. Did M. de Tillemont imagine that
marriages were consummated in the temple of Venus at Rome? Hist.
des Empereurs, tom. iii. p. 389. Note 6.}

\pagenote[4]{Hist. August. p. 65.}

\pagenote[5]{Hist. August. p. 5.}

\pagenote[6]{Dion Cassius, l. lxxvii. p. 1304, 1314.}

\pagenote[7]{See a dissertation of Menage, at the end of his
edition of Diogenes Lærtius, de Fœminis Philosophis.}

\pagenote[8]{Dion, l. lxxvi. p. 1285. Aurelius Victor.}

Two sons, Caracalla\textsuperscript{9} and Geta, were the fruit of this marriage,
and the destined heirs of the empire. The fond hopes of the
father, and of the Roman world, were soon disappointed by these
vain youths, who displayed the indolent security of hereditary
princes; and a presumption that fortune would supply the place of
merit and application. Without any emulation of virtue or
talents, they discovered, almost from their infancy, a fixed and
implacable antipathy for each other.

\pagenote[9]{Bassianus was his first name, as it had been that of
his maternal grandfather. During his reign, he assumed the
appellation of Antoninus, which is employed by lawyers and
ancient historians. After his death, the public indignation
loaded him with the nicknames of Tarantus and Caracalla. The
first was borrowed from a celebrated Gladiator, the second from a
long Gallic gown which he distributed to the people of Rome.}

Their aversion, confirmed by years, and fomented by the arts of
their interested favorites, broke out in childish, and gradually
in more serious competitions; and, at length, divided the
theatre, the circus, and the court, into two factions, actuated
by the hopes and fears of their respective leaders. The prudent
emperor endeavored, by every expedient of advice and authority,
to allay this growing animosity. The unhappy discord of his sons
clouded all his prospects, and threatened to overturn a throne
raised with so much labor, cemented with so much blood, and
guarded with every defence of arms and treasure. With an
impartial hand he maintained between them an exact balance of
favor, conferred on both the rank of Augustus, with the revered
name of Antoninus; and for the first time the Roman world beheld
three emperors.\textsuperscript{10} Yet even this equal conduct served only to
inflame the contest, whilst the fierce Caracalla asserted the
right of primogeniture, and the milder Geta courted the
affections of the people and the soldiers. In the anguish of a
disappointed father, Severus foretold that the weaker of his sons
would fall a sacrifice to the stronger; who, in his turn, would
be ruined by his own vices.\textsuperscript{11}

\pagenote[10]{The elevation of Caracalla is fixed by the accurate
M. de Tillemont to the year 198; the association of Geta to the
year 208.}

\pagenote[11]{Herodian, l. iii. p. 130. The lives of Caracalla
and Geta, in the Augustan History.}

In these circumstances the intelligence of a war in Britain, and
of an invasion of the province by the barbarians of the North,
was received with pleasure by Severus. Though the vigilance of
his lieutenants might have been sufficient to repel the distant
enemy, he resolved to embrace the honorable pretext of
withdrawing his sons from the luxury of Rome, which enervated
their minds and irritated their passions; and of inuring their
youth to the toils of war and government. Notwithstanding his
advanced age, (for he was above threescore,) and his gout, which
obliged him to be carried in a litter, he transported himself in
person into that remote island, attended by his two sons, his
whole court, and a formidable army. He immediately passed the
walls of Hadrian and Antoninus, and entered the enemy’s country,
with a design of completing the long attempted conquest of
Britain. He penetrated to the northern extremity of the island,
without meeting an enemy. But the concealed ambuscades of the
Caledonians, who hung unseen on the rear and flanks of his army,
the coldness of the climate and the severity of a winter march
across the hills and morasses of Scotland, are reported to have
cost the Romans above fifty thousand men. The Caledonians at
length yielded to the powerful and obstinate attack, sued for
peace, and surrendered a part of their arms, and a large tract of
territory. But their apparent submission lasted no longer than
the present terror. As soon as the Roman legions had retired,
they resumed their hostile independence. Their restless spirit
provoked Severus to send a new army into Caledonia, with the most
bloody orders, not to subdue, but to extirpate the natives. They
were saved by the death of their haughty enemy.\textsuperscript{12}

\pagenote[12]{Dion, l. lxxvi. p. 1280, \&c. Herodian, l. iii. p.
132, \&c.}

This Caledonian war, neither marked by decisive events, nor
attended with any important consequences, would ill deserve our
attention; but it is supposed, not without a considerable degree
of probability, that the invasion of Severus is connected with
the most shining period of the British history or fable. Fingal,
whose fame, with that of his heroes and bards, has been revived
in our language by a recent publication, is said to have
commanded the Caledonians in that memorable juncture, to have
eluded the power of Severus, and to have obtained a signal
victory on the banks of the Carun, in which the son of \textit{the King
of the World}, Caracul, fled from his arms along the fields of
his pride.\textsuperscript{13} Something of a doubtful mist still hangs over these
Highland traditions; nor can it be entirely dispelled by the most
ingenious researches of modern criticism;\textsuperscript{14} but if we could,
with safety, indulge the pleasing supposition, that Fingal lived,
and that Ossian sung, the striking contrast of the situation and
manners of the contending nations might amuse a philosophic mind.

The parallel would be little to the advantage of the more
civilized people, if we compared the unrelenting revenge of
Severus with the generous clemency of Fingal; the timid and
brutal cruelty of Caracalla with the bravery, the tenderness, the
elegant genius of Ossian; the mercenary chiefs, who, from motives
of fear or interest, served under the imperial standard, with the
free-born warriors who started to arms at the voice of the king
of Morven; if, in a word, we contemplated the untutored
Caledonians, glowing with the warm virtues of nature, and the
degenerate Romans, polluted with the mean vices of wealth and
slavery.

\pagenote[13]{Ossian’s Poems, vol. i. p. 175.}

\pagenote[14]{That the Caracul of Ossian is the Caracalla of the
Roman History, is, perhaps, the only point of British antiquity
in which Mr. Macpherson and Mr. Whitaker are of the same opinion;
and yet the opinion is not without difficulty. In the Caledonian
war, the son of Severus was known only by the appellation of
Antoninus, and it may seem strange that the Highland bard should
describe him by a nickname, invented four years afterwards,
scarcely used by the Romans till after the death of that emperor,
and seldom employed by the most ancient historians. See Dion, l.
lxxvii. p. 1317. Hist. August. p. 89 Aurel. Victor. Euseb. in
Chron. ad ann. 214. Note: The historical authority of
Macpherson’s Ossian has not increased since Gibbon wrote. We may,
indeed, consider it exploded. Mr. Whitaker, in a letter to Gibbon
(Misc. Works, vol. ii. p. 100,) attempts, not very successfully,
to weaken this objection of the historian.—M.}

The declining health and last illness of Severus inflamed the
wild ambition and black passions of Caracalla’s soul. Impatient
of any delay or division of empire, he attempted, more than once,
to shorten the small remainder of his father’s days, and
endeavored, but without success, to excite a mutiny among the
troops.\textsuperscript{15} The old emperor had often censured the misguided
lenity of Marcus, who, by a single act of justice, might have
saved the Romans from the tyranny of his worthless son. Placed in
the same situation, he experienced how easily the rigor of a
judge dissolves away in the tenderness of a parent. He
deliberated, he threatened, but he could not punish; and this
last and only instance of mercy was more fatal to the empire than
a long series of cruelty.\textsuperscript{16} The disorder of his mind irritated
the pains of his body; he wished impatiently for death, and
hastened the instant of it by his impatience. He expired at York
in the sixty-fifth year of his life, and in the eighteenth of a
glorious and successful reign. In his last moments he recommended
concord to his sons, and his sons to the army. The salutary
advice never reached the heart, or even the understanding, of the
impetuous youths; but the more obedient troops, mindful of their
oath of allegiance, and of the authority of their deceased
master, resisted the solicitations of Caracalla, and proclaimed
both brothers emperors of Rome. The new princes soon left the
Caledonians in peace, returned to the capital, celebrated their
father’s funeral with divine honors, and were cheerfully
acknowledged as lawful sovereigns, by the senate, the people, and
the provinces. Some preeminence of rank seems to have been
allowed to the elder brother; but they both administered the
empire with equal and independent power.\textsuperscript{17}

\pagenote[15]{Dion, l. lxxvi. p. 1282. Hist. August. p. 71.
Aurel. Victor.}

\pagenote[16]{Dion, l. lxxvi. p. 1283. Hist. August. p. 89}

\pagenote[17]{Footnote 17: Dion, l. lxxvi. p. 1284. Herodian, l.
iii. p. 135.}

Such a divided form of government would have proved a source of
discord between the most affectionate brothers. It was impossible
that it could long subsist between two implacable enemies, who
neither desired nor could trust a reconciliation. It was visible
that one only could reign, and that the other must fall; and each
of them, judging of his rival’s designs by his own, guarded his
life with the most jealous vigilance from the repeated attacks of
poison or the sword. Their rapid journey through Gaul and Italy,
during which they never ate at the same table, or slept in the
same house, displayed to the provinces the odious spectacle of
fraternal discord. On their arrival at Rome, they immediately
divided the vast extent of the imperial palace.\textsuperscript{18} No
communication was allowed between their apartments; the doors and
passages were diligently fortified, and guards posted and
relieved with the same strictness as in a besieged place. The
emperors met only in public, in the presence of their afflicted
mother; and each surrounded by a numerous train of armed
followers. Even on these occasions of ceremony, the dissimulation
of courts could ill disguise the rancor of their hearts.\textsuperscript{19}

\pagenote[18]{Mr. Hume is justly surprised at a passage of
Herodian, (l. iv. p. 139,) who, on this occasion, represents the
Imperial palace as equal in extent to the rest of Rome. The whole
region of the Palatine Mount, on which it was built, occupied, at
most, a circumference of eleven or twelve thousand feet, (see the
Notitia and Victor, in Nardini’s Roma Antica.) But we should
recollect that the opulent senators had almost surrounded the
city with their extensive gardens and suburb palaces, the
greatest part of which had been gradually confiscated by the
emperors. If Geta resided in the gardens that bore his name on
the Janiculum, and if Caracalla inhabited the gardens of Mæcenas
on the Esquiline, the rival brothers were separated from each
other by the distance of several miles; and yet the intermediate
space was filled by the Imperial gardens of Sallust, of Lucullus,
of Agrippa, of Domitian, of Caius, \&c., all skirting round the
city, and all connected with each other, and with the palace, by
bridges thrown over the Tiber and the streets. But this
explanation of Herodian would require, though it ill deserves, a
particular dissertation, illustrated by a map of ancient Rome.
(Hume, Essay on Populousness of Ancient Nations.—M.)}

\pagenote[19]{Herodian, l. iv. p. 139}

This latent civil war already distracted the whole government,
when a scheme was suggested that seemed of mutual benefit to the
hostile brothers. It was proposed, that since it was impossible
to reconcile their minds, they should separate their interest,
and divide the empire between them. The conditions of the treaty
were already drawn with some accuracy. It was agreed that
Caracalla, as the elder brother should remain in possession of
Europe and the western Africa; and that he should relinquish the
sovereignty of Asia and Egypt to Geta, who might fix his
residence at Alexandria or Antioch, cities little inferior to
Rome itself in wealth and greatness; that numerous armies should
be constantly encamped on either side of the Thracian Bosphorus,
to guard the frontiers of the rival monarchies; and that the
senators of European extraction should acknowledge the sovereign
of Rome, whilst the natives of Asia followed the emperor of the
East. The tears of the empress Julia interrupted the negotiation,
the first idea of which had filled every Roman breast with
surprise and indignation. The mighty mass of conquest was so
intimately united by the hand of time and policy, that it
required the most forcible violence to rend it asunder. The
Romans had reason to dread, that the disjointed members would
soon be reduced by a civil war under the dominion of one master;
but if the separation was permanent, the division of the
provinces must terminate in the dissolution of an empire whose
unity had hitherto remained inviolate.\textsuperscript{20}

\pagenote[20]{Herodian, l. iv. p. 144.}

Had the treaty been carried into execution, the sovereign of
Europe might soon have been the conqueror of Asia; but Caracalla
obtained an easier, though a more guilty, victory. He artfully
listened to his mother’s entreaties, and consented to meet his
brother in her apartment, on terms of peace and reconciliation.
In the midst of their conversation, some centurions, who had
contrived to conceal themselves, rushed with drawn swords upon
the unfortunate Geta. His distracted mother strove to protect him
in her arms; but, in the unavailing struggle, she was wounded in
the hand, and covered with the blood of her younger son, while
she saw the elder animating and assisting\textsuperscript{21} the fury of the
assassins. As soon as the deed was perpetrated, Caracalla, with
hasty steps, and horror in his countenance, ran towards the
Prætorian camp, as his only refuge, and threw himself on the
ground before the statues of the tutelar deities.\textsuperscript{22} The soldiers
attempted to raise and comfort him. In broken and disordered
words he informed them of his imminent danger, and fortunate
escape; insinuating that he had prevented the designs of his
enemy, and declared his resolution to live and die with his
faithful troops. Geta had been the favorite of the soldiers; but
complaint was useless, revenge was dangerous, and they still
reverenced the son of Severus. Their discontent died away in idle
murmurs, and Caracalla soon convinced them of the justice of his
cause, by distributing in one lavish donative the accumulated
treasures of his father’s reign.\textsuperscript{23} The real \textit{sentiments} of the
soldiers alone were of importance to his power or safety. Their
declaration in his favor commanded the dutiful \textit{professions} of
the senate. The obsequious assembly was always prepared to ratify
the decision of fortune;\textsuperscript{231} but as Caracalla wished to assuage
the first emotions of public indignation, the name of Geta was
mentioned with decency, and he received the funeral honors of a
Roman emperor.\textsuperscript{24} Posterity, in pity to his misfortune, has cast
a veil over his vices. We consider that young prince as the
innocent victim of his brother’s ambition, without recollecting
that he himself wanted power, rather than inclination, to
consummate the same attempts of revenge and murder.\textsuperscript{241}

\pagenote[21]{Caracalla consecrated, in the temple of Serapis,
the sword with which, as he boasted, he had slain his brother
Geta. Dion, l. lxxvii p. 1307.}

\pagenote[22]{Herodian, l. iv. p. 147. In every Roman camp there
was a small chapel near the head-quarters, in which the statues
of the tutelar deities were preserved and adored; and we may
remark that the eagles, and other military ensigns, were in the
first rank of these deities; an excellent institution, which
confirmed discipline by the sanction of religion. See Lipsius de
Militia Romana, iv. 5, v. 2.}

\pagenote[23]{Herodian, l. iv. p. 148. Dion, l. lxxvii. p. 1289.}

\pagenote[231]{The account of this transaction, in a new passage
of Dion, varies in some degree from this statement. It adds that
the next morning, in the senate, Antoninus requested their
indulgence, not because he had killed his brother, but because he
was hoarse, and could not address them. Mai. Fragm. p. 228.—M.}

\pagenote[24]{Geta was placed among the gods. Sit divus, dum non
sit vivus said his brother. Hist. August. p. 91. Some marks of
Geta’s consecration are still found upon medals.}

\pagenote[241]{The favorable judgment which history has given of
Geta is not founded solely on a feeling of pity; it is supported
by the testimony of contemporary historians: he was too fond of
the pleasures of the table, and showed great mistrust of his
brother; but he was humane, well instructed; he often endeavored
to mitigate the rigorous decrees of Severus and Caracalla. Herod
iv. 3. Spartian in Geta.—W.}

The crime went not unpunished. Neither business, nor pleasure,
nor flattery, could defend Caracalla from the stings of a guilty
conscience; and he confessed, in the anguish of a tortured mind,
that his disordered fancy often beheld the angry forms of his
father and his brother rising into life, to threaten and upbraid
him.\textsuperscript{25} The consciousness of his crime should have induced him to
convince mankind, by the virtues of his reign, that the bloody
deed had been the involuntary effect of fatal necessity. But the
repentance of Caracalla only prompted him to remove from the
world whatever could remind him of his guilt, or recall the
memory of his murdered brother. On his return from the senate to
the palace, he found his mother in the company of several noble
matrons, weeping over the untimely fate of her younger son. The
jealous emperor threatened them with instant death; the sentence
was executed against Fadilla, the last remaining daughter of the
emperor Marcus;\textsuperscript{251} and even the afflicted Julia was obliged to
silence her lamentations, to suppress her sighs, and to receive
the assassin with smiles of joy and approbation. It was computed
that, under the vague appellation of the friends of Geta, above
twenty thousand persons of both sexes suffered death. His guards
and freedmen, the ministers of his serious business, and the
companions of his looser hours, those who by his interest had
been promoted to any commands in the army or provinces, with the
long connected chain of their dependants, were included in the
proscription; which endeavored to reach every one who had
maintained the smallest correspondence with Geta, who lamented
his death, or who even mentioned his name.\textsuperscript{26} Helvius Pertinax,
son to the prince of that name, lost his life by an unseasonable
witticism.\textsuperscript{27} It was a sufficient crime of Thrasea Priscus to be
descended from a family in which the love of liberty seemed an
hereditary quality.\textsuperscript{28} The particular causes of calumny and
suspicion were at length exhausted; and when a senator was
accused of being a secret enemy to the government, the emperor
was satisfied with the general proof that he was a man of
property and virtue. From this well-grounded principle he
frequently drew the most bloody inferences.\textsuperscript{281}

\pagenote[25]{Dion, l. lxxvii. p. 1307}

\pagenote[251]{The most valuable paragraph of dion, which the
industry of M. Manas recovered, relates to this daughter of
Marcus, executed by Caracalla. Her name, as appears from Fronto,
as well as from Dion, was Cornificia. When commanded to choose
the kind of death she was to suffer, she burst into womanish
tears; but remembering her father Marcus, she thus spoke:—“O my
hapless soul, (... animula,) now imprisoned in the body, burst
forth! be free! show them, however reluctant to believe it, that
thou art the daughter of Marcus.” She then laid aside all her
ornaments, and preparing herself for death, ordered her veins to
be opened. Mai. Fragm. Vatican ii p. 220.—M.}

\pagenote[26]{Dion, l. lxxvii. p. 1290. Herodian, l. iv. p. 150.
Dion (p. 2298) says, that the comic poets no longer durst employ
the name of Geta in their plays, and that the estates of those
who mentioned it in their testaments were confiscated.}

\pagenote[27]{Caracalla had assumed the names of several
conquered nations; Pertinax observed, that the name of Geticus
(he had obtained some advantage over the Goths, or Getæ) would be
a proper addition to Parthieus, Alemannicus, \&c. Hist. August. p.
89.}

\pagenote[28]{Dion, l. lxxvii. p. 1291. He was probably descended
from Helvidius Priscus, and Thrasea Pætus, those patriots, whose
firm, but useless and unseasonable, virtue has been immortalized
by Tacitus. Note: M. Guizot is indignant at this “cold”
observation of Gibbon on the noble character of Thrasea; but he
admits that his virtue was useless to the public, and
unseasonable amidst the vices of his age.—M.}

\pagenote[281]{Caracalla reproached all those who demanded no
favors of him. “It is clear that if you make me no requests, you
do not trust me; if you do not trust me, you suspect me; if you
suspect me, you fear me; if you fear me, you hate me.” And
forthwith he condemned them as conspirators, a good specimen of
the sorites in a tyrant’s logic. See Fragm. Vatican p.—M.}

\section{Part \thesection.}

The execution of so many innocent citizens was bewailed by the
secret tears of their friends and families. The death of
Papinian, the Prætorian Præfect, was lamented as a public
calamity.\textsuperscript{282} During the last seven years of Severus, he had
exercised the most important offices of the state, and, by his
salutary influence, guided the emperor’s steps in the paths of
justice and moderation. In full assurance of his virtue and
abilities, Severus, on his death-bed, had conjured him to watch
over the prosperity and union of the Imperial family.\textsuperscript{29} The
honest labors of Papinian served only to inflame the hatred which
Caracalla had already conceived against his father’s minister.
After the murder of Geta, the Præfect was commanded to exert the
powers of his skill and eloquence in a studied apology for that
atrocious deed. The philosophic Seneca had condescended to
compose a similar epistle to the senate, in the name of the son
and assassin of Agrippina.\textsuperscript{30} “That it was easier to commit than
to justify a parricide,” was the glorious reply of Papinian;\textsuperscript{31}
who did not hesitate between the loss of life and that of honor.
Such intrepid virtue, which had escaped pure and unsullied from
the intrigues of courts, the habits of business, and the arts of
his profession, reflects more lustre on the memory of Papinian,
than all his great employments, his numerous writings, and the
superior reputation as a lawyer, which he has preserved through
every age of the Roman jurisprudence.\textsuperscript{32}

\pagenote[282]{Papinian was no longer Prætorian Præfect.
Caracalla had deprived him of that office immediately after the
death of Severus. Such is the statement of Dion; and the
testimony of Spartian, who gives Papinian the Prætorian
præfecture till his death, is of little weight opposed to that of
a senator then living at Rome.—W.}

\pagenote[29]{It is said that Papinian was himself a relation of
the empress Julia.}

\pagenote[30]{Tacit. Annal. xiv. 2.}

\pagenote[31]{Hist. August. p. 88.}

\pagenote[32]{With regard to Papinian, see Heineccius’s Historia
Juris Roma ni, l. 330, \&c.}

It had hitherto been the peculiar felicity of the Romans, and in
the worst of times the consolation, that the virtue of the
emperors was active, and their vice indolent. Augustus, Trajan,
Hadrian, and Marcus visited their extensive dominions in person,
and their progress was marked by acts of wisdom and beneficence.
The tyranny of Tiberius, Nero, and Domitian, who resided almost
constantly at Rome, or in the adjacent was confined to the
senatorial and equestrian orders.\textsuperscript{33} But Caracalla was the common
enemy of mankind. He left the capital (and he never returned to
it) about a year after the murder of Geta. The rest of his reign
was spent in the several provinces of the empire, particularly
those of the East, and every province was by turns the scene of
his rapine and cruelty. The senators, compelled by fear to attend
his capricious motions, were obliged to provide daily
entertainments at an immense expense, which he abandoned with
contempt to his guards; and to erect, in every city, magnificent
palaces and theatres, which he either disdained to visit, or
ordered immediately thrown down. The most wealthy families were
ruined by partial fines and confiscations, and the great body of
his subjects oppressed by ingenious and aggravated taxes.\textsuperscript{34} In
the midst of peace, and upon the slightest provocation, he issued
his commands, at Alexandria, in Egypt for a general massacre.
From a secure post in the temple of Serapis, he viewed and
directed the slaughter of many thousand citizens, as well as
strangers, without distinguishing the number or the crime of the
sufferers; since as he coolly informed the senate, \textit{all} the
Alexandrians, those who had perished, and those who had escaped,
were alike guilty.\textsuperscript{35}

\pagenote[33]{Tiberius and Domitian never moved from the
neighborhood of Rome. Nero made a short journey into Greece. “Et
laudatorum Principum usus ex æquo, quamvis procul agentibus. Sævi
proximis ingruunt.” Tacit. Hist. iv. 74.}

\pagenote[34]{Dion, l. lxxvii. p. 1294.}

\pagenote[35]{Dion, l. lxxvii. p. 1307. Herodian, l. iv. p. 158.
The former represents it as a cruel massacre, the latter as a
perfidious one too. It seems probable that the Alexandrians has
irritated the tyrant by their railleries, and perhaps by their
tumults. * Note: After these massacres, Caracalla also deprived
the Alexandrians of their spectacles and public feasts; he
divided the city into two parts by a wall with towers at
intervals, to prevent the peaceful communications of the
citizens. Thus was treated the unhappy Alexandria, says Dion, by
the savage beast of Ausonia. This, in fact, was the epithet which
the oracle had applied to him; it is said, indeed, that he was
much pleased with the name and often boasted of it. Dion, lxxvii.
p. 1307.—G.}

The wise instructions of Severus never made any lasting
impression on the mind of his son, who, although not destitute of
imagination and eloquence, was equally devoid of judgment and
humanity.\textsuperscript{36} One dangerous maxim, worthy of a tyrant, was
remembered and abused by Caracalla. “To secure the affections of
the army, and to esteem the rest of his subjects as of little
moment.”\textsuperscript{37} But the liberality of the father had been restrained
by prudence, and his indulgence to the troops was tempered by
firmness and authority. The careless profusion of the son was the
policy of one reign, and the inevitable ruin both of the army and
of the empire. The vigor of the soldiers, instead of being
confirmed by the severe discipline of camps, melted away in the
luxury of cities. The excessive increase of their pay and
donatives\textsuperscript{38} exhausted the state to enrich the military order,
whose modesty in peace, and service in war, is best secured by an
honorable poverty. The demeanor of Caracalla was haughty and full
of pride; but with the troops he forgot even the proper dignity
of his rank, encouraged their insolent familiarity, and,
neglecting the essential duties of a general, affected to imitate
the dress and manners of a common soldier.

\pagenote[36]{Dion, l. lxxvii. p. 1296.}

\pagenote[37]{Dion, l. lxxvi. p. 1284. Mr. Wotton (Hist. of Rome,
p. 330) suspects that this maxim was invented by Caracalla
himself, and attributed to his father.}

\pagenote[38]{Dion (l. lxxviii. p. 1343) informs us that the
extraordinary gifts of Caracalla to the army amounted annually to
seventy millions of drachmæ (about two millions three hundred and
fifty thousand pounds.) There is another passage in Dion,
concerning the military pay, infinitely curious, were it not
obscure, imperfect, and probably corrupt. The best sense seems to
be, that the Prætorian guards received twelve hundred and fifty
drachmæ, (forty pounds a year,) (Dion, l. lxxvii. p. 1307.) Under
the reign of Augustus, they were paid at the rate of two drachmæ,
or denarii, per day, 720 a year, (Tacit. Annal. i. 17.) Domitian,
who increased the soldiers’ pay one fourth, must have raised the
Prætorians to 960 drachmæ, (Gronoviue de Pecunia Veteri, l. iii.
c. 2.) These successive augmentations ruined the empire; for,
with the soldiers’ pay, their numbers too were increased. We have
seen the Prætorians alone increased from 10,000 to 50,000 men.
Note: Valois and Reimar have explained in a very simple and
probable manner this passage of Dion, which Gibbon seems to me
not to have understood. He ordered that the soldiers should
receive, as the reward of their services the Prætorians 1250
drachms, the other 5000 drachms. Valois thinks that the numbers
have been transposed, and that Caracalla added 5000 drachms to
the donations made to the Prætorians, 1250 to those of the
legionaries. The Prætorians, in fact, always received more than
the others. The error of Gibbon arose from his considering that
this referred to the annual pay of the soldiers, while it relates
to the sum they received as a reward for their services on their
discharge: donatives means recompense for service. Augustus had
settled that the Prætorians, after sixteen campaigns, should
receive 5000 drachms: the legionaries received only 3000 after
twenty years. Caracalla added 5000 drachms to the donative of the
Prætorians, 1250 to that of the legionaries. Gibbon appears to
have been mistaken both in confounding this donative on discharge
with the annual pay, and in not paying attention to the remark of
Valois on the transposition of the numbers in the text.—G}

It was impossible that such a character, and such conduct as that
of Caracalla, could inspire either love or esteem; but as long as
his vices were beneficial to the armies, he was secure from the
danger of rebellion. A secret conspiracy, provoked by his own
jealousy, was fatal to the tyrant. The Prætorian præfecture was
divided between two ministers. The military department was
intrusted to Adventus, an experienced rather than able soldier;
and the civil affairs were transacted by Opilius Macrinus, who,
by his dexterity in business, had raised himself, with a fair
character, to that high office. But his favor varied with the
caprice of the emperor, and his life might depend on the
slightest suspicion, or the most casual circumstance. Malice or
fanaticism had suggested to an African, deeply skilled in the
knowledge of futurity, a very dangerous prediction, that Macrinus
and his son were destined to reign over the empire. The report
was soon diffused through the province; and when the man was sent
in chains to Rome, he still asserted, in the presence of the
præfect of the city, the faith of his prophecy. That magistrate,
who had received the most pressing instructions to inform himself
of the \textit{successors} of Caracalla, immediately communicated the
examination of the African to the Imperial court, which at that
time resided in Syria. But, notwithstanding the diligence of the
public messengers, a friend of Macrinus found means to apprise
him of the approaching danger. The emperor received the letters
from Rome; and as he was then engaged in the conduct of a chariot
race, he delivered them unopened to the Prætorian Præfect,
directing him to despatch the ordinary affairs, and to report the
more important business that might be contained in them. Macrinus
read his fate, and resolved to prevent it. He inflamed the
discontents of some inferior officers, and employed the hand of
Martialis, a desperate soldier, who had been refused the rank of
centurion. The devotion of Caracalla prompted him to make a
pilgrimage from Edessa to the celebrated temple of the Moon at
Carrhæ.\textsuperscript{381} He was attended by a body of cavalry: but having
stopped on the road for some necessary occasion, his guards
preserved a respectful distance, and Martialis, approaching his
person under a presence of duty, stabbed him with a dagger. The
bold assassin was instantly killed by a Scythian archer of the
Imperial guard. Such was the end of a monster whose life
disgraced human nature, and whose reign accused the patience of
the Romans.\textsuperscript{39} The grateful soldiers forgot his vices, remembered
only his partial liberality, and obliged the senate to prostitute
their own dignity and that of religion, by granting him a place
among the gods. Whilst he was upon earth, Alexander the Great was
the only hero whom this god deemed worthy his admiration. He
assumed the name and ensigns of Alexander, formed a Macedonian
phalanx of guards, persecuted the disciples of Aristotle, and
displayed, with a puerile enthusiasm, the only sentiment by which
he discovered any regard for virtue or glory. We can easily
conceive, that after the battle of Narva, and the conquest of
Poland, Charles XII. (though he still wanted the more elegant
accomplishments of the son of Philip) might boast of having
rivalled his valor and magnanimity; but in no one action of his
life did Caracalla express the faintest resemblance of the
Macedonian hero, except in the murder of a great number of his
own and of his father’s friends.\textsuperscript{40}

\pagenote[381]{Carrhæ, now Harran, between Edessan and Nisibis,
famous for the defeat of Crassus—the Haran from whence Abraham
set out for the land of Canaan. This city has always been
remarkable for its attachment to Sabaism—G}

\pagenote[39]{Dion, l. lxxviii. p. 1312. Herodian, l. iv. p.
168.}

\pagenote[40]{The fondness of Caracalla for the name and ensigns
of Alexander is still preserved on the medals of that emperor.
See Spanheim, de Usu Numismatum, Dissertat. xii. Herodian (l. iv.
p. 154) had seen very ridiculous pictures, in which a figure was
drawn with one side of the face like Alexander, and the other
like Caracalla.}

After the extinction of the house of Severus, the Roman world
remained three days without a master. The choice of the army (for
the authority of a distant and feeble senate was little regarded)
hung in anxious suspense, as no candidate presented himself whose
distinguished birth and merit could engage their attachment and
unite their suffrages. The decisive weight of the Prætorian
guards elevated the hopes of their præfects, and these powerful
ministers began to assert their \textit{legal} claim to fill the vacancy
of the Imperial throne. Adventus, however, the senior præfect,
conscious of his age and infirmities, of his small reputation,
and his smaller abilities, resigned the dangerous honor to the
crafty ambition of his colleague Macrinus, whose well-dissembled
grief removed all suspicion of his being accessary to his
master’s death.\textsuperscript{41} The troops neither loved nor esteemed his
character. They cast their eyes around in search of a competitor,
and at last yielded with reluctance to his promises of unbounded
liberality and indulgence. A short time after his accession, he
conferred on his son Diadumenianus, at the age of only ten years,
the Imperial title, and the popular name of Antoninus. The
beautiful figure of the youth, assisted by an additional
donative, for which the ceremony furnished a pretext, might
attract, it was hoped, the favor of the army, and secure the
doubtful throne of Macrinus.

\pagenote[41]{Herodian, l. iv. p. 169. Hist. August. p. 94.}

The authority of the new sovereign had been ratified by the
cheerful submission of the senate and provinces. They exulted in
their unexpected deliverance from a hated tyrant, and it seemed
of little consequence to examine into the virtues of the
successor of Caracalla. But as soon as the first transports of
joy and surprise had subsided, they began to scrutinize the
merits of Macrinus with a critical severity, and to arraign the
nasty choice of the army. It had hitherto been considered as a
fundamental maxim of the constitution, that the emperor must be
always chosen in the senate, and the sovereign power, no longer
exercised by the whole body, was always delegated to one of its
members. But Macrinus was not a senator.\textsuperscript{42} The sudden elevation
of the Prætorian præfects betrayed the meanness of their origin;
and the equestrian order was still in possession of that great
office, which commanded with arbitrary sway the lives and
fortunes of the senate. A murmur of indignation was heard, that a
man, whose obscure\textsuperscript{43} extraction had never been illustrated by
any signal service, should dare to invest himself with the
purple, instead of bestowing it on some distinguished senator,
equal in birth and dignity to the splendor of the Imperial
station. As soon as the character of Macrinus was surveyed by the
sharp eye of discontent, some vices, and many defects, were
easily discovered. The choice of his ministers was in many
instances justly censured, and the dissatisfied people, with
their usual candor, accused at once his indolent tameness and his
excessive severity.\textsuperscript{44}

\pagenote[42]{Dion, l. lxxxviii. p. 1350. Elagabalus reproached
his predecessor with daring to seat himself on the throne;
though, as Prætorian præfect, he could not have been admitted
into the senate after the voice of the crier had cleared the
house. The personal favor of Plautianus and Sejanus had broke
through the established rule. They rose, indeed, from the
equestrian order; but they preserved the præfecture, with the
rank of senator and even with the annulship.}

\pagenote[43]{He was a native of Cæsarea, in Numidia, and began
his fortune by serving in the household of Plautian, from whose
ruin he narrowly escaped. His enemies asserted that he was born a
slave, and had exercised, among other infamous professions, that
of Gladiator. The fashion of aspersing the birth and condition of
an adversary seems to have lasted from the time of the Greek
orators to the learned grammarians of the last age.}

\pagenote[44]{Both Dion and Herodian speak of the virtues and
vices of Macrinus with candor and impartiality; but the author of
his life, in the Augustan History, seems to have implicitly
copied some of the venal writers, employed by Elagabalus, to
blacken the memory of his predecessor.}

His rash ambition had climbed a height where it was difficult to
stand with firmness, and impossible to fall without instant
destruction. Trained in the arts of courts and the forms of civil
business, he trembled in the presence of the fierce and
undisciplined multitude, over whom he had assumed the command;
his military talents were despised, and his personal courage
suspected; a whisper that circulated in the camp, disclosed the
fatal secret of the conspiracy against the late emperor,
aggravated the guilt of murder by the baseness of hypocrisy, and
heightened contempt by detestation. To alienate the soldiers, and
to provoke inevitable ruin, the character of a reformer was only
wanting; and such was the peculiar hardship of his fate, that
Macrinus was compelled to exercise that invidious office. The
prodigality of Caracalla had left behind it a long train of ruin
and disorder; and if that worthless tyrant had been capable of
reflecting on the sure consequences of his own conduct, he would
perhaps have enjoyed the dark prospect of the distress and
calamities which he bequeathed to his successors.

In the management of this necessary reformation, Macrinus
proceeded with a cautious prudence, which would have restored
health and vigor to the Roman army in an easy and almost
imperceptible manner. To the soldiers already engaged in the
service, he was constrained to leave the dangerous privileges and
extravagant pay given by Caracalla; but the new recruits were
received on the more moderate though liberal establishment of
Severus, and gradually formed to modesty and obedience.\textsuperscript{45} One
fatal error destroyed the salutary effects of this judicious
plan. The numerous army, assembled in the East by the late
emperor, instead of being immediately dispersed by Macrinus
through the several provinces, was suffered to remain united in
Syria, during the winter that followed his elevation. In the
luxurious idleness of their quarters, the troops viewed their
strength and numbers, communicated their complaints, and revolved
in their minds the advantages of another revolution. The
veterans, instead of being flattered by the advantageous
distinction, were alarmed by the first steps of the emperor,
which they considered as the presage of his future intentions.
The recruits, with sullen reluctance, entered on a service, whose
labors were increased while its rewards were diminished by a
covetous and unwarlike sovereign. The murmurs of the army swelled
with impunity into seditious clamors; and the partial mutinies
betrayed a spirit of discontent and disaffection that waited only
for the slightest occasion to break out on every side into a
general rebellion. To minds thus disposed, the occasion soon
presented itself.

\pagenote[45]{Dion, l. lxxxiii. p. 1336. The sense of the author
is as the intention of the emperor; but Mr. Wotton has mistaken
both, by understanding the distinction, not of veterans and
recruits, but of old and new legions. History of Rome, p. 347.}

The empress Julia had experienced all the vicissitudes of
fortune. From an humble station she had been raised to greatness,
only to taste the superior bitterness of an exalted rank. She was
doomed to weep over the death of one of her sons, and over the
life of the other. The cruel fate of Caracalla, though her good
sense must have long taught her to expect it, awakened the
feelings of a mother and of an empress. Notwithstanding the
respectful civility expressed by the usurper towards the widow of
Severus, she descended with a painful struggle into the condition
of a subject, and soon withdrew herself, by a voluntary death,
from the anxious and humiliating dependence.\textsuperscript{46} \textsuperscript{461} Julia Mæsa,
her sister, was ordered to leave the court and Antioch. She
retired to Emesa with an immense fortune, the fruit of twenty
years’ favor accompanied by her two daughters, Soæmias and Mamæ,
each of whom was a widow, and each had an only son. Bassianus,\textsuperscript{462}
for that was the name of the son of Soæmias, was consecrated
to the honorable ministry of high priest of the Sun; and this
holy vocation, embraced either from prudence or superstition,
contributed to raise the Syrian youth to the empire of Rome. A
numerous body of troops was stationed at Emesa; and as the severe
discipline of Macrinus had constrained them to pass the winter
encamped, they were eager to revenge the cruelty of such
unaccustomed hardships. The soldiers, who resorted in crowds to
the temple of the Sun, beheld with veneration and delight the
elegant dress and figure of the young pontiff; they recognized,
or they thought that they recognized, the features of Caracalla,
whose memory they now adored. The artful Mæsa saw and cherished
their rising partiality, and readily sacrificing her daughter’s
reputation to the fortune of her grandson, she insinuated that
Bassianus was the natural son of their murdered sovereign. The
sums distributed by her emissaries with a lavish hand silenced
every objection, and the profusion sufficiently proved the
affinity, or at least the resemblance, of Bassianus with the
great original. The young Antoninus (for he had assumed and
polluted that respectable name) was declared emperor by the
troops of Emesa, asserted his hereditary right, and called aloud
on the armies to follow the standard of a young and liberal
prince, who had taken up arms to revenge his father’s death and
the oppression of the military order.\textsuperscript{47}

\pagenote[46]{Dion, l. lxxviii. p. 1330. The abridgment of
Xiphilin, though less particular, is in this place clearer than
the original.}

\pagenote[461]{As soon as this princess heard of the death of
Caracalla, she wished to starve herself to death: the respect
shown to her by Macrinus, in making no change in her attendants
or her court, induced her to prolong her life. But it appears, as
far as the mutilated text of Dion and the imperfect epitome of
Xiphilin permit us to judge, that she conceived projects of
ambition, and endeavored to raise herself to the empire. She
wished to tread in the steps of Semiramis and Nitocris, whose
country bordered on her own. Macrinus sent her an order
immediately to leave Antioch, and to retire wherever she chose.
She returned to her former purpose, and starved herself to
death.—G.}

\pagenote[462]{He inherited this name from his great-grandfather
of the mother’s side, Bassianus, father of Julia Mæsa, his
grandmother, and of Julia Domna, wife of Severus. Victor (in his
epitome) is perhaps the only historian who has given the key to
this genealogy, when speaking of Caracalla. His Bassianus ex avi
materni nomine dictus. Caracalla, Elagabalus, and Alexander
Seyerus, bore successively this name.—G.}

\pagenote[47]{According to Lampridius, (Hist. August. p. 135,)
Alexander Severus lived twenty-nine years three months and seven
days. As he was killed March 19, 235, he was born December 12,
205 and was consequently about this time thirteen years old, as
his elder cousin might be about seventeen. This computation suits
much better the history of the young princes than that of
Herodian, (l. v. p. 181,) who represents them as three years
younger; whilst, by an opposite error of chronology, he lengthens
the reign of Elagabalus two years beyond its real duration. For
the particulars of the conspiracy, see Dion, l. lxxviii. p. 1339.
Herodian, l. v. p. 184.}

Whilst a conspiracy of women and eunuchs was concerted with
prudence, and conducted with rapid vigor, Macrinus, who, by a
decisive motion, might have crushed his infant enemy, floated
between the opposite extremes of terror and security, which alike
fixed him inactive at Antioch. A spirit of rebellion diffused
itself through all the camps and garrisons of Syria, successive
detachments murdered their officers,\textsuperscript{48} and joined the party of
the rebels; and the tardy restitution of military pay and
privileges was imputed to the acknowledged weakness of Macrinus.
At length he marched out of Antioch, to meet the increasing and
zealous army of the young pretender. His own troops seemed to
take the field with faintness and reluctance; but, in the heat of
the battle,\textsuperscript{49} the Prætorian guards, almost by an involuntary
impulse, asserted the superiority of their valor and discipline.
The rebel ranks were broken; when the mother and grandmother of
the Syrian prince, who, according to their eastern custom, had
attended the army, threw themselves from their covered chariots,
and, by exciting the compassion of the soldiers, endeavored to
animate their drooping courage. Antoninus himself, who, in the
rest of his life, never acted like a man, in this important
crisis of his fate, approved himself a hero, mounted his horse,
and, at the head of his rallied troops, charged sword in hand
among the thickest of the enemy; whilst the eunuch Gannys,\textsuperscript{491}
whose occupations had been confined to female cares and the soft
luxury of Asia, displayed the talents of an able and experienced
general. The battle still raged with doubtful violence, and
Macrinus might have obtained the victory, had he not betrayed his
own cause by a shameful and precipitate flight. His cowardice
served only to protract his life a few days, and to stamp
deserved ignominy on his misfortunes. It is scarcely necessary to
add, that his son Diadumenianus was involved in the same fate.

As soon as the stubborn Prætorians could be convinced that they
fought for a prince who had basely deserted them, they
surrendered to the conqueror: the contending parties of the Roman
army, mingling tears of joy and tenderness, united under the
banners of the imagined son of Caracalla, and the East
acknowledged with pleasure the first emperor of Asiatic
extraction.

\pagenote[48]{By a most dangerous proclamation of the pretended
Antoninus, every soldier who brought in his officer’s head became
entitled to his private estate, as well as to his military
commission.}

\pagenote[49]{Dion, l. lxxviii. p. 1345. Herodian, l. v. p. 186.
The battle was fought near the village of Immæ, about
two-and-twenty miles from Antioch.}

\pagenote[491]{Gannys was not a eunuch. Dion, p. 1355.—W}

The letters of Macrinus had condescended to inform the senate of
the slight disturbance occasioned by an impostor in Syria, and a
decree immediately passed, declaring the rebel and his family
public enemies; with a promise of pardon, however, to such of his
deluded adherents as should merit it by an immediate return to
their duty. During the twenty days that elapsed from the
declaration of the victory of Antoninus (for in so short an
interval was the fate of the Roman world decided,) the capital
and the provinces, more especially those of the East, were
distracted with hopes and fears, agitated with tumult, and
stained with a useless effusion of civil blood, since whosoever
of the rivals prevailed in Syria must reign over the empire. The
specious letters in which the young conqueror announced his
victory to the obedient senate were filled with professions of
virtue and moderation; the shining examples of Marcus and
Augustus, he should ever consider as the great rule of his
administration; and he affected to dwell with pride on the
striking resemblance of his own age and fortunes with those of
Augustus, who in the earliest youth had revenged, by a successful
war, the murder of his father. By adopting the style of Marcus
Aurelius Antoninus, son of Antoninus and grandson of Severus, he
tacitly asserted his hereditary claim to the empire; but, by
assuming the tribunitian and proconsular powers before they had
been conferred on him by a decree of the senate, he offended the
delicacy of Roman prejudice. This new and injudicious violation
of the constitution was probably dictated either by the ignorance
of his Syrian courtiers, or the fierce disdain of his military
followers.\textsuperscript{50}

\pagenote[50]{Dion, l. lxxix. p. 1353.}

As the attention of the new emperor was diverted by the most
trifling amusements, he wasted many months in his luxurious
progress from Syria to Italy, passed at Nicomedia his first
winter after his victory, and deferred till the ensuing summer
his triumphal entry into the capital. A faithful picture,
however, which preceded his arrival, and was placed by his
immediate order over the altar of Victory in the senate house,
conveyed to the Romans the just but unworthy resemblance of his
person and manners. He was drawn in his sacerdotal robes of silk
and gold, after the loose flowing fashion of the Medes and
Phœnicians; his head was covered with a lofty tiara, his numerous
collars and bracelets were adorned with gems of an inestimable
value. His eyebrows were tinged with black, and his cheeks
painted with an artificial red and white.\textsuperscript{51} The grave senators
confessed with a sigh, that, after having long experienced the
stern tyranny of their own countrymen, Rome was at length humbled
beneath the effeminate luxury of Oriental despotism.

\pagenote[51]{Dion, l. lxxix. p. 1363. Herodian, l. v. p. 189.}

The Sun was worshipped at Emesa, under the name of Elagabalus, \textsuperscript{52}
and under the form of a black conical stone, which, as it was
universally believed, had fallen from heaven on that sacred
place. To this protecting deity, Antoninus, not without some
reason, ascribed his elevation to the throne. The display of
superstitious gratitude was the only serious business of his
reign. The triumph of the god of Emesa over all the religions of
the earth, was the great object of his zeal and vanity; and the
appellation of Elagabalus (for he presumed as pontiff and
favorite to adopt that sacred name) was dearer to him than all
the titles of Imperial greatness. In a solemn procession through
the streets of Rome, the way was strewed with gold dust; the
black stone, set in precious gems, was placed on a chariot drawn
by six milk-white horses richly caparisoned. The pious emperor
held the reins, and, supported by his ministers, moved slowly
backwards, that he might perpetually enjoy the felicity of the
divine presence. In a magnificent temple raised on the Palatine
Mount, the sacrifices of the god Elagabalus were celebrated with
every circumstance of cost and solemnity. The richest wines, the
most extraordinary victims, and the rarest aromatics, were
profusely consumed on his altar. Around the altar, a chorus of
Syrian damsels performed their lascivious dances to the sound of
barbarian music, whilst the gravest personages of the state and
army, clothed in long Phœnician tunics, officiated in the meanest
functions, with affected zeal and secret indignation.\textsuperscript{53}

\pagenote[52]{This name is derived by the learned from two Syrian
words, Ela a God, and Gabal, to form, the forming or plastic god,
a proper, and even happy epithet for the sun. Wotton’s History of
Rome, p. 378 Note: The name of Elagabalus has been disfigured in
various ways. Herodian calls him; Lampridius, and the more modern
writers, make him Heliogabalus. Dion calls him Elegabalus; but
Elegabalus was the true name, as it appears on the medals.
(Eckhel. de Doct. num. vet. t. vii. p. 250.) As to its etymology,
that which Gibbon adduces is given by Bochart, Chan. ii. 5; but
Salmasius, on better grounds. (not. in Lamprid. in Elagab.,)
derives the name of Elagabalus from the idol of that god,
represented by Herodian and the medals in the form of a mountain,
(gibel in Hebrew,) or great stone cut to a point, with marks
which represent the sun. As it was not permitted, at Hierapolis,
in Syria, to make statues of the sun and moon, because, it was
said, they are themselves sufficiently visible, the sun was
represented at Emesa in the form of a great stone, which, as it
appeared, had fallen from heaven. Spanheim, Cæsar. notes, p.
46.—G. The name of Elagabalus, in “nummis rarius legetur.”
Rasche, Lex. Univ. Ref. Numm. Rasche quotes two.—M}

\pagenote[53]{Herodian, l. v. p. 190.}

\section{Part \thesection.}

To this temple, as to the common centre of religious worship, the
Imperial fanatic attempted to remove the Ancilia, the Palladium,\textsuperscript{54}
and all the sacred pledges of the faith of Numa. A crowd of
inferior deities attended in various stations the majesty of the
god of Emesa; but his court was still imperfect, till a female of
distinguished rank was admitted to his bed. Pallas had been first
chosen for his consort; but as it was dreaded lest her warlike
terrors might affright the soft delicacy of a Syrian deity, the
Moon, adored by the Africans under the name of Astarte, was
deemed a more suitable companion for the Sun. Her image, with the
rich offerings of her temple as a marriage portion, was
transported with solemn pomp from Carthage to Rome, and the day
of these mystic nuptials was a general festival in the capital
and throughout the empire.\textsuperscript{55}

\pagenote[54]{He broke into the sanctuary of Vesta, and carried
away a statue, which he supposed to be the palladium; but the
vestals boasted that, by a pious fraud, they had imposed a
counterfeit image on the profane intruder. Hist. August., p.
103.}

\pagenote[55]{Dion, l. lxxix. p. 1360. Herodian, l. v. p. 193.
The subjects of the empire were obliged to make liberal presents
to the new married couple; and whatever they had promised during
the life of Elagabalus was carefully exacted under the
administration of Mamæa.}

A rational voluptuary adheres with invariable respect to the
temperate dictates of nature, and improves the gratifications of
sense by social intercourse, endearing connections, and the soft
coloring of taste and the imagination. But Elagabalus, (I speak
of the emperor of that name,) corrupted by his youth, his
country, and his fortune, abandoned himself to the grossest
pleasures with ungoverned fury, and soon found disgust and
satiety in the midst of his enjoyments. The inflammatory powers
of art were summoned to his aid: the confused multitude of women,
of wines, and of dishes, and the studied variety of attitude and
sauces, served to revive his languid appetites. New terms and new
inventions in these sciences, the only ones cultivated and
patronized by the monarch,\textsuperscript{56} signalized his reign, and
transmitted his infamy to succeeding times. A capricious
prodigality supplied the want of taste and elegance; and whilst
Elagabalus lavished away the treasures of his people in the
wildest extravagance, his own voice and that of his flatterers
applauded a spirit of magnificence unknown to the tameness of his
predecessors. To confound the order of seasons and climates,\textsuperscript{57}
to sport with the passions and prejudices of his subjects, and to
subvert every law of nature and decency, were in the number of
his most delicious amusements. A long train of concubines, and a
rapid succession of wives, among whom was a vestal virgin,
ravished by force from her sacred asylum,\textsuperscript{58} were insufficient to
satisfy the impotence of his passions. The master of the Roman
world affected to copy the dress and manners of the female sex,
preferred the distaff to the sceptre, and dishonored the
principal dignities of the empire by distributing them among his
numerous lovers; one of whom was publicly invested with the title
and authority of the emperor’s, or, as he more properly styled
himself, of the empress’s husband.\textsuperscript{59}

\pagenote[56]{The invention of a new sauce was liberally
rewarded; but if it was not relished, the inventor was confined
to eat of nothing else till he had discovered another more
agreeable to the Imperial palate Hist. August. p. 111.}

\pagenote[57]{He never would eat sea-fish except at a great
distance from the sea; he then would distribute vast quantities
of the rarest sorts, brought at an immense expense, to the
peasants of the inland country. Hist. August. p. 109.}

\pagenote[58]{Dion, l. lxxix. p. 1358. Herodian, l. v. p. 192.}

\pagenote[59]{Hierocles enjoyed that honor; but he would have
been supplanted by one Zoticus, had he not contrived, by a
potion, to enervate the powers of his rival, who, being found on
trial unequal to his reputation, was driven with ignominy from
the palace. Dion, l. lxxix. p. 1363, 1364. A dancer was made
præfect of the city, a charioteer præfect of the watch, a barber
præfect of the provisions. These three ministers, with many
inferior officers, were all recommended enormitate membrorum.
Hist. August. p. 105.}

It may seem probable, the vices and follies of Elagabalus have
been adorned by fancy, and blackened by prejudice.\textsuperscript{60} Yet,
confining ourselves to the public scenes displayed before the
Roman people, and attested by grave and contemporary historians,
their inexpressible infamy surpasses that of any other age or
country. The license of an eastern monarch is secluded from the
eye of curiosity by the inaccessible walls of his seraglio. The
sentiments of honor and gallantry have introduced a refinement of
pleasure, a regard for decency, and a respect for the public
opinion, into the modern courts of Europe;\textsuperscript{601} but the corrupt
and opulent nobles of Rome gratified every vice that could be
collected from the mighty conflux of nations and manners. Secure
of impunity, careless of censure, they lived without restraint in
the patient and humble society of their slaves and parasites. The
emperor, in his turn, viewing every rank of his subjects with the
same contemptuous indifference, asserted without control his
sovereign privilege of lust and luxury.

\pagenote[60]{Even the credulous compiler of his life, in the
Augustan History (p. 111) is inclined to suspect that his vices
may have been exaggerated.}

\pagenote[601]{Wenck has justly observed that Gibbon should have
reckoned the influence of Christianity in this great change. In
the most savage times, and the most corrupt courts, since the
introduction of Christianity there have been no Neros or
Domitians, no Commodus or Elagabalus.—M.}

The most worthless of mankind are not afraid to condemn in others
the same disorders which they allow in themselves; and can
readily discover some nice difference of age, character, or
station, to justify the partial distinction. The licentious
soldiers, who had raised to the throne the dissolute son of
Caracalla, blushed at their ignominious choice, and turned with
disgust from that monster, to contemplate with pleasure the
opening virtues of his cousin Alexander, the son of Mamæa. The
crafty Mæsa, sensible that her grandson Elagabalus must
inevitably destroy himself by his own vices, had provided another
and surer support of her family. Embracing a favorable moment of
fondness and devotion, she had persuaded the young emperor to
adopt Alexander, and to invest him with the title of Cæsar, that
his own divine occupations might be no longer interrupted by the
care of the earth. In the second rank that amiable prince soon
acquired the affections of the public, and excited the tyrant’s
jealousy, who resolved to terminate the dangerous competition,
either by corrupting the manners, or by taking away the life, of
his rival. His arts proved unsuccessful; his vain designs were
constantly discovered by his own loquacious folly, and
disappointed by those virtuous and faithful servants whom the
prudence of Mamæa had placed about the person of her son. In a
hasty sally of passion, Elagabalus resolved to execute by force
what he had been unable to compass by fraud, and by a despotic
sentence degraded his cousin from the rank and honors of Cæsar.
The message was received in the senate with silence, and in the
camp with fury. The Prætorian guards swore to protect Alexander,
and to revenge the dishonored majesty of the throne. The tears
and promises of the trembling Elagabalus, who only begged them to
spare his life, and to leave him in the possession of his beloved
Hierocles, diverted their just indignation; and they contented
themselves with empowering their præfects to watch over the
safety of Alexander, and the conduct of the emperor.\textsuperscript{61}

\pagenote[61]{Dion, l. lxxix. p. 1365. Herodian, l. v. p.
195—201. Hist. August. p. 105. The last of the three historians
seems to have followed the best authors in his account of the
revolution.}

It was impossible that such a reconciliation should last, or that
even the mean soul of Elagabalus could hold an empire on such
humiliating terms of dependence. He soon attempted, by a
dangerous experiment, to try the temper of the soldiers. The
report of the death of Alexander, and the natural suspicion that
he had been murdered, inflamed their passions into fury, and the
tempest of the camp could only be appeased by the presence and
authority of the popular youth. Provoked at this new instance of
their affection for his cousin, and their contempt for his
person, the emperor ventured to punish some of the leaders of the
mutiny. His unseasonable severity proved instantly fatal to his
minions, his mother, and himself. Elagabalus was massacred by the
indignant Prætorians, his mutilated corpse dragged through the
streets of the city, and thrown into the Tiber. His memory was
branded with eternal infamy by the senate; the justice of whose
decree has been ratified by posterity.\textsuperscript{62}

\pagenote[62]{The æra of the death of Elagabalus, and of the
accession of Alexander, has employed the learning and ingenuity
of Pagi, Tillemont, Valsecchi, Vignoli, and Torre, bishop of
Adria. The question is most assuredly intricate; but I still
adhere to the authority of Dion, the truth of whose calculations
is undeniable, and the purity of whose text is justified by the
agreement of Xiphilin, Zonaras, and Cedrenus. Elagabalus reigned
three years nine months and four days, from his victory over
Macrinus, and was killed March 10, 222. But what shall we reply
to the medals, undoubtedly genuine, which reckon the fifth year
of his tribunitian power? We shall reply, with the learned
Valsecchi, that the usurpation of Macrinus was annihilated, and
that the son of Caracalla dated his reign from his father’s
death? After resolving this great difficulty, the smaller knots
of this question may be easily untied, or cut asunder. Note: This
opinion of Valsecchi has been triumphantly contested by Eckhel,
who has shown the impossibility of reconciling it with the medals
of Elagabalus, and has given the most satisfactory explanation of
the five tribunates of that emperor. He ascended the throne and
received the tribunitian power the 16th of May, in the year of
Rome 971; and on the 1st January of the next year, 972, he began
a new tribunate, according to the custom established by preceding
emperors. During the years 972, 973, 974, he enjoyed the
tribunate, and commenced his fifth in the year 975, during which
he was killed on the 10th March. Eckhel de Doct. Num. viii. 430
\&c.—G.}

In the room of Elagabalus, his cousin Alexander was raised to the
throne by the Prætorian guards. His relation to the family of
Severus, whose name he assumed, was the same as that of his
predecessor; his virtue and his danger had already endeared him
to the Romans, and the eager liberality of the senate conferred
upon him, in one day, the various titles and powers of the
Imperial dignity.\textsuperscript{63} But as Alexander was a modest and dutiful
youth, of only seventeen years of age, the reins of government
were in the hands of two women, of his mother, Mamæa, and of
Mæsa, his grandmother. After the death of the latter, who
survived but a short time the elevation of Alexander, Mamæa
remained the sole regent of her son and of the empire.

\pagenote[63]{Hist. August. p. 114. By this unusual
precipitation, the senate meant to confound the hopes of
pretenders, and prevent the factions of the armies.}

In every age and country, the wiser, or at least the stronger, of
the two sexes, has usurped the powers of the state, and confined
the other to the cares and pleasures of domestic life. In
hereditary monarchies, however, and especially in those of modern
Europe, the gallant spirit of chivalry, and the law of
succession, have accustomed us to allow a singular exception; and
a woman is often acknowledged the absolute sovereign of a great
kingdom, in which she would be deemed incapable of exercising the
smallest employment, civil or military. But as the Roman emperors
were still considered as the generals and magistrates of the
republic, their wives and mothers, although distinguished by the
name of Augusta, were never associated to their personal honors;
and a female reign would have appeared an inexpiable prodigy in
the eyes of those primitive Romans, who married without love, or
loved without delicacy and respect.\textsuperscript{64} The haughty Agrippina
aspired, indeed, to share the honors of the empire which she had
conferred on her son; but her mad ambition, detested by every
citizen who felt for the dignity of Rome, was disappointed by the
artful firmness of Seneca and Burrhus.\textsuperscript{65} The good sense, or the
indifference, of succeeding princes, restrained them from
offending the prejudices of their subjects; and it was reserved
for the profligate Elagabalus to discharge the acts of the senate
with the name of his mother Soæmias, who was placed by the side
of the consuls, and subscribed, as a regular member, the decrees
of the legislative assembly. Her more prudent sister, Mamæa,
declined the useless and odious prerogative, and a solemn law was
enacted, excluding women forever from the senate, and devoting to
the infernal gods the head of the wretch by whom this sanction
should be violated.\textsuperscript{66} The substance, not the pageantry, of power
was the object of Mamæa’s manly ambition. She maintained an
absolute and lasting empire over the mind of her son, and in his
affection the mother could not brook a rival. Alexander, with her
consent, married the daughter of a patrician; but his respect for
his father-in-law, and love for the empress, were inconsistent
with the tenderness of interest of Mamæa. The patrician was
executed on the ready accusation of treason, and the wife of
Alexander driven with ignominy from the palace, and banished into
Africa.\textsuperscript{67}

\pagenote[64]{Metellus Numidicus, the censor, acknowledged to the
Roman people, in a public oration, that had kind nature allowed
us to exist without the help of women, we should be delivered
from a very troublesome companion; and he could recommend
matrimony only as the sacrifice of private pleasure to public
duty. Aulus Gellius, i. 6.}

\pagenote[65]{Tacit. Annal. xiii. 5.}

\pagenote[66]{Hist. August. p. 102, 107.}

\pagenote[67]{Dion, l. lxxx. p. 1369. Herodian, l. vi. p. 206.
Hist. August. p. 131. Herodian represents the patrician as
innocent. The Augustian History, on the authority of Dexippus,
condemns him, as guilty of a conspiracy against the life of
Alexander. It is impossible to pronounce between them; but Dion
is an irreproachable witness of the jealousy and cruelty of Mamæa
towards the young empress, whose hard fate Alexander lamented,
but durst not oppose.}

Notwithstanding this act of jealous cruelty, as well as some
instances of avarice, with which Mamæa is charged, the general
tenor of her administration was equally for the benefit of her
son and of the empire. With the approbation of the senate, she
chose sixteen of the wisest and most virtuous senators as a
perpetual council of state, before whom every public business of
moment was debated and determined. The celebrated Ulpian, equally
distinguished by his knowledge of, and his respect for, the laws
of Rome, was at their head; and the prudent firmness of this
aristocracy restored order and authority to the government. As
soon as they had purged the city from foreign superstition and
luxury, the remains of the capricious tyranny of Elagabalus, they
applied themselves to remove his worthless creatures from every
department of the public administration, and to supply their
places with men of virtue and ability. Learning, and the love of
justice, became the only recommendations for civil offices;
valor, and the love of discipline, the only qualifications for
military employments.\textsuperscript{68}

\pagenote[68]{Herodian, l. vi. p. 203. Hist. August. p. 119. The
latter insinuates, that when any law was to be passed, the
council was assisted by a number of able lawyers and experienced
senators, whose opinions were separately given, and taken down in
writing.}

But the most important care of Mamæa and her wise counsellors,
was to form the character of the young emperor, on whose personal
qualities the happiness or misery of the Roman world must
ultimately depend. The fortunate soil assisted, and even
prevented, the hand of cultivation. An excellent understanding
soon convinced Alexander of the advantages of virtue, the
pleasure of knowledge, and the necessity of labor. A natural
mildness and moderation of temper preserved him from the assaults
of passion, and the allurements of vice. His unalterable regard
for his mother, and his esteem for the wise Ulpian, guarded his
unexperienced youth from the poison of flattery.\textsuperscript{581}

\pagenote[581]{Alexander received into his chapel all the
religions which prevailed in the empire; he admitted Jesus
Christ, Abraham, Orpheus, Apollonius of Tyana, \&c. It was almost
certain that his mother Mamæa had instructed him in the morality
of Christianity. Historians in general agree in calling her a
Christian; there is reason to believe that she had begun to have
a taste for the principles of Christianity. (See Tillemont,
Alexander Severus) Gibbon has not noticed this circumstance; he
appears to have wished to lower the character of this empress; he
has throughout followed the narrative of Herodian, who, by the
acknowledgment of Capitolinus himself, detested Alexander.
Without believing the exaggerated praises of Lampridius, he ought
not to have followed the unjust severity of Herodian, and, above
all, not to have forgotten to say that the virtuous Alexander
Severus had insured to the Jews the preservation of their
privileges, and permitted the exercise of Christianity. Hist.
Aug. p. 121. The Christians had established their worship in a
public place, of which the victuallers (cauponarii) claimed, not
the property, but possession by custom. Alexander answered, that
it was better that the place should be used for the service of
God, in any form, than for victuallers.—G. I have scrupled to
omit this note, as it contains some points worthy of notice; but
it is very unjust to Gibbon, who mentions almost all the
circumstances, which he is accused of omitting, in another, and,
according to his plan, a better place, and, perhaps, in stronger
terms than M. Guizot. See Chap. xvi.— M.}

The simple journal of his ordinary occupations exhibits a
pleasing picture of an accomplished emperor,\textsuperscript{69} and, with some
allowance for the difference of manners, might well deserve the
imitation of modern princes. Alexander rose early: the first
moments of the day were consecrated to private devotion, and his
domestic chapel was filled with the images of those heroes, who,
by improving or reforming human life, had deserved the grateful
reverence of posterity. But as he deemed the service of mankind
the most acceptable worship of the gods, the greatest part of his
morning hours was employed in his council, where he discussed
public affairs, and determined private causes, with a patience
and discretion above his years. The dryness of business was
relieved by the charms of literature; and a portion of time was
always set apart for his favorite studies of poetry, history, and
philosophy. The works of Virgil and Horace, the republics of
Plato and Cicero, formed his taste, enlarged his understanding,
and gave him the noblest ideas of man and government. The
exercises of the body succeeded to those of the mind; and
Alexander, who was tall, active, and robust, surpassed most of
his equals in the gymnastic arts. Refreshed by the use of the
bath and a slight dinner, he resumed, with new vigor, the
business of the day; and, till the hour of supper, the principal
meal of the Romans, he was attended by his secretaries, with whom
he read and answered the multitude of letters, memorials, and
petitions, that must have been addressed to the master of the
greatest part of the world. His table was served with the most
frugal simplicity, and whenever he was at liberty to consult his
own inclination, the company consisted of a few select friends,
men of learning and virtue, amongst whom Ulpian was constantly
invited. Their conversation was familiar and instructive; and the
pauses were occasionally enlivened by the recital of some
pleasing composition, which supplied the place of the dancers,
comedians, and even gladiators, so frequently summoned to the
tables of the rich and luxurious Romans.\textsuperscript{70} The dress of
Alexander was plain and modest, his demeanor courteous and
affable: at the proper hours his palace was open to all his
subjects, but the voice of a crier was heard, as in the
Eleusinian mysteries, pronouncing the same salutary admonition:
“Let none enter these holy walls, unless he is conscious of a
pure and innocent mind.”\textsuperscript{71}

\pagenote[69]{See his life in the Augustan History. The
undistinguishing compiler has buried these interesting anecdotes
under a load of trivial unmeaning circumstances.}

\pagenote[70]{See the 13th Satire of Juvenal.}

\pagenote[71]{Hist. August. p. 119.}

Such a uniform tenor of life, which left not a moment for vice or
folly, is a better proof of the wisdom and justice of Alexander’s
government, than all the trifling details preserved in the
compilation of Lampridius. Since the accession of Commodus, the
Roman world had experienced, during the term of forty years, the
successive and various vices of four tyrants. From the death of
Elagabalus, it enjoyed an auspicious calm of thirteen years.\textsuperscript{711}
The provinces, relieved from the oppressive taxes invented by
Caracalla and his pretended son, flourished in peace and
prosperity, under the administration of magistrates who were
convinced by experience that to deserve the love of the subjects
was their best and only method of obtaining the favor of their
sovereign. While some gentle restraints were imposed on the
innocent luxury of the Roman people, the price of provisions and
the interest of money, were reduced by the paternal care of
Alexander, whose prudent liberality, without distressing the
industrious, supplied the wants and amusements of the populace.
The dignity, the freedom, the authority of the senate was
restored; and every virtuous senator might approach the person of
the emperor without a fear and without a blush.

\pagenote[711]{Wenck observes that Gibbon, enchanted with the
virtue of Alexander has heightened, particularly in this
sentence, its effect on the state of the world. His own account,
which follows, of the insurrections and foreign wars, is not in
harmony with this beautiful picture.—M.}

The name of Antoninus, ennobled by the virtues of Pius and
Marcus, had been communicated by adoption to the dissolute Verus,
and by descent to the cruel Commodus. It became the honorable
appellation of the sons of Severus, was bestowed on young
Diadumenianus, and at length prostituted to the infamy of the
high priest of Emesa. Alexander, though pressed by the studied,
and, perhaps, sincere importunity of the senate, nobly refused
the borrowed lustre of a name; whilst in his whole conduct he
labored to restore the glories and felicity of the age of the
genuine Antonines.\textsuperscript{72}

\pagenote[72]{See, in the Hist. August. p. 116, 117, the whole
contest between Alexander and the senate, extracted from the
journals of that assembly. It happened on the sixth of March,
probably of the year 223, when the Romans had enjoyed, almost a
twelvemonth, the blessings of his reign. Before the appellation
of Antoninus was offered him as a title of honor, the senate
waited to see whether Alexander would not assume it as a family
name.}

In the civil administration of Alexander, wisdom was enforced by
power, and the people, sensible of the public felicity, repaid
their benefactor with their love and gratitude. There still
remained a greater, a more necessary, but a more difficult
enterprise; the reformation of the military order, whose interest
and temper, confirmed by long impunity, rendered them impatient
of the restraints of discipline, and careless of the blessings of
public tranquillity. In the execution of his design, the emperor
affected to display his love, and to conceal his fear of the
army. The most rigid economy in every other branch of the
administration supplied a fund of gold and silver for the
ordinary pay and the extraordinary rewards of the troops. In
their marches he relaxed the severe obligation of carrying
seventeen days’ provision on their shoulders. Ample magazines
were formed along the public roads, and as soon as they entered
the enemy’s country, a numerous train of mules and camels waited
on their haughty laziness. As Alexander despaired of correcting
the luxury of his soldiers, he attempted, at least, to direct it
to objects of martial pomp and ornament, fine horses, splendid
armor, and shields enriched with silver and gold. He shared
whatever fatigues he was obliged to impose, visited, in person,
the sick and wounded, preserved an exact register of their
services and his own gratitude, and expressed on every occasion,
the warmest regard for a body of men, whose welfare, as he
affected to declare, was so closely connected with that of the
state.\textsuperscript{73} By the most gentle arts he labored to inspire the
fierce multitude with a sense of duty, and to restore at least a
faint image of that discipline to which the Romans owed their
empire over so many other nations, as warlike and more powerful
than themselves. But his prudence was vain, his courage fatal,
and the attempt towards a reformation served only to inflame the
ills it was meant to cure.

\pagenote[73]{It was a favorite saying of the emperor’s Se
milites magis servare, quam seipsum, quod salus publica in his
esset. Hist. Aug. p. 130.}

The Prætorian guards were attached to the youth of Alexander.
They loved him as a tender pupil, whom they had saved from a
tyrant’s fury, and placed on the Imperial throne. That amiable
prince was sensible of the obligation; but as his gratitude was
restrained within the limits of reason and justice, they soon
were more dissatisfied with the virtues of Alexander, than they
had ever been with the vices of Elagabalus. Their præfect, the
wise Ulpian, was the friend of the laws and of the people; he was
considered as the enemy of the soldiers, and to his pernicious
counsels every scheme of reformation was imputed. Some trifling
accident blew up their discontent into a furious mutiny; and the
civil war raged, during three days, in Rome, whilst the life of
that excellent minister was defended by the grateful people.
Terrified, at length, by the sight of some houses in flames, and
by the threats of a general conflagration, the people yielded
with a sigh, and left the virtuous but unfortunate Ulpian to his
fate. He was pursued into the Imperial palace, and massacred at
the feet of his master, who vainly strove to cover him with the
purple, and to obtain his pardon from the inexorable soldiers.\textsuperscript{731}
Such was the deplorable weakness of government, that the
emperor was unable to revenge his murdered friend and his
insulted dignity, without stooping to the arts of patience and
dissimulation. Epagathus, the principal leader of the mutiny, was
removed from Rome, by the honorable employment of præfect of
Egypt: from that high rank he was gently degraded to the
government of Crete; and when at length, his popularity among the
guards was effaced by time and absence, Alexander ventured to
inflict the tardy but deserved punishment of his crimes.\textsuperscript{74} Under
the reign of a just and virtuous prince, the tyranny of the army
threatened with instant death his most faithful ministers, who
were suspected of an intention to correct their intolerable
disorders. The historian Dion Cassius had commanded the Pannonian
legions with the spirit of ancient discipline. Their brethren of
Rome, embracing the common cause of military license, demanded
the head of the reformer. Alexander, however, instead of yielding
to their seditious clamors, showed a just sense of his merit and
services, by appointing him his colleague in the consulship, and
defraying from his own treasury the expense of that vain dignity:
but as was justly apprehended, that if the soldiers beheld him
with the ensigns of his office, they would revenge the insult in
his blood, the nominal first magistrate of the state retired, by
the emperor’s advice, from the city, and spent the greatest part
of his consulship at his villas in Campania.\textsuperscript{75} \textsuperscript{751}

\pagenote[731]{Gibbon has confounded two events altogether
different— the quarrel of the people with the Prætorians, which
lasted three days, and the assassination of Ulpian by the latter.
Dion relates first the death of Ulpian, afterwards, reverting
back according to a manner which is usual with him, he says that
during the life of Ulpian, there had been a war of three days
between the Prætorians and the people. But Ulpian was not the
cause. Dion says, on the contrary, that it was occasioned by some
unimportant circumstance; whilst he assigns a weighty reason for
the murder of Ulpian, the judgment by which that Prætorian
præfect had condemned his predecessors, Chrestus and Flavian, to
death, whom the soldiers wished to revenge. Zosimus (l. 1, c.
xi.) attributes this sentence to Mamæra; but, even then, the
troops might have imputed it to Ulpian, who had reaped all the
advantage and was otherwise odious to them.—W.}

\pagenote[74]{Though the author of the life of Alexander (Hist.
August. p. 182) mentions the sedition raised against Ulpian by
the soldiers, he conceals the catastrophe, as it might discover a
weakness in the administration of his hero. From this designed
omission, we may judge of the weight and candor of that author.}

\pagenote[75]{For an account of Ulpian’s fate and his own danger,
see the mutilated conclusion of Dion’s History, l. lxxx. p.
1371.}

\pagenote[751]{Dion possessed no estates in Campania, and was not
rich. He only says that the emperor advised him to reside, during
his consulate, in some place out of Rome; that he returned to
Rome after the end of his consulate, and had an interview with
the emperor in Campania. He asked and obtained leave to pass the
rest of his life in his native city, (Nice, in Bithynia:) it was
there that he finished his history, which closes with his second
consulship.—W.}

\section{Part \thesection.}

The lenity of the emperor confirmed the insolence of the troops;
the legions imitated the example of the guards, and defended
their prerogative of licentiousness with the same furious
obstinacy. The administration of Alexander was an unavailing
struggle against the corruption of his age. In llyricum, in
Mauritania, in Armenia, in Mesopotamia, in Germany, fresh
mutinies perpetually broke out; his officers were murdered, his
authority was insulted, and his life at last sacrificed to the
fierce discontents of the army.\textsuperscript{76} One particular fact well
deserves to be recorded, as it illustrates the manners of the
troops, and exhibits a singular instance of their return to a
sense of duty and obedience. Whilst the emperor lay at Antioch,
in his Persian expedition, the particulars of which we shall
hereafter relate, the punishment of some soldiers, who had been
discovered in the baths of women, excited a sedition in the
legion to which they belonged. Alexander ascended his tribunal,
and with a modest firmness represented to the armed multitude the
absolute necessity, as well as his inflexible resolution, of
correcting the vices introduced by his impure predecessor, and of
maintaining the discipline, which could not be relaxed without
the ruin of the Roman name and empire. Their clamors interrupted
his mild expostulation. “Reserve your shout,” said the undaunted
emperor, “till you take the field against the Persians, the
Germans, and the Sarmatians. Be silent in the presence of your
sovereign and benefactor, who bestows upon you the corn, the
clothing, and the money of the provinces. Be silent, or I shall
no longer style you solders, but \textit{citizens},\textsuperscript{77} if those indeed
who disclaim the laws of Rome deserve to be ranked among the
meanest of the people.” His menaces inflamed the fury of the
legion, and their brandished arms already threatened his person.
“Your courage,” resumed the intrepid Alexander, “would be more
nobly displayed in the field of battle; \textit{me} you may destroy, you
cannot intimidate; and the severe justice of the republic would
punish your crime and revenge my death.” The legion still
persisted in clamorous sedition, when the emperor pronounced,
with a loud voice, the decisive sentence, “\textit{Citizens!} lay down
your arms, and depart in peace to your respective habitations.”
The tempest was instantly appeased: the soldiers, filled with
grief and shame, silently confessed the justice of their
punishment, and the power of discipline, yielded up their arms
and military ensigns, and retired in confusion, not to their
camp, but to the several inns of the city. Alexander enjoyed,
during thirty days, the edifying spectacle of their repentance;
nor did he restore them to their former rank in the army, till he
had punished with death those tribunes whose connivance had
occasioned the mutiny. The grateful legion served the emperor
whilst living, and revenged him when dead.\textsuperscript{78}

\pagenote[76]{Annot. Reimar. ad Dion Cassius, l. lxxx. p. 1369.}

\pagenote[77]{Julius Cæsar had appeased a sedition with the same
word, Quirites; which, thus opposed to soldiers, was used in a
sense of contempt, and reduced the offenders to the less
honorable condition of mere citizens. Tacit. Annal. i. 43.}

\pagenote[78]{Hist. August. p. 132.}

The resolutions of the multitude generally depend on a moment;
and the caprice of passion might equally determine the seditious
legion to lay down their arms at the emperor’s feet, or to plunge
them into his breast. Perhaps, if this singular transaction had
been investigated by the penetration of a philosopher, we should
discover the secret causes which on that occasion authorized the
boldness of the prince, and commanded the obedience of the
troops; and perhaps, if it had been related by a judicious
historian, we should find this action, worthy of Cæsar himself,
reduced nearer to the level of probability and the common
standard of the character of Alexander Severus. The abilities of
that amiable prince seem to have been inadequate to the
difficulties of his situation, the firmness of his conduct
inferior to the purity of his intentions. His virtues, as well as
the vices of Elagabalus, contracted a tincture of weakness and
effeminacy from the soft climate of Syria, of which he was a
native; though he blushed at his foreign origin, and listened
with a vain complacency to the flattering genealogists, who
derived his race from the ancient stock of Roman nobility.\textsuperscript{79} The
pride and avarice of his mother cast a shade on the glories of
his reign; and by exacting from his riper years the same dutiful
obedience which she had justly claimed from his unexperienced
youth, Mamæa exposed to public ridicule both her son’s character
and her own.\textsuperscript{80} The fatigues of the Persian war irritated the
military discontent; the unsuccessful event\textsuperscript{801} degraded the
reputation of the emperor as a general, and even as a soldier.
Every cause prepared, and every circumstance hastened, a
revolution, which distracted the Roman empire with a long series
of intestine calamities.

\pagenote[79]{From the Metelli. Hist. August. p. 119. The choice
was judicious. In one short period of twelve years, the Metelli
could reckon seven consulships and five triumphs. See Velleius
Paterculus, ii. 11, and the Fasti.}

\pagenote[80]{The life of Alexander, in the Augustan History, is
the mere idea of a perfect prince, an awkward imitation of the
Cyropædia. The account of his reign, as given by Herodian, is
rational and moderate, consistent with the general history of the
age; and, in some of the most invidious particulars, confirmed by
the decisive fragments of Dion. Yet from a very paltry prejudice,
the greater number of our modern writers abuse Herodian, and copy
the Augustan History. See Mess de Tillemont and Wotton. From the
opposite prejudice, the emperor Julian (in Cæsarib. p. 315)
dwells with a visible satisfaction on the effeminate weakness of
the Syrian, and the ridiculous avarice of his mother.}

\pagenote[801]{Historians are divided as to the success of the
campaign against the Persians; Herodian alone speaks of defeat.
Lampridius, Eutropius, Victor, and others, say that it was very
glorious to Alexander; that he beat Artaxerxes in a great battle,
and repelled him from the frontiers of the empire. This much is
certain, that Alexander, on his return to Rome, (Lamp. Hist. Aug.
c. 56, 133, 134,) received the honors of a triumph, and that he
said, in his oration to the people. Quirites, vicimus Persas,
milites divites reduximus, vobis congiarium pollicemur, cras
ludos circenses Persicos donabimus. Alexander, says Eckhel, had
too much modesty and wisdom to permit himself to receive honors
which ought only to be the reward of victory, if he had not
deserved them; he would have contented himself with dissembling
his losses. Eckhel, Doct. Num. vet. vii. 276. The medals
represent him as in triumph; one, among others, displays him
crowned by Victory between two rivers, the Euphrates and the
Tigris. P. M. TR. P. xii. Cos. iii. PP. Imperator paludatus D.
hastam. S. parazonium, stat inter duos fluvios humi jacentes, et
ab accedente retro Victoria coronatur. Æ. max. mod. (Mus. Reg.
Gall.) Although Gibbon treats this question more in detail when
he speaks of the Persian monarchy, I have thought fit to place
here what contradicts his opinion.—G}

The dissolute tyranny of Commodus, the civil wars occasioned by
his death, and the new maxims of policy introduced by the house
of Severus, had all contributed to increase the dangerous power
of the army, and to obliterate the faint image of laws and
liberty that was still impressed on the minds of the Romans. The
internal change, which undermined the foundations of the empire,
we have endeavored to explain with some degree of order and
perspicuity. The personal characters of the emperors, their
victories, laws, follies, and fortunes, can interest us no
farther than as they are connected with the general history of
the Decline and Fall of the monarchy. Our constant attention to
that great object will not suffer us to overlook a most important
edict of Antoninus Caracalla, which communicated to all the free
inhabitants of the empire the name and privileges of Roman
citizens. His unbounded liberality flowed not, however, from the
sentiments of a generous mind; it was the sordid result of
avarice, and will naturally be illustrated by some observations
on the finances of that state, from the victorious ages of the
commonwealth to the reign of Alexander Severus.

The siege of Veii in Tuscany, the first considerable enterprise
of the Romans, was protracted to the tenth year, much less by the
strength of the place than by the unskilfulness of the besiegers.
The unaccustomed hardships of so many winter campaigns, at the
distance of near twenty miles from home,\textsuperscript{81} required more than
common encouragements; and the senate wisely prevented the
clamors of the people, by the institution of a regular pay for
the soldiers, which was levied by a general tribute, assessed
according to an equitable proportion on the property of the
citizens.\textsuperscript{82} During more than two hundred years after the
conquest of Veii, the victories of the republic added less to the
wealth than to the power of Rome. The states of Italy paid their
tribute in military service only, and the vast force, both by sea
and land, which was exerted in the Punic wars, was maintained at
the expense of the Romans themselves. That high-spirited people
(such is often the generous enthusiasm of freedom) cheerfully
submitted to the most excessive but voluntary burdens, in the
just confidence that they should speedily enjoy the rich harvest
of their labors. Their expectations were not disappointed. In the
course of a few years, the riches of Syracuse, of Carthage, of
Macedonia, and of Asia, were brought in triumph to Rome. The
treasures of Perseus alone amounted to near two millions
sterling, and the Roman people, the sovereign of so many nations,
was forever delivered from the weight of taxes.\textsuperscript{83} The increasing
revenue of the provinces was found sufficient to defray the
ordinary establishment of war and government, and the superfluous
mass of gold and silver was deposited in the temple of Saturn,
and reserved for any unforeseen emergency of the state.\textsuperscript{84}

\pagenote[81]{According to the more accurate Dionysius, the city
itself was only a hundred stadia, or twelve miles and a half,
from Rome, though some out-posts might be advanced farther on the
side of Etruria. Nardini, in a professed treatise, has combated
the popular opinion and the authority of two popes, and has
removed Veii from Civita Castellana, to a little spot called
Isola, in the midway between Rome and the Lake Bracianno. * Note:
See the interesting account of the site and ruins of Veii in Sir
W Gell’s topography of Rome and its Vicinity. v. ii. p. 303.—M.}

\pagenote[82]{See the 4th and 5th books of Livy. In the Roman
census, property, power, and taxation were commensurate with each
other.}

\pagenote[83]{Plin. Hist. Natur. l. xxxiii. c. 3. Cicero de
Offic. ii. 22. Plutarch, P. Æmil. p. 275.}

\pagenote[84]{See a fine description of this accumulated wealth
of ages in Phars. l. iii. v. 155, \&c.}

History has never, perhaps, suffered a greater or more
irreparable injury than in the loss of the curious register\textsuperscript{841}
bequeathed by Augustus to the senate, in which that experienced
prince so accurately balanced the revenues and expenses of the
Roman empire.\textsuperscript{85} Deprived of this clear and comprehensive
estimate, we are reduced to collect a few imperfect hints from
such of the ancients as have accidentally turned aside from the
splendid to the more useful parts of history. We are informed
that, by the conquests of Pompey, the tributes of Asia were
raised from fifty to one hundred and thirty-five millions of
drachms; or about four millions and a half sterling.\textsuperscript{86} \textsuperscript{861} Under
the last and most indolent of the Ptolemies, the revenue of Egypt
is said to have amounted to twelve thousand five hundred talents;
a sum equivalent to more than two millions and a half of our
money, but which was afterwards considerably improved by the more
exact economy of the Romans, and the increase of the trade of
Æthiopia and India.\textsuperscript{87} Gaul was enriched by rapine, as Egypt was
by commerce, and the tributes of those two great provinces have
been compared as nearly equal to each other in value.\textsuperscript{88} The ten
thousand Euboic or Phœnician talents, about four millions
sterling,\textsuperscript{89} which vanquished Carthage was condemned to pay
within the term of fifty years, were a slight acknowledgment of
the superiority of Rome,\textsuperscript{90} and cannot bear the least proportion
with the taxes afterwards raised both on the lands and on the
persons of the inhabitants, when the fertile coast of Africa was
reduced into a province.\textsuperscript{91}

\pagenote[841]{See Rationarium imperii. Compare besides Tacitus,
Suet. Aug. c. ult. Dion, p. 832. Other emperors kept and
published similar registers. See a dissertation of Dr. Wolle, de
Rationario imperii Rom. Leipsig, 1773. The last book of Appian
also contained the statistics of the Roman empire, but it is
lost.—W.}

\pagenote[85]{Tacit. in Annal. i. ll. It seems to have existed in
the time of Appian.}

\pagenote[86]{Plutarch, in Pompeio, p. 642.}

\pagenote[861]{Wenck contests the accuracy of Gibbon’s version of
Plutarch, and supposes that Pompey only raised the revenue from
50,000,000 to 85,000,000 of drachms; but the text of Plutarch
seems clearly to mean that his conquests added 85,000,000 to the
ordinary revenue. Wenck adds, “Plutarch says in another part,
that Antony made Asia pay, at one time, 200,000 talents, that is
to say, 38,875,000 L. sterling.” But Appian explains this by
saying that it was the revenue of ten years, which brings the
annual revenue, at the time of Antony, to 3,875,000 L.
sterling.—M.}

\pagenote[87]{Strabo, l. xvii. p. 798.}

\pagenote[88]{Velleius Paterculus, l. ii. c. 39. He seems to give
the preference to the revenue of Gaul.}

\pagenote[89]{The Euboic, the Phœnician, and the Alexandrian
talents were double in weight to the Attic. See Hooper on ancient
weights and measures, p. iv. c. 5. It is very probable that the
same talent was carried from Tyre to Carthage.}

\pagenote[90]{Polyb. l. xv. c. 2.}

\pagenote[91]{Appian in Punicis, p. 84.}

Spain, by a very singular fatality, was the Peru and Mexico of
the old world. The discovery of the rich western continent by the
Phœnicians, and the oppression of the simple natives, who were
compelled to labor in their own mines for the benefit of
strangers, form an exact type of the more recent history of
Spanish America.\textsuperscript{92} The Phœnicians were acquainted only with the
sea-coast of Spain; avarice, as well as ambition, carried the
arms of Rome and Carthage into the heart of the country, and
almost every part of the soil was found pregnant with copper,
silver, and gold.\textsuperscript{921} Mention is made of a mine near Carthagena
which yielded every day twenty-five thousand drachmns of silver,
or about three hundred thousand pounds a year.\textsuperscript{93} Twenty thousand
pound weight of gold was annually received from the provinces of
Asturia, Gallicia, and Lusitania.\textsuperscript{94}

\pagenote[92]{Diodorus Siculus, l. 5. Oadiz was built by the
Phœnicians a little more than a thousand years before Christ. See
Vell. Pa ter. i.2.}

\pagenote[921]{Compare Heeren’s Researches vol. i. part ii. p.}

\pagenote[93]{Strabo, l. iii. p. 148.}

\pagenote[94]{Plin. Hist. Natur. l. xxxiii. c. 3. He mentions
likewise a silver mine in Dalmatia, that yielded every day fifty
pounds to the state.}

We want both leisure and materials to
pursue this curious inquiry through the many potent states that
were annihilated in the Roman empire. Some notion, however, may
be formed of the revenue of the provinces where considerable
wealth had been deposited by nature, or collected by man, if we
observe the severe attention that was directed to the abodes of
solitude and sterility. Augustus once received a petition from
the inhabitants of Gyarus, humbly praying that they might be
relieved from one third of their excessive impositions. Their
whole tax amounted indeed to no more than one hundred and fifty
drachms, or about five pounds: but Gyarus was a little island, or
rather a rock, of the Ægean Sea, destitute of fresh water and
every necessary of life, and inhabited only by a few wretched
fishermen.\textsuperscript{95}

\pagenote[95]{Strabo, l. x. p. 485. Tacit. Annal. iu. 69, and iv.
30. See Tournefort (Voyages au Levant, Lettre viii.) a very
lively picture of the actual misery of Gyarus.}

From the faint glimmerings of such doubtful and scattered lights,
we should be inclined to believe, 1st, That (with every fair
allowance for the differences of times and circumstances) the
general income of the Roman provinces could seldom amount to less
than fifteen or twenty millions of our money;\textsuperscript{96} and, 2dly, That
so ample a revenue must have been fully adequate to all the
expenses of the moderate government instituted by Augustus, whose
court was the modest family of a private senator, and whose
military establishment was calculated for the defence of the
frontiers, without any aspiring views of conquest, or any serious
apprehension of a foreign invasion.

\pagenote[96]{Lipsius de magnitudine Romana (l. ii. c. 3)
computes the revenue at one hundred and fifty millions of gold
crowns; but his whole book, though learned and ingenious, betrays
a very heated imagination. Note: If Justus Lipsius has
exaggerated the revenue of the Roman empire Gibbon, on the other
hand, has underrated it. He fixes it at fifteen or twenty
millions of our money. But if we take only, on a moderate
calculation, the taxes in the provinces which he has already
cited, they will amount, considering the augmentations made by
Augustus, to nearly that sum. There remain also the provinces of
Italy, of Rhætia, of Noricum, Pannonia, and Greece, \&c., \&c. Let
us pay attention, besides, to the prodigious expenditure of some
emperors, (Suet. Vesp. 16;) we shall see that such a revenue
could not be sufficient. The authors of the Universal History,
part xii., assign forty millions sterling as the sum to about
which the public revenue might amount.—G. from W.}

Notwithstanding the seeming probability of both these
conclusions, the latter of them at least is positively disowned
by the language and conduct of Augustus. It is not easy to
determine whether, on this occasion, he acted as the common
father of the Roman world, or as the oppressor of liberty;
whether he wished to relieve the provinces, or to impoverish the
senate and the equestrian order. But no sooner had he assumed the
reins of government, than he frequently intimated the
insufficiency of the tributes, and the necessity of throwing an
equitable proportion of the public burden upon Rome and Italy.\textsuperscript{961}
In the prosecution of this unpopular design, he advanced,
however, by cautious and well-weighed steps. The introduction of
customs was followed by the establishment of an excise, and the
scheme of taxation was completed by an artful assessment on the
real and personal property of the Roman citizens, who had been
exempted from any kind of contribution above a century and a
half.

\pagenote[961]{It is not astonishing that Augustus held this
language. The senate declared also under Nero, that the state
could not exist without the imposts as well augmented as founded
by Augustus. Tac. Ann. xiii. 50. After the abolition of the
different tributes paid by Italy, an abolition which took place
A. U. 646, 694, and 695, the state derived no revenues from that
great country, but the twentieth part of the manumissions,
(vicesima manumissionum,) and Ciero laments this in many places,
particularly in his epistles to ii. 15.—G. from W.}

I. In a great empire like that of Rome, a natural balance of
money must have gradually established itself. It has been already
observed, that as the wealth of the provinces was attracted to
the capital by the strong hand of conquest and power, so a
considerable part of it was restored to the industrious provinces
by the gentle influence of commerce and arts. In the reign of
Augustus and his successors, duties were imposed on every kind of
merchandise, which through a thousand channels flowed to the
great centre of opulence and luxury; and in whatsoever manner the
law was expressed, it was the Roman purchaser, and not the
provincial merchant, who paid the tax.\textsuperscript{97} The rate of the customs
varied from the eighth to the fortieth part of the value of the
commodity; and we have a right to suppose that the variation was
directed by the unalterable maxims of policy; that a higher duty
was fixed on the articles of luxury than on those of necessity,
and that the productions raised or manufactured by the labor of
the subjects of the empire were treated with more indulgence than
was shown to the pernicious, or at least the unpopular, commerce
of Arabia and India.\textsuperscript{98} There is still extant a long but
imperfect catalogue of eastern commodities, which about the time
of Alexander Severus were subject to the payment of duties;
cinnamon, myrrh, pepper, ginger, and the whole tribe of
aromatics; a great variety of precious stones, among which the
diamond was the most remarkable for its price, and the emerald
for its beauty;\textsuperscript{99} Parthian and Babylonian leather, cottons,
silks, both raw and manufactured, ebony ivory, and eunuchs.\textsuperscript{100}
We may observe that the use and value of those effeminate slaves
gradually rose with the decline of the empire.

\pagenote[97]{Tacit. Annal. xiii. 31. * Note: The customs
(portoria) existed in the times of the ancient kings of Rome.
They were suppressed in Italy, A. U. 694, by the Prætor, Cecilius
Matellus Nepos. Augustus only reestablished them. See note
above.—W.}

\pagenote[98]{See Pliny, (Hist. Natur. l. vi. c. 23, lxii. c. 18.)
His observation that the Indian commodities were sold at Rome at
a hundred times their original price, may give us some notion of
the produce of the customs, since that original price amounted to
more than eight hundred thousand pounds.}

\pagenote[99]{The ancients were unacquainted with the art of
cutting diamonds.}

\pagenote[100]{M. Bouchaud, in his treatise de l’Impot chez les
Romains, has transcribed this catalogue from the Digest, and
attempts to illustrate it by a very prolix commentary. * Note: In
the Pandects, l. 39, t. 14, de Publican. Compare Cicero in
Verrem. c. 72—74.—W.}

II. The excise, introduced by Augustus after the civil wars, was
extremely moderate, but it was general. It seldom exceeded one
\textit{per cent}.; but it comprehended whatever was sold in the markets
or by public auction, from the most considerable purchases of
lands and houses, to those minute objects which can only derive a
value from their infinite multitude and daily consumption. Such a
tax, as it affects the body of the people, has ever been the
occasion of clamor and discontent. An emperor well acquainted
with the wants and resources of the state was obliged to declare,
by a public edict, that the support of the army depended in a
great measure on the produce of the excise.\textsuperscript{101}

\pagenote[101]{Tacit. Annal. i. 78. Two years afterwards, the
reduction of the poor kingdom of Cappadocia gave Tiberius a
pretence for diminishing the excise of one half, but the relief
was of very short duration.}

III. When Augustus resolved to establish a permanent military
force for the defence of his government against foreign and
domestic enemies, he instituted a peculiar treasury for the pay
of the soldiers, the rewards of the veterans, and the
extra-ordinary expenses of war. The ample revenue of the excise,
though peculiarly appropriated to those uses, was found
inadequate. To supply the deficiency, the emperor suggested a new
tax of five per cent. on all legacies and inheritances. But the
nobles of Rome were more tenacious of property than of freedom.
Their indignant murmurs were received by Augustus with his usual
temper. He candidly referred the whole business to the senate,
and exhorted them to provide for the public service by some other
expedient of a less odious nature. They were divided and
perplexed. He insinuated to them, that their obstinacy would
oblige him to \textit{propose} a general land tax and capitation. They
acquiesced in silence.\textsuperscript{102} The new imposition on legacies and
inheritances was, however, mitigated by some restrictions. It did
not take place unless the object was of a certain value, most
probably of fifty or a hundred pieces of gold;\textsuperscript{103} nor could it
be exacted from the nearest of kin on the father’s side.\textsuperscript{104} When
the rights of nature and poverty were thus secured, it seemed
reasonable, that a stranger, or a distant relation, who acquired
an unexpected accession of fortune, should cheerfully resign a
twentieth part of it, for the benefit of the state. \textsuperscript{105}

\pagenote[102]{Dion Cassius, l. lv. p. 794, l. lvi. p. 825. Note:
Dion neither mentions this proposition nor the capitation. He
only says that the emperor imposed a tax upon landed property,
and sent every where men employed to make a survey, without
fixing how much, and for how much each was to pay. The senators
then preferred giving the tax on legacies and inheritances.—W.}

\pagenote[103]{The sum is only fixed by conjecture.}

\pagenote[104]{As the Roman law subsisted for many ages, the
Cognati, or relations on the mother’s side, were not called to
the succession. This harsh institution was gradually undermined
by humanity, and finally abolished by Justinian.}

\pagenote[105]{Plin. Panegyric. c. 37.}

Such a tax, plentiful as it must prove in every wealthy
community, was most happily suited to the situation of the
Romans, who could frame their arbitrary wills, according to the
dictates of reason or caprice, without any restraint from the
modern fetters of entails and settlements. From various causes,
the partiality of paternal affection often lost its influence
over the stern patriots of the commonwealth, and the dissolute
nobles of the empire; and if the father bequeathed to his son the
fourth part of his estate, he removed all ground of legal
complaint.\textsuperscript{106} But a rich childish old man was a domestic tyrant,
and his power increased with his years and infirmities. A servile
crowd, in which he frequently reckoned prætors and consuls,
courted his smiles, pampered his avarice, applauded his follies,
served his passions, and waited with impatience for his death.
The arts of attendance and flattery were formed into a most
lucrative science; those who professed it acquired a peculiar
appellation; and the whole city, according to the lively
descriptions of satire, was divided between two parties, the
hunters and their game.\textsuperscript{107} Yet, while so many unjust and
extravagant wills were every day dictated by cunning and
subscribed by folly, a few were the result of rational esteem and
virtuous gratitude. Cicero, who had so often defended the lives
and fortunes of his fellow-citizens, was rewarded with legacies
to the amount of a hundred and seventy thousand pounds;\textsuperscript{108} nor
do the friends of the younger Pliny seem to have been less
generous to that amiable orator.\textsuperscript{109} Whatever was the motive of
the testator, the treasury claimed, without distinction, the
twentieth part of his estate: and in the course of two or three
generations, the whole property of the subject must have
gradually passed through the coffers of the state.

\pagenote[106]{See Heineccius in the Antiquit. Juris Romani, l.
ii.}

\pagenote[107]{Horat. l. ii. Sat. v. Potron. c. 116, \&c. Plin. l.
ii. Epist. 20.}

\pagenote[108]{Cicero in Philip. ii. c. 16.}

\pagenote[109]{See his epistles. Every such will gave him an
occasion of displaying his reverence to the dead, and his justice
to the living. He reconciled both in his behavior to a son who
had been disinherited by his mother, (v.l.)}

In the first and golden years of the reign of Nero, that prince,
from a desire of popularity, and perhaps from a blind impulse of
benevolence, conceived a wish of abolishing the oppression of the
customs and excise. The wisest senators applauded his
magnanimity: but they diverted him from the execution of a design
which would have dissolved the strength and resources of the
republic.\textsuperscript{110} Had it indeed been possible to realize this dream
of fancy, such princes as Trajan and the Antonines would surely
have embraced with ardor the glorious opportunity of conferring
so signal an obligation on mankind. Satisfied, however, with
alleviating the public burden, they attempted not to remove it.
The mildness and precision of their laws ascertained the rule and
measure of taxation, and protected the subject of every rank
against arbitrary interpretations, antiquated claims, and the
insolent vexation of the farmers of the revenue.\textsuperscript{111} For it is
somewhat singular, that, in every age, the best and wisest of the
Roman governors persevered in this pernicious method of
collecting the principal branches at least of the excise and
customs.\textsuperscript{112}

\pagenote[110]{Tacit. Annal. xiii. 50. Esprit des Loix, l. xii.
c. 19.}

\pagenote[111]{See Pliny’s Panegyric, the Augustan History, and
Burman de Vectigal. passim.}

\pagenote[112]{The tributes (properly so called) were not farmed;
since the good princes often remitted many millions of arrears.}

The sentiments, and, indeed, the situation, of Caracalla were
very different from those of the Antonines. Inattentive, or
rather averse, to the welfare of his people, he found himself
under the necessity of gratifying the insatiate avarice which he
had excited in the army. Of the several impositions introduced by
Augustus, the twentieth on inheritances and legacies was the most
fruitful, as well as the most comprehensive. As its influence was
not confined to Rome or Italy, the produce continually increased
with the gradual extension of the Roman City. The new citizens,
though charged, on equal terms,\textsuperscript{113} with the payment of new
taxes, which had not affected them as subjects, derived an ample
compensation from the rank they obtained, the privileges they
acquired, and the fair prospect of honors and fortune that was
thrown open to their ambition. But the favor which implied a
distinction was lost in the prodigality of Caracalla, and the
reluctant provincials were compelled to assume the vain title,
and the real obligations, of Roman citizens.\textsuperscript{1131} Nor was the
rapacious son of Severus contented with such a measure of
taxation as had appeared sufficient to his moderate predecessors.
Instead of a twentieth, he exacted a tenth of all legacies and
inheritances; and during his reign (for the ancient proportion
was restored after his death) he crushed alike every part of the
empire under the weight of his iron sceptre.\textsuperscript{114}

\pagenote[113]{The situation of the new citizens is minutely
described by Pliny, (Panegyric, c. 37, 38, 39). Trajan published
a law very much in their favor.}

\pagenote[1131]{Gibbon has adopted the opinion of Spanheim and of
Burman, which attributes to Caracalla this edict, which gave the
right of the city to all the inhabitants of the provinces. This
opinion may be disputed. Several passages of Spartianus, of
Aurelius Victor, and of Aristides, attribute this edict to Marc.
Aurelius. See a learned essay, entitled Joh. P. Mahneri Comm. de
Marc. Aur. Antonino Constitutionis de Civitate Universo Orbi
Romano data auctore. Halæ, 1772, 8vo. It appears that Marc.
Aurelius made some modifications of this edict, which released
the provincials from some of the charges imposed by the right of
the city, and deprived them of some of the advantages which it
conferred. Caracalla annulled these modifications.—W.}

\pagenote[114]{Dion, l. lxxvii. p. 1295.}

When all the provincials became liable to the peculiar
impositions of Roman citizens, they seemed to acquire a legal
exemption from the tributes which they had paid in their former
condition of subjects. Such were not the maxims of government
adopted by Caracalla and his pretended son. The old as well as
the new taxes were, at the same time, levied in the provinces. It
was reserved for the virtue of Alexander to relieve them in a
great measure from this intolerable grievance, by reducing the
tributes to a thirteenth part of the sum exacted at the time of
his accession.\textsuperscript{115} It is impossible to conjecture the motive that
engaged him to spare so trifling a remnant of the public evil;
but the noxious weed, which had not been totally eradicated,
again sprang up with the most luxuriant growth, and in the
succeeding age darkened the Roman world with its deadly shade. In
the course of this history, we shall be too often summoned to
explain the land tax, the capitation, and the heavy contributions
of corn, wine, oil, and meat, which were exacted from the
provinces for the use of the court, the army, and the capital.

\pagenote[115]{He who paid ten aurei, the usual tribute, was
charged with no more than the third part of an aureus, and
proportional pieces of gold were coined by Alexander’s order.
Hist. August. p. 127, with the commentary of Salmasius.}

As long as Rome and Italy were respected as the centre of
government, a national spirit was preserved by the ancient, and
insensibly imbibed by the adopted, citizens. The principal
commands of the army were filled by men who had received a
liberal education, were well instructed in the advantages of laws
and letters, and who had risen, by equal steps, through the
regular succession of civil and military honors.\textsuperscript{116} To their
influence and example we may partly ascribe the modest obedience
of the legions during the two first centuries of the Imperial
history.

\pagenote[116]{See the lives of Agricola, Vespasian, Trajan,
Severus, and his three competitors; and indeed of all the eminent
men of those times.}

But when the last enclosure of the Roman constitution was
trampled down by Caracalla, the separation of professions
gradually succeeded to the distinction of ranks. The more
polished citizens of the internal provinces were alone qualified
to act as lawyers and magistrates. The rougher trade of arms was
abandoned to the peasants and barbarians of the frontiers, who
knew no country but their camp, no science but that of war, no
civil laws, and scarcely those of military discipline. With
bloody hands, savage manners, and desperate resolutions, they
sometimes guarded, but much oftener subverted, the throne of the
emperors.

