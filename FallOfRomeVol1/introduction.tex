\section{Introduction}
\subsection{Preface By The Editor.}
\markboth{Introduction}{Editor's Preface}
\thispagestyle{simple}

The great work of Gibbon is indispensable to the student of history. The literature of Europe offers no substitute for “The Decline and Fall of the Roman Empire.” It has obtained undisputed possession, as rightful occupant, of the vast period which it comprehends. However some subjects, which it embraces, may have undergone more complete investigation, on the general view of the whole period, this history is the sole undisputed authority to which all defer, and from which few appeal to the original writers, or to more modern compilers. The inherent interest of the subject, the inexhaustible labor employed upon it; the immense condensation of matter; the luminous arrangement; the general accuracy; the style, which, however monotonous from its uniform stateliness, and sometimes wearisome from its elaborate ar., is throughout vigorous, animated, often picturesque, always commands attention, always conveys its meaning with emphatic energy, describes with singular breadth and fidelity, and generalizes with unrivalled felicity of expression; all these high qualifications have secured, and seem likely to secure, its permanent place in historic literature.

This vast design of Gibbon, the magnificent whole into which he has cast the decay and ruin of the ancient civilization, the formation and birth of the new order of things, will of itself, independent of the laborious execution of his immense plan, render “The Decline and Fall of the Roman Empire” an unapproachable subject to the future historian:\footnotemark[101] in the eloquent language of his recent French editor, M. Guizot:—

\footnotetext[101]{A considerable portion of this preface has already appeared before us public in the Quarterly Review.}

“The gradual decline of the most extraordinary dominion which has ever invaded and oppressed the world; the fall of that immense empire, erected on the ruins of so many kingdoms, republics, and states both barbarous and civilized; and forming in its turn, by its dismemberment, a multitude of states, republics, and kingdoms; the annihilation of the religion of Greece and Rome; the birth and the progress of the two new religions which have shared the most beautiful regions of the earth; the decrepitude of the ancient world, the spectacle of its expiring glory and degenerate manners; the infancy of the modern world, the picture of its first progress, of the new direction given to the mind and character of man—such a subject must necessarily fix the attention and excite the interest of men, who cannot behold with indifference those memorable epochs, during which, in the fine language of Corneille—

‘Un grand destin commence, un grand destin s’achève.’”

This extent and harmony of design is unquestionably that which distinguishes the work of Gibbon from all other great historical compositions. He has first bridged the abyss between ancient and modern times, and connected together the two great worlds of history. The great advantage which the classical historians possess over those of modern times is in unity of plan, of course greatly facilitated by the narrower sphere to which their researches were confined. Except Herodotus, the great historians of Greece—we exclude the more modern compilers, like Diodorus Siculus—limited themselves to a single period, or at ‘east to the contracted sphere of Grecian affairs. As far as the Barbarians trespassed within the Grecian boundary, or were necessarily mingled up with Grecian politics, they were admitted into the pale of Grecian history; but to Thucydides and to Xenophon, excepting in the Persian inroad of the latter, Greece was the world. Natural unity confined their narrative almost to chronological order, the episodes were of rare occurrence and extremely brief. To the Roman historians the course was equally clear and defined. Rome was their centre of unity; and the uniformity with which the circle of the Roman dominion spread around, the regularity with which their civil polity expanded, forced, as it were, upon the Roman historian that plan which Polybius announces as the subject of his history, the means and the manner by which the whole world became subject to the Roman sway. How different the complicated politics of the European kingdoms! Every national history, to be complete, must, in a certain sense, be the history of Europe; there is no knowing to how remote a quarter it may be necessary to trace our most domestic events; from a country, how apparently disconnected, may originate the impulse which gives its direction to the whole course of affairs.

In imitation of his classical models, Gibbon places \textit{Rome} as the cardinal point from which his inquiries diverge, and to which they bear constant reference; yet how immeasurable the space over which those inquiries range! how complicated, how confused, how apparently inextricable the causes which tend to the decline of the Roman empire! how countless the nations which swarm forth, in mingling and indistinct hordes, constantly changing the geographical limits—incessantly confounding the natural boundaries! At first sight, the whole period, the whole state of the world, seems to offer no more secure footing to an historical adventurer than the chaos of Milton—to be in a state of irreclaimable disorder, best described in the language of the poet:—
\begin{verse}
—“A dark Illimitable ocean, without bound,\\
Without dimension, where length, breadth, and height,\\
And time, and place, are lost: where eldest Night\\
And Chaos, ancestors of Nature, hold\\
Eternal anarchy, amidst the noise\\
Of endless wars, and by confusion stand.”
\end{verse}
We feel that the unity and harmony of narrative, which shall comprehend this period of social disorganization, must be ascribed entirely to the skill and luminous disposition of the historian. It is in this sublime Gothic architecture of his work, in which the boundless range, the infinite variety, the, at first sight, incongruous gorgeousness of the separate parts, nevertheless are all subordinate to one main and predominant idea, that Gibbon is unrivalled. We cannot but admire the manner in which he masses his materials, and arranges his facts in successive groups, not according to chronological order, but to their moral or political connection; the distinctness with which he marks his periods of gradually increasing decay; and the skill with which, though advancing on separate parallels of history, he shows the common tendency of the slower or more rapid religious or civil innovations. However these principles of composition may demand more than ordinary attention on the part of the reader, they can alone impress upon the memory the real course, and the relative importance of the events. Whoever would justly appreciate the superiority of Gibbon’s lucid arrangement, should attempt to make his way through the regular but wearisome annals of Tillemont, or even the less ponderous volumes of Le Beau. Both these writers adhere, almost entirely, to chronological order; the consequence is, that we are twenty times called upon to break off, and resume the thread of six or eight wars in different parts of the empire; to suspend the operations of a military expedition for a court intrigue; to hurry away from a siege to a council; and the same page places us in the middle of a campaign against the barbarians, and in the depths of the Monophysite controversy. In Gibbon it is not always easy to bear in mind the exact dates but the course of events is ever clear and distinct; like a skilful general, though his troops advance from the most remote and opposite quarters, they are constantly bearing down and concentrating themselves on one point—that which is still occupied by the name, and by the waning power of Rome. Whether he traces the progress of hostile religions, or leads from the shores of the Baltic, or the verge of the Chinese empire, the successive hosts of barbarians—though one wave has hardly burst and discharged itself, before another swells up and approaches—all is made to flow in the same direction, and the impression which each makes upon the tottering fabric of the Roman greatness, connects their distant movements, and measures the relative importance assigned to them in the panoramic history. The more peaceful and didactic episodes on the development of the Roman law, or even on the details of ecclesiastical history, interpose themselves as resting-places or divisions between the periods of barbaric invasion. In short, though distracted first by the two capitals, and afterwards by the formal partition of the empire, the extraordinary felicity of arrangement maintains an order and a regular progression. As our horizon expands to reveal to us the gathering tempests which are forming far beyond the boundaries of the civilized world—as we follow their successive approach to the trembling frontier—the compressed and receding line is still distinctly visible; though gradually dismembered and the broken fragments assuming the form of regular states and kingdoms, the real relation of those kingdoms to the empire is maintained and defined; and even when the Roman dominion has shrunk into little more than the province of Thrace—when the name of Rome, confined, in Italy, to the walls of the city—yet it is still the memory, the shade of the Roman greatness, which extends over the wide sphere into which the historian expands his later narrative; the whole blends into the unity, and is manifestly essential to the double catastrophe of his tragic drama.

But the amplitude, the magnificence, or the harmony of design, are, though imposing, yet unworthy claims on our admiration, unless the details are filled up with correctness and accuracy. No writer has been more severely tried on this point than Gibbon. He has undergone the triple scrutiny of theological zeal quickened by just resentment, of literary emulation, and of that mean and invidious vanity which delights in detecting errors in writers of established fame. On the result of the trial, we may be permitted to summon competent witnesses before we deliver our own judgment.

M. Guizot, in his preface, after stating that in France and Germany, as well as in England, in the most enlightened countries of Europe, Gibbon is constantly cited as an authority, thus proceeds:—

“I have had occasion, during my labors, to consult the writings of philosophers, who have treated on the finances of the Roman empire; of scholars, who have investigated the chronology; of theologians, who have searched the depths of ecclesiastical history; of writers on law, who have studied with care the Roman jurisprudence; of Orientalists, who have occupied themselves with the Arabians and the Koran; of modern historians, who have entered upon extensive researches touching the crusades and their influence; each of these writers has remarked and pointed out, in the ‘History of the Decline and Fall of the Roman Empire,’ some negligences, some false or imperfect views, some omissions, which it is impossible not to suppose voluntary; they have rectified some facts, combated with advantage some assertions; but in general they have taken the researches and the ideas of Gibbon, as points of departure, or as proofs of the researches or of the new opinions which they have advanced.”

M. Guizot goes on to state his own impressions on reading Gibbon’s history, and no authority will have greater weight with those to whom the extent and accuracy of his historical researches are known:—

“After a first rapid perusal, which allowed me to feel nothing but the interest of a narrative, always animated, and, notwithstanding its extent and the variety of objects which it makes to pass before the view, always perspicuous, I entered upon a minute examination of the details of which it was composed; and the opinion which I then formed was, I confess, singularly severe. I discovered, in certain chapters, errors which appeared to me sufficiently important and numerous to make me believe that they had been written with extreme negligence; in others, I was struck with a certain tinge of partiality and prejudice, which imparted to the exposition of the facts that want of truth and justice, which the English express by their happy term \textit{misrepresentation}. Some imperfect (\textit{tronquées}) quotations; some passages, omitted unintentionally or designedly cast a suspicion on the honesty (\textit{bonne foi}) of the author; and his violation of the first law of history—increased to my eye by the prolonged attention with which I occupied myself with every phrase, every note, every reflection—caused me to form upon the whole work, a judgment far too rigorous. After having finished my labors, I allowed some time to elapse before I reviewed the whole. A second attentive and regular perusal of the entire work, of the notes of the author, and of those which I had thought it right to subjoin, showed me how much I had exaggerated the importance of the reproaches which Gibbon really deserved; I was struck with the same errors, the same partiality on certain subjects; but I had been far from doing adequate justice to the immensity of his researches, the variety of his knowledge, and above all, to that truly philosophical discrimination (\textit{justesse d’esprit}) which judges the past as it would judge the present; which does not permit itself to be blinded by the clouds which time gathers around the dead, and which prevent us from seeing that, under the toga, as under the modern dress, in the senate as in our councils, men were what they still are, and that events took place eighteen centuries ago, as they take place in our days. I then felt that his book, in spite of its faults, will always be a noble work—and that we may correct his errors and combat his prejudices, without ceasing to admit that few men have combined, if we are not to say in so high a degree, at least in a manner so complete, and so well regulated, the necessary qualifications for a writer of history.”

The present editor has followed the track of Gibbon through many parts of his work; he has read his authorities with constant reference to his pages, and must pronounce his deliberate judgment, in terms of the highest admiration as to his general accuracy. Many of his seeming errors are almost inevitable from the close condensation of his matter. From the immense range of his history, it was sometimes necessary to compress into a single sentence, a whole vague and diffuse page of a Byzantine chronicler. Perhaps something of importance may have thus escaped, and his expressions may not quite contain the whole substance of the passage from which they are taken. His limits, at times, compel him to sketch; where that is the case, it is not fair to expect the full details of the finished picture. At times he can only deal with important results; and in his account of a war, it sometimes requires great attention to discover that the events which seem to be comprehended in a single campaign, occupy several years. But this admirable skill in selecting and giving prominence to the points which are of real weight and importance—this distribution of light and shade—though perhaps it may occasionally betray him into vague and imperfect statements, is one of the highest excellencies of Gibbon’s historic manner. It is the more striking, when we pass from the works of his chief authorities, where, after laboring through long, minute, and wearisome descriptions of the accessary and subordinate circumstances, a single unmarked and undistinguished sentence, which we may overlook from the inattention of fatigue, contains the great moral and political result.

Gibbon’s method of arrangement, though on the whole most favorable to the clear comprehension of the events, leads likewise to apparent inaccuracy. That which we expect to find in one part is reserved for another. The estimate which we are to form, depends on the accurate balance of statements in remote parts of the work; and we have sometimes to correct and modify opinions, formed from one chapter by those of another. Yet, on the other hand, it is astonishing how rarely we detect contradiction; the mind of the author has already harmonized the whole result to truth and probability; the general impression is almost invariably the same. The quotations of Gibbon have likewise been called in question;—I have, \textit{in general}, been more inclined to admire their exactitude, than to complain of their indistinctness, or incompleteness. Where they are imperfect, it is commonly from the study of brevity, and rather from the desire of compressing the substance of his notes into pointed and emphatic sentences, than from dishonesty, or uncandid suppression of truth.

These observations apply more particularly to the accuracy and fidelity of the historian as to his facts; his inferences, of course, are more liable to exception. It is almost impossible to trace the line between unfairness and unfaithfulness; between intentional misrepresentation and undesigned false coloring. The relative magnitude and importance of events must, in some respect, depend upon the mind before which they are presented; the estimate of character, on the habits and feelings of the reader. Christians, like M. Guizot and ourselves, will see some things, and some persons, in a different light from the historian of the Decline and Fall. We may deplore the bias of his mind; we may ourselves be on our guard against the danger of being misled, and be anxious to warn less wary readers against the same perils; but we must not confound this secret and unconscious departure from truth, with the deliberate violation of that veracity which is the only title of an historian to our confidence. Gibbon, it may be fearlessly asserted, is rarely chargeable even with the suppression of any material fact, which bears upon individual character; he may, with apparently invidious hostility, enhance the errors and crimes, and disparage the virtues of certain persons; yet, in general, he leaves us the materials for forming a fairer judgment; and if he is not exempt from his own prejudices, perhaps we might write \textit{passions}, yet it must be candidly acknowledged, that his philosophical bigotry is not more unjust than the theological partialities of those ecclesiastical writers who were before in undisputed possession of this province of history.

We are thus naturally led to that great misrepresentation which pervades his history—his false estimate of the nature and influence of Christianity.

But on this subject some preliminary caution is necessary, lest that should be expected from a new edition, which it is impossible that it should completely accomplish. We must first be prepared with the only sound preservative against the false impression likely to be produced by the perusal of Gibbon; and we must see clearly the real cause of that false impression. The former of these cautions will be briefly suggested in its proper place, but it may be as well to state it, here, somewhat more at length. The art of Gibbon, or at least the unfair impression produced by his two memorable chapters, consists in his confounding together, in one indistinguishable mass, the \textit{origin} and \textit{apostolic} propagation of the new religion, with its \textit{later} progress. No argument for the divine authority of Christianity has been urged with greater force, or traced with higher eloquence, than that deduced from its primary development, explicable on no other hypothesis than a heavenly origin, and from its rapid extension through great part of the Roman empire. But this argument—one, when confined within reasonable limits, of unanswerable force—becomes more feeble and disputable in proportion as it recedes from the birthplace, as it were, of the religion. The further Christianity advanced, the more causes purely human were enlisted in its favor; nor can it be doubted that those developed with such artful exclusiveness by Gibbon did concur most essentially to its establishment. It is in the Christian dispensation, as in the material world. In both it is as the great First Cause, that the Deity is most undeniably manifest. When once launched in regular motion upon the bosom of space, and endowed with all their properties and relations of weight and mutual attraction, the heavenly bodies appear to pursue their courses according to secondary laws, which account for all their sublime regularity. So Christianity proclaims its Divine Author chiefly in its first origin and development. When it had once received its impulse from above—when it had once been infused into the minds of its first teachers—when it had gained full possession of the reason and affections of the favored few—it \textit{might be}—and to the Protestant, the rationa Christian, it is impossible to define \textit{when} it really \textit{was}—left to make its way by its native force, under the ordinary secret agencies of all-ruling Providence. The main question, the \textit{divine origin of the religion}, was dexterously eluded, or speciously conceded by Gibbon; his plan enabled him to commence his account, in most parts, \textit{below the apostolic times;} and it was only by the strength of the dark coloring with which he brought out the failings and the follies of the succeeding ages, that a shadow of doubt and suspicion was thrown back upon the primitive period of Christianity.

“The theologian,” says Gibbon, “may indulge the pleasing task of describing religion as she descended from heaven, arrayed in her native purity; a more melancholy duty is imposed upon the historian:—he must discover the inevitable mixture of error and corruption which she contracted in a long residence upon earth among a weak and degenerate race of beings.” Divest this passage of the latent sarcasm betrayed by the subsequent tone of the whole disquisition, and it might commence a Christian history written in the most Christian spirit of candor. But as the historian, by seeming to respect, yet by dexterously confounding the limits of the sacred land, contrived to insinuate that it was an Utopia which had no existence but in the imagination of the theologian—as he \textit{suggested} rather than affirmed that the days of Christian purity were a kind of poetic golden age;—so the theologian, by venturing too far into the domain of the historian, has been perpetually obliged to contest points on which he had little chance of victory—to deny facts established on unshaken evidence—and thence, to retire, if not with the shame of defeat, yet with but doubtful and imperfect success. Paley, with his intuitive sagacity, saw through the difficulty of answering Gibbon by the ordinary arts of controversy; his emphatic sentence, “Who can refute a sneer?” contains as much truth as point. But full and pregnant as this phrase is, it is not quite the whole truth; it is the tone in which the progress of Christianity is traced, in \textit{comparison} with the rest of the splendid and prodigally ornamented work, which is the radical defect in the “Decline and Fall.” Christianity alone receives no embellishment from the magic of Gibbon’s language; his imagination is dead to its moral dignity; it is kept down by a general zone of jealous disparagement, or neutralized by a painfully elaborate exposition of its darker and degenerate periods. There are occasions, indeed, when its pure and exalted humanity, when its manifestly beneficial influence, can compel even him, as it were, to fairness, and kindle his unguarded eloquence to its usual fervor; but, in general, he soon relapses into a frigid apathy; \textit{affects} an ostentatiously severe impartiality; notes all the faults of Christians in every age with bitter and almost malignant sarcasm; reluctantly, and with exception and reservation, admits their claim to admiration. This inextricable bias appears even to influence his manner of composition. While all the other assailants of the Roman empire, whether warlike or religious, the Goth, the Hun, the Arab, the Tartar, Alaric and Attila, Mahomet, and Zengis, and Tamerlane, are each introduced upon the scene almost with dramatic animation—their progress related in a full, complete, and unbroken narrative—the triumph of Christianity alone takes the form of a cold and critical disquisition. The successes of barbarous energy and brute force call forth all the consummate skill of composition; while the moral triumphs of Christian benevolence—the tranquil heroism of endurance, the blameless purity, the contempt of guilty fame and of honors destructive to the human race, which, had they assumed the proud name of philosophy, would have been blazoned in his brightest words, because they own religion as their principle—sink into narrow asceticism. The \textit{glories} of Christianity, in short, touch on no chord in the heart of the writer; his imagination remains unkindled; his words, though they maintain their stately and measured march, have become cool, argumentative, and inanimate. Who would obscure one hue of that gorgeous coloring in which Gibbon has invested the dying forms of Paganism, or darken one paragraph in his splendid view of the rise and progress of Mahometanism? But who would not have wished that the same equal justice had been done to Christianity; that its real character and deeply penetrating influence had been traced with the same philosophical sagacity, and represented with more sober, as would become its quiet course, and perhaps less picturesque, but still with lively and attractive, descriptiveness? He might have thrown aside, with the same scorn, the mass of ecclesiastical fiction which envelops the early history of the church, stripped off the legendary romance, and brought out the facts in their primitive nakedness and simplicity—if he had but allowed those facts the benefit of the glowing eloquence which he denied to them alone. He might have annihilated the whole fabric of post-apostolic miracles, if he had left uninjured by sarcastic insinuation those of the New Testament; he might have cashiered, with Dodwell, the whole host of martyrs, which owe their existence to the prodigal invention of later days, had he but bestowed fair room, and dwelt with his ordinary energy on the sufferings of the genuine witnesses to the truth of Christianity, the Polycarps, or the martyrs of Vienne. And indeed, if, after all, the view of the early progress of Christianity be melancholy and humiliating we must beware lest we charge the whole of this on the infidelity of the historian. It is idle, it is disingenuous, to deny or to dissemble the early depravations of Christianity, its gradual but rapid departure from its primitive simplicity and purity, still more, from its spirit of universal love. It may be no unsalutary lesson to the Christian world, that this silent, this unavoidable, perhaps, yet fatal change shall have been drawn by an impartial, or even an hostile hand. The Christianity of every age may take warning, lest by its own narrow views, its want of wisdom, and its want of charity, it give the same advantage to the future unfriendly historian, and disparage the cause of true religion.

The design of the present edition is partly corrective, partly supplementary: corrective, by notes, which point out (it is hoped, in a perfectly candid and dispassionate spirit with no desire but to establish the truth) such inaccuracies or misstatements as may have been detected, particularly with regard to Christianity; and which thus, with the previous caution, may counteract to a considerable extent the unfair and unfavorable impression created against rational religion: supplementary, by adding such additional information as the editor’s reading may have been able to furnish, from original documents or books, not accessible at the time when Gibbon wrote.

The work originated in the editor’s habit of noting on the margin of his copy of Gibbon references to such authors as had discovered errors, or thrown new light on the subjects treated by Gibbon. These had grown to some extent, and seemed to him likely to be of use to others. The annotations of M. Guizot also appeared to him worthy of being better known to the English public than they were likely to be, as appended to the French translation.

The chief works from which the editor has derived his materials are, I. The French translation, with notes by M. Guizot; 2d edition, Paris, 1828. The editor has translated almost all the notes of M. Guizot. Where he has not altogether agreed with him, his respect for the learning and judgment of that writer has, in general, induced him to retain the statement from which he has ventured to differ, with the grounds on which he formed his own opinion. In the notes on Christianity, he has retained all those of M. Guizot, with his own, from the conviction, that on such a subject, to many, the authority of a French statesman, a Protestant, and a rational and sincere Christian, would appear more independent and unbiassed, and therefore be more commanding, than that of an English clergyman.

The editor has not scrupled to transfer the notes of M. Guizot to the present work. The well-known zeal for knowledge, displayed in all the writings of that distinguished historian, has led to the natural inference, that he would not be displeased at the attempt to make them of use to the English readers of Gibbon. The notes of M. Guizot are signed with the letter G.

II. The German translation, with the notes of Wenck. Unfortunately this learned translator died, after having completed only the first volume; the rest of the work was executed by a very inferior hand.

The notes of Wenck are extremely valuable; many of them have been adopted by M. Guizot; they are distinguished by the letter W.\footnotemark[102]

\footnotetext[102]{The editor regrets that he has not been able to find the Italian translation, mentioned by Gibbon himself with some respect. It is not in our great libraries, the Museum or the Bodleian; and he has never found any bookseller in London who has seen it.}

III. The new edition of Le Beau’s “Histoire du Bas Empire, with notes by M. St. Martin, and M. Brosset.” That distinguished Armenian scholar, M. St. Martin (now, unhappily, deceased) had added much information from Oriental writers, particularly from those of Armenia, as well as from more general sources. Many of his observations have been found as applicable to the work of Gibbon as to that of Le Beau.

IV. The editor has consulted the various answers made to Gibbon on the first appearance of his work; he must confess, with little profit. They were, in general, hastily compiled by inferior and now forgotten writers, with the exception of Bishop Watson, whose able apology is rather a general argument, than an examination of misstatements. The name of Milner stands higher with a certain class of readers, but will not carry much weight with the severe investigator of history.

V. Some few classical works and fragments have come to light, since the appearance of Gibbon’s History, and have been noticed in their respective places; and much use has been made, in the latter volumes particularly, of the increase to our stores of Oriental literature. The editor cannot, indeed, pretend to have followed his author, in these gleanings, over the whole vast field of his inquiries; he may have overlooked or may not have been able to command some works, which might have thrown still further light on these subjects; but he trusts that what he has adduced will be of use to the student of historic truth.

The editor would further observe, that with regard to some other objectionable passages, which do not involve misstatement or inaccuracy, he has intentionally abstained from directing particular attention towards them by any special protest.

The editor’s notes are marked M.

A considerable part of the quotations (some of which in the later editions had fallen into great confusion) have been verified, and have been corrected by the latest and best editions of the authors.

June, 1845.

In this new edition, the text and the notes have been carefully revised, the latter by the editor.

Some additional notes have been subjoined, distinguished by the signature M. 1845.

\subsection{Preface Of The Author.}
\markboth{Introduction}{Author's Preface}

It is not my intention to detain the reader by expatiating on the variety or the importance of the subject, which I have undertaken to treat; since the merit of the choice would serve to render the weakness of the execution still more apparent, and still less excusable. But as I have presumed to lay before the public a \textit{first} volume only\footnotemark[1] of the History of the Decline and Fall of the Roman Empire, it will, perhaps, be expected that I should explain, in a few words, the nature and limits of my general plan.

\footnotetext[1]{The first volume of the quarto, which contained the sixteen first chapters.}

The memorable series of revolutions, which in the course of about thirteen centuries gradually undermined, and at length destroyed, the solid fabric of human greatness, may, with some propriety, be divided into the three following periods:

I. The first of these periods may be traced from the age of Trajan and the Antonines, when the Roman monarchy, having attained its full strength and maturity, began to verge towards its decline; and will extend to the subversion of the Western Empire, by the barbarians of Germany and Scythia, the rude ancestors of the most polished nations of modern Europe. This extraordinary revolution, which subjected Rome to the power of a Gothic conqueror, was completed about the beginning of the sixth century.

II. The second period of the Decline and Fall of Rome may be supposed to commence with the reign of Justinian, who, by his laws, as well as by his victories, restored a transient splendor to the Eastern Empire. It will comprehend the invasion of Italy by the Lombards; the conquest of the Asiatic and African provinces by the Arabs, who embraced the religion of Mahomet; the revolt of the Roman people against the feeble princes of Constantinople; and the elevation of Charlemagne, who, in the year eight hundred, established the second, or German Empire of the West.

III. The last and longest of these periods includes about six centuries and a half; from the revival of the Western Empire, till the taking of Constantinople by the Turks, and the extinction of a degenerate race of princes, who continued to assume the titles of Cæsar and Augustus, after their dominions were contracted to the limits of a single city; in which the language, as well as manners, of the ancient Romans, had been long since forgotten. The writer who should undertake to relate the events of this period, would find himself obliged to enter into the general history of the Crusades, as far as they contributed to the ruin of the Greek Empire; and he would scarcely be able to restrain his curiosity from making some inquiry into the state of the city of Rome, during the darkness and confusion of the middle ages.

As I have ventured, perhaps too hastily, to commit to the press a work which in every sense of the word, deserves the epithet of imperfect. I consider myself as contracting an engagement to finish, most probably in a second volume,\footnotemark[2] the first of these memorable periods; and to deliver to the Public the complete History of the Decline and Fall of Rome, from the age of the Antonines to the subversion of the Western Empire. With regard to the subsequent periods, though I may entertain some hopes, I dare not presume to give any assurances. The execution of the extensive plan which I have described, would connect the ancient and modern history of the world; but it would require many years of health, of leisure, and of perseverance.

\footnotetext[2] {The Author, as it frequently happens, took an inadequate measure of his growing work. The remainder of the first period has filled \textit{two} volumes in quarto, being the third, fourth, fifth, and sixth volumes of the octavo edition.}

BENTINCK STREET, \textit{February} 1, 1776.

P. S. The entire History, which is now published, of the Decline and Fall of the Roman Empire in the West, abundantly discharges my engagements with the Public. Perhaps their favorable opinion may encourage me to prosecute a work, which, however laborious it may seem, is the most agreeable occupation of my leisure hours.

BENTINCK STREET, \textit{March} 1, 1781.

An Author easily persuades himself that the public opinion is still favorable to his labors; and I have now embraced the serious resolution of proceeding to the last period of my original design, and of the Roman Empire, the taking of Constantinople by the Turks, in the year one thousand four hundred and fifty-three. The most patient Reader, who computes that three ponderous\footnotemark[3] volumes have been already employed on the events of four centuries, may, perhaps, be alarmed at the long prospect of nine hundred years. But it is not my intention to expatiate with the same minuteness on the whole series of the Byzantine history. At our entrance into this period, the reign of Justinian, and the conquests of the Mahometans, will deserve and detain our attention, and the last age of Constantinople (the Crusades and the Turks) is connected with the revolutions of Modern Europe. From the seventh to the eleventh century, the obscure interval will be supplied by a concise narrative of such facts as may still appear either interesting or important.

BENTINCK STREET, \textit{March} 1, 1782.

\footnotetext[3]{The first six volumes of the octavo edition.}


\subsection{Preface To The First Volume.}
\markboth{Introduction}{Preface to Vol I.}


Diligence and accuracy are the only merits which an historical writer may ascribe to himself; if any merit, indeed, can be assumed from the performance of an indispensable duty. I may therefore be allowed to say, that I have carefully examined all the original materials that could illustrate the subject which I had undertaken to treat. Should I ever complete the extensive design which has been sketched out in the Preface, I might perhaps conclude it with a critical account of the authors consulted during the progress of the whole work; and however such an attempt might incur the censure of ostentation, I am persuaded that it would be susceptible of entertainment, as well as information.

At present I shall content myself with a single observation.

The biographers, who, under the reigns of Diocletian and Constantine, composed, or rather compiled, the lives of the Emperors, from Hadrian to the sons of Carus, are usually mentioned under the names of Ælius Spartianus, Julius Capitolinus, Ælius Lampridius, Vulcatius Gallicanus, Trebellius Pollio and Flavius Vopiscus. But there is so much perplexity in the titles of the MSS., and so many disputes have arisen among the critics (see Fabricius, Biblioth. Latin. l. iii. c. 6) concerning their number, their names, and their respective property, that for the most part I have quoted them without distinction, under the general and well-known title of the \textit{Augustan History}.


\subsection{Preface To The Fourth Volume Of The Original Quarto Edition.}
\markboth{Introduction}{Preface to Quarto Vol IV.}

I now discharge my promise, and complete my design, of writing the History of the Decline and Fall of the Roman Empire, both in the West and the East. The whole period extends from the age of Trajan and the Antonines, to the taking of Constantinople by Mahomet the Second; and includes a review of the Crusades, and the state of Rome during the middle ages. Since the publication of the first volume, twelve years have elapsed; twelve years, according to my wish, “of health, of leisure, and of perseverance.” I may now congratulate my deliverance from a long and laborious service, and my satisfaction will be pure and perfect, if the public favor should be extended to the conclusion of my work.

It was my first intention to have collected, under one view, the numerous authors, of every age and language, from whom I have derived the materials of this history; and I am still convinced that the apparent ostentation would be more than compensated by real use. If I have renounced this idea, if I have declined an undertaking which had obtained the approbation of a master-artist,\footnotemark[4] my excuse may be found in the extreme difficulty of assigning a proper measure to such a catalogue. A naked list of names and editions would not be satisfactory either to myself or my readers: the characters of the principal Authors of the Roman and Byzantine History have been occasionally connected with the events which they describe; a more copious and critical inquiry might indeed deserve, but it would demand, an elaborate volume, which might swell by degrees into a general library of historical writers. For the present, I shall content myself with renewing my serious protestation, that I have always endeavored to draw from the fountain-head; that my curiosity, as well as a sense of duty, has always urged me to study the originals; and that, if they have sometimes eluded my search, I have carefully marked the secondary evidence, on whose faith a passage or a fact were reduced to depend.

\footnotetext[4]{See Dr. Robertson’s Preface to his History of America.}

  I shall soon revisit the banks of the Lake of Lausanne, a country which I have known and loved from my early youth. Under a mild government, amidst a beauteous landscape, in a life of leisure and independence, and among a people of easy and elegant manners, I have enjoyed, and may again hope to enjoy, the varied pleasures of retirement and society. But I shall ever glory in the name and character of an Englishman: I am proud of my birth in a free and enlightened country; and the approbation of that country is the best and most honorable reward of my labors. Were I ambitious of any other Patron than the Public, I would inscribe this work to a Statesman, who, in a long, a stormy, and at length an unfortunate administration, had many political opponents, almost without a personal enemy; who has retained, in his fall from power, many faithful and disinterested friends; and who, under the pressure of severe infirmity, enjoys the lively vigor of his mind, and the felicity of his incomparable temper. Lord North will permit me to express the feelings of friendship in the language of truth: but even truth and friendship should be silent, if he still dispensed the favors of the crown.

In a remote solitude, vanity may still whisper in my ear, that my readers, perhaps, may inquire whether, in the conclusion of the present work, I am now taking an everlasting farewell. They shall hear all that I know myself, and all that I could reveal to the most intimate friend. The motives of action or silence are now equally balanced; nor can I pronounce, in my most secret thoughts, on which side the scale will preponderate. I cannot dissemble that six quartos must have tried, and may have exhausted, the indulgence of the Public; that, in the repetition of similar attempts, a successful Author has much more to lose than he can hope to gain; that I am now descending into the vale of years; and that the most respectable of my countrymen, the men whom I aspire to imitate, have resigned the pen of history about the same period of their lives. Yet I consider that the annals of ancient and modern times may afford many rich and interesting subjects; that I am still possessed of health and leisure; that by the practice of writing, some skill and facility must be acquired; and that, in the ardent pursuit of truth and knowledge, I am not conscious of decay. To an active mind, indolence is more painful than labor; and the first months of my liberty will be occupied and amused in the excursions of curiosity and taste. By such temptations, I have been sometimes seduced from the rigid duty even of a pleasing and voluntary task: but my time will now be my own; and in the use or abuse of independence, I shall no longer fear my own reproaches or those of my friends. I am fairly entitled to a year of jubilee: next summer and the following winter will rapidly pass away; and experience only can determine whether I shall still prefer the freedom and variety of study to the design and composition of a regular work, which animates, while it confines, the daily application of the Author.

Caprice and accident may influence my choice; but the dexterity of self-love will contrive to applaud either active industry or philosophic repose.

DOWNING STREET, \textit{May} 1, 1788.

P. S. I shall embrace this opportunity of introducing two \textit{verbal} remarks, which have not conveniently offered themselves to my notice. 1. As often as I use the definitions of \textit{beyond} the Alps, the Rhine, the Danube, \&c., I generally suppose myself at Rome, and afterwards at Constantinople; without observing whether this relative geography may agree with the local, but variable, situation of the reader, or the historian. 2. In proper names of foreign, and especially of Oriental origin, it should be always our aim to express, in our English version, a faithful copy of the original. But this rule, which is founded on a just regard to uniformity and truth, must often be relaxed; and the exceptions will be limited or enlarged by the custom of the language and the taste of the interpreter. Our alphabets may be often defective; a harsh sound, an uncouth spelling, might offend the ear or the eye of our countrymen; and some words, notoriously corrupt, are fixed, and, as it were, naturalized in the vulgar tongue. The prophet \textit{Mohammed} can no longer be stripped of the famous, though improper, appellation of Mahomet: the well-known cities of Aleppo, Damascus, and Cairo, would almost be lost in the strange descriptions of \textit{Haleb, Demashk}, and \textit{Al Cahira:} the titles and offices of the Ottoman empire are fashioned by the practice of three hundred years; and we are pleased to blend the three Chinese monosyllables, \textit{Con-fû-tzee}, in the respectable name of Confucius, or even to adopt the Portuguese corruption of Mandarin. But I would vary the use of Zoroaster and \textit{Zerdusht}, as I drew my information from Greece or Persia: since our connection with India, the genuine \textit{Timour} is restored to the throne of Tamerlane: our most correct writers have retrenched the \textit{Al}, the superfluous article, from the Koran; and we escape an ambiguous termination, by adopting \textit{Moslem} instead of Musulman, in the plural number. In these, and in a thousand examples, the shades of distinction are often minute; and I can feel, where I cannot explain, the motives of my choice.

