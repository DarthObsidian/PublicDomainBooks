\section{Part \thesection.}
\thispagestyle{simple}

The camp of a Roman legion presented the appearance of a
fortified city.\footnotemark[60] As soon as the space was marked out, the
pioneers carefully levelled the ground, and removed every
impediment that might interrupt its perfect regularity. Its form
was an exact quadrangle; and we may calculate, that a square of
about seven hundred yards was sufficient for the encampment of
twenty thousand Romans; though a similar number of our own troops
would expose to the enemy a front of more than treble that
extent. In the midst of the camp, the prætorium, or general’s
quarters, rose above the others; the cavalry, the infantry, and
the auxiliaries occupied their respective stations; the streets
were broad and perfectly straight, and a vacant space of two
hundred feet was left on all sides between the tents and the
rampart. The rampart itself was usually twelve feet high, armed
with a line of strong and intricate palisades, and defended by a
ditch of twelve feet in depth as well as in breadth. This
important labor was performed by the hands of the legionaries
themselves; to whom the use of the spade and the pickaxe was no
less familiar than that of the sword or \textit{pilum}. Active valor may
often be the present of nature; but such patient diligence can be
the fruit only of habit and discipline.\footnotemark[61]

\footnotetext[60]{Vegetius finishes his second book, and the
description of the legion, with the following emphatic
words:—“Universa quæ in quoque belli genere necessaria esse
creduntur, secum legio debet ubique portare, ut in quovis loco
fixerit castra, armatam faciat civitatem.”}

\footnotetext[61]{For the Roman Castrametation, see Polybius, l. vi.
with Lipsius de Militia Romana, Joseph. de Bell. Jud. l. iii. c.
5. Vegetius, i. 21—25, iii. 9, and Memoires de Guichard, tom. i.
c. 1.}

Whenever the trumpet gave the signal of departure, the camp was
almost instantly broke up, and the troops fell into their ranks
without delay or confusion. Besides their arms, which the
legionaries scarcely considered as an encumbrance, they were
laden with their kitchen furniture, the instruments of
fortification, and the provision of many days.\footnotemark[62] Under this
weight, which would oppress the delicacy of a modern soldier,
they were trained by a regular step to advance, in about six
hours, near twenty miles.\footnotemark[63] On the appearance of an enemy, they
threw aside their baggage, and by easy and rapid evolutions
converted the column of march into an order of battle.\footnotemark[64] The
slingers and archers skirmished in the front; the auxiliaries
formed the first line, and were seconded or sustained by the
strength of the legions; the cavalry covered the flanks, and the
military engines were placed in the rear.

\footnotetext[62]{Cicero in Tusculan. ii. 37, [15.]—Joseph. de Bell.
Jud. l. iii. 5, Frontinus, iv. 1.}

\footnotetext[63]{Vegetius, i. 9. See Memoires de l’Academie des
Inscriptions, tom. xxv. p. 187.}

\footnotetext[64]{See those evolutions admirably well explained by M.
Guichard Nouveaux Memoires, tom. i. p. 141—234.}

Such were the arts of war, by which the Roman emperors defended
their extensive conquests, and preserved a military spirit, at a
time when every other virtue was oppressed by luxury and
despotism. If, in the consideration of their armies, we pass from
their discipline to their numbers, we shall not find it easy to
define them with any tolerable accuracy. We may compute, however,
that the legion, which was itself a body of six thousand eight
hundred and thirty-one Romans, might, with its attendant
auxiliaries, amount to about twelve thousand five hundred men.
The peace establishment of Hadrian and his successors was
composed of no less than thirty of these formidable brigades; and
most probably formed a standing force of three hundred and
seventy-five thousand men. Instead of being confined within the
walls of fortified cities, which the Romans considered as the
refuge of weakness or pusillanimity, the legions were encamped on
the banks of the great rivers, and along the frontiers of the
barbarians. As their stations, for the most part, remained fixed
and permanent, we may venture to describe the distribution of the
troops. Three legions were sufficient for Britain. The principal
strength lay upon the Rhine and Danube, and consisted of sixteen
legions, in the following proportions: two in the Lower, and
three in the Upper Germany; one in Rhætia, one in Noricum, four
in Pannonia, three in Mæsia, and two in Dacia. The defence of the
Euphrates was intrusted to eight legions, six of whom were
planted in Syria, and the other two in Cappadocia. With regard to
Egypt, Africa, and Spain, as they were far removed from any
important scene of war, a single legion maintained the domestic
tranquillity of each of those great provinces. Even Italy was not
left destitute of a military force. Above twenty thousand chosen
soldiers, distinguished by the titles of City Cohorts and
Prætorian Guards, watched over the safety of the monarch and the
capital. As the authors of almost every revolution that
distracted the empire, the Prætorians will, very soon, and very
loudly, demand our attention; but, in their arms and
institutions, we cannot find any circumstance which discriminated
them from the legions, unless it were a more splendid appearance,
and a less rigid discipline.\footnotemark[65]

\footnotetext[65]{Tacitus (Annal. iv. 5) has given us a state of the
legions under Tiberius; and Dion Cassius (l. lv. p. 794) under
Alexander Severus. I have endeavored to fix on the proper medium
between these two periods. See likewise Lipsius de Magnitudine
Romana, l. i. c. 4, 5.}

The navy maintained by the emperors might seem inadequate to
their greatness; but it was fully sufficient for every useful
purpose of government. The ambition of the Romans was confined to
the land; nor was that warlike people ever actuated by the
enterprising spirit which had prompted the navigators of Tyre, of
Carthage, and even of Marseilles, to enlarge the bounds of the
world, and to explore the most remote coasts of the ocean. To the
Romans the ocean remained an object of terror rather than of
curiosity;\footnotemark[66] the whole extent of the Mediterranean, after the
destruction of Carthage, and the extirpation of the pirates, was
included within their provinces. The policy of the emperors was
directed only to preserve the peaceful dominion of that sea, and
to protect the commerce of their subjects. With these moderate
views, Augustus stationed two permanent fleets in the most
convenient ports of Italy, the one at Ravenna, on the Adriatic,
the other at Misenum, in the Bay of Naples. Experience seems at
length to have convinced the ancients, that as soon as their
galleys exceeded two, or at the most three ranks of oars, they
were suited rather for vain pomp than for real service. Augustus
himself, in the victory of Actium, had seen the superiority of
his own light frigates (they were called Liburnians) over the
lofty but unwieldy castles of his rival.\footnotemark[67] Of these Liburnians
he composed the two fleets of Ravenna and Misenum, destined to
command, the one the eastern, the other the western division of
the Mediterranean; and to each of the squadrons he attached a
body of several thousand marines. Besides these two ports, which
may be considered as the principal seats of the Roman navy, a
very considerable force was stationed at Frejus, on the coast of
Provence, and the Euxine was guarded by forty ships, and three
thousand soldiers. To all these we add the fleet which preserved
the communication between Gaul and Britain, and a great number of
vessels constantly maintained on the Rhine and Danube, to harass
the country, or to intercept the passage of the barbarians.\footnotemark[68] If
we review this general state of the Imperial forces; of the
cavalry as well as infantry; of the legions, the auxiliaries, the
guards, and the navy; the most liberal computation will not allow
us to fix the entire establishment by sea and by land at more
than four hundred and fifty thousand men: a military power,
which, however formidable it may seem, was equalled by a monarch
of the last century, whose kingdom was confined within a single
province of the Roman empire.\footnotemark[69]

\footnotetext[66]{The Romans tried to disguise, by the pretence of
religious awe their ignorance and terror. See Tacit. Germania, c.
34.}

\footnotetext[67]{Plutarch, in Marc. Anton. [c. 67.] And yet, if we
may credit Orosius, these monstrous castles were no more than ten
feet above the water, vi. 19.}

\footnotetext[68]{See Lipsius, de Magnitud. Rom. l. i. c. 5. The
sixteen last chapters of Vegetius relate to naval affairs.}

\footnotetext[69]{Voltaire, Siecle de Louis XIV. c. 29. It must,
however, be remembered, that France still feels that
extraordinary effort.}

We have attempted to explain the spirit which moderated, and the
strength which supported, the power of Hadrian and the Antonines.
We shall now endeavor, with clearness and precision, to describe
the provinces once united under their sway, but, at present,
divided into so many independent and hostile states. Spain, the
western extremity of the empire, of Europe, and of the ancient
world, has, in every age, invariably preserved the same natural
limits; the Pyrenæan Mountains, the Mediterranean, and the
Atlantic Ocean. That great peninsula, at present so unequally
divided between two sovereigns, was distributed by Augustus into
three provinces, Lusitania, Bætica, and Tarraconensis. The
kingdom of Portugal now fills the place of the warlike country of
the Lusitanians; and the loss sustained by the former on the side
of the East, is compensated by an accession of territory towards
the North. The confines of Grenada and Andalusia correspond with
those of ancient Bætica. The remainder of Spain, Gallicia, and
the Asturias, Biscay, and Navarre, Leon, and the two Castiles,
Murcia, Valencia, Catalonia, and Arragon, all contributed to form
the third and most considerable of the Roman governments, which,
from the name of its capital, was styled the province of
Tarragona.\footnotemark[70] Of the native barbarians, the Celtiberians were the
most powerful, as the Cantabrians and Asturians proved the most
obstinate. Confident in the strength of their mountains, they
were the last who submitted to the arms of Rome, and the first
who threw off the yoke of the Arabs.

\footnotetext[70]{See Strabo, l. ii. It is natural enough to suppose,
that Arragon is derived from Tarraconensis, and several moderns
who have written in Latin use those words as synonymous. It is,
however, certain, that the Arragon, a little stream which falls
from the Pyrenees into the Ebro, first gave its name to a
country, and gradually to a kingdom. See d’Anville, Geographie du
Moyen Age, p. 181.}

Ancient Gaul, as it contained the whole country between the
Pyrenees, the Alps, the Rhine, and the Ocean, was of greater
extent than modern France. To the dominions of that powerful
monarchy, with its recent acquisitions of Alsace and Lorraine, we
must add the duchy of Savoy, the cantons of Switzerland, the four
electorates of the Rhine, and the territories of Liege,
Luxemburgh, Hainault, Flanders, and Brabant. When Augustus gave
laws to the conquests of his father, he introduced a division of
Gaul, equally adapted to the progress of the legions, to the
course of the rivers, and to the principal national distinctions,
which had comprehended above a hundred independent states.\footnotemark[71] The
sea-coast of the Mediterranean, Languedoc, Provence, and
Dauphiné, received their provincial appellation from the colony
of Narbonne. The government of Aquitaine was extended from the
Pyrenees to the Loire. The country between the Loire and the
Seine was styled the Celtic Gaul, and soon borrowed a new
denomination from the celebrated colony of Lugdunum, or Lyons.
The Belgic lay beyond the Seine, and in more ancient times had
been bounded only by the Rhine; but a little before the age of
Cæsar, the Germans, abusing their superiority of valor, had
occupied a considerable portion of the Belgic territory. The
Roman conquerors very eagerly embraced so flattering a
circumstance, and the Gallic frontier of the Rhine, from Basil to
Leyden, received the pompous names of the Upper and the Lower
Germany.\footnotemark[72] Such, under the reign of the Antonines, were the six
provinces of Gaul; the Narbonnese, Aquitaine, the Celtic, or
Lyonnese, the Belgic, and the two Germanies.

\footnotetext[71]{One hundred and fifteen \textit{cities} appear in the
Notitia of Gaul; and it is well known that this appellation was
applied not only to the capital town, but to the whole territory
of each state. But Plutarch and Appian increase the number of
tribes to three or four hundred.}

\footnotetext[72]{D’Anville. Notice de l’Ancienne Gaule.}

We have already had occasion to mention the conquest of Britain,
and to fix the boundary of the Roman Province in this island. It
comprehended all England, Wales, and the Lowlands of Scotland, as
far as the Friths of Dumbarton and Edinburgh. Before Britain lost
her freedom, the country was irregularly divided between thirty
tribes of barbarians, of whom the most considerable were the
Belgæ in the West, the Brigantes in the North, the Silures in
South Wales, and the Iceni in Norfolk and Suffolk.\footnotemark[73] As far as
we can either trace or credit the resemblance of manners and
language, Spain, Gaul, and Britain were peopled by the same hardy
race of savages. Before they yielded to the Roman arms, they
often disputed the field, and often renewed the contest. After
their submission, they constituted the western division of the
European provinces, which extended from the columns of Hercules
to the wall of Antoninus, and from the mouth of the Tagus to the
sources of the Rhine and Danube.

\footnotetext[73]{Whittaker’s History of Manchester, vol. i. c. 3.}
Before the Roman conquest, the country which is now called
Lombardy, was not considered as a part of Italy. It had been
occupied by a powerful colony of Gauls, who, settling themselves
along the banks of the Po, from Piedmont to Romagna, carried
their arms and diffused their name from the Alps to the Apennine.

The Ligurians dwelt on the rocky coast which now forms the
republic of Genoa. Venice was yet unborn; but the territories of
that state, which lie to the east of the Adige, were inhabited by
the Venetians.\footnotemark[74] The middle part of the peninsula, that now
composes the duchy of Tuscany and the ecclesiastical state, was
the ancient seat of the Etruscans and Umbrians; to the former of
whom Italy was indebted for the first rudiments of civilized
life.\footnotemark[75] The Tyber rolled at the foot of the seven hills of Rome,
and the country of the Sabines, the Latins, and the Volsci, from
that river to the frontiers of Naples, was the theatre of her
infant victories. On that celebrated ground the first consuls
deserved triumphs, their successors adorned villas, and \textit{their}
posterity have erected convents.\footnotemark[76] Capua and Campania possessed
the immediate territory of Naples; the rest of the kingdom was
inhabited by many warlike nations, the Marsi, the Samnites, the
Apulians, and the Lucanians; and the sea-coasts had been covered
by the flourishing colonies of the Greeks. We may remark, that
when Augustus divided Italy into eleven regions, the little
province of Istria was annexed to that seat of Roman sovereignty.\footnotemark[77]

\footnotetext[74]{The Italian Veneti, though often confounded with
the Gauls, were more probably of Illyrian origin. See M. Freret,
Mémoires de l’Académie des Inscriptions, tom. xviii. * Note: Or
Liburnian, according to Niebuhr. Vol. i. p. 172.—M.}

\footnotetext[75]{See Maffei Verona illustrata, l. i. * Note: Add
Niebuhr, vol. i., and Otfried Müller, \textit{die Etrusker}, which
contains much that is known, and much that is conjectured, about
this remarkable people. Also Micali, Storia degli antichi popoli
Italiani. Florence, 1832—M.}

\footnotetext[76]{The first contrast was observed by the ancients.
See Florus, i. 11. The second must strike every modern
traveller.}

\footnotetext[77]{Pliny (Hist. Natur. l. iii.) follows the division
of Italy by Augustus.}

The European provinces of Rome were protected by the course of
the Rhine and the Danube. The latter of those mighty streams,
which rises at the distance of only thirty miles from the former,
flows above thirteen hundred miles, for the most part to the
south-east, collects the tribute of sixty navigable rivers, and
is, at length, through six mouths, received into the Euxine,
which appears scarcely equal to such an accession of waters.\footnotemark[78]
The provinces of the Danube soon acquired the general appellation
of Illyricum, or the Illyrian frontier,\footnotemark[79] and were esteemed the
most warlike of the empire; but they deserve to be more
particularly considered under the names of Rhætia, Noricum,
Pannonia, Dalmatia, Dacia, Mæsia, Thrace, Macedonia, and Greece.

\footnotetext[78]{Tournefort, Voyages en Grece et Asie Mineure,
lettre xviii.}

\footnotetext[79]{The name of Illyricum originally belonged to the
sea-coast of the Adriatic, and was gradually extended by the
Romans from the Alps to the Euxine Sea. See Severini Pannonia, l.
i. c. 3.}

The province of Rhætia, which soon extinguished the name of the
Vindelicians, extended from the summit of the Alps to the banks
of the Danube; from its source, as far as its conflux with the
Inn. The greatest part of the flat country is subject to the
elector of Bavaria; the city of Augsburg is protected by the
constitution of the German empire; the Grisons are safe in their
mountains, and the country of Tirol is ranked among the numerous
provinces of the house of Austria.

The wide extent of territory which is included between the Inn,
the Danube, and the Save,—Austria, Styria, Carinthia, Carniola,
the Lower Hungary, and Sclavonia,—was known to the ancients under
the names of Noricum and Pannonia. In their original state of
independence, their fierce inhabitants were intimately connected.
Under the Roman government they were frequently united, and they
still remain the patrimony of a single family. They now contain
the residence of a German prince, who styles himself Emperor of
the Romans, and form the centre, as well as strength, of the
Austrian power. It may not be improper to observe, that if we
except Bohemia, Moravia, the northern skirts of Austria, and a
part of Hungary between the Teyss and the Danube, all the other
dominions of the House of Austria were comprised within the
limits of the Roman Empire.

Dalmatia, to which the name of Illyricum more properly belonged,
was a long, but narrow tract, between the Save and the Adriatic.
The best part of the sea-coast, which still retains its ancient
appellation, is a province of the Venetian state, and the seat of
the little republic of Ragusa. The inland parts have assumed the
Sclavonian names of Croatia and Bosnia; the former obeys an
Austrian governor, the latter a Turkish pacha; but the whole
country is still infested by tribes of barbarians, whose savage
independence irregularly marks the doubtful limit of the
Christian and Mahometan power.\footnotemark[80]

\footnotetext[80]{A Venetian traveller, the Abbate Fortis, has lately
given us some account of those very obscure countries. But the
geography and antiquities of the western Illyricum can be
expected only from the munificence of the emperor, its
sovereign.}

After the Danube had received the waters of the Teyss and the
Save, it acquired, at least among the Greeks, the name of Ister.\footnotemark[81]
It formerly divided Mæsia and Dacia, the latter of which, as
we have already seen, was a conquest of Trajan, and the only
province beyond the river. If we inquire into the present state
of those countries, we shall find that, on the left hand of the
Danube, Temeswar and Transylvania have been annexed, after many
revolutions, to the crown of Hungary; whilst the principalities
of Moldavia and Wallachia acknowledge the supremacy of the
Ottoman Porte. On the right hand of the Danube, Mæsia, which,
during the middle ages, was broken into the barbarian kingdoms of
Servia and Bulgaria, is again united in Turkish slavery.

\footnotetext[81]{The Save rises near the confines of \textit{Istria}, and
was considered by the more early Greeks as the principal stream
of the Danube.}

The appellation of Roumelia, which is still bestowed by the Turks
on the extensive countries of Thrace, Macedonia, and Greece,
preserves the memory of their ancient state under the Roman
empire. In the time of the Antonines, the martial regions of
Thrace, from the mountains of Hæmus and Rhodope, to the Bosphorus
and the Hellespont, had assumed the form of a province.
Notwithstanding the change of masters and of religion, the new
city of Rome, founded by Constantine on the banks of the
Bosphorus, has ever since remained the capital of a great
monarchy. The kingdom of Macedonia, which, under the reign of
Alexander, gave laws to Asia, derived more solid advantages from
the policy of the two Philips; and with its dependencies of
Epirus and Thessaly, extended from the Ægean to the Ionian Sea.
When we reflect on the fame of Thebes and Argos, of Sparta and
Athens, we can scarcely persuade ourselves, that so many immortal
republics of ancient Greece were lost in a single province of the
Roman empire, which, from the superior influence of the Achæan
league, was usually denominated the province of Achaia.

Such was the state of Europe under the Roman emperors. The
provinces of Asia, without excepting the transient conquests of
Trajan, are all comprehended within the limits of the Turkish
power. But, instead of following the arbitrary divisions of
despotism and ignorance, it will be safer for us, as well as more
agreeable, to observe the indelible characters of nature. The
name of Asia Minor is attributed with some propriety to the
peninsula, which, confined betwixt the Euxine and the
Mediterranean, advances from the Euphrates towards Europe. The
most extensive and flourishing district, westward of Mount Taurus
and the River Halys, was dignified by the Romans with the
exclusive title of Asia. The jurisdiction of that province
extended over the ancient monarchies of Troy, Lydia, and Phrygia,
the maritime countries of the Pamphylians, Lycians, and Carians,
and the Grecian colonies of Ionia, which equalled in arts, though
not in arms, the glory of their parent. The kingdoms of Bithynia
and Pontus possessed the northern side of the peninsula from
Constantinople to Trebizond. On the opposite side, the province
of Cilicia was terminated by the mountains of Syria: the inland
country, separated from the Roman Asia by the River Halys, and
from Armenia by the Euphrates, had once formed the independent
kingdom of Cappadocia. In this place we may observe, that the
northern shores of the Euxine, beyond Trebizond in Asia, and
beyond the Danube in Europe, acknowledged the sovereignty of the
emperors, and received at their hands either tributary princes or
Roman garrisons. Budzak, Crim Tartary, Circassia, and Mingrelia,
are the modern appellations of those savage countries.\footnotemark[82]

\footnotetext[82]{See the Periplus of Arrian. He examined the coasts
of the Euxine, when he was governor of Cappadocia.}

Under the successors of Alexander, Syria was the seat of the
Seleucidæ, who reigned over Upper Asia, till the successful
revolt of the Parthians confined their dominions between the
Euphrates and the Mediterranean. When Syria became subject to the
Romans, it formed the eastern frontier of their empire: nor did
that province, in its utmost latitude, know any other bounds than
the mountains of Cappadocia to the north, and towards the south,
the confines of Egypt, and the Red Sea. Phœnicia and Palestine
were sometimes annexed to, and sometimes separated from, the
jurisdiction of Syria. The former of these was a narrow and rocky
coast; the latter was a territory scarcely superior to Wales,
either in fertility or extent.\footnotemark[821] Yet Phœnicia and Palestine
will forever live in the memory of mankind; since America, as
well as Europe, has received letters from the one, and religion
from the other.\footnotemark[83] A sandy desert, alike destitute of wood and
water, skirts along the doubtful confine of Syria, from the
Euphrates to the Red Sea. The wandering life of the Arabs was
inseparably connected with their independence; and wherever, on
some spots less barren than the rest, they ventured to for many
settled habitations, they soon became subjects to the Roman
empire.\footnotemark[84]

\footnotetext[821]{This comparison is exaggerated, with the
intention, no doubt, of attacking the authority of the Bible,
which boasts of the fertility of Palestine. Gibbon’s only
authorities were that of Strabo (l. xvi. 1104) and the present
state of the country. But Strabo only speaks of the neighborhood
of Jerusalem, which he calls barren and arid to the extent of
sixty stadia round the city: in other parts he gives a favorable
testimony to the fertility of many parts of Palestine: thus he
says, “Near Jericho there is a grove of palms, and a country of a
hundred stadia, full of springs, and well peopled.” Moreover,
Strabo had never seen Palestine; he spoke only after reports,
which may be as inaccurate as those according to which he has
composed that description of Germany, in which Gluverius has
detected so many errors. (Gluv. Germ. iii. 1.) Finally, his
testimony is contradicted and refuted by that of other ancient
authors, and by medals. Tacitus says, in speaking of Palestine,
“The inhabitants are healthy and robust; the rains moderate; the
soil fertile.” (Hist. v. 6.) Ammianus Macellinus says also, “The
last of the Syrias is Palestine, a country of considerable
extent, abounding in clean and well-cultivated land, and
containing some fine cities, none of which yields to the other;
but, as it were, being on a parallel, are rivals.”—xiv. 8. See
also the historian Josephus, Hist. vi. 1. Procopius of Cæserea,
who lived in the sixth century, says that Chosroes, king of
Persia, had a great desire to make himself master of Palestine,
\textit{on account of its} extraordinary fertility, its opulence, and
the great number of its inhabitants. The Saracens thought the
same, and were afraid that Omar. when he went to Jerusalem,
charmed with the fertility of the soil and the purity of the air,
would never return to Medina. (Ockley, Hist. of Sarac. i. 232.)
The importance attached by the Romans to the conquest of
Palestine, and the obstacles they encountered, prove also the
richness and population of the country. Vespasian and Titus
caused medals to be struck with trophies, in which Palestine is
represented by a female under a palm-tree, to signify the
richness of he country, with this legend: \textit{Judæa capta}. Other
medals also indicate this fertility; for instance, that of Herod
holding a bunch of grapes, and that of the young Agrippa
displaying fruit. As to the present state of he country, one
perceives that it is not fair to draw any inference against its
ancient fertility: the disasters through which it has passed, the
government to which it is subject, the disposition of the
inhabitants, explain sufficiently the wild and uncultivated
appearance of the land, where, nevertheless, fertile and
cultivated districts are still found, according to the testimony
of travellers; among others, of Shaw, Maundrel, La Rocque, \&c.—G.
The Abbé Guénée, in his \textit{Lettres de quelques Juifs à Mons. de
Voltaire}, has exhausted the subject of the fertility of
Palestine; for Voltaire had likewise indulged in sarcasm on this
subject. Gibbon was assailed on this point, not, indeed, by Mr.
Davis, who, he slyly insinuates, was prevented by his patriotism
as a Welshman from resenting the comparison with Wales, but by
other writers. In his Vindication, he first established the
correctness of his measurement of Palestine, which he estimates
as 7600 square English miles, while Wales is about 7011. As to
fertility, he proceeds in the following dexterously composed and
splendid passage: “The emperor Frederick II., the enemy and the
victim of the clergy, is accused of saying, after his return from
his crusade, that the God of the Jews would have despised his
promised land, if he had once seen the fruitful realms of Sicily
and Naples.” (See Giannone, Istor. Civ. del R. di Napoli, ii.
245.) This raillery, which malice has, perhaps, falsely imputed
to Frederick, is inconsistent with truth and piety; yet it must
be confessed that the soil of Palestine does not contain that
inexhaustible, and, as it were, spontaneous principle of
fertility, which, under the most unfavorable circumstances, has
covered with rich harvests the banks of the Nile, the fields of
Sicily, or the plains of Poland. The Jordan is the only navigable
river of Palestine: a considerable part of the narrow space is
occupied, or rather lost, in the \textit{Dead Sea} whose horrid aspect
inspires every sensation of disgust, and countenances every tale
of horror. The districts which border on Arabia partake of the
sandy quality of the adjacent desert. The face of the country,
except the sea-coast, and the valley of the Jordan, is covered
with mountains, which appear, for the most part, as naked and
barren rocks; and in the neighborhood of Jerusalem, there is a
real scarcity of the two elements of earth and water. (See
Maundrel’s Travels, p. 65, and Reland’s Palestin. i. 238, 395.)
These disadvantages, which now operate in their fullest extent,
were formerly corrected by the labors of a numerous people, and
the active protection of a wise government. The hills were
clothed with rich beds of artificial mould, the rain was
collected in vast cisterns, a supply of fresh water was conveyed
by pipes and aqueducts to the dry lands. The breed of cattle was
encouraged in those parts which were not adapted for tillage, and
almost every spot was compelled to yield some production for the
use of the inhabitants.

Pater ispe colendi Haud facilem esse viam voluit, primusque par
artem Movit agros; curis acuens mortalia corda, Nec torpere gravi
passus sua Regna veterno. Gibbon, Misc. Works, iv. 540.

But Gibbon has here eluded the question about the land “flowing
with milk and honey.” He is describing Judæa only, without
comprehending Galilee, or the rich pastures beyond the Jordan,
even now proverbial for their flocks and herds. (See Burckhardt’s
Travels, and Hist of Jews, i. 178.) The following is believed to
be a fair statement: “The extraordinary fertility of the whole
country must be taken into the account. No part was waste; very
little was occupied by unprofitable wood; the more fertile hills
were cultivated in artificial terraces, others were hung with
orchards of fruit trees the more rocky and barren districts were
covered with vineyards.” Even in the present day, the wars and
misgovernment of ages have not exhausted the natural richness of
the soil. “Galilee,” says Malte Brun, “would be a paradise were
it inhabited by an industrious people under an enlightened
government. No land could be less dependent on foreign
importation; it bore within itself every thing that could be
necessary for the subsistence and comfort of a simple
agricultural people. The climate was healthy, the seasons
regular; the former rains, which fell about October, after the
vintage, prepared the ground for the seed; that latter, which
prevailed during March and the beginning of April, made it grow
rapidly. Directly the rains ceased, the grain ripened with still
greater rapidity, and was gathered in before the end of May. The
summer months were dry and very hot, but the nights cool and
refreshed by copious dews. In September, the vintage was
gathered. Grain of all kinds, wheat, barley, millet, zea, and
other sorts, grew in abundance; the wheat commonly yielded thirty
for one. Besides the vine and the olive, the almond, the date,
figs of many kinds, the orange, the pomegranate, and many other
fruit trees, flourished in the greatest luxuriance. Great
quantity of honey was collected. The balm-tree, which produced
the opobalsamum, a great object of trade, was probably introduced
from Arabia, in the time of Solomon. It flourished about Jericho
and in Gilead.”—Milman’s Hist. of Jews. i. 177.—M.}

\footnotetext[83]{The progress of religion is well known. The use of
letter was introduced among the savages of Europe about fifteen
hundred years before Christ; and the Europeans carried them to
America about fifteen centuries after the Christian Æra. But in a
period of three thousand years, the Phœnician alphabet received
considerable alterations, as it passed through the hands of the
Greeks and Romans.}

\footnotetext[84]{Dion Cassius, lib. lxviii. p. 1131.}

The geographers of antiquity have frequently hesitated to what
portion of the globe they should ascribe Egypt.\footnotemark[85] By its
situation that celebrated kingdom is included within the immense
peninsula of Africa; but it is accessible only on the side of
Asia, whose revolutions, in almost every period of history, Egypt
has humbly obeyed. A Roman præfect was seated on the splendid
throne of the Ptolemies; and the iron sceptre of the Mamelukes is
now in the hands of a Turkish pacha. The Nile flows down the
country, above five hundred miles from the tropic of Cancer to
the Mediterranean, and marks on either side the extent of
fertility by the measure of its inundations. Cyrene, situate
towards the west, and along the sea-coast, was first a Greek
colony, afterwards a province of Egypt, and is now lost in the
desert of Barca.\footnotemark[851]

\footnotetext[85]{Ptolemy and Strabo, with the modern geographers,
fix the Isthmus of Suez as the boundary of Asia and Africa.
Dionysius, Mela, Pliny, Sallust, Hirtius, and Solinus, have
preferred for that purpose the western branch of the Nile, or
even the great Catabathmus, or descent, which last would assign
to Asia, not only Egypt, but part of Libya.}

\footnotetext[851]{The French editor has a long and unnecessary note
on the History of Cyrene. For the present state of that coast and
country, the volume of Captain Beechey is full of interesting
details. Egypt, now an independent and improving kingdom,
appears, under the enterprising rule of Mahommed Ali, likely to
revenge its former oppression upon the decrepit power of the
Turkish empire.—M.—This note was written in 1838. The future
destiny of Egypt is an important problem, only to be solved by
time. This observation will also apply to the new French colony
in Algiers.—M. 1845.}

From Cyrene to the ocean, the coast of Africa extends above
fifteen hundred miles; yet so closely is it pressed between the
Mediterranean and the Sahara, or sandy desert, that its breadth
seldom exceeds fourscore or a hundred miles. The eastern division
was considered by the Romans as the more peculiar and proper
province of Africa. Till the arrival of the Phœnician colonies,
that fertile country was inhabited by the Libyans, the most
savage of mankind. Under the immediate jurisdiction of Carthage,
it became the centre of commerce and empire; but the republic of
Carthage is now degenerated into the feeble and disorderly states
of Tripoli and Tunis. The military government of Algiers
oppresses the wide extent of Numidia, as it was once united under
Massinissa and Jugurtha; but in the time of Augustus, the limits
of Numidia were contracted; and, at least, two thirds of the
country acquiesced in the name of Mauritania, with the epithet of
Cæsariensis. The genuine Mauritania, or country of the Moors,
which, from the ancient city of Tingi, or Tangier, was
distinguished by the appellation of Tingitana, is represented by
the modern kingdom of Fez. Salle, on the Ocean, so infamous at
present for its piratical depredations, was noticed by the
Romans, as the extreme object of their power, and almost of their
geography. A city of their foundation may still be discovered
near Mequinez, the residence of the barbarian whom we condescend
to style the Emperor of Morocco; but it does not appear, that his
more southern dominions, Morocco itself, and Segelmessa, were
ever comprehended within the Roman province. The western parts of
Africa are intersected by the branches of Mount Atlas, a name so
idly celebrated by the fancy of poets;\footnotemark[86] but which is now
diffused over the immense ocean that rolls between the ancient
and the new continent.\footnotemark[87]

\footnotetext[86]{The long range, moderate height, and gentle
declivity of Mount Atlas, (see Shaw’s Travels, p. 5,) are very
unlike a solitary mountain which rears its head into the clouds,
and seems to support the heavens. The peak of Teneriff, on the
contrary, rises a league and a half above the surface of the sea;
and, as it was frequently visited by the Phœnicians, might engage
the notice of the Greek poets. See Buffon, Histoire Naturelle,
tom. i. p. 312. Histoire des Voyages, tom. ii.}

\footnotetext[87]{M. de Voltaire, tom. xiv. p. 297, unsupported by
either fact or probability, has generously bestowed the Canary
Islands on the Roman empire.}

Having now finished the circuit of the Roman empire, we may
observe, that Africa is divided from Spain by a narrow strait of
about twelve miles, through which the Atlantic flows into the
Mediterranean. The columns of Hercules, so famous among the
ancients, were two mountains which seemed to have been torn
asunder by some convulsion of the elements; and at the foot of
the European mountain, the fortress of Gibraltar is now seated.
The whole extent of the Mediterranean Sea, its coasts and its
islands, were comprised within the Roman dominion. Of the larger
islands, the two Baleares, which derive their name of Majorca and
Minorca from their respective size, are subject at present, the
former to Spain, the latter to Great Britain.\footnotemark[871] It is easier to
deplore the fate, than to describe the actual condition, of
Corsica.\footnotemark[872] Two Italian sovereigns assume a regal title from
Sardinia and Sicily. Crete, or Candia, with Cyprus, and most of
the smaller islands of Greece and Asia, have been subdued by the
Turkish arms, whilst the little rock of Malta defies their power,
and has emerged, under the government of its military Order, into
fame and opulence.\footnotemark[873]

\footnotetext[871]{Minorca was lost to Great Britain in 1782. Ann.
Register for that year.—M.}

\footnotetext[872]{The gallant struggles of the Corsicans for their
independence, under Paoli, were brought to a close in the year
1769. This volume was published in 1776. See Botta, Storia
d’Italia, vol. xiv.—M.}

\footnotetext[873]{Malta, it need scarcely be said, is now in the
possession of the English. We have not, however, thought it
necessary to notice every change in the political state of the
world, since the time of Gibbon.—M}

This long enumeration of provinces, whose broken fragments have
formed so many powerful kingdoms, might almost induce us to
forgive the vanity or ignorance of the ancients. Dazzled with the
extensive sway, the irresistible strength, and the real or
affected moderation of the emperors, they permitted themselves to
despise, and sometimes to forget, the outlying countries which
had been left in the enjoyment of a barbarous independence; and
they gradually usurped the license of confounding the Roman
monarchy with the globe of the earth.\footnotemark[88] But the temper, as well
as knowledge, of a modern historian, require a more sober and
accurate language. He may impress a juster image of the greatness
of Rome, by observing that the empire was above two thousand
miles in breadth, from the wall of Antoninus and the northern
limits of Dacia, to Mount Atlas and the tropic of Cancer; that it
extended in length more than three thousand miles from the
Western Ocean to the Euphrates; that it was situated in the
finest part of the Temperate Zone, between the twenty-fourth and
fifty-sixth degrees of northern latitude; and that it was
supposed to contain above sixteen hundred thousand square miles,
for the most part of fertile and well-cultivated land.\footnotemark[89]

\footnotetext[88]{Bergier, Hist. des Grands Chemins, l. iii. c. 1, 2,
3, 4, a very useful collection.}

\footnotetext[89]{See Templeman’s Survey of the Globe; but I distrust
both the Doctor’s learning and his maps.}

