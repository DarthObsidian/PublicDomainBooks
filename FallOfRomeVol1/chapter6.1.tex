\chapter{Death Of Severus, Tyranny Of Caracalla, Usurpation Of Marcinus.}
\section{Part \thesection.}

The Death Of Severus. — Tyranny Of Caracalla. — Usurpation Of
Macrinus. — Follies Of Elagabalus. — Virtues Of Alexander
Severus. — Licentiousness Of The Army. — General State Of The Roman
Finances.
\vspace{\onelineskip}

The ascent to greatness, however steep and dangerous, may
entertain an active spirit with the consciousness and exercise of
its own powers: but the possession of a throne could never yet
afford a lasting satisfaction to an ambitious mind. This
melancholy truth was felt and acknowledged by Severus. Fortune
and merit had, from an humble station, elevated him to the first
place among mankind. “He had been all things,” as he said
himself, “and all was of little value.”\footnotemark[1] Distracted with the
care, not of acquiring, but of preserving an empire, oppressed
with age and infirmities, careless of fame,\footnotemark[2] and satiated with
power, all his prospects of life were closed. The desire of
perpetuating the greatness of his family was the only remaining
wish of his ambition and paternal tenderness.

\footnotetext[1]{Hist. August. p. 71. “Omnia fui, et nihil expedit.”}

\footnotetext[2]{Dion Cassius, l. lxxvi. p. 1284.}

Like most of the Africans, Severus was passionately addicted to
the vain studies of magic and divination, deeply versed in the
interpretation of dreams and omens, and perfectly acquainted with
the science of judicial astrology; which, in almost every age
except the present, has maintained its dominion over the mind of
man. He had lost his first wife, while he was governor of the
Lionnese Gaul.\footnotemark[3] In the choice of a second, he sought only to
connect himself with some favorite of fortune; and as soon as he
had discovered that the young lady of Emesa in Syria had \textit{a royal
nativity}, he solicited and obtained her hand.\footnotemark[4] Julia Domna (for
that was her name) deserved all that the stars could promise her.

She possessed, even in advanced age, the attractions of beauty,\footnotemark[5]
and united to a lively imagination a firmness of mind, and
strength of judgment, seldom bestowed on her sex. Her amiable
qualities never made any deep impression on the dark and jealous
temper of her husband; but in her son’s reign, she administered
the principal affairs of the empire, with a prudence that
supported his authority, and with a moderation that sometimes
corrected his wild extravagancies.\footnotemark[6] Julia applied herself to
letters and philosophy, with some success, and with the most
splendid reputation. She was the patroness of every art, and the
friend of every man of genius.\footnotemark[7] The grateful flattery of the
learned has celebrated her virtues; but, if we may credit the
scandal of ancient history, chastity was very far from being the
most conspicuous virtue of the empress Julia.\footnotemark[8]

\footnotetext[3]{About the year 186. M. de Tillemont is miserably
embarrassed with a passage of Dion, in which the empress
Faustina, who died in the year 175, is introduced as having
contributed to the marriage of Severus and Julia, (l. lxxiv. p.
1243.) The learned compiler forgot that Dion is relating not a
real fact, but a dream of Severus; and dreams are circumscribed
to no limits of time or space. Did M. de Tillemont imagine that
marriages were consummated in the temple of Venus at Rome? Hist.
des Empereurs, tom. iii. p. 389. Note 6.}

\footnotetext[4]{Hist. August. p. 65.}

\footnotetext[5]{Hist. August. p. 5.}

\footnotetext[6]{Dion Cassius, l. lxxvii. p. 1304, 1314.}

\footnotetext[7]{See a dissertation of Menage, at the end of his
edition of Diogenes Lærtius, de Fœminis Philosophis.}

\footnotetext[8]{Dion, l. lxxvi. p. 1285. Aurelius Victor.}

Two sons, Caracalla\footnotemark[9] and Geta, were the fruit of this marriage,
and the destined heirs of the empire. The fond hopes of the
father, and of the Roman world, were soon disappointed by these
vain youths, who displayed the indolent security of hereditary
princes; and a presumption that fortune would supply the place of
merit and application. Without any emulation of virtue or
talents, they discovered, almost from their infancy, a fixed and
implacable antipathy for each other.

\footnotetext[9]{Bassianus was his first name, as it had been that of
his maternal grandfather. During his reign, he assumed the
appellation of Antoninus, which is employed by lawyers and
ancient historians. After his death, the public indignation
loaded him with the nicknames of Tarantus and Caracalla. The
first was borrowed from a celebrated Gladiator, the second from a
long Gallic gown which he distributed to the people of Rome.}

Their aversion, confirmed by years, and fomented by the arts of
their interested favorites, broke out in childish, and gradually
in more serious competitions; and, at length, divided the
theatre, the circus, and the court, into two factions, actuated
by the hopes and fears of their respective leaders. The prudent
emperor endeavored, by every expedient of advice and authority,
to allay this growing animosity. The unhappy discord of his sons
clouded all his prospects, and threatened to overturn a throne
raised with so much labor, cemented with so much blood, and
guarded with every defence of arms and treasure. With an
impartial hand he maintained between them an exact balance of
favor, conferred on both the rank of Augustus, with the revered
name of Antoninus; and for the first time the Roman world beheld
three emperors.\footnotemark[10] Yet even this equal conduct served only to
inflame the contest, whilst the fierce Caracalla asserted the
right of primogeniture, and the milder Geta courted the
affections of the people and the soldiers. In the anguish of a
disappointed father, Severus foretold that the weaker of his sons
would fall a sacrifice to the stronger; who, in his turn, would
be ruined by his own vices.\footnotemark[11]

\footnotetext[10]{The elevation of Caracalla is fixed by the accurate
M. de Tillemont to the year 198; the association of Geta to the
year 208.}

\footnotetext[11]{Herodian, l. iii. p. 130. The lives of Caracalla
and Geta, in the Augustan History.}

In these circumstances the intelligence of a war in Britain, and
of an invasion of the province by the barbarians of the North,
was received with pleasure by Severus. Though the vigilance of
his lieutenants might have been sufficient to repel the distant
enemy, he resolved to embrace the honorable pretext of
withdrawing his sons from the luxury of Rome, which enervated
their minds and irritated their passions; and of inuring their
youth to the toils of war and government. Notwithstanding his
advanced age, (for he was above threescore,) and his gout, which
obliged him to be carried in a litter, he transported himself in
person into that remote island, attended by his two sons, his
whole court, and a formidable army. He immediately passed the
walls of Hadrian and Antoninus, and entered the enemy’s country,
with a design of completing the long attempted conquest of
Britain. He penetrated to the northern extremity of the island,
without meeting an enemy. But the concealed ambuscades of the
Caledonians, who hung unseen on the rear and flanks of his army,
the coldness of the climate and the severity of a winter march
across the hills and morasses of Scotland, are reported to have
cost the Romans above fifty thousand men. The Caledonians at
length yielded to the powerful and obstinate attack, sued for
peace, and surrendered a part of their arms, and a large tract of
territory. But their apparent submission lasted no longer than
the present terror. As soon as the Roman legions had retired,
they resumed their hostile independence. Their restless spirit
provoked Severus to send a new army into Caledonia, with the most
bloody orders, not to subdue, but to extirpate the natives. They
were saved by the death of their haughty enemy.\footnotemark[12]

\footnotetext[12]{Dion, l. lxxvi. p. 1280, \&c. Herodian, l. iii. p.
132, \&c.}

This Caledonian war, neither marked by decisive events, nor
attended with any important consequences, would ill deserve our
attention; but it is supposed, not without a considerable degree
of probability, that the invasion of Severus is connected with
the most shining period of the British history or fable. Fingal,
whose fame, with that of his heroes and bards, has been revived
in our language by a recent publication, is said to have
commanded the Caledonians in that memorable juncture, to have
eluded the power of Severus, and to have obtained a signal
victory on the banks of the Carun, in which the son of \textit{the King
of the World}, Caracul, fled from his arms along the fields of
his pride.\footnotemark[13] Something of a doubtful mist still hangs over these
Highland traditions; nor can it be entirely dispelled by the most
ingenious researches of modern criticism;\footnotemark[14] but if we could,
with safety, indulge the pleasing supposition, that Fingal lived,
and that Ossian sung, the striking contrast of the situation and
manners of the contending nations might amuse a philosophic mind.

The parallel would be little to the advantage of the more
civilized people, if we compared the unrelenting revenge of
Severus with the generous clemency of Fingal; the timid and
brutal cruelty of Caracalla with the bravery, the tenderness, the
elegant genius of Ossian; the mercenary chiefs, who, from motives
of fear or interest, served under the imperial standard, with the
free-born warriors who started to arms at the voice of the king
of Morven; if, in a word, we contemplated the untutored
Caledonians, glowing with the warm virtues of nature, and the
degenerate Romans, polluted with the mean vices of wealth and
slavery.

\footnotetext[13]{Ossian’s Poems, vol. i. p. 175.}

\footnotetext[14]{That the Caracul of Ossian is the Caracalla of the
Roman History, is, perhaps, the only point of British antiquity
in which Mr. Macpherson and Mr. Whitaker are of the same opinion;
and yet the opinion is not without difficulty. In the Caledonian
war, the son of Severus was known only by the appellation of
Antoninus, and it may seem strange that the Highland bard should
describe him by a nickname, invented four years afterwards,
scarcely used by the Romans till after the death of that emperor,
and seldom employed by the most ancient historians. See Dion, l.
lxxvii. p. 1317. Hist. August. p. 89 Aurel. Victor. Euseb. in
Chron. ad ann. 214. Note: The historical authority of
Macpherson’s Ossian has not increased since Gibbon wrote. We may,
indeed, consider it exploded. Mr. Whitaker, in a letter to Gibbon
(Misc. Works, vol. ii. p. 100,) attempts, not very successfully,
to weaken this objection of the historian.—M.}

The declining health and last illness of Severus inflamed the
wild ambition and black passions of Caracalla’s soul. Impatient
of any delay or division of empire, he attempted, more than once,
to shorten the small remainder of his father’s days, and
endeavored, but without success, to excite a mutiny among the
troops.\footnotemark[15] The old emperor had often censured the misguided
lenity of Marcus, who, by a single act of justice, might have
saved the Romans from the tyranny of his worthless son. Placed in
the same situation, he experienced how easily the rigor of a
judge dissolves away in the tenderness of a parent. He
deliberated, he threatened, but he could not punish; and this
last and only instance of mercy was more fatal to the empire than
a long series of cruelty.\footnotemark[16] The disorder of his mind irritated
the pains of his body; he wished impatiently for death, and
hastened the instant of it by his impatience. He expired at York
in the sixty-fifth year of his life, and in the eighteenth of a
glorious and successful reign. In his last moments he recommended
concord to his sons, and his sons to the army. The salutary
advice never reached the heart, or even the understanding, of the
impetuous youths; but the more obedient troops, mindful of their
oath of allegiance, and of the authority of their deceased
master, resisted the solicitations of Caracalla, and proclaimed
both brothers emperors of Rome. The new princes soon left the
Caledonians in peace, returned to the capital, celebrated their
father’s funeral with divine honors, and were cheerfully
acknowledged as lawful sovereigns, by the senate, the people, and
the provinces. Some preeminence of rank seems to have been
allowed to the elder brother; but they both administered the
empire with equal and independent power.\footnotemark[17]

\footnotetext[15]{Dion, l. lxxvi. p. 1282. Hist. August. p. 71.
Aurel. Victor.}

\footnotetext[16]{Dion, l. lxxvi. p. 1283. Hist. August. p. 89}

\footnotetext[17]{Footnote 17: Dion, l. lxxvi. p. 1284. Herodian, l.
iii. p. 135.}

Such a divided form of government would have proved a source of
discord between the most affectionate brothers. It was impossible
that it could long subsist between two implacable enemies, who
neither desired nor could trust a reconciliation. It was visible
that one only could reign, and that the other must fall; and each
of them, judging of his rival’s designs by his own, guarded his
life with the most jealous vigilance from the repeated attacks of
poison or the sword. Their rapid journey through Gaul and Italy,
during which they never ate at the same table, or slept in the
same house, displayed to the provinces the odious spectacle of
fraternal discord. On their arrival at Rome, they immediately
divided the vast extent of the imperial palace.\footnotemark[18] No
communication was allowed between their apartments; the doors and
passages were diligently fortified, and guards posted and
relieved with the same strictness as in a besieged place. The
emperors met only in public, in the presence of their afflicted
mother; and each surrounded by a numerous train of armed
followers. Even on these occasions of ceremony, the dissimulation
of courts could ill disguise the rancor of their hearts.\footnotemark[19]

\footnotetext[18]{Mr. Hume is justly surprised at a passage of
Herodian, (l. iv. p. 139,) who, on this occasion, represents the
Imperial palace as equal in extent to the rest of Rome. The whole
region of the Palatine Mount, on which it was built, occupied, at
most, a circumference of eleven or twelve thousand feet, (see the
Notitia and Victor, in Nardini’s Roma Antica.) But we should
recollect that the opulent senators had almost surrounded the
city with their extensive gardens and suburb palaces, the
greatest part of which had been gradually confiscated by the
emperors. If Geta resided in the gardens that bore his name on
the Janiculum, and if Caracalla inhabited the gardens of Mæcenas
on the Esquiline, the rival brothers were separated from each
other by the distance of several miles; and yet the intermediate
space was filled by the Imperial gardens of Sallust, of Lucullus,
of Agrippa, of Domitian, of Caius, \&c., all skirting round the
city, and all connected with each other, and with the palace, by
bridges thrown over the Tiber and the streets. But this
explanation of Herodian would require, though it ill deserves, a
particular dissertation, illustrated by a map of ancient Rome.
(Hume, Essay on Populousness of Ancient Nations.—M.)}

\footnotetext[19]{Herodian, l. iv. p. 139}

This latent civil war already distracted the whole government,
when a scheme was suggested that seemed of mutual benefit to the
hostile brothers. It was proposed, that since it was impossible
to reconcile their minds, they should separate their interest,
and divide the empire between them. The conditions of the treaty
were already drawn with some accuracy. It was agreed that
Caracalla, as the elder brother should remain in possession of
Europe and the western Africa; and that he should relinquish the
sovereignty of Asia and Egypt to Geta, who might fix his
residence at Alexandria or Antioch, cities little inferior to
Rome itself in wealth and greatness; that numerous armies should
be constantly encamped on either side of the Thracian Bosphorus,
to guard the frontiers of the rival monarchies; and that the
senators of European extraction should acknowledge the sovereign
of Rome, whilst the natives of Asia followed the emperor of the
East. The tears of the empress Julia interrupted the negotiation,
the first idea of which had filled every Roman breast with
surprise and indignation. The mighty mass of conquest was so
intimately united by the hand of time and policy, that it
required the most forcible violence to rend it asunder. The
Romans had reason to dread, that the disjointed members would
soon be reduced by a civil war under the dominion of one master;
but if the separation was permanent, the division of the
provinces must terminate in the dissolution of an empire whose
unity had hitherto remained inviolate.\footnotemark[20]

\footnotetext[20]{Herodian, l. iv. p. 144.}

Had the treaty been carried into execution, the sovereign of
Europe might soon have been the conqueror of Asia; but Caracalla
obtained an easier, though a more guilty, victory. He artfully
listened to his mother’s entreaties, and consented to meet his
brother in her apartment, on terms of peace and reconciliation.
In the midst of their conversation, some centurions, who had
contrived to conceal themselves, rushed with drawn swords upon
the unfortunate Geta. His distracted mother strove to protect him
in her arms; but, in the unavailing struggle, she was wounded in
the hand, and covered with the blood of her younger son, while
she saw the elder animating and assisting\footnotemark[21] the fury of the
assassins. As soon as the deed was perpetrated, Caracalla, with
hasty steps, and horror in his countenance, ran towards the
Prætorian camp, as his only refuge, and threw himself on the
ground before the statues of the tutelar deities.\footnotemark[22] The soldiers
attempted to raise and comfort him. In broken and disordered
words he informed them of his imminent danger, and fortunate
escape; insinuating that he had prevented the designs of his
enemy, and declared his resolution to live and die with his
faithful troops. Geta had been the favorite of the soldiers; but
complaint was useless, revenge was dangerous, and they still
reverenced the son of Severus. Their discontent died away in idle
murmurs, and Caracalla soon convinced them of the justice of his
cause, by distributing in one lavish donative the accumulated
treasures of his father’s reign.\footnotemark[23] The real \textit{sentiments} of the
soldiers alone were of importance to his power or safety. Their
declaration in his favor commanded the dutiful \textit{professions} of
the senate. The obsequious assembly was always prepared to ratify
the decision of fortune;\footnotemark[231] but as Caracalla wished to assuage
the first emotions of public indignation, the name of Geta was
mentioned with decency, and he received the funeral honors of a
Roman emperor.\footnotemark[24] Posterity, in pity to his misfortune, has cast
a veil over his vices. We consider that young prince as the
innocent victim of his brother’s ambition, without recollecting
that he himself wanted power, rather than inclination, to
consummate the same attempts of revenge and murder.\footnotemark[241]

\footnotetext[21]{Caracalla consecrated, in the temple of Serapis,
the sword with which, as he boasted, he had slain his brother
Geta. Dion, l. lxxvii p. 1307.}

\footnotetext[22]{Herodian, l. iv. p. 147. In every Roman camp there
was a small chapel near the head-quarters, in which the statues
of the tutelar deities were preserved and adored; and we may
remark that the eagles, and other military ensigns, were in the
first rank of these deities; an excellent institution, which
confirmed discipline by the sanction of religion. See Lipsius de
Militia Romana, iv. 5, v. 2.}

\footnotetext[23]{Herodian, l. iv. p. 148. Dion, l. lxxvii. p. 1289.}

\footnotetext[231]{The account of this transaction, in a new passage
of Dion, varies in some degree from this statement. It adds that
the next morning, in the senate, Antoninus requested their
indulgence, not because he had killed his brother, but because he
was hoarse, and could not address them. Mai. Fragm. p. 228.—M.}

\footnotetext[24]{Geta was placed among the gods. Sit divus, dum non
sit vivus said his brother. Hist. August. p. 91. Some marks of
Geta’s consecration are still found upon medals.}

\footnotetext[241]{The favorable judgment which history has given of
Geta is not founded solely on a feeling of pity; it is supported
by the testimony of contemporary historians: he was too fond of
the pleasures of the table, and showed great mistrust of his
brother; but he was humane, well instructed; he often endeavored
to mitigate the rigorous decrees of Severus and Caracalla. Herod
iv. 3. Spartian in Geta.—W.}

The crime went not unpunished. Neither business, nor pleasure,
nor flattery, could defend Caracalla from the stings of a guilty
conscience; and he confessed, in the anguish of a tortured mind,
that his disordered fancy often beheld the angry forms of his
father and his brother rising into life, to threaten and upbraid
him.\footnotemark[25] The consciousness of his crime should have induced him to
convince mankind, by the virtues of his reign, that the bloody
deed had been the involuntary effect of fatal necessity. But the
repentance of Caracalla only prompted him to remove from the
world whatever could remind him of his guilt, or recall the
memory of his murdered brother. On his return from the senate to
the palace, he found his mother in the company of several noble
matrons, weeping over the untimely fate of her younger son. The
jealous emperor threatened them with instant death; the sentence
was executed against Fadilla, the last remaining daughter of the
emperor Marcus;\footnotemark[251] and even the afflicted Julia was obliged to
silence her lamentations, to suppress her sighs, and to receive
the assassin with smiles of joy and approbation. It was computed
that, under the vague appellation of the friends of Geta, above
twenty thousand persons of both sexes suffered death. His guards
and freedmen, the ministers of his serious business, and the
companions of his looser hours, those who by his interest had
been promoted to any commands in the army or provinces, with the
long connected chain of their dependants, were included in the
proscription; which endeavored to reach every one who had
maintained the smallest correspondence with Geta, who lamented
his death, or who even mentioned his name.\footnotemark[26] Helvius Pertinax,
son to the prince of that name, lost his life by an unseasonable
witticism.\footnotemark[27] It was a sufficient crime of Thrasea Priscus to be
descended from a family in which the love of liberty seemed an
hereditary quality.\footnotemark[28] The particular causes of calumny and
suspicion were at length exhausted; and when a senator was
accused of being a secret enemy to the government, the emperor
was satisfied with the general proof that he was a man of
property and virtue. From this well-grounded principle he
frequently drew the most bloody inferences.\footnotemark[281]

\footnotetext[25]{Dion, l. lxxvii. p. 1307}

\footnotetext[251]{The most valuable paragraph of dion, which the
industry of M. Manas recovered, relates to this daughter of
Marcus, executed by Caracalla. Her name, as appears from Fronto,
as well as from Dion, was Cornificia. When commanded to choose
the kind of death she was to suffer, she burst into womanish
tears; but remembering her father Marcus, she thus spoke:—“O my
hapless soul, (... animula,) now imprisoned in the body, burst
forth! be free! show them, however reluctant to believe it, that
thou art the daughter of Marcus.” She then laid aside all her
ornaments, and preparing herself for death, ordered her veins to
be opened. Mai. Fragm. Vatican ii p. 220.—M.}

\footnotetext[26]{Dion, l. lxxvii. p. 1290. Herodian, l. iv. p. 150.
Dion (p. 2298) says, that the comic poets no longer durst employ
the name of Geta in their plays, and that the estates of those
who mentioned it in their testaments were confiscated.}

\footnotetext[27]{Caracalla had assumed the names of several
conquered nations; Pertinax observed, that the name of Geticus
(he had obtained some advantage over the Goths, or Getæ) would be
a proper addition to Parthieus, Alemannicus, \&c. Hist. August. p.
89.}

\footnotetext[28]{Dion, l. lxxvii. p. 1291. He was probably descended
from Helvidius Priscus, and Thrasea Pætus, those patriots, whose
firm, but useless and unseasonable, virtue has been immortalized
by Tacitus. Note: M. Guizot is indignant at this “cold”
observation of Gibbon on the noble character of Thrasea; but he
admits that his virtue was useless to the public, and
unseasonable amidst the vices of his age.—M.}

\footnotetext[281]{Caracalla reproached all those who demanded no
favors of him. “It is clear that if you make me no requests, you
do not trust me; if you do not trust me, you suspect me; if you
suspect me, you fear me; if you fear me, you hate me.” And
forthwith he condemned them as conspirators, a good specimen of
the sorites in a tyrant’s logic. See Fragm. Vatican p.—M.}

