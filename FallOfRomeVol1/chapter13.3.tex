\section{Part \thesection.}
\thispagestyle{simple}

While the East anxiously expected the decision of this great
contest, the emperor Diocletian, having assembled in Syria a
strong army of observation, displayed from a distance the
resources of the Roman power, and reserved himself for any future
emergency of the war. On the intelligence of the victory he
condescended to advance towards the frontier, with a view of
moderating, by his presence and counsels, the pride of Galerius.
The interview of the Roman princes at Nisibis was accompanied
with every expression of respect on one side, and of esteem on
the other. It was in that city that they soon afterwards gave
audience to the ambassador of the Great King.\footnotemark[74] The power, or at
least the spirit, of Narses, had been broken by his last defeat;
and he considered an immediate peace as the only means that could
stop the progress of the Roman arms. He despatched Apharban, a
servant who possessed his favor and confidence, with a commission
to negotiate a treaty, or rather to receive whatever conditions
the conqueror should impose. Apharban opened the conference by
expressing his master’s gratitude for the generous treatment of
his family, and by soliciting the liberty of those illustrious
captives. He celebrated the valor of Galerius, without degrading
the reputation of Narses, and thought it no dishonor to confess
the superiority of the victorious Cæsar, over a monarch who had
surpassed in glory all the princes of his race. Notwithstanding
the justice of the Persian cause, he was empowered to submit the
present differences to the decision of the emperors themselves;
convinced as he was, that, in the midst of prosperity, they would
not be unmindful of the vicissitudes of fortune. Apharban
concluded his discourse in the style of eastern allegory, by
observing that the Roman and Persian monarchies were the two eyes
of the world, which would remain imperfect and mutilated if
either of them should be put out.

\footnotetext[74]{The account of the negotiation is taken from the
fragments of Peter the Patrician, in the Excerpta Legationum,
published in the Byzantine Collection. Peter lived under
Justinian; but it is very evident, by the nature of his
materials, that they are drawn from the most authentic and
respectable writers.}

“It well becomes the Persians,” replied Galerius, with a
transport of fury, which seemed to convulse his whole frame, “it
well becomes the Persians to expatiate on the vicissitudes of
fortune, and calmly to read us lectures on the virtues of
moderation. Let them remember their own \textit{moderation} towards the
unhappy Valerian. They vanquished him by fraud, they treated him
with indignity. They detained him till the last moment of his
life in shameful captivity, and after his death they exposed his
body to perpetual ignominy.” Softening, however, his tone,
Galerius insinuated to the ambassador, that it had never been the
practice of the Romans to trample on a prostrate enemy; and that,
on this occasion, they should consult their own dignity rather
than the Persian merit. He dismissed Apharban with a hope that
Narses would soon be informed on what conditions he might obtain,
from the clemency of the emperors, a lasting peace, and the
restoration of his wives and children. In this conference we may
discover the fierce passions of Galerius, as well as his
deference to the superior wisdom and authority of Diocletian. The
ambition of the former grasped at the conquest of the East, and
had proposed to reduce Persia into the state of a province. The
prudence of the latter, who adhered to the moderate policy of
Augustus and the Antonines, embraced the favorable opportunity of
terminating a successful war by an honorable and advantageous
peace.\footnotemark[75]

\footnotetext[75]{Adeo victor (says Aurelius) ut ni Valerius, cujus
nutu omnis gerebantur, abnuisset, Romani fasces in provinciam
novam ferrentur Verum pars terrarum tamen nobis utilior quæsita.}

In pursuance of their promise, the emperors soon afterwards
appointed Sicorius Probus, one of their secretaries, to acquaint
the Persian court with their final resolution. As the minister of
peace, he was received with every mark of politeness and
friendship; but, under the pretence of allowing him the necessary
repose after so long a journey, the audience of Probus was
deferred from day to day; and he attended the slow motions of the
king, till at length he was admitted to his presence, near the
River Asprudus in Media. The secret motive of Narses, in this
delay, had been to collect such a military force as might enable
him, though sincerely desirous of peace, to negotiate with the
greater weight and dignity. Three persons only assisted at this
important conference, the minister Apharban, the præfect of the
guards, and an officer who had commanded on the Armenian
frontier.\footnotemark[76] The first condition proposed by the ambassador is
not at present of a very intelligible nature; that the city of
Nisibis might be established for the place of mutual exchange,
or, as we should formerly have termed it, for the staple of
trade, between the two empires. There is no difficulty in
conceiving the intention of the Roman princes to improve their
revenue by some restraints upon commerce; but as Nisibis was
situated within their own dominions, and as they were masters
both of the imports and exports, it should seem that such
restraints were the objects of an internal law, rather than of a
foreign treaty. To render them more effectual, some stipulations
were probably required on the side of the king of Persia, which
appeared so very repugnant either to his interest or to his
dignity, that Narses could not be persuaded to subscribe them. As
this was the only article to which he refused his consent, it was
no longer insisted on; and the emperors either suffered the trade
to flow in its natural channels, or contented themselves with
such restrictions, as it depended on their own authority to
establish.

\footnotetext[76]{He had been governor of Sumium, (Pot. Patricius in
Excerpt. Legat. p. 30.) This province seems to be mentioned by
Moses of Chorene, (Geograph. p. 360,) and lay to the east of
Mount Ararat. * Note: The Siounikh of the Armenian writers St.
Martin i. 142.—M.}

As soon as this difficulty was removed, a solemn peace was
concluded and ratified between the two nations. The conditions of
a treaty so glorious to the empire, and so necessary to Persia,
may deserve a more peculiar attention, as the history of Rome
presents very few transactions of a similar nature; most of her
wars having either been terminated by absolute conquest, or waged
against barbarians ignorant of the use of letters. I. The Aboras,
or, as it is called by Xenophon, the Araxes, was fixed as the
boundary between the two monarchies.\footnotemark[77] That river, which rose
near the Tigris, was increased, a few miles below Nisibis, by the
little stream of the Mygdonius, passed under the walls of
Singara, and fell into the Euphrates at Circesium, a frontier
town, which, by the care of Diocletian, was very strongly
fortified.\footnotemark[78] Mesopotomia, the object of so many wars, was ceded
to the empire; and the Persians, by this treaty, renounced all
pretensions to that great province. II. They relinquished to the
Romans five provinces beyond the Tigris.\footnotemark[79] Their situation
formed a very useful barrier, and their natural strength was soon
improved by art and military skill. Four of these, to the north
of the river, were districts of obscure fame and inconsiderable
extent; Intiline, Zabdicene, Arzanene, and Moxoene;\footnotemark[791] but on
the east of the Tigris, the empire acquired the large and
mountainous territory of Carduene, the ancient seat of the
Carduchians, who preserved for many ages their manly freedom in
the heart of the despotic monarchies of Asia. The ten thousand
Greeks traversed their country, after a painful march, or rather
engagement, of seven days; and it is confessed by their leader,
in his incomparable relation of the retreat, that they suffered
more from the arrows of the Carduchians, than from the power of
the Great King.\footnotemark[80] Their posterity, the Curds, with very little
alteration either of name or manners,\footnotemark[801] acknowledged the
nominal sovereignty of the Turkish sultan. III. It is almost
needless to observe, that Tiridates, the faithful ally of Rome,
was restored to the throne of his fathers, and that the rights of
the Imperial supremacy were fully asserted and secured. The
limits of Armenia were extended as far as the fortress of Sintha
in Media, and this increase of dominion was not so much an act of
liberality as of justice. Of the provinces already mentioned
beyond the Tigris, the four first had been dismembered by the
Parthians from the crown of Armenia;\footnotemark[81] and when the Romans
acquired the possession of them, they stipulated, at the expense
of the usurpers, an ample compensation, which invested their ally
with the extensive and fertile country of Atropatene. Its
principal city, in the same situation perhaps as the modern
Tauris, was frequently honored by the residence of Tiridates; and
as it sometimes bore the name of Ecbatana, he imitated, in the
buildings and fortifications, the splendid capital of the Medes.\footnotemark[82]
IV. The country of Iberia was barren, its inhabitants rude and
savage. But they were accustomed to the use of arms, and they
separated from the empire barbarians much fiercer and more
formidable than themselves. The narrow defiles of Mount Caucasus
were in their hands, and it was in their choice, either to admit
or to exclude the wandering tribes of Sarmatia, whenever a
rapacious spirit urged them to penetrate into the richer climes
of the South.\footnotemark[83] The nomination of the kings of Iberia, which was
resigned by the Persian monarch to the emperors, contributed to
the strength and security of the Roman power in Asia.\footnotemark[84] The East
enjoyed a profound tranquillity during forty years; and the
treaty between the rival monarchies was strictly observed till
the death of Tiridates; when a new generation, animated with
different views and different passions, succeeded to the
government of the world; and the grandson of Narses undertook a
long and memorable war against the princes of the house of
Constantine.

\footnotetext[77]{By an error of the geographer Ptolemy, the position
of Singara is removed from the Aboras to the Tigris, which may
have produced the mistake of Peter, in assigning the latter river
for the boundary, instead of the former. The line of the Roman
frontier traversed, but never followed, the course of the Tigris.
* Note: There are here several errors. Gibbon has confounded the
streams, and the towns which they pass. The Aboras, or rather the
Chaboras, the Araxes of Xenophon, has its source above Ras-Ain or
Re-Saina, (Theodosiopolis,) about twenty-seven leagues from the
Tigris; it receives the waters of the Mygdonius, or Saocoras,
about thirty-three leagues below Nisibis. at a town now called Al
Nahraim; it does not pass under the walls of Singara; it is the
Saocoras that washes the walls of that town: the latter river has
its source near Nisibis. at five leagues from the Tigris. See
D’Anv. l’Euphrate et le Tigre, 46, 49, 50, and the map.—— To the
east of the Tigris is another less considerable river, named also
the Chaboras, which D’Anville calls the Centrites, Khabour,
Nicephorius, without quoting the authorities on which he gives
those names. Gibbon did not mean to speak of this river, which
does not pass by Singara, and does not fall into the Euphrates.
See Michaelis, Supp. ad Lex. Hebraica. 3d part, p. 664, 665.—G.}

\footnotetext[78]{Procopius de Edificiis, l. ii. c. 6.}

\footnotetext[79]{Three of the provinces, Zabdicene, Arzanene, and
Carduene, are allowed on all sides. But instead of the other two,
Peter (in Excerpt. Leg. p. 30) inserts Rehimene and Sophene. I
have preferred Ammianus, (l. xxv. 7,) because it might be proved
that Sophene was never in the hands of the Persians, either
before the reign of Diocletian, or after that of Jovian. For want
of correct maps, like those of M. d’Anville, almost all the
moderns, with Tillemont and Valesius at their head, have
imagined, that it was in respect to Persia, and not to Rome, that
the five provinces were situate beyond the Tigris.}

\footnotetext[791]{See St. Martin, note on Le Beau, i. 380. He would
read, for Intiline, Ingeleme, the name of a small province of
Armenia, near the sources of the Tigris, mentioned by St.
Epiphanius, (Hæres, 60;) for the unknown name Arzacene, with
Gibbon, Arzanene. These provinces do not appear to have made an
integral part of the Roman empire; Roman garrisons replaced those
of Persia, but the sovereignty remained in the hands of the
feudatory princes of Armenia. A prince of Carduene, ally or
dependent on the empire, with the Roman name of Jovianus, occurs
in the reign of Julian.—M.}

\footnotetext[80]{Xenophon’s Anabasis, l. iv. Their bows were three
cubits in length, their arrows two; they rolled down stones that
were each a wagon load. The Greeks found a great many villages in
that rude country.}

\footnotetext[801]{I travelled through this country in 1810, and
should judge, from what I have read and seen of its inhabitants,
that they have remained unchanged in their appearance and
character for more than twenty centuries Malcolm, note to Hist.
of Persia, vol. i. p. 82.—M.}

\footnotetext[81]{According to Eutropius, (vi. 9, as the text is
represented by the best Mss.,) the city of Tigranocerta was in
Arzanene. The names and situation of the other three may be
faintly traced.}

\footnotetext[82]{Compare Herodotus, l. i. c. 97, with Moses
Choronens. Hist Armen. l. ii. c. 84, and the map of Armenia given
by his editors.}

\footnotetext[83]{Hiberi, locorum potentes, Caspia via Sarmatam in
Armenios raptim effundunt. Tacit. Annal. vi. 34. See Strabon.
Geograph. l. xi. p. 764, edit. Casaub.}

\footnotetext[84]{Peter Patricius (in Excerpt. Leg. p. 30) is the
only writer who mentions the Iberian article of the treaty.}

The arduous work of rescuing the distressed empire from tyrants
and barbarians had now been completely achieved by a succession
of Illyrian peasants. As soon as Diocletian entered into the
twentieth year of his reign, he celebrated that memorable æra, as
well as the success of his arms, by the pomp of a Roman triumph.\footnotemark[85]
Maximian, the equal partner of his power, was his only
companion in the glory of that day. The two Cæsars had fought and
conquered, but the merit of their exploits was ascribed,
according to the rigor of ancient maxims, to the auspicious
influence of their fathers and emperors.\footnotemark[86] The triumph of
Diocletian and Maximian was less magnificent, perhaps, than those
of Aurelian and Probus, but it was dignified by several
circumstances of superior fame and good fortune. Africa and
Britain, the Rhine, the Danube, and the Nile, furnished their
respective trophies; but the most distinguished ornament was of a
more singular nature, a Persian victory followed by an important
conquest. The representations of rivers, mountains, and
provinces, were carried before the Imperial car. The images of
the captive wives, the sisters, and the children of the Great
King, afforded a new and grateful spectacle to the vanity of the
people.\footnotemark[87] In the eyes of posterity, this triumph is remarkable,
by a distinction of a less honorable kind. It was the last that
Rome ever beheld. Soon after this period, the emperors ceased to
vanquish, and Rome ceased to be the capital of the empire.

\footnotetext[85]{Euseb. in Chron. Pagi ad annum. Till the discovery
of the treatise De Mortibus Persecutorum, it was not certain that
the triumph and the Vicennalia was celebrated at the same time.}

\footnotetext[86]{At the time of the Vicennalia, Galerius seems to
have kept station on the Danube. See Lactant. de M. P. c. 38.}

\footnotetext[87]{Eutropius (ix. 27) mentions them as a part of the
triumph. As the persons had been restored to Narses, nothing more
than their images could be exhibited.}

The spot on which Rome was founded had been consecrated by
ancient ceremonies and imaginary miracles. The presence of some
god, or the memory of some hero, seemed to animate every part of
the city, and the empire of the world had been promised to the
Capitol.\footnotemark[88] The native Romans felt and confessed the power of
this agreeable illusion. It was derived from their ancestors, had
grown up with their earliest habits of life, and was protected,
in some measure, by the opinion of political utility. The form
and the seat of government were intimately blended together, nor
was it esteemed possible to transport the one without destroying
the other.\footnotemark[89] But the sovereignty of the capital was gradually
annihilated in the extent of conquest; the provinces rose to the
same level, and the vanquished nations acquired the name and
privileges, without imbibing the partial affections, of Romans.
During a long period, however, the remains of the ancient
constitution, and the influence of custom, preserved the dignity
of Rome. The emperors, though perhaps of African or Illyrian
extraction, respected their adopted country, as the seat of their
power, and the centre of their extensive dominions. The
emergencies of war very frequently required their presence on the
frontiers; but Diocletian and Maximian were the first Roman
princes who fixed, in time of peace, their ordinary residence in
the provinces; and their conduct, however it might be suggested
by private motives, was justified by very specious considerations
of policy. The court of the emperor of the West was, for the most
part, established at Milan, whose situation, at the foot of the
Alps, appeared far more convenient than that of Rome, for the
important purpose of watching the motions of the barbarians of
Germany. Milan soon assumed the splendor of an Imperial city. The
houses are described as numerous and well built; the manners of
the people as polished and liberal. A circus, a theatre, a mint,
a palace, baths, which bore the name of their founder Maximian;
porticos adorned with statues, and a double circumference of
walls, contributed to the beauty of the new capital; nor did it
seem oppressed even by the proximity of Rome.\footnotemark[90] To rival the
majesty of Rome was the ambition likewise of Diocletian, who
employed his leisure, and the wealth of the East, in the
embellishment of Nicomedia, a city placed on the verge of Europe
and Asia, almost at an equal distance between the Danube and the
Euphrates. By the taste of the monarch, and at the expense of the
people, Nicomedia acquired, in the space of a few years, a degree
of magnificence which might appear to have required the labor of
ages, and became inferior only to Rome, Alexandria, and Antioch,
in extent of populousness.\footnotemark[91] The life of Diocletian and Maximian
was a life of action, and a considerable portion of it was spent
in camps, or in the long and frequent marches; but whenever the
public business allowed them any relaxation, they seemed to have
retired with pleasure to their favorite residences of Nicomedia
and Milan. Till Diocletian, in the twentieth year of his reign,
celebrated his Roman triumph, it is extremely doubtful whether he
ever visited the ancient capital of the empire. Even on that
memorable occasion his stay did not exceed two months. Disgusted
with the licentious familiarity of the people, he quitted Rome
with precipitation thirteen days before it was expected that he
should have appeared in the senate, invested with the ensigns of
the consular dignity.\footnotemark[92]

\footnotetext[88]{Livy gives us a speech of Camillus on that subject,
(v. 51—55,) full of eloquence and sensibility, in opposition to a
design of removing the seat of government from Rome to the
neighboring city of Veii.}

\footnotetext[89]{Julius Cæsar was reproached with the intention of
removing the empire to Ilium or Alexandria. See Sueton. in Cæsar.
c. 79. According to the ingenious conjecture of Le Fevre and
Dacier, the ode of the third book of Horace was intended to
divert from the execution of a similar design.}

\footnotetext[90]{See Aurelius Victor, who likewise mentions the
buildings erected by Maximian at Carthage, probably during the
Moorish war. We shall insert some verses of Ausonius de Clar.
Urb. v.—— Et Mediolani miræomnia: copia rerum; Innumeræ cultæque
domus; facunda virorum Ingenia, et mores læti: tum duplice muro
Amplificata loci species; populique voluptas Circus; et inclusi
moles cuneata Theatri; Templa, Palatinæque arces, opulensque
Moneta, Et regio Herculei celebris sub honore lavacri. Cunctaque
marmoreis ornata Peristyla signis; Moeniaque in valli formam
circumdata labro, Omnia quæ magnis operum velut æmula formis
Excellunt: nec juncta premit vicinia Romæ.}

\footnotetext[91]{Lactant. de M. P. c. 17. Libanius, Orat. viii. p.
203.}

\footnotetext[92]{Lactant. de M. P. c. 17. On a similar occasion,
Ammianus mentions the dicacitas plebis, as not very agreeable to
an Imperial ear. (See l. xvi. c. 10.)}

The dislike expressed by Diocletian towards Rome and Roman
freedom was not the effect of momentary caprice, but the result
of the most artful policy. That crafty prince had framed a new
system of Imperial government, which was afterwards completed by
the family of Constantine; and as the image of the old
constitution was religiously preserved in the senate, he resolved
to deprive that order of its small remains of power and
consideration. We may recollect, about eight years before the
elevation of Diocletian, the transient greatness, and the
ambitious hopes, of the Roman senate. As long as that enthusiasm
prevailed, many of the nobles imprudently displayed their zeal in
the cause of freedom; and after the successes of Probus had
withdrawn their countenance from the republican party, the
senators were unable to disguise their impotent resentment.

As the sovereign of Italy, Maximian was intrusted with the care
of extinguishing this troublesome, rather than dangerous spirit,
and the task was perfectly suited to his cruel temper. The most
illustrious members of the senate, whom Diocletian always
affected to esteem, were involved, by his colleague, in the
accusation of imaginary plots; and the possession of an elegant
villa, or a well-cultivated estate, was interpreted as a
convincing evidence of guilt.\footnotemark[93] The camp of the Prætorians,
which had so long oppressed, began to protect, the majesty of
Rome; and as those haughty troops were conscious of the decline
of their power, they were naturally disposed to unite their
strength with the authority of the senate. By the prudent
measures of Diocletian, the numbers of the Prætorians were
insensibly reduced, their privileges abolished,\footnotemark[94] and their
place supplied by two faithful legions of Illyricum, who, under
the new titles of Jovians and Herculians, were appointed to
perform the service of the Imperial guards.\footnotemark[95] But the most fatal
though secret wound, which the senate received from the hands of
Diocletian and Maximian, was inflicted by the inevitable
operation of their absence. As long as the emperors resided at
Rome, that assembly might be oppressed, but it could scarcely be
neglected. The successors of Augustus exercised the power of
dictating whatever laws their wisdom or caprice might suggest;
but those laws were ratified by the sanction of the senate. The
model of ancient freedom was preserved in its deliberations and
decrees; and wise princes, who respected the prejudices of the
Roman people, were in some measure obliged to assume the language
and behavior suitable to the general and first magistrate of the
republic. In the armies and in the provinces, they displayed the
dignity of monarchs; and when they fixed their residence at a
distance from the capital, they forever laid aside the
dissimulation which Augustus had recommended to his successors.
In the exercise of the legislative as well as the executive
power, the sovereign advised with his ministers, instead of
consulting the great council of the nation. The name of the
senate was mentioned with honor till the last period of the
empire; the vanity of its members was still flattered with
honorary distinctions;\footnotemark[96] but the assembly which had so long been
the source, and so long the instrument of power, was respectfully
suffered to sink into oblivion. The senate of Rome, losing all
connection with the Imperial court and the actual constitution,
was left a venerable but useless monument of antiquity on the
Capitoline hill.

\footnotetext[93]{Lactantius accuses Maximian of destroying fictis
criminationibus lumina senatus, (De M. P. c. 8.) Aurelius Victor
speaks very doubtfully of the faith of Diocletian towards his
friends.}

\footnotetext[94]{Truncatæ vires urbis, imminuto prætoriarum
cohortium atque in armis vulgi numero. Aurelius Victor.
Lactantius attributes to Galerius the prosecution of the same
plan, (c. 26.)}

\footnotetext[95]{They were old corps stationed in Illyricum; and
according to the ancient establishment, they each consisted of
six thousand men. They had acquired much reputation by the use of
the plumbatæ, or darts loaded with lead. Each soldier carried
five of these, which he darted from a considerable distance, with
great strength and dexterity. See Vegetius, i. 17.}

\footnotetext[96]{See the Theodosian Code, l. vi. tit. ii. with
Godefroy’s commentary.}

