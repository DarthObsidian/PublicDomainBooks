\chapter{Emperors Decius, Gallus, Æmilianus, Valerian And Gallienus}
\section{Part \thesection}

\textit{The Emperors Decius, Gallus, Æmilianus, Valerian, And
Gallienus.—The General Irruption Of The Barbari Ans.—The Thirty
Tyrants.}
\vspace{\onelineskip}

From the great secular games celebrated by Philip, to the death
of the emperor Gallienus, there elapsed twenty years of shame and
misfortune. During that calamitous period, every instant of time
was marked, every province of the Roman world was afflicted, by
barbarous invaders, and military tyrants, and the ruined empire
seemed to approach the last and fatal moment of its dissolution.
The confusion of the times, and the scarcity of authentic
memorials, oppose equal difficulties to the historian, who
attempts to preserve a clear and unbroken thread of narration.
Surrounded with imperfect fragments, always concise, often
obscure, and sometimes contradictory, he is reduced to collect,
to compare, and to conjecture: and though he ought never to place
his conjectures in the rank of facts, yet the knowledge of human
nature, and of the sure operation of its fierce and unrestrained
passions, might, on some occasions, supply the want of historical
materials.

There is not, for instance, any difficulty in conceiving, that
the successive murders of so many emperors had loosened all the
ties of allegiance between the prince and people; that all the
generals of Philip were disposed to imitate the example of their
master; and that the caprice of armies, long since habituated to
frequent and violent revolutions, might every day raise to the
throne the most obscure of their fellow-soldiers. History can
only add, that the rebellion against the emperor Philip broke out
in the summer of the year two hundred and forty-nine, among the
legions of Mæsia; and that a subaltern officer,\footnotemark[1] named Marinus,
was the object of their seditious choice. Philip was alarmed. He
dreaded lest the treason of the Mæsian army should prove the
first spark of a general conflagration. Distracted with the
consciousness of his guilt and of his danger, he communicated the
intelligence to the senate. A gloomy silence prevailed, the
effect of fear, and perhaps of disaffection; till at length
Decius, one of the assembly, assuming a spirit worthy of his
noble extraction, ventured to discover more intrepidity than the
emperor seemed to possess. He treated the whole business with
contempt, as a hasty and inconsiderate tumult, and Philip’s rival
as a phantom of royalty, who in a very few days would be
destroyed by the same inconstancy that had created him. The
speedy completion of the prophecy inspired Philip with a just
esteem for so able a counsellor; and Decius appeared to him the
only person capable of restoring peace and discipline to an army
whose tumultuous spirit did not immediately subside after the
murder of Marinus. Decius,\footnotemark[2] who long resisted his own
nomination, seems to have insinuated the danger of presenting a
leader of merit to the angry and apprehensive minds of the
soldiers; and his prediction was again confirmed by the event.
The legions of Mæsia forced their judge to become their
accomplice. They left him only the alternative of death or the
purple. His subsequent conduct, after that decisive measure, was
unavoidable. He conducted, or followed, his army to the confines
of Italy, whither Philip, collecting all his force to repel the
formidable competitor whom he had raised up, advanced to meet
him. The Imperial troops were superior in number; but the rebels
formed an army of veterans, commanded by an able and experienced
leader. Philip was either killed in the battle, or put to death a
few days afterwards at Verona. His son and associate in the
empire was massacred at Rome by the Prætorian guards; and the
victorious Decius, with more favorable circumstances than the
ambition of that age can usually plead, was universally
acknowledged by the senate and provinces. It is reported, that,
immediately after his reluctant acceptance of the title of
Augustus, he had assured Philip, by a private message, of his
innocence and loyalty, solemnly protesting, that, on his arrival
on Italy, he would resign the Imperial ornaments, and return to
the condition of an obedient subject. His professions might be
sincere; but in the situation where fortune had placed him, it
was scarcely possible that he could either forgive or be
forgiven.\footnotemark[3]

\footnotetext[1]{The expression used by Zosimus and Zonaras may
signify that Marinus commanded a century, a cohort, or a legion.}

\footnotetext[2]{His birth at Bubalia, a little village in Pannonia,
(Eutrop. ix. Victor. in Cæsarib. et Epitom.,) seems to
contradict, unless it was merely accidental, his supposed descent
from the Decii. Six hundred years had bestowed nobility on the
Decii: but at the commencement of that period, they were only
plebeians of merit, and among the first who shared the consulship
with the haughty patricians. Plebeine Deciorum animæ, \&c.
Juvenal, Sat. viii. 254. See the spirited speech of Decius, in
Livy. x. 9, 10.}

\footnotetext[3]{Zosimus, l. i. p. 20, c. 22. Zonaras, l. xii. p.
624, edit. Louvre.}

The emperor Decius had employed a few months in the works of
peace and the administration of justice, when he was summoned to
the banks of the Danube by the invasion of the Goths. This is the
first considerable occasion in which history mentions that great
people, who afterwards broke the Roman power, sacked the Capitol,
and reigned in Gaul, Spain, and Italy. So memorable was the part
which they acted in the subversion of the Western empire, that
the name of Goths is frequently but improperly used as a general
appellation of rude and warlike barbarism.

In the beginning of the sixth century, and after the conquest of
Italy, the Goths, in possession of present greatness, very
naturally indulged themselves in the prospect of past and of
future glory. They wished to preserve the memory of their
ancestors, and to transmit to posterity their own achievements.
The principal minister of the court of Ravenna, the learned
Cassiodorus, gratified the inclination of the conquerors in a
Gothic history, which consisted of twelve books, now reduced to
the imperfect abridgment of Jornandes.\footnotemark[4] These writers passed
with the most artful conciseness over the misfortunes of the
nation, celebrated its successful valor, and adorned the triumph
with many Asiatic trophies, that more properly belonged to the
people of Scythia. On the faith of ancient songs, the uncertain,
but the only memorials of barbarians, they deduced the first
origin of the Goths from the vast island, or peninsula, of
Scandinavia.\footnotemark[5] \footnotemark[501] That extreme country of the North was not
unknown to the conquerors of Italy: the ties of ancient
consanguinity had been strengthened by recent offices of
friendship; and a Scandinavian king had cheerfully abdicated his
savage greatness, that he might pass the remainder of his days in
the peaceful and polished court of Ravenna.\footnotemark[6] Many vestiges,
which cannot be ascribed to the arts of popular vanity, attest
the ancient residence of the Goths in the countries beyond the
Rhine. From the time of the geographer Ptolemy, the southern part
of Sweden seems to have continued in the possession of the less
enterprising remnant of the nation, and a large territory is even
at present divided into east and west Gothland. During the middle
ages, (from the ninth to the twelfth century,) whilst
Christianity was advancing with a slow progress into the North,
the Goths and the Swedes composed two distinct and sometimes
hostile members of the same monarchy.\footnotemark[7] The latter of these two
names has prevailed without extinguishing the former. The Swedes,
who might well be satisfied with their own fame in arms, have, in
every age, claimed the kindred glory of the Goths. In a moment of
discontent against the court of Rome, Charles the Twelfth
insinuated, that his victorious troops were not degenerated from
their brave ancestors, who had already subdued the mistress of
the world.\footnotemark[8]

\footnotetext[4]{See the prefaces of Cassiodorus and Jornandes; it is
surprising that the latter should be omitted in the excellent
edition, published by Grotius, of the Gothic writers.}

\footnotetext[5]{On the authority of Ablavius, Jornandes quotes some
old Gothic chronicles in verse. De Reb. Geticis, c. 4.}

\footnotetext[501]{The Goths have inhabited Scandinavia, but it was
not their original habitation. This great nation was anciently of
the Suevian race; it occupied, in the time of Tacitus, and long
before, Mecklenburgh, Pomerania Southern Prussia and the
north-west of Poland. A little before the birth of J. C., and in
the first years of that century, they belonged to the kingdom of
Marbod, king of the Marcomanni: but Cotwalda, a young Gothic
prince, delivered them from that tyranny, and established his own
power over the kingdom of the Marcomanni, already much weakened
by the victories of Tiberius. The power of the Goths at that time
must have been great: it was probably from them that the Sinus
Codanus (the Baltic) took this name, as it was afterwards called
Mare Suevicum, and Mare Venedicum, during the superiority of the
proper Suevi and the Venedi. The epoch in which the Goths passed
into Scandinavia is unknown. See Adelung, Hist. of Anc. Germany,
p. 200. Gatterer, Hist. Univ. 458.—G. ——M. St. Martin observes,
that the Scandinavian descent of the Goths rests on the authority
of Jornandes, who professed to derive it from the traditions of
the Goths. He is supported by Procopius and Paulus Diaconus. Yet
the Goths are unquestionably the same with the Getæ of the
earlier historians. St. Martin, note on Le Beau, Hist. du bas
Empire, iii. 324. The identity of the Getæ and Goths is by no
means generally admitted. On the whole, they seem to be one vast
branch of the Indo-Teutonic race, who spread irregularly towards
the north of Europe, and at different periods, and in different
regions, came in contact with the more civilized nations of the
south. At this period, there seems to have been a reflux of these
Gothic tribes from the North. Malte Brun considers that there are
strong grounds for receiving the Islandic traditions commented by
the Danish Varro, M. Suhm. From these, and the voyage of Pytheas,
which Malte Brun considers genuine, the Goths were in possession
of Scandinavia, Ey-Gothland, 250 years before J. C., and of a
tract on the continent (Reid-Gothland) between the mouths of the
Vistula and the Oder. In their southern migration, they followed
the course of the Vistula; afterwards, of the Dnieper. Malte
Brun, Geogr. i. p. 387, edit. 1832. Geijer, the historian of
Sweden, ably maintains the Scandinavian origin of the Goths. The
Gothic language, according to Bopp, is the link between the
Sanscrit and the modern Teutonic dialects: “I think that I am
reading Sanscrit when I am reading Olphilas.” Bopp, Conjugations
System der Sanscrit Sprache, preface, p. x—M.}

\footnotetext[6]{Jornandes, c. 3.}

\footnotetext[7]{See in the Prolegomena of Grotius some large
extracts from Adam of Bremen, and Saxo-Grammaticus. The former
wrote in the year 1077, the latter flourished about the year
1200.}

\footnotetext[8]{Voltaire, Histoire de Charles XII. l. iii. When the
Austrians desired the aid of the court of Rome against Gustavus
Adolphus, they always represented that conqueror as the lineal
successor of Alaric. Harte’s History of Gustavus, vol. ii. p.
123.}

Till the end of the eleventh century, a celebrated temple
subsisted at Upsal, the most considerable town of the Swedes and
Goths. It was enriched with the gold which the Scandinavians had
acquired in their piratical adventures, and sanctified by the
uncouth representations of the three principal deities, the god
of war, the goddess of generation, and the god of thunder. In the
general festival, that was solemnized every ninth year, nine
animals of every species (without excepting the human) were
sacrificed, and their bleeding bodies suspended in the sacred
grove adjacent to the temple.\footnotemark[9] The only traces that now subsist
of this barbaric superstition are contained in the Edda,\footnotemark[901] a
system of mythology, compiled in Iceland about the thirteenth
century, and studied by the learned of Denmark and Sweden, as the
most valuable remains of their ancient traditions.

\footnotetext[9]{See Adam of Bremen in Grotii Prolegomenis, p. 105.
The temple of Upsal was destroyed by Ingo, king of Sweden, who
began his reign in the year 1075, and about fourscore years
afterwards, a Christian cathedral was erected on its ruins. See
Dalin’s History of Sweden, in the Bibliotheque Raisonee.}

\footnotetext[901]{The Eddas have at length been made accessible to
European scholars by the completion of the publication of the
Sæmundine Edda by the Arna Magnæan Commission, in 3 vols. 4to.,
with a copious lexicon of northern mythology.—M.}

Notwithstanding the mysterious obscurity of the Edda, we can
easily distinguish two persons confounded under the name of Odin;
the god of war, and the great legislator of Scandinavia. The
latter, the Mahomet of the North, instituted a religion adapted
to the climate and to the people. Numerous tribes on either side
of the Baltic were subdued by the invincible valor of Odin, by
his persuasive eloquence, and by the fame which he acquired of a
most skilful magician. The faith that he had propagated, during a
long and prosperous life, he confirmed by a voluntary death.
Apprehensive of the ignominious approach of disease and
infirmity, he resolved to expire as became a warrior. In a solemn
assembly of the Swedes and Goths, he wounded himself in nine
mortal places, hastening away (as he asserted with his dying
voice) to prepare the feast of heroes in the palace of the God of
war.\footnotemark[10]

\footnotetext[10]{Mallet, Introduction a l’Histoire du Dannemarc.}

The native and proper habitation of Odin is distinguished by the
appellation of As-gard. The happy resemblance of that name with
As-burg, or As-of,\footnotemark[11] words of a similar signification, has given
rise to an historical system of so pleasing a contexture, that we
could almost wish to persuade ourselves of its truth. It is
supposed that Odin was the chief of a tribe of barbarians which
dwelt on the banks of the Lake Mæotis, till the fall of
Mithridates and the arms of Pompey menaced the North with
servitude. That Odin, yielding with indignant fury to a power he
was unable to resist, conducted his tribe from the frontiers of
the Asiatic Sarmatia into Sweden, with the great design of
forming, in that inaccessible retreat of freedom, a religion and
a people which, in some remote age, might be subservient to his
immortal revenge; when his invincible Goths, armed with martial
fanaticism, should issue in numerous swarms from the neighborhood
of the Polar circle, to chastise the oppressors of mankind.\footnotemark[12]

\footnotetext[11]{Mallet, c. iv. p. 55, has collected from Strabo,
Pliny, Ptolemy, and Stephanus Byzantinus, the vestiges of such a
city and people.}

\footnotetext[12]{This wonderful expedition of Odin, which, by
deducting the enmity of the Goths and Romans from so memorable a
cause, might supply the noble groundwork of an epic poem, cannot
safely be received as authentic history. According to the obvious
sense of the Edda, and the interpretation of the most skilful
critics, As-gard, instead of denoting a real city of the Asiatic
Sarmatia, is the fictitious appellation of the mystic abode of
the gods, the Olympus of Scandinavia; from whence the prophet was
supposed to descend, when he announced his new religion to the
Gothic nations, who were already seated in the southern parts of
Sweden. * Note: A curious letter may be consulted on this subject
from the Swede, Ihre counsellor in the Chancery of Upsal, printed
at Upsal by Edman, in 1772 and translated into German by M.
Schlozer. Gottingen, printed for Dietericht, 1779.—G. ——Gibbon,
at a later period of his work, recanted his opinion of the truth
of this expedition of Odin. The Asiatic origin of the Goths is
almost certain from the affinity of their language to the
Sanscrit and Persian; but their northern writers, when all
mythology was reduced to hero worship.—M.}

If so many successive generations of Goths were capable of
preserving a faint tradition of their Scandinavian origin, we
must not expect, from such unlettered barbarians, any distinct
account of the time and circumstances of their emigration. To
cross the Baltic was an easy and natural attempt. The inhabitants
of Sweden were masters of a sufficient number of large vessels,
with oars,\footnotemark[13] and the distance is little more than one hundred
miles from Carlscroon to the nearest ports of Pomerania and
Prussia. Here, at length, we land on firm and historic ground. At
least as early as the Christian æra,\footnotemark[14] and as late as the age of
the Antonines,\footnotemark[15] the Goths were established towards the mouth of
the Vistula, and in that fertile province where the commercial
cities of Thorn, Elbing, Köningsberg, and Dantzick, were long
afterwards founded.\footnotemark[16] Westward of the Goths, the numerous tribes
of the Vandals were spread along the banks of the Oder, and the
sea-coast of Pomerania and Mecklenburgh. A striking resemblance
of manners, complexion, religion, and language, seemed to
indicate that the Vandals and the Goths were originally one great
people.\footnotemark[17] The latter appear to have been subdivided into
Ostrogoths, Visigoths, and Gepidæ.\footnotemark[18] The distinction among the
Vandals was more strongly marked by the independent names of
Heruli, Burgundians, Lombards, and a variety of other petty
states, many of which, in a future age, expanded themselves into
powerful monarchies.\footnotemark[181]

\footnotetext[13]{Tacit. Germania, c. 44.}

\footnotetext[14]{Tacit. Annal. ii. 62. If we could yield a firm
assent to the navigations of Pytheas of Marseilles, we must allow
that the Goths had passed the Baltic at least three hundred years
before Christ.}

\footnotetext[15]{Ptolemy, l. ii.}

\footnotetext[16]{By the German colonies who followed the arms of the
Teutonic knights. The conquest and conversion of Prussia were
completed by those adventurers in the thirteenth century.}

\footnotetext[17]{Pliny (Hist. Natur. iv. 14) and Procopius (in Bell.
Vandal. l. i. c. l) agree in this opinion. They lived in distant
ages, and possessed different means of investigating the truth.}

\footnotetext[18]{The Ostro and Visi, the eastern and western Goths,
obtained those denominations from their original seats in
Scandinavia. In all their future marches and settlements they
preserved, with their names, the same relative situation. When
they first departed from Sweden, the infant colony was contained
in three vessels. The third, being a heavy sailer, lagged behind,
and the crew, which afterwards swelled into a nation, received
from that circumstance the appellation of Gepidæ or Loiterers.
Jornandes, c. 17. * Note: It was not in Scandinavia that the
Goths were divided into Ostrogoths and Visigoths; that division
took place after their irruption into Dacia in the third century:
those who came from Mecklenburgh and Pomerania were called
Visigoths; those who came from the south of Prussia, and the
northwest of Poland, called themselves Ostrogoths. Adelung, Hist.
All. p. 202 Gatterer, Hist. Univ. 431.—G.}

\footnotetext[181]{This opinion is by no means probable. The Vandals
and the Goths equally belonged to the great division of the
Suevi, but the two tribes were very different. Those who have
treated on this part of history, appear to me to have neglected
to remark that the ancients almost always gave the name of the
dominant and conquering people to all the weaker and conquered
races. So Pliny calls Vindeli, Vandals, all the people of the
north-east of Europe, because at that epoch the Vandals were
doubtless the conquering tribe. Cæsar, on the contrary, ranges
under the name of Suevi, many of the tribes whom Pliny reckons as
Vandals, because the Suevi, properly so called, were then the
most powerful tribe in Germany. When the Goths, become in their
turn conquerors, had subjugated the nations whom they encountered
on their way, these nations lost their name with their liberty,
and became of Gothic origin. The Vandals themselves were then
considered as Goths; the Heruli, the Gepidæ, \&c., suffered the
same fate. A common origin was thus attributed to tribes who had
only been united by the conquests of some dominant nation, and
this confusion has given rise to a number of historical
errors.—G. ——M. St. Martin has a learned note (to Le Beau, v.
261) on the origin of the Vandals. The difficulty appears to be
in rejecting the close analogy of the name with the Vend or
Wendish race, who were of Sclavonian, not of Suevian or German,
origin. M. St. Martin supposes that the different races spread
from the head of the Adriatic to the Baltic, and even the Veneti,
on the shores of the Adriatic, the Vindelici, the tribes which
gave their name to Vindobena, Vindoduna, Vindonissa, were
branches of the same stock with the Sclavonian Venedi, who at one
time gave their name to the Baltic; that they all spoke dialects
of the Wendish language, which still prevails in Carinthia,
Carniola, part of Bohemia, and Lusatia, and is hardly extinct in
Mecklenburgh and Pomerania. The Vandal race, once so fearfully
celebrated in the annals of mankind, has so utterly perished from
the face of the earth, that we are not aware that any vestiges of
their language can be traced, so as to throw light on the
disputed question of their German, their Sclavonian, or
independent origin. The weight of ancient authority seems against
M. St. Martin’s opinion. Compare, on the Vandals, Malte Brun.
394. Also Gibbon’s note, c. xli. n. 38.—M.}

In the age of the Antonines, the Goths were still seated in
Prussia. About the reign of Alexander Severus, the Roman province
of Dacia had already experienced their proximity by frequent and
destructive inroads.\footnotemark[19] In this interval, therefore, of about
seventy years we must place the second migration of the Goths
from the Baltic to the Euxine; but the cause that produced it
lies concealed among the various motives which actuate the
conduct of unsettled barbarians. Either a pestilence or a famine,
a victory or a defeat, an oracle of the gods or the eloquence of
a daring leader, were sufficient to impel the Gothic arms on the
milder climates of the south. Besides the influence of a martial
religion, the numbers and spirit of the Goths were equal to the
most dangerous adventures. The use of round bucklers and short
swords rendered them formidable in a close engagement; the manly
obedience which they yielded to hereditary kings, gave uncommon
union and stability to their councils;\footnotemark[20] and the renowned Amala,
the hero of that age, and the tenth ancestor of Theodoric, king
of Italy, enforced, by the ascendant of personal merit, the
prerogative of his birth, which he derived from the \textit{Anses}, or
demigods of the Gothic nation.\footnotemark[21]

\footnotetext[19]{See a fragment of Peter Patricius in the Excerpta
Legationum and with regard to its probable date, see Tillemont,
Hist, des Empereurs, tom. iii. p. 346.}

\footnotetext[20]{Omnium harum gentium insigne, rotunda scuta, breves
gladii, et erga rages obsequium. Tacit. Germania, c. 43. The
Goths probably acquired their iron by the commerce of amber.}

\footnotetext[21]{Jornandes, c. 13, 14.}

The fame of a great enterprise excited the bravest warriors from
all the Vandalic states of Germany, many of whom are seen a few
years afterwards combating under the common standard of the
Goths.\footnotemark[22] The first motions of the emigrants carried them to the
banks of the Prypec, a river universally conceived by the
ancients to be the southern branch of the Borysthenes.\footnotemark[23] The
windings of that great stream through the plains of Poland and
Russia gave a direction to their line of march, and a constant
supply of fresh water and pasturage to their numerous herds of
cattle. They followed the unknown course of the river, confident
in their valor, and careless of whatever power might oppose their
progress. The Bastarnæ and the Venedi were the first who
presented themselves; and the flower of their youth, either from
choice or compulsion, increased the Gothic army. The Bastarnæ
dwelt on the northern side of the Carpathian Mountains: the
immense tract of land that separated the Bastarnæ from the
savages of Finland was possessed, or rather wasted, by the
Venedi;\footnotemark[24] we have some reason to believe that the first of these
nations, which distinguished itself in the Macedonian war,\footnotemark[25] and
was afterwards divided into the formidable tribes of the Peucini,
the Borani, the Carpi, \&c., derived its origin from the Germans.\footnotemark[251]
With better authority, a Sarmatian extraction may be assigned
to the Venedi, who rendered themselves so famous in the middle
ages.\footnotemark[26] But the confusion of blood and manners on that doubtful
frontier often perplexed the most accurate observers.\footnotemark[27] As the
Goths advanced near the Euxine Sea, they encountered a purer race
of Sarmatians, the Jazyges, the Alani,\footnotemark[271] and the Roxolani; and
they were probably the first Germans who saw the mouths of the
Borysthenes, and of the Tanais. If we inquire into the
characteristic marks of the people of Germany and of Sarmatia, we
shall discover that those two great portions of human kind were
principally distinguished by fixed huts or movable tents, by a
close dress or flowing garments, by the marriage of one or of
several wives, by a military force, consisting, for the most
part, either of infantry or cavalry; and above all, by the use of
the Teutonic, or of the Sclavonian language; the last of which
has been diffused by conquest, from the confines of Italy to the
neighborhood of Japan.

\footnotetext[22]{The Heruli, and the Uregundi or Burgundi, are
particularly mentioned. See Mascou’s History of the Germans, l.
v. A passage in the Augustan History, p. 28, seems to allude to
this great emigration. The Marcomannic war was partly occasioned
by the pressure of barbarous tribes, who fled before the arms of
more northern barbarians.}

\footnotetext[23]{D’Anville, Geographie Ancienne, and the third part
of his incomparable map of Europe.}

\footnotetext[24]{Tacit. Germania, c. 46.}

\footnotetext[25]{Cluver. Germ. Antiqua, l. iii. c. 43.}

\footnotetext[251]{The Bastarnæ cannot be considered original
inhabitants of Germany Strabo and Tacitus appear to doubt it;
Pliny alone calls them Germans: Ptolemy and Dion treat them as
Scythians, a vague appellation at this period of history; Livy,
Plutarch, and Diodorus Siculus, call them Gauls, and this is the
most probable opinion. They descended from the Gauls who entered
Germany under Signoesus. They are always found associated with
other Gaulish tribes, such as the Boll, the Taurisci, \&c., and
not to the German tribes. The names of their chiefs or princes,
Chlonix, Chlondicus. Deldon, are not German names. Those who were
settled in the island of Peuce in the Danube, took the name of
Peucini. The Carpi appear in 237 as a Suevian tribe who had made
an irruption into Mæsia. Afterwards they reappear under the
Ostrogoths, with whom they were probably blended. Adelung, p.
236, 278.—G.}

\footnotetext[26]{The Venedi, the Slavi, and the Antes, were the
three great tribes of the same people. Jornandes, 24. * Note
Dagger: They formed the great Sclavonian nation.—G.}

\footnotetext[27]{Tacitus most assuredly deserves that title, and
even his cautious suspense is a proof of his diligent inquiries.}

\footnotetext[271]{Jac. Reineggs supposed that he had found, in the
mountains of Caucasus, some descendants of the Alani. The Tartars
call them Edeki-Alan: they speak a peculiar dialect of the
ancient language of the Tartars of Caucasus. See J. Reineggs’
Descr. of Caucasus, p. 11, 13.—G. According to Klaproth, they are
the Ossetes of the present day in Mount Caucasus and were the
same with the Albanians of antiquity. Klaproth, Hist. de l’Asie,
p. 180.—M.}

