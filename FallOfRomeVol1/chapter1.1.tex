\chapter{The Extent Of The Empire In The Age Of The Antonines.}
\section{Part \thesection.}
\begin{center}
\textbf{\large Introduction.}
\end{center}

\textit{The Extent And Military Force Of The Empire In The Age Of The Antonines.}
\vspace{\onelineskip}

In the second century of the Christian Æra, the empire of Rome
comprehended the fairest part of the earth, and the most
civilized portion of mankind. The frontiers of that extensive
monarchy were guarded by ancient renown and disciplined valor.
The gentle but powerful influence of laws and manners had
gradually cemented the union of the provinces. Their peaceful
inhabitants enjoyed and abused the advantages of wealth and
luxury. The image of a free constitution was preserved with
decent reverence: the Roman senate appeared to possess the
sovereign authority, and devolved on the emperors all the
executive powers of government. During a happy period of more
than fourscore years, the public administration was conducted by
the virtue and abilities of Nerva, Trajan, Hadrian, and the two
Antonines. It is the design of this, and of the two succeeding
chapters, to describe the prosperous condition of their empire;
and afterwards, from the death of Marcus Antoninus, to deduce the
most important circumstances of its decline and fall; a
revolution which will ever be remembered, and is still felt by
the nations of the earth.

The principal conquests of the Romans were achieved under the
republic; and the emperors, for the most part, were satisfied
with preserving those dominions which had been acquired by the
policy of the senate, the active emulations of the consuls, and
the martial enthusiasm of the people. The seven first centuries
were filled with a rapid succession of triumphs; but it was
reserved for Augustus to relinquish the ambitious design of
subduing the whole earth, and to introduce a spirit of moderation
into the public councils. Inclined to peace by his temper and
situation, it was easy for him to discover that Rome, in her
present exalted situation, had much less to hope than to fear
from the chance of arms; and that, in the prosecution of remote
wars, the undertaking became every day more difficult, the event
more doubtful, and the possession more precarious, and less
beneficial. The experience of Augustus added weight to these
salutary reflections, and effectually convinced him that, by the
prudent vigor of his counsels, it would be easy to secure every
concession which the safety or the dignity of Rome might require
from the most formidable barbarians. Instead of exposing his
person and his legions to the arrows of the Parthians, he
obtained, by an honorable treaty, the restitution of the
standards and prisoners which had been taken in the defeat of
Crassus.\footnotemark[1]

\footnotetext[1]{Dion Cassius, (l. liv. p. 736,) with the annotations
of Reimar, who has collected all that Roman vanity has left upon
the subject. The marble of Ancyra, on which Augustus recorded his
own exploits, asserted that \textit{he compelled} the Parthians to
restore the ensigns of Crassus.}

His generals, in the early part of his reign, attempted the
reduction of Ethiopia and Arabia Felix. They marched near a
thousand miles to the south of the tropic; but the heat of the
climate soon repelled the invaders, and protected the un-warlike
natives of those sequestered regions.\footnotemark[2] The northern countries
of Europe scarcely deserved the expense and labor of conquest.
The forests and morasses of Germany were filled with a hardy race
of barbarians, who despised life when it was separated from
freedom; and though, on the first attack, they seemed to yield to
the weight of the Roman power, they soon, by a signal act of
despair, regained their independence, and reminded Augustus of
the vicissitude of fortune.\footnotemark[3] On the death of that emperor, his
testament was publicly read in the senate. He bequeathed, as a
valuable legacy to his successors, the advice of confining the
empire within those limits which nature seemed to have placed as
its permanent bulwarks and boundaries: on the west, the Atlantic
Ocean; the Rhine and Danube on the north; the Euphrates on the
east; and towards the south, the sandy deserts of Arabia and
Africa.\footnotemark[4]

\footnotetext[2]{Strabo, (l. xvi. p. 780,) Pliny the elder, (Hist.
Natur. l. vi. c. 32, 35, [28, 29,]) and Dion Cassius, (l. liii.
p. 723, and l. liv. p. 734,) have left us very curious details
concerning these wars. The Romans made themselves masters of
Mariaba, or Merab, a city of Arabia Felix, well known to the
Orientals. (See Abulfeda and the Nubian geography, p. 52) They
were arrived within three days’ journey of the spice country, the
rich object of their invasion.

Note: It is this city of Merab that the Arabs say was the
residence of Belkis, queen of Saba, who desired to see Solomon. A
dam, by which the waters collected in its neighborhood were kept
back, having been swept away, the sudden inundation destroyed
this city, of which, nevertheless, vestiges remain. It bordered
on a country called Adramout, where a particular aromatic plant
grows: it is for this reason that we real in the history of the
Roman expedition, that they were arrived within three days’
journey of the spice country.—G. Compare \textit{Malte-Brun, Geogr}.
Eng. trans. vol. ii. p. 215. The period of this flood has been
copiously discussed by Reiske, (\textit{Program. de vetustâ Epochâ
Arabum, rupturâ cataractæ Merabensis}.) Add. Johannsen, \textit{Hist.
Yemanæ}, p. 282. Bonn, 1828; and see Gibbon, note 16. to Chap.
L.—M.

Note: Two, according to Strabo. The detailed account of Strabo
makes the invaders fail before Marsuabæ this cannot be the same
place as Mariaba. Ukert observes, that Ælius Gallus would not
have failed for want of water before Mariaba. (See M. Guizot’s
note above.) “Either, therefore, they were different places, or
Strabo is mistaken.” (Ukert, \textit{Geographie der Griechen und Römer},
vol. i. p. 181.) Strabo, indeed, mentions Mariaba distinct from
Marsuabæ. Gibbon has followed Pliny in reckoning Mariaba among
the conquests of Gallus. There can be little doubt that he is
wrong, as Gallus did not approach the capital of Sabæa. Compare
the note of the Oxford editor of Strabo.—M.}

\footnotetext[3]{By the slaughter of Varus and his three legions.
See the first book of the Annals of Tacitus. Sueton. in August.
c. 23, and Velleius Paterculus, l. ii. c. 117, \&c. Augustus did
not receive the melancholy news with all the temper and firmness
that might have been expected from his character.}

\footnotetext[4]{Tacit. Annal. l. ii. Dion Cassius, l. lvi. p. 833,
and the speech of Augustus himself, in Julian’s Cæsars. It
receives great light from the learned notes of his French
translator, M. Spanheim.}

Happily for the repose of mankind, the moderate system
recommended by the wisdom of Augustus, was adopted by the fears
and vices of his immediate successors. Engaged in the pursuit of
pleasure, or in the exercise of tyranny, the first Cæsars seldom
showed themselves to the armies, or to the provinces; nor were
they disposed to suffer, that those triumphs which \textit{their}
indolence neglected, should be usurped by the conduct and valor
of their lieutenants. The military fame of a subject was
considered as an insolent invasion of the Imperial prerogative;
and it became the duty, as well as interest, of every Roman
general, to guard the frontiers intrusted to his care, without
aspiring to conquests which might have proved no less fatal to
himself than to the vanquished barbarians.\footnotemark[5]

\footnotetext[5]{Germanicus, Suetonius Paulinus, and Agricola were
checked and recalled in the course of their victories. Corbulo
was put to death. Military merit, as it is admirably expressed by
Tacitus, was, in the strictest sense of the word, \textit{imperatoria
virtus}.}

The only accession which the Roman empire received, during the
first century of the Christian Æra, was the province of Britain.
In this single instance, the successors of Cæsar and Augustus
were persuaded to follow the example of the former, rather than
the precept of the latter. The proximity of its situation to the
coast of Gaul seemed to invite their arms; the pleasing though
doubtful intelligence of a pearl fishery attracted their avarice;\footnotemark[6]
and as Britain was viewed in the light of a distinct and
insulated world, the conquest scarcely formed any exception to
the general system of continental measures. After a war of about
forty years, undertaken by the most stupid,\footnotemark[7] maintained by the
most dissolute, and terminated by the most timid of all the
emperors, the far greater part of the island submitted to the
Roman yoke.\footnotemark[8] The various tribes of Britain possessed valor
without conduct, and the love of freedom without the spirit of
union. They took up arms with savage fierceness; they laid them
down, or turned them against each other, with wild inconsistency;
and while they fought singly, they were successively subdued.
Neither the fortitude of Caractacus, nor the despair of Boadicea,
nor the fanaticism of the Druids, could avert the slavery of
their country, or resist the steady progress of the Imperial
generals, who maintained the national glory, when the throne was
disgraced by the weakest, or the most vicious of mankind. At the
very time when Domitian, confined to his palace, felt the terrors
which he inspired, his legions, under the command of the virtuous
Agricola, defeated the collected force of the Caledonians, at the
foot of the Grampian Hills; and his fleets, venturing to explore
an unknown and dangerous navigation, displayed the Roman arms
round every part of the island. The conquest of Britain was
considered as already achieved; and it was the design of Agricola
to complete and insure his success, by the easy reduction of
Ireland, for which, in his opinion, one legion and a few
auxiliaries were sufficient.\footnotemark[9] The western isle might be improved
into a valuable possession, and the Britons would wear their
chains with the less reluctance, if the prospect and example of
freedom were on every side removed from before their eyes.

\footnotetext[6]{Cæsar himself conceals that ignoble motive; but it
is mentioned by Suetonius, c. 47. The British pearls proved,
however, of little value, on account of their dark and livid
color. Tacitus observes, with reason, (in Agricola, c. 12,) that
it was an inherent defect. “Ego facilius crediderim, naturam
margaritis deesse quam nobis avaritiam.”}

\footnotetext[7]{Claudius, Nero, and Domitian. A hope is expressed by
Pomponius Mela, l. iii. c. 6, (he wrote under Claudius,) that, by
the success of the Roman arms, the island and its savage
inhabitants would soon be better known. It is amusing enough to
peruse such passages in the midst of London.}

\footnotetext[8]{See the admirable abridgment given by Tacitus, in
the life of Agricola, and copiously, though perhaps not
completely, illustrated by our own antiquarians, Camden and
Horsley.}

\footnotetext[9]{The Irish writers, jealous of their national honor,
are extremely provoked on this occasion, both with Tacitus and
with Agricola.}

But the superior merit of Agricola soon occasioned his removal
from the government of Britain; and forever disappointed this
rational, though extensive scheme of conquest. Before his
departure, the prudent general had provided for security as well
as for dominion. He had observed, that the island is almost
divided into two unequal parts by the opposite gulfs, or, as they
are now called, the Friths of Scotland. Across the narrow
interval of about forty miles, he had drawn a line of military
stations, which was afterwards fortified, in the reign of
Antoninus Pius, by a turf rampart, erected on foundations of
stone.\footnotemark[10] This wall of Antoninus, at a small distance beyond the
modern cities of Edinburgh and Glasgow, was fixed as the limit of
the Roman province. The native Caledonians preserved, in the
northern extremity of the island, their wild independence, for
which they were not less indebted to their poverty than to their
valor. Their incursions were frequently repelled and chastised;
but their country was never subdued.\footnotemark[11] The masters of the
fairest and most wealthy climates of the globe turned with
contempt from gloomy hills, assailed by the winter tempest, from
lakes concealed in a blue mist, and from cold and lonely heaths,
over which the deer of the forest were chased by a troop of naked
barbarians.\footnotemark[12]

\footnotetext[10]{See Horsley’s Britannia Romana, l. i. c. 10. Note:
Agricola fortified the line from Dumbarton to Edinburgh,
consequently within Scotland. The emperor Hadrian, during his
residence in Britain, about the year 121, caused a rampart of
earth to be raised between Newcastle and Carlisle. Antoninus
Pius, having gained new victories over the Caledonians, by the
ability of his general, Lollius, Urbicus, caused a new rampart of
earth to be constructed between Edinburgh and Dumbarton. Lastly,
Septimius Severus caused a wall of stone to be built parallel to
the rampart of Hadrian, and on the same locality. See John
Warburton’s Vallum Romanum, or the History and Antiquities of the
Roman Wall. London, 1754, 4to.—W. See likewise a good note on the
Roman wall in Lingard’s History of England, vol. i. p. 40, 4to
edit—M.}

\footnotetext[11]{The poet Buchanan celebrates with elegance and
spirit (see his Sylvæ, v.) the unviolated independence of his
native country. But, if the single testimony of Richard of
Cirencester was sufficient to create a Roman province of
Vespasiana to the north of the wall, that independence would be
reduced within very narrow limits.}

\footnotetext[12]{See Appian (in Proœm.) and the uniform imagery of
Ossian’s Poems, which, according to every hypothesis, were
composed by a native Caledonian.}

Such was the state of the Roman frontiers, and such the maxims of
Imperial policy, from the death of Augustus to the accession of
Trajan. That virtuous and active prince had received the
education of a soldier, and possessed the talents of a general.\footnotemark[13]
The peaceful system of his predecessors was interrupted by
scenes of war and conquest; and the legions, after a long
interval, beheld a military emperor at their head. The first
exploits of Trajan were against the Dacians, the most warlike of
men, who dwelt beyond the Danube, and who, during the reign of
Domitian, had insulted, with impunity, the Majesty of Rome.\footnotemark[14] To
the strength and fierceness of barbarians they added a contempt
for life, which was derived from a warm persuasion of the
immortality and transmigration of the soul.\footnotemark[15] Decebalus, the
Dacian king, approved himself a rival not unworthy of Trajan; nor
did he despair of his own and the public fortune, till, by the
confession of his enemies, he had exhausted every resource both
of valor and policy.\footnotemark[16] This memorable war, with a very short
suspension of hostilities, lasted five years; and as the emperor
could exert, without control, the whole force of the state, it
was terminated by an absolute submission of the barbarians.\footnotemark[17]
The new province of Dacia, which formed a second exception to the
precept of Augustus, was about thirteen hundred miles in
circumference. Its natural boundaries were the Niester, the Teyss
or Tibiscus, the Lower Danube, and the Euxine Sea. The vestiges
of a military road may still be traced from the banks of the
Danube to the neighborhood of Bender, a place famous in modern
history, and the actual frontier of the Turkish and Russian
empires.\footnotemark[18]

\footnotetext[13]{See Pliny’s Panegyric, which seems founded on
facts.}

\footnotetext[14]{Dion Cassius, l. lxvii.}

\footnotetext[15]{Herodotus, l. iv. c. 94. Julian in the Cæsars, with
Spanheims observations.}

\footnotetext[16]{Plin. Epist. viii. 9.}

\footnotetext[17]{Dion Cassius, l. lxviii. p. 1123, 1131. Julian in
Cæsaribus Eutropius, viii. 2, 6. Aurelius Victor in Epitome.}

\footnotetext[18]{See a Memoir of M. d’Anville, on the Province of
Dacia, in the Academie des Inscriptions, tom. xxviii. p.
444—468.}

Trajan was ambitious of fame; and as long as mankind shall
continue to bestow more liberal applause on their destroyers than
on their benefactors, the thirst of military glory will ever be
the vice of the most exalted characters. The praises of
Alexander, transmitted by a succession of poets and historians,
had kindled a dangerous emulation in the mind of Trajan. Like
him, the Roman emperor undertook an expedition against the
nations of the East; but he lamented with a sigh, that his
advanced age scarcely left him any hopes of equalling the renown
of the son of Philip.\footnotemark[19] Yet the success of Trajan, however
transient, was rapid and specious. The degenerate Parthians,
broken by intestine discord, fled before his arms. He descended
the River Tigris in triumph, from the mountains of Armenia to the
Persian Gulf. He enjoyed the honor of being the first, as he was
the last, of the Roman generals, who ever navigated that remote
sea. His fleets ravaged the coast of Arabia; and Trajan vainly
flattered himself that he was approaching towards the confines of
India.\footnotemark[20] Every day the astonished senate received the
intelligence of new names and new nations, that acknowledged his
sway. They were informed that the kings of Bosphorus, Colchos,
Iberia, Albania, Osrhoene, and even the Parthian monarch himself,
had accepted their diadems from the hands of the emperor; that
the independent tribes of the Median and Carduchian hills had
implored his protection; and that the rich countries of Armenia,
Mesopotamia, and Assyria, were reduced into the state of
provinces.\footnotemark[21] But the death of Trajan soon clouded the splendid
prospect; and it was justly to be dreaded, that so many distant
nations would throw off the unaccustomed yoke, when they were no
longer restrained by the powerful hand which had imposed it.

\footnotetext[19]{Trajan’s sentiments are represented in a very just
and lively manner in the Cæsars of Julian.}

\footnotetext[20]{Eutropius and Sextus Rufus have endeavored to
perpetuate the illusion. See a very sensible dissertation of M.
Freret in the Académie des Inscriptions, tom. xxi. p. 55.}

\footnotetext[21]{Dion Cassius, l. lxviii.; and the Abbreviators.}

