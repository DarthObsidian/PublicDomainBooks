\chapter{Tyranny Of Maximin, Rebellion, Civil Wars, Death Of Maximin.}
\section{Part \thesection.}

\textit{The Elevation And Tyranny Of Maximin. — Rebellion In Africa And
Italy, Under The Authority Of The Senate. — Civil Wars And
Seditions. — Violent Deaths Of Maximin And His Son, Of Maximus And
Balbinus, And Of The Three Gordians. — Usurpation And Secular Games
Of Philip.}
\vspace{\onelineskip}

Of the various forms of government which have prevailed in the
world, an hereditary monarchy seems to present the fairest scope
for ridicule. Is it possible to relate without an indignant
smile, that, on the father’s decease, the property of a nation,
like that of a drove of oxen, descends to his infant son, as yet
unknown to mankind and to himself; and that the bravest warriors
and the wisest statesmen, relinquishing their natural right to
empire, approach the royal cradle with bended knees and
protestations of inviolable fidelity? Satire and declamation may
paint these obvious topics in the most dazzling colors, but our
more serious thoughts will respect a useful prejudice, that
establishes a rule of succession, independent of the passions of
mankind; and we shall cheerfully acquiesce in any expedient which
deprives the multitude of the dangerous, and indeed the ideal,
power of giving themselves a master.

In the cool shade of retirement, we may easily devise imaginary
forms of government, in which the sceptre shall be constantly
bestowed on the most worthy, by the free and incorrupt suffrage
of the whole community. Experience overturns these airy fabrics,
and teaches us, that in a large society, the election of a
monarch can never devolve to the wisest, or to the most numerous
part of the people. The army is the only order of men
sufficiently united to concur in the same sentiments, and
powerful enough to impose them on the rest of their
fellow-citizens; but the temper of soldiers, habituated at once
to violence and to slavery, renders them very unfit guardians of
a legal, or even a civil constitution. Justice, humanity, or
political wisdom, are qualities they are too little acquainted
with in themselves, to appreciate them in others. Valor will
acquire their esteem, and liberality will purchase their
suffrage; but the first of these merits is often lodged in the
most savage breasts; the latter can only exert itself at the
expense of the public; and both may be turned against the
possessor of the throne, by the ambition of a daring rival.

The superior prerogative of birth, when it has obtained the
sanction of time and popular opinion, is the plainest and least
invidious of all distinctions among mankind. The acknowledged
right extinguishes the hopes of faction, and the conscious
security disarms the cruelty of the monarch. To the firm
establishment of this idea we owe the peaceful succession and
mild administration of European monarchies. To the defect of it
we must attribute the frequent civil wars, through which an
Asiatic despot is obliged to cut his way to the throne of his
fathers. Yet, even in the East, the sphere of contention is
usually limited to the princes of the reigning house, and as soon
as the more fortunate competitor has removed his brethren by the
sword and the bowstring, he no longer entertains any jealousy of
his meaner subjects. But the Roman empire, after the authority of
the senate had sunk into contempt, was a vast scene of confusion.
The royal, and even noble, families of the provinces had long
since been led in triumph before the car of the haughty
republicans. The ancient families of Rome had successively fallen
beneath the tyranny of the Cæsars; and whilst those princes were
shackled by the forms of a commonwealth, and disappointed by the
repeated failure of their posterity,\footnotemark[1] it was impossible that any
idea of hereditary succession should have taken root in the minds
of their subjects. The right to the throne, which none could
claim from birth, every one assumed from merit. The daring hopes
of ambition were set loose from the salutary restraints of law
and prejudice; and the meanest of mankind might, without folly,
entertain a hope of being raised by valor and fortune to a rank
in the army, in which a single crime would enable him to wrest
the sceptre of the world from his feeble and unpopular master.
After the murder of Alexander Severus, and the elevation of
Maximin, no emperor could think himself safe upon the throne, and
every barbarian peasant of the frontier might aspire to that
august, but dangerous station.

\footnotetext[1]{There had been no example of three successive
generations on the throne; only three instances of sons who
succeeded their fathers. The marriages of the Cæsars
(notwithstanding the permission, and the frequent practice of
divorces) were generally unfruitful.}

About thirty-two years before that event, the emperor Severus,
returning from an eastern expedition, halted in Thrace, to
celebrate, with military games, the birthday of his younger son,
Geta. The country flocked in crowds to behold their sovereign,
and a young barbarian of gigantic stature earnestly solicited, in
his rude dialect, that he might be allowed to contend for the
prize of wrestling. As the pride of discipline would have been
disgraced in the overthrow of a Roman soldier by a Thracian
peasant, he was matched with the stoutest followers of the camp,
sixteen of whom he successively laid on the ground. His victory
was rewarded by some trifling gifts, and a permission to enlist
in the troops. The next day, the happy barbarian was
distinguished above a crowd of recruits, dancing and exulting
after the fashion of his country. As soon as he perceived that he
had attracted the emperor’s notice, he instantly ran up to his
horse, and followed him on foot, without the least appearance of
fatigue, in a long and rapid career. “Thracian,” said Severus
with astonishment, “art thou disposed to wrestle after thy race?”
“Most willingly, sir,” replied the unwearied youth; and, almost
in a breath, overthrew seven of the strongest soldiers in the
army. A gold collar was the prize of his matchless vigor and
activity, and he was immediately appointed to serve in the
horseguards who always attended on the person of the sovereign.\footnotemark[2]

\footnotetext[2]{Hist. August p. 138.}

Maximin, for that was his name, though born on the territories of
the empire, descended from a mixed race of barbarians. His father
was a Goth, and his mother of the nation of the Alani. He
displayed on every occasion a valor equal to his strength; and
his native fierceness was soon tempered or disguised by the
knowledge of the world. Under the reign of Severus and his son,
he obtained the rank of centurion, with the favor and esteem of
both those princes, the former of whom was an excellent judge of
merit. Gratitude forbade Maximin to serve under the assassin of
Caracalla. Honor taught him to decline the effeminate insults of
Elagabalus. On the accession of Alexander he returned to court,
and was placed by that prince in a station useful to the service,
and honorable to himself. The fourth legion, to which he was
appointed tribune, soon became, under his care, the best
disciplined of the whole army. With the general applause of the
soldiers, who bestowed on their favorite hero the names of Ajax
and Hercules, he was successively promoted to the first military
command;\footnotemark[3] and had not he still retained too much of his savage
origin, the emperor might perhaps have given his own sister in
marriage to the son of Maximin.\footnotemark[4]

\footnotetext[3]{Hist. August. p. 140. Herodian, l. vi. p. 223.
Aurelius Victor. By comparing these authors, it should seem that
Maximin had the particular command of the Tribellian horse, with
the general commission of disciplining the recruits of the whole
army. His biographer ought to have marked, with more care, his
exploits, and the successive steps of his military promotions.}

\footnotetext[4]{See the original letter of Alexander Severus, Hist.
August. p. 149.}

Instead of securing his fidelity, these favors served only to
inflame the ambition of the Thracian peasant, who deemed his
fortune inadequate to his merit, as long as he was constrained to
acknowledge a superior. Though a stranger to real wisdom, he was
not devoid of a selfish cunning, which showed him that the
emperor had lost the affection of the army, and taught him to
improve their discontent to his own advantage. It is easy for
faction and calumny to shed their poison on the administration of
the best of princes, and to accuse even their virtues by artfully
confounding them with those vices to which they bear the nearest
affinity. The troops listened with pleasure to the emissaries of
Maximin. They blushed at their own ignominious patience, which,
during thirteen years, had supported the vexatious discipline
imposed by an effeminate Syrian, the timid slave of his mother
and of the senate. It was time, they cried, to cast away that
useless phantom of the civil power, and to elect for their prince
and general a real soldier, educated in camps, exercised in war,
who would assert the glory, and distribute among his companions
the treasures, of the empire. A great army was at that time
assembled on the banks of the Rhine, under the command of the
emperor himself, who, almost immediately after his return from
the Persian war, had been obliged to march against the barbarians
of Germany. The important care of training and reviewing the new
levies was intrusted to Maximin. One day, as he entered the field
of exercise, the troops, either from a sudden impulse, or a
formed conspiracy, saluted him emperor, silenced by their loud
acclamations his obstinate refusal, and hastened to consummate
their rebellion by the murder of Alexander Severus.

The circumstances of his death are variously related. The
writers, who suppose that he died in ignorance of the ingratitude
and ambition of Maximin affirm that, after taking a frugal repast
in the sight of the army, he retired to sleep, and that, about
the seventh hour of the day, a part of his own guards broke into
the imperial tent, and, with many wounds, assassinated their
virtuous and unsuspecting prince.\footnotemark[5] If we credit another, and
indeed a more probable account, Maximin was invested with the
purple by a numerous detachment, at the distance of several miles
from the head-quarters; and he trusted for success rather to the
secret wishes than to the public declarations of the great army.
Alexander had sufficient time to awaken a faint sense of loyalty
among the troops; but their reluctant professions of fidelity
quickly vanished on the appearance of Maximin, who declared
himself the friend and advocate of the military order, and was
unanimously acknowledged emperor of the Romans by the applauding
legions. The son of Mamæa, betrayed and deserted, withdrew into
his tent, desirous at least to conceal his approaching fate from
the insults of the multitude. He was soon followed by a tribune
and some centurions, the ministers of death; but instead of
receiving with manly resolution the inevitable stroke, his
unavailing cries and entreaties disgraced the last moments of his
life, and converted into contempt some portion of the just pity
which his innocence and misfortunes must inspire. His mother,
Mamæa, whose pride and avarice he loudly accused as the cause of
his ruin, perished with her son. The most faithful of his friends
were sacrificed to the first fury of the soldiers. Others were
reserved for the more deliberate cruelty of the usurper; and
those who experienced the mildest treatment, were stripped of
their employments, and ignominiously driven from the court and
army.\footnotemark[6]

\footnotetext[5]{Hist. August. p. 135. I have softened some of the
most improbable circumstances of this wretched biographer. From
his ill-worded narration, it should seem that the prince’s
buffoon having accidentally entered the tent, and awakened the
slumbering monarch, the fear of punishment urged him to persuade
the disaffected soldiers to commit the murder.}

\footnotetext[6]{Herodian, l. vi. 223-227.}

The former tyrants, Caligula and Nero, Commodus, and Caracalla,
were all dissolute and unexperienced youths,\footnotemark[7] educated in the
purple, and corrupted by the pride of empire, the luxury of Rome,
and the perfidious voice of flattery. The cruelty of Maximin was
derived from a different source, the fear of contempt. Though he
depended on the attachment of the soldiers, who loved him for
virtues like their own, he was conscious that his mean and
barbarian origin, his savage appearance, and his total ignorance
of the arts and institutions of civil life,\footnotemark[8] formed a very
unfavorable contrast with the amiable manners of the unhappy
Alexander. He remembered, that, in his humbler fortune, he had
often waited before the door of the haughty nobles of Rome, and
had been denied admittance by the insolence of their slaves. He
recollected too the friendship of a few who had relieved his
poverty, and assisted his rising hopes. But those who had
spurned, and those who had protected, the Thracian, were guilty
of the same crime, the knowledge of his original obscurity. For
this crime many were put to death; and by the execution of
several of his benefactors, Maximin published, in characters of
blood, the indelible history of his baseness and ingratitude.\footnotemark[9]

\footnotetext[7]{Caligula, the eldest of the four, was only
twenty-five years of age when he ascended the throne; Caracalla
was twenty-three, Commodus nineteen, and Nero no more than
seventeen.}

\footnotetext[8]{It appears that he was totally ignorant of the Greek
language; which, from its universal use in conversation and
letters, was an essential part of every liberal education.}

\footnotetext[9]{Hist. August. p. 141. Herodian, l. vii. p. 237. The
latter of these historians has been most unjustly censured for
sparing the vices of Maximin.}

The dark and sanguinary soul of the tyrant was open to every
suspicion against those among his subjects who were the most
distinguished by their birth or merit. Whenever he was alarmed
with the sound of treason, his cruelty was unbounded and
unrelenting. A conspiracy against his life was either discovered
or imagined, and Magnus, a consular senator, was named as the
principal author of it. Without a witness, without a trial, and
without an opportunity of defence, Magnus, with four thousand of
his supposed accomplices, was put to death. Italy and the whole
empire were infested with innumerable spies and informers. On the
slightest accusation, the first of the Roman nobles, who had
governed provinces, commanded armies, and been adorned with the
consular and triumphal ornaments, were chained on the public
carriages, and hurried away to the emperor’s presence.
Confiscation, exile, or simple death, were esteemed uncommon
instances of his lenity. Some of the unfortunate sufferers he
ordered to be sewed up in the hides of slaughtered animals,
others to be exposed to wild beasts, others again to be beaten to
death with clubs. During the three years of his reign, he
disdained to visit either Rome or Italy. His camp, occasionally
removed from the banks of the Rhine to those of the Danube, was
the seat of his stern despotism, which trampled on every
principle of law and justice, and was supported by the avowed
power of the sword.\footnotemark[10] No man of noble birth, elegant
accomplishments, or knowledge of civil business, was suffered
near his person; and the court of a Roman emperor revived the
idea of those ancient chiefs of slaves and gladiators, whose
savage power had left a deep impression of terror and
detestation.\footnotemark[11]

\footnotetext[10]{The wife of Maximin, by insinuating wise counsels
with female gentleness, sometimes brought back the tyrant to the
way of truth and humanity. See Ammianus Marcellinus, l. xiv. c.
l, where he alludes to the fact which he had more fully related
under the reign of the Gordians. We may collect from the medals,
that Paullina was the name of this benevolent empress; and from
the title of Diva, that she died before Maximin. (Valesius ad
loc. cit. Ammian.) Spanheim de U. et P. N. tom. ii. p. 300. Note:
If we may believe Syrcellus and Zonaras, in was Maximin himself
who ordered her death—G}

\footnotetext[11]{He was compared to Spartacus and Athenio. Hist.
August p. 141.}

As long as the cruelty of Maximin was confined to the illustrious
senators, or even to the bold adventurers, who in the court or
army expose themselves to the caprice of fortune, the body of the
people viewed their sufferings with indifference, or perhaps with
pleasure. But the tyrant’s avarice, stimulated by the insatiate
desires of the soldiers, at length attacked the public property.
Every city of the empire was possessed of an independent revenue,
destined to purchase corn for the multitude, and to supply the
expenses of the games and entertainments. By a single act of
authority, the whole mass of wealth was at once confiscated for
the use of the Imperial treasury. The temples were stripped of
their most valuable offerings of gold and silver, and the statues
of gods, heroes, and emperors, were melted down and coined into
money. These impious orders could not be executed without tumults
and massacres, as in many places the people chose rather to die
in the defence of their altars, than to behold in the midst of
peace their cities exposed to the rapine and cruelty of war. The
soldiers themselves, among whom this sacrilegious plunder was
distributed, received it with a blush; and hardened as they were
in acts of violence, they dreaded the just reproaches of their
friends and relations. Throughout the Roman world a general cry
of indignation was heard, imploring vengeance on the common enemy
of human kind; and at length, by an act of private oppression, a
peaceful and unarmed province was driven into rebellion against
him.\footnotemark[12]

\footnotetext[12]{Herodian, l. vii. p. 238. Zosim. l. i. p. 15.}

The procurator of Africa was a servant worthy of such a master,
who considered the fines and confiscations of the rich as one of
the most fruitful branches of the Imperial revenue. An iniquitous
sentence had been pronounced against some opulent youths of that
country, the execution of which would have stripped them of far
the greater part of their patrimony. In this extremity, a
resolution that must either complete or prevent their ruin, was
dictated by despair. A respite of three days, obtained with
difficulty from the rapacious treasurer, was employed in
collecting from their estates a great number of slaves and
peasants blindly devoted to the commands of their lords, and
armed with the rustic weapons of clubs and axes. The leaders of
the conspiracy, as they were admitted to the audience of the
procurator, stabbed him with the daggers concealed under their
garments, and, by the assistance of their tumultuary train,
seized on the little town of Thysdrus,\footnotemark[13] and erected the
standard of rebellion against the sovereign of the Roman empire.
They rested their hopes on the hatred of mankind against Maximin,
and they judiciously resolved to oppose to that detested tyrant
an emperor whose mild virtues had already acquired the love and
esteem of the Romans, and whose authority over the province would
give weight and stability to the enterprise. Gordianus, their
proconsul, and the object of their choice, refused, with
unfeigned reluctance, the dangerous honor, and begged with tears,
that they would suffer him to terminate in peace a long and
innocent life, without staining his feeble age with civil blood.
Their menaces compelled him to accept the Imperial purple, his
only refuge, indeed, against the jealous cruelty of Maximin;
since, according to the reasoning of tyrants, those who have been
esteemed worthy of the throne deserve death, and those who
deliberate have already rebelled.\footnotemark[14]

\footnotetext[13]{In the fertile territory of Byzacium, one hundred
and fifty miles to the south of Carthage. This city was
decorated, probably by the Gordians, with the title of colony,
and with a fine amphitheatre, which is still in a very perfect
state. See Intinerar. Wesseling, p. 59; and Shaw’s Travels, p.
117.}

\footnotetext[14]{Herodian, l. vii. p. 239. Hist. August. p. 153.}

The family of Gordianus was one of the most illustrious of the
Roman senate. On the father’s side he was descended from the
Gracchi; on his mother’s, from the emperor Trajan. A great estate
enabled him to support the dignity of his birth, and in the
enjoyment of it, he displayed an elegant taste and beneficent
disposition. The palace in Rome, formerly inhabited by the great
Pompey, had been, during several generations, in the possession
of Gordian’s family.\footnotemark[15] It was distinguished by ancient trophies
of naval victories, and decorated with the works of modern
painting. His villa on the road to Præneste was celebrated for
baths of singular beauty and extent, for three stately rooms of a
hundred feet in length, and for a magnificent portico, supported
by two hundred columns of the four most curious and costly sorts
of marble.\footnotemark[16] The public shows exhibited at his expense, and in
which the people were entertained with many hundreds of wild
beasts and gladiators,\footnotemark[17] seem to surpass the fortune of a
subject; and whilst the liberality of other magistrates was
confined to a few solemn festivals at Rome, the magnificence of
Gordian was repeated, when he was ædile, every month in the year,
and extended, during his consulship, to the principal cities of
Italy. He was twice elevated to the last-mentioned dignity, by
Caracalla and by Alexander; for he possessed the uncommon talent
of acquiring the esteem of virtuous princes, without alarming the
jealousy of tyrants. His long life was innocently spent in the
study of letters and the peaceful honors of Rome; and, till he
was named proconsul of Africa by the voice of the senate and the
approbation of Alexander,\footnotemark[18] he appears prudently to have
declined the command of armies and the government of provinces.\footnotemark[181]
As long as that emperor lived, Africa was happy under the
administration of his worthy representative: after the barbarous
Maximin had usurped the throne, Gordianus alleviated the miseries
which he was unable to prevent. When he reluctantly accepted the
purple, he was above fourscore years old; a last and valuable
remains of the happy age of the Antonines, whose virtues he
revived in his own conduct, and celebrated in an elegant poem of
thirty books. With the venerable proconsul, his son, who had
accompanied him into Africa as his lieutenant, was likewise
declared emperor. His manners were less pure, but his character
was equally amiable with that of his father. Twenty-two
acknowledged concubines, and a library of sixty-two thousand
volumes, attested the variety of his inclinations; and from the
productions which he left behind him, it appears that the former
as well as the latter were designed for use rather than for
ostentation.\footnotemark[19] The Roman people acknowledged in the features of
the younger Gordian the resemblance of Scipio Africanus,\footnotemark[191]
recollected with pleasure that his mother was the granddaughter
of Antoninus Pius, and rested the public hope on those latent
virtues which had hitherto, as they fondly imagined, lain
concealed in the luxurious indolence of private life.

\footnotetext[15]{Hist. Aug. p. 152. The celebrated house of Pompey
in carinis was usurped by Marc Antony, and consequently became,
after the Triumvir’s death, a part of the Imperial domain. The
emperor Trajan allowed, and even encouraged, the rich senators to
purchase those magnificent and useless places, (Plin. Panegyric.
c. 50;) and it may seem probable, that, on this occasion,
Pompey’s house came into the possession of Gordian’s
great-grandfather.}

\footnotetext[16]{The Claudian, the Numidian, the Carystian, and the
Synnadian. The colors of Roman marbles have been faintly
described and imperfectly distinguished. It appears, however,
that the Carystian was a sea-green, and that the marble of
Synnada was white mixed with oval spots of purple. See Salmasius
ad Hist. August. p. 164.}

\footnotetext[17]{Hist. August. p. 151, 152. He sometimes gave five
hundred pair of gladiators, never less than one hundred and
fifty. He once gave for the use of the circus one hundred
Sicilian, and as many Cappæcian Cappadecian horses. The animals
designed for hunting were chiefly bears, boars, bulls, stags,
elks, wild asses, \&c. Elephants and lions seem to have been
appropriated to Imperial magnificence.}

\footnotetext[18]{See the original letter, in the Augustan History,
p. 152, which at once shows Alexander’s respect for the authority
of the senate, and his esteem for the proconsul appointed by that
assembly.}

\footnotetext[181]{Herodian expressly says that he had administered
many provinces, lib. vii. 10.—W.}

\footnotetext[19]{By each of his concubines, the younger Gordian left
three or four children. His literary productions, though less
numerous, were by no means contemptible.}

\footnotetext[191]{Not the personal likeness, but the family descent
from the Scipiod.—W.}

As soon as the Gordians had appeased the first tumult of a
popular election, they removed their court to Carthage. They were
received with the acclamations of the Africans, who honored their
virtues, and who, since the visit of Hadrian, had never beheld
the majesty of a Roman emperor. But these vain acclamations
neither strengthened nor confirmed the title of the Gordians.
They were induced by principle, as well as interest, to solicit
the approbation of the senate; and a deputation of the noblest
provincials was sent, without delay, to Rome, to relate and
justify the conduct of their countrymen, who, having long
suffered with patience, were at length resolved to act with
vigor. The letters of the new princes were modest and respectful,
excusing the necessity which had obliged them to accept the
Imperial title; but submitting their election and their fate to
the supreme judgment of the senate.\footnotemark[20]

\footnotetext[20]{Herodian, l. vii. p. 243. Hist. August. p. 144.}

The inclinations of the senate were neither doubtful nor divided.
The birth and noble alliances of the Gordians had intimately
connected them with the most illustrious houses of Rome. Their
fortune had created many dependants in that assembly, their merit
had acquired many friends. Their mild administration opened the
flattering prospect of the restoration, not only of the civil but
even of the republican government. The terror of military
violence, which had first obliged the senate to forget the murder
of Alexander, and to ratify the election of a barbarian peasant,\footnotemark[21]
now produced a contrary effect, and provoked them to assert
the injured rights of freedom and humanity. The hatred of Maximin
towards the senate was declared and implacable; the tamest
submission had not appeased his fury, the most cautious innocence
would not remove his suspicions; and even the care of their own
safety urged them to share the fortune of an enterprise, of which
(if unsuccessful) they were sure to be the first victims. These
considerations, and perhaps others of a more private nature, were
debated in a previous conference of the consuls and the
magistrates. As soon as their resolution was decided, they
convoked in the temple of Castor the whole body of the senate,
according to an ancient form of secrecy,\footnotemark[22] calculated to awaken
their attention, and to conceal their decrees. “Conscript
fathers,” said the consul Syllanus, “the two Gordians, both of
consular dignity, the one your proconsul, the other your
lieutenant, have been declared emperors by the general consent of
Africa. Let us return thanks,” he boldly continued, “to the youth
of Thysdrus; let us return thanks to the faithful people of
Carthage, our generous deliverers from a horrid monster—Why do
you hear me thus coolly, thus timidly? Why do you cast those
anxious looks on each other? Why hesitate? Maximin is a public
enemy! may his enmity soon expire with him, and may we long enjoy
the prudence and felicity of Gordian the father, the valor and
constancy of Gordian the son!”\footnotemark[23] The noble ardor of the consul
revived the languid spirit of the senate. By a unanimous decree,
the election of the Gordians was ratified, Maximin, his son, and
his adherents, were pronounced enemies of their country, and
liberal rewards were offered to whomsoever had the courage and
good fortune to destroy them.

\footnotetext[21]{Quod. tamen patres dum periculosum existimant;
inermes armato esistere approbaverunt. —Aurelius Victor.}

\footnotetext[22]{Even the servants of the house, the scribes, \&c.,
were excluded, and their office was filled by the senators
themselves. We are obliged to the Augustan History. p. 159, for
preserving this curious example of the old discipline of the
commonwealth.}

\footnotetext[23]{This spirited speech, translated from the Augustan
historian, p. 156, seems transcribed by him from the origina
registers of the senate}

During the emperor’s absence, a detachment of the Prætorian
guards remained at Rome, to protect, or rather to command, the
capital. The præfect Vitalianus had signalized his fidelity to
Maximin, by the alacrity with which he had obeyed, and even
prevented the cruel mandates of the tyrant. His death alone could
rescue the authority of the senate, and the lives of the senators
from a state of danger and suspense. Before their resolves had
transpired, a quæstor and some tribunes were commissioned to take
his devoted life. They executed the order with equal boldness and
success; and, with their bloody daggers in their hands, ran
through the streets, proclaiming to the people and the soldiers
the news of the happy revolution. The enthusiasm of liberty was
seconded by the promise of a large donative, in lands and money;
the statues of Maximin were thrown down; the capital of the
empire acknowledged, with transport, the authority of the two
Gordians and the senate;\footnotemark[24] and the example of Rome was followed
by the rest of Italy.

\footnotetext[24]{Herodian, l. vii. p. 244}

A new spirit had arisen in that assembly, whose long patience had
been insulted by wanton despotism and military license. The
senate assumed the reins of government, and, with a calm
intrepidity, prepared to vindicate by arms the cause of freedom.
Among the consular senators recommended by their merit and
services to the favor of the emperor Alexander, it was easy to
select twenty, not unequal to the command of an army, and the
conduct of a war. To these was the defence of Italy intrusted.
Each was appointed to act in his respective department,
authorized to enroll and discipline the Italian youth; and
instructed to fortify the ports and highways, against the
impending invasion of Maximin. A number of deputies, chosen from
the most illustrious of the senatorian and equestrian orders,
were despatched at the same time to the governors of the several
provinces, earnestly conjuring them to fly to the assistance of
their country, and to remind the nations of their ancient ties of
friendship with the Roman senate and people. The general respect
with which these deputies were received, and the zeal of Italy
and the provinces in favor of the senate, sufficiently prove that
the subjects of Maximin were reduced to that uncommon distress,
in which the body of the people has more to fear from oppression
than from resistance. The consciousness of that melancholy truth,
inspires a degree of persevering fury, seldom to be found in
those civil wars which are artificially supported for the benefit
of a few factious and designing leaders.\footnotemark[25]

\footnotetext[25]{Herodian, l. vii. p. 247, l. viii. p. 277. Hist.
August. p 156-158.}

For while the cause of the Gordians was embraced with such
diffusive ardor, the Gordians themselves were no more. The feeble
court of Carthage was alarmed by the rapid approach of
Capelianus, governor of Mauritania, who, with a small band of
veterans, and a fierce host of barbarians, attacked a faithful,
but unwarlike province. The younger Gordian sallied out to meet
the enemy at the head of a few guards, and a numerous
undisciplined multitude, educated in the peaceful luxury of
Carthage. His useless valor served only to procure him an
honorable death on the field of battle. His aged father, whose
reign had not exceeded thirty-six days, put an end to his life on
the first news of the defeat. Carthage, destitute of defence,
opened her gates to the conqueror, and Africa was exposed to the
rapacious cruelty of a slave, obliged to satisfy his unrelenting
master with a large account of blood and treasure.\footnotemark[26]

\footnotetext[26]{Herodian, l. vii. p. 254. Hist. August. p. 150-160.
We may observe, that one month and six days, for the reign of
Gordian, is a just correction of Casaubon and Panvinius, instead
of the absurd reading of one year and six months. See Commentar.
p. 193. Zosimus relates, l. i. p. 17, that the two Gordians
perished by a tempest in the midst of their navigation. A strange
ignorance of history, or a strange abuse of metaphors!}

The fate of the Gordians filled Rome with just but unexpected
terror. The senate, convoked in the temple of Concord, affected
to transact the common business of the day; and seemed to
decline, with trembling anxiety, the consideration of their own
and the public danger. A silent consternation prevailed in the
assembly, till a senator, of the name and family of Trajan,
awakened his brethren from their fatal lethargy. He represented
to them that the choice of cautious, dilatory measures had been
long since out of their power; that Maximin, implacable by
nature, and exasperated by injuries, was advancing towards Italy,
at the head of the military force of the empire; and that their
only remaining alternative was either to meet him bravely in the
field, or tamely to expect the tortures and ignominious death
reserved for unsuccessful rebellion. “We have lost,” continued
he, “two excellent princes; but unless we desert ourselves, the
hopes of the republic have not perished with the Gordians. Many
are the senators whose virtues have deserved, and whose abilities
would sustain, the Imperial dignity. Let us elect two emperors,
one of whom may conduct the war against the public enemy, whilst
his colleague remains at Rome to direct the civil administration.
I cheerfully expose myself to the danger and envy of the
nomination, and give my vote in favor of Maximus and Balbinus.
Ratify my choice, conscript fathers, or appoint in their place,
others more worthy of the empire.” The general apprehension
silenced the whispers of jealousy; the merit of the candidates
was universally acknowledged; and the house resounded with the
sincere acclamations of “Long life and victory to the emperors
Maximus and Balbinus. You are happy in the judgment of the
senate; may the republic be happy under your administration!”\footnotemark[27]

\footnotetext[27]{See the Augustan History, p. 166, from the
registers of the senate; the date is confessedly faulty but the
coincidence of the Apollinatian games enables us to correct it.}

