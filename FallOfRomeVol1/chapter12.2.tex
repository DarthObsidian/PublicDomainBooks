\section{Part \thesection.}
\thispagestyle{simple}

The strength of Aurelian had crushed on every side the enemies of
Rome. After his death they seemed to revive with an increase of
fury and of numbers. They were again vanquished by the active
vigor of Probus, who, in a short reign of about six years,\footnotemark[29]
equalled the fame of ancient heroes, and restored peace and order
to every province of the Roman world. The dangerous frontier of
Rhætia he so firmly secured, that he left it without the
suspicion of an enemy. He broke the wandering power of the
Sarmatian tribes, and by the terror of his arms compelled those
barbarians to relinquish their spoil. The Gothic nation courted
the alliance of so warlike an emperor.\footnotemark[30] He attacked the
Isaurians in their mountains, besieged and took several of their
strongest castles,\footnotemark[31] and flattered himself that he had forever
suppressed a domestic foe, whose independence so deeply wounded
the majesty of the empire. The troubles excited by the usurper
Firmus in the Upper Egypt had never been perfectly appeased, and
the cities of Ptolemais and Coptos, fortified by the alliance of
the Blemmyes, still maintained an obscure rebellion. The
chastisement of those cities, and of their auxiliaries the
savages of the South, is said to have alarmed the court of
Persia,\footnotemark[32] and the Great King sued in vain for the friendship of
Probus. Most of the exploits which distinguished his reign were
achieved by the personal valor and conduct of the emperor,
insomuch that the writer of his life expresses some amazement
how, in so short a time, a single man could be present in so many
distant wars. The remaining actions he intrusted to the care of
his lieutenants, the judicious choice of whom forms no
inconsiderable part of his glory. Carus, Diocletian, Maximian,
Constantius, Galerius, Asclepiodatus, Annibalianus, and a crowd
of other chiefs, who afterwards ascended or supported the throne,
were trained to arms in the severe school of Aurelian and Probus.\footnotemark[33]

\footnotetext[29]{The date and duration of the reign of Probus are
very correctly ascertained by Cardinal Noris in his learned work,
De Epochis Syro-Macedonum, p. 96—105. A passage of Eusebius
connects the second year of Probus with the æras of several of
the Syrian cities.}

\footnotetext[30]{Vopiscus in Hist. August. p. 239.}

\footnotetext[31]{Zosimus (l. i. p. 62—65) tells us a very long and
trifling story of Lycius, the Isaurian robber.}

\footnotetext[32]{Zosim. l. i. p. 65. Vopiscus in Hist. August. p.
239, 240. But it seems incredible that the defeat of the savages
of Æthiopia could affect the Persian monarch.}

\footnotetext[33]{Besides these well-known chiefs, several others are
named by Vopiscus, (Hist. August. p. 241,) whose actions have not
reached knowledge.}

But the most important service which Probus rendered to the
republic was the deliverance of Gaul, and the recovery of seventy
flourishing cities oppressed by the barbarians of Germany, who,
since the death of Aurelian, had ravaged that great province with
impunity.\footnotemark[34] Among the various multitude of those fierce invaders
we may distinguish, with some degree of clearness, three great
armies, or rather nations, successively vanquished by the valor
of Probus. He drove back the Franks into their morasses; a
descriptive circumstance from whence we may infer, that the
confederacy known by the manly appellation of \textit{Free}, already
occupied the flat maritime country, intersected and almost
overflown by the stagnating waters of the Rhine, and that several
tribes of the Frisians and Batavians had acceded to their
alliance. He vanquished the Burgundians, a considerable people of
the Vandalic race.\footnotemark[341] They had wandered in quest of booty from
the banks of the Oder to those of the Seine. They esteemed
themselves sufficiently fortunate to purchase, by the restitution
of all their booty, the permission of an undisturbed retreat.
They attempted to elude that article of the treaty. Their
punishment was immediate and terrible.\footnotemark[35] But of all the invaders
of Gaul, the most formidable were the Lygians, a distant people,
who reigned over a wide domain on the frontiers of Poland and
Silesia.\footnotemark[36] In the Lygian nation, the Arii held the first rank by
their numbers and fierceness. “The Arii” (it is thus that they
are described by the energy of Tacitus) “study to improve by art
and circumstances the innate terrors of their barbarism. Their
shields are black, their bodies are painted black. They choose
for the combat the darkest hour of the night. Their host
advances, covered as it were with a funeral shade;\footnotemark[37] nor do they
often find an enemy capable of sustaining so strange and infernal
an aspect. Of all our senses, the eyes are the first vanquished
in battle.”\footnotemark[38] Yet the arms and discipline of the Romans easily
discomfited these horrid phantoms. The Lygii were defeated in a
general engagement, and Semno, the most renowned of their chiefs,
fell alive into the hands of Probus. That prudent emperor,
unwilling to reduce a brave people to despair, granted them an
honorable capitulation, and permitted them to return in safety to
their native country. But the losses which they suffered in the
march, the battle, and the retreat, broke the power of the
nation: nor is the Lygian name ever repeated in the history
either of Germany or of the empire. The deliverance of Gaul is
reported to have cost the lives of four hundred thousand of the
invaders; a work of labor to the Romans, and of expense to the
emperor, who gave a piece of gold for the head of every
barbarian.\footnotemark[39] But as the fame of warriors is built on the
destruction of human kind, we may naturally suspect that the
sanguinary account was multiplied by the avarice of the soldiers,
and accepted without any very severe examination by the liberal
vanity of Probus.

\footnotetext[34]{See the Cæsars of Julian, and Hist. August. p. 238,
240, 241.}

\footnotetext[341]{It was only under the emperors Diocletian and
Maximian, that the Burgundians, in concert with the Alemanni,
invaded the interior of Gaul; under the reign of Probus, they did
no more than pass the river which separated them from the Roman
Empire: they were repelled. Gatterer presumes that this river was
the Danube; a passage in Zosimus appears to me rather to indicate
the Rhine. Zos. l. i. p. 37, edit H. Etienne, 1581.—G. On the
origin of the Burgundians may be consulted Malte Brun, Geogr vi.
p. 396, (edit. 1831,) who observes that all the remains of the
Burgundian language indicate that they spoke a Gothic
dialect.—M.}

\footnotetext[35]{Zosimus, l. i. p. 62. Hist. August. p. 240. But the
latter supposes the punishment inflicted with the consent of
their kings: if so, it was partial, like the offence.}

\footnotetext[36]{See Cluver. Germania Antiqua, l. iii. Ptolemy
places in their country the city of Calisia, probably Calish in
Silesia. * Note: Luden (vol ii. 501) supposes that these have
been erroneously identified with the Lygii of Tacitus. Perhaps
one fertile source of mistakes has been, that the Romans have
turned appellations into national names. Malte Brun observes of
the Lygii, “that their name appears Sclavonian, and signifies
‘inhabitants of plains;’ they are probably the Lieches of the
middle ages, and the ancestors of the Poles. We find among the
Arii the worship of the two twin gods known in the Sclavian
mythology.” Malte Brun, vol. i. p. 278, (edit. 1831.)—M. But
compare Schafarik, Slawische Alterthumer, 1, p. 406. They were of
German or Keltish descent, occupying the Wendish (or Slavian)
district, Luhy.—M. 1845.}

\footnotetext[37]{Feralis umbra, is the expression of Tacitus: it is
surely a very bold one.}

\footnotetext[38]{Tacit. Germania, (c. 43.)}

\footnotetext[39]{Vopiscus in Hist. August. p. 238}

Since the expedition of Maximin, the Roman generals had confined
their ambition to a defensive war against the nations of Germany,
who perpetually pressed on the frontiers of the empire. The more
daring Probus pursued his Gallic victories, passed the Rhine, and
displayed his invincible eagles on the banks of the Elbe and the
Neckar. He was fully convinced that nothing could reconcile the
minds of the barbarians to peace, unless they experienced, in
their own country, the calamities of war. Germany, exhausted by
the ill success of the last emigration, was astonished by his
presence. Nine of the most considerable princes repaired to his
camp, and fell prostrate at his feet. Such a treaty was humbly
received by the Germans, as it pleased the conqueror to dictate.
He exacted a strict restitution of the effects and captives which
they had carried away from the provinces; and obliged their own
magistrates to punish the more obstinate robbers who presumed to
detain any part of the spoil. A considerable tribute of corn,
cattle, and horses, the only wealth of barbarians, was reserved
for the use of the garrisons which Probus established on the
limits of their territory. He even entertained some thoughts of
compelling the Germans to relinquish the exercise of arms, and to
trust their differences to the justice, their safety to the
power, of Rome. To accomplish these salutary ends, the constant
residence of an Imperial governor, supported by a numerous army,
was indispensably requisite. Probus therefore judged it more
expedient to defer the execution of so great a design; which was
indeed rather of specious than solid utility.\footnotemark[40] Had Germany been
reduced into the state of a province, the Romans, with immense
labor and expense, would have acquired only a more extensive
boundary to defend against the fiercer and more active barbarians
of Scythia.

\footnotetext[40]{Hist. August. 238, 239. Vopiscus quotes a letter
from the emperor to the senate, in which he mentions his design
of reducing Germany into a province.}

Instead of reducing the warlike natives of Germany to the
condition of subjects, Probus contented himself with the humble
expedient of raising a bulwark against their inroads. The country
which now forms the circle of Swabia had been left desert in the
age of Augustus by the emigration of its ancient inhabitants.\footnotemark[41]
The fertility of the soil soon attracted a new colony from the
adjacent provinces of Gaul. Crowds of adventurers, of a roving
temper and of desperate fortunes, occupied the doubtful
possession, and acknowledged, by the payment of tithes, the
majesty of the empire.\footnotemark[42] To protect these new subjects, a line
of frontier garrisons was gradually extended from the Rhine to
the Danube. About the reign of Hadrian, when that mode of defence
began to be practised, these garrisons were connected and covered
by a strong intrenchment of trees and palisades. In the place of
so rude a bulwark, the emperor Probus constructed a stone wall of
a considerable height, and strengthened it by towers at
convenient distances. From the neighborhood of Neustadt and
Ratisbon on the Danube, it stretched across hills, valleys,
rivers, and morasses, as far as Wimpfen on the Neckar, and at
length terminated on the banks of the Rhine, after a winding
course of near two hundred miles.\footnotemark[43] This important barrier,
uniting the two mighty streams that protected the provinces of
Europe, seemed to fill up the vacant space through which the
barbarians, and particularly the Alemanni, could penetrate with
the greatest facility into the heart of the empire. But the
experience of the world, from China to Britain, has exposed the
vain attempt of fortifying any extensive tract of country.\footnotemark[44] An
active enemy, who can select and vary his points of attack, must,
in the end, discover some feeble spot, or some unguarded moment.
The strength, as well as the attention, of the defenders is
divided; and such are the blind effects of terror on the firmest
troops, that a line broken in a single place is almost instantly
deserted. The fate of the wall which Probus erected may confirm
the general observation. Within a few years after his death, it
was overthrown by the Alemanni. Its scattered ruins, universally
ascribed to the power of the Dæmon, now serve only to excite the
wonder of the Swabian peasant.

\footnotetext[41]{Strabo, l. vii. According to Valleius Paterculus,
(ii. 108,) Maroboduus led his Marcomanni into Bohemia; Cluverius
(German. Antiq. iii. 8) proves that it was from Swabia.}

\footnotetext[42]{These settlers, from the payment of tithes, were
denominated Decunates. Tacit. Germania, c. 29}

\footnotetext[43]{See notes de l’Abbé de la Bleterie a la Germanie de
Tacite, p. 183. His account of the wall is chiefly borrowed (as
he says himself) from the Alsatia Illustrata of Schoepflin.}

\footnotetext[44]{See Recherches sur les Chinois et les Egyptiens,
tom. ii. p. 81—102. The anonymous author is well acquainted with
the globe in general, and with Germany in particular: with regard
to the latter, he quotes a work of M. Hanselman; but he seems to
confound the wall of Probus, designed against the Alemanni, with
the fortification of the Mattiaci, constructed in the
neighborhood of Frankfort against the Catti. * Note: De Pauw is
well known to have been the author of this work, as of the
Recherches sur les Americains before quoted. The judgment of M.
Remusat on this writer is in a very different, I fear a juster
tone. Quand au lieu de rechercher, d’examiner, d’etudier, on se
borne, comme cet ecrivain, a juger a prononcer, a decider, sans
connoitre ni l’histoire. ni les langues, sans recourir aux
sources, sans meme se douter de leur existence, on peut en
imposer pendant quelque temps a des lecteurs prevenus ou peu
instruits; mais le mepris qui ne manque guere de succeder a cet
engouement fait bientot justice de ces assertions hazardees, et
elles retombent dans l’oubli d’autant plus promptement, qu’elles
ont ete posees avec plus de confiance. Sur les l angues Tartares,
p. 231.—M.}

Among the useful conditions of peace imposed by Probus on the
vanquished nations of Germany, was the obligation of supplying
the Roman army with sixteen thousand recruits, the bravest and
most robust of their youth. The emperor dispersed them through
all the provinces, and distributed this dangerous reënforcement,
in small bands of fifty or sixty each, among the national troops;
judiciously observing, that the aid which the republic derived
from the barbarians should be felt but not seen.\footnotemark[45] Their aid was
now become necessary. The feeble elegance of Italy and the
internal provinces could no longer support the weight of arms.
The hardy frontiers of the Rhine and Danube still produced minds
and bodies equal to the labors of the camp; but a perpetual
series of wars had gradually diminished their numbers. The
infrequency of marriage, and the ruin of agriculture, affected
the principles of population, and not only destroyed the strength
of the present, but intercepted the hope of future, generations.
The wisdom of Probus embraced a great and beneficial plan of
replenishing the exhausted frontiers, by new colonies of captive
or fugitive barbarians, on whom he bestowed lands, cattle,
instruments of husbandry, and every encouragement that might
engage them to educate a race of soldiers for the service of the
republic. Into Britain, and most probably into Cambridgeshire,\footnotemark[46]
he transported a considerable body of Vandals. The impossibility
of an escape reconciled them to their situation, and in the
subsequent troubles of that island, they approved themselves the
most faithful servants of the state.\footnotemark[47] Great numbers of Franks
and Gepidæ were settled on the banks of the Danube and the Rhine.
A hundred thousand Bastarnæ, expelled from their own country,
cheerfully accepted an establishment in Thrace, and soon imbibed
the manners and sentiments of Roman subjects. \footnotemark[48] But the
expectations of Probus were too often disappointed. The
impatience and idleness of the barbarians could ill brook the
slow labors of agriculture. Their unconquerable love of freedom,
rising against despotism, provoked them into hasty rebellions,
alike fatal to themselves and to the provinces;\footnotemark[49] nor could
these artificial supplies, however repeated by succeeding
emperors, restore the important limit of Gaul and Illyricum to
its ancient and native vigor.

\footnotetext[45]{He distributed about fifty or sixty barbarians to a
Numerus, as it was then called, a corps with whose established
number we are not exactly acquainted.}

\footnotetext[46]{Camden’s Britannia, Introduction, p. 136; but he
speaks from a very doubtful conjecture.}

\footnotetext[47]{Zosimus, l. i. p. 62. According to Vopiscus,
another body of Vandals was less faithful.}

\footnotetext[48]{Footnote 48: Hist. August. p. 240. They were
probably expelled by the Goths. Zosim. l. i. p. 66.}

\footnotetext[49]{Hist. August. p. 240.}

Of all the barbarians who abandoned their new settlements, and
disturbed the public tranquillity, a very small number returned
to their own country. For a short season they might wander in
arms through the empire; but in the end they were surely
destroyed by the power of a warlike emperor. The successful
rashness of a party of Franks was attended, however, with such
memorable consequences, that it ought not to be passed unnoticed.
They had been established by Probus, on the sea-coast of Pontus,
with a view of strengthening the frontier against the inroads of
the Alani. A fleet stationed in one of the harbors of the Euxine
fell into the hands of the Franks; and they resolved, through
unknown seas, to explore their way from the mouth of the Phasis
to that of the Rhine. They easily escaped through the Bosphorus
and the Hellespont, and cruising along the Mediterranean,
indulged their appetite for revenge and plunder by frequent
descents on the unsuspecting shores of Asia, Greece, and Africa.
The opulent city of Syracuse, in whose port the navies of Athens
and Carthage had formerly been sunk, was sacked by a handful of
barbarians, who massacred the greatest part of the trembling
inhabitants. From the island of Sicily the Franks proceeded to
the columns of Hercules, trusted themselves to the ocean, coasted
round Spain and Gaul, and steering their triumphant course
through the British Channel, at length finished their surprising
voyage, by landing in safety on the Batavian or Frisian shores.\footnotemark[50]
The example of their success, instructing their countrymen to
conceive the advantages and to despise the dangers of the sea,
pointed out to their enterprising spirit a new road to wealth and
glory.

\footnotetext[50]{Panegyr. Vet. v. 18. Zosimus, l. i. p. 66.}

Notwithstanding the vigilance and activity of Probus, it was
almost impossible that he could at once contain in obedience
every part of his wide-extended dominions. The barbarians, who
broke their chains, had seized the favorable opportunity of a
domestic war. When the emperor marched to the relief of Gaul, he
devolved the command of the East on Saturninus. That general, a
man of merit and experience, was driven into rebellion by the
absence of his sovereign, the levity of the Alexandrian people,
the pressing instances of his friends, and his own fears; but
from the moment of his elevation, he never entertained a hope of
empire, or even of life. “Alas!” he said, “the republic has lost
a useful servant, and the rashness of an hour has destroyed the
services of many years. You know not,” continued he, “the misery
of sovereign power; a sword is perpetually suspended over our
head. We dread our very guards, we distrust our companions. The
choice of action or of repose is no longer in our disposition,
nor is there any age, or character, or conduct, that can protect
us from the censure of envy. In thus exalting me to the throne,
you have doomed me to a life of cares, and to an untimely fate.
The only consolation which remains is the assurance that I shall
not fall alone.”\footnotemark[51] But as the former part of his prediction was
verified by the victory, so the latter was disappointed by the
clemency, of Probus. That amiable prince attempted even to save
the unhappy Saturninus from the fury of the soldiers. He had more
than once solicited the usurper himself to place some confidence
in the mercy of a sovereign who so highly esteemed his character,
that he had punished, as a malicious informer, the first who
related the improbable news of his disaffection.\footnotemark[52] Saturninus
might, perhaps, have embraced the generous offer, had he not been
restrained by the obstinate distrust of his adherents. Their
guilt was deeper, and their hopes more sanguine, than those of
their experienced leader.

\footnotetext[51]{Vopiscus in Hist. August. p. 245, 246. The
unfortunate orator had studied rhetoric at Carthage; and was
therefore more probably a Moor (Zosim. l. i. p. 60) than a Gaul,
as Vopiscus calls him.}

\footnotetext[52]{Zonaras, l. xii. p. 638.}

The revolt of Saturninus was scarcely extinguished in the East,
before new troubles were excited in the West, by the rebellion of
Bonosus and Proculus, in Gaul. The most distinguished merit of
those two officers was their respective prowess, of the one in
the combats of Bacchus, of the other in those of Venus,\footnotemark[53] yet
neither of them was destitute of courage and capacity, and both
sustained, with honor, the august character which the fear of
punishment had engaged them to assume, till they sunk at length
beneath the superior genius of Probus. He used the victory with
his accustomed moderation, and spared the fortune, as well as the
lives of their innocent families.\footnotemark[54]

\footnotetext[53]{A very surprising instance is recorded of the
prowess of Proculus. He had taken one hundred Sarmatian virgins.
The rest of the story he must relate in his own language: “Ex his
una necte decem inivi; omnes tamen, quod in me erat, mulieres
intra dies quindecim reddidi.” Vopiscus in Hist. August. p. 246.}

\footnotetext[54]{Proculus, who was a native of Albengue, on the
Genoese coast armed two thousand of his own slaves. His riches
were great, but they were acquired by robbery. It was afterwards
a saying of his family, sibi non placere esse vel principes vel
latrones. Vopiscus in Hist. August. p. 247.}

The arms of Probus had now suppressed all the foreign and
domestic enemies of the state. His mild but steady administration
confirmed the re-ëstablishment of the public tranquillity; nor
was there left in the provinces a hostile barbarian, a tyrant, or
even a robber, to revive the memory of past disorders. It was
time that the emperor should revisit Rome, and celebrate his own
glory and the general happiness. The triumph due to the valor of
Probus was conducted with a magnificence suitable to his fortune,
and the people, who had so lately admired the trophies of
Aurelian, gazed with equal pleasure on those of his heroic
successor.\footnotemark[55] We cannot, on this occasion, forget the desperate
courage of about fourscore gladiators, reserved, with near six
hundred others, for the inhuman sports of the amphitheatre.
Disdaining to shed their blood for the amusement of the populace,
they killed their keepers, broke from the place of their
confinement, and filled the streets of Rome with blood and
confusion. After an obstinate resistance, they were overpowered
and cut in pieces by the regular forces; but they obtained at
least an honorable death, and the satisfaction of a just revenge.\footnotemark[56]

\footnotetext[55]{Hist. August. p. 240.}

\footnotetext[56]{Zosim. l. i. p. 66.}

The military discipline which reigned in the camps of Probus was
less cruel than that of Aurelian, but it was equally rigid and
exact. The latter had punished the irregularities of the soldiers
with unrelenting severity, the former prevented them by employing
the legions in constant and useful labors. When Probus commanded
in Egypt, he executed many considerable works for the splendor
and benefit of that rich country. The navigation of the Nile, so
important to Rome itself, was improved; and temples, buildings,
porticos, and palaces, were constructed by the hands of the
soldiers, who acted by turns as architects, as engineers, and as
husbandmen.\footnotemark[57] It was reported of Hannibal, that, in order to
preserve his troops from the dangerous temptations of idleness,
he had obliged them to form large plantations of olive-trees
along the coast of Africa.\footnotemark[58] From a similar principle, Probus
exercised his legions in covering with rich vineyards the hills
of Gaul and Pannonia, and two considerable spots are described,
which were entirely dug and planted by military labor.\footnotemark[59] One of
these, known under the name of Mount Almo, was situated near
Sirmium, the country where Probus was born, for which he ever
retained a partial affection, and whose gratitude he endeavored
to secure, by converting into tillage a large and unhealthy tract
of marshy ground. An army thus employed constituted perhaps the
most useful, as well as the bravest, portion of Roman subjects.

\footnotetext[57]{Hist. August. p. 236.}

\footnotetext[58]{Aurel. Victor. in Prob. But the policy of Hannibal,
unnoticed by any more ancient writer, is irreconcilable with the
history of his life. He left Africa when he was nine years old,
returned to it when he was forty-five, and immediately lost his
army in the decisive battle of Zama. Livilus, xxx. 37.}

\footnotetext[59]{Hist. August. p. 240. Eutrop. ix. 17. Aurel.
Victor. in Prob. Victor Junior. He revoked the prohibition of
Domitian, and granted a general permission of planting vines to
the Gauls, the Britons, and the Pannonians.}

But in the prosecution of a favorite scheme, the best of men,
satisfied with the rectitude of their intentions, are subject to
forget the bounds of moderation; nor did Probus himself
sufficiently consult the patience and disposition of his fierce
legionaries.\footnotemark[60] The dangers of the military profession seem only
to be compensated by a life of pleasure and idleness; but if the
duties of the soldier are incessantly aggravated by the labors of
the peasant, he will at last sink under the intolerable burden,
or shake it off with indignation. The imprudence of Probus is
said to have inflamed the discontent of his troops. More
attentive to the interests of mankind than to those of the army,
he expressed the vain hope, that, by the establishment of
universal peace, he should soon abolish the necessity of a
standing and mercenary force.\footnotemark[61] The unguarded expression proved
fatal to him. In one of the hottest days of summer, as he
severely urged the unwholesome labor of draining the marshes of
Sirmium, the soldiers, impatient of fatigue, on a sudden threw
down their tools, grasped their arms, and broke out into a
furious mutiny. The emperor, conscious of his danger, took refuge
in a lofty tower, constructed for the purpose of surveying the
progress of the work.\footnotemark[62] The tower was instantly forced, and a
thousand swords were plunged at once into the bosom of the
unfortunate Probus. The rage of the troops subsided as soon as it
had been gratified. They then lamented their fatal rashness,
forgot the severity of the emperor whom they had massacred, and
hastened to perpetuate, by an honorable monument, the memory of
his virtues and victories.\footnotemark[63]

\footnotetext[60]{Julian bestows a severe, and indeed excessive,
censure on the rigor of Probus, who, as he thinks, almost
deserved his fate.}

\footnotetext[61]{Vopiscus in Hist. August. p. 241. He lavishes on
this idle hope a large stock of very foolish eloquence.}

\footnotetext[62]{Turris ferrata. It seems to have been a movable
tower, and cased with iron.}

\footnotetext[63]{Probus, et vere probus situs est; Victor omnium
gentium Barbararum; victor etiam tyrannorum.}

When the legions had indulged their grief and repentance for the
death of Probus, their unanimous consent declared Carus, his
Prætorian præfect, the most deserving of the Imperial throne.
Every circumstance that relates to this prince appears of a mixed
and doubtful nature. He gloried in the title of Roman Citizen;
and affected to compare the purity of \textit{his} blood with the
foreign and even barbarous origin of the preceding emperors; yet
the most inquisitive of his contemporaries, very far from
admitting his claim, have variously deduced his own birth, or
that of his parents, from Illyricum, from Gaul, or from Africa.\footnotemark[64]
Though a soldier, he had received a learned education; though
a senator, he was invested with the first dignity of the army;
and in an age when the civil and military professions began to be
irrecoverably separated from each other, they were united in the
person of Carus. Notwithstanding the severe justice which he
exercised against the assassins of Probus, to whose favor and
esteem he was highly indebted, he could not escape the suspicion
of being accessory to a deed from whence he derived the principal
advantage. He enjoyed, at least before his elevation, an
acknowledged character of virtue and abilities;\footnotemark[65] but his
austere temper insensibly degenerated into moroseness and
cruelty; and the imperfect writers of his life almost hesitate
whether they shall not rank him in the number of Roman tyrants.\footnotemark[66]
When Carus assumed the purple, he was about sixty years of
age, and his two sons, Carinus and Numerian had already attained
the season of manhood.\footnotemark[67]

\footnotetext[64]{Yet all this may be conciliated. He was born at
Narbonne in Illyricum, confounded by Eutropius with the more
famous city of that name in Gaul. His father might be an African,
and his mother a noble Roman. Carus himself was educated in the
capital. See Scaliger Animadversion. ad Euseb. Chron. p. 241.}

\footnotetext[65]{Probus had requested of the senate an equestrian
statue and a marble palace, at the public expense, as a just
recompense of the singular merit of Carus. Vopiscus in Hist.
August. p. 249.}

\footnotetext[66]{Vopiscus in Hist. August. p. 242, 249. Julian
excludes the emperor Carus and both his sons from the banquet of
the Cæsars.}

\footnotetext[67]{John Malala, tom. i. p. 401. But the authority of
that ignorant Greek is very slight. He ridiculously derives from
Carus the city of Carrhæ, and the province of Caria, the latter
of which is mentioned by Homer.}

The authority of the senate expired with Probus; nor was the
repentance of the soldiers displayed by the same dutiful regard
for the civil power, which they had testified after the
unfortunate death of Aurelian. The election of Carus was decided
without expecting the approbation of the senate, and the new
emperor contented himself with announcing, in a cold and stately
epistle, that he had ascended the vacant throne.\footnotemark[68] A behavior so
very opposite to that of his amiable predecessor afforded no
favorable presage of the new reign: and the Romans, deprived of
power and freedom, asserted their privilege of licentious
murmurs.\footnotemark[69] The voice of congratulation and flattery was not,
however, silent; and we may still peruse, with pleasure and
contempt, an eclogue, which was composed on the accession of the
emperor Carus. Two shepherds, avoiding the noontide heat, retire
into the cave of Faunus. On a spreading beech they discover some
recent characters. The rural deity had described, in prophetic
verses, the felicity promised to the empire under the reign of so
great a prince. Faunus hails the approach of that hero, who,
receiving on his shoulders the sinking weight of the Roman world,
shall extinguish war and faction, and once again restore the
innocence and security of the golden age.\footnotemark[70]

\footnotetext[68]{Hist. August. p. 249. Carus congratulated the
senate, that one of their own order was made emperor.}

\footnotetext[69]{Hist. August. p. 242.}

\footnotetext[70]{See the first eclogue of Calphurnius. The design of
it is preferes by Fontenelle to that of Virgil’s Pollio. See tom.
iii. p. 148.}

It is more than probable, that these elegant trifles never
reached the ears of a veteran general, who, with the consent of
the legions, was preparing to execute the long-suspended design
of the Persian war. Before his departure for this distant
expedition, Carus conferred on his two sons, Carinus and
Numerian, the title of Cæsar, and investing the former with
almost an equal share of the Imperial power, directed the young
prince first to suppress some troubles which had arisen in Gaul,
and afterwards to fix the seat of his residence at Rome, and to
assume the government of the Western provinces.\footnotemark[71] The safety of
Illyricum was confirmed by a memorable defeat of the Sarmatians;
sixteen thousand of those barbarians remained on the field of
battle, and the number of captives amounted to twenty thousand.
The old emperor, animated with the fame and prospect of victory,
pursued his march, in the midst of winter, through the countries
of Thrace and Asia Minor, and at length, with his younger son,
Numerian, arrived on the confines of the Persian monarchy. There,
encamping on the summit of a lofty mountain, he pointed out to
his troops the opulence and luxury of the enemy whom they were
about to invade.

\footnotetext[71]{Hist. August. p. 353. Eutropius, ix. 18. Pagi.
Annal.}

The successor of Artaxerxes,\footnotemark[711] Varanes, or Bahram, though he
had subdued the Segestans, one of the most warlike nations of
Upper Asia,\footnotemark[72] was alarmed at the approach of the Romans, and
endeavored to retard their progress by a negotiation of peace.\footnotemark[721]

His ambassadors entered the camp about sunset, at the time when
the troops were satisfying their hunger with a frugal repast. The
Persians expressed their desire of being introduced to the
presence of the Roman emperor. They were at length conducted to a
soldier, who was seated on the grass. A piece of stale bacon and
a few hard peas composed his supper. A coarse woollen garment of
purple was the only circumstance that announced his dignity. The
conference was conducted with the same disregard of courtly
elegance. Carus, taking off a cap which he wore to conceal his
baldness, assured the ambassadors, that, unless their master
acknowledged the superiority of Rome, he would speedily render
Persia as naked of trees as his own head was destitute of hair.\footnotemark[73]
Notwithstanding some traces of art and preparation, we may
discover in this scene the manners of Carus, and the severe
simplicity which the martial princes, who succeeded Gallienus,
had already restored in the Roman camps. The ministers of the
Great King trembled and retired.

\footnotetext[711]{Three monarchs had intervened, Sapor, (Shahpour,)
Hormisdas, (Hormooz,) Varanes; Baharam the First.—M.}

\footnotetext[72]{Agathias, l. iv. p. 135. We find one of his sayings
in the Bibliotheque Orientale of M. d’Herbelot. “The definition
of humanity includes all other virtues.”}

\footnotetext[721]{The manner in which his life was saved by the
Chief Pontiff from a conspiracy of his nobles, is as remarkable
as his saying. “By the advice (of the Pontiff) all the nobles
absented themselves from court. The king wandered through his
palace alone. He saw no one; all was silence around. He became
alarmed and distressed. At last the Chief Pontiff appeared, and
bowed his head in apparent misery, but spoke not a word. The king
entreated him to declare what had happened. The virtuous man
boldly related all that had passed, and conjured Bahram, in the
name of his glorious ancestors, to change his conduct and save
himself from destruction. The king was much moved, professed
himself most penitent, and said he was resolved his future life
should prove his sincerity. The overjoyed High Priest, delighted
at this success, made a signal, at which all the nobles and
attendants were in an instant, as if by magic, in their usual
places. The monarch now perceived that only one opinion prevailed
on his past conduct. He repeated therefore to his nobles all he
had said to the Chief Pontiff, and his future reign was unstained
by cruelty or oppression.” Malcolm’s Persia,—M.}

\footnotetext[73]{Synesius tells this story of Carinus; and it is
much more natural to understand it of Carus, than (as Petavius
and Tillemont choose to do) of Probus.}

The threats of Carus were not without effect. He ravaged
Mesopotamia, cut in pieces whatever opposed his passage, made
himself master of the great cities of Seleucia and Ctesiphon,
(which seemed to have surrendered without resistance,) and
carried his victorious arms beyond the Tigris.\footnotemark[74] He had seized
the favorable moment for an invasion. The Persian councils were
distracted by domestic factions, and the greater part of their
forces were detained on the frontiers of India. Rome and the East
received with transport the news of such important advantages.
Flattery and hope painted, in the most lively colors, the fall of
Persia, the conquest of Arabia, the submission of Egypt, and a
lasting deliverance from the inroads of the Scythian nations.\footnotemark[75]
But the reign of Carus was destined to expose the vanity of
predictions. They were scarcely uttered before they were
contradicted by his death; an event attended with such ambiguous
circumstances, that it may be related in a letter from his own
secretary to the præfect of the city. “Carus,” says he, “our
dearest emperor, was confined by sickness to his bed, when a
furious tempest arose in the camp. The darkness which overspread
the sky was so thick, that we could no longer distinguish each
other; and the incessant flashes of lightning took from us the
knowledge of all that passed in the general confusion.
Immediately after the most violent clap of thunder, we heard a
sudden cry that the emperor was dead; and it soon appeared, that
his chamberlains, in a rage of grief, had set fire to the royal
pavilion; a circumstance which gave rise to the report that Carus
was killed by lightning. But, as far as we have been able to
investigate the truth, his death was the natural effect of his
disorder.”\footnotemark[76]

\footnotetext[74]{Vopiscus in Hist. August. p. 250. Eutropius, ix.
18. The two Victors.}

\footnotetext[75]{To the Persian victory of Carus I refer the
dialogue of the Philopatris, which has so long been an object of
dispute among the learned. But to explain and justify my opinion,
would require a dissertation. Note: Niebuhr, in the new edition
of the Byzantine Historians, (vol. x.) has boldly assigned the
Philopatris to the tenth century, and to the reign of Nicephorus
Phocas. An opinion so decisively pronounced by Niebuhr and
favorably received by Hase, the learned editor of Leo Diaconus,
commands respectful consideration. But the whole tone of the work
appears to me altogether inconsistent with any period in which
philosophy did not stand, as it were, on some ground of equality
with Christianity. The doctrine of the Trinity is sarcastically
introduced rather as the strange doctrine of a new religion, than
the established tenet of a faith universally prevalent. The
argument, adopted from Solanus, concerning the formula of the
procession of the Holy Ghost, is utterly worthless, as it is a
mere quotation in the words of the Gospel of St. John, xv. 26.
The only argument of any value is the historic one, from the
allusion to the recent violation of many virgins in the Island of
Crete. But neither is the language of Niebuhr quite accurate, nor
his reference to the Acroases of Theodosius satisfactory. When,
then, could this occurrence take place? Why not in the
devastation of the island by the Gothic pirates, during the reign
of Claudius. Hist. Aug. in Claud. p. 814. edit. Var. Lugd. Bat
1661.—M.}

\footnotetext[76]{Hist. August. p. 250. Yet Eutropius, Festus, Rufus,
the two Victors, Jerome, Sidonius Apollinaris, Syncellus, and
Zonaras, all ascribe the death of Carus to lightning.}

