\section{Part \thesection.}
\thispagestyle{simple}

In the general calamities of mankind, the death of an individual,
however exalted, the ruin of an edifice, however famous, are
passed over with careless inattention. Yet we cannot forget that
the temple of Diana at Ephesus, after having risen with
increasing splendor from seven repeated misfortunes,\footnotemark[128] was
finally burnt by the Goths in their third naval invasion. The
arts of Greece, and the wealth of Asia, had conspired to erect
that sacred and magnificent structure. It was supported by a
hundred and twenty-seven marble columns of the Ionic order. They
were the gifts of devout monarchs, and each was sixty feet high.
The altar was adorned with the masterly sculptures of Praxiteles,
who had, perhaps, selected from the favorite legends of the place
the birth of the divine children of Latona, the concealment of
Apollo after the slaughter of the Cyclops, and the clemency of
Bacchus to the vanquished Amazons.\footnotemark[129] Yet the length of the
temple of Ephesus was only four hundred and twenty-five feet,
about two thirds of the measure of the church of St. Peter’s at
Rome.\footnotemark[130] In the other dimensions, it was still more inferior to
that sublime production of modern architecture. The spreading
arms of a Christian cross require a much greater breadth than the
oblong temples of the Pagans; and the boldest artists of
antiquity would have been startled at the proposal of raising in
the air a dome of the size and proportions of the Pantheon. The
temple of Diana was, however, admired as one of the wonders of
the world. Successive empires, the Persian, the Macedonian, and
the Roman, had revered its sanctity and enriched its splendor.\footnotemark[131]
But the rude savages of the Baltic were destitute of a taste
for the elegant arts, and they despised the ideal terrors of a
foreign superstition.\footnotemark[132]

\footnotetext[128]{Hist. Aug. p. 178. Jornandes, c. 20.}

\footnotetext[129]{Strabo, l. xiv. p. 640. Vitruvius, l. i. c. i.
præfat l vii. Tacit Annal. iii. 61. Plin. Hist. Nat. xxxvi. 14.}

\footnotetext[130]{The length of St. Peter’s is 840 Roman palms; each
palm is very little short of nine English inches. See Greaves’s
Miscellanies vol. i. p. 233; on the Roman Foot. * Note: St.
Paul’s Cathedral is 500 feet. Dallaway on Architecture—M.}

\footnotetext[131]{The policy, however, of the Romans induced them to
abridge the extent of the sanctuary or asylum, which by
successive privileges had spread itself two stadia round the
temple. Strabo, l. xiv. p. 641. Tacit. Annal. iii. 60, \&c.}

\footnotetext[132]{They offered no sacrifices to the Grecian gods.
See Epistol Gregor. Thaumat.}

Another circumstance is related of these invasions, which might
deserve our notice, were it not justly to be suspected as the
fanciful conceit of a recent sophist. We are told that in the
sack of Athens the Goths had collected all the libraries, and
were on the point of setting fire to this funeral pile of Grecian
learning, had not one of their chiefs, of more refined policy
than his brethren, dissuaded them from the design; by the
profound observation, that as long as the Greeks were addicted to
the study of books, they would never apply themselves to the
exercise of arms.\footnotemark[133] The sagacious counsellor (should the truth
of the fact be admitted) reasoned like an ignorant barbarian. In
the most polite and powerful nations, genius of every kind has
displayed itself about the same period; and the age of science
has generally been the age of military virtue and success.

\footnotetext[133]{Zonaras, l. xii. p. 635. Such an anecdote was
perfectly suited to the taste of Montaigne. He makes use of it in
his agreeable Essay on Pedantry, l. i. c. 24.}

IV. The new sovereign of Persia, Artaxerxes and his son Sapor,
had triumphed (as we have already seen) over the house of
Arsaces. Of the many princes of that ancient race. Chosroes, king
of Armenia, had alone preserved both his life and his
independence. He defended himself by the natural strength of his
country; by the perpetual resort of fugitives and malecontents;
by the alliance of the Romans, and above all, by his own courage.

Invincible in arms, during a thirty years’ war, he was at length
assassinated by the emissaries of Sapor, king of Persia. The
patriotic satraps of Armenia, who asserted the freedom and
dignity of the crown, implored the protection of Rome in favor of
Tiridates, the lawful heir. But the son of Chosroes was an
infant, the allies were at a distance, and the Persian monarch
advanced towards the frontier at the head of an irresistible
force. Young Tiridates, the future hope of his country, was saved
by the fidelity of a servant, and Armenia continued above
twenty-seven years a reluctant province of the great monarchy of
Persia.\footnotemark[134] Elated with this easy conquest, and presuming on the
distresses or the degeneracy of the Romans, Sapor obliged the
strong garrisons of Carrhæ and Nisibis\footnotemark[1341] to surrender, and
spread devastation and terror on either side of the Euphrates.

\footnotetext[134]{Moses Chorenensis, l. ii. c. 71, 73, 74. Zonaras,
l. xii. p. 628. The anthentic relation of the Armenian historian
serves to rectify the confused account of the Greek. The latter
talks of the children of Tiridates, who at that time was himself
an infant. (Compare St Martin Memoires sur l’Armenie, i. p.
301.—M.)}

\footnotetext[1341]{Nisibis, according to Persian authors, was taken
by a miracle, the wall fell, in compliance with the prayers of
the army. Malcolm’s Persia, l. 76.—M}

The loss of an important frontier, the ruin of a faithful and
natural ally, and the rapid success of Sapor’s ambition, affected
Rome with a deep sense of the insult as well as of the danger.
Valerian flattered himself, that the vigilance of his lieutenants
would sufficiently provide for the safety of the Rhine and of the
Danube; but he resolved, notwithstanding his advanced age, to
march in person to the defence of the Euphrates.

During his progress through Asia Minor, the naval enterprises of
the Goths were suspended, and the afflicted province enjoyed a
transient and fallacious calm. He passed the Euphrates,
encountered the Persian monarch near the walls of Edessa, was
vanquished, and taken prisoner by Sapor. The particulars of this
great event are darkly and imperfectly represented; yet, by the
glimmering light which is afforded us, we may discover a long
series of imprudence, of error, and of deserved misfortunes on
the side of the Roman emperor. He reposed an implicit confidence
in Macrianus, his Prætorian præfect.\footnotemark[135] That worthless minister
rendered his master formidable only to the oppressed subjects,
and contemptible to the enemies of Rome.\footnotemark[136] By his weak or
wicked counsels, the Imperial army was betrayed into a situation
where valor and military skill were equally unavailing.\footnotemark[137] The
vigorous attempt of the Romans to cut their way through the
Persian host was repulsed with great slaughter;\footnotemark[138] and Sapor,
who encompassed the camp with superior numbers, patiently waited
till the increasing rage of famine and pestilence had insured his
victory. The licentious murmurs of the legions soon accused
Valerian as the cause of their calamities; their seditious
clamors demanded an instant capitulation. An immense sum of gold
was offered to purchase the permission of a disgraceful retreat.
But the Persian, conscious of his superiority, refused the money
with disdain; and detaining the deputies, advanced in order of
battle to the foot of the Roman rampart, and insisted on a
personal conference with the emperor. Valerian was reduced to the
necessity of intrusting his life and dignity to the faith of an
enemy. The interview ended as it was natural to expect. The
emperor was made a prisoner, and his astonished troops laid down
their arms.\footnotemark[139] In such a moment of triumph, the pride and policy
of Sapor prompted him to fill the vacant throne with a successor
entirely dependent on his pleasure. Cyriades, an obscure fugitive
of Antioch, stained with every vice, was chosen to dishonor the
Roman purple; and the will of the Persian victor could not fail
of being ratified by the acclamations, however reluctant, of the
captive army.\footnotemark[140]

\footnotetext[135]{Hist. Aug. p. 191. As Macrianus was an enemy to
the Christians, they charged him with being a magician.}

\footnotetext[136]{Zosimus, l. i. p. 33.}

\footnotetext[137]{Hist. Aug. p. 174.}

\footnotetext[138]{Victor in Cæsar. Eutropius, ix. 7.}

\footnotetext[139]{Zosimus, l. i. p. 33. Zonaras, l. xii. p. 630.
Peter Patricius, in the Excerpta Legat. p. 29.}

\footnotetext[140]{Hist. August. p. 185. The reign of Cyriades
appears in that collection prior to the death of Valerian; but I
have preferred a probable series of events to the doubtful
chronology of a most inaccurate writer}

The Imperial slave was eager to secure the favor of his master by
an act of treason to his native country. He conducted Sapor over
the Euphrates, and, by the way of Chalcis, to the metropolis of
the East. So rapid were the motions of the Persian cavalry, that,
if we may credit a very judicious historian,\footnotemark[141] the city of
Antioch was surprised when the idle multitude was fondly gazing
on the amusements of the theatre. The splendid buildings of
Antioch, private as well as public, were either pillaged or
destroyed; and the numerous inhabitants were put to the sword, or
led away into captivity.\footnotemark[142] The tide of devastation was stopped
for a moment by the resolution of the high priest of Emesa.
Arrayed in his sacerdotal robes, he appeared at the head of a
great body of fanatic peasants, armed only with slings, and
defended his god and his property from the sacrilegious hands of
the followers of Zoroaster.\footnotemark[143] But the ruin of Tarsus, and of
many other cities, furnishes a melancholy proof that, except in
this singular instance, the conquest of Syria and Cilicia
scarcely interrupted the progress of the Persian arms. The
advantages of the narrow passes of Mount Taurus were abandoned,
in which an invader, whose principal force consisted in his
cavalry, would have been engaged in a very unequal combat: and
Sapor was permitted to form the siege of Cæsarea, the capital of
Cappadocia; a city, though of the second rank, which was supposed
to contain four hundred thousand inhabitants. Demosthenes
commanded in the place, not so much by the commission of the
emperor, as in the voluntary defence of his country. For a long
time he deferred its fate; and when at last Cæsarea was betrayed
by the perfidy of a physician, he cut his way through the
Persians, who had been ordered to exert their utmost diligence to
take him alive. This heroic chief escaped the power of a foe who
might either have honored or punished his obstinate valor; but
many thousands of his fellow-citizens were involved in a general
massacre, and Sapor is accused of treating his prisoners with
wanton and unrelenting cruelty.\footnotemark[144] Much should undoubtedly be
allowed for national animosity, much for humbled pride and
impotent revenge; yet, upon the whole, it is certain, that the
same prince, who, in Armenia, had displayed the mild aspect of a
legislator, showed himself to the Romans under the stern features
of a conqueror. He despaired of making any permanent
establishment in the empire, and sought only to leave behind him
a wasted desert, whilst he transported into Persia the people and
the treasures of the provinces.\footnotemark[145]

\footnotetext[141]{The sack of Antioch, anticipated by some
historians, is assigned, by the decisive testimony of Ammianus
Marcellinus, to the reign of Gallienus, xxiii. 5. * Note: Heyne,
in his note on Zosimus, contests this opinion of Gibbon and
observes, that the testimony of Ammianus is in fact by no means
clear, decisive. Gallienus and Valerian reigned together.
Zosimus, in a passage, l. iiii. 32, 8, distinctly places this
event before the capture of Valerian.—M.}

\footnotetext[142]{Zosimus, l. i. p. 35.}

\footnotetext[143]{John Malala, tom. i. p. 391. He corrupts this
probable event by some fabulous circumstances.}

\footnotetext[144]{Zonaras, l. xii. p. 630. Deep valleys were filled
up with the slain. Crowds of prisoners were driven to water like
beasts, and many perished for want of food.}

\footnotetext[145]{Zosimus, l. i. p. 25 asserts, that Sapor, had he
not preferred spoil to conquest, might have remained master of
Asia.}

At the time when the East trembled at the name of Sapor, he
received a present not unworthy of the greatest kings; a long
train of camels, laden with the most rare and valuable
merchandises. The rich offering was accompanied with an epistle,
respectful, but not servile, from Odenathus, one of the noblest
and most opulent senators of Palmyra. “Who is this Odenathus,”
(said the haughty victor, and he commanded that the present
should be cast into the Euphrates,) “that he thus insolently
presumes to write to his lord? If he entertains a hope of
mitigating his punishment, let him fall prostrate before the foot
of our throne, with his hands bound behind his back. Should he
hesitate, swift destruction shall be poured on his head, on his
whole race, and on his country.”\footnotemark[146] The desperate extremity to
which the Palmyrenian was reduced, called into action all the
latent powers of his soul. He met Sapor; but he met him in arms.

Infusing his own spirit into a little army collected from the
villages of Syria\footnotemark[147] and the tents of the desert,\footnotemark[148] he hovered
round the Persian host, harassed their retreat, carried off part
of the treasure, and, what was dearer than any treasure, several
of the women of the great king; who was at last obliged to repass
the Euphrates with some marks of haste and confusion.\footnotemark[149] By this
exploit, Odenathus laid the foundations of his future fame and
fortunes. The majesty of Rome, oppressed by a Persian, was
protected by a Syrian or Arab of Palmyra.

\footnotetext[146]{Peter Patricius in Excerpt. Leg. p. 29.}

\footnotetext[147]{Syrorum agrestium manu. Sextus Rufus, c. 23. Rufus
Victor the Augustan History, (p. 192,) and several inscriptions,
agree in making Odenathus a citizen of Palmyra.}

\footnotetext[148]{He possessed so powerful an interest among the
wandering tribes, that Procopius (Bell. Persic. l. ii. c. 5) and
John Malala, (tom. i. p. 391) style him Prince of the Saracens.}

\footnotetext[149]{Peter Patricius, p. 25.}

The voice of history, which is often little more than the organ
of hatred or flattery, reproaches Sapor with a proud abuse of the
rights of conquest. We are told that Valerian, in chains, but
invested with the Imperial purple, was exposed to the multitude,
a constant spectacle of fallen greatness; and that whenever the
Persian monarch mounted on horseback, he placed his foot on the
neck of a Roman emperor. Notwithstanding all the remonstrances of
his allies, who repeatedly advised him to remember the
vicissitudes of fortune, to dread the returning power of Rome,
and to make his illustrious captive the pledge of peace, not the
object of insult, Sapor still remained inflexible. When Valerian
sunk under the weight of shame and grief, his skin, stuffed with
straw, and formed into the likeness of a human figure, was
preserved for ages in the most celebrated temple of Persia; a
more real monument of triumph, than the fancied trophies of brass
and marble so often erected by Roman vanity.\footnotemark[150] The tale is
moral and pathetic, but the truth\footnotemark[1501] of it may very fairly be
called in question. The letters still extant from the princes of
the East to Sapor are manifest forgeries;\footnotemark[151] nor is it natural
to suppose that a jealous monarch should, even in the person of a
rival, thus publicly degrade the majesty of kings. Whatever
treatment the unfortunate Valerian might experience in Persia, it
is at least certain that the only emperor of Rome who had ever
fallen into the hands of the enemy, languished away his life in
hopeless captivity.

\footnotetext[150]{The Pagan writers lament, the Christian insult,
the misfortunes of Valerian. Their various testimonies are
accurately collected by Tillemont, tom. iii. p. 739, \&c. So
little has been preserved of eastern history before Mahomet, that
the modern Persians are totally ignorant of the victory Sapor, an
event so glorious to their nation. See Bibliotheque Orientale. *
Note: Malcolm appears to write from Persian authorities, i.
76.—M.}

\footnotetext[1501]{Yet Gibbon himself records a speech of the
emperor Galerius, which alludes to the cruelties exercised
against the living, and the indignities to which they exposed the
dead Valerian, vol. ii. ch. 13. Respect for the kingly character
would by no means prevent an eastern monarch from ratifying his
pride and his vengeance on a fallen foe.—M.}

\footnotetext[151]{One of these epistles is from Artavasdes, king of
Armenia; since Armenia was then a province of Persia, the king,
the kingdom, and the epistle must be fictitious.}

The emperor Gallienus, who had long supported with impatience the
censorial severity of his father and colleague, received the
intelligence of his misfortunes with secret pleasure and avowed
indifference. “I knew that my father was a mortal,” said he; “and
since he has acted as it becomes a brave man, I am satisfied.”
Whilst Rome lamented the fate of her sovereign, the savage
coldness of his son was extolled by the servile courtiers as the
perfect firmness of a hero and a stoic.\footnotemark[152] It is difficult to
paint the light, the various, the inconstant character of
Gallienus, which he displayed without constraint, as soon as he
became sole possessor of the empire. In every art that he
attempted, his lively genius enabled him to succeed; and as his
genius was destitute of judgment, he attempted every art, except
the important ones of war and government. He was a master of
several curious, but useless sciences, a ready orator, an elegant
poet,\footnotemark[153] a skilful gardener, an excellent cook, and most
contemptible prince. When the great emergencies of the state
required his presence and attention, he was engaged in
conversation with the philosopher Plotinus,\footnotemark[154] wasting his time
in trifling or licentious pleasures, preparing his initiation to
the Grecian mysteries, or soliciting a place in the Areopagus of
Athens. His profuse magnificence insulted the general poverty;
the solemn ridicule of his triumphs impressed a deeper sense of
the public disgrace.\footnotemark[155] The repeated intelligence of invasions,
defeats, and rebellions, he received with a careless smile; and
singling out, with affected contempt, some particular production
of the lost province, he carelessly asked, whether Rome must be
ruined, unless it was supplied with linen from Egypt, and arras
cloth from Gaul. There were, however, a few short moments in the
life of Gallienus, when, exasperated by some recent injury, he
suddenly appeared the intrepid soldier and the cruel tyrant;
till, satiated with blood, or fatigued by resistance, he
insensibly sunk into the natural mildness and indolence of his
character.\footnotemark[156]

\footnotetext[152]{See his life in the Augustan History.}

\footnotetext[153]{There is still extant a very pretty Epithalamium,
composed by Gallienus for the nuptials of his nephews:—“Ite ait,
O juvenes, pariter sudate medullis Omnibus, inter vos: non
murmura vestra columbæ, Brachia non hederæ, non vincant oscula
conchæ.”}

\footnotetext[154]{He was on the point of giving Plotinus a ruined
city of Campania to try the experiment of realizing Plato’s
Republic. See the Life of Plotinus, by Porphyry, in Fabricius’s
Biblioth. Græc. l. iv.}

\footnotetext[155]{A medal which bears the head of Gallienus has
perplexed the antiquarians by its legend and reverse; the former
Gallienæ Augustæ, the latter Ubique Pax. M. Spanheim supposes
that the coin was struck by some of the enemies of Gallienus, and
was designed as a severe satire on that effeminate prince. But as
the use of irony may seem unworthy of the gravity of the Roman
mint, M. de Vallemont has deduced from a passage of Trebellius
Pollio (Hist. Aug. p. 198) an ingenious and natural solution.
Galliena was first cousin to the emperor. By delivering Africa
from the usurper Celsus, she deserved the title of Augusta. On a
medal in the French king’s collection, we read a similar
inscription of Faustina Augusta round the head of Marcus
Aurelius. With regard to the Ubique Pax, it is easily explained
by the vanity of Gallienus, who seized, perhaps, the occasion of
some momentary calm. See Nouvelles de la Republique des Lettres,
Janvier, 1700, p. 21—34.}

\footnotetext[156]{This singular character has, I believe, been
fairly transmitted to us. The reign of his immediate successor
was short and busy; and the historians who wrote before the
elevation of the family of Constantine could not have the most
remote interest to misrepresent the character of Gallienus.}

At the time when the reins of government were held with so loose
a hand, it is not surprising that a crowd of usurpers should
start up in every province of the empire against the son of
Valerian. It was probably some ingenious fancy, of comparing the
thirty tyrants of Rome with the thirty tyrants of Athens, that
induced the writers of the Augustan History to select that
celebrated number, which has been gradually received into a
popular appellation.\footnotemark[157] But in every light the parallel is idle
and defective. What resemblance can we discover between a council
of thirty persons, the united oppressors of a single city, and an
uncertain list of independent rivals, who rose and fell in
irregular succession through the extent of a vast empire? Nor can
the number of thirty be completed, unless we include in the
account the women and children who were honored with the Imperial
title. The reign of Gallienus, distracted as it was, produced
only nineteen pretenders to the throne: Cyriades, Macrianus,
Balista, Odenathus, and Zenobia, in the East; in Gaul, and the
western provinces, Posthumus, Lollianus, Victorinus, and his
mother Victoria, Marius, and Tetricus; in Illyricum and the
confines of the Danube, Ingenuus, Regillianus, and Aureolus; in
Pontus,\footnotemark[158] Saturninus; in Isauria, Trebellianus; Piso in
Thessaly; Valens in Achaia; Æmilianus in Egypt; and Celsus in
Africa.\footnotemark[1581] To illustrate the obscure monuments of the life and
death of each individual, would prove a laborious task, alike
barren of instruction and of amusement. We may content ourselves
with investigating some general characters, that most strongly
mark the condition of the times, and the manners of the men,
their pretensions, their motives, their fate, and the destructive
consequences of their usurpation.\footnotemark[159]

\footnotetext[157]{Pollio expresses the most minute anxiety to
complete the number. * Note: Compare a dissertation of Manso on
the thirty tyrants at the end of his Leben Constantius des
Grossen. Breslau, 1817.—M.}

\footnotetext[158]{The place of his reign is somewhat doubtful; but
there was a tyrant in Pontus, and we are acquainted with the seat
of all the others.}

\footnotetext[1581]{Captain Smyth, in his “Catalogue of Medals,” p.
307, substitutes two new names to make up the number of nineteen,
for those of Odenathus and Zenobia. He subjoins this list:—1. 2.
3. Of those whose coins Those whose coins Those of whom no are
undoubtedly true. are suspected. coins are known. Posthumus.
Cyriades. Valens. Lælianus, (Lollianus, G.) Ingenuus. Balista
Victorinus Celsus. Saturninus. Marius. Piso Frugi. Trebellianus.
Tetricus. —M. 1815 Macrianus. Quietus. Regalianus (Regillianus,
G.) Alex. Æmilianus. Aureolus. Sulpicius Antoninus}

\footnotetext[159]{Tillemont, tom. iii. p. 1163, reckons them
somewhat differently.}

It is sufficiently known, that the odious appellation of \textit{Tyrant}
was often employed by the ancients to express the illegal seizure
of supreme power, without any reference to the abuse of it.
Several of the pretenders, who raised the standard of rebellion
against the emperor Gallienus, were shining models of virtue, and
almost all possessed a considerable share of vigor and ability.
Their merit had recommended them to the favor of Valerian, and
gradually promoted them to the most important commands of the
empire. The generals, who assumed the title of Augustus, were
either respected by their troops for their able conduct and
severe discipline, or admired for valor and success in war, or
beloved for frankness and generosity. The field of victory was
often the scene of their election; and even the armorer Marius,
the most contemptible of all the candidates for the purple, was
distinguished, however, by intrepid courage, matchless strength,
and blunt honesty.\footnotemark[160] His mean and recent trade cast, indeed, an
air of ridicule on his elevation;\footnotemark[1601] but his birth could not be
more obscure than was that of the greater part of his rivals, who
were born of peasants, and enlisted in the army as private
soldiers. In times of confusion every active genius finds the
place assigned him by nature: in a general state of war military
merit is the road to glory and to greatness. Of the nineteen
tyrants Tetricus only was a senator; Piso alone was a noble. The
blood of Numa, through twenty-eight successive generations, ran
in the veins of Calphurnius Piso,\footnotemark[161] who, by female alliances,
claimed a right of exhibiting, in his house, the images of
Crassus and of the great Pompey.\footnotemark[162] His ancestors had been
repeatedly dignified with all the honors which the commonwealth
could bestow; and of all the ancient families of Rome, the
Calphurnian alone had survived the tyranny of the Cæsars. The
personal qualities of Piso added new lustre to his race. The
usurper Valens, by whose order he was killed, confessed, with
deep remorse, that even an enemy ought to have respected the
sanctity of Piso; and although he died in arms against Gallienus,
the senate, with the emperor’s generous permission, decreed the
triumphal ornaments to the memory of so virtuous a rebel.\footnotemark[163]

\footnotetext[160]{See the speech of Marius in the Augustan History,
p. 197. The accidental identity of names was the only
circumstance that could tempt Pollio to imitate Sallust.}

\footnotetext[1601]{Marius was killed by a soldier, who had formerly
served as a workman in his shop, and who exclaimed, as he struck,
“Behold the sword which thyself hast forged.” Trob vita.—G.}

\footnotetext[161]{“Vos, O Pompilius sanguis!” is Horace’s address to
the Pisos See Art. Poet. v. 292, with Dacier’s and Sanadon’s
notes.}

\footnotetext[162]{Tacit. Annal. xv. 48. Hist. i. 15. In the former
of these passages we may venture to change paterna into materna.
In every generation from Augustus to Alexander Severus, one or
more Pisos appear as consuls. A Piso was deemed worthy of the
throne by Augustus, (Tacit. Annal. i. 13;) a second headed a
formidable conspiracy against Nero; and a third was adopted, and
declared Cæsar, by Galba.}

\footnotetext[163]{Hist. August. p. 195. The senate, in a moment of
enthusiasm, seems to have presumed on the approbation of
Gallienus.}

The lieutenants of Valerian were grateful to the father, whom
they esteemed. They disdained to serve the luxurious indolence of
his unworthy son. The throne of the Roman world was unsupported
by any principle of loyalty; and treason against such a prince
might easily be considered as patriotism to the state. Yet if we
examine with candor the conduct of these usurpers, it will
appear, that they were much oftener driven into rebellion by
their fears, than urged to it by their ambition. They dreaded the
cruel suspicions of Gallienus; they equally dreaded the
capricious violence of their troops. If the dangerous favor of
the army had imprudently declared them deserving of the purple,
they were marked for sure destruction; and even prudence would
counsel them to secure a short enjoyment of empire, and rather to
try the fortune of war than to expect the hand of an executioner.

When the clamor of the soldiers invested the reluctant victims
with the ensigns of sovereign authority, they sometimes mourned
in secret their approaching fate. “You have lost,” said
Saturninus, on the day of his elevation, “you have lost a useful
commander, and you have made a very wretched emperor.”\footnotemark[164]

\footnotetext[164]{Hist. August p. 196.}

The apprehensions of Saturninus were justified by the repeated
experience of revolutions. Of the nineteen tyrants who started up
under the reign of Gallienus, there was not one who enjoyed a
life of peace, or a natural death. As soon as they were invested
with the bloody purple, they inspired their adherents with the
same fears and ambition which had occasioned their own revolt.
Encompassed with domestic conspiracy, military sedition, and
civil war, they trembled on the edge of precipices, in which,
after a longer or shorter term of anxiety, they were inevitably
lost. These precarious monarchs received, however, such honors as
the flattery of their respective armies and provinces could
bestow; but their claim, founded on rebellion, could never obtain
the sanction of law or history. Italy, Rome, and the senate,
constantly adhered to the cause of Gallienus, and he alone was
considered as the sovereign of the empire. That prince
condescended, indeed, to acknowledge the victorious arms of
Odenathus, who deserved the honorable distinction, by the
respectful conduct which he always maintained towards the son of
Valerian. With the general applause of the Romans, and the
consent of Gallienus, the senate conferred the title of Augustus
on the brave Palmyrenian; and seemed to intrust him with the
government of the East, which he already possessed, in so
independent a manner, that, like a private succession, he
bequeathed it to his illustrious widow, Zenobia.\footnotemark[165]

\footnotetext[165]{The association of the brave Palmyrenian was the
most popular act of the whole reign of Gallienus. Hist. August.
p. 180.}

The rapid and perpetual transitions from the cottage to the
throne, and from the throne to the grave, might have amused an
indifferent philosopher; were it possible for a philosopher to
remain indifferent amidst the general calamities of human kind.
The election of these precarious emperors, their power and their
death, were equally destructive to their subjects and adherents.
The price of their fatal elevation was instantly discharged to
the troops by an immense donative, drawn from the bowels of the
exhausted people. However virtuous was their character, however
pure their intentions, they found themselves reduced to the hard
necessity of supporting their usurpation by frequent acts of
rapine and cruelty. When they fell, they involved armies and
provinces in their fall. There is still extant a most savage
mandate from Gallienus to one of his ministers, after the
suppression of Ingenuus, who had assumed the purple in Illyricum.

“It is not enough,” says that soft but inhuman prince, “that you
exterminate such as have appeared in arms; the chance of battle
might have served me as effectually. The male sex of every age
must be extirpated; provided that, in the execution of the
children and old men, you can contrive means to save our
reputation. Let every one die who has dropped an expression, who
has entertained a thought against me, against \textit{me}, the son of
Valerian, the father and brother of so many princes.\footnotemark[166] Remember
that Ingenuus was made emperor: tear, kill, hew in pieces. I
write to you with my own hand, and would inspire you with my own
feelings.”\footnotemark[167] Whilst the public forces of the state were
dissipated in private quarrels, the defenceless provinces lay
exposed to every invader. The bravest usurpers were compelled, by
the perplexity of their situation, to conclude ignominious
treaties with the common enemy, to purchase with oppressive
tributes the neutrality or services of the Barbarians, and to
introduce hostile and independent nations into the heart of the
Roman monarchy.\footnotemark[168]

\footnotetext[166]{Gallienus had given the titles of Cæsar and
Augustus to his son Saloninus, slain at Cologne by the usurper
Posthumus. A second son of Gallienus succeeded to the name and
rank of his elder brother Valerian, the brother of Gallienus, was
also associated to the empire: several other brothers, sisters,
nephews, and nieces of the emperor formed a very numerous royal
family. See Tillemont, tom iii, and M. de Brequigny in the
Memoires de l’Academie, tom xxxii p. 262.}

\footnotetext[167]{Hist. August. p. 188.}

\footnotetext[168]{Regillianus had some bands of Roxolani in his
service; Posthumus a body of Franks. It was, perhaps, in the
character of auxiliaries that the latter introduced themselves
into Spain.}

Such were the barbarians, and such the tyrants, who, under the
reigns of Valerian and Gallienus, dismembered the provinces, and
reduced the empire to the lowest pitch of disgrace and ruin, from
whence it seemed impossible that it should ever emerge. As far as
the barrenness of materials would permit, we have attempted to
trace, with order and perspicuity, the general events of that
calamitous period. There still remain some particular facts; I.
The disorders of Sicily; II. The tumults of Alexandria; and, III.
The rebellion of the Isaurians, which may serve to reflect a
strong light on the horrid picture.

I. Whenever numerous troops of banditti, multiplied by success
and impunity, publicly defy, instead of eluding, the justice of
their country, we may safely infer that the excessive weakness of
the country is felt and abused by the lowest ranks of the
community. The situation of Sicily preserved it from the
Barbarians; nor could the disarmed province have supported a
usurper. The sufferings of that once flourishing and still
fertile island were inflicted by baser hands. A licentious crowd
of slaves and peasants reigned for a while over the plundered
country, and renewed the memory of the servile wars of more
ancient times.\footnotemark[169] Devastations, of which the husbandman was
either the victim or the accomplice, must have ruined the
agriculture of Sicily; and as the principal estates were the
property of the opulent senators of Rome, who often enclosed
within a farm the territory of an old republic, it is not
improbable, that this private injury might affect the capital
more deeply, than all the conquests of the Goths or the Persians.

\footnotetext[169]{The Augustan History, p. 177. See Diodor. Sicul.
l. xxxiv.}

II. The foundation of Alexandria was a noble design, at once
conceived and executed by the son of Philip. The beautiful and
regular form of that great city, second only to Rome itself,
comprehended a circumference of fifteen miles;\footnotemark[170] it was peopled
by three hundred thousand free inhabitants, besides at least an
equal number of slaves.\footnotemark[171] The lucrative trade of Arabia and
India flowed through the port of Alexandria, to the capital and
provinces of the empire.\footnotemark[1711] Idleness was unknown. Some were
employed in blowing of glass, others in weaving of linen, others
again manufacturing the papyrus. Either sex, and every age, was
engaged in the pursuits of industry, nor did even the blind or
the lame want occupations suited to their condition.\footnotemark[172] But the
people of Alexandria, a various mixture of nations, united the
vanity and inconstancy of the Greeks with the superstition and
obstinacy of the Egyptians. The most trifling occasion, a
transient scarcity of flesh or lentils, the neglect of an
accustomed salutation, a mistake of precedency in the public
baths, or even a religious dispute,\footnotemark[173] were at any time
sufficient to kindle a sedition among that vast multitude, whose
resentments were furious and implacable.\footnotemark[174] After the captivity
of Valerian and the insolence of his son had relaxed the
authority of the laws, the Alexandrians abandoned themselves to
the ungoverned rage of their passions, and their unhappy country
was the theatre of a civil war, which continued (with a few short
and suspicious truces) above twelve years.\footnotemark[175] All intercourse
was cut off between the several quarters of the afflicted city,
every street was polluted with blood, every building of strength
converted into a citadel; nor did the tumults subside till a
considerable part of Alexandria was irretrievably ruined. The
spacious and magnificent district of Bruchion,\footnotemark[1751] with its
palaces and musæum, the residence of the kings and philosophers
of Egypt, is described above a century afterwards, as already
reduced to its present state of dreary solitude.\footnotemark[176]

\footnotetext[170]{Plin. Hist. Natur. v. 10.}

\footnotetext[171]{Diodor. Sicul. l. xvii. p. 590, edit. Wesseling.}

\footnotetext[1711]{Berenice, or Myos-Hormos, on the Red Sea,
received the eastern commodities. From thence they were
transported to the Nile, and down the Nile to Alexandria.—M.}

\footnotetext[172]{See a very curious letter of Hadrian, in the
Augustan History, p. 245.}

\footnotetext[173]{Such as the sacrilegious murder of a divine cat.
See Diodor. Sicul. l. i. * Note: The hostility between the Jewish
and Grecian part of the population afterwards between the two
former and the Christian, were unfailing causes of tumult,
sedition, and massacre. In no place were the religious disputes,
after the establishment of Christianity, more frequent or more
sanguinary. See Philo. de Legat. Hist. of Jews, ii. 171, iii.
111, 198. Gibbon, iii c. xxi. viii. c. xlvii.—M.}

\footnotetext[174]{Hist. August. p. 195. This long and terrible
sedition was first occasioned by a dispute between a soldier and
a townsman about a pair of shoes.}

\footnotetext[175]{Dionysius apud. Euses. Hist. Eccles. vii. p. 21.
Ammian xxii. 16.}

\footnotetext[1751]{The Bruchion was a quarter of Alexandria which
extended along the largest of the two ports, and contained many
palaces, inhabited by the Ptolemies. D’Anv. Geogr. Anc. iii.
10.—G.}

\footnotetext[176]{Scaliger. Animadver. ad Euseb. Chron. p. 258.
Three dissertations of M. Bonamy, in the Mem. de l’Academie, tom.
ix.}

he obscure rebellion of Trebellianus, who assumed the purple in
a, a petty province of Asia Minor, was attended with strange and
ble consequences. The pageant of royalty was soon destroyed by an
r of Gallienus; but his followers, despairing of mercy, resolved
ke off their allegiance, not only to the emperor, but to the
, and suddenly returned to the savage manners from which they had
perfectly been reclaimed. Their craggy rocks, a branch of the
xtended Taurus, protected their inaccessible retreat. The tillage
e fertile valleys\footnotemark[177] supplied them with necessaries, and a habit
ine with the luxuries of life. In the heart of the Roman
hy, the Isaurians long continued a nation of wild barbarians.
ding princes, unable to reduce them to obedience, either by arms
icy, were compelled to acknowledge their weakness, by surrounding
stile and independent spot with a strong chain of fortifications,
ich often proved insufficient to restrain the incursions of these
ic foes. The Isaurians, gradually extending their territory to
a-coast, subdued the western and mountainous part of Cilicia,
ly the nest of those daring pirates, against whom the republic
ce been obliged to exert its utmost force, under the conduct of
eat Pompey.\footnotemark[179]

\footnotetext[177]{Strabo, l. xiii. p. 569.}

\footnotetext[178]{Hist. August. p. 197.}

\footnotetext[179]{See Cellarius, Geogr Antiq. tom. ii. p. 137, upon
the limits of Isauria.}

Our habits of thinking so fondly connect the order of the
universe with the fate of man, that this gloomy period of history
has been decorated with inundations, earthquakes, uncommon
meteors, preternatural darkness, and a crowd of prodigies
fictitious or exaggerated.\footnotemark[180] But a long and general famine was
a calamity of a more serious kind. It was the inevitable
consequence of rapine and oppression, which extirpated the
produce of the present and the hope of future harvests. Famine is
almost always followed by epidemical diseases, the effect of
scanty and unwholesome food. Other causes must, however, have
contributed to the furious plague, which, from the year two
hundred and fifty to the year two hundred and sixty-five, raged
without interruption in every province, every city, and almost
every family, of the Roman empire. During some time five thousand
persons died daily in Rome; and many towns, that had escaped the
hands of the Barbarians, were entirely depopulated.\footnotemark[181]

\footnotetext[180]{Hist August p 177.}

\footnotetext[181]{Hist. August. p. 177. Zosimus, l. i. p. 24.
Zonaras, l. xii. p. 623. Euseb. Chronicon. Victor in Epitom.
Victor in Cæsar. Eutropius, ix. 5. Orosius, vii. 21.}

We have the knowledge of a very curious circumstance, of some use
perhaps in the melancholy calculation of human calamities. An
exact register was kept at Alexandria of all the citizens
entitled to receive the distribution of corn. It was found, that
the ancient number of those comprised between the ages of forty
and seventy, had been equal to the whole sum of claimants, from
fourteen to fourscore years of age, who remained alive after the
reign of Gallienus.\footnotemark[182] Applying this authentic fact to the most
correct tables of mortality, it evidently proves, that above half
the people of Alexandria had perished; and could we venture to
extend the analogy to the other provinces, we might suspect, that
war, pestilence, and famine, had consumed, in a few years, the
moiety of the human species.\footnotemark[183]

\footnotetext[182]{Euseb. Hist. Eccles. vii. 21. The fact is taken
from the Letters of Dionysius, who, in the time of those
troubles, was bishop of Alexandria.}

\footnotetext[183]{In a great number of parishes, 11,000 persons were
found between fourteen and eighty; 5365 between forty and
seventy. See Buffon, Histoire Naturelle, tom. ii. p. 590.}

