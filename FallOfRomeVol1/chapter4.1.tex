\chapter{The Cruelty, Follies And Murder Of Commodus.}
\section{Part \thesection.}

The Cruelty, Follies, And Murder Of Commodus — Election Of
Pertinax — His Attempts To Reform The State — His Assassination By The
Prætorian Guards.
\vspace{\onelineskip}

The mildness of Marcus, which the rigid discipline of the Stoics
was unable to eradicate, formed, at the same time, the most
amiable, and the only defective part of his character. His
excellent understanding was often deceived by the unsuspecting
goodness of his heart. Artful men, who study the passions of
princes, and conceal their own, approached his person in the
disguise of philosophic sanctity, and acquired riches and honors
by affecting to despise them.\footnotemark[1] His excessive indulgence to his
brother, \footnotemark[105] his wife, and his son, exceeded the bounds of
private virtue, and became a public injury, by the example and
consequences of their vices.

\footnotetext[1]{See the complaints of Avidius Cassius, Hist. August.
p. 45. These are, it is true, the complaints of faction; but even
faction exaggerates, rather than invents.}

\footnotetext[105]{His brother by adoption, and his colleague, L.
Verus. Marcus Aurelius had no other brother.—W.}

Faustina, the daughter of Pius and the wife of Marcus, has been
as much celebrated for her gallantries as for her beauty. The
grave simplicity of the philosopher was ill calculated to engage
her wanton levity, or to fix that unbounded passion for variety,
which often discovered personal merit in the meanest of mankind.\footnotemark[2]
The Cupid of the ancients was, in general, a very sensual
deity; and the amours of an empress, as they exact on her side
the plainest advances, are seldom susceptible of much sentimental
delicacy. Marcus was the only man in the empire who seemed
ignorant or insensible of the irregularities of Faustina; which,
according to the prejudices of every age, reflected some disgrace
on the injured husband. He promoted several of her lovers to
posts of honor and profit,\footnotemark[3] and during a connection of thirty
years, invariably gave her proofs of the most tender confidence,
and of a respect which ended not with her life. In his
Meditations, he thanks the gods, who had bestowed on him a wife
so faithful, so gentle, and of such a wonderful simplicity of
manners.\footnotemark[4] The obsequious senate, at his earnest request,
declared her a goddess. She was represented in her temples, with
the attributes of Juno, Venus, and Ceres; and it was decreed,
that, on the day of their nuptials, the youth of either sex
should pay their vows before the altar of their chaste patroness.\footnotemark[5]

\footnotetext[2]{Faustinam satis constat apud Cajetam conditiones
sibi et nauticas et gladiatorias, elegisse. Hist. August. p. 30.
Lampridius explains the sort of merit which Faustina chose, and
the conditions which she exacted. Hist. August. p. 102.}

\footnotetext[3]{Hist. August. p. 34.}

\footnotetext[4]{Meditat. l. i. The world has laughed at the
credulity of Marcus but Madam Dacier assures us, (and we may
credit a lady,) that the husband will always be deceived, if the
wife condescends to dissemble.}

\footnotetext[5]{Footnote 5: Dion Cassius, l. lxxi. [c. 31,] p. 1195.
Hist. August. p. 33. Commentaire de Spanheim sur les Cæsars de
Julien, p. 289. The deification of Faustina is the only defect
which Julian’s criticism is able to discover in the
all-accomplished character of Marcus.}

The monstrous vices of the son have cast a shade on the purity of
the father’s virtues. It has been objected to Marcus, that he
sacrificed the happiness of millions to a fond partiality for a
worthless boy; and that he chose a successor in his own family,
rather than in the republic. Nothing however, was neglected by
the anxious father, and by the men of virtue and learning whom he
summoned to his assistance, to expand the narrow mind of young
Commodus, to correct his growing vices, and to render him worthy
of the throne for which he was designed. But the power of
instruction is seldom of much efficacy, except in those happy
dispositions where it is almost superfluous. The distasteful
lesson of a grave philosopher was, in a moment, obliterated by
the whisper of a profligate favorite; and Marcus himself blasted
the fruits of this labored education, by admitting his son, at
the age of fourteen or fifteen, to a full participation of the
Imperial power. He lived but four years afterwards: but he lived
long enough to repent a rash measure, which raised the impetuous
youth above the restraint of reason and authority.

Most of the crimes which disturb the internal peace of society,
are produced by the restraints which the necessary but unequal
laws of property have imposed on the appetites of mankind, by
confining to a few the possession of those objects that are
coveted by many. Of all our passions and appetites, the love of
power is of the most imperious and unsociable nature, since the
pride of one man requires the submission of the multitude. In the
tumult of civil discord, the laws of society lose their force,
and their place is seldom supplied by those of humanity. The
ardor of contention, the pride of victory, the despair of
success, the memory of past injuries, and the fear of future
dangers, all contribute to inflame the mind, and to silence the
voice of pity. From such motives almost every page of history has
been stained with civil blood; but these motives will not account
for the unprovoked cruelties of Commodus, who had nothing to wish
and every thing to enjoy. The beloved son of Marcus succeeded to
his father, amidst the acclamations of the senate and armies;\footnotemark[6]
and when he ascended the throne, the happy youth saw round him
neither competitor to remove, nor enemies to punish. In this
calm, elevated station, it was surely natural that he should
prefer the love of mankind to their detestation, the mild glories
of his five predecessors to the ignominious fate of Nero and
Domitian.

\footnotetext[6]{Commodus was the first \textit{Porphyrogenitus}, (born
since his father’s accession to the throne.) By a new strain of
flattery, the Egyptian medals date by the years of his life; as
if they were synonymous to those of his reign. Tillemont, Hist.
des Empereurs, tom. ii. p. 752.}

Yet Commodus was not, as he has been represented, a tiger born
with an insatiate thirst of human blood, and capable, from his
infancy, of the most inhuman actions.\footnotemark[7] Nature had formed him of
a weak rather than a wicked disposition. His simplicity and
timidity rendered him the slave of his attendants, who gradually
corrupted his mind. His cruelty, which at first obeyed the
dictates of others, degenerated into habit, and at length became
the ruling passion of his soul.\footnotemark[8]

\footnotetext[7]{Hist. August. p. 46.}

\footnotetext[8]{Dion Cassius, l. lxxii. p. 1203.}

Upon the death of his father, Commodus found himself embarrassed
with the command of a great army, and the conduct of a difficult
war against the Quadi and Marcomanni.\footnotemark[9] The servile and
profligate youths whom Marcus had banished, soon regained their
station and influence about the new emperor. They exaggerated the
hardships and dangers of a campaign in the wild countries beyond
the Danube; and they assured the indolent prince that the terror
of his name, and the arms of his lieutenants, would be sufficient
to complete the conquest of the dismayed barbarians, or to impose
such conditions as were more advantageous than any conquest. By a
dexterous application to his sensual appetites, they compared the
tranquillity, the splendor, the refined pleasures of Rome, with
the tumult of a Pannonian camp, which afforded neither leisure
nor materials for luxury.\footnotemark[10] Commodus listened to the pleasing
advice; but whilst he hesitated between his own inclination and
the awe which he still retained for his father’s counsellors, the
summer insensibly elapsed, and his triumphal entry into the
capital was deferred till the autumn. His graceful person,\footnotemark[11]
popular address, and imagined virtues, attracted the public
favor; the honorable peace which he had recently granted to the
barbarians, diffused a universal joy;\footnotemark[12] his impatience to
revisit Rome was fondly ascribed to the love of his country; and
his dissolute course of amusements was faintly condemned in a
prince of nineteen years of age.

\footnotetext[9]{According to Tertullian, (Apolog. c. 25,) he died at
Sirmium. But the situation of Vindobona, or Vienna, where both
the Victors place his death, is better adapted to the operations
of the war against the Marcomanni and Quadi.}

\footnotetext[10]{Herodian, l. i. p. 12.}

\footnotetext[11]{Herodian, l. i. p. 16.}

\footnotetext[12]{This universal joy is well described (from the
medals as well as historians) by Mr. Wotton, Hist. of Rome, p.
192, 193.] During the three first years of his reign, the forms,
and even the spirit, of the old administration, were maintained
by those faithful counsellors, to whom Marcus had recommended his
son, and for whose wisdom and integrity Commodus still
entertained a reluctant esteem. The young prince and his
profligate favorites revelled in all the license of sovereign
power; but his hands were yet unstained with blood; and he had
even displayed a generosity of sentiment, which might perhaps
have ripened into solid virtue.\footnotemark[13] A fatal incident decided his
fluctuating character.}

\footnotetext[13]{Manilius, the confidential secretary of Avidius
Cassius, was discovered after he had lain concealed several
years. The emperor nobly relieved the public anxiety by refusing
to see him, and burning his papers without opening them. Dion
Cassius, l. lxxii. p. 1209.}

One evening, as the emperor was returning to the palace, through
a dark and narrow portico in the amphitheatre,\footnotemark[14] an assassin,
who waited his passage, rushed upon him with a drawn sword,
loudly exclaiming, “\textit{The senate sends you this}.” The menace
prevented the deed; the assassin was seized by the guards, and
immediately revealed the authors of the conspiracy. It had been
formed, not in the state, but within the walls of the palace.
Lucilla, the emperor’s sister, and widow of Lucius Verus,
impatient of the second rank, and jealous of the reigning
empress, had armed the murderer against her brother’s life. She
had not ventured to communicate the black design to her second
husband, Claudius Pompeiarus, a senator of distinguished merit
and unshaken loyalty; but among the crowd of her lovers (for she
imitated the manners of Faustina) she found men of desperate
fortunes and wild ambition, who were prepared to serve her more
violent, as well as her tender passions. The conspirators
experienced the rigor of justice, and the abandoned princess was
punished, first with exile, and afterwards with death.\footnotemark[15]

\footnotetext[14]{See Maffei degli Amphitheatri, p. 126.}

\footnotetext[15]{Dion, l. lxxi. p. 1205 Herodian, l. i. p. 16 Hist.
August p. 46.}

But the words of the assassin sunk deep into the mind of
Commodus, and left an indelible impression of fear and hatred
against the whole body of the senate.\footnotemark[151] Those whom he had
dreaded as importunate ministers, he now suspected as secret
enemies. The Delators, a race of men discouraged, and almost
extinguished, under the former reigns, again became formidable,
as soon as they discovered that the emperor was desirous of
finding disaffection and treason in the senate. That assembly,
whom Marcus had ever considered as the great council of the
nation, was composed of the most distinguished of the Romans; and
distinction of every kind soon became criminal. The possession of
wealth stimulated the diligence of the informers; rigid virtue
implied a tacit censure of the irregularities of Commodus;
important services implied a dangerous superiority of merit; and
the friendship of the father always insured the aversion of the
son. Suspicion was equivalent to proof; trial to condemnation.
The execution of a considerable senator was attended with the
death of all who might lament or revenge his fate; and when
Commodus had once tasted human blood, he became incapable of pity
or remorse.

\footnotetext[151]{The conspirators were senators, even the assassin
himself. Herod. 81.—G.}

Of these innocent victims of tyranny, none died more lamented
than the two brothers of the Quintilian family, Maximus and
Condianus; whose fraternal love has saved their names from
oblivion, and endeared their memory to posterity. Their studies
and their occupations, their pursuits and their pleasures, were
still the same. In the enjoyment of a great estate, they never
admitted the idea of a separate interest: some fragments are now
extant of a treatise which they composed in common;\footnotemark[152] and in
every action of life it was observed that their two bodies were
animated by one soul. The Antonines, who valued their virtues,
and delighted in their union, raised them, in the same year, to
the consulship; and Marcus afterwards intrusted to their joint
care the civil administration of Greece, and a great military
command, in which they obtained a signal victory over the
Germans. The kind cruelty of Commodus united them in death.\footnotemark[16]

\footnotetext[152]{This work was on agriculture, and is often quoted
by later writers. See P. Needham, Proleg. ad Geoponic. Camb.
1704.—W.}

\footnotetext[16]{In a note upon the Augustan History, Casaubon has
collected a number of particulars concerning these celebrated
brothers. See p. 96 of his learned commentary.}

The tyrant’s rage, after having shed the noblest blood of the
senate, at length recoiled on the principal instrument of his
cruelty. Whilst Commodus was immersed in blood and luxury, he
devolved the detail of the public business on Perennis, a servile
and ambitious minister, who had obtained his post by the murder
of his predecessor, but who possessed a considerable share of
vigor and ability. By acts of extortion, and the forfeited
estates of the nobles sacrificed to his avarice, he had
accumulated an immense treasure. The Prætorian guards were under
his immediate command; and his son, who already discovered a
military genius, was at the head of the Illyrian legions.
Perennis aspired to the empire; or what, in the eyes of Commodus,
amounted to the same crime, he was capable of aspiring to it, had
he not been prevented, surprised, and put to death. The fall of a
minister is a very trifling incident in the general history of
the empire; but it was hastened by an extraordinary circumstance,
which proved how much the nerves of discipline were already
relaxed. The legions of Britain, discontented with the
administration of Perennis, formed a deputation of fifteen
hundred select men, with instructions to march to Rome, and lay
their complaints before the emperor. These military petitioners,
by their own determined behaviour, by inflaming the divisions of
the guards, by exaggerating the strength of the British army, and
by alarming the fears of Commodus, exacted and obtained the
minister’s death, as the only redress of their grievances.\footnotemark[17]
This presumption of a distant army, and their discovery of the
weakness of government, was a sure presage of the most dreadful
convulsions.

\footnotetext[17]{Dion, l. lxxii. p. 1210. Herodian, l. i. p. 22.
Hist. August. p. 48. Dion gives a much less odious character of
Perennis, than the other historians. His moderation is almost a
pledge of his veracity. Note: Gibbon praises Dion for the
moderation with which he speaks of Perennis: he follows,
nevertheless, in his own narrative, Herodian and Lampridius. Dion
speaks of Perennis not only with moderation, but with admiration;
he represents him as a great man, virtuous in his life, and
blameless in his death: perhaps he may be suspected of
partiality; but it is singular that Gibbon, having adopted, from
Herodian and Lampridius, their judgment on this minister, follows
Dion’s improbable account of his death. What likelihood, in fact,
that fifteen hundred men should have traversed Gaul and Italy,
and have arrived at Rome without any understanding with the
Prætorians, or without detection or opposition from Perennis, the
Prætorian præfect? Gibbon, foreseeing, perhaps, this difficulty,
has added, that the military deputation inflamed the divisions of
the guards; but Dion says expressly that they did not reach Rome,
but that the emperor went out to meet them: he even reproaches
him for not having opposed them with the guards, who were
superior in number. Herodian relates that Commodus, having
learned, from a soldier, the ambitious designs of Perennis and
his son, caused them to be attacked and massacred by night.—G.
from W. Dion’s narrative is remarkably circumstantial, and his
authority higher than either of the other writers. He hints that
Cleander, a new favorite, had already undermined the influence of
Perennis.—M.}

The negligence of the public administration was betrayed, soon
afterwards, by a new disorder, which arose from the smallest
beginnings. A spirit of desertion began to prevail among the
troops: and the deserters, instead of seeking their safety in
flight or concealment, infested the highways. Maternus, a private
soldier, of a daring boldness above his station, collected these
bands of robbers into a little army, set open the prisons,
invited the slaves to assert their freedom, and plundered with
impunity the rich and defenceless cities of Gaul and Spain. The
governors of the provinces, who had long been the spectators, and
perhaps the partners, of his depredations, were, at length,
roused from their supine indolence by the threatening commands of
the emperor. Maternus found that he was encompassed, and foresaw
that he must be overpowered. A great effort of despair was his
last resource. He ordered his followers to disperse, to pass the
Alps in small parties and various disguises, and to assemble at
Rome, during the licentious tumult of the festival of Cybele.\footnotemark[18]
To murder Commodus, and to ascend the vacant throne, was the
ambition of no vulgar robber. His measures were so ably concerted
that his concealed troops already filled the streets of Rome. The
envy of an accomplice discovered and ruined this singular
enterprise, in a moment when it was ripe for execution.\footnotemark[19]

\footnotetext[18]{During the second Punic war, the Romans imported
from Asia the worship of the mother of the gods. Her festival,
the Megalesia, began on the fourth of April, and lasted six days.
The streets were crowded with mad processions, the theatres with
spectators, and the public tables with unbidden guests. Order and
police were suspended, and pleasure was the only serious business
of the city. See Ovid. de Fastis, l. iv. 189, \&c.}

\footnotetext[19]{Herodian, l. i. p. 23, 23.}

Suspicious princes often promote the last of mankind, from a vain
persuasion, that those who have no dependence, except on their
favor, will have no attachment, except to the person of their
benefactor. Cleander, the successor of Perennis, was a Phrygian
by birth; of a nation over whose stubborn, but servile temper,
blows only could prevail.\footnotemark[20] He had been sent from his native
country to Rome, in the capacity of a slave. As a slave he
entered the Imperial palace, rendered himself useful to his
master’s passions, and rapidly ascended to the most exalted
station which a subject could enjoy. His influence over the mind
of Commodus was much greater than that of his predecessor; for
Cleander was devoid of any ability or virtue which could inspire
the emperor with envy or distrust. Avarice was the reigning
passion of his soul, and the great principle of his
administration. The rank of Consul, of Patrician, of Senator, was
exposed to public sale; and it would have been considered as
disaffection, if any one had refused to purchase these empty and
disgraceful honors with the greatest part of his fortune.\footnotemark[21] In
the lucrative provincial employments, the minister shared with
the governor the spoils of the people. The execution of the laws
was penal and arbitrary. A wealthy criminal might obtain, not
only the reversal of the sentence by which he was justly
condemned, but might likewise inflict whatever punishment he
pleased on the accuser, the witnesses, and the judge.

\footnotetext[20]{Cicero pro Flacco, c. 27.}

\footnotetext[21]{One of these dear-bought promotions occasioned a
current... that Julius Solon was banished into the senate.}

By these means, Cleander, in the space of three years, had
accumulated more wealth than had ever yet been possessed by any
freedman.\footnotemark[22] Commodus was perfectly satisfied with the
magnificent presents which the artful courtier laid at his feet
in the most seasonable moments. To divert the public envy,
Cleander, under the emperor’s name, erected baths, porticos, and
places of exercise, for the use of the people.\footnotemark[23] He flattered
himself that the Romans, dazzled and amused by this apparent
liberality, would be less affected by the bloody scenes which
were daily exhibited; that they would forget the death of
Byrrhus, a senator to whose superior merit the late emperor had
granted one of his daughters; and that they would forgive the
execution of Arrius Antoninus, the last representative of the
name and virtues of the Antonines. The former, with more
integrity than prudence, had attempted to disclose, to his
brother-in-law, the true character of Cleander. An equitable
sentence pronounced by the latter, when proconsul of Asia,
against a worthless creature of the favorite, proved fatal to
him.\footnotemark[24] After the fall of Perennis, the terrors of Commodus had,
for a short time, assumed the appearance of a return to virtue.
He repealed the most odious of his acts; loaded his memory with
the public execration, and ascribed to the pernicious counsels of
that wicked minister all the errors of his inexperienced youth.
But his repentance lasted only thirty days; and, under Cleander’s
tyranny, the administration of Perennis was often regretted.

\footnotetext[22]{Dion (l. lxxii. p. 12, 13) observes, that no
freedman had possessed riches equal to those of Cleander. The
fortune of Pallas amounted, however, to upwards of five and
twenty hundred thousand pounds; Ter millies.}

\footnotetext[23]{Dion, l. lxxii. p. 12, 13. Herodian, l. i. p. 29.
Hist. August. p. 52. These baths were situated near the Porta
Capena. See Nardini Roma Antica, p. 79.}

\footnotetext[24]{Hist. August. p. 79.}

