\chapter{The Constitution In The Age Of The Antonines.}
\section{Part \thesection.}

\textit{Of The Constitution Of The Roman Empire, In The Age Of The
Antonines.}
\vspace{\onelineskip}

The obvious definition of a monarchy seems to be that of a state,
in which a single person, by whatsoever name he may be
distinguished, is intrusted with the execution of the laws, the
management of the revenue, and the command of the army. But,
unless public liberty is protected by intrepid and vigilant
guardians, the authority of so formidable a magistrate will soon
degenerate into despotism. The influence of the clergy, in an age
of superstition, might be usefully employed to assert the rights
of mankind; but so intimate is the connection between the throne
and the altar, that the banner of the church has very seldom been
seen on the side of the people.\footnotemark[101] A martial nobility and
stubborn commons, possessed of arms, tenacious of property, and
collected into constitutional assemblies, form the only balance
capable of preserving a free constitution against enterprises of
an aspiring prince.

\footnotetext[101]{Often enough in the ages of superstition, but not
in the interest of the people or the state, but in that of the
church to which all others were subordinate. Yet the power of the
pope has often been of great service in repressing the excesses
of sovereigns, and in softening manners.—W. The history of the
Italian republics proves the error of Gibbon, and the justice of
his German translator’s comment.—M.}

Every barrier of the Roman constitution had been levelled by the
vast ambition of the dictator; every fence had been extirpated by
the cruel hand of the triumvir. After the victory of Actium, the
fate of the Roman world depended on the will of Octavianus,
surnamed Cæsar, by his uncle’s adoption, and afterwards Augustus,
by the flattery of the senate. The conqueror was at the head of
forty-four veteran legions,\footnotemark[1] conscious of their own strength,
and of the weakness of the constitution, habituated, during
twenty years’ civil war, to every act of blood and violence, and
passionately devoted to the house of Cæsar, from whence alone
they had received, and expected the most lavish rewards. The
provinces, long oppressed by the ministers of the republic,
sighed for the government of a single person, who would be the
master, not the accomplice, of those petty tyrants. The people of
Rome, viewing, with a secret pleasure, the humiliation of the
aristocracy, demanded only bread and public shows; and were
supplied with both by the liberal hand of Augustus. The rich and
polite Italians, who had almost universally embraced the
philosophy of Epicurus, enjoyed the present blessings of ease and
tranquillity, and suffered not the pleasing dream to be
interrupted by the memory of their old tumultuous freedom. With
its power, the senate had lost its dignity; many of the most
noble families were extinct. The republicans of spirit and
ability had perished in the field of battle, or in the
proscription. The door of the assembly had been designedly left
open, for a mixed multitude of more than a thousand persons, who
reflected disgrace upon their rank, instead of deriving honor
from it.\footnotemark[2]

\footnotetext[1]{Orosius, vi. 18. * Note: Dion says twenty-five, (or
three,) (lv. 23.) The united triumvirs had but forty-three.
(Appian. Bell. Civ. iv. 3.) The testimony of Orosius is of little
value when more certain may be had.—W. But all the legions,
doubtless, submitted to Augustus after the battle of Actium.—M.}

\footnotetext[2]{Julius Cæsar introduced soldiers, strangers, and
half-barbarians into the senate (Sueton. in Cæsar. c. 77, 80.)
The abuse became still more scandalous after his death.}

The reformation of the senate was one of the first steps in which
Augustus laid aside the tyrant, and professed himself the father
of his country. He was elected censor; and, in concert with his
faithful Agrippa, he examined the list of the senators, expelled
a few members,\footnotemark[201] whose vices or whose obstinacy required a
public example, persuaded near two hundred to prevent the shame
of an expulsion by a voluntary retreat, raised the qualification
of a senator to about ten thousand pounds, created a sufficient
number of patrician families, and accepted for himself the
honorable title of Prince of the Senate, \footnotemark[202] which had always
been bestowed, by the censors, on the citizen the most eminent
for his honors and services. \footnotemark[3] But whilst he thus restored the
dignity, he destroyed the independence, of the senate. The
principles of a free constitution are irrecoverably lost, when
the legislative power is nominated by the executive.

\footnotetext[201]{Of these Dion and Suetonius knew nothing.—W. Dion
says the contrary.—M.}

\footnotetext[202]{But Augustus, then Octavius, was censor, and in
virtue of that office, even according to the constitution of the
free republic, could reform the senate, expel unworthy members,
name the Princeps Senatus, \&c. That was called, as is well known,
Senatum legere. It was customary, during the free republic, for
the censor to be named Princeps Senatus, (S. Liv. l. xxvii. c.
11, l. xl. c. 51;) and Dion expressly says, that this was done
according to ancient usage. He was empowered by a decree of the
senate to admit a number of families among the patricians.
Finally, the senate was not the legislative power.—W}

\footnotetext[3]{Dion Cassius, l. liii. p. 693. Suetonius in August.
c. 35.}

Before an assembly thus modelled and prepared, Augustus
pronounced a studied oration, which displayed his patriotism, and
disguised his ambition. “He lamented, yet excused, his past
conduct. Filial piety had required at his hands the revenge of
his father’s murder; the humanity of his own nature had sometimes
given way to the stern laws of necessity, and to a forced
connection with two unworthy colleagues: as long as Antony lived,
the republic forbade him to abandon her to a degenerate Roman,
and a barbarian queen. He was now at liberty to satisfy his duty
and his inclination. He solemnly restored the senate and people
to all their ancient rights; and wished only to mingle with the
crowd of his fellow-citizens, and to share the blessings which he
had obtained for his country.”\footnotemark[4]

\footnotetext[4]{Dion (l. liii. p. 698) gives us a prolix and bombast
speech on this great occasion. I have borrowed from Suetonius and
Tacitus the general language of Augustus.}

It would require the pen of Tacitus (if Tacitus had assisted at
this assembly) to describe the various emotions of the senate,
those that were suppressed, and those that were affected. It was
dangerous to trust the sincerity of Augustus; to seem to distrust
it was still more dangerous. The respective advantages of
monarchy and a republic have often divided speculative inquirers;
the present greatness of the Roman state, the corruption of
manners, and the license of the soldiers, supplied new arguments
to the advocates of monarchy; and these general views of
government were again warped by the hopes and fears of each
individual. Amidst this confusion of sentiments, the answer of
the senate was unanimous and decisive. They refused to accept the
resignation of Augustus; they conjured him not to desert the
republic, which he had saved. After a decent resistance, the
crafty tyrant submitted to the orders of the senate; and
consented to receive the government of the provinces, and the
general command of the Roman armies, under the well-known names
of PROCONSUL and IMPERATOR.\footnotemark[5] But he would receive them only for
ten years. Even before the expiration of that period, he hope
that the wounds of civil discord would be completely healed, and
that the republic, restored to its pristine health and vigor,
would no longer require the dangerous interposition of so
extraordinary a magistrate. The memory of this comedy, repeated
several times during the life of Augustus, was preserved to the
last ages of the empire, by the peculiar pomp with which the
perpetual monarchs of Rome always solemnized the tenth years of
their reign.\footnotemark[6]

\footnotetext[5]{Imperator (from which we have derived Emperor)
signified under her republic no more than general, and was
emphatically bestowed by the soldiers, when on the field of
battle they proclaimed their victorious leader worthy of that
title. When the Roman emperors assumed it in that sense, they
placed it after their name, and marked how often they had taken
it.}

\footnotetext[6]{Dion. l. liii. p. 703, \&c.}

Without any violation of the principles of the constitution, the
general of the Roman armies might receive and exercise an
authority almost despotic over the soldiers, the enemies, and the
subjects of the republic. With regard to the soldiers, the
jealousy of freedom had, even from the earliest ages of Rome,
given way to the hopes of conquest, and a just sense of military
discipline. The dictator, or consul, had a right to command the
service of the Roman youth; and to punish an obstinate or
cowardly disobedience by the most severe and ignominious
penalties, by striking the offender out of the list of citizens,
by confiscating his property, and by selling his person into
slavery.\footnotemark[7] The most sacred rights of freedom, confirmed by the
Porcian and Sempronian laws, were suspended by the military
engagement. In his camp the general exercised an absolute power
of life and death; his jurisdiction was not confined by any forms
of trial, or rules of proceeding, and the execution of the
sentence was immediate and without appeal.\footnotemark[8] The choice of the
enemies of Rome was regularly decided by the legislative
authority. The most important resolutions of peace and war were
seriously debated in the senate, and solemnly ratified by the
people. But when the arms of the legions were carried to a great
distance from Italy, the general assumed the liberty of directing
them against whatever people, and in whatever manner, they judged
most advantageous for the public service. It was from the
success, not from the justice, of their enterprises, that they
expected the honors of a triumph. In the use of victory,
especially after they were no longer controlled by the
commissioners of the senate, they exercised the most unbounded
despotism. When Pompey commanded in the East, he rewarded his
soldiers and allies, dethroned princes, divided kingdoms, founded
colonies, and distributed the treasures of Mithridates. On his
return to Rome, he obtained, by a single act of the senate and
people, the universal ratification of all his proceedings.\footnotemark[9] Such
was the power over the soldiers, and over the enemies of Rome,
which was either granted to, or assumed by, the generals of the
republic. They were, at the same time, the governors, or rather
monarchs, of the conquered provinces, united the civil with the
military character, administered justice as well as the finances,
and exercised both the executive and legislative power of the
state.

\footnotetext[7]{Livy Epitom. l. xiv. [c. 27.] Valer. Maxim. vi. 3.}

\footnotetext[8]{See, in the viiith book of Livy, the conduct of
Manlius Torquatus and Papirius Cursor. They violated the laws of
nature and humanity, but they asserted those of military
discipline; and the people, who abhorred the action, was obliged
to respect the principle.}

\footnotetext[9]{By the lavish but unconstrained suffrages of the
people, Pompey had obtained a military command scarcely inferior
to that of Augustus. Among the extraordinary acts of power
executed by the former we may remark the foundation of
twenty-nine cities, and the distribution of three or four
millions sterling to his troops. The ratification of his acts met
with some opposition and delays in the senate See Plutarch,
Appian, Dion Cassius, and the first book of the epistles to
Atticus.}

From what has already been observed in the first chapter of this
work, some notion may be formed of the armies and provinces thus
intrusted to the ruling hand of Augustus. But as it was
impossible that he could personally command the regions of so
many distant frontiers, he was indulged by the senate, as Pompey
had already been, in the permission of devolving the execution of
his great office on a sufficient number of lieutenants. In rank
and authority these officers seemed not inferior to the ancient
proconsuls; but their station was dependent and precarious. They
received and held their commissions at the will of a superior, to
whose auspicious influence the merit of their action was legally
attributed.\footnotemark[10] They were the representatives of the emperor. The
emperor alone was the general of the republic, and his
jurisdiction, civil as well as military, extended over all the
conquests of Rome. It was some satisfaction, however, to the
senate, that he always delegated his power to the members of
their body. The imperial lieutenants were of consular or
prætorian dignity; the legions were commanded by senators, and
the præfecture of Egypt was the only important trust committed to
a Roman knight.

\footnotetext[10]{Under the commonwealth, a triumph could only be
claimed by the general, who was authorized to take the Auspices
in the name of the people. By an exact consequence, drawn from
this principle of policy and religion, the triumph was reserved
to the emperor; and his most successful lieutenants were
satisfied with some marks of distinction, which, under the name
of triumphal honors, were invented in their favor.}

Within six days after Augustus had been compelled to accept so
very liberal a grant, he resolved to gratify the pride of the
senate by an easy sacrifice. He represented to them, that they
had enlarged his powers, even beyond that degree which might be
required by the melancholy condition of the times. They had not
permitted him to refuse the laborious command of the armies and
the frontiers; but he must insist on being allowed to restore the
more peaceful and secure provinces to the mild administration of
the civil magistrate. In the division of the provinces, Augustus
provided for his own power and for the dignity of the republic.
The proconsuls of the senate, particularly those of Asia, Greece,
and Africa, enjoyed a more honorable character than the
lieutenants of the emperor, who commanded in Gaul or Syria. The
former were attended by lictors, the latter by soldiers. 105 A
law was passed, that wherever the emperor was present, his
extraordinary commission should supersede the ordinary
jurisdiction of the governor; a custom was introduced, that the
new conquests belonged to the imperial portion; and it was soon
discovered that the authority of the \textit{Prince}, the favorite
epithet of Augustus, was the same in every part of the empire.

\footnotetext[105]{This distinction is without foundation. The
lieutenants of the emperor, who were called Proprætors, whether
they had been prætors or consuls, were attended by six lictors;
those who had the right of the sword, (of life and death over the
soldiers.—M.) bore the military habit (paludamentum) and the
sword. The provincial governors commissioned by the senate, who,
whether they had been consuls or not, were called Pronconsuls,
had twelve lictors when they had been consuls, and six only when
they had but been prætors. The provinces of Africa and Asia were
only given to ex-consuls. See, on the Organization of the
Provinces, Dion, liii. 12, 16 Strabo, xvii 840.—W}

In return for this imaginary concession, Augustus obtained an
important privilege, which rendered him master of Rome and Italy.
By a dangerous exception to the ancient maxims, he was authorized
to preserve his military command, supported by a numerous body of
guards, even in time of peace, and in the heart of the capital.
His command, indeed, was confined to those citizens who were
engaged in the service by the military oath; but such was the
propensity of the Romans to servitude, that the oath was
voluntarily taken by the magistrates, the senators, and the
equestrian order, till the homage of flattery was insensibly
converted into an annual and solemn protestation of fidelity.

Although Augustus considered a military force as the firmest
foundation, he wisely rejected it, as a very odious instrument of
government. It was more agreeable to his temper, as well as to
his policy, to reign under the venerable names of ancient
magistracy, and artfully to collect, in his own person, all the
scattered rays of civil jurisdiction. With this view, he
permitted the senate to confer upon him, for his life, the powers
of the consular\footnotemark[11] and tribunitian offices,\footnotemark[12] which were, in the
same manner, continued to all his successors. The consuls had
succeeded to the kings of Rome, and represented the dignity of
the state. They superintended the ceremonies of religion, levied
and commanded the legions, gave audience to foreign ambassadors,
and presided in the assemblies both of the senate and people. The
general control of the finances was intrusted to their care; and
though they seldom had leisure to administer justice in person,
they were considered as the supreme guardians of law, equity, and
the public peace. Such was their ordinary jurisdiction; but
whenever the senate empowered the first magistrate to consult the
safety of the commonwealth, he was raised by that decree above
the laws, and exercised, in the defence of liberty, a temporary
despotism.\footnotemark[13] The character of the tribunes was, in every
respect, different from that of the consuls. The appearance of
the former was modest and humble; but their persons were sacred
and inviolable. Their force was suited rather for opposition than
for action. They were instituted to defend the oppressed, to
pardon offences, to arraign the enemies of the people, and, when
they judged it necessary, to stop, by a single word, the whole
machine of government. As long as the republic subsisted, the
dangerous influence, which either the consul or the tribune might
derive from their respective jurisdiction, was diminished by
several important restrictions. Their authority expired with the
year in which they were elected; the former office was divided
between two, the latter among ten persons; and, as both in their
private and public interest they were averse to each other, their
mutual conflicts contributed, for the most part, to strengthen
rather than to destroy the balance of the constitution. 131 But
when the consular and tribunitian powers were united, when they
were vested for life in a single person, when the general of the
army was, at the same time, the minister of the senate and the
representative of the Roman people, it was impossible to resist
the exercise, nor was it easy to define the limits, of his
imperial prerogative.

\footnotetext[11]{Cicero (de Legibus, iii. 3) gives the consular
office the name of egia potestas; and Polybius (l. vi. c. 3)
observes three powers in the Roman constitution. The monarchical
was represented and exercised by the consuls.}

\footnotetext[12]{As the tribunitian power (distinct from the annual
office) was first invented by the dictator Cæsar, (Dion, l. xliv.
p. 384,) we may easily conceive, that it was given as a reward
for having so nobly asserted, by arms, the sacred rights of the
tribunes and people. See his own Commentaries, de Bell. Civil. l.
i.}

\footnotetext[13]{Augustus exercised nine annual consulships without
interruption. He then most artfully refused the magistracy, as
well as the dictatorship, absented himself from Rome, and waited
till the fatal effects of tumult and faction forced the senate to
invest him with a perpetual consulship. Augustus, as well as his
successors, affected, however, to conceal so invidious a title.}

\footnotetext[131]{The note of M. Guizot on the tribunitian power
applies to the French translation rather than to the original.
The former has, maintenir la balance toujours egale, which
implies much more than Gibbon’s general expression. The note
belongs rather to the history of the Republic than that of the
Empire.—M}

To these accumulated honors, the policy of Augustus soon added
the splendid as well as important dignities of supreme pontiff,
and of censor. By the former he acquired the management of the
religion, and by the latter a legal inspection over the manners
and fortunes, of the Roman people. If so many distinct and
independent powers did not exactly unite with each other, the
complaisance of the senate was prepared to supply every
deficiency by the most ample and extraordinary concessions. The
emperors, as the first ministers of the republic, were exempted
from the obligation and penalty of many inconvenient laws: they
were authorized to convoke the senate, to make several motions in
the same day, to recommend candidates for the honors of the
state, to enlarge the bounds of the city, to employ the revenue
at their discretion, to declare peace and war, to ratify
treaties; and by a most comprehensive clause, they were empowered
to execute whatsoever they should judge advantageous to the
empire, and agreeable to the majesty of things private or public,
human of divine.\footnotemark[14]

\footnotetext[14]{See a fragment of a Decree of the Senate,
conferring on the emperor Vespasian all the powers granted to his
predecessors, Augustus, Tiberius, and Claudius. This curious and
important monument is published in Gruter’s Inscriptions, No.
ccxlii. * Note: It is also in the editions of Tacitus by Ryck,
(Annal. p. 420, 421,) and Ernesti, (Excurs. ad lib. iv. 6;) but
this fragment contains so many inconsistencies, both in matter
and form, that its authenticity may be doubted—W.}

When all the various powers of executive government were
committed to the \textit{Imperial magistrate}, the ordinary magistrates
of the commonwealth languished in obscurity, without vigor, and
almost without business. The names and forms of the ancient
administration were preserved by Augustus with the most anxious
care. The usual number of consuls, prætors, and tribunes,\footnotemark[15] were
annually invested with their respective ensigns of office, and
continued to discharge some of their least important functions.
Those honors still attracted the vain ambition of the Romans; and
the emperors themselves, though invested for life with the powers
of the consulship, frequently aspired to the title of that annual
dignity, which they condescended to share with the most
illustrious of their fellow-citizens.\footnotemark[16] In the election of these
magistrates, the people, during the reign of Augustus, were
permitted to expose all the inconveniences of a wild democracy.
That artful prince, instead of discovering the least symptom of
impatience, humbly solicited their suffrages for himself or his
friends, and scrupulously practised all the duties of an ordinary
candidate.\footnotemark[17] But we may venture to ascribe to his councils the
first measure of the succeeding reign, by which the elections
were transferred to the senate.\footnotemark[18] The assemblies of the people
were forever abolished, and the emperors were delivered from a
dangerous multitude, who, without restoring liberty, might have
disturbed, and perhaps endangered, the established government.

\footnotetext[15]{Two consuls were created on the Calends of January;
but in the course of the year others were substituted in their
places, till the annual number seems to have amounted to no less
than twelve. The prætors were usually sixteen or eighteen,
(Lipsius in Excurs. D. ad Tacit. Annal. l. i.) I have not
mentioned the Ædiles or Quæstors Officers of the police or
revenue easily adapt themselves to any form of government. In the
time of Nero, the tribunes legally possessed the right of
intercession, though it might be dangerous to exercise it (Tacit.
Annal. xvi. 26.) In the time of Trajan, it was doubtful whether
the tribuneship was an office or a name, (Plin. Epist. i. 23.)}

\footnotetext[16]{The tyrants themselves were ambitious of the
consulship. The virtuous princes were moderate in the pursuit,
and exact in the discharge of it. Trajan revived the ancient
oath, and swore before the consul’s tribunal that he would
observe the laws, (Plin. Panegyric c. 64.)}

\footnotetext[17]{Quoties Magistratuum Comitiis interesset. Tribus
cum candidatis suis circunbat: supplicabatque more solemni.
Ferebat et ipse suffragium in tribubus, ut unus e populo.
Suetonius in August c. 56.}

\footnotetext[18]{Tum primum Comitia e campo ad patres translata
sunt. Tacit. Annal. i. 15. The word primum seems to allude to
some faint and unsuccessful efforts which were made towards
restoring them to the people. Note: The emperor Caligula made the
attempt: he rest red the Comitia to the people, but, in a short
time, took them away again. Suet. in Caio. c. 16. Dion. lix. 9,
20. Nevertheless, at the time of Dion, they preserved still the
form of the Comitia. Dion. lviii. 20.—W.}

By declaring themselves the protectors of the people, Marius and
Cæsar had subverted the constitution of their country. But as
soon as the senate had been humbled and disarmed, such an
assembly, consisting of five or six hundred persons, was found a
much more tractable and useful instrument of dominion. It was on
the dignity of the senate that Augustus and his successors
founded their new empire; and they affected, on every occasion,
to adopt the language and principles of Patricians. In the
administration of their own powers, they frequently consulted the
great national council, and \textit{seemed} to refer to its decision the
most important concerns of peace and war. Rome, Italy, and the
internal provinces, were subject to the immediate jurisdiction of
the senate. With regard to civil objects, it was the supreme
court of appeal; with regard to criminal matters, a tribunal,
constituted for the trial of all offences that were committed by
men in any public station, or that affected the peace and majesty
of the Roman people. The exercise of the judicial power became
the most frequent and serious occupation of the senate; and the
important causes that were pleaded before them afforded a last
refuge to the spirit of ancient eloquence. As a council of state,
and as a court of justice, the senate possessed very considerable
prerogatives; but in its legislative capacity, in which it was
supposed virtually to represent the people, the rights of
sovereignty were acknowledged to reside in that assembly. Every
power was derived from their authority, every law was ratified by
their sanction. Their regular meetings were held on three stated
days in every month, the Calends, the Nones, and the Ides. The
debates were conducted with decent freedom; and the emperors
themselves, who gloried in the name of senators, sat, voted, and
divided with their equals. To resume, in a few words, the system
of the Imperial government; as it was instituted by Augustus, and
maintained by those princes who understood their own interest and
that of the people, it may be defined an absolute monarchy
disguised by the forms of a commonwealth. The masters of the
Roman world surrounded their throne with darkness, concealed
their irresistible strength, and humbly professed themselves the
accountable ministers of the senate, whose supreme decrees they
dictated and obeyed.\footnotemark[19]

\footnotetext[19]{Dion Cassius (l. liii. p. 703—714) has given a very
loose and partial sketch of the Imperial system. To illustrate
and often to correct him, I have meditated Tacitus, examined
Suetonius, and consulted the following moderns: the Abbé de la
Bleterie, in the Memoires de l’Academie des Inscriptions, tom.
xix. xxi. xxiv. xxv. xxvii. Beaufort Republique Romaine, tom. i.
p. 255—275. The Dissertations of Noodt and Gronovius de lege
Regia, printed at Leyden, in the year 1731 Gravina de Imperio
Romano, p. 479—544 of his Opuscula. Maffei, Verona Illustrata, p.
i. p. 245, \&c.}

The face of the court corresponded with the forms of the
administration. The emperors, if we except those tyrants whose
capricious folly violated every law of nature and decency,
disdained that pomp and ceremony which might offend their
countrymen, but could add nothing to their real power. In all the
offices of life, they affected to confound themselves with their
subjects, and maintained with them an equal intercourse of visits
and entertainments. Their habit, their palace, their table, were
suited only to the rank of an opulent senator. Their family,
however numerous or splendid, was composed entirely of their
domestic slaves and freedmen.\footnotemark[20] Augustus or Trajan would have
blushed at employing the meanest of the Romans in those menial
offices, which, in the household and bedchamber of a limited
monarch, are so eagerly solicited by the proudest nobles of
Britain.

\footnotetext[20]{A weak prince will always be governed by his
domestics. The power of slaves aggravated the shame of the
Romans; and the senate paid court to a Pallas or a Narcissus.
There is a chance that a modern favorite may be a gentleman.}

The deification of the emperors\footnotemark[21] is the only instance in which
they departed from their accustomed prudence and modesty. The
Asiatic Greeks were the first inventors, the successors of
Alexander the first objects, of this servile and impious mode of
adulation.\footnotemark[211] It was easily transferred from the kings to the
governors of Asia; and the Roman magistrates very frequently were
adored as provincial deities, with the pomp of altars and
temples, of festivals and sacrifices.\footnotemark[22] It was natural that the
emperors should not refuse what the proconsuls had accepted; and
the divine honors which both the one and the other received from
the provinces, attested rather the despotism than the servitude
of Rome. But the conquerors soon imitated the vanquished nations
in the arts of flattery; and the imperious spirit of the first
Cæsar too easily consented to assume, during his lifetime, a
place among the tutelar deities of Rome. The milder temper of his
successor declined so dangerous an ambition, which was never
afterwards revived, except by the madness of Caligula and
Domitian. Augustus permitted indeed some of the provincial cities
to erect temples to his honor, on condition that they should
associate the worship of Rome with that of the sovereign; he
tolerated private superstition, of which he might be the object;\footnotemark[23]
but he contented himself with being revered by the senate and
the people in his human character, and wisely left to his
successor the care of his public deification. A regular custom
was introduced, that on the decease of every emperor who had
neither lived nor died like a tyrant, the senate by a solemn
decree should place him in the number of the gods: and the
ceremonies of his apotheosis were blended with those of his
funeral.\footnotemark[231] This legal, and, as it should seem, injudicious
profanation, so abhorrent to our stricter principles, was
received with a very faint murmur,\footnotemark[24] by the easy nature of
Polytheism; but it was received as an institution, not of
religion, but of policy. We should disgrace the virtues of the
Antonines by comparing them with the vices of Hercules or
Jupiter. Even the characters of Cæsar or Augustus were far
superior to those of the popular deities. But it was the
misfortune of the former to live in an enlightened age, and their
actions were too faithfully recorded to admit of such a mixture
of fable and mystery, as the devotion of the vulgar requires. As
soon as their divinity was established by law, it sunk into
oblivion, without contributing either to their own fame, or to
the dignity of succeeding princes.

\footnotetext[21]{See a treatise of Vandale de Consecratione
Principium. It would be easier for me to copy, than it has been
to verify, the quotations of that learned Dutchman.}

\footnotetext[211]{This is inaccurate. The successors of Alexander
were not the first deified sovereigns; the Egyptians had deified
and worshipped many of their kings; the Olympus of the Greeks was
peopled with divinities who had reigned on earth; finally,
Romulus himself had received the honors of an apotheosis (Tit.
Liv. i. 16) a long time before Alexander and his successors. It
is also an inaccuracy to confound the honors offered in the
provinces to the Roman governors, by temples and altars, with the
true apotheosis of the emperors; it was not a religious worship,
for it had neither priests nor sacrifices. Augustus was severely
blamed for having permitted himself to be worshipped as a god in
the provinces, (Tac. Ann. i. 10: ) he would not have incurred
that blame if he had only done what the governors were accustomed
to do.—G. from W. M. Guizot has been guilty of a still greater
inaccuracy in confounding the deification of the living with the
apotheosis of the dead emperors. The nature of the king-worship
of Egypt is still very obscure; the hero-worship of the Greeks
very different from the adoration of the “præsens numen” in the
reigning sovereign.—M.}

\footnotetext[22]{See a dissertation of the Abbé Mongault in the
first volume of the Academy of Inscriptions.}

\footnotetext[23]{Jurandasque tuum per nomen ponimus aras, says
Horace to the emperor himself, and Horace was well acquainted
with the court of Augustus. Note: The good princes were not those
who alone obtained the honors of an apotheosis: it was conferred
on many tyrants. See an excellent treatise of Schæpflin, de
Consecratione Imperatorum Romanorum, in his Commentationes
historicæ et criticæ. Bale, 1741, p. 184.—W.}

\footnotetext[231]{The curious satire in the works of Seneca, is the
strongest remonstrance of profaned religion.—M.}

\footnotetext[24]{See Cicero in Philippic. i. 6. Julian in Cæsaribus.
Inque Deum templis jurabit Roma per umbras, is the indignant
expression of Lucan; but it is a patriotic rather than a devout
indignation.}

In the consideration of the Imperial government, we have
frequently mentioned the artful founder, under his well-known
title of Augustus, which was not, however, conferred upon him
till the edifice was almost completed. The obscure name of
Octavianus he derived from a mean family, in the little town of
Aricia.\footnotemark[241] It was stained with the blood of the proscription;
and he was desirous, had it been possible, to erase all memory of
his former life. The illustrious surname of Cæsar he had assumed,
as the adopted son of the dictator: but he had too much good
sense, either to hope to be confounded, or to wish to be compared
with that extraordinary man. It was proposed in the senate to
dignify their minister with a new appellation; and after a
serious discussion, that of Augustus was chosen, among several
others, as being the most expressive of the character of peace
and sanctity, which he uniformly affected.\footnotemark[25] \textit{Augustus} was
therefore a personal, \textit{Cæsar} a family distinction. The former
should naturally have expired with the prince on whom it was
bestowed; and however the latter was diffused by adoption and
female alliance, Nero was the last prince who could allege any
hereditary claim to the honors of the Julian line. But, at the
time of his death, the practice of a century had inseparably
connected those appellations with the Imperial dignity, and they
have been preserved by a long succession of emperors, Romans,
Greeks, Franks, and Germans, from the fall of the republic to the
present time. A distinction was, however, soon introduced. The
sacred title of Augustus was always reserved for the monarch,
whilst the name of Cæsar was more freely communicated to his
relations; and, from the reign of Hadrian, at least, was
appropriated to the second person in the state, who was
considered as the presumptive heir of the empire.\footnotemark[251]

\footnotetext[241]{Octavius was not of an obscure family, but of a
considerable one of the equestrian order. His father, C.
Octavius, who possessed great property, had been prætor, governor
of Macedonia, adorned with the title of Imperator, and was on the
point of becoming consul when he died. His mother Attia, was
daughter of M. Attius Balbus, who had also been prætor. M.
Anthony reproached Octavius with having been born in Aricia,
which, nevertheless, was a considerable municipal city: he was
vigorously refuted by Cicero. Philip. iii. c. 6.—W. Gibbon
probably meant that the family had but recently emerged into
notice.—M.}

\footnotetext[25]{Dion. Cassius, l. liii. p. 710, with the curious
Annotations of Reimar.}

\footnotetext[251]{The princes who by their birth or their adoption
belonged to the family of the Cæsars, took the name of Cæsar.
After the death of Nero, this name designated the Imperial
dignity itself, and afterwards the appointed successor. The time
at which it was employed in the latter sense, cannot be fixed
with certainty. Bach (Hist. Jurisprud. Rom. 304) affirms from
Tacitus, H. i. 15, and Suetonius, Galba, 17, that Galba conferred
on Piso Lucinianus the title of Cæsar, and from that time the
term had this meaning: but these two historians simply say that
he appointed Piso his successor, and do not mention the word
Cæsar. Aurelius Victor (in Traj. 348, ed. Artzen) says that
Hadrian first received this title on his adoption; but as the
adoption of Hadrian is still doubtful, and besides this, as
Trajan, on his death-bed, was not likely to have created a new
title for his successor, it is more probable that Ælius Verus was
the first who was called Cæsar when adopted by Hadrian. Spart. in
Ælio Vero, 102.—W.}

