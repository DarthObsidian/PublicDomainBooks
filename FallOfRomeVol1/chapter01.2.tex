\section{Part \thesection.}
\thispagestyle{simple}

It was an ancient tradition, that when the Capitol was founded by
\markboth{Chapter \thechapter.}{Part \thesection.}
one of the Roman kings, the god Terminus (who presided over
boundaries, and was represented, according to the fashion of that
age, by a large stone) alone, among all the inferior deities,
refused to yield his place to Jupiter himself. A favorable
inference was drawn from his obstinacy, which was interpreted by
the augurs as a sure presage that the boundaries of the Roman
power would never recede.\footnotemark[22] During many ages, the prediction, as
it is usual, contributed to its own accomplishment. But though
Terminus had resisted the Majesty of Jupiter, he submitted to the
authority of the emperor Hadrian.\footnotemark[23] The resignation of all the
eastern conquests of Trajan was the first measure of his reign.
He restored to the Parthians the election of an independent
sovereign; withdrew the Roman garrisons from the provinces of
Armenia, Mesopotamia, and Assyria; and, in compliance with the
precept of Augustus, once more established the Euphrates as the
frontier of the empire.\footnotemark[24] Censure, which arraigns the public
actions and the private motives of princes, has ascribed to envy,
a conduct which might be attributed to the prudence and
moderation of Hadrian. The various character of that emperor,
capable, by turns, of the meanest and the most generous
sentiments, may afford some color to the suspicion. It was,
however, scarcely in his power to place the superiority of his
predecessor in a more conspicuous light, than by thus confessing
himself unequal to the task of defending the conquests of Trajan.

\footnotetext[22]{Ovid. Fast. l. ii. ver. 667. See Livy, and
Dionysius of Halicarnassus, under the reign of Tarquin.}

\footnotetext[23]{St. Augustin is highly delighted with the proof of
the weakness of Terminus, and the vanity of the Augurs. See De
Civitate Dei, iv. 29. * Note: The turn of Gibbon’s sentence is
Augustin’s: “Plus Hadrianum regem hominum, quam regem Deorum
timuisse videatur.”—M}

\footnotetext[24]{See the Augustan History, p. 5, Jerome’s Chronicle,
and all the Epitomizers. It is somewhat surprising, that this
memorable event should be omitted by Dion, or rather by
Xiphilin.}

The martial and ambitious spirit of Trajan formed a very singular
contrast with the moderation of his successor. The restless
activity of Hadrian was not less remarkable when compared with
the gentle repose of Antoninus Pius. The life of the former was
almost a perpetual journey; and as he possessed the various
talents of the soldier, the statesman, and the scholar, he
gratified his curiosity in the discharge of his duty.

Careless of the difference of seasons and of climates, he marched
on foot, and bare-headed, over the snows of Caledonia, and the
sultry plains of the Upper Egypt; nor was there a province of the
empire which, in the course of his reign, was not honored with
the presence of the monarch.\footnotemark[25] But the tranquil life of
Antoninus Pius was spent in the bosom of Italy, and, during the
twenty-three years that he directed the public administration,
the longest journeys of that amiable prince extended no farther
than from his palace in Rome to the retirement of his Lanuvian
villa.\footnotemark[26]

\footnotetext[25]{Dion, l. lxix. p. 1158. Hist. August. p. 5, 8. If
all our historians were lost, medals, inscriptions, and other
monuments, would be sufficient to record the onecolumonecolumnf Hadrian.
Note: The journeys of Hadrian are traced in a note on Solvet’s
translation of Hegewisch, Essai sur l’Epoque de Histoire Romaine
la plus heureuse pour Genre Humain Paris, 1834, p. 123.—M.}

\footnotetext[26]{See the Augustan History and the Epitomes.}

Notwithstanding this difference in their personal conduct, the
general system of Augustus was equally adopted and uniformly
pursued by Hadrian and by the two Antonines. They persisted in
the design of maintaining the dignity of the empire, without
attempting to enlarge its limits. By every honorable expedient
they invited the friendship of the barbarians; and endeavored to
convince mankind that the Roman power, raised above the
temptation of conquest, was actuated only by the love of order
and justice. During a long period of forty-three years, their
virtuous labors were crowned with success; and if we except a few
slight hostilities, that served to exercise the legions of the
frontier, the reigns of Hadrian and Antoninus Pius offer the fair
prospect of universal peace.\footnotemark[27] The Roman name was revered among
the most remote nations of the earth. The fiercest barbarians
frequently submitted their differences to the arbitration of the
emperor; and we are informed by a contemporary historian that he
had seen ambassadors who were refused the honor which they came
to solicit of being admitted into the rank of subjects.\footnotemark[28]

\footnotetext[27]{We must, however, remember, that in the time of
Hadrian, a rebellion of the Jews raged with religious fury,
though only in a single province. Pausanias (l. viii. c. 43)
mentions two necessary and successful wars, conducted by the
generals of Pius: 1st. Against the wandering Moors, who were
driven into the solitudes of Atlas. 2d. Against the Brigantes of
Britain, who had invaded the Roman province. Both these wars
(with several other hostilities) are mentioned in the Augustan
History, p. 19.}

\footnotetext[28]{Appian of Alexandria, in the preface to his History
of the Roman Wars.}

The terror of the Roman arms added weight and dignity to the
moderation of the emperors. They preserved peace by a constant
preparation for war; and while justice regulated their conduct,
they announced to the nations on their confines, that they were
as little disposed to endure, as to offer an injury. The military
strength, which it had been sufficient for Hadrian and the elder
Antoninus to display, was exerted against the Parthians and the
Germans by the emperor Marcus. The hostilities of the barbarians
provoked the resentment of that philosophic monarch, and, in the
prosecution of a just defence, Marcus and his generals obtained
many signal victories, both on the Euphrates and on the Danube.\footnotemark[29]
The military establishment of the Roman empire, which thus
assured either its tranquillity or success, will now become the
proper and important object of our attention.

\footnotetext[29]{Dion, l. lxxi. Hist. August. in Marco. The Parthian
victories gave birth to a crowd of contemptible historians, whose
memory has been rescued from oblivion and exposed to ridicule, in
a very lively piece of criticism of Lucian.}

In the purer ages of the commonwealth, the use of arms was
reserved for those ranks of citizens who had a country to love, a
property to defend, and some share in enacting those laws, which
it was their interest as well as duty to maintain. But in
proportion as the public freedom was lost in extent of conquest,
war was gradually improved into an art, and degraded into a
trade.\footnotemark[30] The legions themselves, even at the time when they were
recruited in the most distant provinces, were supposed to consist
of Roman citizens. That distinction was generally considered,
either as a legal qualification or as a proper recompense for the
soldier; but a more serious regard was paid to the essential
merit of age, strength, and military stature.\footnotemark[31] In all levies, a
just preference was given to the climates of the North over those
of the South: the race of men born to the exercise of arms was
sought for in the country rather than in cities; and it was very
reasonably presumed, that the hardy occupations of smiths,
carpenters, and huntsmen, would supply more vigor and resolution
than the sedentary trades which are employed in the service of
luxury.\footnotemark[32] After every qualification of property had been laid
aside, the armies of the Roman emperors were still commanded, for
the most part, by officers of liberal birth and education; but
the common soldiers, like the mercenary troops of modern Europe,
were drawn from the meanest, and very frequently from the most
profligate, of mankind.

\footnotetext[30]{The poorest rank of soldiers possessed above forty
pounds sterling, (Dionys. Halicarn. iv. 17,) a very high
qualification at a time when money was so scarce, that an ounce
of silver was equivalent to seventy pounds weight of brass. The
populace, excluded by the ancient constitution, were
indiscriminately admitted by Marius. See Sallust. de Bell.
Jugurth. c. 91. * Note: On the uncertainty of all these
estimates, and the difficulty of fixing the relative value of
brass and silver, compare Niebuhr, vol. i. p. 473, \&c. Eng.
trans. p. 452. According to Niebuhr, the relative disproportion
in value, between the two metals, arose, in a great degree from
the abundance of brass or copper.—M. Compare also Dureau ‘de la
Malle Economie Politique des Romains especially L. l. c. ix.—M.
1845.}

\footnotetext[31]{Cæsar formed his legion Alauda of Gauls and
strangers; but it was during the license of civil war; and after
the victory, he gave them the freedom of the city for their
reward.}

\footnotetext[32]{See Vegetius, de Re Militari, l. i. c. 2—7.}

That public virtue, which among the ancients was denominated
patriotism, is derived from a strong sense of our own interest in
the preservation and prosperity of the free government of which
we are members. Such a sentiment, which had rendered the legions
of the republic almost invincible, could make but a very feeble
impression on the mercenary servants of a despotic prince; and it
became necessary to supply that defect by other motives, of a
different, but not less forcible nature—honor and religion. The
peasant, or mechanic, imbibed the useful prejudice that he was
advanced to the more dignified profession of arms, in which his
rank and reputation would depend on his own valor; and that,
although the prowess of a private soldier must often escape the
notice of fame, his own behavior might sometimes confer glory or
disgrace on the company, the legion, or even the army, to whose
honors he was associated. On his first entrance into the service,
an oath was administered to him with every circumstance of
solemnity. He promised never to desert his standard, to submit
his own will to the commands of his leaders, and to sacrifice his
life for the safety of the emperor and the empire.\footnotemark[33] The
attachment of the Roman troops to their standards was inspired by
the united influence of religion and of honor. The golden eagle,
which glittered in the front of the legion, was the object of
their fondest devotion; nor was it esteemed less impious than it
was ignominious, to abandon that sacred ensign in the hour of
danger.\footnotemark[34] These motives, which derived their strength from the
imagination, were enforced by fears and hopes of a more
substantial kind. Regular pay, occasional donatives, and a stated
recompense, after the appointed time of service, alleviated the
hardships of the military life,\footnotemark[35] whilst, on the other hand, it
was impossible for cowardice or disobedience to escape the
severest punishment. The centurions were authorized to chastise
with blows, the generals had a right to punish with death; and it
was an inflexible maxim of Roman discipline, that a good soldier
should dread his officers far more than the enemy. From such
laudable arts did the valor of the Imperial troops receive a
degree of firmness and docility unattainable by the impetuous and
irregular passions of barbarians.

\footnotetext[33]{The oath of service and fidelity to the emperor was
annually renewed by the troops on the first of January.}

\footnotetext[34]{Tacitus calls the Roman eagles, Bellorum Deos. They
were placed in a chapel in the camp, and with the other deities
received the religious worship of the troops. * Note: See also
Dio. Cass. xl. c. 18. —M.}

\footnotetext[35]{See Gronovius de Pecunia vetere, l. iii. p. 120,
\&c. The emperor Domitian raised the annual stipend of the
legionaries to twelve pieces of gold, which, in his time, was
equivalent to about ten of our guineas. This pay, somewhat higher
than our own, had been, and was afterwards, gradually increased,
according to the progress of wealth and military government.
After twenty years’ service, the veteran received three thousand
denarii, (about one hundred pounds sterling,) or a proportionable
allowance of land. The pay and advantages of the guards were, in
general, about double those of the legions.}

And yet so sensible were the Romans of the imperfection of valor
without skill and practice, that, in their language, the name of
an army was borrowed from the word which signified exercise.\footnotemark[36]
Military exercises were the important and unremitted object of
their discipline. The recruits and young soldiers were constantly
trained, both in the morning and in the evening, nor was age or
knowledge allowed to excuse the veterans from the daily
repetition of what they had completely learnt. Large sheds were
erected in the winter-quarters of the troops, that their useful
labors might not receive any interruption from the most
tempestuous weather; and it was carefully observed, that the arms
destined to this imitation of war, should be of double the weight
which was required in real action.\footnotemark[37] It is not the purpose of
this work to enter into any minute description of the Roman
exercises. We shall only remark, that they comprehended whatever
could add strength to the body, activity to the limbs, or grace
to the motions. The soldiers were diligently instructed to march,
to run, to leap, to swim, to carry heavy burdens, to handle every
species of arms that was used either for offence or for defence,
either in distant engagement or in a closer onset; to form a
variety of evolutions; and to move to the sound of flutes in the
Pyrrhic or martial dance.\footnotemark[38] In the midst of peace, the Roman
troops familiarized themselves with the practice of war; and it
is prettily remarked by an ancient historian who had fought
against them, that the effusion of blood was the only
circumstance which distinguished a field of battle from a field
of exercise.\footnotemark[39] It was the policy of the ablest generals, and
even of the emperors themselves, to encourage these military
studies by their presence and example; and we are informed that
Hadrian, as well as Trajan, frequently condescended to instruct
the unexperienced soldiers, to reward the diligent, and sometimes
to dispute with them the prize of superior strength or dexterity.\footnotemark[40]
Under the reigns of those princes, the science of tactics was
cultivated with success; and as long as the empire retained any
vigor, their military instructions were respected as the most
perfect model of Roman discipline.

\footnotetext[36]{\textit{Exercitus ab exercitando}, Varro de Lingua Latina,
l. iv. Cicero in Tusculan. l. ii. 37. 15. There is room for a
very interesting work, which should lay open the connection
between the languages and manners of nations. * Note I am not
aware of the existence, at present, of such a work; but the
profound observations of the late William von Humboldt, in the
introduction to his posthumously published Essay on the Language
of the Island of Java, (uber die Kawi-sprache, Berlin, 1836,) may
cause regret that this task was not completed by that
accomplished and universal scholar.—M.}

\footnotetext[37]{Vegatius, l. ii. and the rest of his first book.}

\footnotetext[38]{The Pyrrhic dance is extremely well illustrated by
M. le Beau, in the Academie des Inscriptions, tom. xxxv. p. 262,
\&c. That learned academician, in a series of memoirs, has
collected all the passages of the ancients that relate to the
Roman legion.}

\footnotetext[39]{Joseph. de Bell. Judaico, l. iii. c. 5. We are
indebted to this Jew for some very curious details of Roman
discipline.}

\footnotetext[40]{Plin. Panegyr. c. 13. Life of Hadrian, in the
Augustan History.}

Nine centuries of war had gradually introduced into the service
many alterations and improvements. The legions, as they are
described by Polybius,\footnotemark[41] in the time of the Punic wars, differed
very materially from those which achieved the victories of Cæsar,
or defended the monarchy of Hadrian and the Antonines. The
constitution of the Imperial legion may be described in a few
words.\footnotemark[42] The heavy-armed infantry, which composed its principal
strength,\footnotemark[43] was divided into ten cohorts, and fifty-five
companies, under the orders of a correspondent number of tribunes
and centurions. The first cohort, which always claimed the post
of honor and the custody of the eagle, was formed of eleven
hundred and five soldiers, the most approved for valor and
fidelity. The remaining nine cohorts consisted each of five
hundred and fifty-five; and the whole body of legionary infantry
amounted to six thousand one hundred men. Their arms were
uniform, and admirably adapted to the nature of their service: an
open helmet, with a lofty crest; a breastplate, or coat of mail;
greaves on their legs, and an ample buckler on their left arm.
The buckler was of an oblong and concave figure, four feet in
length, and two and a half in breadth, framed of a light wood,
covered with a bull’s hide, and strongly guarded with plates of
brass. Besides a lighter spear, the legionary soldier grasped in
his right hand the formidable \textit{pilum}, a ponderous javelin, whose
utmost length was about six feet, and which was terminated by a
massy triangular point of steel of eighteen inches.\footnotemark[44] This
instrument was indeed much inferior to our modern fire-arms;
since it was exhausted by a single discharge, at the distance of
only ten or twelve paces. Yet when it was launched by a firm and
skilful hand, there was not any cavalry that durst venture within
its reach, nor any shield or corselet that could sustain the
impetuosity of its weight. As soon as the Roman had darted his
\textit{pilum}, he drew his sword, and rushed forwards to close with the
enemy. His sword was a short well-tempered Spanish blade, that
carried a double edge, and was alike suited to the purpose of
striking or of pushing; but the soldier was always instructed to
prefer the latter use of his weapon, as his own body remained
less exposed, whilst he inflicted a more dangerous wound on his
adversary.\footnotemark[45] The legion was usually drawn up eight deep; and the
regular distance of three feet was left between the files as well
as ranks.\footnotemark[46] A body of troops, habituated to preserve this open
order, in a long front and a rapid charge, found themselves
prepared to execute every disposition which the circumstances of
war, or the skill of their leader, might suggest. The soldier
possessed a free space for his arms and motions, and sufficient
intervals were allowed, through which seasonable reinforcements
might be introduced to the relief of the exhausted combatants.\footnotemark[47]
The tactics of the Greeks and Macedonians were formed on very
different principles. The strength of the phalanx depended on
sixteen ranks of long pikes, wedged together in the closest
array.\footnotemark[48] But it was soon discovered by reflection, as well as by
the event, that the strength of the phalanx was unable to contend
with the activity of the legion.\footnotemark[49]

\footnotetext[41]{See an admirable digression on the Roman
discipline, in the sixth book of his History.}

\footnotetext[42]{Vegetius de Re Militari, l. ii. c. 4, \&c.
Considerable part of his very perplexed abridgment was taken from
the regulations of Trajan and Hadrian; and the legion, as he
describes it, cannot suit any other age of the Roman empire.}

\footnotetext[43]{Vegetius de Re Militari, l. ii. c. 1. In the purer
age of Cæsar and Cicero, the word miles was almost confined to
the infantry. Under the lower empire, and the times of chivalry,
it was appropriated almost as exclusively to the men at arms, who
fought on horseback.}

\footnotetext[44]{In the time of Polybius and Dionysius of
Halicarnassus, (l. v. c. 45,) the steel point of the pilum seems
to have been much longer. In the time of Vegetius, it was reduced
to a foot, or even nine inches. I have chosen a medium.}

\footnotetext[45]{For the legionary arms, see Lipsius de Militia
Romana, l. iii. c. 2—7.}

\footnotetext[46]{See the beautiful comparison of Virgil, Georgic ii.
v. 279.}

\footnotetext[47]{M. Guichard, Memoires Militaires, tom. i. c. 4, and
Nouveaux Memoires, tom. i. p. 293—311, has treated the subject
like a scholar and an officer.}

\footnotetext[48]{See Arrian’s Tactics. With the true partiality of a
Greek, Arrian rather chose to describe the phalanx, of which he
had read, than the legions which he had commanded.}

\footnotetext[49]{Polyb. l. xvii. (xviii. 9.)}

The cavalry, without which the force of the legion would have
remained imperfect, was divided into ten troops or squadrons; the
first, as the companion of the first cohort, consisted of a
hundred and thirty-two men; whilst each of the other nine
amounted only to sixty-six. The entire establishment formed a
regiment, if we may use the modern expression, of seven hundred
and twenty-six horse, naturally connected with its respective
legion, but occasionally separated to act in the line, and to
compose a part of the wings of the army. \footnotemark[50] The cavalry of the
emperors was no longer composed, like that of the ancient
republic, of the noblest youths of Rome and Italy, who, by
performing their military service on horseback, prepared
themselves for the offices of senator and consul; and solicited,
by deeds of valor, the future suffrages of their countrymen.\footnotemark[51]
Since the alteration of manners and government, the most wealthy
of the equestrian order were engaged in the administration of
justice, and of the revenue;\footnotemark[52] and whenever they embraced the
profession of arms, they were immediately intrusted with a troop
of horse, or a cohort of foot.\footnotemark[53] Trajan and Hadrian formed their
cavalry from the same provinces, and the same class of their
subjects, which recruited the ranks of the legion. The horses
were bred, for the most part, in Spain or Cappadocia. The Roman
troopers despised the complete armor with which the cavalry of
the East was encumbered. \textit{Their} more useful arms consisted in a
helmet, an oblong shield, light boots, and a coat of mail. A
javelin, and a long broad sword, were their principal weapons of
offence. The use of lances and of iron maces they seem to have
borrowed from the barbarians.\footnotemark[54]

\footnotetext[50]{Veget. de Re Militari, l. ii. c. 6. His positive
testimony, which might be supported by circumstantial evidence,
ought surely to silence those critics who refuse the Imperial
legion its proper body of cavalry. Note: See also Joseph. B. J.
iii. vi. 2.—M.}

\footnotetext[51]{See Livy almost throughout, particularly xlii. 61.}

\footnotetext[52]{Plin. Hist. Natur. xxxiii. 2. The true sense of
that very curious passage was first discovered and illustrated by
M. de Beaufort, Republique Romaine, l. ii. c. 2.}

\footnotetext[53]{As in the instance of Horace and Agricola. This
appears to have been a defect in the Roman discipline; which
Hadrian endeavored to remedy by ascertaining the legal age of a
tribune. * Note: These details are not altogether accurate.
Although, in the latter days of the republic, and under the first
emperors, the young Roman nobles obtained the command of a
squadron or a cohort with greater facility than in the former
times, they never obtained it without passing through a tolerably
long military service. Usually they served first in the prætorian
cohort, which was intrusted with the guard of the general: they
were received into the companionship (contubernium) of some
superior officer, and were there formed for duty. Thus Julius
Cæsar, though sprung from a great family, served first as
contubernalis under the prætor, M. Thermus, and later under
Servilius the Isaurian. (Suet. Jul. 2, 5. Plut. in Par. p. 516.
Ed. Froben.) The example of Horace, which Gibbon adduces to prove
that young knights were made tribunes immediately on entering the
service, proves nothing. In the first place, Horace was not a
knight; he was the son of a freedman of Venusia, in Apulia, who
exercised the humble office of coactor exauctionum, (collector of
payments at auctions.) (Sat. i. vi. 45, or 86.) Moreover, when
the poet was made tribune, Brutus, whose army was nearly entirely
composed of Orientals, gave this title to all the Romans of
consideration who joined him. The emperors were still less
difficult in their choice; the number of tribunes was augmented;
the title and honors were conferred on persons whom they wished
to attack to the court. Augustus conferred on the sons of
senators, sometimes the tribunate, sometimes the command of a
squadron. Claudius gave to the knights who entered into the
service, first the command of a cohort of auxiliaries, later that
of a squadron, and at length, for the first time, the tribunate.
(Suet in Claud. with the notes of Ernesti.) The abuses that arose
caused by the edict of Hadrian, which fixed the age at which that
honor could be attained. (Spart. in Had. \&c.) This edict was
subsequently obeyed; for the emperor Valerian, in a letter
addressed to Mulvius Gallinnus, prætorian præfect, excuses
himself for having violated it in favor of the young Probus
afterwards emperor, on whom he had conferred the tribunate at an
earlier age on account of his rare talents. (Vopisc. in Prob.
iv.)—W. and G. Agricola, though already invested with the title
of tribune, was contubernalis in Britain with Suetonius Paulinus.
Tac. Agr. v.—M.}

\footnotetext[54]{See Arrian’s Tactics.}

The safety and honor of the empire was principally intrusted to
the legions, but the policy of Rome condescended to adopt every
useful instrument of war. Considerable levies were regularly made
among the provincials, who had not yet deserved the honorable
distinction of Romans. Many dependent princes and communities,
dispersed round the frontiers, were permitted, for a while, to
hold their freedom and security by the tenure of military
service.\footnotemark[55] Even select troops of hostile barbarians were
frequently compelled or persuaded to consume their dangerous
valor in remote climates, and for the benefit of the state.\footnotemark[56]
All these were included under the general name of auxiliaries;
and howsoever they might vary according to the difference of
times and circumstances, their numbers were seldom much inferior
to those of the legions themselves.\footnotemark[57] Among the auxiliaries, the
bravest and most faithful bands were placed under the command of
præfects and centurions, and severely trained in the arts of
Roman discipline; but the far greater part retained those arms,
to which the nature of their country, or their early habits of
life, more peculiarly adapted them. By this institution, each
legion, to whom a certain proportion of auxiliaries was allotted,
contained within itself every species of lighter troops, and of
missile weapons; and was capable of encountering every nation,
with the advantages of its respective arms and discipline.\footnotemark[58] Nor
was the legion destitute of what, in modern language, would be
styled a train of artillery. It consisted in ten military engines
of the largest, and fifty-five of a smaller size; but all of
which, either in an oblique or horizontal manner, discharged
stones and darts with irresistible violence.\footnotemark[59]

\footnotetext[55]{Such, in particular, was the state of the
Batavians. Tacit. Germania, c. 29.}

\footnotetext[56]{Marcus Antoninus obliged the vanquished Quadi and
Marcomanni to supply him with a large body of troops, which he
immediately sent into Britain. Dion Cassius, l. lxxi. (c. 16.)}

\footnotetext[57]{Tacit. Annal. iv. 5. Those who fix a regular
proportion of as many foot, and twice as many horse, confound the
auxiliaries of the emperors with the Italian allies of the
republic.}

\footnotetext[58]{Vegetius, ii. 2. Arrian, in his order of march and
battle against the Alani.}

\footnotetext[59]{The subject of the ancient machines is treated with
great knowledge and ingenuity by the Chevalier Folard, (Polybe,
tom. ii. p. 233-290.) He prefers them in many respects to our
modern cannon and mortars. We may observe, that the use of them
in the field gradually became more prevalent, in proportion as
personal valor and military skill declined with the Roman empire.
When men were no longer found, their place was supplied by
machines. See Vegetius, ii. 25. Arrian.}

