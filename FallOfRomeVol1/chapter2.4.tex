\section{Part \thesection.}
\thispagestyle{simple}

All these cities were connected with each other, and with the
capital, by the public highways, which, issuing from the Forum of
Rome, traversed Italy, pervaded the provinces, and were
terminated only by the frontiers of the empire. If we carefully
trace the distance from the wall of Antoninus to Rome, and from
thence to Jerusalem, it will be found that the great chain of
communication, from the north-west to the south-east point of the
empire, was drawn out to the length of four thousand and eighty
Roman miles.\footnotemark[85] The public roads were accurately divided by
mile-stones, and ran in a direct line from one city to another,
with very little respect for the obstacles either of nature or
private property. Mountains were perforated, and bold arches
thrown over the broadest and most rapid streams.\footnotemark[86] The middle
part of the road was raised into a terrace which commanded the
adjacent country, consisted of several strata of sand, gravel,
and cement, and was paved with large stones, or, in some places
near the capital, with granite.\footnotemark[87] Such was the solid
construction of the Roman highways, whose firmness has not
entirely yielded to the effort of fifteen centuries. They united
the subjects of the most distant provinces by an easy and
familiar intercourse; but their primary object had been to
facilitate the marches of the legions; nor was any country
considered as completely subdued, till it had been rendered, in
all its parts, pervious to the arms and authority of the
conqueror. The advantage of receiving the earliest intelligence,
and of conveying their orders with celerity, induced the emperors
to establish, throughout their extensive dominions, the regular
institution of posts.\footnotemark[88] Houses were everywhere erected at the
distance only of five or six miles; each of them was constantly
provided with forty horses, and by the help of these relays, it
was easy to travel a hundred miles in a day along the Roman
roads.\footnotemark[89] \footnotemark[891] The use of posts was allowed to those who claimed
it by an Imperial mandate; but though originally intended for the
public service, it was sometimes indulged to the business or
conveniency of private citizens.\footnotemark[90] Nor was the communication of
the Roman empire less free and open by sea than it was by land.
The provinces surrounded and enclosed the Mediterranean: and
Italy, in the shape of an immense promontory, advanced into the
midst of that great lake. The coasts of Italy are, in general,
destitute of safe harbors; but human industry had corrected the
deficiencies of nature; and the artificial port of Ostia, in
particular, situate at the mouth of the Tyber, and formed by the
emperor Claudius, was a useful monument of Roman greatness.\footnotemark[91]
From this port, which was only sixteen miles from the capital, a
favorable breeze frequently carried vessels in seven days to the
columns of Hercules, and in nine or ten, to Alexandria in Egypt.\footnotemark[92]

\footnotetext[85]{The following Itinerary may serve to convey some
idea of the direction of the road, and of the distance between
the principal towns. I. From the wall of Antoninus to York, 222
Roman miles. II. London, 227. III. Rhutupiæ or Sandwich, 67. IV.
The navigation to Boulogne, 45. V. Rheims, 174. VI. Lyons, 330.
VII. Milan, 324. VIII. Rome, 426. IX. Brundusium, 360. X. The
navigation to Dyrrachium, 40. XI. Byzantium, 711. XII. Ancyra,
283. XIII. Tarsus, 301. XIV. Antioch, 141. XV. Tyre, 252. XVI.
Jerusalem, 168. In all 4080 Roman, or 3740 English miles. See the
Itineraries published by Wesseling, his annotations; Gale and
Stukeley for Britain, and M. d’Anville for Gaul and Italy.}

\footnotetext[86]{Montfaucon, l’Antiquite Expliquee, (tom. 4, p. 2,
l. i. c. 5,) has described the bridges of Narni, Alcantara,
Nismes, \&c.}

\footnotetext[87]{Bergier, Histoire des grands Chemins de l’Empire
Romain, l. ii. c. l. l—28.}

\footnotetext[88]{Procopius in Hist. Arcana, c. 30. Bergier, Hist.
des grands Chemins, l. iv. Codex Theodosian. l. viii. tit. v.
vol. ii. p. 506—563 with Godefroy’s learned commentary.}

\footnotetext[89]{In the time of Theodosius, Cæsarius, a magistrate
of high rank, went post from Antioch to Constantinople. He began
his journey at night, was in Cappadocia (165 miles from Antioch)
the ensuing evening, and arrived at Constantinople the sixth day
about noon. The whole distance was 725 Roman, or 665 English
miles. See Libanius, Orat. xxii., and the Itineria, p. 572—581.
Note: A courier is mentioned in Walpole’s Travels, ii. 335, who
was to travel from Aleppo to Constantinople, more than 700 miles,
in eight days, an unusually short journey.—M.}

\footnotetext[891]{Posts for the conveyance of intelligence were
established by Augustus. Suet. Aug. 49. The couriers travelled
with amazing speed. Blair on Roman Slavery, note, p. 261. It is
probable that the posts, from the time of Augustus, were confined
to the public service, and supplied by impressment Nerva, as it
appears from a coin of his reign, made an important change; “he
established posts upon all the public roads of Italy, and made
the service chargeable upon his own exchequer. Hadrian,
perceiving the advantage of this improvement, extended it to all
the provinces of the empire.” Cardwell on Coins, p. 220.—M.}

\footnotetext[90]{Pliny, though a favorite and a minister, made an
apology for granting post-horses to his wife on the most urgent
business. Epist. x. 121, 122.}

\footnotetext[91]{Bergier, Hist. des grands Chemins, l. iv. c. 49.}

\footnotetext[92]{Plin. Hist. Natur. xix. i. [In Proœm.] * Note:
Pliny says Puteoli, which seems to have been the usual landing
place from the East. See the voyages of St. Paul, Acts xxviii.
13, and of Josephus, Vita, c. 3—M.}

Whatever evils either reason or declamation have imputed to
extensive empire, the power of Rome was attended with some
beneficial consequences to mankind; and the same freedom of
intercourse which extended the vices, diffused likewise the
improvements, of social life. In the more remote ages of
antiquity, the world was unequally divided. The East was in the
immemorial possession of arts and luxury; whilst the West was
inhabited by rude and warlike barbarians, who either disdained
agriculture, or to whom it was totally unknown. Under the
protection of an established government, the productions of
happier climates, and the industry of more civilized nations,
were gradually introduced into the western countries of Europe;
and the natives were encouraged, by an open and profitable
commerce, to multiply the former, as well as to improve the
latter. It would be almost impossible to enumerate all the
articles, either of the animal or the vegetable reign, which were
successively imported into Europe from Asia and Egypt:\footnotemark[93] but it
will not be unworthy of the dignity, and much less of the
utility, of an historical work, slightly to touch on a few of the
principal heads. 1. Almost all the flowers, the herbs, and the
fruits, that grow in our European gardens, are of foreign
extraction, which, in many cases, is betrayed even by their
names: the apple was a native of Italy, and when the Romans had
tasted the richer flavor of the apricot, the peach, the
pomegranate, the citron, and the orange, they contented
themselves with applying to all these new fruits the common
denomination of apple, discriminating them from each other by the
additional epithet of their country. 2. In the time of Homer, the
vine grew wild in the island of Sicily, and most probably in the
adjacent continent; but it was not improved by the skill, nor did
it afford a liquor grateful to the taste, of the savage
inhabitants.\footnotemark[94] A thousand years afterwards, Italy could boast,
that of the fourscore most generous and celebrated wines, more
than two thirds were produced from her soil.\footnotemark[95] The blessing was
soon communicated to the Narbonnese province of Gaul; but so
intense was the cold to the north of the Cevennes, that, in the
time of Strabo, it was thought impossible to ripen the grapes in
those parts of Gaul.\footnotemark[96] This difficulty, however, was gradually
vanquished; and there is some reason to believe, that the
vineyards of Burgundy are as old as the age of the Antonines.\footnotemark[97]
3. The olive, in the western world, followed the progress of
peace, of which it was considered as the symbol. Two centuries
after the foundation of Rome, both Italy and Africa were
strangers to that useful plant: it was naturalized in those
countries; and at length carried into the heart of Spain and
Gaul. The timid errors of the ancients, that it required a
certain degree of heat, and could only flourish in the
neighborhood of the sea, were insensibly exploded by industry and
experience. 4. The cultivation of flax was transported from Egypt
to Gaul, and enriched the whole country, however it might
impoverish the particular lands on which it was sown.\footnotemark[99] 5. The
use of artificial grasses became familiar to the farmers both of
Italy and the provinces, particularly the Lucerne, which derived
its name and origin from Media.\footnotetext[100] The assured supply of
wholesome and plentiful food for the cattle during winter,
multiplied the number of the docks and herds, which in their turn
contributed to the fertility of the soil. To all these
improvements may be added an assiduous attention to mines and
fisheries, which, by employing a multitude of laborious hands,
serve to increase the pleasures of the rich and the subsistence
of the poor. The elegant treatise of Columella describes the
advanced state of the Spanish husbandry under the reign of
Tiberius; and it may be observed, that those famines, which so
frequently afflicted the infant republic, were seldom or never
experienced by the extensive empire of Rome. The accidental
scarcity, in any single province, was immediately relieved by the
plenty of its more fortunate neighbors.

\footnotetext[93]{It is not improbable that the Greeks and Phœnicians
introduced some new arts and productions into the neighborhood of
Marseilles and Gades.}

\footnotetext[94]{See Homer, Odyss. l. ix. v. 358.}

\footnotetext[95]{Plin. Hist. Natur. l. xiv.}

\footnotetext[96]{Strab. Geograph. l. iv. p. 269. The intense cold of
a Gallic winter was almost proverbial among the ancients. * Note:
Strabo only says that the grape does not ripen. Attempts had been
made in the time of Augustus to naturalize the vine in the north
of Gaul; but the cold was too great. Diod. Sic. edit. Rhodom. p.
304.—W. Diodorus (lib. v. 26) gives a curious picture of the
Italian traders bartering, with the savages of Gaul, a cask of
wine for a slave.—M. —It appears from the newly discovered
treatise of Cicero de Republica, that there was a law of the
republic prohibiting the culture of the vine and olive beyond the
Alps, in order to keep up the value of those in Italy. Nos
justissimi homines, qui transalpinas gentes oleam et vitem serere
non sinimus, quo pluris sint nostra oliveta nostræque vineæ. Lib.
iii. 9. The restrictive law of Domitian was veiled under the
decent pretext of encouraging the cultivation of grain. Suet.
Dom. vii. It was repealed by Probus Vopis Strobus, 18.—M.}

\footnotetext[97]{In the beginning of the fourth century, the orator
Eumenius (Panegyr. Veter. viii. 6, edit. Delphin.) speaks of the
vines in the territory of Autun, which were decayed through age,
and the first plantation of which was totally unknown. The Pagus
Arebrignus is supposed by M. d’Anville to be the district of
Beaune, celebrated, even at present for one of the first growths
of Burgundy. * Note: This is proved by a passage of Pliny the
Elder, where he speaks of a certain kind of grape (vitis picata.
vinum picatum) which grows naturally to the district of Vienne,
and had recently been transplanted into the country of the
Arverni, (Auvergne,) of the Helvii, (the Vivarias.) and the
Burgundy and Franche Compte. Pliny wrote A.D. 77. Hist. Nat. xiv.
1.— W.}

\footnotetext[99]{Plin. Hist. Natur. l. xix.}

\footnotetext[100]{See the agreeable Essays on Agriculture by Mr.
Harte, in which he has collected all that the ancients and
moderns have said of Lucerne.}

Agriculture is the foundation of manufactures; since the
productions of nature are the materials of art. Under the Roman
empire, the labor of an industrious and ingenious people was
variously, but incessantly, employed in the service of the rich.
In their dress, their table, their houses, and their furniture,
the favorites of fortune united every refinement of conveniency,
of elegance, and of splendor, whatever could soothe their pride
or gratify their sensuality. Such refinements, under the odious
name of luxury, have been severely arraigned by the moralists of
every age; and it might perhaps be more conducive to the virtue,
as well as happiness, of mankind, if all possessed the
necessaries, and none the superfluities, of life. But in the
present imperfect condition of society, luxury, though it may
proceed from vice or folly, seems to be the only means that can
correct the unequal distribution of property. The diligent
mechanic, and the skilful artist, who have obtained no share in
the division of the earth, receive a voluntary tax from the
possessors of land; and the latter are prompted, by a sense of
interest, to improve those estates, with whose produce they may
purchase additional pleasures. This operation, the particular
effects of which are felt in every society, acted with much more
diffusive energy in the Roman world. The provinces would soon
have been exhausted of their wealth, if the manufactures and
commerce of luxury had not insensibly restored to the industrious
subjects the sums which were exacted from them by the arms and
authority of Rome. As long as the circulation was confined within
the bounds of the empire, it impressed the political machine with
a new degree of activity, and its consequences, sometimes
beneficial, could never become pernicious.

But it is no easy task to confine luxury within the limits of an
empire. The most remote countries of the ancient world were
ransacked to supply the pomp and delicacy of Rome. The forests of
Scythia afforded some valuable furs. Amber was brought over land
from the shores of the Baltic to the Danube; and the barbarians
were astonished at the price which they received in exchange for
so useless a commodity.\footnotemark[101] There was a considerable demand for
Babylonian carpets, and other manufactures of the East; but the
most important and unpopular branch of foreign trade was carried
on with Arabia and India. Every year, about the time of the
summer solstice, a fleet of a hundred and twenty vessels sailed
from Myos-hormos, a port of Egypt, on the Red Sea. By the
periodical assistance of the monsoons, they traversed the ocean
in about forty days. The coast of Malabar, or the island of
Ceylon,\footnotemark[102] was the usual term of their navigation, and it was in
those markets that the merchants from the more remote countries
of Asia expected their arrival. The return of the fleet of Egypt
was fixed to the months of December or January; and as soon as
their rich cargo had been transported on the backs of camels,
from the Red Sea to the Nile, and had descended that river as far
as Alexandria, it was poured, without delay, into the capital of
the empire.\footnotemark[103] The objects of oriental traffic were splendid and
trifling; silk, a pound of which was esteemed not inferior in
value to a pound of gold;\footnotemark[104] precious stones, among which the
pearl claimed the first rank after the diamond;\footnotemark[105] and a variety
of aromatics, that were consumed in religious worship and the
pomp of funerals. The labor and risk of the voyage was rewarded
with almost incredible profit; but the profit was made upon Roman
subjects, and a few individuals were enriched at the expense of
the public. As the natives of Arabia and India were contented
with the productions and manufactures of their own country,
silver, on the side of the Romans, was the principal, if not the
only\footnotemark[1051] instrument of commerce. It was a complaint worthy of
the gravity of the senate, that, in the purchase of female
ornaments, the wealth of the state was irrecoverably given away
to foreign and hostile nations.\footnotemark[106] The annual loss is computed,
by a writer of an inquisitive but censorious temper, at upwards
of eight hundred thousand pounds sterling.\footnotemark[107] Such was the style
of discontent, brooding over the dark prospect of approaching
poverty. And yet, if we compare the proportion between gold and
silver, as it stood in the time of Pliny, and as it was fixed in
the reign of Constantine, we shall discover within that period a
very considerable increase.\footnotemark[108] There is not the least reason to
suppose that gold was become more scarce; it is therefore evident
that silver was grown more common; that whatever might be the
amount of the Indian and Arabian exports, they were far from
exhausting the wealth of the Roman world; and that the produce of
the mines abundantly supplied the demands of commerce.

\footnotetext[101]{Tacit. Germania, c. 45. Plin. Hist. Nat. xxxvii.
13. The latter observed, with some humor, that even fashion had
not yet found out the use of amber. Nero sent a Roman knight to
purchase great quantities on the spot where it was produced, the
coast of modern Prussia.}

\footnotetext[102]{Called Taprobana by the Romans, and Serindib by
the Arabs. It was discovered under the reign of Claudius, and
gradually became the principal mart of the East.}

\footnotetext[103]{Plin. Hist. Natur. l. vi. Strabo, l. xvii.}

\footnotetext[104]{Hist. August. p. 224. A silk garment was
considered as an ornament to a woman, but as a disgrace to a
man.}

\footnotetext[105]{The two great pearl fisheries were the same as at
present, Ormuz and Cape Comorin. As well as we can compare
ancient with modern geography, Rome was supplied with diamonds
from the mine of Jumelpur, in Bengal, which is described in the
Voyages de Tavernier, tom. ii. p. 281.}

\footnotetext[1051]{Certainly not the only one. The Indians were not
so contented with regard to foreign productions. Arrian has a
long list of European wares, which they received in exchange for
their own; Italian and other wines, brass, tin, lead, coral,
chrysolith, storax, glass, dresses of one or many colors, zones,
\&c. See Periplus Maris Erythræi in Hudson, Geogr. Min. i. p.
27.—W. The German translator observes that Gibbon has confined
the use of aromatics to religious worship and funerals. His error
seems the omission of other spices, of which the Romans must have
consumed great quantities in their cookery. Wenck, however,
admits that silver was the chief article of exchange.—M. In 1787,
a peasant (near Nellore in the Carnatic) struck, in digging, on
the remains of a Hindu temple; he found, also, a pot which
contained Roman coins and medals of the second century, mostly
Trajans, Adrians, and Faustinas, all of gold, many of them fresh
and beautiful, others defaced or perforated, as if they had been
worn as ornaments. (Asiatic Researches, ii. 19.)—M.}

\footnotetext[106]{Tacit. Annal. iii. 53. In a speech of Tiberius.}

\footnotetext[107]{Plin. Hist. Natur. xii. 18. In another place he
computes half that sum; Quingenties H. S. for India exclusive of
Arabia.}

\footnotetext[108]{The proportion, which was 1 to 10, and 12 1/2,
rose to 14 2/5, the legal regulation of Constantine. See
Arbuthnot’s Tables of ancient Coins, c. 5.}

Notwithstanding the propensity of mankind to exalt the past, and
to depreciate the present, the tranquil and prosperous state of
the empire was warmly felt, and honestly confessed, by the
provincials as well as Romans. “They acknowledged that the true
principles of social life, laws, agriculture, and science, which
had been first invented by the wisdom of Athens, were now firmly
established by the power of Rome, under whose auspicious
influence the fiercest barbarians were united by an equal
government and common language. They affirm, that with the
improvement of arts, the human species were visibly multiplied.
They celebrate the increasing splendor of the cities, the
beautiful face of the country, cultivated and adorned like an
immense garden; and the long festival of peace which was enjoyed
by so many nations, forgetful of the ancient animosities, and
delivered from the apprehension of future danger.”\footnotemark[109] Whatever
suspicions may be suggested by the air of rhetoric and
declamation, which seems to prevail in these passages, the
substance of them is perfectly agreeable to historic truth.

\footnotetext[109]{Among many other passages, see Pliny, (Hist.
Natur. iii. 5.) Aristides, (de Urbe Roma,) and Tertullian, (de
Anima, c. 30.)}

It was scarcely possible that the eyes of contemporaries should
discover in the public felicity the latent causes of decay and
corruption. This long peace, and the uniform government of the
Romans, introduced a slow and secret poison into the vitals of
the empire. The minds of men were gradually reduced to the same
level, the fire of genius was extinguished, and even the military
spirit evaporated. The natives of Europe were brave and robust.
Spain, Gaul, Britain, and Illyricum supplied the legions with
excellent soldiers, and constituted the real strength of the
monarchy. Their personal valor remained, but they no longer
possessed that public courage which is nourished by the love of
independence, the sense of national honor, the presence of
danger, and the habit of command. They received laws and
governors from the will of their sovereign, and trusted for their
defence to a mercenary army. The posterity of their boldest
leaders was contented with the rank of citizens and subjects. The
most aspiring spirits resorted to the court or standard of the
emperors; and the deserted provinces, deprived of political
strength or union, insensibly sunk into the languid indifference
of private life.

The love of letters, almost inseparable from peace and
refinement, was fashionable among the subjects of Hadrian and the
Antonines, who were themselves men of learning and curiosity. It
was diffused over the whole extent of their empire; the most
northern tribes of Britons had acquired a taste for rhetoric;
Homer as well as Virgil were transcribed and studied on the banks
of the Rhine and Danube; and the most liberal rewards sought out
the faintest glimmerings of literary merit.\footnotemark[110] The sciences of
physic and astronomy were successfully cultivated by the Greeks;
the observations of Ptolemy and the writings of Galen are studied
by those who have improved their discoveries and corrected their
errors; but if we except the inimitable Lucian, this age of
indolence passed away without having produced a single writer of
original genius, or who excelled in the arts of elegant
composition.\footnotemark[1101] The authority of Plato and Aristotle, of Zeno
and Epicurus, still reigned in the schools; and their systems,
transmitted with blind deference from one generation of disciples
to another, precluded every generous attempt to exercise the
powers, or enlarge the limits, of the human mind. The beauties of
the poets and orators, instead of kindling a fire like their own,
inspired only cold and servile imitations: or if any ventured to
deviate from those models, they deviated at the same time from
good sense and propriety. On the revival of letters, the youthful
vigor of the imagination, after a long repose, national
emulation, a new religion, new languages, and a new world, called
forth the genius of Europe. But the provincials of Rome, trained
by a uniform artificial foreign education, were engaged in a very
unequal competition with those bold ancients, who, by expressing
their genuine feelings in their native tongue, had already
occupied every place of honor. The name of Poet was almost
forgotten; that of Orator was usurped by the sophists. A cloud of
critics, of compilers, of commentators, darkened the face of
learning, and the decline of genius was soon followed by the
corruption of taste.

\footnotetext[110]{Herodes Atticus gave the sophist Polemo above
eight thousand pounds for three declamations. See Philostrat. l.
i. p. 538. The Antonines founded a school at Athens, in which
professors of grammar, rhetoric, politics, and the four great
sects of philosophy were maintained at the public expense for the
instruction of youth. The salary of a philosopher was ten
thousand drachmæ, between three and four hundred pounds a year.
Similar establishments were formed in the other great cities of
the empire. See Lucian in Eunuch. tom. ii. p. 352, edit. Reitz.
Philostrat. l. ii. p. 566. Hist. August. p. 21. Dion Cassius, l.
lxxi. p. 1195. Juvenal himself, in a morose satire, which in
every line betrays his own disappointment and envy, is obliged,
however, to say,—“—O Juvenes, circumspicit et stimulat vos.
Materiamque sibi Ducis indulgentia quærit.”—Satir. vii. 20. Note:
Vespasian first gave a salary to professors: he assigned to each
professor of rhetoric, Greek and Roman, centena sestertia.
(Sueton. in Vesp. 18). Hadrian and the Antonines, though still
liberal, were less profuse.—G. from W. Suetonius wrote annua
centena L. 807, 5, 10.—M.}

\footnotetext[1101]{This judgment is rather severe: besides the
physicians, astronomers, and grammarians, among whom there were
some very distinguished men, there were still, under Hadrian,
Suetonius, Florus, Plutarch; under the Antonines, Arrian,
Pausanias, Appian, Marcus Aurelius himself, Sextus Empiricus, \&c.
Jurisprudence gained much by the labors of Salvius Julianus,
Julius Celsus, Sex. Pomponius, Caius, and others.—G. from W. Yet
where, among these, is the writer of original genius, unless,
perhaps Plutarch? or even of a style really elegant?— M.}

The sublime Longinus, who, in somewhat a later period, and in the
court of a Syrian queen, preserved the spirit of ancient Athens,
observes and laments this degeneracy of his contemporaries, which
debased their sentiments, enervated their courage, and depressed
their talents. “In the same manner,” says he, “as some children
always remain pygmies, whose infant limbs have been too closely
confined, thus our tender minds, fettered by the prejudices and
habits of a just servitude, are unable to expand themselves, or
to attain that well-proportioned greatness which we admire in the
ancients; who, living under a popular government, wrote with the
same freedom as they acted.”\footnotemark[111] This diminutive stature of
mankind, if we pursue the metaphor, was daily sinking below the
old standard, and the Roman world was indeed peopled by a race of
pygmies; when the fierce giants of the north broke in, and mended
the puny breed. They restored a manly spirit of freedom; and
after the revolution of ten centuries, freedom became the happy
parent of taste and science.

\footnotetext[111]{Longin. de Sublim. c. 44, p. 229, edit. Toll.
Here, too, we may say of Longinus, “his own example strengthens
all his laws.” Instead of proposing his sentiments with a manly
boldness, he insinuates them with the most guarded caution; puts
them into the mouth of a friend, and as far as we can collect
from a corrupted text, makes a show of refuting them himself.}

