\chapter{The Internal Prosperity In The Age Of The Antonines.}
\section{Part \thesection.}

\textit{Of The Union And Internal Prosperity Of The Roman Empire, In The
Age Of The Antonines.}
\vspace{\onelineskip}

It is not alone by the rapidity, or extent of conquest, that we
should estimate the greatness of Rome. The sovereign of the
Russian deserts commands a larger portion of the globe. In the
seventh summer after his passage of the Hellespont, Alexander
erected the Macedonian trophies on the banks of the Hyphasis.\footnotemark[1]
Within less than a century, the irresistible Zingis, and the
Mogul princes of his race, spread their cruel devastations and
transient empire from the Sea of China, to the confines of Egypt
and Germany.\footnotemark[2] But the firm edifice of Roman power was raised and
preserved by the wisdom of ages. The obedient provinces of Trajan
and the Antonines were united by laws, and adorned by arts. They
might occasionally suffer from the partial abuse of delegated
authority; but the general principle of government was wise,
simple, and beneficent. They enjoyed the religion of their
ancestors, whilst in civil honors and advantages they were
exalted, by just degrees, to an equality with their conquerors.

\footnotetext[1]{They were erected about the midway between Lahor and
Delhi. The conquests of Alexander in Hindostan were confined to
the Punjab, a country watered by the five great streams of the
Indus. * Note: The Hyphasis is one of the five rivers which join
the Indus or the Sind, after having traversed the province of the
Pendj-ab—a name which in Persian, signifies \textit{five rivers}. * * *
G. The five rivers were, 1. The Hydaspes, now the Chelum, Behni,
or Bedusta, (\textit{Sanscrit}, Vitashà, Arrow-swift.) 2. The Acesines,
the Chenab, (\textit{Sanscrit}, Chandrabhágâ, Moon-gift.) 3. Hydraotes,
the Ravey, or Iraoty, (\textit{Sanscrit}, Irâvatî.) 4. Hyphasis, the
Beyah, (\textit{Sanscrit}, Vepâsà, Fetterless.) 5. The Satadru,
(\textit{Sanscrit}, the Hundred Streamed,) the Sutledj, known first to
the Greeks in the time of Ptolemy. Rennel. Vincent, Commerce of
Anc. book 2. Lassen, Pentapotam. Ind. Wilson’s Sanscrit Dict.,
and the valuable memoir of Lieut. Burnes, Journal of London
Geogr. Society, vol. iii. p. 2, with the travels of that very
able writer. Compare Gibbon’s own note, c. lxv. note 25.—M
substit. for G.}

\footnotetext[2]{See M. de Guignes, Histoire des Huns, l. xv. xvi.
and xvii.}

I. The policy of the emperors and the senate, as far as it
concerned religion, was happily seconded by the reflections of
the enlightened, and by the habits of the superstitious, part of
their subjects. The various modes of worship, which prevailed in
the Roman world, were all considered by the people, as equally
true; by the philosopher, as equally false; and by the
magistrate, as equally useful. And thus toleration produced not
only mutual indulgence, but even religious concord.

The superstition of the people was not imbittered by any mixture
of theological rancor; nor was it confined by the chains of any
speculative system. The devout polytheist, though fondly attached
to his national rites, admitted with implicit faith the different
religions of the earth.\footnotemark[3] Fear, gratitude, and curiosity, a dream
or an omen, a singular disorder, or a distant journey,
perpetually disposed him to multiply the articles of his belief,
and to enlarge the list of his protectors. The thin texture of
the Pagan mythology was interwoven with various but not
discordant materials. As soon as it was allowed that sages and
heroes, who had lived or who had died for the benefit of their
country, were exalted to a state of power and immortality, it was
universally confessed, that they deserved, if not the adoration,
at least the reverence, of all mankind. The deities of a thousand
groves and a thousand streams possessed, in peace, their local
and respective influence; nor could the Romans who deprecated the
wrath of the Tiber, deride the Egyptian who presented his
offering to the beneficent genius of the Nile. The visible powers
of nature, the planets, and the elements were the same throughout
the universe. The invisible governors of the moral world were
inevitably cast in a similar mould of fiction and allegory. Every
virtue, and even vice, acquired its divine representative; every
art and profession its patron, whose attributes, in the most
distant ages and countries, were uniformly derived from the
character of their peculiar votaries. A republic of gods of such
opposite tempers and interests required, in every system, the
moderating hand of a supreme magistrate, who, by the progress of
knowledge and flattery, was gradually invested with the sublime
perfections of an Eternal Parent, and an Omnipotent Monarch.\footnotemark[4]
Such was the mild spirit of antiquity, that the nations were less
attentive to the difference, than to the resemblance, of their
religious worship. The Greek, the Roman, and the Barbarian, as
they met before their respective altars, easily persuaded
themselves, that under various names, and with various
ceremonies, they adored the same deities.\footnotemark[5] The elegant mythology
of Homer gave a beautiful, and almost a regular form, to the
polytheism of the ancient world.

\footnotetext[3]{There is not any writer who describes in so lively a
manner as Herodotus the true genius of polytheism. The best
commentary may be found in Mr. Hume’s Natural History of
Religion; and the best contrast in Bossuet’s Universal History.
Some obscure traces of an intolerant spirit appear in the conduct
of the Egyptians, (see Juvenal, Sat. xv.;) and the Christians, as
well as Jews, who lived under the Roman empire, formed a very
important exception; so important indeed, that the discussion
will require a distinct chapter of this work. * Note: M.
Constant, in his very learned and eloquent work, “Sur la
Religion,” with the two additional volumes, “Du Polytheisme
Romain,” has considered the whole history of polytheism in a tone
of philosophy, which, without subscribing to all his opinions, we
may be permitted to admire. “The boasted tolerance of polytheism
did not rest upon the respect due from society to the freedom of
individual opinion. The polytheistic nations, tolerant as they
were towards each other, as separate states, were not the less
ignorant of the eternal principle, the only basis of enlightened
toleration, that every one has a right to worship God in the
manner which seems to him the best. Citizens, on the contrary,
were bound to conform to the religion of the state; they had not
the liberty to adopt a foreign religion, though that religion
might be legally recognized in their own city, for the strangers
who were its votaries.” —Sur la Religion, v. 184. Du. Polyth.
Rom. ii. 308. At this time, the growing religious indifference,
and the general administration of the empire by Romans, who,
being strangers, would do no more than protect, not enlist
themselves in the cause of the local superstitions, had
introduced great laxity. But intolerance was clearly the theory
both of the Greek and Roman law. The subject is more fully
considered in another place.—M.}

\footnotetext[4]{The rights, powers, and pretensions of the sovereign
of Olympus are very clearly described in the xvth book of the
Iliad; in the Greek original, I mean; for Mr. Pope, without
perceiving it, has improved the theology of Homer. * Note: There
is a curious coincidence between Gibbon’s expressions and those
of the newly-recovered “De Republica” of Cicero, though the
argument is rather the converse, lib. i. c. 36. “Sive hæc ad
utilitatem vitæ constitute sint a principibus rerum publicarum,
ut rex putaretur unus esse in cœlo, qui nutu, ut ait Homerus,
totum Olympum converteret, idemque et rex et patos haberetur
omnium.”—M.}

\footnotetext[5]{See, for instance, Cæsar de Bell. Gall. vi. 17.
Within a century or two, the Gauls themselves applied to their
gods the names of Mercury, Mars, Apollo, \&c.}

The philosophers of Greece deduced their morals from the nature
of man, rather than from that of God. They meditated, however, on
the Divine Nature, as a very curious and important speculation;
and in the profound inquiry, they displayed the strength and
weakness of the human understanding.\footnotemark[6] Of the four most
celebrated schools, the Stoics and the Platonists endeavored to
reconcile the jaring interests of reason and piety. They have
left us the most sublime proofs of the existence and perfections
of the first cause; but, as it was impossible for them to
conceive the creation of matter, the workman in the Stoic
philosophy was not sufficiently distinguished from the work;
whilst, on the contrary, the spiritual God of Plato and his
disciples resembled an idea, rather than a substance. The
opinions of the Academics and Epicureans were of a less religious
cast; but whilst the modest science of the former induced them to
doubt, the positive ignorance of the latter urged them to deny,
the providence of a Supreme Ruler. The spirit of inquiry,
prompted by emulation, and supported by freedom, had divided the
public teachers of philosophy into a variety of contending sects;
but the ingenious youth, who, from every part, resorted to
Athens, and the other seats of learning in the Roman empire, were
alike instructed in every school to reject and to despise the
religion of the multitude. How, indeed, was it possible that a
philosopher should accept, as divine truths, the idle tales of
the poets, and the incoherent traditions of antiquity; or that he
should adore, as gods, those imperfect beings whom he must have
despised, as men? Against such unworthy adversaries, Cicero
condescended to employ the arms of reason and eloquence; but the
satire of Lucian was a much more adequate, as well as more
efficacious, weapon. We may be well assured, that a writer,
conversant with the world, would never have ventured to expose
the gods of his country to public ridicule, had they not already
been the objects of secret contempt among the polished and
enlightened orders of society.\footnotemark[7]

\footnotetext[6]{The admirable work of Cicero de Natura Deorum is the
best clew we have to guide us through the dark and profound
abyss. He represents with candor, and confutes with subtlety, the
opinions of the philosophers.}

\footnotetext[7]{I do not pretend to assert, that, in this
irreligious age, the natural terrors of superstition, dreams,
omens, apparitions, \&c., had lost their efficacy.}

Notwithstanding the fashionable irreligion which prevailed in the
age of the Antonines, both the interest of the priests and the
credulity of the people were sufficiently respected. In their
writings and conversation, the philosophers of antiquity asserted
the independent dignity of reason; but they resigned their
actions to the commands of law and of custom. Viewing, with a
smile of pity and indulgence, the various errors of the vulgar,
they diligently practised the ceremonies of their fathers,
devoutly frequented the temples of the gods; and sometimes
condescending to act a part on the theatre of superstition, they
concealed the sentiments of an atheist under the sacerdotal
robes. Reasoners of such a temper were scarcely inclined to
wrangle about their respective modes of faith, or of worship. It
was indifferent to them what shape the folly of the multitude
might choose to assume; and they approached with the same inward
contempt, and the same external reverence, the altars of the
Libyan, the Olympian, or the Capitoline Jupiter.\footnotemark[8]

\footnotetext[8]{Socrates, Epicurus, Cicero, and Plutarch always
inculcated a decent reverence for the religion of their own
country, and of mankind. The devotion of Epicurus was assiduous
and exemplary. Diogen. Lært. x. 10.}

It is not easy to conceive from what motives a spirit of
persecution could introduce itself into the Roman councils. The
magistrates could not be actuated by a blind, though honest
bigotry, since the magistrates were themselves philosophers; and
the schools of Athens had given laws to the senate. They could
not be impelled by ambition or avarice, as the temporal and
ecclesiastical powers were united in the same hands. The pontiffs
were chosen among the most illustrious of the senators; and the
office of Supreme Pontiff was constantly exercised by the
emperors themselves. They knew and valued the advantages of
religion, as it is connected with civil government. They
encouraged the public festivals which humanize the manners of the
people. They managed the arts of divination as a convenient
instrument of policy; and they respected, as the firmest bond of
society, the useful persuasion, that, either in this or in a
future life, the crime of perjury is most assuredly punished by
the avenging gods.\footnotemark[9] But whilst they acknowledged the general
advantages of religion, they were convinced that the various
modes of worship contributed alike to the same salutary purposes;
and that, in every country, the form of superstition, which had
received the sanction of time and experience, was the best
adapted to the climate, and to its inhabitants. Avarice and taste
very frequently despoiled the vanquished nations of the elegant
statues of their gods, and the rich ornaments of their temples;\footnotemark[10]
but, in the exercise of the religion which they derived from
their ancestors, they uniformly experienced the indulgence, and
even protection, of the Roman conquerors. The province of Gaul
seems, and indeed only seems, an exception to this universal
toleration. Under the specious pretext of abolishing human
sacrifices, the emperors Tiberius and Claudius suppressed the
dangerous power of the Druids:\footnotemark[11] but the priests themselves,
their gods and their altars, subsisted in peaceful obscurity till
the final destruction of Paganism.\footnotemark[12]

\footnotetext[9]{Polybius, l. vi. c. 53, 54. Juvenal, Sat. xiii.
laments that in his time this apprehension had lost much of its
effect.}

\footnotetext[10]{See the fate of Syracuse, Tarentum, Ambracia,
Corinth, \&c., the conduct of Verres, in Cicero, (Actio ii. Orat.
4,) and the usual practice of governors, in the viiith Satire of
Juvenal.}

\footnotetext[11]{Seuton. in Claud.—Plin. Hist. Nat. xxx. 1.}

\footnotetext[12]{Pelloutier, Histoire des Celtes, tom. vi. p.
230—252.}

Rome, the capital of a great monarchy, was incessantly filled
with subjects and strangers from every part of the world,\footnotemark[13] who
all introduced and enjoyed the favorite superstitions of their
native country.\footnotemark[14] Every city in the empire was justified in
maintaining the purity of its ancient ceremonies; and the Roman
senate, using the common privilege, sometimes interposed, to
check this inundation of foreign rites.\footnotemark[141] The Egyptian
superstition, of all the most contemptible and abject, was
frequently prohibited: the temples of Serapis and Isis
demolished, and their worshippers banished from Rome and Italy.\footnotemark[15]
But the zeal of fanaticism prevailed over the cold and feeble
efforts of policy. The exiles returned, the proselytes
multiplied, the temples were restored with increasing splendor,
and Isis and Serapis at length assumed their place among the
Roman Deities.\footnotemark[151]\footnotemark[16] Nor was this indulgence a departure from
the old maxims of government. In the purest ages of the
commonwealth, Cybele and Æsculapius had been invited by solemn
embassies;\footnotemark[17] and it was customary to tempt the protectors of
besieged cities, by the promise of more distinguished honors than
they possessed in their native country.\footnotemark[18] Rome gradually became
the common temple of her subjects; and the freedom of the city
was bestowed on all the gods of mankind.\footnotemark[19]

\footnotetext[13]{Seneca, Consolat. ad Helviam, p. 74. Edit., Lips.}

\footnotetext[14]{Dionysius Halicarn. Antiquitat. Roman. l. ii. (vol.
i. p. 275, edit. Reiske.)}

\footnotetext[141]{Yet the worship of foreign gods at Rome was only
guarantied to the natives of those countries from whence they
came. The Romans administered the priestly offices only to the
gods of their fathers. Gibbon, throughout the whole preceding
sketch of the opinions of the Romans and their subjects, has
shown through what causes they were free from religious hatred
and its consequences. But, on the other hand the internal state
of these religions, the infidelity and hypocrisy of the upper
orders, the indifference towards all religion, in even the better
part of the common people, during the last days of the republic,
and under the Cæsars, and the corrupting principles of the
philosophers, had exercised a very pernicious influence on the
manners, and even on the constitution.—W.}

\footnotetext[15]{In the year of Rome 701, the temple of Isis and
Serapis was demolished by the order of the Senate, (Dion Cassius,
l. xl. p. 252,) and even by the hands of the consul, (Valerius
Maximus, l. 3.) After the death of Cæsar it was restored at the
public expense, (Dion. l. xlvii. p. 501.) When Augustus was in
Egypt, he revered the majesty of Serapis, (Dion, l. li. p. 647;)
but in the Pomærium of Rome, and a mile round it, he prohibited
the worship of the Egyptian gods, (Dion, l. liii. p. 679; l. liv.
p. 735.) They remained, however, very fashionable under his reign
(Ovid. de Art. Amand. l. i.) and that of his successor, till the
justice of Tiberius was provoked to some acts of severity. (See
Tacit. Annal. ii. 85. Joseph. Antiquit. l. xviii. c. 3.) * Note:
See, in the pictures from the walls of Pompeii, the
representation of an Isiac temple and worship. Vestiges of
Egyptian worship have been traced in Gaul, and, I am informed,
recently in Britain, in excavations at York.— M.}

\footnotetext[151]{Gibbon here blends into one, two events, distant a
hundred and sixty-six years from each other. It was in the year
of Rome 535, that the senate having ordered the destruction of
the temples of Isis and Serapis, the workman would lend his hand;
and the consul, L. Paulus himself (Valer. Max. 1, 3) seized the
axe, to give the first blow. Gibbon attribute this circumstance
to the second demolition, which took place in the year 701 and
which he considers as the first.—W.}

\footnotetext[16]{Tertullian in Apologetic. c. 6, p. 74. Edit.
Havercamp. I am inclined to attribute their establishment to the
devotion of the Flavian family.}

\footnotetext[17]{See Livy, l. xi. [Suppl.] and xxix.}

\footnotetext[18]{Macrob. Saturnalia, l. iii. c. 9. He gives us a
form of evocation.}

\footnotetext[19]{Minutius Fælix in Octavio, p. 54. Arnobius, l. vi.
p. 115.}

II. The narrow policy of preserving, without any foreign mixture,
the pure blood of the ancient citizens, had checked the fortune,
and hastened the ruin, of Athens and Sparta. The aspiring genius
of Rome sacrificed vanity to ambition, and deemed it more
prudent, as well as honorable, to adopt virtue and merit for her
own wheresoever they were found, among slaves or strangers,
enemies or barbarians.\footnotemark[20] During the most flourishing æra of the
Athenian commonwealth, the number of citizens gradually decreased
from about thirty\footnotemark[21] to twenty-one thousand.\footnotemark[22] If, on the
contrary, we study the growth of the Roman republic, we may
discover, that, notwithstanding the incessant demands of wars and
colonies, the citizens, who, in the first census of Servius
Tullius, amounted to no more than eighty-three thousand, were
multiplied, before the commencement of the social war, to the
number of four hundred and sixty-three thousand men, able to bear
arms in the service of their country.\footnotemark[23] When the allies of Rome
claimed an equal share of honors and privileges, the senate
indeed preferred the chance of arms to an ignominious concession.
The Samnites and the Lucanians paid the severe penalty of their
rashness; but the rest of the Italian states, as they
successively returned to their duty, were admitted into the bosom
of the republic,\footnotemark[24] and soon contributed to the ruin of public
freedom. Under a democratical government, the citizens exercise
the powers of sovereignty; and those powers will be first abused,
and afterwards lost, if they are committed to an unwieldy
multitude. But when the popular assemblies had been suppressed by
the administration of the emperors, the conquerors were
distinguished from the vanquished nations, only as the first and
most honorable order of subjects; and their increase, however
rapid, was no longer exposed to the same dangers. Yet the wisest
princes, who adopted the maxims of Augustus, guarded with the
strictest care the dignity of the Roman name, and diffused the
freedom of the city with a prudent liberality.\footnotemark[25]

\footnotetext[20]{Tacit. Annal. xi. 24. The Orbis Romanus of the
learned Spanheim is a complete history of the progressive
admission of Latium, Italy, and the provinces, to the freedom of
Rome. * Note: Democratic states, observes Denina, (delle Revoluz.
d’ Italia, l. ii. c. l.), are most jealous of communication the
privileges of citizenship; monarchies or oligarchies willingly
multiply the numbers of their free subjects. The most remarkable
accessions to the strength of Rome, by the aggregation of
conquered and foreign nations, took place under the regal and
patrician—we may add, the Imperial government.—M.}

\footnotetext[21]{Herodotus, v. 97. It should seem, however, that he
followed a large and popular estimation.}

\footnotetext[22]{Athenæus, Deipnosophist. l. vi. p. 272. Edit.
Casaubon. Meursius de Fortunâ Atticâ, c. 4. * Note: On the number
of citizens in Athens, compare Bœckh, Public Economy of Athens,
(English Tr.,) p. 45, et seq. Fynes Clinton, Essay in Fasti Hel
lenici, vol. i. 381.—M.}

\footnotetext[23]{See a very accurate collection of the numbers of
each Lustrum in M. de Beaufort, Republique Romaine, l. iv. c. 4.
Note: All these questions are placed in an entirely new point of
view by Niebuhr, (Römische Geschichte, vol. i. p. 464.) He
rejects the census of Servius fullius as unhistoric, (vol. ii. p.
78, et seq.,) and he establishes the principle that the census
comprehended all the confederate cities which had the right of
Isopolity.—M.}

\footnotetext[24]{Appian. de Bell. Civil. l. i. Velleius Paterculus,
l. ii. c. 15, 16, 17.}

\footnotetext[25]{Mæcenas had advised him to declare, by one edict,
all his subjects citizens. But we may justly suspect that the
historian Dion was the author of a counsel so much adapted to the
practice of his own age, and so little to that of Augustus.}

