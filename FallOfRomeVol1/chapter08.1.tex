\chapter{State Of Persia And Restoration Of The Monarchy.}
\section{Part \thesection.}

\textit{Of The State Of Persia After The Restoration Of The Monarchy By Artaxerxes.}
\vspace{\onelineskip}

Whenever Tacitus indulges himself in those beautiful episodes, in
which he relates some domestic transaction of the Germans or of
the Parthians, his principal object is to relieve the attention
of the reader from a uniform scene of vice and misery. From the
reign of Augustus to the time of Alexander Severus, the enemies
of Rome were in her bosom—the tyrants and the soldiers; and her
prosperity had a very distant and feeble interest in the
revolutions that might happen beyond the Rhine and the Euphrates.
But when the military order had levelled, in wild anarchy, the
power of the prince, the laws of the senate, and even the
discipline of the camp, the barbarians of the North and of the
East, who had long hovered on the frontier, boldly attacked the
provinces of a declining monarchy. Their vexatious inroads were
changed into formidable irruptions, and, after a long vicissitude
of mutual calamities, many tribes of the victorious invaders
established themselves in the provinces of the Roman Empire. To
obtain a clearer knowledge of these great events, we shall
endeavor to form a previous idea of the character, forces, and
designs of those nations who avenged the cause of Hannibal and
Mithridates.

In the more early ages of the world, whilst the forest that
covered Europe afforded a retreat to a few wandering savages, the
inhabitants of Asia were already collected into populous cities,
and reduced under extensive empires the seat of the arts, of
luxury, and of despotism. The Assyrians reigned over the East,\footnotemark[1]
till the sceptre of Ninus and Semiramis dropped from the hands of
their enervated successors. The Medes and the Babylonians divided
their power, and were themselves swallowed up in the monarchy of
the Persians, whose arms could not be confined within the narrow
limits of Asia. Followed, as it is said, by two millions of
\textit{men}, Xerxes, the descendant of Cyrus, invaded Greece.

Thirty thousand \textit{soldiers}, under the command of Alexander, the
son of Philip, who was intrusted by the Greeks with their glory
and revenge, were sufficient to subdue Persia. The princes of the
house of Seleucus usurped and lost the Macedonian command over
the East. About the same time, that, by an ignominious treaty,
they resigned to the Romans the country on this side Mount Tarus,
they were driven by the Parthians,\footnotemark[1001] an obscure horde of
Scythian origin, from all the provinces of Upper Asia. The
formidable power of the Parthians, which spread from India to the
frontiers of Syria, was in its turn subverted by Ardshir, or
Artaxerxes; the founder of a new dynasty, which, under the name
of Sassanides, governed Persia till the invasion of the Arabs.
This great revolution, whose fatal influence was soon experienced
by the Romans, happened in the fourth year of Alexander Severus,
two hundred and twenty-six years after the Christian era.\footnotemark[2] \footnotemark[201]

\footnotetext[1]{An ancient chronologist, quoted by Valleius
Paterculus, (l. i. c. 6,) observes, that the Assyrians, the
Medes, the Persians, and the Macedonians, reigned over Asia one
thousand nine hundred and ninety-five years, from the accession
of Ninus to the defeat of Antiochus by the Romans. As the latter
of these great events happened 289 years before Christ, the
former may be placed 2184 years before the same æra. The
Astronomical Observations, found at Babylon, by Alexander, went
fifty years higher.}

\footnotetext[1001]{The Parthians were a tribe of the Indo-Germanic
branch which dwelt on the south-east of the Caspian, and belonged
to the same race as the Getæ, the Massagetæ, and other nations,
confounded by the ancients under the vague denomination of
Scythians. Klaproth, Tableaux Hist. d l’Asie, p. 40. Strabo (p.
747) calls the Parthians Carduchi, i.e., the inhabitants of
Curdistan.—M.}

\footnotetext[2]{In the five hundred and thirty-eighth year of the
æra of Seleucus. See Agathias, l. ii. p. 63. This great event
(such is the carelessness of the Orientals) is placed by
Eutychius as high as the tenth year of Commodus, and by Moses of
Chorene as low as the reign of Philip. Ammianus Marcellinus has
so servilely copied (xxiii. 6) his ancient materials, which are
indeed very good, that he describes the family of the Arsacides
as still seated on the Persian throne in the middle of the fourth
century.}

\footnotetext[201]{The Persian History, if the poetry of the Shah
Nameh, the Book of Kings, may deserve that name mentions four
dynasties from the earliest ages to the invasion of the Saracens.
The Shah Nameh was composed with the view of perpetuating the
remains of the original Persian records or traditions which had
survived the Saracenic invasion. The task was undertaken by the
poet Dukiki, and afterwards, under the patronage of Mahmood of
Ghazni, completed by Ferdusi. The first of these dynasties is
that of Kaiomors, as Sir W. Jones observes, the dark and fabulous
period; the second, that of the Kaianian, the heroic and
poetical, in which the earned have discovered some curious, and
imagined some fanciful, analogies with the Jewish, the Greek, and
the Roman accounts of the eastern world. See, on the Shah Nameh,
Translation by Goerres, with Von Hammer’s Review, Vienna Jahrbuch
von Lit. 17, 75, 77. Malcolm’s Persia, 8vo. ed. i. 503. Macan’s
Preface to his Critical Edition of the Shah Nameh. On the early
Persian History, a very sensible abstract of various opinions in
Malcolm’s Hist. of Persian.—M.}

Artaxerxes had served with great reputation in the armies of
Artaban, the last king of the Parthians, and it appears that he
was driven into exile and rebellion by royal ingratitude, the
customary reward for superior merit. His birth was obscure, and
the obscurity equally gave room to the aspersions of his enemies,
and the flattery of his adherents. If we credit the scandal of
the former, Artaxerxes sprang from the illegitimate commerce of a
tanner’s wife with a common soldier.\footnotemark[3] The latter represent him
as descended from a branch of the ancient kings of Persian,
though time and misfortune had gradually reduced his ancestors to
the humble station of private citizens.\footnotemark[4] As the lineal heir of
the monarchy, he asserted his right to the throne, and challenged
the noble task of delivering the Persians from the oppression
under which they groaned above five centuries since the death of
Darius. The Parthians were defeated in three great battles.\footnotemark[401]
In the last of these their king Artaban was slain, and the spirit
of the nation was forever broken.\footnotemark[5] The authority of Artaxerxes
was solemnly acknowledged in a great assembly held at Balch in
Khorasan.\footnotemark[501] Two younger branches of the royal house of Arsaces
were confounded among the prostrate satraps. A third, more
mindful of ancient grandeur than of present necessity, attempted
to retire, with a numerous train of vessels, towards their
kinsman, the king of Armenia; but this little army of deserters
was intercepted, and cut off, by the vigilance of the conqueror,\footnotemark[6]
who boldly assumed the double diadem, and the title of King of
Kings, which had been enjoyed by his predecessor. But these
pompous titles, instead of gratifying the vanity of the Persian,
served only to admonish him of his duty, and to inflame in his
soul the ambition of restoring in their full splendor, the
religion and empire of Cyrus.

\footnotetext[3]{The tanner’s name was Babec; the soldier’s, Sassan:
from the former Artaxerxes obtained the surname of Babegan, from
the latter all his descendants have been styled Sassanides.}

\footnotetext[4]{D’Herbelot, Bibliotheque Orientale, Ardshir.}

\footnotetext[401]{In the plain of Hoormuz, the son of Babek was
hailed in the field with the proud title of Shahan Shah, king of
kings—a name ever since assumed by the sovereigns of Persia.
Malcolm, i. 71.—M.}

\footnotetext[5]{Dion Cassius, l. lxxx. Herodian, l. vi. p. 207.
Abulpharagins Dynast. p. 80.}

\footnotetext[501]{See the Persian account of the rise of Ardeschir
Babegan in Malcolm l 69.—M.}

\footnotetext[6]{See Moses Chorenensis, l. ii. c. 65—71.}

I. During the long servitude of Persia under the Macedonian and
the Parthian yoke, the nations of Europe and Asia had mutually
adopted and corrupted each other’s superstitions. The Arsacides,
indeed, practised the worship of the Magi; but they disgraced and
polluted it with a various mixture of foreign idolatry.\footnotemark[601] The
memory of Zoroaster, the ancient prophet and philosopher of the
Persians,\footnotemark[7] was still revered in the East; but the obsolete and
mysterious language, in which the Zendavesta was composed,\footnotemark[8]
opened a field of dispute to seventy sects, who variously
explained the fundamental doctrines of their religion, and were
all indifferently devided by a crowd of infidels, who rejected
the divine mission and miracles of the prophet. To suppress the
idolaters, reunite the schismatics, and confute the unbelievers,
by the infallible decision of a general council, the pious
Artaxerxes summoned the Magi from all parts of his dominions.
These priests, who had so long sighed in contempt and obscurity
obeyed the welcome summons; and, on the appointed day, appeared,
to the number of about eighty thousand. But as the debates of so
tumultuous an assembly could not have been directed by the
authority of reason, or influenced by the art of policy, the
Persian synod was reduced, by successive operations, to forty
thousand, to four thousand, to four hundred, to forty, and at
last to seven Magi, the most respected for their learning and
piety. One of these, Erdaviraph, a young but holy prelate,
received from the hands of his brethren three cups of
soporiferous wine. He drank them off, and instantly fell into a
long and profound sleep. As soon as he waked, he related to the
king and to the believing multitude, his journey to heaven, and
his intimate conferences with the Deity. Every doubt was silenced
by this supernatural evidence; and the articles of the faith of
Zoroaster were fixed with equal authority and precision.\footnotemark[9] A
short delineation of that celebrated system will be found useful,
not only to display the character of the Persian nation, but to
illustrate many of their most important transactions, both in
peace and war, with the Roman empire.\footnotemark[10]

\footnotetext[601]{Silvestre de Sacy (Antiquites de la Perse) had
proved the neglect of the Zoroastrian religion under the Parthian
kings.—M.}

\footnotetext[7]{Hyde and Prideaux, working up the Persian legends
and their own conjectures into a very agreeable story, represent
Zoroaster as a contemporary of Darius Hystaspes. But it is
sufficient to observe, that the Greek writers, who lived almost
in the age of Darius, agree in placing the æra of Zoroaster many
hundred, or even thousand, years before their own time. The
judicious criticisms of Mr. Moyle perceived, and maintained
against his uncle, Dr. Prideaux, the antiquity of the Persian
prophet. See his work, vol. ii. * Note: There are three leading
theories concerning the age of Zoroaster: 1. That which assigns
him to an age of great and almost indefinite antiquity—it is that
of Moyle, adopted by Gibbon, Volney, Recherches sur l’Histoire,
ii. 2. Rhode, also, (die Heilige Sage, \&c.,) in a very ingenious
and ably-developed theory, throws the Bactrian prophet far back
into antiquity 2. Foucher, (Mem. de l’Acad. xxvii. 253,) Tychsen,
(in Com. Soc. Gott. ii. 112), Heeren, (ldeen. i. 459,) and
recently Holty, identify the Gushtasp of the Persian mythological
history with Cyaxares the First, the king of the Medes, and
consider the religion to be Median in its origin. M. Guizot
considers this opinion most probable, note in loc. 3. Hyde,
Prideaux, Anquetil du Perron, Kleuker, Herder, Goerres,
(Mythen-Geschichte,) Von Hammer. (Wien. Jahrbuch, vol. ix.,)
Malcolm, (i. 528,) De Guigniaut, (Relig. de l’Antiq. 2d part,
vol. iii.,) Klaproth, (Tableaux de l’Asie, p. 21,) make Gushtasp
Darius Hystaspes, and Zoroaster his contemporary. The silence of
Herodotus appears the great objection to this theory. Some
writers, as M. Foucher (resting, as M. Guizot observes, on the
doubtful authority of Pliny,) make more than one Zoroaster, and
so attempt to reconcile the conflicting theories.— M.}

\footnotetext[8]{That ancient idiom was called the Zend. The language
of the commentary, the Pehlvi, though much more modern, has
ceased many ages ago to be a living tongue. This fact alone (if
it is allowed as authentic) sufficiently warrants the antiquity
of those writings which M d’Anquetil has brought into Europe, and
translated into French. * Note: Zend signifies life, living. The
word means, either the collection of the canonical books of the
followers of Zoroaster, or the language itself in which they are
written. They are the books that contain the word of life whether
the language was originally called Zend, or whether it was so
called from the contents of the books. Avesta means word, oracle,
revelation: this term is not the title of a particular work, but
of the collection of the books of Zoroaster, as the revelation of
Ormuzd. This collection is sometimes called Zendavesta, sometimes
briefly Zend. The Zend was the ancient language of Media, as is
proved by its affinity with the dialects of Armenia and Georgia;
it was already a dead language under the Arsacides in the country
which was the scene of the events recorded in the Zendavesta.
Some critics, among others Richardson and Sir W. Jones, have
called in question the antiquity of these books. The former
pretended that Zend had never been a written or spoken language,
but had been invented in the later times by the Magi, for the
purposes of their art; but Kleuker, in the dissertations which he
added to those of Anquetil and the Abbé Foucher, has proved that
the Zend was a living and spoken language.—G. Sir W. Jones
appears to have abandoned his doubts, on discovering the affinity
between the Zend and the Sanskrit. Since the time of Kleuker,
this question has been investigated by many learned scholars. Sir
W. Jones, Leyden, (Asiat. Research. x. 283,) and Mr. Erskine,
(Bombay Trans. ii. 299,) consider it a derivative from the
Sanskrit. The antiquity of the Zendavesta has likewise been
asserted by Rask, the great Danish linguist, who, according to
Malcolm, brought back from the East fresh transcripts and
additions to those published by Anquetil. According to Rask, the
Zend and Sanskrit are sister dialects; the one the parent of the
Persian, the other of the Indian family of languages.—G. and
M.——But the subject is more satisfactorily illustrated in Bopp’s
comparative Grammar of the Sanscrit, Zend, Greek, Latin,
Lithuanian, Gothic, and German languages. Berlin. 1833-5.
According to Bopp, the Zend is, in some respects, of a more
remarkable structure than the Sanskrit. Parts of the Zendavesta
have been published in the original, by M. Bournouf, at Paris,
and M. Ol. shausen, in Hamburg.—M.——The Pehlvi was the language
of the countries bordering on Assyria, and probably of Assyria
itself. Pehlvi signifies valor, heroism; the Pehlvi, therefore,
was the language of the ancient heroes and kings of Persia, the
valiant. (Mr. Erskine prefers the derivation from Pehla, a
border.—M.) It contains a number of Aramaic roots. Anquetil
considered it formed from the Zend. Kleuker does not adopt this
opinion. The Pehlvi, he says, is much more flowing, and less
overcharged with vowels, than the Zend. The books of Zoroaster,
first written in Zend, were afterwards translated into Pehlvi and
Parsi. The Pehlvi had fallen into disuse under the dynasty of the
Sassanides, but the learned still wrote it. The Parsi, the
dialect of Pars or Farristan, was then prevailing dialect.
Kleuker, Anhang zum Zend Avesta, 2, ii. part i. p. 158, part ii.
31.—G.——Mr. Erskine (Bombay Transactions) considers the existing
Zendavesta to have been compiled in the time of Ardeschir
Babegan.—M.}

\footnotetext[9]{Hyde de Religione veterum Pers. c. 21.}

\footnotetext[10]{I have principally drawn this account from the
Zendavesta of M. d’Anquetil, and the Sadder, subjoined to Dr.
Hyde’s treatise. It must, however, be confessed, that the studied
obscurity of a prophet, the figurative style of the East, and the
deceitful medium of a French or Latin version may have betrayed
us into error and heresy, in this abridgment of Persian theology.
* Note: It is to be regretted that Gibbon followed the
post-Mahometan Sadder of Hyde.—M.}

The great and fundamental article of the system was the
celebrated doctrine of the two principles; a bold and injudicious
attempt of Eastern philosophy to reconcile the existence of moral
and physical evil with the attributes of a beneficent Creator and
Governor of the world. The first and original Being, in whom, or
by whom, the universe exists, is denominated in the writings of
Zoroaster, \textit{Time without bounds};\footnotemark[1001] but it must be confessed,
that this infinite substance seems rather a metaphysical
abstraction of the mind than a real object endowed with
self-consciousness, or possessed of moral perfections. From
either the blind or the intelligent operation of this infinite
Time, which bears but too near an affinity with the chaos of the
Greeks, the two secondary but active principles of the universe
were from all eternity produced, Ormusd and Ahriman, each of them
possessed of the powers of creation, but each disposed, by his
invariable nature, to exercise them with different designs.\footnotemark[1002]
The principle of good is eternally aborbed in light; the
principle of evil eternally buried in darkness. The wise
benevolence of Ormusd formed man capable of virtue, and
abundantly provided his fair habitation with the materials of
happiness. By his vigilant providence, the motion of the planets,
the order of the seasons, and the temperate mixture of the
elements, are preserved. But the malice of Ahriman has long since
pierced \textit{Ormusd’s egg;} or, in other words, has violated the
harmony of his works. Since that fatal eruption, the most minute
articles of good and evil are intimately intermingled and
agitated together; the rankest poisons spring up amidst the most
salutary plants; deluges, earthquakes, and conflagrations attest
the conflict of Nature, and the little world of man is
perpetually shaken by vice and misfortune. Whilst the rest of
human kind are led away captives in the chains of their infernal
enemy, the faithful Persian alone reserves his religious
adoration for his friend and protector Ormusd, and fights under
his banner of light, in the full confidence that he shall, in the
last day, share the glory of his triumph. At that decisive
period, the enlightened wisdom of goodness will render the power
of Ormusd superior to the furious malice of his rival. Ahriman
and his followers, disarmed and subdued, will sink into their
native darkness; and virtue will maintain the eternal peace and
harmony of the universe.\footnotemark[11] \footnotemark[1101]

\footnotetext[1001]{Zeruane Akerene, so translated by Anquetil and
Kleuker. There is a dissertation of Foucher on this subject, Mem.
de l’Acad. des Inscr. t. xxix. According to Bohlen (das alte
Indien) it is the Sanskrit Sarvan Akaranam, the Uncreated Whole;
or, according to Fred. Schlegel, Sarvan Akharyam the Uncreate
Indivisible.—M.}

\footnotetext[1002]{This is an error. Ahriman was not forced by his
invariable nature to do evil; the Zendavesta expressly recognizes
(see the Izeschne) that he was born good, that in his origin he
was light; envy rendered him evil; he became jealous of the power
and attributes of Ormuzd; then light was changed into darkness,
and Ahriman was precipitated into the abyss. See the Abridgment
of the Doctrine of the Ancient Persians, by Anquetil, c. ii
Section 2.—G.}

\footnotetext[11]{The modern Parsees (and in some degree the Sadder)
exalt Ormusd into the first and omnipotent cause, whilst they
degrade Ahriman into an inferior but rebellious spirit. Their
desire of pleasing the Mahometans may have contributed to refine
their theological systems.}

\footnotetext[1101]{According to the Zendavesta, Ahriman will not be
annihilated or precipitated forever into darkness: at the
resurrection of the dead he will be entirely defeated by Ormuzd,
his power will be destroyed, his kingdom overthrown to its
foundations, he will himself be purified in torrents of melting
metal; he will change his heart and his will, become holy,
heavenly establish in his dominions the law and word of Ormuzd,
unite himself with him in everlasting friendship, and both will
sing hymns in honor of the Great Eternal. See Anquetil’s
Abridgment. Kleuker, Anhang part iii. p 85, 36; and the Izeschne,
one of the books of the Zendavesta. According to the Sadder
Bun-Dehesch, a more modern work, Ahriman is to be annihilated:
but this is contrary to the text itself of the Zendavesta, and to
the idea its author gives of the kingdom of Eternity, after the
twelve thousand years assigned to the contest between Good and
Evil.—G.}

