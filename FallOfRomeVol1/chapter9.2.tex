\section{Part \thesection.}
\thispagestyle{simple}

There is not anywhere upon the globe a large tract of country,
which we have discovered destitute of inhabitants, or whose first
population can be fixed with any degree of historical certainty.
And yet, as the most philosophic minds can seldom refrain from
investigating the infancy of great nations, our curiosity
consumes itself in toilsome and disappointed efforts. When
Tacitus considered the purity of the German blood, and the
forbidding aspect of the country, he was disposed to pronounce
those barbarians \textit{Indigenæ}, or natives of the soil. We may allow
with safety, and perhaps with truth, that ancient Germany was not
originally peopled by any foreign colonies already formed into a
political society;\footnotemark[12] but that the name and nation received their
existence from the gradual union of some wandering savages of the
Hercynian woods. To assert those savages to have been the
spontaneous production of the earth which they inhabited would be
a rash inference, condemned by religion, and unwarranted by
reason.

\footnotetext[12]{Facit. Germ. c. 3. The emigration of the Gauls
followed the course of the Danube, and discharged itself on
Greece and Asia. Tacitus could discover only one inconsiderable
tribe that retained any traces of a Gallic origin. * Note: The
Gothini, who must not be confounded with the Gothi, a Suevian
tribe. In the time of Cæsar many other tribes of Gaulish origin
dwelt along the course of the Danube, who could not long resist
the attacks of the Suevi. The Helvetians, who dwelt on the
borders of the Black Forest, between the Maine and the Danube,
had been expelled long before the time of Cæsar. He mentions also
the Volci Tectosagi, who came from Languedoc and settled round
the Black Forest. The Boii, who had penetrated into that forest,
and also have left traces of their name in Bohemia, were subdued
in the first century by the Marcomanni. The Boii settled in
Noricum, were mingled afterwards with the Lombards, and received
the name of Boio Arii (Bavaria) or Boiovarii: var, in some German
dialects, appearing to mean remains, descendants. Compare Malte
B-m, Geography, vol. i. p. 410, edit 1832—M.}

Such rational doubt is but ill suited with the genius of popular
vanity. Among the nations who have adopted the Mosaic history of
the world, the ark of Noah has been of the same use, as was
formerly to the Greeks and Romans the siege of Troy. On a narrow
basis of acknowledged truth, an immense but rude superstructure
of fable has been erected; and the wild Irishman,\footnotemark[13] as well as
the wild Tartar,\footnotemark[14] could point out the individual son of Japhet,
from whose loins his ancestors were lineally descended. The last
century abounded with antiquarians of profound learning and easy
faith, who, by the dim light of legends and traditions, of
conjectures and etymologies, conducted the great grandchildren of
Noah from the Tower of Babel to the extremities of the globe. Of
these judicious critics, one of the most entertaining was Olaus
Rudbeck, professor in the university of Upsal.\footnotemark[15] Whatever is
celebrated either in history or fable this zealous patriot
ascribes to his country. From Sweden (which formed so
considerable a part of ancient Germany) the Greeks themselves
derived their alphabetical characters, their astronomy, and their
religion. Of that delightful region (for such it appeared to the
eyes of a native) the Atlantis of Plato, the country of the
Hyperboreans, the gardens of the Hesperides, the Fortunate
Islands, and even the Elysian Fields, were all but faint and
imperfect transcripts. A clime so profusely favored by Nature
could not long remain desert after the flood. The learned Rudbeck
allows the family of Noah a few years to multiply from eight to
about twenty thousand persons. He then disperses them into small
colonies to replenish the earth, and to propagate the human
species. The German or Swedish detachment (which marched, if I am
not mistaken, under the command of Askenaz, the son of Gomer, the
son of Japhet) distinguished itself by a more than common
diligence in the prosecution of this great work. The northern
hive cast its swarms over the greatest part of Europe, Africa,
and Asia; and (to use the author’s metaphor) the blood circulated
from the extremities to the heart.

\footnotetext[13]{According to Dr. Keating, (History of Ireland, p.
13, 14,) the giant Portholanus, who was the son of Seara, the son
of Esra, the son of Sru, the son of Framant, the son of
Fathaclan, the son of Magog, the son of Japhet, the son of Noah,
landed on the coast of Munster the 14th day of May, in the year
of the world one thousand nine hundred and seventy-eight. Though
he succeeded in his great enterprise, the loose behavior of his
wife rendered his domestic life very unhappy, and provoked him to
such a degree, that he killed—her favorite greyhound. This, as
the learned historian very properly observes, was the first
instance of female falsehood and infidelity ever known in
Ireland.}

\footnotetext[14]{Genealogical History of the Tartars, by Abulghazi
Bahadur Khan.}

\footnotetext[15]{His work, entitled Atlantica, is uncommonly scarce.
Bayle has given two most curious extracts from it. Republique des
Lettres Janvier et Fevrier, 1685.}

But all this well-labored system of German antiquities is
annihilated by a single fact, too well attested to admit of any
doubt, and of too decisive a nature to leave room for any reply.
The Germans, in the age of Tacitus, were unacquainted with the
use of letters;\footnotemark[16] and the use of letters is the principal
circumstance that distinguishes a civilized people from a herd of
savages incapable of knowledge or reflection. Without that
artificial help, the human memory soon dissipates or corrupts the
ideas intrusted to her charge; and the nobler faculties of the
mind, no longer supplied with models or with materials, gradually
forget their powers; the judgment becomes feeble and lethargic,
the imagination languid or irregular. Fully to apprehend this
important truth, let us attempt, in an improved society, to
calculate the immense distance between the man of learning and
the \textit{illiterate} peasant. The former, by reading and reflection,
multiplies his own experience, and lives in distant ages and
remote countries; whilst the latter, rooted to a single spot, and
confined to a few years of existence, surpasses but very little
his fellow-laborer, the ox, in the exercise of his mental
faculties. The same, and even a greater, difference will be found
between nations than between individuals; and we may safely
pronounce, that without some species of writing, no people has
ever preserved the faithful annals of their history, ever made
any considerable progress in the abstract sciences, or ever
possessed, in any tolerable degree of perfection, the useful and
agreeable arts of life.

\footnotetext[16]{Tacit. Germ. ii. 19. Literarum secreta viri pariter
ac fœminæ ignorant. We may rest contented with this decisive
authority, without entering into the obscure disputes concerning
the antiquity of the Runic characters. The learned Celsius, a
Swede, a scholar, and a philosopher, was of opinion, that they
were nothing more than the Roman letters, with the curves changed
into straight lines for the ease of engraving. See Pelloutier,
Histoire des Celtes, l. ii. c. 11. Dictionnaire Diplomatique,
tom. i. p. 223. We may add, that the oldest Runic inscriptions
are supposed to be of the third century, and the most ancient
writer who mentions the Runic characters is Venan tius
Frotunatus, (Carm. vii. 18,) who lived towards the end of the
sixth century. Barbara fraxineis pingatur Runa tabellis. * Note:
The obscure subject of the Runic characters has exercised the
industry and ingenuity of the modern scholars of the north. There
are three distinct theories; one, maintained by Schlozer,
(Nordische Geschichte, p. 481, \&c.,) who considers their sixteen
letters to be a corruption of the Roman alphabet, post-Christian
in their date, and Schlozer would attribute their introduction
into the north to the Alemanni. The second, that of Frederick
Schlegel, (Vorlesungen uber alte und neue Literatur,) supposes
that these characters were left on the coasts of the
Mediterranean and Northern Seas by the Phœnicians, preserved by
the priestly castes, and employed for purposes of magic. Their
common origin from the Phœnician would account for heir
similarity to the Roman letters. The last, to which we incline,
claims much higher and more venerable antiquity for the Runic,
and supposes them to have been the original characters of the
Indo-Teutonic tribes, brought from the East, and preserved among
the different races of that stock. See Ueber Deutsche Runen von
W. C. Grimm, 1821. A Memoir by Dr. Legis. Fundgruben des alten
Nordens. Foreign Quarterly Review vol. ix. p. 438.—M.}

Of these arts, the ancient Germans were wretchedly destitute.\footnotemark[1601]
They passed their lives in a state of ignorance and poverty,
which it has pleased some declaimers to dignify with the
appellation of virtuous simplicity. Modern Germany is said to
contain about two thousand three hundred walled towns.\footnotemark[17] In a
much wider extent of country, the geographer Ptolemy could
discover no more than ninety places which he decorates with the
name of cities;\footnotemark[18] though, according to our ideas, they would but
ill deserve that splendid title. We can only suppose them to have
been rude fortifications, constructed in the centre of the woods,
and designed to secure the women, children, and cattle, whilst
the warriors of the tribe marched out to repel a sudden invasion.\footnotemark[19]
But Tacitus asserts, as a well-known fact, that the Germans,
in his time, had \textit{no} cities;\footnotemark[20] and that they affected to
despise the works of Roman industry, as places of confinement
rather than of security.\footnotemark[21] Their edifices were not even
contiguous, or formed into regular villas;\footnotemark[22] each barbarian
fixed his independent dwelling on the spot to which a plain, a
wood, or a stream of fresh water, had induced him to give the
preference. Neither stone, nor brick, nor tiles, were employed in
these slight habitations.\footnotemark[23] They were indeed no more than low
huts, of a circular figure, built of rough timber, thatched with
straw, and pierced at the top to leave a free passage for the
smoke. In the most inclement winter, the hardy German was
satisfied with a scanty garment made of the skin of some animal.
The nations who dwelt towards the North clothed themselves in
furs; and the women manufactured for their own use a coarse kind
of linen.\footnotemark[24] The game of various sorts, with which the forests of
Germany were plentifully stocked, supplied its inhabitants with
food and exercise.\footnotemark[25] Their monstrous herds of cattle, less
remarkable indeed for their beauty than for their utility,\footnotemark[26]
formed the principal object of their wealth. A small quantity of
corn was the only produce exacted from the earth; the use of
orchards or artificial meadows was unknown to the Germans; nor
can we expect any improvements in agriculture from a people,
whose prosperity every year experienced a general change by a new
division of the arable lands, and who, in that strange operation,
avoided disputes, by suffering a great part of their territory to
lie waste and without tillage.\footnotemark[27]

\footnotetext[1601]{Luden (the author of the Geschichte des Teutschen
Volkes) has surpassed most writers in his patriotic enthusiasm
for the virtues and noble manners of his ancestors. Even the cold
of the climate, and the want of vines and fruit trees, as well as
the barbarism of the inhabitants, are calumnies of the luxurious
Italians. M. Guizot, on the other side, (in his Histoire de la
Civilisation, vol. i. p. 272, \&c.,) has drawn a curious parallel
between the Germans of Tacitus and the North American
Indians.—M.}

\footnotetext[17]{Recherches Philosophiques sur les Americains, tom.
iii. p. 228. The author of that very curious work is, if I am not
misinformed, a German by birth. (De Pauw.)}

\footnotetext[18]{The Alexandrian Geographer is often criticized by
the accurate Cluverius.}

\footnotetext[19]{See Cæsar, and the learned Mr. Whitaker in his
History of Manchester, vol. i.}

\footnotetext[20]{Tacit. Germ. 15.}

\footnotetext[21]{When the Germans commanded the Ubii of Cologne to
cast off the Roman yoke, and with their new freedom to resume
their ancient manners, they insisted on the immediate demolition
of the walls of the colony. “Postulamus a vobis, muros coloniæ,
munimenta servitii, detrahatis; etiam fera animalia, si clausa
teneas, virtutis obliviscuntur.” Tacit. Hist. iv. 64.}

\footnotetext[22]{The straggling villages of Silesia are several
miles in length. See Cluver. l. i. c. 13.}

\footnotetext[23]{One hundred and forty years after Tacitus, a few
more regular structures were erected near the Rhine and Danube.
Herodian, l. vii. p. 234.}

\footnotetext[24]{Tacit. Germ. 17.}

\footnotetext[25]{Tacit. Germ. 5.}

\footnotetext[26]{Cæsar de Bell. Gall. vi. 21.}

\footnotetext[27]{Tacit. Germ. 26. Cæsar, vi. 22.}

Gold, silver, and iron, were extremely scarce in Germany. Its
barbarous inhabitants wanted both skill and patience to
investigate those rich veins of silver, which have so liberally
rewarded the attention of the princes of Brunswick and Saxony.
Sweden, which now supplies Europe with iron, was equally ignorant
of its own riches; and the appearance of the arms of the Germans
furnished a sufficient proof how little iron they were able to
bestow on what they must have deemed the noblest use of that
metal. The various transactions of peace and war had introduced
some Roman coins (chiefly silver) among the borderers of the
Rhine and Danube; but the more distant tribes were absolutely
unacquainted with the use of money, carried on their confined
traffic by the exchange of commodities, and prized their rude
earthen vessels as of equal value with the silver vases, the
presents of Rome to their princes and ambassadors.\footnotemark[28] To a mind
capable of reflection, such leading facts convey more
instruction, than a tedious detail of subordinate circumstances.
The value of money has been settled by general consent to express
our wants and our property, as letters were invented to express
our ideas; and both these institutions, by giving a more active
energy to the powers and passions of human nature, have
contributed to multiply the objects they were designed to
represent. The use of gold and silver is in a great measure
factitious; but it would be impossible to enumerate the important
and various services which agriculture, and all the arts, have
received from iron, when tempered and fashioned by the operation
of fire and the dexterous hand of man. Money, in a word, is the
most universal incitement, iron the most powerful instrument, of
human industry; and it is very difficult to conceive by what
means a people, neither actuated by the one, nor seconded by the
other, could emerge from the grossest barbarism.\footnotemark[29]

\footnotetext[28]{Tacit. Germ. 6.}

\footnotetext[29]{It is said that the Mexicans and Peruvians, without
the use of either money or iron, had made a very great progress
in the arts. Those arts, and the monuments they produced, have
been strangely magnified. See Recherches sur les Americains, tom.
ii. p. 153, \&c}

If we contemplate a savage nation in any part of the globe, a
supine indolence and a carelessness of futurity will be found to
constitute their general character. In a civilized state every
faculty of man is expanded and exercised; and the great chain of
mutual dependence connects and embraces the several members of
society. The most numerous portion of it is employed in constant
and useful labor. The select few, placed by fortune above that
necessity, can, however, fill up their time by the pursuits of
interest or glory, by the improvement of their estate or of their
understanding, by the duties, the pleasures, and even the follies
of social life. The Germans were not possessed of these varied
resources. The care of the house and family, the management of
the land and cattle, were delegated to the old and the infirm, to
women and slaves. The lazy warrior, destitute of every art that
might employ his leisure hours, consumed his days and nights in
the animal gratifications of sleep and food. And yet, by a
wonderful diversity of nature, (according to the remark of a
writer who had pierced into its darkest recesses,) the same
barbarians are by turns the most indolent and the most restless
of mankind. They delight in sloth, they detest tranquility.\footnotemark[30]
The languid soul, oppressed with its own weight, anxiously
required some new and powerful sensation; and war and danger were
the only amusements adequate to its fierce temper. The sound that
summoned the German to arms was grateful to his ear. It roused
him from his uncomfortable lethargy, gave him an active pursuit,
and, by strong exercise of the body, and violent emotions of the
mind, restored him to a more lively sense of his existence. In
the dull intervals of peace, these barbarians were immoderately
addicted to deep gaming and excessive drinking; both of which, by
different means, the one by inflaming their passions, the other
by extinguishing their reason, alike relieved them from the pain
of thinking. They gloried in passing whole days and nights at
table; and the blood of friends and relations often stained their
numerous and drunken assemblies.\footnotemark[31] Their debts of honor (for in
that light they have transmitted to us those of play) they
discharged with the most romantic fidelity. The desperate
gamester, who had staked his person and liberty on a last throw
of the dice, patiently submitted to the decision of fortune, and
suffered himself to be bound, chastised, and sold into remote
slavery, by his weaker but more lucky antagonist.\footnotemark[32]

\footnotetext[30]{Tacit. Germ. 15.}

\footnotetext[31]{Tacit. Germ. 22, 23.}

\footnotetext[32]{Id. 24. The Germans might borrow the arts of play
from the Romans, but the passion is wonderfully inherent in the
human species.}

Strong beer, a liquor extracted with very little art from wheat
or barley, and \textit{corrupted} (as it is strongly expressed by
Tacitus) into a certain semblance of wine, was sufficient for the
gross purposes of German debauchery. But those who had tasted the
rich wines of Italy, and afterwards of Gaul, sighed for that more
delicious species of intoxication. They attempted not, however,
(as has since been executed with so much success,) to naturalize
the vine on the banks of the Rhine and Danube; nor did they
endeavor to procure by industry the materials of an advantageous
commerce. To solicit by labor what might be ravished by arms, was
esteemed unworthy of the German spirit.\footnotemark[33] The intemperate thirst
of strong liquors often urged the barbarians to invade the
provinces on which art or nature had bestowed those much envied
presents. The Tuscan who betrayed his country to the Celtic
nations, attracted them into Italy by the prospect of the rich
fruits and delicious wines, the productions of a happier climate.\footnotemark[34]
And in the same manner the German auxiliaries, invited into
France during the civil wars of the sixteenth century, were
allured by the promise of plenteous quarters in the provinces of
Champaigne and Burgundy.\footnotemark[35] Drunkenness, the most illiberal, but
not the most dangerous of \textit{our} vices, was sometimes capable, in
a less civilized state of mankind, of occasioning a battle, a
war, or a revolution.

\footnotetext[33]{Tacit. Germ. 14.}

\footnotetext[34]{Plutarch. in Camillo. T. Liv. v. 33.}

\footnotetext[35]{Dubos. Hist. de la Monarchie Francoise, tom. i. p.
193.}

The climate of ancient Germany has been modified, and the soil
fertilized, by the labor of ten centuries from the time of
Charlemagne. The same extent of ground which at present
maintains, in ease and plenty, a million of husbandmen and
artificers, was unable to supply a hundred thousand lazy warriors
with the simple necessaries of life.\footnotemark[36] The Germans abandoned
their immense forests to the exercise of hunting, employed in
pasturage the most considerable part of their lands, bestowed on
the small remainder a rude and careless cultivation, and then
accused the scantiness and sterility of a country that refused to
maintain the multitude of its inhabitants. When the return of
famine severely admonished them of the importance of the arts,
the national distress was sometimes alleviated by the emigration
of a third, perhaps, or a fourth part of their youth.\footnotemark[37] The
possession and the enjoyment of property are the pledges which
bind a civilized people to an improved country. But the Germans,
who carried with them what they most valued, their arms, their
cattle, and their women, cheerfully abandoned the vast silence of
their woods for the unbounded hopes of plunder and conquest. The
innumerable swarms that issued, or seemed to issue, from the
great storehouse of nations, were multiplied by the fears of the
vanquished, and by the credulity of succeeding ages. And from
facts thus exaggerated, an opinion was gradually established, and
has been supported by writers of distinguished reputation, that,
in the age of Cæsar and Tacitus, the inhabitants of the North
were far more numerous than they are in our days.\footnotemark[38] A more
serious inquiry into the causes of population seems to have
convinced modern philosophers of the falsehood, and indeed the
impossibility, of the supposition. To the names of Mariana and of
Machiavel,\footnotemark[39] we can oppose the equal names of Robertson and
Hume.\footnotemark[40]

\footnotetext[36]{The Helvetian nation, which issued from a country
called Switzerland, contained, of every age and sex, 368,000
persons, (Cæsar de Bell. Gal. i. 29.) At present, the number of
people in the Pays de Vaud (a small district on the banks of the
Leman Lake, much more distinguished for politeness than for
industry) amounts to 112,591. See an excellent tract of M. Muret,
in the Memoires de la Societe de Born.}

\footnotetext[37]{Paul Diaconus, c. 1, 2, 3. Machiavel, Davila, and
the rest of Paul’s followers, represent these emigrations too
much as regular and concerted measures.}

\footnotetext[38]{Sir William Temple and Montesquieu have indulged,
on this subject, the usual liveliness of their fancy.}

\footnotetext[39]{Machiavel, Hist. di Firenze, l. i. Mariana, Hist.
Hispan. l. v. c. 1}

\footnotetext[40]{Robertson’s Charles V. Hume’s Political Essays.
Note: It is a wise observation of Malthus, that these nations
“were not populous in proportion to the land they occupied, but
to the food they produced.” They were prolific from their pure
morals and constitutions, but their institutions were not
calculated to produce food for those whom they brought into
being.—M—1845.}

A warlike nation like the Germans, without either cities,
letters, arts, or money, found some compensation for this savage
state in the enjoyment of liberty. Their poverty secured their
freedom, since our desires and our possessions are the strongest
fetters of despotism. “Among the Suiones (says Tacitus) riches
are held in honor. They are \textit{therefore} subject to an absolute
monarch, who, instead of intrusting his people with the free use
of arms, as is practised in the rest of Germany, commits them to
the safe custody, not of a citizen, or even of a freedman, but of
a slave. The neighbors of the Suiones, the Sitones, are sunk even
below servitude; they obey a woman.”\footnotemark[41] In the mention of these
exceptions, the great historian sufficiently acknowledges the
general theory of government. We are only at a loss to conceive
by what means riches and despotism could penetrate into a remote
corner of the North, and extinguish the generous flame that
blazed with such fierceness on the frontier of the Roman
provinces, or how the ancestors of those Danes and Norwegians, so
distinguished in latter ages by their unconquered spirit, could
thus tamely resign the great character of German liberty.\footnotemark[42] Some
tribes, however, on the coast of the Baltic, acknowledged the
authority of kings, though without relinquishing the rights of
men,\footnotemark[43] but in the far greater part of Germany, the form of
government was a democracy, tempered, indeed, and controlled, not
so much by general and positive laws, as by the occasional
ascendant of birth or valor, of eloquence or superstition.\footnotemark[44]

\footnotetext[41]{Tacit. German. 44, 45. Freinshemius (who dedicated
his supplement to Livy to Christina of Sweden) thinks proper to
be very angry with the Roman who expressed so very little
reverence for Northern queens. Note: The Suiones and the Sitones
are the ancient inhabitants of Scandinavia, their name may be
traced in that of Sweden; they did not belong to the race of the
Suevi, but that of the non-Suevi or Cimbri, whom the Suevi, in
very remote times, drove back part to the west, part to the
north; they were afterwards mingled with Suevian tribes, among
others the Goths, who have traces of their name and power in the
isle of Gothland.—G}

\footnotetext[42]{May we not suspect that superstition was the parent
of despotism? The descendants of Odin, (whose race was not
extinct till the year 1060) are said to have reigned in Sweden
above a thousand years. The temple of Upsal was the ancient seat
of religion and empire. In the year 1153 I find a singular law,
prohibiting the use and profession of arms to any except the
king’s guards. Is it not probable that it was colored by the
pretence of reviving an old institution? See Dalin’s History of
Sweden in the Bibliotheque Raisonneo tom. xl. and xlv.}

\footnotetext[43]{Tacit. Germ. c. 43.}

\footnotetext[44]{Id. c. 11, 12, 13, \& c.}

Civil governments, in their first institution, are voluntary
associations for mutual defence. To obtain the desired end, it is
absolutely necessary that each individual should conceive himself
obliged to submit his private opinions and actions to the
judgment of the greater number of his associates. The German
tribes were contented with this rude but liberal outline of
political society. As soon as a youth, born of free parents, had
attained the age of manhood, he was introduced into the general
council of his countrymen, solemnly invested with a shield and
spear, and adopted as an equal and worthy member of the military
commonwealth. The assembly of the warriors of the tribe was
convened at stated seasons, or on sudden emergencies. The trial
of public offences, the election of magistrates, and the great
business of peace and war, were determined by its independent
voice. Sometimes indeed, these important questions were
previously considered and prepared in a more select council of
the principal chieftains.\footnotemark[45] The magistrates might deliberate and
persuade, the people only could resolve and execute; and the
resolutions of the Germans were for the most part hasty and
violent. Barbarians accustomed to place their freedom in
gratifying the present passion, and their courage in overlooking
all future consequences, turned away with indignant contempt from
the remonstrances of justice and policy, and it was the practice
to signify by a hollow murmur their dislike of such timid
counsels. But whenever a more popular orator proposed to
vindicate the meanest citizen from either foreign or domestic
injury, whenever he called upon his fellow-countrymen to assert
the national honor, or to pursue some enterprise full of danger
and glory, a loud clashing of shields and spears expressed the
eager applause of the assembly. For the Germans always met in
arms, and it was constantly to be dreaded, lest an irregular
multitude, inflamed with faction and strong liquors, should use
those arms to enforce, as well as to declare, their furious
resolves. We may recollect how often the diets of Poland have
been polluted with blood, and the more numerous party has been
compelled to yield to the more violent and seditious.\footnotemark[46]

\footnotetext[45]{Grotius changes an expression of Tacitus,
pertractantur into Prætractantur. The correction is equally just
and ingenious.}

\footnotetext[46]{Even in our ancient parliament, the barons often
carried a question, not so much by the number of votes, as by
that of their armed followers.}

A general of the tribe was elected on occasions of danger; and,
if the danger was pressing and extensive, several tribes
concurred in the choice of the same general. The bravest warrior
was named to lead his countrymen into the field, by his example
rather than by his commands. But this power, however limited, was
still invidious. It expired with the war, and in time of peace
the German tribes acknowledged not any supreme chief.\footnotemark[47]
\textit{Princes} were, however, appointed, in the general assembly, to
administer justice, or rather to compose differences,\footnotemark[48] in their
respective districts. In the choice of these magistrates, as much
regard was shown to birth as to merit.\footnotemark[49] To each was assigned,
by the public, a guard, and a council of a hundred persons, and
the first of the princes appears to have enjoyed a preeminence of
rank and honor which sometimes tempted the Romans to compliment
him with the regal title.\footnotemark[50]

\footnotetext[47]{Cæsar de Bell. Gal. vi. 23.}

\footnotetext[48]{Minuunt controversias, is a very happy expression
of Cæsar’s.}

\footnotetext[49]{Reges ex nobilitate, duces ex virtute sumunt. Tacit
Germ. 7}

\footnotetext[50]{Cluver. Germ. Ant. l. i. c. 38.}

The comparative view of the powers of the magistrates, in two
remarkable instances, is alone sufficient to represent the whole
system of German manners. The disposal of the landed property
within their district was absolutely vested in their hands, and
they distributed it every year according to a new division.\footnotemark[51] At
the same time they were not authorized to punish with death, to
imprison, or even to strike a private citizen.\footnotemark[52] A people thus
jealous of their persons, and careless of their possessions, must
have been totally destitute of industry and the arts, but
animated with a high sense of honor and independence.

\footnotetext[51]{Cæsar, vi. 22. Tacit Germ. 26.}

\footnotetext[52]{Tacit. Germ. 7.}

