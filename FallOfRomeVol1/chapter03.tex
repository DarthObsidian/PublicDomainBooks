\chapter{The Constitution In The Age Of The Antonines.}
\section{Part \thesection.}

\textit{Of The Constitution Of The Roman Empire, In The Age Of The
Antonines.}
\vspace{\onelineskip}

The obvious definition of a monarchy seems to be that of a state,
in which a single person, by whatsoever name he may be
distinguished, is intrusted with the execution of the laws, the
management of the revenue, and the command of the army. But,
unless public liberty is protected by intrepid and vigilant
guardians, the authority of so formidable a magistrate will soon
degenerate into despotism. The influence of the clergy, in an age
of superstition, might be usefully employed to assert the rights
of mankind; but so intimate is the connection between the throne
and the altar, that the banner of the church has very seldom been
seen on the side of the people.\textsuperscript{101} A martial nobility and
stubborn commons, possessed of arms, tenacious of property, and
collected into constitutional assemblies, form the only balance
capable of preserving a free constitution against enterprises of
an aspiring prince.

\pagenote[101]{Often enough in the ages of superstition, but not
in the interest of the people or the state, but in that of the
church to which all others were subordinate. Yet the power of the
pope has often been of great service in repressing the excesses
of sovereigns, and in softening manners.—W. The history of the
Italian republics proves the error of Gibbon, and the justice of
his German translator’s comment.—M.}

Every barrier of the Roman constitution had been levelled by the
vast ambition of the dictator; every fence had been extirpated by
the cruel hand of the triumvir. After the victory of Actium, the
fate of the Roman world depended on the will of Octavianus,
surnamed Cæsar, by his uncle’s adoption, and afterwards Augustus,
by the flattery of the senate. The conqueror was at the head of
forty-four veteran legions,\textsuperscript{1} conscious of their own strength,
and of the weakness of the constitution, habituated, during
twenty years’ civil war, to every act of blood and violence, and
passionately devoted to the house of Cæsar, from whence alone
they had received, and expected the most lavish rewards. The
provinces, long oppressed by the ministers of the republic,
sighed for the government of a single person, who would be the
master, not the accomplice, of those petty tyrants. The people of
Rome, viewing, with a secret pleasure, the humiliation of the
aristocracy, demanded only bread and public shows; and were
supplied with both by the liberal hand of Augustus. The rich and
polite Italians, who had almost universally embraced the
philosophy of Epicurus, enjoyed the present blessings of ease and
tranquillity, and suffered not the pleasing dream to be
interrupted by the memory of their old tumultuous freedom. With
its power, the senate had lost its dignity; many of the most
noble families were extinct. The republicans of spirit and
ability had perished in the field of battle, or in the
proscription. The door of the assembly had been designedly left
open, for a mixed multitude of more than a thousand persons, who
reflected disgrace upon their rank, instead of deriving honor
from it.\textsuperscript{2}

\pagenote[1]{Orosius, vi. 18. * Note: Dion says twenty-five, (or
three,) (lv. 23.) The united triumvirs had but forty-three.
(Appian. Bell. Civ. iv. 3.) The testimony of Orosius is of little
value when more certain may be had.—W. But all the legions,
doubtless, submitted to Augustus after the battle of Actium.—M.}

\pagenote[2]{Julius Cæsar introduced soldiers, strangers, and
half-barbarians into the senate (Sueton. in Cæsar. c. 77, 80.)
The abuse became still more scandalous after his death.}

The reformation of the senate was one of the first steps in which
Augustus laid aside the tyrant, and professed himself the father
of his country. He was elected censor; and, in concert with his
faithful Agrippa, he examined the list of the senators, expelled
a few members,\textsuperscript{201} whose vices or whose obstinacy required a
public example, persuaded near two hundred to prevent the shame
of an expulsion by a voluntary retreat, raised the qualification
of a senator to about ten thousand pounds, created a sufficient
number of patrician families, and accepted for himself the
honorable title of Prince of the Senate, \textsuperscript{202} which had always
been bestowed, by the censors, on the citizen the most eminent
for his honors and services. \textsuperscript{3} But whilst he thus restored the
dignity, he destroyed the independence, of the senate. The
principles of a free constitution are irrecoverably lost, when
the legislative power is nominated by the executive.

\pagenote[201]{Of these Dion and Suetonius knew nothing.—W. Dion
says the contrary.—M.}

\pagenote[202]{But Augustus, then Octavius, was censor, and in
virtue of that office, even according to the constitution of the
free republic, could reform the senate, expel unworthy members,
name the Princeps Senatus, \&c. That was called, as is well known,
Senatum legere. It was customary, during the free republic, for
the censor to be named Princeps Senatus, (S. Liv. l. xxvii. c.
11, l. xl. c. 51;) and Dion expressly says, that this was done
according to ancient usage. He was empowered by a decree of the
senate to admit a number of families among the patricians.
Finally, the senate was not the legislative power.—W}

\pagenote[3]{Dion Cassius, l. liii. p. 693. Suetonius in August.
c. 35.}

Before an assembly thus modelled and prepared, Augustus
pronounced a studied oration, which displayed his patriotism, and
disguised his ambition. “He lamented, yet excused, his past
conduct. Filial piety had required at his hands the revenge of
his father’s murder; the humanity of his own nature had sometimes
given way to the stern laws of necessity, and to a forced
connection with two unworthy colleagues: as long as Antony lived,
the republic forbade him to abandon her to a degenerate Roman,
and a barbarian queen. He was now at liberty to satisfy his duty
and his inclination. He solemnly restored the senate and people
to all their ancient rights; and wished only to mingle with the
crowd of his fellow-citizens, and to share the blessings which he
had obtained for his country.”\textsuperscript{4}

\pagenote[4]{Dion (l. liii. p. 698) gives us a prolix and bombast
speech on this great occasion. I have borrowed from Suetonius and
Tacitus the general language of Augustus.}

It would require the pen of Tacitus (if Tacitus had assisted at
this assembly) to describe the various emotions of the senate,
those that were suppressed, and those that were affected. It was
dangerous to trust the sincerity of Augustus; to seem to distrust
it was still more dangerous. The respective advantages of
monarchy and a republic have often divided speculative inquirers;
the present greatness of the Roman state, the corruption of
manners, and the license of the soldiers, supplied new arguments
to the advocates of monarchy; and these general views of
government were again warped by the hopes and fears of each
individual. Amidst this confusion of sentiments, the answer of
the senate was unanimous and decisive. They refused to accept the
resignation of Augustus; they conjured him not to desert the
republic, which he had saved. After a decent resistance, the
crafty tyrant submitted to the orders of the senate; and
consented to receive the government of the provinces, and the
general command of the Roman armies, under the well-known names
of PROCONSUL and IMPERATOR.\textsuperscript{5} But he would receive them only for
ten years. Even before the expiration of that period, he hope
that the wounds of civil discord would be completely healed, and
that the republic, restored to its pristine health and vigor,
would no longer require the dangerous interposition of so
extraordinary a magistrate. The memory of this comedy, repeated
several times during the life of Augustus, was preserved to the
last ages of the empire, by the peculiar pomp with which the
perpetual monarchs of Rome always solemnized the tenth years of
their reign.\textsuperscript{6}

\pagenote[5]{Imperator (from which we have derived Emperor)
signified under her republic no more than general, and was
emphatically bestowed by the soldiers, when on the field of
battle they proclaimed their victorious leader worthy of that
title. When the Roman emperors assumed it in that sense, they
placed it after their name, and marked how often they had taken
it.}

\pagenote[6]{Dion. l. liii. p. 703, \&c.}

Without any violation of the principles of the constitution, the
general of the Roman armies might receive and exercise an
authority almost despotic over the soldiers, the enemies, and the
subjects of the republic. With regard to the soldiers, the
jealousy of freedom had, even from the earliest ages of Rome,
given way to the hopes of conquest, and a just sense of military
discipline. The dictator, or consul, had a right to command the
service of the Roman youth; and to punish an obstinate or
cowardly disobedience by the most severe and ignominious
penalties, by striking the offender out of the list of citizens,
by confiscating his property, and by selling his person into
slavery.\textsuperscript{7} The most sacred rights of freedom, confirmed by the
Porcian and Sempronian laws, were suspended by the military
engagement. In his camp the general exercised an absolute power
of life and death; his jurisdiction was not confined by any forms
of trial, or rules of proceeding, and the execution of the
sentence was immediate and without appeal.\textsuperscript{8} The choice of the
enemies of Rome was regularly decided by the legislative
authority. The most important resolutions of peace and war were
seriously debated in the senate, and solemnly ratified by the
people. But when the arms of the legions were carried to a great
distance from Italy, the general assumed the liberty of directing
them against whatever people, and in whatever manner, they judged
most advantageous for the public service. It was from the
success, not from the justice, of their enterprises, that they
expected the honors of a triumph. In the use of victory,
especially after they were no longer controlled by the
commissioners of the senate, they exercised the most unbounded
despotism. When Pompey commanded in the East, he rewarded his
soldiers and allies, dethroned princes, divided kingdoms, founded
colonies, and distributed the treasures of Mithridates. On his
return to Rome, he obtained, by a single act of the senate and
people, the universal ratification of all his proceedings.\textsuperscript{9} Such
was the power over the soldiers, and over the enemies of Rome,
which was either granted to, or assumed by, the generals of the
republic. They were, at the same time, the governors, or rather
monarchs, of the conquered provinces, united the civil with the
military character, administered justice as well as the finances,
and exercised both the executive and legislative power of the
state.

\pagenote[7]{Livy Epitom. l. xiv. [c. 27.] Valer. Maxim. vi. 3.}

\pagenote[8]{See, in the viiith book of Livy, the conduct of
Manlius Torquatus and Papirius Cursor. They violated the laws of
nature and humanity, but they asserted those of military
discipline; and the people, who abhorred the action, was obliged
to respect the principle.}

\pagenote[9]{By the lavish but unconstrained suffrages of the
people, Pompey had obtained a military command scarcely inferior
to that of Augustus. Among the extraordinary acts of power
executed by the former we may remark the foundation of
twenty-nine cities, and the distribution of three or four
millions sterling to his troops. The ratification of his acts met
with some opposition and delays in the senate See Plutarch,
Appian, Dion Cassius, and the first book of the epistles to
Atticus.}

From what has already been observed in the first chapter of this
work, some notion may be formed of the armies and provinces thus
intrusted to the ruling hand of Augustus. But as it was
impossible that he could personally command the regions of so
many distant frontiers, he was indulged by the senate, as Pompey
had already been, in the permission of devolving the execution of
his great office on a sufficient number of lieutenants. In rank
and authority these officers seemed not inferior to the ancient
proconsuls; but their station was dependent and precarious. They
received and held their commissions at the will of a superior, to
whose auspicious influence the merit of their action was legally
attributed.\textsuperscript{10} They were the representatives of the emperor. The
emperor alone was the general of the republic, and his
jurisdiction, civil as well as military, extended over all the
conquests of Rome. It was some satisfaction, however, to the
senate, that he always delegated his power to the members of
their body. The imperial lieutenants were of consular or
prætorian dignity; the legions were commanded by senators, and
the præfecture of Egypt was the only important trust committed to
a Roman knight.

\pagenote[10]{Under the commonwealth, a triumph could only be
claimed by the general, who was authorized to take the Auspices
in the name of the people. By an exact consequence, drawn from
this principle of policy and religion, the triumph was reserved
to the emperor; and his most successful lieutenants were
satisfied with some marks of distinction, which, under the name
of triumphal honors, were invented in their favor.}

Within six days after Augustus had been compelled to accept so
very liberal a grant, he resolved to gratify the pride of the
senate by an easy sacrifice. He represented to them, that they
had enlarged his powers, even beyond that degree which might be
required by the melancholy condition of the times. They had not
permitted him to refuse the laborious command of the armies and
the frontiers; but he must insist on being allowed to restore the
more peaceful and secure provinces to the mild administration of
the civil magistrate. In the division of the provinces, Augustus
provided for his own power and for the dignity of the republic.
The proconsuls of the senate, particularly those of Asia, Greece,
and Africa, enjoyed a more honorable character than the
lieutenants of the emperor, who commanded in Gaul or Syria. The
former were attended by lictors, the latter by soldiers.\textsuperscript{105} A
law was passed, that wherever the emperor was present, his
extraordinary commission should supersede the ordinary
jurisdiction of the governor; a custom was introduced, that the
new conquests belonged to the imperial portion; and it was soon
discovered that the authority of the \textit{Prince}, the favorite
epithet of Augustus, was the same in every part of the empire.

\pagenote[105]{This distinction is without foundation. The
lieutenants of the emperor, who were called Proprætors, whether
they had been prætors or consuls, were attended by six lictors;
those who had the right of the sword, (of life and death over the
soldiers.—M.) bore the military habit (paludamentum) and the
sword. The provincial governors commissioned by the senate, who,
whether they had been consuls or not, were called Pronconsuls,
had twelve lictors when they had been consuls, and six only when
they had but been prætors. The provinces of Africa and Asia were
only given to ex-consuls. See, on the Organization of the
Provinces, Dion, liii. 12, 16 Strabo, xvii 840.—W}

In return for this imaginary concession, Augustus obtained an
important privilege, which rendered him master of Rome and Italy.
By a dangerous exception to the ancient maxims, he was authorized
to preserve his military command, supported by a numerous body of
guards, even in time of peace, and in the heart of the capital.
His command, indeed, was confined to those citizens who were
engaged in the service by the military oath; but such was the
propensity of the Romans to servitude, that the oath was
voluntarily taken by the magistrates, the senators, and the
equestrian order, till the homage of flattery was insensibly
converted into an annual and solemn protestation of fidelity.

Although Augustus considered a military force as the firmest
foundation, he wisely rejected it, as a very odious instrument of
government. It was more agreeable to his temper, as well as to
his policy, to reign under the venerable names of ancient
magistracy, and artfully to collect, in his own person, all the
scattered rays of civil jurisdiction. With this view, he
permitted the senate to confer upon him, for his life, the powers
of the consular\textsuperscript{11} and tribunitian offices,\textsuperscript{12} which were, in the
same manner, continued to all his successors. The consuls had
succeeded to the kings of Rome, and represented the dignity of
the state. They superintended the ceremonies of religion, levied
and commanded the legions, gave audience to foreign ambassadors,
and presided in the assemblies both of the senate and people. The
general control of the finances was intrusted to their care; and
though they seldom had leisure to administer justice in person,
they were considered as the supreme guardians of law, equity, and
the public peace. Such was their ordinary jurisdiction; but
whenever the senate empowered the first magistrate to consult the
safety of the commonwealth, he was raised by that decree above
the laws, and exercised, in the defence of liberty, a temporary
despotism.\textsuperscript{13} The character of the tribunes was, in every
respect, different from that of the consuls. The appearance of
the former was modest and humble; but their persons were sacred
and inviolable. Their force was suited rather for opposition than
for action. They were instituted to defend the oppressed, to
pardon offences, to arraign the enemies of the people, and, when
they judged it necessary, to stop, by a single word, the whole
machine of government. As long as the republic subsisted, the
dangerous influence, which either the consul or the tribune might
derive from their respective jurisdiction, was diminished by
several important restrictions. Their authority expired with the
year in which they were elected; the former office was divided
between two, the latter among ten persons; and, as both in their
private and public interest they were averse to each other, their
mutual conflicts contributed, for the most part, to strengthen
rather than to destroy the balance of the constitution.\textsuperscript{131} But
when the consular and tribunitian powers were united, when they
were vested for life in a single person, when the general of the
army was, at the same time, the minister of the senate and the
representative of the Roman people, it was impossible to resist
the exercise, nor was it easy to define the limits, of his
imperial prerogative.

\pagenote[11]{Cicero (de Legibus, iii. 3) gives the consular
office the name of egia potestas; and Polybius (l. vi. c. 3)
observes three powers in the Roman constitution. The monarchical
was represented and exercised by the consuls.}

\pagenote[12]{As the tribunitian power (distinct from the annual
office) was first invented by the dictator Cæsar, (Dion, l. xliv.
p. 384,) we may easily conceive, that it was given as a reward
for having so nobly asserted, by arms, the sacred rights of the
tribunes and people. See his own Commentaries, de Bell. Civil. l.
i.}

\pagenote[13]{Augustus exercised nine annual consulships without
interruption. He then most artfully refused the magistracy, as
well as the dictatorship, absented himself from Rome, and waited
till the fatal effects of tumult and faction forced the senate to
invest him with a perpetual consulship. Augustus, as well as his
successors, affected, however, to conceal so invidious a title.}

\pagenote[131]{The note of M. Guizot on the tribunitian power
applies to the French translation rather than to the original.
The former has, maintenir la balance toujours egale, which
implies much more than Gibbon’s general expression. The note
belongs rather to the history of the Republic than that of the
Empire.—M}

To these accumulated honors, the policy of Augustus soon added
the splendid as well as important dignities of supreme pontiff,
and of censor. By the former he acquired the management of the
religion, and by the latter a legal inspection over the manners
and fortunes, of the Roman people. If so many distinct and
independent powers did not exactly unite with each other, the
complaisance of the senate was prepared to supply every
deficiency by the most ample and extraordinary concessions. The
emperors, as the first ministers of the republic, were exempted
from the obligation and penalty of many inconvenient laws: they
were authorized to convoke the senate, to make several motions in
the same day, to recommend candidates for the honors of the
state, to enlarge the bounds of the city, to employ the revenue
at their discretion, to declare peace and war, to ratify
treaties; and by a most comprehensive clause, they were empowered
to execute whatsoever they should judge advantageous to the
empire, and agreeable to the majesty of things private or public,
human of divine.\textsuperscript{14}

\pagenote[14]{See a fragment of a Decree of the Senate,
conferring on the emperor Vespasian all the powers granted to his
predecessors, Augustus, Tiberius, and Claudius. This curious and
important monument is published in Gruter’s Inscriptions, No.
ccxlii. * Note: It is also in the editions of Tacitus by Ryck,
(Annal. p. 420, 421,) and Ernesti, (Excurs. ad lib. iv. 6;) but
this fragment contains so many inconsistencies, both in matter
and form, that its authenticity may be doubted—W.}

When all the various powers of executive government were
committed to the \textit{Imperial magistrate}, the ordinary magistrates
of the commonwealth languished in obscurity, without vigor, and
almost without business. The names and forms of the ancient
administration were preserved by Augustus with the most anxious
care. The usual number of consuls, prætors, and tribunes,\textsuperscript{15} were
annually invested with their respective ensigns of office, and
continued to discharge some of their least important functions.
Those honors still attracted the vain ambition of the Romans; and
the emperors themselves, though invested for life with the powers
of the consulship, frequently aspired to the title of that annual
dignity, which they condescended to share with the most
illustrious of their fellow-citizens.\textsuperscript{16} In the election of these
magistrates, the people, during the reign of Augustus, were
permitted to expose all the inconveniences of a wild democracy.
That artful prince, instead of discovering the least symptom of
impatience, humbly solicited their suffrages for himself or his
friends, and scrupulously practised all the duties of an ordinary
candidate.\textsuperscript{17} But we may venture to ascribe to his councils the
first measure of the succeeding reign, by which the elections
were transferred to the senate.\textsuperscript{18} The assemblies of the people
were forever abolished, and the emperors were delivered from a
dangerous multitude, who, without restoring liberty, might have
disturbed, and perhaps endangered, the established government.

\pagenote[15]{Two consuls were created on the Calends of January;
but in the course of the year others were substituted in their
places, till the annual number seems to have amounted to no less
than twelve. The prætors were usually sixteen or eighteen,
(Lipsius in Excurs. D. ad Tacit. Annal. l. i.) I have not
mentioned the Ædiles or Quæstors Officers of the police or
revenue easily adapt themselves to any form of government. In the
time of Nero, the tribunes legally possessed the right of
intercession, though it might be dangerous to exercise it (Tacit.
Annal. xvi. 26.) In the time of Trajan, it was doubtful whether
the tribuneship was an office or a name, (Plin. Epist. i. 23.)}

\pagenote[16]{The tyrants themselves were ambitious of the
consulship. The virtuous princes were moderate in the pursuit,
and exact in the discharge of it. Trajan revived the ancient
oath, and swore before the consul’s tribunal that he would
observe the laws, (Plin. Panegyric c. 64.)}

\pagenote[17]{Quoties Magistratuum Comitiis interesset. Tribus
cum candidatis suis circunbat: supplicabatque more solemni.
Ferebat et ipse suffragium in tribubus, ut unus e populo.
Suetonius in August c. 56.}

\pagenote[18]{Tum primum Comitia e campo ad patres translata
sunt. Tacit. Annal. i. 15. The word primum seems to allude to
some faint and unsuccessful efforts which were made towards
restoring them to the people. Note: The emperor Caligula made the
attempt: he rest red the Comitia to the people, but, in a short
time, took them away again. Suet. in Caio. c. 16. Dion. lix. 9,
20. Nevertheless, at the time of Dion, they preserved still the
form of the Comitia. Dion. lviii. 20.—W.}

By declaring themselves the protectors of the people, Marius and
Cæsar had subverted the constitution of their country. But as
soon as the senate had been humbled and disarmed, such an
assembly, consisting of five or six hundred persons, was found a
much more tractable and useful instrument of dominion. It was on
the dignity of the senate that Augustus and his successors
founded their new empire; and they affected, on every occasion,
to adopt the language and principles of Patricians. In the
administration of their own powers, they frequently consulted the
great national council, and \textit{seemed} to refer to its decision the
most important concerns of peace and war. Rome, Italy, and the
internal provinces, were subject to the immediate jurisdiction of
the senate. With regard to civil objects, it was the supreme
court of appeal; with regard to criminal matters, a tribunal,
constituted for the trial of all offences that were committed by
men in any public station, or that affected the peace and majesty
of the Roman people. The exercise of the judicial power became
the most frequent and serious occupation of the senate; and the
important causes that were pleaded before them afforded a last
refuge to the spirit of ancient eloquence. As a council of state,
and as a court of justice, the senate possessed very considerable
prerogatives; but in its legislative capacity, in which it was
supposed virtually to represent the people, the rights of
sovereignty were acknowledged to reside in that assembly. Every
power was derived from their authority, every law was ratified by
their sanction. Their regular meetings were held on three stated
days in every month, the Calends, the Nones, and the Ides. The
debates were conducted with decent freedom; and the emperors
themselves, who gloried in the name of senators, sat, voted, and
divided with their equals. To resume, in a few words, the system
of the Imperial government; as it was instituted by Augustus, and
maintained by those princes who understood their own interest and
that of the people, it may be defined an absolute monarchy
disguised by the forms of a commonwealth. The masters of the
Roman world surrounded their throne with darkness, concealed
their irresistible strength, and humbly professed themselves the
accountable ministers of the senate, whose supreme decrees they
dictated and obeyed.\textsuperscript{19}

\pagenote[19]{Dion Cassius (l. liii. p. 703—714) has given a very
loose and partial sketch of the Imperial system. To illustrate
and often to correct him, I have meditated Tacitus, examined
Suetonius, and consulted the following moderns: the Abbé de la
Bleterie, in the Memoires de l’Academie des Inscriptions, tom.
xix. xxi. xxiv. xxv. xxvii. Beaufort Republique Romaine, tom. i.
p. 255—275. The Dissertations of Noodt and Gronovius de lege
Regia, printed at Leyden, in the year 1731 Gravina de Imperio
Romano, p. 479—544 of his Opuscula. Maffei, Verona Illustrata, p.
i. p. 245, \&c.}

The face of the court corresponded with the forms of the
administration. The emperors, if we except those tyrants whose
capricious folly violated every law of nature and decency,
disdained that pomp and ceremony which might offend their
countrymen, but could add nothing to their real power. In all the
offices of life, they affected to confound themselves with their
subjects, and maintained with them an equal intercourse of visits
and entertainments. Their habit, their palace, their table, were
suited only to the rank of an opulent senator. Their family,
however numerous or splendid, was composed entirely of their
domestic slaves and freedmen.\textsuperscript{20} Augustus or Trajan would have
blushed at employing the meanest of the Romans in those menial
offices, which, in the household and bedchamber of a limited
monarch, are so eagerly solicited by the proudest nobles of
Britain.

\pagenote[20]{A weak prince will always be governed by his
domestics. The power of slaves aggravated the shame of the
Romans; and the senate paid court to a Pallas or a Narcissus.
There is a chance that a modern favorite may be a gentleman.}

The deification of the emperors\textsuperscript{21} is the only instance in which
they departed from their accustomed prudence and modesty. The
Asiatic Greeks were the first inventors, the successors of
Alexander the first objects, of this servile and impious mode of
adulation.\textsuperscript{211} It was easily transferred from the kings to the
governors of Asia; and the Roman magistrates very frequently were
adored as provincial deities, with the pomp of altars and
temples, of festivals and sacrifices.\textsuperscript{22} It was natural that the
emperors should not refuse what the proconsuls had accepted; and
the divine honors which both the one and the other received from
the provinces, attested rather the despotism than the servitude
of Rome. But the conquerors soon imitated the vanquished nations
in the arts of flattery; and the imperious spirit of the first
Cæsar too easily consented to assume, during his lifetime, a
place among the tutelar deities of Rome. The milder temper of his
successor declined so dangerous an ambition, which was never
afterwards revived, except by the madness of Caligula and
Domitian. Augustus permitted indeed some of the provincial cities
to erect temples to his honor, on condition that they should
associate the worship of Rome with that of the sovereign; he
tolerated private superstition, of which he might be the object;\textsuperscript{23}
but he contented himself with being revered by the senate and
the people in his human character, and wisely left to his
successor the care of his public deification. A regular custom
was introduced, that on the decease of every emperor who had
neither lived nor died like a tyrant, the senate by a solemn
decree should place him in the number of the gods: and the
ceremonies of his apotheosis were blended with those of his
funeral.\textsuperscript{231} This legal, and, as it should seem, injudicious
profanation, so abhorrent to our stricter principles, was
received with a very faint murmur,\textsuperscript{24} by the easy nature of
Polytheism; but it was received as an institution, not of
religion, but of policy. We should disgrace the virtues of the
Antonines by comparing them with the vices of Hercules or
Jupiter. Even the characters of Cæsar or Augustus were far
superior to those of the popular deities. But it was the
misfortune of the former to live in an enlightened age, and their
actions were too faithfully recorded to admit of such a mixture
of fable and mystery, as the devotion of the vulgar requires. As
soon as their divinity was established by law, it sunk into
oblivion, without contributing either to their own fame, or to
the dignity of succeeding princes.

\pagenote[21]{See a treatise of Vandale de Consecratione
Principium. It would be easier for me to copy, than it has been
to verify, the quotations of that learned Dutchman.}

\pagenote[211]{This is inaccurate. The successors of Alexander
were not the first deified sovereigns; the Egyptians had deified
and worshipped many of their kings; the Olympus of the Greeks was
peopled with divinities who had reigned on earth; finally,
Romulus himself had received the honors of an apotheosis (Tit.
Liv. i. 16) a long time before Alexander and his successors. It
is also an inaccuracy to confound the honors offered in the
provinces to the Roman governors, by temples and altars, with the
true apotheosis of the emperors; it was not a religious worship,
for it had neither priests nor sacrifices. Augustus was severely
blamed for having permitted himself to be worshipped as a god in
the provinces, (Tac. Ann. i. 10: ) he would not have incurred
that blame if he had only done what the governors were accustomed
to do.—G. from W. M. Guizot has been guilty of a still greater
inaccuracy in confounding the deification of the living with the
apotheosis of the dead emperors. The nature of the king-worship
of Egypt is still very obscure; the hero-worship of the Greeks
very different from the adoration of the “præsens numen” in the
reigning sovereign.—M.}

\pagenote[22]{See a dissertation of the Abbé Mongault in the
first volume of the Academy of Inscriptions.}

\pagenote[23]{Jurandasque tuum per nomen ponimus aras, says
Horace to the emperor himself, and Horace was well acquainted
with the court of Augustus. Note: The good princes were not those
who alone obtained the honors of an apotheosis: it was conferred
on many tyrants. See an excellent treatise of Schæpflin, de
Consecratione Imperatorum Romanorum, in his Commentationes
historicæ et criticæ. Bale, 1741, p. 184.—W.}

\pagenote[231]{The curious satire in the works of Seneca, is the
strongest remonstrance of profaned religion.—M.}

\pagenote[24]{See Cicero in Philippic. i. 6. Julian in Cæsaribus.
Inque Deum templis jurabit Roma per umbras, is the indignant
expression of Lucan; but it is a patriotic rather than a devout
indignation.}

In the consideration of the Imperial government, we have
frequently mentioned the artful founder, under his well-known
title of Augustus, which was not, however, conferred upon him
till the edifice was almost completed. The obscure name of
Octavianus he derived from a mean family, in the little town of
Aricia.\textsuperscript{241} It was stained with the blood of the proscription;
and he was desirous, had it been possible, to erase all memory of
his former life. The illustrious surname of Cæsar he had assumed,
as the adopted son of the dictator: but he had too much good
sense, either to hope to be confounded, or to wish to be compared
with that extraordinary man. It was proposed in the senate to
dignify their minister with a new appellation; and after a
serious discussion, that of Augustus was chosen, among several
others, as being the most expressive of the character of peace
and sanctity, which he uniformly affected.\textsuperscript{25} \textit{Augustus} was
therefore a personal, \textit{Cæsar} a family distinction. The former
should naturally have expired with the prince on whom it was
bestowed; and however the latter was diffused by adoption and
female alliance, Nero was the last prince who could allege any
hereditary claim to the honors of the Julian line. But, at the
time of his death, the practice of a century had inseparably
connected those appellations with the Imperial dignity, and they
have been preserved by a long succession of emperors, Romans,
Greeks, Franks, and Germans, from the fall of the republic to the
present time. A distinction was, however, soon introduced. The
sacred title of Augustus was always reserved for the monarch,
whilst the name of Cæsar was more freely communicated to his
relations; and, from the reign of Hadrian, at least, was
appropriated to the second person in the state, who was
considered as the presumptive heir of the empire.\textsuperscript{251}

\pagenote[241]{Octavius was not of an obscure family, but of a
considerable one of the equestrian order. His father, C.
Octavius, who possessed great property, had been prætor, governor
of Macedonia, adorned with the title of Imperator, and was on the
point of becoming consul when he died. His mother Attia, was
daughter of M. Attius Balbus, who had also been prætor. M.
Anthony reproached Octavius with having been born in Aricia,
which, nevertheless, was a considerable municipal city: he was
vigorously refuted by Cicero. Philip. iii. c. 6.—W. Gibbon
probably meant that the family had but recently emerged into
notice.—M.}

\pagenote[25]{Dion. Cassius, l. liii. p. 710, with the curious
Annotations of Reimar.}

\pagenote[251]{The princes who by their birth or their adoption
belonged to the family of the Cæsars, took the name of Cæsar.
After the death of Nero, this name designated the Imperial
dignity itself, and afterwards the appointed successor. The time
at which it was employed in the latter sense, cannot be fixed
with certainty. Bach (Hist. Jurisprud. Rom. 304) affirms from
Tacitus, H. i. 15, and Suetonius, Galba, 17, that Galba conferred
on Piso Lucinianus the title of Cæsar, and from that time the
term had this meaning: but these two historians simply say that
he appointed Piso his successor, and do not mention the word
Cæsar. Aurelius Victor (in Traj. 348, ed. Artzen) says that
Hadrian first received this title on his adoption; but as the
adoption of Hadrian is still doubtful, and besides this, as
Trajan, on his death-bed, was not likely to have created a new
title for his successor, it is more probable that Ælius Verus was
the first who was called Cæsar when adopted by Hadrian. Spart. in
Ælio Vero, 102.—W.}

\section{Part \thesection.}

The tender respect of Augustus for a free constitution which he
had destroyed, can only be explained by an attentive
consideration of the character of that subtle tyrant. A cool
head, an unfeeling heart, and a cowardly disposition, prompted
him at the age of nineteen to assume the mask of hypocrisy, which
he never afterwards laid aside. With the same hand, and probably
with the same temper, he signed the proscription of Cicero, and
the pardon of Cinna. His virtues, and even his vices, were
artificial; and according to the various dictates of his
interest, he was at first the enemy, and at last the father, of
the Roman world.\textsuperscript{26} When he framed the artful system of the
Imperial authority, his moderation was inspired by his fears. He
wished to deceive the people by an image of civil liberty, and
the armies by an image of civil government.

\pagenote[26]{As Octavianus advanced to the banquet of the
Cæsars, his color changed like that of the chameleon; pale at
first, then red, afterwards black, he at last assumed the mild
livery of Venus and the Graces, (Cæsars, p. 309.) This image,
employed by Julian in his ingenious fiction, is just and elegant;
but when he considers this change of character as real and
ascribes it to the power of philosophy, he does too much honor to
philosophy and to Octavianus.}

I. The death of Cæsar was ever before his eyes. He had lavished
wealth and honors on his adherents; but the most favored friends
of his uncle were in the number of the conspirators. The fidelity
of the legions might defend his authority against open rebellion;
but their vigilance could not secure his person from the dagger
of a determined republican; and the Romans, who revered the
memory of Brutus,\textsuperscript{27} would applaud the imitation of his virtue.
Cæsar had provoked his fate, as much as by the ostentation of his
power, as by his power itself. The consul or the tribune might
have reigned in peace. The title of king had armed the Romans
against his life. Augustus was sensible that mankind is governed
by names; nor was he deceived in his expectation, that the senate
and people would submit to slavery, provided they were
respectfully assured that they still enjoyed their ancient
freedom. A feeble senate and enervated people cheerfully
acquiesced in the pleasing illusion, as long as it was supported
by the virtue, or even by the prudence, of the successors of
Augustus. It was a motive of self-preservation, not a principle
of liberty, that animated the conspirators against Caligula,
Nero, and Domitian. They attacked the person of the tyrant,
without aiming their blow at the authority of the emperor.

\pagenote[27]{Two centuries after the establishment of monarchy,
the emperor Marcus Antoninus recommends the character of Brutus
as a perfect model of Roman virtue. * Note: In a very ingenious
essay, Gibbon has ventured to call in question the preeminent
virtue of Brutus. Misc Works, iv. 95.—M.}

There appears, indeed, \textit{one} memorable occasion, in which the
senate, after seventy years of patience, made an ineffectual
attempt to re-assume its long-forgotten rights. When the throne
was vacant by the murder of Caligula, the consuls convoked that
assembly in the Capitol, condemned the memory of the Cæsars, gave
the watchword \textit{liberty} to the few cohorts who faintly adhered to
their standard, and during eight-and-forty hours acted as the
independent chiefs of a free commonwealth. But while they
deliberated, the prætorian guards had resolved. The stupid
Claudius, brother of Germanicus, was already in their camp,
invested with the Imperial purple, and prepared to support his
election by arms. The dream of liberty was at an end; and the
senate awoke to all the horrors of inevitable servitude. Deserted
by the people, and threatened by a military force, that feeble
assembly was compelled to ratify the choice of the prætorians,
and to embrace the benefit of an amnesty, which Claudius had the
prudence to offer, and the generosity to observe.\textsuperscript{28}

\pagenote[28]{It is much to be regretted that we have lost the
part of Tacitus which treated of that transaction. We are forced
to content ourselves with the popular rumors of Josephus, and the
imperfect hints of Dion and Suetonius.}

II. The insolence of the armies inspired Augustus with fears of a
still more alarming nature. The despair of the citizens could
only attempt, what the power of the soldiers was, at any time,
able to execute. How precarious was his own authority over men
whom he had taught to violate every social duty! He had heard
their seditious clamors; he dreaded their calmer moments of
reflection. One revolution had been purchased by immense rewards;
but a second revolution might double those rewards. The troops
professed the fondest attachment to the house of Cæsar; but the
attachments of the multitude are capricious and inconstant.
Augustus summoned to his aid whatever remained in those fierce
minds of Roman prejudices; enforced the rigor of discipline by
the sanction of law; and, interposing the majesty of the senate
between the emperor and the army, boldly claimed their
allegiance, as the first magistrate of the republic.

During a long period of two hundred and twenty years from the
establishment of this artful system to the death of Commodus, the
dangers inherent to a military government were, in a great
measure, suspended. The soldiers were seldom roused to that fatal
sense of their own strength, and of the weakness of the civil
authority, which was, before and afterwards, productive of such
dreadful calamities. Caligula and Domitian were assassinated in
their palace by their own domestics:\textsuperscript{281} the convulsions which
agitated Rome on the death of the former, were confined to the
walls of the city. But Nero involved the whole empire in his
ruin. In the space of eighteen months, four princes perished by
the sword; and the Roman world was shaken by the fury of the
contending armies. Excepting only this short, though violent
eruption of military license, the two centuries from Augustus\textsuperscript{29}
to Commodus passed away unstained with civil blood, and
undisturbed by revolutions. The emperor was elected by \textit{the
authority of the senate}, and \textit{the consent of the soldiers}.\textsuperscript{30}
The legions respected their oath of fidelity; and it requires a
minute inspection of the Roman annals to discover three
inconsiderable rebellions, which were all suppressed in a few
months, and without even the hazard of a battle.\textsuperscript{31}

\pagenote[281]{Caligula perished by a conspiracy formed by the
officers of the prætorian troops, and Domitian would not,
perhaps, have been assassinated without the participation of the
two chiefs of that guard in his death.—W.}

\pagenote[29]{Augustus restored the ancient severity of
discipline. After the civil wars, he dropped the endearing name
of Fellow-Soldiers, and called them only Soldiers, (Sueton. in
August. c. 25.) See the use Tiberius made of the Senate in the
mutiny of the Pannonian legions, (Tacit. Annal. i.)}

\pagenote[30]{These words seem to have been the constitutional
language. See Tacit. Annal. xiii. 4. * Note: This panegyric on
the soldiery is rather too liberal. Claudius was obliged to
purchase their consent to his coronation: the presents which he
made, and those which the prætorians received on other occasions,
considerably embarrassed the finances. Moreover, this formidable
guard favored, in general, the cruelties of the tyrants. The
distant revolts were more frequent than Gibbon thinks: already,
under Tiberius, the legions of Germany would have seditiously
constrained Germanicus to assume the Imperial purple. On the
revolt of Claudius Civilis, under Vespasian, the legions of Gaul
murdered their general, and offered their assistance to the Gauls
who were in insurrection. Julius Sabinus made himself be
proclaimed emperor, \&c. The wars, the merit, and the severe
discipline of Trajan, Hadrian, and the two Antonines,
established, for some time, a greater degree of subordination.—W}

\pagenote[31]{The first was Camillus Scribonianus, who took up
arms in Dalmatia against Claudius, and was deserted by his own
troops in five days, the second, L. Antonius, in Germany, who
rebelled against Domitian; and the third, Avidius Cassius, in the
reign of M. Antoninus. The two last reigned but a few months, and
were cut off by their own adherents. We may observe, that both
Camillus and Cassius colored their ambition with the design of
restoring the republic; a task, said Cassius peculiarly reserved
for his name and family.}

In elective monarchies, the vacancy of the throne is a moment big
with danger and mischief. The Roman emperors, desirous to spare
the legions that interval of suspense, and the temptation of an
irregular choice, invested their designed successor with so large
a share of present power, as should enable him, after their
decease, to assume the remainder, without suffering the empire to
perceive the change of masters. Thus Augustus, after all his
fairer prospects had been snatched from him by untimely deaths,
rested his last hopes on Tiberius, obtained for his adopted son
the censorial and tribunitian powers, and dictated a law, by
which the future prince was invested with an authority equal to
his own, over the provinces and the armies.\textsuperscript{32} Thus Vespasian
subdued the generous mind of his eldest son. Titus was adored by
the eastern legions, which, under his command, had recently
achieved the conquest of Judæa. His power was dreaded, and, as
his virtues were clouded by the intemperance of youth, his
designs were suspected. Instead of listening to such unworthy
suspicions, the prudent monarch associated Titus to the full
powers of the Imperial dignity; and the grateful son ever
approved himself the humble and faithful minister of so indulgent
a father.\textsuperscript{33}

\pagenote[32]{Velleius Paterculus, l. ii. c. 121. Sueton. in
Tiber. c. 26.}

\pagenote[33]{Sueton. in Tit. c. 6. Plin. in Præfat. Hist.
Natur.}

The good sense of Vespasian engaged him indeed to embrace every
measure that might confirm his recent and precarious elevation.
The military oath, and the fidelity of the troops, had been
consecrated, by the habits of a hundred years, to the name and
family of the Cæsars; and although that family had been continued
only by the fictitious rite of adoption, the Romans still
revered, in the person of Nero, the grandson of Germanicus, and
the lineal successor of Augustus. It was not without reluctance
and remorse, that the prætorian guards had been persuaded to
abandon the cause of the tyrant.\textsuperscript{34} The rapid downfall of Galba,
Otho, and Vitellus, taught the armies to consider the emperors as
the creatures of \textit{their} will, and the instruments of \textit{their}
license. The birth of Vespasian was mean: his grandfather had
been a private soldier, his father a petty officer of the
revenue;\textsuperscript{35} his own merit had raised him, in an advanced age, to
the empire; but his merit was rather useful than shining, and his
virtues were disgraced by a strict and even sordid parsimony.
Such a prince consulted his true interest by the association of a
son, whose more splendid and amiable character might turn the
public attention from the obscure origin, to the future glories,
of the Flavian house. Under the mild administration of Titus, the
Roman world enjoyed a transient felicity, and his beloved memory
served to protect, above fifteen years, the vices of his brother
Domitian.

\pagenote[34]{This idea is frequently and strongly inculcated by
Tacitus. See Hist. i. 5, 16, ii. 76.}

\pagenote[35]{The emperor Vespasian, with his usual good sense,
laughed at the genealogists, who deduced his family from Flavius,
the founder of Reate, (his native country,) and one of the
companions of Hercules Suet in Vespasian, c. 12.}

Nerva had scarcely accepted the purple from the assassins of
Domitian, before he discovered that his feeble age was unable to
stem the torrent of public disorders, which had multiplied under
the long tyranny of his predecessor. His mild disposition was
respected by the good; but the degenerate Romans required a more
vigorous character, whose justice should strike terror into the
guilty. Though he had several relations, he fixed his choice on a
stranger. He adopted Trajan, then about forty years of age, and
who commanded a powerful army in the Lower Germany; and
immediately, by a decree of the senate, declared him his
colleague and successor in the empire.\textsuperscript{36} It is sincerely to be
lamented, that whilst we are fatigued with the disgustful
relation of Nero’s crimes and follies, we are reduced to collect
the actions of Trajan from the glimmerings of an abridgment, or
the doubtful light of a panegyric. There remains, however, one
panegyric far removed beyond the suspicion of flattery. Above two
hundred and fifty years after the death of Trajan, the senate, in
pouring out the customary acclamations on the accession of a new
emperor, wished that he might surpass the felicity of Augustus,
and the virtue of Trajan.\textsuperscript{37}

\pagenote[36]{Dion, l. lxviii. p. 1121. Plin. Secund. in
Panegyric.}

\pagenote[37]{Felicior Augusto, Melior Trajano. Eutrop. viii. 5.}

We may readily believe, that the father of his country hesitated
whether he ought to intrust the various and doubtful character of
his kinsman Hadrian with sovereign power. In his last moments the
arts of the empress Plotina either fixed the irresolution of
Trajan, or boldly supposed a fictitious adoption;\textsuperscript{38} the truth of
which could not be safely disputed, and Hadrian was peaceably
acknowledged as his lawful successor. Under his reign, as has
been already mentioned, the empire flourished in peace and
prosperity. He encouraged the arts, reformed the laws, asserted
military discipline, and visited all his provinces in person. His
vast and active genius was equally suited to the most enlarged
views, and the minute details of civil policy. But the ruling
passions of his soul were curiosity and vanity. As they
prevailed, and as they were attracted by different objects,
Hadrian was, by turns, an excellent prince, a ridiculous sophist,
and a jealous tyrant. The general tenor of his conduct deserved
praise for its equity and moderation. Yet in the first days of
his reign, he put to death four consular senators, his personal
enemies, and men who had been judged worthy of empire; and the
tediousness of a painful illness rendered him, at last, peevish
and cruel. The senate doubted whether they should pronounce him a
god or a tyrant; and the honors decreed to his memory were
granted to the prayers of the pious Antoninus.\textsuperscript{39}

\pagenote[38]{Dion (l. lxix. p. 1249) affirms the whole to have
been a fiction, on the authority of his father, who, being
governor of the province where Trajan died, had very good
opportunities of sifting this mysterious transaction. Yet Dodwell
(Prælect. Camden. xvii.) has maintained that Hadrian was called
to the certain hope of the empire, during the lifetime of
Trajan.}

\pagenote[39]{Dion, (l. lxx. p. 1171.) Aurel. Victor.}

The caprice of Hadrian influenced his choice of a successor.

After revolving in his mind several men of distinguished merit,
whom he esteemed and hated, he adopted Ælius Verus a gay and
voluptuous nobleman, recommended by uncommon beauty to the lover
of Antinous.\textsuperscript{40} But whilst Hadrian was delighting himself with
his own applause, and the acclamations of the soldiers, whose
consent had been secured by an immense donative, the new Cæsar\textsuperscript{41}
was ravished from his embraces by an untimely death. He left only
one son. Hadrian commended the boy to the gratitude of the
Antonines. He was adopted by Pius; and, on the accession of
Marcus, was invested with an equal share of sovereign power.
Among the many vices of this younger Verus, he possessed one
virtue; a dutiful reverence for his wiser colleague, to whom he
willingly abandoned the ruder cares of empire. The philosophic
emperor dissembled his follies, lamented his early death, and
cast a decent veil over his memory.

\pagenote[40]{The deification of Antinous, his medals, his
statues, temples, city, oracles, and constellation, are well
known, and still dishonor the memory of Hadrian. Yet we may
remark, that of the first fifteen emperors, Claudius was the only
one whose taste in love was entirely correct. For the honors of
Antinous, see Spanheim, Commentaire sui les Cæsars de Julien, p.
80.}

\pagenote[41]{Hist. August. p. 13. Aurelius Victor in Epitom.}

As soon as Hadrian’s passion was either gratified or
disappointed, he resolved to deserve the thanks of posterity, by
placing the most exalted merit on the Roman throne. His
discerning eye easily discovered a senator about fifty years of
age, blameless in all the offices of life; and a youth of about
seventeen, whose riper years opened a fair prospect of every
virtue: the elder of these was declared the son and successor of
Hadrian, on condition, however, that he himself should
immediately adopt the younger. The two Antonines (for it is of
them that we are now speaking,) governed the Roman world
forty-two years, with the same invariable spirit of wisdom and
virtue. Although Pius had two sons,\textsuperscript{42} he preferred the welfare
of Rome to the interest of his family, gave his daughter
Faustina, in marriage to young Marcus, obtained from the senate
the tribunitian and proconsular powers, and, with a noble
disdain, or rather ignorance of jealousy, associated him to all
the labors of government. Marcus, on the other hand, revered the
character of his benefactor, loved him as a parent, obeyed him as
his sovereign,\textsuperscript{43} and, after he was no more, regulated his own
administration by the example and maxims of his predecessor.
Their united reigns are possibly the only period of history in
which the happiness of a great people was the sole object of
government.

\pagenote[42]{Without the help of medals and inscriptions, we
should be ignorant of this fact, so honorable to the memory of
Pius. Note: Gibbon attributes to Antoninus Pius a merit which he
either did not possess, or was not in a situation to display.

1. He was adopted only on the condition that he would adopt, in
his turn, Marcus Aurelius and L. Verus.

2. His two sons died children, and one of them, M. Galerius,
alone, appears to have survived, for a few years, his father’s
coronation. Gibbon is also mistaken when he says (note 42) that
“without the help of medals and inscriptions, we should be
ignorant that Antoninus had two sons.” Capitolinus says
expressly, (c. 1,) Filii mares duo, duæ-fœminæ; we only owe
their names to the medals. Pagi. Cont. Baron, i. 33, edit
Paris.—W.}

\pagenote[43]{During the twenty-three years of Pius’s reign,
Marcus was only two nights absent from the palace, and even those
were at different times. Hist. August. p. 25.}

Titus Antoninus Pius has been justly denominated a second Numa.
The same love of religion, justice, and peace, was the
distinguishing characteristic of both princes. But the situation
of the latter opened a much larger field for the exercise of
those virtues. Numa could only prevent a few neighboring villages
from plundering each other’s harvests. Antoninus diffused order
and tranquillity over the greatest part of the earth. His reign
is marked by the rare advantage of furnishing very few materials
for history; which is, indeed, little more than the register of
the crimes, follies, and misfortunes of mankind. In private life,
he was an amiable, as well as a good man. The native simplicity
of his virtue was a stranger to vanity or affectation. He enjoyed
with moderation the conveniences of his fortune, and the innocent
pleasures of society;\textsuperscript{44} and the benevolence of his soul
displayed itself in a cheerful serenity of temper.

\pagenote[44]{He was fond of the theatre, and not insensible to
the charms of the fair sex. Marcus Antoninus, i. 16. Hist.
August. p. 20, 21. Julian in Cæsar.}

The virtue of Marcus Aurelius Antoninus was of severer and more
laborious kind.\textsuperscript{45} It was the well-earned harvest of many a
learned conference, of many a patient lecture, and many a
midnight lucubration. At the age of twelve years he embraced the
rigid system of the Stoics, which taught him to submit his body
to his mind, his passions to his reason; to consider virtue as
the only good, vice as the only evil, all things external as
things indifferent.\textsuperscript{46} His meditations, composed in the tumult of
the camp, are still extant; and he even condescended to give
lessons of philosophy, in a more public manner than was perhaps
consistent with the modesty of sage, or the dignity of an
emperor.\textsuperscript{47} But his life was the noblest commentary on the
precepts of Zeno. He was severe to himself, indulgent to the
imperfections of others, just and beneficent to all mankind. He
regretted that Avidius Cassius, who excited a rebellion in Syria,
had disappointed him, by a voluntary death,\textsuperscript{471} of the pleasure
of converting an enemy into a friend;; and he justified the
sincerity of that sentiment, by moderating the zeal of the senate
against the adherents of the traitor.\textsuperscript{48} War he detested, as the
disgrace and calamity of human nature;\textsuperscript{481} but when the necessity
of a just defence called upon him to take up arms, he readily
exposed his person to eight winter campaigns, on the frozen banks
of the Danube, the severity of which was at last fatal to the
weakness of his constitution. His memory was revered by a
grateful posterity, and above a century after his death, many
persons preserved the image of Marcus Antoninus among those of
their household gods.\textsuperscript{49}

\pagenote[45]{The enemies of Marcus charged him with hypocrisy,
and with a want of that simplicity which distinguished Pius and
even Verus. (Hist. August. 6, 34.) This suspicions, unjust as it
was, may serve to account for the superior applause bestowed upon
personal qualifications, in preference to the social virtues.
Even Marcus Antoninus has been called a hypocrite; but the
wildest scepticism never insinuated that Cæsar might probably be
a coward, or Tully a fool. Wit and valor are qualifications more
easily ascertained than humanity or the love of justice.}

\pagenote[46]{Tacitus has characterized, in a few words, the
principles of the portico: Doctores sapientiæ secutus est, qui
sola bona quæ honesta, main tantum quæ turpia; potentiam,
nobilitatem, æteraque extra... bonis neque malis adnumerant.
Tacit. Hist. iv. 5.}

\pagenote[47]{Before he went on the second expedition against the
Germans, he read lectures of philosophy to the Roman people,
during three days. He had already done the same in the cities of
Greece and Asia. Hist. August. in Cassio, c. 3.}

\pagenote[471]{Cassius was murdered by his own partisans. Vulcat.
Gallic. in Cassio, c. 7. Dion, lxxi. c. 27.—W.}

\pagenote[48]{Dion, l. lxxi. p. 1190. Hist. August. in Avid.
Cassio. Note: See one of the newly discovered passages of Dion
Cassius. Marcus wrote to the senate, who urged the execution of
the partisans of Cassius, in these words: “I entreat and beseech
you to preserve my reign unstained by senatorial blood. None of
your order must perish either by your desire or mine.” Mai.
Fragm. Vatican. ii. p. 224.—M.}

\pagenote[481]{Marcus would not accept the services of any of the
barbarian allies who crowded to his standard in the war against
Avidius Cassius. “Barbarians,” he said, with wise but vain
sagacity, “must not become acquainted with the dissensions of the
Roman people.” Mai. Fragm Vatican l. 224.—M.}

\pagenote[49]{Hist. August. in Marc. Antonin. c. 18.}

If a man were called to fix the period in the history of the
world, during which the condition of the human race was most
happy and prosperous, he would, without hesitation, name that
which elapsed from the death of Domitian to the accession of
Commodus. The vast extent of the Roman empire was governed by
absolute power, under the guidance of virtue and wisdom. The
armies were restrained by the firm but gentle hand of four
successive emperors, whose characters and authority commanded
involuntary respect. The forms of the civil administration were
carefully preserved by Nerva, Trajan, Hadrian, and the Antonines,
who delighted in the image of liberty, and were pleased with
considering themselves as the accountable ministers of the laws.
Such princes deserved the honor of restoring the republic, had
the Romans of their days been capable of enjoying a rational
freedom.

The labors of these monarchs were overpaid by the immense reward
that inseparably waited on their success; by the honest pride of
virtue, and by the exquisite delight of beholding the general
happiness of which they were the authors. A just but melancholy
reflection imbittered, however, the noblest of human enjoyments.
They must often have recollected the instability of a happiness
which depended on the character of single man. The fatal moment
was perhaps approaching, when some licentious youth, or some
jealous tyrant, would abuse, to the destruction, that absolute
power, which they had exerted for the benefit of their people.
The ideal restraints of the senate and the laws might serve to
display the virtues, but could never correct the vices, of the
emperor. The military force was a blind and irresistible
instrument of oppression; and the corruption of Roman manners
would always supply flatterers eager to applaud, and ministers
prepared to serve, the fear or the avarice, the lust or the
cruelty, of their master. These gloomy apprehensions had been
already justified by the experience of the Romans. The annals of
the emperors exhibit a strong and various picture of human
nature, which we should vainly seek among the mixed and doubtful
characters of modern history. In the conduct of those monarchs we
may trace the utmost lines of vice and virtue; the most exalted
perfection, and the meanest degeneracy of our own species. The
golden age of Trajan and the Antonines had been preceded by an
age of iron. It is almost superfluous to enumerate the unworthy
successors of Augustus. Their unparalleled vices, and the
splendid theatre on which they were acted, have saved them from
oblivion. The dark, unrelenting Tiberius, the furious Caligula,
the feeble Claudius, the profligate and cruel Nero, the beastly
Vitellius,\textsuperscript{50} and the timid, inhuman Domitian, are condemned to
everlasting infamy. During fourscore years (excepting only the
short and doubtful respite of Vespasian’s reign)\textsuperscript{51} Rome groaned
beneath an unremitting tyranny, which exterminated the ancient
families of the republic, and was fatal to almost every virtue
and every talent that arose in that unhappy period.

\pagenote[50]{Vitellius consumed in mere eating at least six
millions of our money in about seven months. It is not easy to
express his vices with dignity, or even decency. Tacitus fairly
calls him a hog, but it is by substituting for a coarse word a
very fine image. “At Vitellius, umbraculis hortorum abditus, ut
ignava animalia, quibus si cibum suggeras, jacent torpentque,
præterita, instantia, futura, pari oblivione dimiserat. Atque
illum nemore Aricino desidem et marcentum,” \&c. Tacit. Hist. iii.
36, ii. 95. Sueton. in Vitell. c. 13. Dion. Cassius, l xv. p.
1062.}

\pagenote[51]{The execution of Helvidius Priscus, and of the
virtuous Eponina, disgraced the reign of Vespasian.}

Under the reign of these monsters, the slavery of the Romans was
accompanied with two peculiar circumstances, the one occasioned
by their former liberty, the other by their extensive conquests,
which rendered their condition more completely wretched than that
of the victims of tyranny in any other age or country. From these
causes were derived, 1. The exquisite sensibility of the
sufferers; and, 2. The impossibility of escaping from the hand of
the oppressor.

I. When Persia was governed by the descendants of Sefi, a race of
princes whose wanton cruelty often stained their divan, their
table, and their bed, with the blood of their favorites, there is
a saying recorded of a young nobleman, that he never departed
from the sultan’s presence, without satisfying himself whether
his head was still on his shoulders. The experience of every day
might almost justify the scepticism of Rustan.\textsuperscript{52} Yet the fatal
sword, suspended above him by a single thread, seems not to have
disturbed the slumbers, or interrupted the tranquillity, of the
Persian. The monarch’s frown, he well knew, could level him with
the dust; but the stroke of lightning or apoplexy might be
equally fatal; and it was the part of a wise man to forget the
inevitable calamities of human life in the enjoyment of the
fleeting hour. He was dignified with the appellation of the
king’s slave; had, perhaps, been purchased from obscure parents,
in a country which he had never known; and was trained up from
his infancy in the severe discipline of the seraglio.\textsuperscript{53} His
name, his wealth, his honors, were the gift of a master, who
might, without injustice, resume what he had bestowed. Rustan’s
knowledge, if he possessed any, could only serve to confirm his
habits by prejudices. His language afforded not words for any
form of government, except absolute monarchy. The history of the
East informed him, that such had ever been the condition of
mankind.\textsuperscript{54} The Koran, and the interpreters of that divine book,
inculcated to him, that the sultan was the descendant of the
prophet, and the vicegerent of heaven; that patience was the
first virtue of a Mussulman, and unlimited obedience the great
duty of a subject.

\pagenote[52]{Voyage de Chardin en Perse, vol. iii. p. 293.}

\pagenote[53]{The practice of raising slaves to the great offices
of state is still more common among the Turks than among the
Persians. The miserable countries of Georgia and Circassia supply
rulers to the greatest part of the East.}

\pagenote[54]{Chardin says, that European travellers have
diffused among the Persians some ideas of the freedom and
mildness of our governments. They have done them a very ill
office.}

The minds of the Romans were very differently prepared for
slavery. Oppressed beneath the weight of their own corruption and
of military violence, they for a long while preserved the
sentiments, or at least the ideas, of their free-born ancestors.
The education of Helvidius and Thrasea, of Tacitus and Pliny, was
the same as that of Cato and Cicero. From Grecian philosophy,
they had imbibed the justest and most liberal notions of the
dignity of human nature, and the origin of civil society. The
history of their own country had taught them to revere a free, a
virtuous, and a victorious commonwealth; to abhor the successful
crimes of Cæsar and Augustus; and inwardly to despise those
tyrants whom they adored with the most abject flattery. As
magistrates and senators they were admitted into the great
council, which had once dictated laws to the earth, whose
authority was so often prostituted to the vilest purposes of
tyranny. Tiberius, and those emperors who adopted his maxims,
attempted to disguise their murders by the formalities of
justice, and perhaps enjoyed a secret pleasure in rendering the
senate their accomplice as well as their victim. By this
assembly, the last of the Romans were condemned for imaginary
crimes and real virtues. Their infamous accusers assumed the
language of independent patriots, who arraigned a dangerous
citizen before the tribunal of his country; and the public
service was rewarded by riches and honors.\textsuperscript{55} The servile judges
professed to assert the majesty of the commonwealth, violated in
the person of its first magistrate,\textsuperscript{56} whose clemency they most
applauded when they trembled the most at his inexorable and
impending cruelty.\textsuperscript{57} The tyrant beheld their baseness with just
contempt, and encountered their secret sentiments of detestation
with sincere and avowed hatred for the whole body of the senate.

\pagenote[55]{They alleged the example of Scipio and Cato,
(Tacit. Annal. iii. 66.) Marcellus Epirus and Crispus Vibius had
acquired two millions and a half under Nero. Their wealth, which
aggravated their crimes, protected them under Vespasian. See
Tacit. Hist. iv. 43. Dialog. de Orator. c. 8. For one accusation,
Regulus, the just object of Pliny’s satire, received from the
senate the consular ornaments, and a present of sixty thousand
pounds.}

\pagenote[56]{The crime of majesty was formerly a treasonable
offence against the Roman people. As tribunes of the people,
Augustus and Tiberius applied tit to their own persons, and
extended it to an infinite latitude. Note: It was Tiberius, not
Augustus, who first took in this sense the words crimen læsæ
majestatis. Bachii Trajanus, 27. —W.}

\pagenote[57]{After the virtuous and unfortunate widow of
Germanicus had been put to death, Tiberius received the thanks of
the senate for his clemency. she had not been publicly strangled;
nor was the body drawn with a hook to the Gemoniæ, where those of
common male factors were exposed. See Tacit. Annal. vi. 25.
Sueton. in Tiberio c. 53.}

II. The division of Europe into a number of independent states,
connected, however, with each other by the general resemblance of
religion, language, and manners, is productive of the most
beneficial consequences to the liberty of mankind. A modern
tyrant, who should find no resistance either in his own breast,
or in his people, would soon experience a gentle restraint from
the example of his equals, the dread of present censure, the
advice of his allies, and the apprehension of his enemies. The
object of his displeasure, escaping from the narrow limits of his
dominions, would easily obtain, in a happier climate, a secure
refuge, a new fortune adequate to his merit, the freedom of
complaint, and perhaps the means of revenge. But the empire of
the Romans filled the world, and when the empire fell into the
hands of a single person, the world became a safe and dreary
prison for his enemies. The slave of Imperial despotism, whether
he was condemned to drag his gilded chain in rome and the senate,
or to were out a life of exile on the barren rock of Seriphus, or
the frozen bank of the Danube, expected his fate in silent
despair.\textsuperscript{58} To resist was fatal, and it was impossible to fly. On
every side he was encompassed with a vast extent of sea and land,
which he could never hope to traverse without being discovered,
seized, and restored to his irritated master. Beyond the
frontiers, his anxious view could discover nothing, except the
ocean, inhospitable deserts, hostile tribes of barbarians, of
fierce manners and unknown language, or dependent kings, who
would gladly purchase the emperor’s protection by the sacrifice
of an obnoxious fugitive.\textsuperscript{59} “Wherever you are,” said Cicero to
the exiled Marcellus, “remember that you are equally within the
power of the conqueror.”\textsuperscript{60}

\pagenote[58]{Seriphus was a small rocky island in the Ægean Sea,
the inhabitants of which were despised for their ignorance and
obscurity. The place of Ovid’s exile is well known, by his just,
but unmanly lamentations. It should seem, that he only received
an order to leave rome in so many days, and to transport himself
to Tomi. Guards and jailers were unnecessary.}

\pagenote[59]{Under Tiberius, a Roman knight attempted to fly to
the Parthians. He was stopped in the straits of Sicily; but so
little danger did there appear in the example, that the most
jealous of tyrants disdained to punish it. Tacit. Annal. vi. 14.}

\pagenote[60]{Cicero ad Familiares, iv. 7.}

