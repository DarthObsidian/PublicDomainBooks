\chapter{Emperors Decius, Gallus, Æmilianus, Valerian And Gallienus}
\section{Part \thesection}

\textit{The Emperors Decius, Gallus, Æmilianus, Valerian, And
Gallienus.—The General Irruption Of The Barbari Ans.—The Thirty
Tyrants.}
\vspace{\onelineskip}

From the great secular games celebrated by Philip, to the death
of the emperor Gallienus, there elapsed twenty years of shame and
misfortune. During that calamitous period, every instant of time
was marked, every province of the Roman world was afflicted, by
barbarous invaders, and military tyrants, and the ruined empire
seemed to approach the last and fatal moment of its dissolution.
The confusion of the times, and the scarcity of authentic
memorials, oppose equal difficulties to the historian, who
attempts to preserve a clear and unbroken thread of narration.
Surrounded with imperfect fragments, always concise, often
obscure, and sometimes contradictory, he is reduced to collect,
to compare, and to conjecture: and though he ought never to place
his conjectures in the rank of facts, yet the knowledge of human
nature, and of the sure operation of its fierce and unrestrained
passions, might, on some occasions, supply the want of historical
materials.

There is not, for instance, any difficulty in conceiving, that
the successive murders of so many emperors had loosened all the
ties of allegiance between the prince and people; that all the
generals of Philip were disposed to imitate the example of their
master; and that the caprice of armies, long since habituated to
frequent and violent revolutions, might every day raise to the
throne the most obscure of their fellow-soldiers. History can
only add, that the rebellion against the emperor Philip broke out
in the summer of the year two hundred and forty-nine, among the
legions of Mæsia; and that a subaltern officer,\textsuperscript{1} named Marinus,
was the object of their seditious choice. Philip was alarmed. He
dreaded lest the treason of the Mæsian army should prove the
first spark of a general conflagration. Distracted with the
consciousness of his guilt and of his danger, he communicated the
intelligence to the senate. A gloomy silence prevailed, the
effect of fear, and perhaps of disaffection; till at length
Decius, one of the assembly, assuming a spirit worthy of his
noble extraction, ventured to discover more intrepidity than the
emperor seemed to possess. He treated the whole business with
contempt, as a hasty and inconsiderate tumult, and Philip’s rival
as a phantom of royalty, who in a very few days would be
destroyed by the same inconstancy that had created him. The
speedy completion of the prophecy inspired Philip with a just
esteem for so able a counsellor; and Decius appeared to him the
only person capable of restoring peace and discipline to an army
whose tumultuous spirit did not immediately subside after the
murder of Marinus. Decius,\textsuperscript{2} who long resisted his own
nomination, seems to have insinuated the danger of presenting a
leader of merit to the angry and apprehensive minds of the
soldiers; and his prediction was again confirmed by the event.
The legions of Mæsia forced their judge to become their
accomplice. They left him only the alternative of death or the
purple. His subsequent conduct, after that decisive measure, was
unavoidable. He conducted, or followed, his army to the confines
of Italy, whither Philip, collecting all his force to repel the
formidable competitor whom he had raised up, advanced to meet
him. The Imperial troops were superior in number; but the rebels
formed an army of veterans, commanded by an able and experienced
leader. Philip was either killed in the battle, or put to death a
few days afterwards at Verona. His son and associate in the
empire was massacred at Rome by the Prætorian guards; and the
victorious Decius, with more favorable circumstances than the
ambition of that age can usually plead, was universally
acknowledged by the senate and provinces. It is reported, that,
immediately after his reluctant acceptance of the title of
Augustus, he had assured Philip, by a private message, of his
innocence and loyalty, solemnly protesting, that, on his arrival
on Italy, he would resign the Imperial ornaments, and return to
the condition of an obedient subject. His professions might be
sincere; but in the situation where fortune had placed him, it
was scarcely possible that he could either forgive or be
forgiven.\textsuperscript{3}

\pagenote[1]{The expression used by Zosimus and Zonaras may
signify that Marinus commanded a century, a cohort, or a legion.}

\pagenote[2]{His birth at Bubalia, a little village in Pannonia,
(Eutrop. ix. Victor. in Cæsarib. et Epitom.,) seems to
contradict, unless it was merely accidental, his supposed descent
from the Decii. Six hundred years had bestowed nobility on the
Decii: but at the commencement of that period, they were only
plebeians of merit, and among the first who shared the consulship
with the haughty patricians. Plebeine Deciorum animæ, \&c.
Juvenal, Sat. viii. 254. See the spirited speech of Decius, in
Livy. x. 9, 10.}

\pagenote[3]{Zosimus, l. i. p. 20, c. 22. Zonaras, l. xii. p.
624, edit. Louvre.}

The emperor Decius had employed a few months in the works of
peace and the administration of justice, when he was summoned to
the banks of the Danube by the invasion of the Goths. This is the
first considerable occasion in which history mentions that great
people, who afterwards broke the Roman power, sacked the Capitol,
and reigned in Gaul, Spain, and Italy. So memorable was the part
which they acted in the subversion of the Western empire, that
the name of Goths is frequently but improperly used as a general
appellation of rude and warlike barbarism.

In the beginning of the sixth century, and after the conquest of
Italy, the Goths, in possession of present greatness, very
naturally indulged themselves in the prospect of past and of
future glory. They wished to preserve the memory of their
ancestors, and to transmit to posterity their own achievements.
The principal minister of the court of Ravenna, the learned
Cassiodorus, gratified the inclination of the conquerors in a
Gothic history, which consisted of twelve books, now reduced to
the imperfect abridgment of Jornandes.\textsuperscript{4} These writers passed
with the most artful conciseness over the misfortunes of the
nation, celebrated its successful valor, and adorned the triumph
with many Asiatic trophies, that more properly belonged to the
people of Scythia. On the faith of ancient songs, the uncertain,
but the only memorials of barbarians, they deduced the first
origin of the Goths from the vast island, or peninsula, of
Scandinavia.\textsuperscript{5} \textsuperscript{501} That extreme country of the North was not
unknown to the conquerors of Italy: the ties of ancient
consanguinity had been strengthened by recent offices of
friendship; and a Scandinavian king had cheerfully abdicated his
savage greatness, that he might pass the remainder of his days in
the peaceful and polished court of Ravenna.\textsuperscript{6} Many vestiges,
which cannot be ascribed to the arts of popular vanity, attest
the ancient residence of the Goths in the countries beyond the
Rhine. From the time of the geographer Ptolemy, the southern part
of Sweden seems to have continued in the possession of the less
enterprising remnant of the nation, and a large territory is even
at present divided into east and west Gothland. During the middle
ages, (from the ninth to the twelfth century,) whilst
Christianity was advancing with a slow progress into the North,
the Goths and the Swedes composed two distinct and sometimes
hostile members of the same monarchy.\textsuperscript{7} The latter of these two
names has prevailed without extinguishing the former. The Swedes,
who might well be satisfied with their own fame in arms, have, in
every age, claimed the kindred glory of the Goths. In a moment of
discontent against the court of Rome, Charles the Twelfth
insinuated, that his victorious troops were not degenerated from
their brave ancestors, who had already subdued the mistress of
the world.\textsuperscript{8}

\pagenote[4]{See the prefaces of Cassiodorus and Jornandes; it is
surprising that the latter should be omitted in the excellent
edition, published by Grotius, of the Gothic writers.}

\pagenote[5]{On the authority of Ablavius, Jornandes quotes some
old Gothic chronicles in verse. De Reb. Geticis, c. 4.}

\pagenote[501]{The Goths have inhabited Scandinavia, but it was
not their original habitation. This great nation was anciently of
the Suevian race; it occupied, in the time of Tacitus, and long
before, Mecklenburgh, Pomerania Southern Prussia and the
north-west of Poland. A little before the birth of J. C., and in
the first years of that century, they belonged to the kingdom of
Marbod, king of the Marcomanni: but Cotwalda, a young Gothic
prince, delivered them from that tyranny, and established his own
power over the kingdom of the Marcomanni, already much weakened
by the victories of Tiberius. The power of the Goths at that time
must have been great: it was probably from them that the Sinus
Codanus (the Baltic) took this name, as it was afterwards called
Mare Suevicum, and Mare Venedicum, during the superiority of the
proper Suevi and the Venedi. The epoch in which the Goths passed
into Scandinavia is unknown. See Adelung, Hist. of Anc. Germany,
p. 200. Gatterer, Hist. Univ. 458.—G. ——M. St. Martin observes,
that the Scandinavian descent of the Goths rests on the authority
of Jornandes, who professed to derive it from the traditions of
the Goths. He is supported by Procopius and Paulus Diaconus. Yet
the Goths are unquestionably the same with the Getæ of the
earlier historians. St. Martin, note on Le Beau, Hist. du bas
Empire, iii. 324. The identity of the Getæ and Goths is by no
means generally admitted. On the whole, they seem to be one vast
branch of the Indo-Teutonic race, who spread irregularly towards
the north of Europe, and at different periods, and in different
regions, came in contact with the more civilized nations of the
south. At this period, there seems to have been a reflux of these
Gothic tribes from the North. Malte Brun considers that there are
strong grounds for receiving the Islandic traditions commented by
the Danish Varro, M. Suhm. From these, and the voyage of Pytheas,
which Malte Brun considers genuine, the Goths were in possession
of Scandinavia, Ey-Gothland, 250 years before J. C., and of a
tract on the continent (Reid-Gothland) between the mouths of the
Vistula and the Oder. In their southern migration, they followed
the course of the Vistula; afterwards, of the Dnieper. Malte
Brun, Geogr. i. p. 387, edit. 1832. Geijer, the historian of
Sweden, ably maintains the Scandinavian origin of the Goths. The
Gothic language, according to Bopp, is the link between the
Sanscrit and the modern Teutonic dialects: “I think that I am
reading Sanscrit when I am reading Olphilas.” Bopp, Conjugations
System der Sanscrit Sprache, preface, p. x—M.}

\pagenote[6]{Jornandes, c. 3.}

\pagenote[7]{See in the Prolegomena of Grotius some large
extracts from Adam of Bremen, and Saxo-Grammaticus. The former
wrote in the year 1077, the latter flourished about the year
1200.}

\pagenote[8]{Voltaire, Histoire de Charles XII. l. iii. When the
Austrians desired the aid of the court of Rome against Gustavus
Adolphus, they always represented that conqueror as the lineal
successor of Alaric. Harte’s History of Gustavus, vol. ii. p.
123.}

Till the end of the eleventh century, a celebrated temple
subsisted at Upsal, the most considerable town of the Swedes and
Goths. It was enriched with the gold which the Scandinavians had
acquired in their piratical adventures, and sanctified by the
uncouth representations of the three principal deities, the god
of war, the goddess of generation, and the god of thunder. In the
general festival, that was solemnized every ninth year, nine
animals of every species (without excepting the human) were
sacrificed, and their bleeding bodies suspended in the sacred
grove adjacent to the temple.\textsuperscript{9} The only traces that now subsist
of this barbaric superstition are contained in the Edda,\textsuperscript{901} a
system of mythology, compiled in Iceland about the thirteenth
century, and studied by the learned of Denmark and Sweden, as the
most valuable remains of their ancient traditions.

\pagenote[9]{See Adam of Bremen in Grotii Prolegomenis, p. 105.
The temple of Upsal was destroyed by Ingo, king of Sweden, who
began his reign in the year 1075, and about fourscore years
afterwards, a Christian cathedral was erected on its ruins. See
Dalin’s History of Sweden, in the Bibliotheque Raisonee.}

\pagenote[901]{The Eddas have at length been made accessible to
European scholars by the completion of the publication of the
Sæmundine Edda by the Arna Magnæan Commission, in 3 vols. 4to.,
with a copious lexicon of northern mythology.—M.}

Notwithstanding the mysterious obscurity of the Edda, we can
easily distinguish two persons confounded under the name of Odin;
the god of war, and the great legislator of Scandinavia. The
latter, the Mahomet of the North, instituted a religion adapted
to the climate and to the people. Numerous tribes on either side
of the Baltic were subdued by the invincible valor of Odin, by
his persuasive eloquence, and by the fame which he acquired of a
most skilful magician. The faith that he had propagated, during a
long and prosperous life, he confirmed by a voluntary death.
Apprehensive of the ignominious approach of disease and
infirmity, he resolved to expire as became a warrior. In a solemn
assembly of the Swedes and Goths, he wounded himself in nine
mortal places, hastening away (as he asserted with his dying
voice) to prepare the feast of heroes in the palace of the God of
war.\textsuperscript{10}

\pagenote[10]{Mallet, Introduction a l’Histoire du Dannemarc.}

The native and proper habitation of Odin is distinguished by the
appellation of As-gard. The happy resemblance of that name with
As-burg, or As-of,\textsuperscript{11} words of a similar signification, has given
rise to an historical system of so pleasing a contexture, that we
could almost wish to persuade ourselves of its truth. It is
supposed that Odin was the chief of a tribe of barbarians which
dwelt on the banks of the Lake Mæotis, till the fall of
Mithridates and the arms of Pompey menaced the North with
servitude. That Odin, yielding with indignant fury to a power he
was unable to resist, conducted his tribe from the frontiers of
the Asiatic Sarmatia into Sweden, with the great design of
forming, in that inaccessible retreat of freedom, a religion and
a people which, in some remote age, might be subservient to his
immortal revenge; when his invincible Goths, armed with martial
fanaticism, should issue in numerous swarms from the neighborhood
of the Polar circle, to chastise the oppressors of mankind.\textsuperscript{12}

\pagenote[11]{Mallet, c. iv. p. 55, has collected from Strabo,
Pliny, Ptolemy, and Stephanus Byzantinus, the vestiges of such a
city and people.}

\pagenote[12]{This wonderful expedition of Odin, which, by
deducting the enmity of the Goths and Romans from so memorable a
cause, might supply the noble groundwork of an epic poem, cannot
safely be received as authentic history. According to the obvious
sense of the Edda, and the interpretation of the most skilful
critics, As-gard, instead of denoting a real city of the Asiatic
Sarmatia, is the fictitious appellation of the mystic abode of
the gods, the Olympus of Scandinavia; from whence the prophet was
supposed to descend, when he announced his new religion to the
Gothic nations, who were already seated in the southern parts of
Sweden. * Note: A curious letter may be consulted on this subject
from the Swede, Ihre counsellor in the Chancery of Upsal, printed
at Upsal by Edman, in 1772 and translated into German by M.
Schlozer. Gottingen, printed for Dietericht, 1779.—G. ——Gibbon,
at a later period of his work, recanted his opinion of the truth
of this expedition of Odin. The Asiatic origin of the Goths is
almost certain from the affinity of their language to the
Sanscrit and Persian; but their northern writers, when all
mythology was reduced to hero worship.—M.}

If so many successive generations of Goths were capable of
preserving a faint tradition of their Scandinavian origin, we
must not expect, from such unlettered barbarians, any distinct
account of the time and circumstances of their emigration. To
cross the Baltic was an easy and natural attempt. The inhabitants
of Sweden were masters of a sufficient number of large vessels,
with oars,\textsuperscript{13} and the distance is little more than one hundred
miles from Carlscroon to the nearest ports of Pomerania and
Prussia. Here, at length, we land on firm and historic ground. At
least as early as the Christian æra,\textsuperscript{14} and as late as the age of
the Antonines,\textsuperscript{15} the Goths were established towards the mouth of
the Vistula, and in that fertile province where the commercial
cities of Thorn, Elbing, Köningsberg, and Dantzick, were long
afterwards founded.\textsuperscript{16} Westward of the Goths, the numerous tribes
of the Vandals were spread along the banks of the Oder, and the
sea-coast of Pomerania and Mecklenburgh. A striking resemblance
of manners, complexion, religion, and language, seemed to
indicate that the Vandals and the Goths were originally one great
people.\textsuperscript{17} The latter appear to have been subdivided into
Ostrogoths, Visigoths, and Gepidæ.\textsuperscript{18} The distinction among the
Vandals was more strongly marked by the independent names of
Heruli, Burgundians, Lombards, and a variety of other petty
states, many of which, in a future age, expanded themselves into
powerful monarchies.\textsuperscript{181}

\pagenote[13]{Tacit. Germania, c. 44.}

\pagenote[14]{Tacit. Annal. ii. 62. If we could yield a firm
assent to the navigations of Pytheas of Marseilles, we must allow
that the Goths had passed the Baltic at least three hundred years
before Christ.}

\pagenote[15]{Ptolemy, l. ii.}

\pagenote[16]{By the German colonies who followed the arms of the
Teutonic knights. The conquest and conversion of Prussia were
completed by those adventurers in the thirteenth century.}

\pagenote[17]{Pliny (Hist. Natur. iv. 14) and Procopius (in Bell.
Vandal. l. i. c. l) agree in this opinion. They lived in distant
ages, and possessed different means of investigating the truth.}

\pagenote[18]{The Ostro and Visi, the eastern and western Goths,
obtained those denominations from their original seats in
Scandinavia. In all their future marches and settlements they
preserved, with their names, the same relative situation. When
they first departed from Sweden, the infant colony was contained
in three vessels. The third, being a heavy sailer, lagged behind,
and the crew, which afterwards swelled into a nation, received
from that circumstance the appellation of Gepidæ or Loiterers.
Jornandes, c. 17. * Note: It was not in Scandinavia that the
Goths were divided into Ostrogoths and Visigoths; that division
took place after their irruption into Dacia in the third century:
those who came from Mecklenburgh and Pomerania were called
Visigoths; those who came from the south of Prussia, and the
northwest of Poland, called themselves Ostrogoths. Adelung, Hist.
All. p. 202 Gatterer, Hist. Univ. 431.—G.}

\pagenote[181]{This opinion is by no means probable. The Vandals
and the Goths equally belonged to the great division of the
Suevi, but the two tribes were very different. Those who have
treated on this part of history, appear to me to have neglected
to remark that the ancients almost always gave the name of the
dominant and conquering people to all the weaker and conquered
races. So Pliny calls Vindeli, Vandals, all the people of the
north-east of Europe, because at that epoch the Vandals were
doubtless the conquering tribe. Cæsar, on the contrary, ranges
under the name of Suevi, many of the tribes whom Pliny reckons as
Vandals, because the Suevi, properly so called, were then the
most powerful tribe in Germany. When the Goths, become in their
turn conquerors, had subjugated the nations whom they encountered
on their way, these nations lost their name with their liberty,
and became of Gothic origin. The Vandals themselves were then
considered as Goths; the Heruli, the Gepidæ, \&c., suffered the
same fate. A common origin was thus attributed to tribes who had
only been united by the conquests of some dominant nation, and
this confusion has given rise to a number of historical
errors.—G. ——M. St. Martin has a learned note (to Le Beau, v.
261) on the origin of the Vandals. The difficulty appears to be
in rejecting the close analogy of the name with the Vend or
Wendish race, who were of Sclavonian, not of Suevian or German,
origin. M. St. Martin supposes that the different races spread
from the head of the Adriatic to the Baltic, and even the Veneti,
on the shores of the Adriatic, the Vindelici, the tribes which
gave their name to Vindobena, Vindoduna, Vindonissa, were
branches of the same stock with the Sclavonian Venedi, who at one
time gave their name to the Baltic; that they all spoke dialects
of the Wendish language, which still prevails in Carinthia,
Carniola, part of Bohemia, and Lusatia, and is hardly extinct in
Mecklenburgh and Pomerania. The Vandal race, once so fearfully
celebrated in the annals of mankind, has so utterly perished from
the face of the earth, that we are not aware that any vestiges of
their language can be traced, so as to throw light on the
disputed question of their German, their Sclavonian, or
independent origin. The weight of ancient authority seems against
M. St. Martin’s opinion. Compare, on the Vandals, Malte Brun.
394. Also Gibbon’s note, c. xli. n. 38.—M.}

In the age of the Antonines, the Goths were still seated in
Prussia. About the reign of Alexander Severus, the Roman province
of Dacia had already experienced their proximity by frequent and
destructive inroads.\textsuperscript{19} In this interval, therefore, of about
seventy years we must place the second migration of the Goths
from the Baltic to the Euxine; but the cause that produced it
lies concealed among the various motives which actuate the
conduct of unsettled barbarians. Either a pestilence or a famine,
a victory or a defeat, an oracle of the gods or the eloquence of
a daring leader, were sufficient to impel the Gothic arms on the
milder climates of the south. Besides the influence of a martial
religion, the numbers and spirit of the Goths were equal to the
most dangerous adventures. The use of round bucklers and short
swords rendered them formidable in a close engagement; the manly
obedience which they yielded to hereditary kings, gave uncommon
union and stability to their councils;\textsuperscript{20} and the renowned Amala,
the hero of that age, and the tenth ancestor of Theodoric, king
of Italy, enforced, by the ascendant of personal merit, the
prerogative of his birth, which he derived from the \textit{Anses}, or
demigods of the Gothic nation.\textsuperscript{21}

\pagenote[19]{See a fragment of Peter Patricius in the Excerpta
Legationum and with regard to its probable date, see Tillemont,
Hist, des Empereurs, tom. iii. p. 346.}

\pagenote[20]{Omnium harum gentium insigne, rotunda scuta, breves
gladii, et erga rages obsequium. Tacit. Germania, c. 43. The
Goths probably acquired their iron by the commerce of amber.}

\pagenote[21]{Jornandes, c. 13, 14.}

The fame of a great enterprise excited the bravest warriors from
all the Vandalic states of Germany, many of whom are seen a few
years afterwards combating under the common standard of the
Goths.\textsuperscript{22} The first motions of the emigrants carried them to the
banks of the Prypec, a river universally conceived by the
ancients to be the southern branch of the Borysthenes.\textsuperscript{23} The
windings of that great stream through the plains of Poland and
Russia gave a direction to their line of march, and a constant
supply of fresh water and pasturage to their numerous herds of
cattle. They followed the unknown course of the river, confident
in their valor, and careless of whatever power might oppose their
progress. The Bastarnæ and the Venedi were the first who
presented themselves; and the flower of their youth, either from
choice or compulsion, increased the Gothic army. The Bastarnæ
dwelt on the northern side of the Carpathian Mountains: the
immense tract of land that separated the Bastarnæ from the
savages of Finland was possessed, or rather wasted, by the
Venedi;\textsuperscript{24} we have some reason to believe that the first of these
nations, which distinguished itself in the Macedonian war,\textsuperscript{25} and
was afterwards divided into the formidable tribes of the Peucini,
the Borani, the Carpi, \&c., derived its origin from the Germans.\textsuperscript{251}
With better authority, a Sarmatian extraction may be assigned
to the Venedi, who rendered themselves so famous in the middle
ages.\textsuperscript{26} But the confusion of blood and manners on that doubtful
frontier often perplexed the most accurate observers.\textsuperscript{27} As the
Goths advanced near the Euxine Sea, they encountered a purer race
of Sarmatians, the Jazyges, the Alani,\textsuperscript{271} and the Roxolani; and
they were probably the first Germans who saw the mouths of the
Borysthenes, and of the Tanais. If we inquire into the
characteristic marks of the people of Germany and of Sarmatia, we
shall discover that those two great portions of human kind were
principally distinguished by fixed huts or movable tents, by a
close dress or flowing garments, by the marriage of one or of
several wives, by a military force, consisting, for the most
part, either of infantry or cavalry; and above all, by the use of
the Teutonic, or of the Sclavonian language; the last of which
has been diffused by conquest, from the confines of Italy to the
neighborhood of Japan.

\pagenote[22]{The Heruli, and the Uregundi or Burgundi, are
particularly mentioned. See Mascou’s History of the Germans, l.
v. A passage in the Augustan History, p. 28, seems to allude to
this great emigration. The Marcomannic war was partly occasioned
by the pressure of barbarous tribes, who fled before the arms of
more northern barbarians.}

\pagenote[23]{D’Anville, Geographie Ancienne, and the third part
of his incomparable map of Europe.}

\pagenote[24]{Tacit. Germania, c. 46.}

\pagenote[25]{Cluver. Germ. Antiqua, l. iii. c. 43.}

\pagenote[251]{The Bastarnæ cannot be considered original
inhabitants of Germany Strabo and Tacitus appear to doubt it;
Pliny alone calls them Germans: Ptolemy and Dion treat them as
Scythians, a vague appellation at this period of history; Livy,
Plutarch, and Diodorus Siculus, call them Gauls, and this is the
most probable opinion. They descended from the Gauls who entered
Germany under Signoesus. They are always found associated with
other Gaulish tribes, such as the Boll, the Taurisci, \&c., and
not to the German tribes. The names of their chiefs or princes,
Chlonix, Chlondicus. Deldon, are not German names. Those who were
settled in the island of Peuce in the Danube, took the name of
Peucini. The Carpi appear in 237 as a Suevian tribe who had made
an irruption into Mæsia. Afterwards they reappear under the
Ostrogoths, with whom they were probably blended. Adelung, p.
236, 278.—G.}

\pagenote[26]{The Venedi, the Slavi, and the Antes, were the
three great tribes of the same people. Jornandes, 24. * Note
Dagger: They formed the great Sclavonian nation.—G.}

\pagenote[27]{Tacitus most assuredly deserves that title, and
even his cautious suspense is a proof of his diligent inquiries.}

\pagenote[271]{Jac. Reineggs supposed that he had found, in the
mountains of Caucasus, some descendants of the Alani. The Tartars
call them Edeki-Alan: they speak a peculiar dialect of the
ancient language of the Tartars of Caucasus. See J. Reineggs’
Descr. of Caucasus, p. 11, 13.—G. According to Klaproth, they are
the Ossetes of the present day in Mount Caucasus and were the
same with the Albanians of antiquity. Klaproth, Hist. de l’Asie,
p. 180.—M.}

\section{Part \thesection.}

The Goths were now in possession of the Ukraine, a country of
considerable extent and uncommon fertility, intersected with
navigable rivers, which, from either side, discharge themselves
into the Borysthenes; and interspersed with large and lofty
forests of oaks. The plenty of game and fish, the innumerable
bee-hives deposited in the hollow of old trees, and in the
cavities of rocks, and forming, even in that rude age, a valuable
branch of commerce, the size of the cattle, the temperature of
the air, the aptness of the soil for every species of grain, and
the luxuriancy of the vegetation, all displayed the liberality of
Nature, and tempted the industry of man.\textsuperscript{28} But the Goths
withstood all these temptations, and still adhered to a life of
idleness, of poverty, and of rapine.

\pagenote[28]{Genealogical History of the Tartars, p. 593. Mr.
Bell (vol. ii. p 379) traversed the Ukraine, in his journey from
Petersburgh to Constantinople. The modern face of the country is
a just representation of the ancient, since, in the hands of the
Cossacks, it still remains in a state of nature.}

The Scythian hordes, which, towards the east, bordered on the new
settlements of the Goths, presented nothing to their arms, except
the doubtful chance of an unprofitable victory. But the prospect
of the Roman territories was far more alluring; and the fields of
Dacia were covered with rich harvests, sown by the hands of an
industrious, and exposed to be gathered by those of a warlike,
people. It is probable that the conquests of Trajan, maintained
by his successors, less for any real advantage than for ideal
dignity, had contributed to weaken the empire on that side. The
new and unsettled province of Dacia was neither strong enough to
resist, nor rich enough to satiate, the rapaciousness of the
barbarians. As long as the remote banks of the Niester were
considered as the boundary of the Roman power, the fortifications
of the Lower Danube were more carelessly guarded, and the
inhabitants of Mæsia lived in supine security, fondly conceiving
themselves at an inaccessible distance from any barbarian
invaders. The irruptions of the Goths, under the reign of Philip,
fatally convinced them of their mistake. The king, or leader, of
that fierce nation, traversed with contempt the province of
Dacia, and passed both the Niester and the Danube without
encountering any opposition capable of retarding his progress.
The relaxed discipline of the Roman troops betrayed the most
important posts, where they were stationed, and the fear of
deserved punishment induced great numbers of them to enlist under
the Gothic standard. The various multitude of barbarians
appeared, at length, under the walls of Marcianopolis, a city
built by Trajan in honor of his sister, and at that time the
capital of the second Mæsia.\textsuperscript{29} The inhabitants consented to
ransom their lives and property by the payment of a large sum of
money, and the invaders retreated back into their deserts,
animated, rather than satisfied, with the first success of their
arms against an opulent but feeble country. Intelligence was soon
transmitted to the emperor Decius, that Cniva, king of the Goths,
had passed the Danube a second time, with more considerable
forces; that his numerous detachments scattered devastation over
the province of Mæsia, whilst the main body of the army,
consisting of seventy thousand Germans and Sarmatians, a force
equal to the most daring achievements, required the presence of
the Roman monarch, and the exertion of his military power.

\pagenote[29]{In the sixteenth chapter of Jornandes, instead of
secundo Mæsiam we may venture to substitute secundam, the second
Mæsia, of which Marcianopolis was certainly the capital. (See
Hierocles de Provinciis, and Wesseling ad locum, p. 636.
Itinerar.) It is surprising how this palpable error of the scribe
should escape the judicious correction of Grotius. Note: Luden
has observed that Jornandes mentions two passages over the
Danube; this relates to the second irruption into Mæsia.
Geschichte des T V. ii. p. 448.—M.}

Decius found the Goths engaged before Nicopolis, one of the many
monuments of Trajan’s victories.\textsuperscript{30} On his approach they raised
the siege, but with a design only of marching away to a conquest
of greater importance, the siege of Philippopolis, a city of
Thrace, founded by the father of Alexander, near the foot of
Mount Hæmus.\textsuperscript{31} Decius followed them through a difficult country,
and by forced marches; but when he imagined himself at a
considerable distance from the rear of the Goths, Cniva turned
with rapid fury on his pursuers. The camp of the Romans was
surprised and pillaged, and, for the first time, their emperor
fled in disorder before a troop of half-armed barbarians. After a
long resistance, Philoppopolis, destitute of succor, was taken by
storm. A hundred thousand persons are reported to have been
massacred in the sack of that great city.\textsuperscript{32} Many prisoners of
consequence became a valuable accession to the spoil; and
Priscus, a brother of the late emperor Philip, blushed not to
assume the purple, under the protection of the barbarous enemies
of Rome.\textsuperscript{33} The time, however, consumed in that tedious siege,
enabled Decius to revive the courage, restore the discipline, and
recruit the numbers of his troops. He intercepted several parties
of Carpi, and other Germans, who were hastening to share the
victory of their countrymen,\textsuperscript{34} intrusted the passes of the
mountains to officers of approved valor and fidelity,\textsuperscript{35} repaired
and strengthened the fortifications of the Danube, and exerted
his utmost vigilance to oppose either the progress or the retreat
of the Goths. Encouraged by the return of fortune, he anxiously
waited for an opportunity to retrieve, by a great and decisive
blow, his own glory, and that of the Roman arms.\textsuperscript{36}

\pagenote[30]{The place is still called Nicop. D’Anville,
Geographie Ancienne, tom. i. p. 307. The little stream, on whose
banks it stood, falls into the Danube.}

\pagenote[31]{Stephan. Byzant. de Urbibus, p. 740. Wesseling,
Itinerar. p. 136. Zonaras, by an odd mistake, ascribes the
foundation of Philippopolis to the immediate predecessor of
Decius. * Note: Now Philippopolis or Philiba; its situation among
the hills caused it to be also called Trimontium. D’Anville,
Geog. Anc. i. 295.—G.}

\pagenote[32]{Ammian. xxxi. 5.}

\pagenote[33]{Aurel. Victor. c. 29.}

\pagenote[34]{Victoriæ Carpicæ, on some medals of Decius,
insinuate these advantages.}

\pagenote[35]{Claudius (who afterwards reigned with so much
glory) was posted in the pass of Thermopylæ with 200 Dardanians,
100 heavy and 160 light horse, 60 Cretan archers, and 1000
well-armed recruits. See an original letter from the emperor to
his officer, in the Augustan History, p. 200.}

\pagenote[36]{Jornandes, c. 16—18. Zosimus, l. i. p. 22. In the
general account of this war, it is easy to discover the opposite
prejudices of the Gothic and the Grecian writer. In carelessness
alone they are alike.}

At the same time when Decius was struggling with the violence of
the tempest, his mind, calm and deliberate amidst the tumult of
war, investigated the more general causes that, since the age of
the Antonines, had so impetuously urged the decline of the Roman
greatness. He soon discovered that it was impossible to replace
that greatness on a permanent basis without restoring public
virtue, ancient principles and manners, and the oppressed majesty
of the laws. To execute this noble but arduous design, he first
resolved to revive the obsolete office of censor; an office
which, as long as it had subsisted in its pristine integrity, had
so much contributed to the perpetuity of the state,\textsuperscript{37} till it
was usurped and gradually neglected by the Cæsars.\textsuperscript{38} Conscious
that the favor of the sovereign may confer power, but that the
esteem of the people can alone bestow authority, he submitted the
choice of the censor to the unbiased voice of the senate. By
their unanimous votes, or rather acclamations, Valerian, who was
afterwards emperor, and who then served with distinction in the
army of Decius, was declared the most worthy of that exalted
honor. As soon as the decree of the senate was transmitted to the
emperor, he assembled a great council in his camp, and before the
investiture of the censor elect, he apprised him of the
difficulty and importance of his great office. “Happy Valerian,”
said the prince to his distinguished subject, “happy in the
general approbation of the senate and of the Roman republic!
Accept the censorship of mankind; and judge of our manners. You
will select those who deserve to continue members of the senate;
you will restore the equestrian order to its ancient splendor;
you will improve the revenue, yet moderate the public burdens.
You will distinguish into regular classes the various and
infinite multitude of citizens, and accurately view the military
strength, the wealth, the virtue, and the resources of Rome. Your
decisions shall obtain the force of laws. The army, the palace,
the ministers of justice, and the great officers of the empire,
are all subject to your tribunal. None are exempted, excepting
only the ordinary consuls,\textsuperscript{39} the præfect of the city, the king
of the sacrifices, and (as long as she preserves her chastity
inviolate) the eldest of the vestal virgins. Even these few, who
may not dread the severity, will anxiously solicit the esteem, of
the Roman censor.”\textsuperscript{40}

\pagenote[37]{Montesquieu, Grandeur et Decadence des Romains, c.
viii. He illustrates the nature and use of the censorship with
his usual ingenuity, and with uncommon precision.}

\pagenote[38]{Vespasian and Titus were the last censors, (Pliny,
Hist. Natur vii. 49. Censorinus de Die Natali.) The modesty of
Trajan refused an honor which he deserved, and his example became
a law to the Antonines. See Pliny’s Panegyric, c. 45 and 60.}

\pagenote[39]{Yet in spite of his exemption, Pompey appeared
before that tribunal during his consulship. The occasion, indeed,
was equally singular and honorable. Plutarch in Pomp. p. 630.}

\pagenote[40]{See the original speech in the Augustan Hist. p.
173-174.}

A magistrate, invested with such extensive powers, would have
appeared not so much the minister, as the colleague of his
sovereign.\textsuperscript{41} Valerian justly dreaded an elevation so full of
envy and of suspicion. He modestly argued the alarming greatness
of the trust, his own insufficiency, and the incurable corruption
of the times. He artfully insinuated, that the office of censor
was inseparable from the Imperial dignity, and that the feeble
hands of a subject were unequal to the support of such an immense
weight of cares and of power.\textsuperscript{42} The approaching event of war
soon put an end to the prosecution of a project so specious, but
so impracticable; and whilst it preserved Valerian from the
danger, saved the emperor Decius from the disappointment, which
would most probably have attended it. A censor may maintain, he
can never restore, the morals of a state. It is impossible for
such a magistrate to exert his authority with benefit, or even
with effect, unless he is supported by a quick sense of honor and
virtue in the minds of the people, by a decent reverence for the
public opinion, and by a train of useful prejudices combating on
the side of national manners. In a period when these principles
are annihilated, the censorial jurisdiction must either sink into
empty pageantry, or be converted into a partial instrument of
vexatious oppression.\textsuperscript{43} It was easier to vanquish the Goths than
to eradicate the public vices; yet even in the first of these
enterprises, Decius lost his army and his life.

\pagenote[41]{This transaction might deceive Zonaras, who
supposes that Valerian was actually declared the colleague of
Decius, l. xii. p. 625.}

\pagenote[42]{Hist. August. p. 174. The emperor’s reply is
omitted.}

\pagenote[43]{Such as the attempts of Augustus towards a
reformation of manness. Tacit. Annal. iii. 24.}

The Goths were now, on every side, surrounded and pursued by the
Roman arms. The flower of their troops had perished in the long
siege of Philippopolis, and the exhausted country could no longer
afford subsistence for the remaining multitude of licentious
barbarians. Reduced to this extremity, the Goths would gladly
have purchased, by the surrender of all their booty and
prisoners, the permission of an undisturbed retreat. But the
emperor, confident of victory, and resolving, by the chastisement
of these invaders, to strike a salutary terror into the nations
of the North, refused to listen to any terms of accommodation.
The high-spirited barbarians preferred death to slavery. An
obscure town of Mæsia, called Forum Terebronii,\textsuperscript{44} was the scene
of the battle. The Gothic army was drawn up in three lines, and
either from choice or accident, the front of the third line was
covered by a morass. In the beginning of the action, the son of
Decius, a youth of the fairest hopes, and already associated to
the honors of the purple, was slain by an arrow, in the sight of
his afflicted father; who, summoning all his fortitude,
admonished the dismayed troops, that the loss of a single soldier
was of little importance to the republic.\textsuperscript{45} The conflict was
terrible; it was the combat of despair against grief and rage.
The first line of the Goths at length gave way in disorder; the
second, advancing to sustain it, shared its fate; and the third
only remained entire, prepared to dispute the passage of the
morass, which was imprudently attempted by the presumption of the
enemy. “Here the fortune of the day turned, and all things became
adverse to the Romans; the place deep with ooze, sinking under
those who stood, slippery to such as advanced; their armor heavy,
the waters deep; nor could they wield, in that uneasy situation,
their weighty javelins. The barbarians, on the contrary, were
inured to encounter in the bogs, their persons tall, their spears
long, such as could wound at a distance.”\textsuperscript{46} In this morass the
Roman army, after an ineffectual struggle, was irrecoverably
lost; nor could the body of the emperor ever be found.\textsuperscript{47} Such
was the fate of Decius, in the fiftieth year of his age; an
accomplished prince, active in war and affable in peace;\textsuperscript{48} who,
together with his son, has deserved to be compared, both in life
and death, with the brightest examples of ancient virtue.\textsuperscript{49}

\pagenote[44]{Tillemont, Histoire des Empereurs, tom. iii. p.
598. As Zosimus and some of his followers mistake the Danube for
the Tanais, they place the field of battle in the plains of
Scythia.}

\pagenote[45]{Aurelius Victor allows two distinct actions for the
deaths of the two Decii; but I have preferred the account of
Jornandes.}

\pagenote[46]{I have ventured to copy from Tacitus (Annal. i. 64)
the picture of a similar engagement between a Roman army and a
German tribe.}

\pagenote[47]{Jornandes, c. 18. Zosimus, l. i. p. 22, [c. 23.]
Zonaras, l. xii. p. 627. Aurelius Victor.}

\pagenote[48]{The Decii were killed before the end of the year
two hundred and fifty-one, since the new princes took possession
of the consulship on the ensuing calends of January.}

\pagenote[49]{Hist. August. p. 223, gives them a very honorable
place among the small number of good emperors who reigned between
Augustus and Diocletian.}

This fatal blow humbled, for a very little time, the insolence of
the legions. They appeared to have patiently expected, and
submissively obeyed, the decree of the senate which regulated the
succession to the throne. From a just regard for the memory of
Decius, the Imperial title was conferred on Hostilianus, his only
surviving son; but an equal rank, with more effectual power, was
granted to Gallus, whose experience and ability seemed equal to
the great trust of guardian to the young prince and the
distressed empire.\textsuperscript{50} The first care of the new emperor was to
deliver the Illyrian provinces from the intolerable weight of the
victorious Goths. He consented to leave in their hands the rich
fruits of their invasion, an immense booty, and what was still
more disgraceful, a great number of prisoners of the highest
merit and quality. He plentifully supplied their camp with every
conveniency that could assuage their angry spirits or facilitate
their so much wished-for departure; and he even promised to pay
them annually a large sum of gold, on condition they should never
afterwards infest the Roman territories by their incursions.\textsuperscript{51}

\pagenote[50]{Hæc ubi Patres comperere.. .. decernunt. Victor in
Cæsaribus.}

\pagenote[51]{Zonaras, l. xii. p. 628.}

In the age of the Scipios, the most opulent kings of the earth,
who courted the protection of the victorious commonwealth, were
gratified with such trifling presents as could only derive a
value from the hand that bestowed them; an ivory chair, a coarse
garment of purple, an inconsiderable piece of plate, or a
quantity of copper coin.\textsuperscript{52} After the wealth of nations had
centred in Rome, the emperors displayed their greatness, and even
their policy, by the regular exercise of a steady and moderate
liberality towards the allies of the state. They relieved the
poverty of the barbarians, honored their merit, and recompensed
their fidelity. These voluntary marks of bounty were understood
to flow, not from the fears, but merely from the generosity or
the gratitude of the Romans; and whilst presents and subsidies
were liberally distributed among friends and suppliants, they
were sternly refused to such as claimed them as a debt.\textsuperscript{53} But
this stipulation, of an annual payment to a victorious enemy,
appeared without disguise in the light of an ignominious tribute;
the minds of the Romans were not yet accustomed to accept such
unequal laws from a tribe of barbarians; and the prince, who by a
necessary concession had probably saved his country, became the
object of the general contempt and aversion. The death of
Hostiliamus, though it happened in the midst of a raging
pestilence, was interpreted as the personal crime of Gallus;\textsuperscript{54}
and even the defeat of the later emperor was ascribed by the
voice of suspicion to the perfidious counsels of his hated
successor.\textsuperscript{55} The tranquillity which the empire enjoyed during
the first year of his administration,\textsuperscript{56} served rather to inflame
than to appease the public discontent; and as soon as the
apprehensions of war were removed, the infamy of the peace was
more deeply and more sensibly felt.

\pagenote[52]{A \textit{Sella}, a \textit{Toga}, and a golden \textit{Patera} of five
pounds weight, were accepted with joy and gratitude by the
wealthy king of Egypt. (Livy, xxvii. 4.) \textit{Quina millia Æris}, a
weight of copper, in value about eighteen pounds sterling, was
the usual present made to foreign are ambassadors. (Livy, xxxi.
9.)}

\pagenote[53]{See the firmness of a Roman general so late as the
time of Alexander Severus, in the Excerpta Legationum, p. 25,
edit. Louvre.}

\pagenote[54]{For the plague, see Jornandes, c. 19, and Victor in
Cæsaribus.}

\pagenote[55]{These improbable accusations are alleged by
Zosimus, l. i. p. 28, 24.}

\pagenote[56]{Jornandes, c. 19. The Gothic writer at least
observed the peace which his victorious countrymen had sworn to
Gallus.}

But the Romans were irritated to a still higher degree, when they
discovered that they had not even secured their repose, though at
the expense of their honor. The dangerous secret of the wealth
and weakness of the empire had been revealed to the world. New
swarms of barbarians, encouraged by the success, and not
conceiving themselves bound by the obligation of their brethren,
spread devastation though the Illyrian provinces, and terror as
far as the gates of Rome. The defence of the monarchy, which
seemed abandoned by the pusillanimous emperor, was assumed by
Æmilianus, governor of Pannonia and Mæsia; who rallied the
scattered forces, and revived the fainting spirits of the troops.
The barbarians were unexpectedly attacked, routed, chased, and
pursued beyond the Danube. The victorious leader distributed as a
donative the money collected for the tribute, and the
acclamations of the soldiers proclaimed him emperor on the field
of battle.\textsuperscript{57} Gallus, who, careless of the general welfare,
indulged himself in the pleasures of Italy, was almost in the
same instant informed of the success, of the revolt, and of the
rapid approach of his aspiring lieutenant. He advanced to meet
him as far as the plains of Spoleto. When the armies came in
sight of each other, the soldiers of Gallus compared the
ignominious conduct of their sovereign with the glory of his
rival. They admired the valor of Æmilianus; they were attracted
by his liberality, for he offered a considerable increase of pay
to all deserters.\textsuperscript{58} The murder of Gallus, and of his son
Volusianus, put an end to the civil war; and the senate gave a
legal sanction to the rights of conquest. The letters of
Æmilianus to that assembly displayed a mixture of moderation and
vanity. He assured them, that he should resign to their wisdom
the civil administration; and, contenting himself with the
quality of their general, would in a short time assert the glory
of Rome, and deliver the empire from all the barbarians both of
the North and of the East.\textsuperscript{59} His pride was flattered by the
applause of the senate; and medals are still extant, representing
him with the name and attributes of Hercules the Victor, and Mars
the Avenger.\textsuperscript{60}

\pagenote[57]{Zosimus, l. i. p. 25, 26.}

\pagenote[58]{Victor in Cæsaribus.}

\pagenote[59]{Zonaras, l. xii. p. 628.}

\pagenote[60]{Banduri Numismata, p. 94.}

If the new monarch possessed the abilities, he wanted the time,
necessary to fulfil these splendid promises. Less than four
months intervened between his victory and his fall.\textsuperscript{61} He had
vanquished Gallus: he sunk under the weight of a competitor more
formidable than Gallus. That unfortunate prince had sent
Valerian, already distinguished by the honorable title of censor,
to bring the legions of Gaul and Germany\textsuperscript{62} to his aid. Valerian
executed that commission with zeal and fidelity; and as he
arrived too late to save his sovereign, he resolved to revenge
him. The troops of Æmilianus, who still lay encamped in the
plains of Spoleto, were awed by the sanctity of his character,
but much more by the superior strength of his army; and as they
were now become as incapable of personal attachment as they had
always been of constitutional principle, they readily imbrued
their hands in the blood of a prince who so lately had been the
object of their partial choice. The guilt was theirs,\textsuperscript{621} but the
advantage of it was Valerian’s; who obtained the possession of
the throne by the means indeed of a civil war, but with a degree
of innocence singular in that age of revolutions; since he owed
neither gratitude nor allegiance to his predecessor, whom he
dethroned.

\pagenote[61]{Eutropius, l. ix. c. 6, says tertio mense. Eusebio
this emperor.}

\pagenote[62]{Zosimus, l. i. p. 28. Eutropius and Victor station
Valerian’s army in Rhætia.}

\pagenote[621]{Aurelius Victor says that Æmilianus died of a
natural disorder. Tropius, in speaking of his death, does not say
that he was assassinated—G.}

Valerian was about sixty years of age\textsuperscript{63} when he was invested
with the purple, not by the caprice of the populace, or the
clamors of the army, but by the unanimous voice of the Roman
world. In his gradual ascent through the honors of the state, he
had deserved the favor of virtuous princes, and had declared
himself the enemy of tyrants.\textsuperscript{64} His noble birth, his mild but
unblemished manners, his learning, prudence, and experience, were
revered by the senate and people; and if mankind (according to
the observation of an ancient writer) had been left at liberty to
choose a master, their choice would most assuredly have fallen on
Valerian.\textsuperscript{65} Perhaps the merit of this emperor was inadequate to
his reputation; perhaps his abilities, or at least his spirit,
were affected by the languor and coldness of old age. The
consciousness of his decline engaged him to share the throne with
a younger and more active associate;\textsuperscript{66} the emergency of the
times demanded a general no less than a prince; and the
experience of the Roman censor might have directed him where to
bestow the Imperial purple, as the reward of military merit. But
instead of making a judicious choice, which would have confirmed
his reign and endeared his memory, Valerian, consulting only the
dictates of affection or vanity, immediately invested with the
supreme honors his son Gallienus, a youth whose effeminate vices
had been hitherto concealed by the obscurity of a private
station. The joint government of the father and the son subsisted
about seven, and the sole administration of Gallienus continued
about eight, years. But the whole period was one uninterrupted
series of confusion and calamity. As the Roman empire was at the
same time, and on every side, attacked by the blind fury of
foreign invaders, and the wild ambition of domestic usurpers, we
shall consult order and perspicuity, by pursuing, not so much the
doubtful arrangement of dates, as the more natural distribution
of subjects. The most dangerous enemies of Rome, during the
reigns of Valerian and Gallienus, were, 1. The Franks; 2. The
Alemanni; 3. The Goths; and, 4. The Persians. Under these general
appellations, we may comprehend the adventures of less
considerable tribes, whose obscure and uncouth names would only
serve to oppress the memory and perplex the attention of the
reader.

\pagenote[63]{He was about seventy at the time of his accession,
or, as it is more probable, of his death. Hist. August. p. 173.
Tillemont, Hist. des Empereurs, tom. iii. p. 893, note 1.}

\pagenote[64]{Inimicus tyrannorum. Hist. August. p. 173. In the
glorious struggle of the senate against Maximin, Valerian acted a
very spirited part. Hist. August. p. 156.}

\pagenote[65]{According to the distinction of Victor, he seems to
have received the title of Imperator from the army, and that of
Augustus from the senate.}

\pagenote[66]{From Victor and from the medals, Tillemont (tom.
iii. p. 710) very justly infers, that Gallienus was associated to
the empire about the month of August of the year 253.}

I. As the posterity of the Franks compose one of the greatest and
most enlightened nations of Europe, the powers of learning and
ingenuity have been exhausted in the discovery of their
unlettered ancestors. To the tales of credulity have succeeded
the systems of fancy. Every passage has been sifted, every spot
has been surveyed, that might possibly reveal some faint traces
of their origin. It has been supposed that Pannonia,\textsuperscript{67} that
Gaul, that the northern parts of Germany,\textsuperscript{68} gave birth to that
celebrated colony of warriors. At length the most rational
critics, rejecting the fictitious emigrations of ideal
conquerors, have acquiesced in a sentiment whose simplicity
persuades us of its truth.\textsuperscript{69} They suppose, that about the year
two hundred and forty,\textsuperscript{70} a new confederacy was formed under the
name of Franks, by the old inhabitants of the Lower Rhine and the
Weser.\textsuperscript{701} The present circle of Westphalia, the Landgraviate of
Hesse, and the duchies of Brunswick and Luneburg, were the
ancient seat of the Chauci who, in their inaccessible morasses,
defied the Roman arms;\textsuperscript{71} of the Cherusci, proud of the fame of
Arminius; of the Catti, formidable by their firm and intrepid
infantry; and of several other tribes of inferior power and
renown.\textsuperscript{72} The love of liberty was the ruling passion of these
Germans; the enjoyment of it their best treasure; the word that
expressed that enjoyment the most pleasing to their ear. They
deserved, they assumed, they maintained the honorable epithet of
Franks, or Freemen; which concealed, though it did not
extinguish, the peculiar names of the several states of the
confederacy.\textsuperscript{73} Tacit consent, and mutual advantage, dictated the
first laws of the union; it was gradually cemented by habit and
experience. The league of the Franks may admit of some comparison
with the Helvetic body; in which every canton, retaining its
independent sovereignty, consults with its brethren in the common
cause, without acknowledging the authority of any supreme head or
representative assembly.\textsuperscript{74} But the principle of the two
confederacies was extremely different. A peace of two hundred
years has rewarded the wise and honest policy of the Swiss. An
inconstant spirit, the thirst of rapine, and a disregard to the
most solemn treaties, disgraced the character of the Franks.

\pagenote[67]{Various systems have been formed to explain a
difficult passage in Gregory of Tours, l. ii. c. 9.}

\pagenote[68]{The Geographer of Ravenna, i. 11, by mentioning
Mauringania, on the confines of Denmark, as the ancient seat of
the Franks, gave birth to an ingenious system of Leibritz.}

\pagenote[69]{See Cluver. Germania Antiqua, l. iii. c. 20. M.
Freret, in the Memoires de l’Academie des Inscriptions, tom.
xviii.}

\pagenote[70]{Most probably under the reign of Gordian, from an
accidental circumstance fully canvassed by Tillemont, tom. iii.
p. 710, 1181.}

\pagenote[701]{The confederation of the Franks appears to have
been formed, 1. Of the Chauci. 2. Of the Sicambri, the
inhabitants of the duchy of Berg. 3. Of the Attuarii, to the
north of the Sicambri, in the principality of Waldeck, between
the Dimel and the Eder. 4. Of the Bructeri, on the banks of the
Lippe, and in the Hartz. 5. Of the Chamavii, the Gambrivii of
Tacitua, who were established, at the time of the Frankish
confederation, in the country of the Bructeri. 6. Of the Catti,
in Hessia.—G. The Salii and Cherasci are added. Greenwood’s Hist.
of Germans, i 193.—M.}

\pagenote[71]{Plin. Hist. Natur. xvi. l. The Panegyrists
frequently allude to the morasses of the Franks.}

\pagenote[72]{Tacit. Germania, c. 30, 37.}

\pagenote[73]{In a subsequent period, most of those old names are
occasionally mentioned. See some vestiges of them in Cluver.
Germ. Antiq. l. iii.}

\pagenote[74]{Simler de Republica Helvet. cum notis Fuselin.}

\section{Part \thesection.}

The Romans had long experienced the daring valor of the people of
Lower Germany. The union of their strength threatened Gaul with a
more formidable invasion, and required the presence of Gallienus,
the heir and colleague of Imperial power.\textsuperscript{75} Whilst that prince,
and his infant son Salonius, displayed, in the court of Treves,
the majesty of the empire, its armies were ably conducted by
their general, Posthumus, who, though he afterwards betrayed the
family of Valerian, was ever faithful to the great interests of
the monarchy. The treacherous language of panegyrics and medals
darkly announces a long series of victories. Trophies and titles
attest (if such evidence can attest) the fame of Posthumus, who
is repeatedly styled the Conqueror of the Germans, and the Savior
of Gaul.\textsuperscript{76}

\pagenote[75]{Zosimus, l. i. p. 27.}

\pagenote[76]{M. de Brequigny (in the Memoires de l’Academie,
tom. xxx.) has given us a very curious life of Posthumus. A
series of the Augustan History from Medals and Inscriptions has
been more than once planned, and is still much wanted. * Note: M.
Eckhel, Keeper of the Cabinet of Medals, and Professor of
Antiquities at Vienna, lately deceased, has supplied this want by
his excellent work, Doctrina veterum Nummorum, conscripta a Jos.
Eckhel, 8 vol. in 4to Vindobona, 1797.—G. Captain Smyth has
likewise printed (privately) a valuable Descriptive Catologue of
a series of Large Brass Medals of this period Bedford, 1834.—M.
1845.}

But a single fact, the only one indeed of which we have any
distinct knowledge, erases, in a great measure, these monuments
of vanity and adulation. The Rhine, though dignified with the
title of Safeguard of the provinces, was an imperfect barrier
against the daring spirit of enterprise with which the Franks
were actuated. Their rapid devastations stretched from the river
to the foot of the Pyrenees; nor were they stopped by those
mountains. Spain, which had never dreaded, was unable to resist,
the inroads of the Germans. During twelve years, the greatest
part of the reign of Gallienus, that opulent country was the
theatre of unequal and destructive hostilities. Tarragona, the
flourishing capital of a peaceful province, was sacked and almost
destroyed;\textsuperscript{77} and so late as the days of Orosius, who wrote in
the fifth century, wretched cottages, scattered amidst the ruins
of magnificent cities, still recorded the rage of the barbarians.\textsuperscript{78}
When the exhausted country no longer supplied a variety of
plunder, the Franks seized on some vessels in the ports of Spain,\textsuperscript{79}
and transported themselves into Mauritania. The distant
province was astonished with the fury of these barbarians, who
seemed to fall from a new world, as their name, manners, and
complexion, were equally unknown on the coast of Africa.\textsuperscript{80}

\pagenote[77]{Aurel. Victor, c. 33. Instead of Pœne direpto,
both the sense and the expression require deleto; though indeed,
for different reasons, it is alike difficult to correct the text
of the best, and of the worst, writers.}

\pagenote[78]{In the time of Ausonius (the end of the fourth
century) Ilerda or Lerida was in a very ruinous state, (Auson.
Epist. xxv. 58,) which probably was the consequence of this
invasion.}

\pagenote[79]{Valesius is therefore mistaken in supposing that
the Franks had invaded Spain by sea.}

\pagenote[80]{Aurel. Victor. Eutrop. ix. 6.}

II. In that part of Upper Saxony, beyond the Elbe, which is at
present called the Marquisate of Lusace, there existed, in
ancient times, a sacred wood, the awful seat of the superstition
of the Suevi. None were permitted to enter the holy precincts,
without confessing, by their servile bonds and suppliant posture,
the immediate presence of the sovereign Deity.\textsuperscript{81} Patriotism
contributed, as well as devotion, to consecrate the Sonnenwald,
or wood of the Semnones.\textsuperscript{82} It was universally believed, that the
nation had received its first existence on that sacred spot. At
stated periods, the numerous tribes who gloried in the Suevic
blood, resorted thither by their ambassadors; and the memory of
their common extraction was perpetrated by barbaric rites and
human sacrifices. The wide-extended name of Suevi filled the
interior countries of Germany, from the banks of the Oder to
those of the Danube. They were distinguished from the other
Germans by their peculiar mode of dressing their long hair, which
they gathered into a rude knot on the crown of the head; and they
delighted in an ornament that showed their ranks more lofty and
terrible in the eyes of the enemy.\textsuperscript{83} Jealous as the Germans were
of military renown, they all confessed the superior valor of the
Suevi; and the tribes of the Usipetes and Tencteri, who, with a
vast army, encountered the dictator Cæsar, declared that they
esteemed it not a disgrace to have fled before a people to whose
arms the immortal gods themselves were unequal.\textsuperscript{84}

\pagenote[81]{Tacit.Germania, 38.}

\pagenote[82]{Cluver. Germ. Antiq. iii. 25.}

\pagenote[83]{Sic Suevi a ceteris Germanis, sic Suerorum ingenui
a servis separantur. A proud separation!}

\pagenote[84]{Cæsar in Bello Gallico, iv. 7.}

In the reign of the emperor Caracalla, an innumerable swarm of
Suevi appeared on the banks of the Main, and in the neighborhood
of the Roman provinces, in quest either of food, of plunder, or
of glory.\textsuperscript{85} The hasty army of volunteers gradually coalesced
into a great and permanent nation, and, as it was composed from
so many different tribes, assumed the name of Alemanni, \textsuperscript{851} or
\textit{Allmen}, to denote at once their various lineage and their
common bravery.\textsuperscript{86} The latter was soon felt by the Romans in many
a hostile inroad. The Alemanni fought chiefly on horseback; but
their cavalry was rendered still more formidable by a mixture of
light infantry, selected from the bravest and most active of the
youth, whom frequent exercise had inured to accompany the
horsemen in the longest march, the most rapid charge, or the most
precipitate retreat.\textsuperscript{87}

\pagenote[85]{Victor in Caracal. Dion Cassius, lxvii. p. 1350.}

\pagenote[851]{The nation of the Alemanni was not originally
formed by the Suavi properly so called; these have always
preserved their own name. Shortly afterwards they made (A. D.
357) an irruption into Rhætia, and it was not long after that
they were reunited with the Alemanni. Still they have always been
a distinct people; at the present day, the people who inhabit the
north-west of the Black Forest call themselves Schwaben,
Suabians, Sueves, while those who inhabit near the Rhine, in
Ortenau, the Brisgaw, the Margraviate of Baden, do not consider
themselves Suabians, and are by origin Alemanni. The Teucteri and
the Usipetæ, inhabitants of the interior and of the north of
Westphalia, formed, says Gatterer, the nucleus of the Alemannic
nation; they occupied the country where the name of the Alemanni
first appears, as conquered in 213, by Caracalla. They were well
trained to fight on horseback, (according to Tacitus, Germ. c.
32;) and Aurelius Victor gives the same praise to the Alemanni:
finally, they never made part of the Frankish league. The
Alemanni became subsequently a centre round which gathered a
multitude of German tribes, See Eumen. Panegyr. c. 2. Amm. Marc.
xviii. 2, xxix. 4.—G. ——The question whether the Suevi was a
generic name comprehending the clans which peopled central
Germany, is rather hastily decided by M. Guizot Mr. Greenwood,
who has studied the modern German writers on their own origin,
supposes the Suevi, Alemanni, and Marcomanni, one people, under
different appellations. History of Germany, vol i.—M.}

\pagenote[86]{This etymology (far different from those which
amuse the fancy of the learned) is preserved by Asinius
Quadratus, an original historian, quoted by Agathias, i. c. 5.}

\pagenote[87]{The Suevi engaged Cæsar in this manner, and the
manœuvre deserved the approbation of the conqueror, (in Bello
Gallico, i. 48.)}

This warlike people of Germans had been astonished by the immense
preparations of Alexander Severus; they were dismayed by the arms
of his successor, a barbarian equal in valor and fierceness to
themselves. But still hovering on the frontiers of the empire,
they increased the general disorder that ensued after the death
of Decius. They inflicted severe wounds on the rich provinces of
Gaul; they were the first who removed the veil that covered the
feeble majesty of Italy. A numerous body of the Alemanni
penetrated across the Danube and through the Rhætian Alps into
the plains of Lombardy, advanced as far as Ravenna, and displayed
the victorious banners of barbarians almost in sight of Rome.\textsuperscript{88}

\pagenote[88]{Hist. August. p. 215, 216. Dexippus in the
Excerpts. Legationam, p. 8. Hieronym. Chron. Orosius, vii. 22.}

The insult and the danger rekindled in the senate some sparks of
their ancient virtue. Both the emperors were engaged in far
distant wars, Valerian in the East, and Gallienus on the Rhine.
All the hopes and resources of the Romans were in themselves. In
this emergency, the senators resumed the defence of the republic,
drew out the Prætorian guards, who had been left to garrison the
capital, and filled up their numbers, by enlisting into the
public service the stoutest and most willing of the Plebeians.
The Alemanni, astonished with the sudden appearance of an army
more numerous than their own, retired into Germany, laden with
spoil; and their retreat was esteemed as a victory by the
unwarlike Romans.\textsuperscript{89}

\pagenote[89]{Zosimus, l. i. p. 34.}

When Gallienus received the intelligence that his capital was
delivered from the barbarians, he was much less delighted than
alarmed with the courage of the senate, since it might one day
prompt them to rescue the public from domestic tyranny as well as
from foreign invasion. His timid ingratitude was published to his
subjects, in an edict which prohibited the senators from
exercising any military employment, and even from approaching the
camps of the legions. But his fears were groundless. The rich and
luxurious nobles, sinking into their natural character, accepted,
as a favor, this disgraceful exemption from military service; and
as long as they were indulged in the enjoyment of their baths,
their theatres, and their villas, they cheerfully resigned the
more dangerous cares of empire to the rough hands of peasants and
soldiers.\textsuperscript{90}

\pagenote[90]{Aurel. Victor, in Gallieno et Probo. His complaints
breathe as uncommon spirit of freedom.}

Another invasion of the Alemanni, of a more formidable aspect,
but more glorious event, is mentioned by a writer of the lower
empire. Three hundred thousand are said to have been vanquished,
in a battle near Milan, by Gallienus in person, at the head of
only ten thousand Romans.\textsuperscript{91} We may, however, with great
probability, ascribe this incredible victory either to the
credulity of the historian, or to some exaggerated exploits of
one of the emperor’s lieutenants. It was by arms of a very
different nature, that Gallienus endeavored to protect Italy from
the fury of the Germans. He espoused Pipa, the daughter of a king
of the Marcomanni, a Suevic tribe, which was often confounded
with the Alemanni in their wars and conquests.\textsuperscript{92} To the father,
as the price of his alliance, he granted an ample settlement in
Pannonia. The native charms of unpolished beauty seem to have
fixed the daughter in the affections of the inconstant emperor,
and the bands of policy were more firmly connected by those of
love. But the haughty prejudice of Rome still refused the name of
marriage to the profane mixture of a citizen and a barbarian; and
has stigmatized the German princess with the opprobrious title of
concubine of Gallienus.\textsuperscript{93}

\pagenote[91]{Zonaras, l. xii. p. 631.}

\pagenote[92]{One of the Victors calls him king of the
Marcomanni; the other of the Germans.}

\pagenote[93]{See Tillemont, Hist. des Empereurs, tom. iii. p.
398, \&c.}

III. We have already traced the emigration of the Goths from
Scandinavia, or at least from Prussia, to the mouth of the
Borysthenes, and have followed their victorious arms from the
Borysthenes to the Danube. Under the reigns of Valerian and
Gallienus, the frontier of the last-mentioned river was
perpetually infested by the inroads of Germans and Sarmatians;
but it was defended by the Romans with more than usual firmness
and success. The provinces that were the seat of war, recruited
the armies of Rome with an inexhaustible supply of hardy
soldiers; and more than one of these Illyrian peasants attained
the station, and displayed the abilities, of a general. Though
flying parties of the barbarians, who incessantly hovered on the
banks of the Danube, penetrated sometimes to the confines of
Italy and Macedonia, their progress was commonly checked, or
their return intercepted, by the Imperial lieutenants.\textsuperscript{94} But the
great stream of the Gothic hostilities was diverted into a very
different channel. The Goths, in their new settlement of the
Ukraine, soon became masters of the northern coast of the Euxine:
to the south of that inland sea were situated the soft and
wealthy provinces of Asia Minor, which possessed all that could
attract, and nothing that could resist, a barbarian conqueror.

\pagenote[94]{See the lives of Claudius, Aurelian, and Probus, in
the Augustan History.}

The banks of the Borysthenes are only sixty miles distant from
the narrow entrance\textsuperscript{95} of the peninsula of Crim Tartary, known to
the ancients under the name of Chersonesus Taurica.\textsuperscript{96} On that
inhospitable shore, Euripides, embellishing with exquisite art
the tales of antiquity, has placed the scene of one of his most
affecting tragedies.\textsuperscript{97} The bloody sacrifices of Diana, the
arrival of Orestes and Pylades, and the triumph of virtue and
religion over savage fierceness, serve to represent an historical
truth, that the Tauri, the original inhabitants of the peninsula,
were, in some degree, reclaimed from their brutal manners by a
gradual intercourse with the Grecian colonies, which settled
along the maritime coast. The little kingdom of Bosphorus, whose
capital was situated on the Straits, through which the Mæotis
communicates itself to the Euxine, was composed of degenerate
Greeks and half-civilized barbarians. It subsisted, as an
independent state, from the time of the Peloponnesian war,\textsuperscript{98} was
at last swallowed up by the ambition of Mithridates,\textsuperscript{99} and, with
the rest of his dominions, sunk under the weight of the Roman
arms. From the reign of Augustus,\textsuperscript{100} the kings of Bosphorus were
the humble, but not useless, allies of the empire. By presents,
by arms, and by a slight fortification drawn across the Isthmus,
they effectually guarded, against the roving plunderers of
Sarmatia, the access of a country which, from its peculiar
situation and convenient harbors, commanded the Euxine Sea and
Asia Minor.\textsuperscript{101} As long as the sceptre was possessed by a lineal
succession of kings, they acquitted themselves of their important
charge with vigilance and success. Domestic factions, and the
fears, or private interest, of obscure usurpers, who seized on
the vacant throne, admitted the Goths into the heart of
Bosphorus. With the acquisition of a superfluous waste of fertile
soil, the conquerors obtained the command of a naval force,
sufficient to transport their armies to the coast of Asia.\textsuperscript{102}
These ships used in the navigation of the Euxine were of a very
singular construction. They were slight flat-bottomed barks
framed of timber only, without the least mixture of iron, and
occasionally covered with a shelving roof, on the appearance of a
tempest.\textsuperscript{103} In these floating houses, the Goths carelessly
trusted themselves to the mercy of an unknown sea, under the
conduct of sailors pressed into the service, and whose skill and
fidelity were equally suspicious. But the hopes of plunder had
banished every idea of danger, and a natural fearlessness of
temper supplied in their minds the more rational confidence,
which is the just result of knowledge and experience. Warriors of
such a daring spirit must have often murmured against the
cowardice of their guides, who required the strongest assurances
of a settled calm before they would venture to embark; and would
scarcely ever be tempted to lose sight of the land. Such, at
least, is the practice of the modern Turks;\textsuperscript{104} and they are
probably not inferior, in the art of navigation, to the ancient
inhabitants of Bosphorus.

\pagenote[95]{It is about half a league in breadth. Genealogical
History of the Tartars, p 598.}

\pagenote[96]{M. de Peyssonel, who had been French Consul at
Caffa, in his Observations sur les Peuples Barbares, que ont
habite les bords du Danube}

\pagenote[97]{Eeripides in Iphigenia in Taurid.}

\pagenote[98]{Strabo, l. vii. p. 309. The first kings of
Bosphorus were the allies of Athens.}

\pagenote[99]{Appian in Mithridat.}

\pagenote[100]{It was reduced by the arms of Agrippa. Orosius,
vi. 21. Eu tropius, vii. 9. The Romans once advanced within three
days’ march of the Tanais. Tacit. Annal. xii. 17.}

\pagenote[101]{See the Toxaris of Lucian, if we credit the
sincerity and the virtues of the Scythian, who relates a great
war of his nation against the kings of Bosphorus.}

\pagenote[102]{Zosimus, l. i. p. 28.}

\pagenote[103]{Strabo, l. xi. Tacit. Hist. iii. 47. They were
called Camarœ.}

\pagenote[104]{See a very natural picture of the Euxine
navigation, in the xvith letter of Tournefort.}

The fleet of the Goths, leaving the coast of Circassia on the
left hand, first appeared before Pityus,\textsuperscript{105} the utmost limits of
the Roman provinces; a city provided with a convenient port, and
fortified with a strong wall. Here they met with a resistance
more obstinate than they had reason to expect from the feeble
garrison of a distant fortress. They were repulsed; and their
disappointment seemed to diminish the terror of the Gothic name.
As long as Successianus, an officer of superior rank and merit,
defended that frontier, all their efforts were ineffectual; but
as soon as he was removed by Valerian to a more honorable but
less important station, they resumed the attack of Pityus; and by
the destruction of that city, obliterated the memory of their
former disgrace.\textsuperscript{106}

\pagenote[105]{Arrian places the frontier garrison at Dioscurias,
or Sebastopolis, forty-four miles to the east of Pityus. The
garrison of Phasis consisted in his time of only four hundred
foot. See the Periplus of the Euxine. * Note: Pityus is
Pitchinda, according to D’Anville, ii. 115.—G. Rather Boukoun.—M.
Dioscurias is Iskuriah.—G.}

\pagenote[106]{Zosimus, l. i. p. 30.}

Circling round the eastern extremity of the Euxine Sea, the
navigation from Pityus to Trebizond is about three hundred miles.\textsuperscript{107}
The course of the Goths carried them in sight of the country
of Colchis, so famous by the expedition of the Argonauts; and
they even attempted, though without success, to pillage a rich
temple at the mouth of the River Phasis. Trebizond, celebrated in
the retreat of the ten thousand as an ancient colony of Greeks,\textsuperscript{108}
derived its wealth and splendor from the magnificence of the
emperor Hadrian, who had constructed an artificial port on a
coast left destitute by nature of secure harbors.\textsuperscript{109} The city
was large and populous; a double enclosure of walls seemed to
defy the fury of the Goths, and the usual garrison had been
strengthened by a reënforcement of ten thousand men. But there
are not any advantages capable of supplying the absence of
discipline and vigilance. The numerous garrison of Trebizond,
dissolved in riot and luxury, disdained to guard their
impregnable fortifications. The Goths soon discovered the supine
negligence of the besieged, erected a lofty pile of fascines,
ascended the walls in the silence of the night, and entered the
defenceless city sword in hand. A general massacre of the people
ensued, whilst the affrighted soldiers escaped through the
opposite gates of the town. The most holy temples, and the most
splendid edifices, were involved in a common destruction. The
booty that fell into the hands of the Goths was immense: the
wealth of the adjacent countries had been deposited in Trebizond,
as in a secure place of refuge. The number of captives was
incredible, as the victorious barbarians ranged without
opposition through the extensive province of Pontus.\textsuperscript{110} The rich
spoils of Trebizond filled a great fleet of ships that had been
found in the port. The robust youth of the sea-coast were chained
to the oar; and the Goths, satisfied with the success of their
first naval expedition, returned in triumph to their new
establishment in the kingdom of Bosphorus.\textsuperscript{111}

\pagenote[107]{Arrian (in Periplo Maris Euxine, p. 130) calls the
distance 2610 stadia.}

\pagenote[108]{Xenophon, Anabasis, l. iv. p. 348, edit.
Hutchinson. Note: Fallmerayer (Geschichte des Kaiserthums von
Trapezunt, p. 6, \&c) assigns a very ancient date to the first
(Pelasgic) foundation of Trapezun (Trebizond)—M.}

\pagenote[109]{Arrian, p. 129. The general observation is
Tournefort’s.}

\pagenote[110]{See an epistle of Gregory Thaumaturgus, bishop of
Neo-Cæoarea, quoted by Mascou, v. 37.}

\pagenote[111]{Zosimus, l. i. p. 32, 33.}

The second expedition of the Goths was undertaken with greater
powers of men and ships; but they steered a different course,
and, disdaining the exhausted provinces of Pontus, followed the
western coast of the Euxine, passed before the wide mouths of the
Borysthenes, the Niester, and the Danube, and increasing their
fleet by the capture of a great number of fishing barks, they
approached the narrow outlet through which the Euxine Sea pours
its waters into the Mediterranean, and divides the continents of
Europe and Asia. The garrison of Chalcedon was encamped near the
temple of Jupiter Urius, on a promontory that commanded the
entrance of the Strait; and so inconsiderable were the dreaded
invasions of the barbarians that this body of troops surpassed in
number the Gothic army. But it was in numbers alone that they
surpassed it. They deserted with precipitation their advantageous
post, and abandoned the town of Chalcedon, most plentifully
stored with arms and money, to the discretion of the conquerors.
Whilst they hesitated whether they should prefer the sea or land,
Europe or Asia, for the scene of their hostilities, a perfidious
fugitive pointed out Nicomedia,\textsuperscript{1111} once the capital of the
kings of Bithynia, as a rich and easy conquest. He guided the
march, which was only sixty miles from the camp of Chalcedon,\textsuperscript{112}
directed the resistless attack, and partook of the booty; for the
Goths had learned sufficient policy to reward the traitor whom
they detested. Nice, Prusa, Apamæa, Cius,\textsuperscript{1121} cities that had
sometimes rivalled, or imitated, the splendor of Nicomedia, were
involved in the same calamity, which, in a few weeks, raged
without control through the whole province of Bithynia. Three
hundred years of peace, enjoyed by the soft inhabitants of Asia,
had abolished the exercise of arms, and removed the apprehension
of danger. The ancient walls were suffered to moulder away, and
all the revenue of the most opulent cities was reserved for the
construction of baths, temples, and theatres.\textsuperscript{113}

\pagenote[1111]{It has preserved its name, joined to the
preposition of place in that of Nikmid. D’Anv. Geog. Anc. ii.
28.—G.}

\pagenote[112]{Itiner. Hierosolym. p. 572. Wesseling.}

\pagenote[1121]{Now Isnik, Bursa, Mondania Ghio or Kemlik D’Anv.
ii. 23.—G.}

\pagenote[113]{Zosimus, l.. p. 32, 33.}

When the city of Cyzicus withstood the utmost effort of
Mithridates,\textsuperscript{114} it was distinguished by wise laws, a naval power
of two hundred galleys, and three arsenals, of arms, of military
engines, and of corn.\textsuperscript{115} It was still the seat of wealth and
luxury; but of its ancient strength, nothing remained except the
situation, in a little island of the Propontis, connected with
the continent of Asia only by two bridges. From the recent sack
of Prusa, the Goths advanced within eighteen miles\textsuperscript{116} of the
city, which they had devoted to destruction; but the ruin of
Cyzicus was delayed by a fortunate accident. The season was
rainy, and the Lake Apolloniates, the reservoir of all the
springs of Mount Olympus, rose to an uncommon height. The little
river of Rhyndacus, which issues from the lake, swelled into a
broad and rapid stream, and stopped the progress of the Goths.
Their retreat to the maritime city of Heraclea, where the fleet
had probably been stationed, was attended by a long train of
wagons, laden with the spoils of Bithynia, and was marked by the
flames of Nico and Nicomedia, which they wantonly burnt.\textsuperscript{117} Some
obscure hints are mentioned of a doubtful combat that secured
their retreat.\textsuperscript{118} But even a complete victory would have been of
little moment, as the approach of the autumnal equinox summoned
them to hasten their return. To navigate the Euxine before the
month of May, or after that of September, is esteemed by the
modern Turks the most unquestionable instance of rashness and
folly.\textsuperscript{119}

\pagenote[114]{He besieged the place with 400 galleys, 150,000
foot, and a numerous cavalry. See Plutarch in Lucul. Appian in
Mithridat Cicero pro Lege Manilia, c. 8.}

\pagenote[115]{Strabo, l. xii. p. 573.}

\pagenote[116]{Pocock’s Description of the East, l. ii. c. 23,
24.}

\pagenote[117]{Zosimus, l. i. p. 33.}

\pagenote[118]{Syncellus tells an unintelligible story of Prince
Odenathus, who defeated the Goths, and who was killed by Prince
Odenathus.}

\pagenote[119]{Footnote 119: Voyages de Chardin, tom. i. p. 45. He
sailed with the Turks from Constantinople to Caffa.}

When we are informed that the third fleet, equipped by the Goths
in the ports of Bosphorus, consisted of five hundred sails of
ships,\textsuperscript{120} our ready imagination instantly computes and
multiplies the formidable armament; but, as we are assured by the
judicious Strabo,\textsuperscript{121} that the piratical vessels used by the
barbarians of Pontus and the Lesser Scythia, were not capable of
containing more than twenty-five or thirty men we may safely
affirm, that fifteen thousand warriors, at the most, embarked in
this great expedition. Impatient of the limits of the Euxine,
they steered their destructive course from the Cimmerian to the
Thracian Bosphorus. When they had almost gained the middle of the
Straits, they were suddenly driven back to the entrance of them;
till a favorable wind, springing up the next day, carried them in
a few hours into the placid sea, or rather lake, of the
Propontis. Their landing on the little island of Cyzicus was
attended with the ruin of that ancient and noble city. From
thence issuing again through the narrow passage of the
Hellespont, they pursued their winding navigation amidst the
numerous islands scattered over the Archipelago, or the Ægean
Sea. The assistance of captives and deserters must have been very
necessary to pilot their vessels, and to direct their various
incursions, as well on the coast of Greece as on that of Asia. At
length the Gothic fleet anchored in the port of Piræus, five
miles distant from Athens,\textsuperscript{122} which had attempted to make some
preparations for a vigorous defence. Cleodamus, one of the
engineers employed by the emperor’s orders to fortify the
maritime cities against the Goths, had already begun to repair
the ancient walls, fallen to decay since the time of Scylla. The
efforts of his skill were ineffectual, and the barbarians became
masters of the native seat of the muses and the arts. But while
the conquerors abandoned themselves to the license of plunder and
intemperance, their fleet, that lay with a slender guard in the
harbor of Piræus, was unexpectedly attacked by the brave
Dexippus, who, flying with the engineer Cleodamus from the sack
of Athens, collected a hasty band of volunteers, peasants as well
as soldiers, and in some measure avenged the calamities of his
country.\textsuperscript{123}

\pagenote[120]{Syncellus (p. 382) speaks of this expedition, as
undertaken by the Heruli.}

\pagenote[121]{Strabo, l. xi. p. 495.}

\pagenote[122]{Plin. Hist. Natur. iii. 7.}

\pagenote[123]{Hist. August. p. 181. Victor, c. 33. Orosius, vii.
42. Zosimus, l. i. p. 35. Zonaras, l. xii. 635. Syncellus, p.
382. It is not without some attention, that we can explain and
conciliate their imperfect hints. We can still discover some
traces of the partiality of Dexippus, in the relation of his own
and his countrymen’s exploits. * Note: According to a new
fragment of Dexippus, published by Mai, the 2000 men took up a
strong position in a mountainous and woods district, and kept up
a harassing warfare. He expresses a hope of being speedily joined
by the Imperial fleet. Dexippus in rov. Byzantinorum Collect a
Niebuhr, p. 26, 8—M.}

But this exploit, whatever lustre it might shed on the declining
age of Athens, served rather to irritate than to subdue the
undaunted spirit of the northern invaders. A general
conflagration blazed out at the same time in every district of
Greece. Thebes and Argos, Corinth and Sparta, which had formerly
waged such memorable wars against each other, were now unable to
bring an army into the field, or even to defend their ruined
fortifications. The rage of war, both by land and by sea, spread
from the eastern point of Sunium to the western coast of Epirus.
The Goths had already advanced within sight of Italy, when the
approach of such imminent danger awakened the indolent Gallienus
from his dream of pleasure. The emperor appeared in arms; and his
presence seems to have checked the ardor, and to have divided the
strength, of the enemy. Naulobatus, a chief of the Heruli,
accepted an honorable capitulation, entered with a large body of
his countrymen into the service of Rome, and was invested with
the ornaments of the consular dignity, which had never before
been profaned by the hands of a barbarian.\textsuperscript{124} Great numbers of
the Goths, disgusted with the perils and hardships of a tedious
voyage, broke into Mæsia, with a design of forcing their way over
the Danube to their settlements in the Ukraine. The wild attempt
would have proved inevitable destruction, if the discord of the
Roman generals had not opened to the barbarians the means of an
escape.\textsuperscript{125} The small remainder of this destroying host returned
on board their vessels; and measuring back their way through the
Hellespont and the Bosphorus, ravaged in their passage the shores
of Troy, whose fame, immortalized by Homer, will probably survive
the memory of the Gothic conquests. As soon as they found
themselves in safety within the basin of the Euxine, they landed
at Anchialus in Thrace, near the foot of Mount Hæmus; and, after
all their toils, indulged themselves in the use of those pleasant
and salutary hot baths. What remained of the voyage was a short
and easy navigation.\textsuperscript{126} Such was the various fate of this third
and greatest of their naval enterprises. It may seem difficult to
conceive how the original body of fifteen thousand warriors could
sustain the losses and divisions of so bold an adventure. But as
their numbers were gradually wasted by the sword, by shipwrecks,
and by the influence of a warm climate, they were perpetually
renewed by troops of banditti and deserters, who flocked to the
standard of plunder, and by a crowd of fugitive slaves, often of
German or Sarmatian extraction, who eagerly seized the glorious
opportunity of freedom and revenge. In these expeditions, the
Gothic nation claimed a superior share of honor and danger; but
the tribes that fought under the Gothic banners are sometimes
distinguished and sometimes confounded in the imperfect histories
of that age; and as the barbarian fleets seemed to issue from the
mouth of the Tanais, the vague but familiar appellation of
Scythians was frequently bestowed on the mixed multitude.\textsuperscript{127}

\pagenote[124]{Syncellus, p. 382. This body of Heruli was for a
long time faithful and famous.}

\pagenote[125]{Claudius, who commanded on the Danube, thought
with propriety and acted with spirit. His colleague was jealous
of his fame Hist. August. p. 181.}

\pagenote[126]{Jornandes, c. 20.}

\pagenote[127]{Zosimus and the Greeks (as the author of the
Philopatris) give the name of Scythians to those whom Jornandes,
and the Latin writers, constantly represent as Goths.}

\section{Part \thesection.}

In the general calamities of mankind, the death of an individual,
however exalted, the ruin of an edifice, however famous, are
passed over with careless inattention. Yet we cannot forget that
the temple of Diana at Ephesus, after having risen with
increasing splendor from seven repeated misfortunes,\textsuperscript{128} was
finally burnt by the Goths in their third naval invasion. The
arts of Greece, and the wealth of Asia, had conspired to erect
that sacred and magnificent structure. It was supported by a
hundred and twenty-seven marble columns of the Ionic order. They
were the gifts of devout monarchs, and each was sixty feet high.
The altar was adorned with the masterly sculptures of Praxiteles,
who had, perhaps, selected from the favorite legends of the place
the birth of the divine children of Latona, the concealment of
Apollo after the slaughter of the Cyclops, and the clemency of
Bacchus to the vanquished Amazons.\textsuperscript{129} Yet the length of the
temple of Ephesus was only four hundred and twenty-five feet,
about two thirds of the measure of the church of St. Peter’s at
Rome.\textsuperscript{130} In the other dimensions, it was still more inferior to
that sublime production of modern architecture. The spreading
arms of a Christian cross require a much greater breadth than the
oblong temples of the Pagans; and the boldest artists of
antiquity would have been startled at the proposal of raising in
the air a dome of the size and proportions of the Pantheon. The
temple of Diana was, however, admired as one of the wonders of
the world. Successive empires, the Persian, the Macedonian, and
the Roman, had revered its sanctity and enriched its splendor.\textsuperscript{131}
But the rude savages of the Baltic were destitute of a taste
for the elegant arts, and they despised the ideal terrors of a
foreign superstition.\textsuperscript{132}

\pagenote[128]{Hist. Aug. p. 178. Jornandes, c. 20.}

\pagenote[129]{Strabo, l. xiv. p. 640. Vitruvius, l. i. c. i.
præfat l vii. Tacit Annal. iii. 61. Plin. Hist. Nat. xxxvi. 14.}

\pagenote[130]{The length of St. Peter’s is 840 Roman palms; each
palm is very little short of nine English inches. See Greaves’s
Miscellanies vol. i. p. 233; on the Roman Foot. * Note: St.
Paul’s Cathedral is 500 feet. Dallaway on Architecture—M.}

\pagenote[131]{The policy, however, of the Romans induced them to
abridge the extent of the sanctuary or asylum, which by
successive privileges had spread itself two stadia round the
temple. Strabo, l. xiv. p. 641. Tacit. Annal. iii. 60, \&c.}

\pagenote[132]{They offered no sacrifices to the Grecian gods.
See Epistol Gregor. Thaumat.}

Another circumstance is related of these invasions, which might
deserve our notice, were it not justly to be suspected as the
fanciful conceit of a recent sophist. We are told that in the
sack of Athens the Goths had collected all the libraries, and
were on the point of setting fire to this funeral pile of Grecian
learning, had not one of their chiefs, of more refined policy
than his brethren, dissuaded them from the design; by the
profound observation, that as long as the Greeks were addicted to
the study of books, they would never apply themselves to the
exercise of arms.\textsuperscript{133} The sagacious counsellor (should the truth
of the fact be admitted) reasoned like an ignorant barbarian. In
the most polite and powerful nations, genius of every kind has
displayed itself about the same period; and the age of science
has generally been the age of military virtue and success.

\pagenote[133]{Zonaras, l. xii. p. 635. Such an anecdote was
perfectly suited to the taste of Montaigne. He makes use of it in
his agreeable Essay on Pedantry, l. i. c. 24.}

IV. The new sovereign of Persia, Artaxerxes and his son Sapor,
had triumphed (as we have already seen) over the house of
Arsaces. Of the many princes of that ancient race. Chosroes, king
of Armenia, had alone preserved both his life and his
independence. He defended himself by the natural strength of his
country; by the perpetual resort of fugitives and malecontents;
by the alliance of the Romans, and above all, by his own courage.

Invincible in arms, during a thirty years’ war, he was at length
assassinated by the emissaries of Sapor, king of Persia. The
patriotic satraps of Armenia, who asserted the freedom and
dignity of the crown, implored the protection of Rome in favor of
Tiridates, the lawful heir. But the son of Chosroes was an
infant, the allies were at a distance, and the Persian monarch
advanced towards the frontier at the head of an irresistible
force. Young Tiridates, the future hope of his country, was saved
by the fidelity of a servant, and Armenia continued above
twenty-seven years a reluctant province of the great monarchy of
Persia.\textsuperscript{134} Elated with this easy conquest, and presuming on the
distresses or the degeneracy of the Romans, Sapor obliged the
strong garrisons of Carrhæ and Nisibis\textsuperscript{1341} to surrender, and
spread devastation and terror on either side of the Euphrates.

\pagenote[134]{Moses Chorenensis, l. ii. c. 71, 73, 74. Zonaras,
l. xii. p. 628. The anthentic relation of the Armenian historian
serves to rectify the confused account of the Greek. The latter
talks of the children of Tiridates, who at that time was himself
an infant. (Compare St Martin Memoires sur l’Armenie, i. p.
301.—M.)}

\pagenote[1341]{Nisibis, according to Persian authors, was taken
by a miracle, the wall fell, in compliance with the prayers of
the army. Malcolm’s Persia, l. 76.—M}

The loss of an important frontier, the ruin of a faithful and
natural ally, and the rapid success of Sapor’s ambition, affected
Rome with a deep sense of the insult as well as of the danger.
Valerian flattered himself, that the vigilance of his lieutenants
would sufficiently provide for the safety of the Rhine and of the
Danube; but he resolved, notwithstanding his advanced age, to
march in person to the defence of the Euphrates.

During his progress through Asia Minor, the naval enterprises of
the Goths were suspended, and the afflicted province enjoyed a
transient and fallacious calm. He passed the Euphrates,
encountered the Persian monarch near the walls of Edessa, was
vanquished, and taken prisoner by Sapor. The particulars of this
great event are darkly and imperfectly represented; yet, by the
glimmering light which is afforded us, we may discover a long
series of imprudence, of error, and of deserved misfortunes on
the side of the Roman emperor. He reposed an implicit confidence
in Macrianus, his Prætorian præfect.\textsuperscript{135} That worthless minister
rendered his master formidable only to the oppressed subjects,
and contemptible to the enemies of Rome.\textsuperscript{136} By his weak or
wicked counsels, the Imperial army was betrayed into a situation
where valor and military skill were equally unavailing.\textsuperscript{137} The
vigorous attempt of the Romans to cut their way through the
Persian host was repulsed with great slaughter;\textsuperscript{138} and Sapor,
who encompassed the camp with superior numbers, patiently waited
till the increasing rage of famine and pestilence had insured his
victory. The licentious murmurs of the legions soon accused
Valerian as the cause of their calamities; their seditious
clamors demanded an instant capitulation. An immense sum of gold
was offered to purchase the permission of a disgraceful retreat.
But the Persian, conscious of his superiority, refused the money
with disdain; and detaining the deputies, advanced in order of
battle to the foot of the Roman rampart, and insisted on a
personal conference with the emperor. Valerian was reduced to the
necessity of intrusting his life and dignity to the faith of an
enemy. The interview ended as it was natural to expect. The
emperor was made a prisoner, and his astonished troops laid down
their arms.\textsuperscript{139} In such a moment of triumph, the pride and policy
of Sapor prompted him to fill the vacant throne with a successor
entirely dependent on his pleasure. Cyriades, an obscure fugitive
of Antioch, stained with every vice, was chosen to dishonor the
Roman purple; and the will of the Persian victor could not fail
of being ratified by the acclamations, however reluctant, of the
captive army.\textsuperscript{140}

\pagenote[135]{Hist. Aug. p. 191. As Macrianus was an enemy to
the Christians, they charged him with being a magician.}

\pagenote[136]{Zosimus, l. i. p. 33.}

\pagenote[137]{Hist. Aug. p. 174.}

\pagenote[138]{Victor in Cæsar. Eutropius, ix. 7.}

\pagenote[139]{Zosimus, l. i. p. 33. Zonaras, l. xii. p. 630.
Peter Patricius, in the Excerpta Legat. p. 29.}

\pagenote[140]{Hist. August. p. 185. The reign of Cyriades
appears in that collection prior to the death of Valerian; but I
have preferred a probable series of events to the doubtful
chronology of a most inaccurate writer}

The Imperial slave was eager to secure the favor of his master by
an act of treason to his native country. He conducted Sapor over
the Euphrates, and, by the way of Chalcis, to the metropolis of
the East. So rapid were the motions of the Persian cavalry, that,
if we may credit a very judicious historian,\textsuperscript{141} the city of
Antioch was surprised when the idle multitude was fondly gazing
on the amusements of the theatre. The splendid buildings of
Antioch, private as well as public, were either pillaged or
destroyed; and the numerous inhabitants were put to the sword, or
led away into captivity.\textsuperscript{142} The tide of devastation was stopped
for a moment by the resolution of the high priest of Emesa.
Arrayed in his sacerdotal robes, he appeared at the head of a
great body of fanatic peasants, armed only with slings, and
defended his god and his property from the sacrilegious hands of
the followers of Zoroaster.\textsuperscript{143} But the ruin of Tarsus, and of
many other cities, furnishes a melancholy proof that, except in
this singular instance, the conquest of Syria and Cilicia
scarcely interrupted the progress of the Persian arms. The
advantages of the narrow passes of Mount Taurus were abandoned,
in which an invader, whose principal force consisted in his
cavalry, would have been engaged in a very unequal combat: and
Sapor was permitted to form the siege of Cæsarea, the capital of
Cappadocia; a city, though of the second rank, which was supposed
to contain four hundred thousand inhabitants. Demosthenes
commanded in the place, not so much by the commission of the
emperor, as in the voluntary defence of his country. For a long
time he deferred its fate; and when at last Cæsarea was betrayed
by the perfidy of a physician, he cut his way through the
Persians, who had been ordered to exert their utmost diligence to
take him alive. This heroic chief escaped the power of a foe who
might either have honored or punished his obstinate valor; but
many thousands of his fellow-citizens were involved in a general
massacre, and Sapor is accused of treating his prisoners with
wanton and unrelenting cruelty.\textsuperscript{144} Much should undoubtedly be
allowed for national animosity, much for humbled pride and
impotent revenge; yet, upon the whole, it is certain, that the
same prince, who, in Armenia, had displayed the mild aspect of a
legislator, showed himself to the Romans under the stern features
of a conqueror. He despaired of making any permanent
establishment in the empire, and sought only to leave behind him
a wasted desert, whilst he transported into Persia the people and
the treasures of the provinces.\textsuperscript{145}

\pagenote[141]{The sack of Antioch, anticipated by some
historians, is assigned, by the decisive testimony of Ammianus
Marcellinus, to the reign of Gallienus, xxiii. 5. * Note: Heyne,
in his note on Zosimus, contests this opinion of Gibbon and
observes, that the testimony of Ammianus is in fact by no means
clear, decisive. Gallienus and Valerian reigned together.
Zosimus, in a passage, l. iiii. 32, 8, distinctly places this
event before the capture of Valerian.—M.}

\pagenote[142]{Zosimus, l. i. p. 35.}

\pagenote[143]{John Malala, tom. i. p. 391. He corrupts this
probable event by some fabulous circumstances.}

\pagenote[144]{Zonaras, l. xii. p. 630. Deep valleys were filled
up with the slain. Crowds of prisoners were driven to water like
beasts, and many perished for want of food.}

\pagenote[145]{Zosimus, l. i. p. 25 asserts, that Sapor, had he
not preferred spoil to conquest, might have remained master of
Asia.}

At the time when the East trembled at the name of Sapor, he
received a present not unworthy of the greatest kings; a long
train of camels, laden with the most rare and valuable
merchandises. The rich offering was accompanied with an epistle,
respectful, but not servile, from Odenathus, one of the noblest
and most opulent senators of Palmyra. “Who is this Odenathus,”
(said the haughty victor, and he commanded that the present
should be cast into the Euphrates,) “that he thus insolently
presumes to write to his lord? If he entertains a hope of
mitigating his punishment, let him fall prostrate before the foot
of our throne, with his hands bound behind his back. Should he
hesitate, swift destruction shall be poured on his head, on his
whole race, and on his country.”\textsuperscript{146} The desperate extremity to
which the Palmyrenian was reduced, called into action all the
latent powers of his soul. He met Sapor; but he met him in arms.

Infusing his own spirit into a little army collected from the
villages of Syria\textsuperscript{147} and the tents of the desert,\textsuperscript{148} he hovered
round the Persian host, harassed their retreat, carried off part
of the treasure, and, what was dearer than any treasure, several
of the women of the great king; who was at last obliged to repass
the Euphrates with some marks of haste and confusion.\textsuperscript{149} By this
exploit, Odenathus laid the foundations of his future fame and
fortunes. The majesty of Rome, oppressed by a Persian, was
protected by a Syrian or Arab of Palmyra.

\pagenote[146]{Peter Patricius in Excerpt. Leg. p. 29.}

\pagenote[147]{Syrorum agrestium manu. Sextus Rufus, c. 23. Rufus
Victor the Augustan History, (p. 192,) and several inscriptions,
agree in making Odenathus a citizen of Palmyra.}

\pagenote[148]{He possessed so powerful an interest among the
wandering tribes, that Procopius (Bell. Persic. l. ii. c. 5) and
John Malala, (tom. i. p. 391) style him Prince of the Saracens.}

\pagenote[149]{Peter Patricius, p. 25.}

The voice of history, which is often little more than the organ
of hatred or flattery, reproaches Sapor with a proud abuse of the
rights of conquest. We are told that Valerian, in chains, but
invested with the Imperial purple, was exposed to the multitude,
a constant spectacle of fallen greatness; and that whenever the
Persian monarch mounted on horseback, he placed his foot on the
neck of a Roman emperor. Notwithstanding all the remonstrances of
his allies, who repeatedly advised him to remember the
vicissitudes of fortune, to dread the returning power of Rome,
and to make his illustrious captive the pledge of peace, not the
object of insult, Sapor still remained inflexible. When Valerian
sunk under the weight of shame and grief, his skin, stuffed with
straw, and formed into the likeness of a human figure, was
preserved for ages in the most celebrated temple of Persia; a
more real monument of triumph, than the fancied trophies of brass
and marble so often erected by Roman vanity.\textsuperscript{150} The tale is
moral and pathetic, but the truth\textsuperscript{1501} of it may very fairly be
called in question. The letters still extant from the princes of
the East to Sapor are manifest forgeries;\textsuperscript{151} nor is it natural
to suppose that a jealous monarch should, even in the person of a
rival, thus publicly degrade the majesty of kings. Whatever
treatment the unfortunate Valerian might experience in Persia, it
is at least certain that the only emperor of Rome who had ever
fallen into the hands of the enemy, languished away his life in
hopeless captivity.

\pagenote[150]{The Pagan writers lament, the Christian insult,
the misfortunes of Valerian. Their various testimonies are
accurately collected by Tillemont, tom. iii. p. 739, \&c. So
little has been preserved of eastern history before Mahomet, that
the modern Persians are totally ignorant of the victory Sapor, an
event so glorious to their nation. See Bibliotheque Orientale. *
Note: Malcolm appears to write from Persian authorities, i.
76.—M.}

\pagenote[1501]{Yet Gibbon himself records a speech of the
emperor Galerius, which alludes to the cruelties exercised
against the living, and the indignities to which they exposed the
dead Valerian, vol. ii. ch. 13. Respect for the kingly character
would by no means prevent an eastern monarch from ratifying his
pride and his vengeance on a fallen foe.—M.}

\pagenote[151]{One of these epistles is from Artavasdes, king of
Armenia; since Armenia was then a province of Persia, the king,
the kingdom, and the epistle must be fictitious.}

The emperor Gallienus, who had long supported with impatience the
censorial severity of his father and colleague, received the
intelligence of his misfortunes with secret pleasure and avowed
indifference. “I knew that my father was a mortal,” said he; “and
since he has acted as it becomes a brave man, I am satisfied.”
Whilst Rome lamented the fate of her sovereign, the savage
coldness of his son was extolled by the servile courtiers as the
perfect firmness of a hero and a stoic.\textsuperscript{152} It is difficult to
paint the light, the various, the inconstant character of
Gallienus, which he displayed without constraint, as soon as he
became sole possessor of the empire. In every art that he
attempted, his lively genius enabled him to succeed; and as his
genius was destitute of judgment, he attempted every art, except
the important ones of war and government. He was a master of
several curious, but useless sciences, a ready orator, an elegant
poet,\textsuperscript{153} a skilful gardener, an excellent cook, and most
contemptible prince. When the great emergencies of the state
required his presence and attention, he was engaged in
conversation with the philosopher Plotinus,\textsuperscript{154} wasting his time
in trifling or licentious pleasures, preparing his initiation to
the Grecian mysteries, or soliciting a place in the Areopagus of
Athens. His profuse magnificence insulted the general poverty;
the solemn ridicule of his triumphs impressed a deeper sense of
the public disgrace.\textsuperscript{155} The repeated intelligence of invasions,
defeats, and rebellions, he received with a careless smile; and
singling out, with affected contempt, some particular production
of the lost province, he carelessly asked, whether Rome must be
ruined, unless it was supplied with linen from Egypt, and arras
cloth from Gaul. There were, however, a few short moments in the
life of Gallienus, when, exasperated by some recent injury, he
suddenly appeared the intrepid soldier and the cruel tyrant;
till, satiated with blood, or fatigued by resistance, he
insensibly sunk into the natural mildness and indolence of his
character.\textsuperscript{156}

\pagenote[152]{See his life in the Augustan History.}

\pagenote[153]{There is still extant a very pretty Epithalamium,
composed by Gallienus for the nuptials of his nephews:—“Ite ait,
O juvenes, pariter sudate medullis Omnibus, inter vos: non
murmura vestra columbæ, Brachia non hederæ, non vincant oscula
conchæ.”}

\pagenote[154]{He was on the point of giving Plotinus a ruined
city of Campania to try the experiment of realizing Plato’s
Republic. See the Life of Plotinus, by Porphyry, in Fabricius’s
Biblioth. Græc. l. iv.}

\pagenote[155]{A medal which bears the head of Gallienus has
perplexed the antiquarians by its legend and reverse; the former
Gallienæ Augustæ, the latter Ubique Pax. M. Spanheim supposes
that the coin was struck by some of the enemies of Gallienus, and
was designed as a severe satire on that effeminate prince. But as
the use of irony may seem unworthy of the gravity of the Roman
mint, M. de Vallemont has deduced from a passage of Trebellius
Pollio (Hist. Aug. p. 198) an ingenious and natural solution.
Galliena was first cousin to the emperor. By delivering Africa
from the usurper Celsus, she deserved the title of Augusta. On a
medal in the French king’s collection, we read a similar
inscription of Faustina Augusta round the head of Marcus
Aurelius. With regard to the Ubique Pax, it is easily explained
by the vanity of Gallienus, who seized, perhaps, the occasion of
some momentary calm. See Nouvelles de la Republique des Lettres,
Janvier, 1700, p. 21—34.}

\pagenote[156]{This singular character has, I believe, been
fairly transmitted to us. The reign of his immediate successor
was short and busy; and the historians who wrote before the
elevation of the family of Constantine could not have the most
remote interest to misrepresent the character of Gallienus.}

At the time when the reins of government were held with so loose
a hand, it is not surprising that a crowd of usurpers should
start up in every province of the empire against the son of
Valerian. It was probably some ingenious fancy, of comparing the
thirty tyrants of Rome with the thirty tyrants of Athens, that
induced the writers of the Augustan History to select that
celebrated number, which has been gradually received into a
popular appellation.\textsuperscript{157} But in every light the parallel is idle
and defective. What resemblance can we discover between a council
of thirty persons, the united oppressors of a single city, and an
uncertain list of independent rivals, who rose and fell in
irregular succession through the extent of a vast empire? Nor can
the number of thirty be completed, unless we include in the
account the women and children who were honored with the Imperial
title. The reign of Gallienus, distracted as it was, produced
only nineteen pretenders to the throne: Cyriades, Macrianus,
Balista, Odenathus, and Zenobia, in the East; in Gaul, and the
western provinces, Posthumus, Lollianus, Victorinus, and his
mother Victoria, Marius, and Tetricus; in Illyricum and the
confines of the Danube, Ingenuus, Regillianus, and Aureolus; in
Pontus,\textsuperscript{158} Saturninus; in Isauria, Trebellianus; Piso in
Thessaly; Valens in Achaia; Æmilianus in Egypt; and Celsus in
Africa.\textsuperscript{1581} To illustrate the obscure monuments of the life and
death of each individual, would prove a laborious task, alike
barren of instruction and of amusement. We may content ourselves
with investigating some general characters, that most strongly
mark the condition of the times, and the manners of the men,
their pretensions, their motives, their fate, and the destructive
consequences of their usurpation.\textsuperscript{159}

\pagenote[157]{Pollio expresses the most minute anxiety to
complete the number. * Note: Compare a dissertation of Manso on
the thirty tyrants at the end of his Leben Constantius des
Grossen. Breslau, 1817.—M.}

\pagenote[158]{The place of his reign is somewhat doubtful; but
there was a tyrant in Pontus, and we are acquainted with the seat
of all the others.}

\pagenote[1581]{Captain Smyth, in his “Catalogue of Medals,” p.
307, substitutes two new names to make up the number of nineteen,
for those of Odenathus and Zenobia. He subjoins this list:—1. 2.
3. Of those whose coins Those whose coins Those of whom no are
undoubtedly true. are suspected. coins are known. Posthumus.
Cyriades. Valens. Lælianus, (Lollianus, G.) Ingenuus. Balista
Victorinus Celsus. Saturninus. Marius. Piso Frugi. Trebellianus.
Tetricus. —M. 1815 Macrianus. Quietus. Regalianus (Regillianus,
G.) Alex. Æmilianus. Aureolus. Sulpicius Antoninus}

\pagenote[159]{Tillemont, tom. iii. p. 1163, reckons them
somewhat differently.}

It is sufficiently known, that the odious appellation of \textit{Tyrant}
was often employed by the ancients to express the illegal seizure
of supreme power, without any reference to the abuse of it.
Several of the pretenders, who raised the standard of rebellion
against the emperor Gallienus, were shining models of virtue, and
almost all possessed a considerable share of vigor and ability.
Their merit had recommended them to the favor of Valerian, and
gradually promoted them to the most important commands of the
empire. The generals, who assumed the title of Augustus, were
either respected by their troops for their able conduct and
severe discipline, or admired for valor and success in war, or
beloved for frankness and generosity. The field of victory was
often the scene of their election; and even the armorer Marius,
the most contemptible of all the candidates for the purple, was
distinguished, however, by intrepid courage, matchless strength,
and blunt honesty.\textsuperscript{160} His mean and recent trade cast, indeed, an
air of ridicule on his elevation;\textsuperscript{1601} but his birth could not be
more obscure than was that of the greater part of his rivals, who
were born of peasants, and enlisted in the army as private
soldiers. In times of confusion every active genius finds the
place assigned him by nature: in a general state of war military
merit is the road to glory and to greatness. Of the nineteen
tyrants Tetricus only was a senator; Piso alone was a noble. The
blood of Numa, through twenty-eight successive generations, ran
in the veins of Calphurnius Piso,\textsuperscript{161} who, by female alliances,
claimed a right of exhibiting, in his house, the images of
Crassus and of the great Pompey.\textsuperscript{162} His ancestors had been
repeatedly dignified with all the honors which the commonwealth
could bestow; and of all the ancient families of Rome, the
Calphurnian alone had survived the tyranny of the Cæsars. The
personal qualities of Piso added new lustre to his race. The
usurper Valens, by whose order he was killed, confessed, with
deep remorse, that even an enemy ought to have respected the
sanctity of Piso; and although he died in arms against Gallienus,
the senate, with the emperor’s generous permission, decreed the
triumphal ornaments to the memory of so virtuous a rebel.\textsuperscript{163}

\pagenote[160]{See the speech of Marius in the Augustan History,
p. 197. The accidental identity of names was the only
circumstance that could tempt Pollio to imitate Sallust.}

\pagenote[1601]{Marius was killed by a soldier, who had formerly
served as a workman in his shop, and who exclaimed, as he struck,
“Behold the sword which thyself hast forged.” Trob vita.—G.}

\pagenote[161]{“Vos, O Pompilius sanguis!” is Horace’s address to
the Pisos See Art. Poet. v. 292, with Dacier’s and Sanadon’s
notes.}

\pagenote[162]{Tacit. Annal. xv. 48. Hist. i. 15. In the former
of these passages we may venture to change paterna into materna.
In every generation from Augustus to Alexander Severus, one or
more Pisos appear as consuls. A Piso was deemed worthy of the
throne by Augustus, (Tacit. Annal. i. 13;) a second headed a
formidable conspiracy against Nero; and a third was adopted, and
declared Cæsar, by Galba.}

\pagenote[163]{Hist. August. p. 195. The senate, in a moment of
enthusiasm, seems to have presumed on the approbation of
Gallienus.}

The lieutenants of Valerian were grateful to the father, whom
they esteemed. They disdained to serve the luxurious indolence of
his unworthy son. The throne of the Roman world was unsupported
by any principle of loyalty; and treason against such a prince
might easily be considered as patriotism to the state. Yet if we
examine with candor the conduct of these usurpers, it will
appear, that they were much oftener driven into rebellion by
their fears, than urged to it by their ambition. They dreaded the
cruel suspicions of Gallienus; they equally dreaded the
capricious violence of their troops. If the dangerous favor of
the army had imprudently declared them deserving of the purple,
they were marked for sure destruction; and even prudence would
counsel them to secure a short enjoyment of empire, and rather to
try the fortune of war than to expect the hand of an executioner.

When the clamor of the soldiers invested the reluctant victims
with the ensigns of sovereign authority, they sometimes mourned
in secret their approaching fate. “You have lost,” said
Saturninus, on the day of his elevation, “you have lost a useful
commander, and you have made a very wretched emperor.”\textsuperscript{164}

\pagenote[164]{Hist. August p. 196.}

The apprehensions of Saturninus were justified by the repeated
experience of revolutions. Of the nineteen tyrants who started up
under the reign of Gallienus, there was not one who enjoyed a
life of peace, or a natural death. As soon as they were invested
with the bloody purple, they inspired their adherents with the
same fears and ambition which had occasioned their own revolt.
Encompassed with domestic conspiracy, military sedition, and
civil war, they trembled on the edge of precipices, in which,
after a longer or shorter term of anxiety, they were inevitably
lost. These precarious monarchs received, however, such honors as
the flattery of their respective armies and provinces could
bestow; but their claim, founded on rebellion, could never obtain
the sanction of law or history. Italy, Rome, and the senate,
constantly adhered to the cause of Gallienus, and he alone was
considered as the sovereign of the empire. That prince
condescended, indeed, to acknowledge the victorious arms of
Odenathus, who deserved the honorable distinction, by the
respectful conduct which he always maintained towards the son of
Valerian. With the general applause of the Romans, and the
consent of Gallienus, the senate conferred the title of Augustus
on the brave Palmyrenian; and seemed to intrust him with the
government of the East, which he already possessed, in so
independent a manner, that, like a private succession, he
bequeathed it to his illustrious widow, Zenobia.\textsuperscript{165}

\pagenote[165]{The association of the brave Palmyrenian was the
most popular act of the whole reign of Gallienus. Hist. August.
p. 180.}

The rapid and perpetual transitions from the cottage to the
throne, and from the throne to the grave, might have amused an
indifferent philosopher; were it possible for a philosopher to
remain indifferent amidst the general calamities of human kind.
The election of these precarious emperors, their power and their
death, were equally destructive to their subjects and adherents.
The price of their fatal elevation was instantly discharged to
the troops by an immense donative, drawn from the bowels of the
exhausted people. However virtuous was their character, however
pure their intentions, they found themselves reduced to the hard
necessity of supporting their usurpation by frequent acts of
rapine and cruelty. When they fell, they involved armies and
provinces in their fall. There is still extant a most savage
mandate from Gallienus to one of his ministers, after the
suppression of Ingenuus, who had assumed the purple in Illyricum.

“It is not enough,” says that soft but inhuman prince, “that you
exterminate such as have appeared in arms; the chance of battle
might have served me as effectually. The male sex of every age
must be extirpated; provided that, in the execution of the
children and old men, you can contrive means to save our
reputation. Let every one die who has dropped an expression, who
has entertained a thought against me, against \textit{me}, the son of
Valerian, the father and brother of so many princes.\textsuperscript{166} Remember
that Ingenuus was made emperor: tear, kill, hew in pieces. I
write to you with my own hand, and would inspire you with my own
feelings.”\textsuperscript{167} Whilst the public forces of the state were
dissipated in private quarrels, the defenceless provinces lay
exposed to every invader. The bravest usurpers were compelled, by
the perplexity of their situation, to conclude ignominious
treaties with the common enemy, to purchase with oppressive
tributes the neutrality or services of the Barbarians, and to
introduce hostile and independent nations into the heart of the
Roman monarchy.\textsuperscript{168}

\pagenote[166]{Gallienus had given the titles of Cæsar and
Augustus to his son Saloninus, slain at Cologne by the usurper
Posthumus. A second son of Gallienus succeeded to the name and
rank of his elder brother Valerian, the brother of Gallienus, was
also associated to the empire: several other brothers, sisters,
nephews, and nieces of the emperor formed a very numerous royal
family. See Tillemont, tom iii, and M. de Brequigny in the
Memoires de l’Academie, tom xxxii p. 262.}

\pagenote[167]{Hist. August. p. 188.}

\pagenote[168]{Regillianus had some bands of Roxolani in his
service; Posthumus a body of Franks. It was, perhaps, in the
character of auxiliaries that the latter introduced themselves
into Spain.}

Such were the barbarians, and such the tyrants, who, under the
reigns of Valerian and Gallienus, dismembered the provinces, and
reduced the empire to the lowest pitch of disgrace and ruin, from
whence it seemed impossible that it should ever emerge. As far as
the barrenness of materials would permit, we have attempted to
trace, with order and perspicuity, the general events of that
calamitous period. There still remain some particular facts; I.
The disorders of Sicily; II. The tumults of Alexandria; and, III.
The rebellion of the Isaurians, which may serve to reflect a
strong light on the horrid picture.

I. Whenever numerous troops of banditti, multiplied by success
and impunity, publicly defy, instead of eluding, the justice of
their country, we may safely infer that the excessive weakness of
the country is felt and abused by the lowest ranks of the
community. The situation of Sicily preserved it from the
Barbarians; nor could the disarmed province have supported a
usurper. The sufferings of that once flourishing and still
fertile island were inflicted by baser hands. A licentious crowd
of slaves and peasants reigned for a while over the plundered
country, and renewed the memory of the servile wars of more
ancient times.\textsuperscript{169} Devastations, of which the husbandman was
either the victim or the accomplice, must have ruined the
agriculture of Sicily; and as the principal estates were the
property of the opulent senators of Rome, who often enclosed
within a farm the territory of an old republic, it is not
improbable, that this private injury might affect the capital
more deeply, than all the conquests of the Goths or the Persians.

\pagenote[169]{The Augustan History, p. 177. See Diodor. Sicul.
l. xxxiv.}

II. The foundation of Alexandria was a noble design, at once
conceived and executed by the son of Philip. The beautiful and
regular form of that great city, second only to Rome itself,
comprehended a circumference of fifteen miles;\textsuperscript{170} it was peopled
by three hundred thousand free inhabitants, besides at least an
equal number of slaves.\textsuperscript{171} The lucrative trade of Arabia and
India flowed through the port of Alexandria, to the capital and
provinces of the empire.\textsuperscript{1711} Idleness was unknown. Some were
employed in blowing of glass, others in weaving of linen, others
again manufacturing the papyrus. Either sex, and every age, was
engaged in the pursuits of industry, nor did even the blind or
the lame want occupations suited to their condition.\textsuperscript{172} But the
people of Alexandria, a various mixture of nations, united the
vanity and inconstancy of the Greeks with the superstition and
obstinacy of the Egyptians. The most trifling occasion, a
transient scarcity of flesh or lentils, the neglect of an
accustomed salutation, a mistake of precedency in the public
baths, or even a religious dispute,\textsuperscript{173} were at any time
sufficient to kindle a sedition among that vast multitude, whose
resentments were furious and implacable.\textsuperscript{174} After the captivity
of Valerian and the insolence of his son had relaxed the
authority of the laws, the Alexandrians abandoned themselves to
the ungoverned rage of their passions, and their unhappy country
was the theatre of a civil war, which continued (with a few short
and suspicious truces) above twelve years.\textsuperscript{175} All intercourse
was cut off between the several quarters of the afflicted city,
every street was polluted with blood, every building of strength
converted into a citadel; nor did the tumults subside till a
considerable part of Alexandria was irretrievably ruined. The
spacious and magnificent district of Bruchion,\textsuperscript{1751} with its
palaces and musæum, the residence of the kings and philosophers
of Egypt, is described above a century afterwards, as already
reduced to its present state of dreary solitude.\textsuperscript{176}

\pagenote[170]{Plin. Hist. Natur. v. 10.}

\pagenote[171]{Diodor. Sicul. l. xvii. p. 590, edit. Wesseling.}

\pagenote[1711]{Berenice, or Myos-Hormos, on the Red Sea,
received the eastern commodities. From thence they were
transported to the Nile, and down the Nile to Alexandria.—M.}

\pagenote[172]{See a very curious letter of Hadrian, in the
Augustan History, p. 245.}

\pagenote[173]{Such as the sacrilegious murder of a divine cat.
See Diodor. Sicul. l. i. * Note: The hostility between the Jewish
and Grecian part of the population afterwards between the two
former and the Christian, were unfailing causes of tumult,
sedition, and massacre. In no place were the religious disputes,
after the establishment of Christianity, more frequent or more
sanguinary. See Philo. de Legat. Hist. of Jews, ii. 171, iii.
111, 198. Gibbon, iii c. xxi. viii. c. xlvii.—M.}

\pagenote[174]{Hist. August. p. 195. This long and terrible
sedition was first occasioned by a dispute between a soldier and
a townsman about a pair of shoes.}

\pagenote[175]{Dionysius apud. Euses. Hist. Eccles. vii. p. 21.
Ammian xxii. 16.}

\pagenote[1751]{The Bruchion was a quarter of Alexandria which
extended along the largest of the two ports, and contained many
palaces, inhabited by the Ptolemies. D’Anv. Geogr. Anc. iii.
10.—G.}

\pagenote[176]{Scaliger. Animadver. ad Euseb. Chron. p. 258.
Three dissertations of M. Bonamy, in the Mem. de l’Academie, tom.
ix.}

III. The obscure rebellion of Trebellianus, who assumed the purple in
Isauria, a petty province of Asia Minor, was attended with strange and
memorable consequences. The pageant of royalty was soon destroyed by an
officer of Gallienus; but his followers, despairing of mercy, resolved
to shake off their allegiance, not only to the emperor, but to the
empire, and suddenly returned to the savage manners from which they had
never perfectly been reclaimed. Their craggy rocks, a branch of the
wide-extended Taurus, protected their inaccessible retreat. The tillage
of some fertile valleys\textsuperscript{177} supplied them with necessaries, and a habit
of rapine with the luxuries of life. In the heart of the Roman
monarchy, the Isaurians long continued a nation of wild barbarians.
Succeeding princes, unable to reduce them to obedience, either by arms
or policy, were compelled to acknowledge their weakness, by surrounding
the hostile and independent spot with a strong chain of fortifications,\textsuperscript{178}
which often proved insufficient to restrain the incursions of these
domestic foes. The Isaurians, gradually extending their territory to
the sea-coast, subdued the western and mountainous part of Cilicia,
formerly the nest of those daring pirates, against whom the republic
had once been obliged to exert its utmost force, under the conduct of
the great Pompey.\textsuperscript{179}

\pagenote[177]{Strabo, l. xiii. p. 569.}

\pagenote[178]{Hist. August. p. 197.}

\pagenote[179]{See Cellarius, Geogr Antiq. tom. ii. p. 137, upon
the limits of Isauria.}

Our habits of thinking so fondly connect the order of the
universe with the fate of man, that this gloomy period of history
has been decorated with inundations, earthquakes, uncommon
meteors, preternatural darkness, and a crowd of prodigies
fictitious or exaggerated.\textsuperscript{180} But a long and general famine was
a calamity of a more serious kind. It was the inevitable
consequence of rapine and oppression, which extirpated the
produce of the present and the hope of future harvests. Famine is
almost always followed by epidemical diseases, the effect of
scanty and unwholesome food. Other causes must, however, have
contributed to the furious plague, which, from the year two
hundred and fifty to the year two hundred and sixty-five, raged
without interruption in every province, every city, and almost
every family, of the Roman empire. During some time five thousand
persons died daily in Rome; and many towns, that had escaped the
hands of the Barbarians, were entirely depopulated.\textsuperscript{181}

\pagenote[180]{Hist August p 177.}

\pagenote[181]{Hist. August. p. 177. Zosimus, l. i. p. 24.
Zonaras, l. xii. p. 623. Euseb. Chronicon. Victor in Epitom.
Victor in Cæsar. Eutropius, ix. 5. Orosius, vii. 21.}

We have the knowledge of a very curious circumstance, of some use
perhaps in the melancholy calculation of human calamities. An
exact register was kept at Alexandria of all the citizens
entitled to receive the distribution of corn. It was found, that
the ancient number of those comprised between the ages of forty
and seventy, had been equal to the whole sum of claimants, from
fourteen to fourscore years of age, who remained alive after the
reign of Gallienus.\textsuperscript{182} Applying this authentic fact to the most
correct tables of mortality, it evidently proves, that above half
the people of Alexandria had perished; and could we venture to
extend the analogy to the other provinces, we might suspect, that
war, pestilence, and famine, had consumed, in a few years, the
moiety of the human species.\textsuperscript{183}

\pagenote[182]{Euseb. Hist. Eccles. vii. 21. The fact is taken
from the Letters of Dionysius, who, in the time of those
troubles, was bishop of Alexandria.}

\pagenote[183]{In a great number of parishes, 11,000 persons were
found between fourteen and eighty; 5365 between forty and
seventy. See Buffon, Histoire Naturelle, tom. ii. p. 590.}

