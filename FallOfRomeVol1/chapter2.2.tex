\section{Part \thesection.}
\thispagestyle{simple}

Till the privileges of Romans had been progressively extended to
all the inhabitants of the empire, an important distinction was
preserved between Italy and the provinces. The former was
esteemed the centre of public unity, and the firm basis of the
constitution. Italy claimed the birth, or at least the residence,
of the emperors and the senate. 26 The estates of the Italians
were exempt from taxes, their persons from the arbitrary
jurisdiction of governors. Their municipal corporations, formed
after the perfect model of the capital, 261 were intrusted, under
the immediate eye of the supreme power, with the execution of the
laws. From the foot of the Alps to the extremity of Calabria, all
the natives of Italy were born citizens of Rome. Their partial
distinctions were obliterated, and they insensibly coalesced into
one great nation, united by language, manners, and civil
institutions, and equal to the weight of a powerful empire. The
republic gloried in her generous policy, and was frequently
rewarded by the merit and services of her adopted sons. Had she
always confined the distinction of Romans to the ancient families
within the walls of the city, that immortal name would have been
deprived of some of its noblest ornaments. Virgil was a native of
Mantua; Horace was inclined to doubt whether he should call
himself an Apulian or a Lucanian; it was in Padua that an
historian was found worthy to record the majestic series of Roman
victories. The patriot family of the Catos emerged from Tusculum;
and the little town of Arpinum claimed the double honor of
producing Marius and Cicero, the former of whom deserved, after
Romulus and Camillus, to be styled the Third Founder of Rome; and
the latter, after saving his country from the designs of
Catiline, enabled her to contend with Athens for the palm of
eloquence. 27

\footnotetext[26]{The senators were obliged to have one third of
their own landed property in Italy. See Plin. l. vi. ep. 19. The
qualification was reduced by Marcus to one fourth. Since the
reign of Trajan, Italy had sunk nearer to the level of the
provinces.}

\footnotetext[261]{It may be doubted whether the municipal government
of the cities was not the old Italian constitution rather than a
transcript from that of Rome. The free government of the cities,
observes Savigny, was the leading characteristic of Italy.
Geschichte des Römischen Rechts, i. p. G.—M.}

\footnotetext[27]{The first part of the Verona Illustrata of the
Marquis Maffei gives the clearest and most comprehensive view of
the state of Italy under the Cæsars. * Note: Compare Denina,
Revol. d’ Italia, l. ii. c. 6, p. 100, 4 to edit.}

The provinces of the empire (as they have been described in the
preceding chapter) were destitute of any public force, or
constitutional freedom. In Etruria, in Greece, 28 and in Gaul, 29
it was the first care of the senate to dissolve those dangerous
confederacies, which taught mankind that, as the Roman arms
prevailed by division, they might be resisted by union. Those
princes, whom the ostentation of gratitude or generosity
permitted for a while to hold a precarious sceptre, were
dismissed from their thrones, as soon as they had performed their
appointed task of fashioning to the yoke the vanquished nations.
The free states and cities which had embraced the cause of Rome
were rewarded with a nominal alliance, and insensibly sunk into
real servitude. The public authority was everywhere exercised by
the ministers of the senate and of the emperors, and that
authority was absolute, and without control. 291 But the same
salutary maxims of government, which had secured the peace and
obedience of Italy were extended to the most distant conquests. A
nation of Romans was gradually formed in the provinces, by the
double expedient of introducing colonies, and of admitting the
most faithful and deserving of the provincials to the freedom of
Rome.

\footnotetext[28]{See Pausanias, l. vii. The Romans condescended to
restore the names of those assemblies, when they could no longer
be dangerous.}

\footnotetext[29]{They are frequently mentioned by Cæsar. The Abbé
Dubos attempts, with very little success, to prove that the
assemblies of Gaul were continued under the emperors. Histoire de
l’Etablissement de la Monarchie Francoise, l. i. c. 4.}

\footnotetext[291]{This is, perhaps, rather overstated. Most cities
retained the choice of their municipal officers: some retained
valuable privileges; Athens, for instance, in form was still a
confederate city. (Tac. Ann. ii. 53.) These privileges, indeed,
depended entirely on the arbitrary will of the emperor, who
revoked or restored them according to his caprice. See Walther
Geschichte des Römischen Rechts, i. 324—an admirable summary of
the Roman constitutional history.—M.}

“Wheresoever the Roman conquers, he inhabits,” is a very just
observation of Seneca, 30 confirmed by history and experience.
The natives of Italy, allured by pleasure or by interest,
hastened to enjoy the advantages of victory; and we may remark,
that, about forty years after the reduction of Asia, eighty
thousand Romans were massacred in one day, by the cruel orders of
Mithridates. 31 These voluntary exiles were engaged, for the most
part, in the occupations of commerce, agriculture, and the farm
of the revenue. But after the legions were rendered permanent by
the emperors, the provinces were peopled by a race of soldiers;
and the veterans, whether they received the reward of their
service in land or in money, usually settled with their families
in the country, where they had honorably spent their youth.
Throughout the empire, but more particularly in the western
parts, the most fertile districts, and the most convenient
situations, were reserved for the establishment of colonies; some
of which were of a civil, and others of a military nature. In
their manners and internal policy, the colonies formed a perfect
representation of their great parent; and they were soon endeared
to the natives by the ties of friendship and alliance, they
effectually diffused a reverence for the Roman name, and a
desire, which was seldom disappointed, of sharing, in due time,
its honors and advantages. 32 The municipal cities insensibly
equalled the rank and splendor of the colonies; and in the reign
of Hadrian, it was disputed which was the preferable condition,
of those societies which had issued from, or those which had been
received into, the bosom of Rome. 33 The right of Latium, as it
was called, 331 conferred on the cities to which it had been
granted, a more partial favor. The magistrates only, at the
expiration of their office, assumed the quality of Roman
citizens; but as those offices were annual, in a few years they
circulated round the principal families. 34 Those of the
provincials who were permitted to bear arms in the legions; 35
those who exercised any civil employment; all, in a word, who
performed any public service, or displayed any personal talents,
were rewarded with a present, whose value was continually
diminished by the increasing liberality of the emperors. Yet
even, in the age of the Antonines, when the freedom of the city
had been bestowed on the greater number of their subjects, it was
still accompanied with very solid advantages. The bulk of the
people acquired, with that title, the benefit of the Roman laws,
particularly in the interesting articles of marriage, testaments,
and inheritances; and the road of fortune was open to those whose
pretensions were seconded by favor or merit. The grandsons of the
Gauls, who had besieged Julius Cæsar in Alesia, commanded
legions, governed provinces, and were admitted into the senate of
Rome. 36 Their ambition, instead of disturbing the tranquillity
of the state, was intimately connected with its safety and
greatness.

\footnotetext[30]{Seneca in Consolat. ad Helviam, c. 6.}

\footnotetext[31]{Memnon apud Photium, (c. 33,) [c. 224, p. 231, ed
Bekker.] Valer. Maxim. ix. 2. Plutarch and Dion Cassius swell the
massacre to 150,000 citizens; but I should esteem the smaller
number to be more than sufficient.}

\footnotetext[32]{Twenty-five colonies were settled in Spain, (see
Plin. Hist. Nat. iii. 3, 4; iv. 35;) and nine in Britain, of
which London, Colchester, Lincoln, Chester, Gloucester, and Bath
still remain considerable cities. (See Richard of Cirencester, p.
36, and Whittaker’s History of Manchester, l. i. c. 3.)}

\footnotetext[33]{Aul. Gel. Noctes Atticæ, xvi 13. The Emperor
Hadrian expressed his surprise, that the cities of Utica, Gades,
and Italica, which already enjoyed the rights of \textit{Municipia},
should solicit the title of \textit{colonies}. Their example, however,
became fashionable, and the empire was filled with honorary
colonies. See Spanheim, de Usu Numismatum Dissertat. xiii.}

\footnotetext[331]{The right of Latium conferred an exemption from
the government of the Roman præfect. Strabo states this
distinctly, l. iv. p. 295, edit. Cæsar’s. See also Walther, p.
233.—M}

\footnotetext[34]{Spanheim, Orbis Roman. c. 8, p. 62.}

\footnotetext[35]{Aristid. in Romæ Encomio. tom. i. p. 218, edit.
Jebb.}

\footnotetext[36]{Tacit. Annal. xi. 23, 24. Hist. iv. 74.}

So sensible were the Romans of the influence of language over
national manners, that it was their most serious care to extend,
with the progress of their arms, the use of the Latin tongue. 37
The ancient dialects of Italy, the Sabine, the Etruscan, and the
Venetian, sunk into oblivion; but in the provinces, the east was
less docile than the west to the voice of its victorious
preceptors. This obvious difference marked the two portions of
the empire with a distinction of colors, which, though it was in
some degree concealed during the meridian splendor of prosperity,
became gradually more visible, as the shades of night descended
upon the Roman world. The western countries were civilized by the
same hands which subdued them. As soon as the barbarians were
reconciled to obedience, their minds were open to any new
impressions of knowledge and politeness. The language of Virgil
and Cicero, though with some inevitable mixture of corruption,
was so universally adopted in Africa, Spain, Gaul, Britain, and
Pannonia, 38 that the faint traces of the Punic or Celtic idioms
were preserved only in the mountains, or among the peasants. 39
Education and study insensibly inspired the natives of those
countries with the sentiments of Romans; and Italy gave fashions,
as well as laws, to her Latin provincials. They solicited with
more ardor, and obtained with more facility, the freedom and
honors of the state; supported the national dignity in letters 40
and in arms; and at length, in the person of Trajan, produced an
emperor whom the Scipios would not have disowned for their
countryman. The situation of the Greeks was very different from
that of the barbarians. The former had been long since civilized
and corrupted. They had too much taste to relinquish their
language, and too much vanity to adopt any foreign institutions.
Still preserving the prejudices, after they had lost the virtues,
of their ancestors, they affected to despise the unpolished
manners of the Roman conquerors, whilst they were compelled to
respect their superior wisdom and power. 41 Nor was the influence
of the Grecian language and sentiments confined to the narrow
limits of that once celebrated country. Their empire, by the
progress of colonies and conquest, had been diffused from the
Adriatic to the Euphrates and the Nile. Asia was covered with
Greek cities, and the long reign of the Macedonian kings had
introduced a silent revolution into Syria and Egypt. In their
pompous courts, those princes united the elegance of Athens with
the luxury of the East, and the example of the court was
imitated, at an humble distance, by the higher ranks of their
subjects. Such was the general division of the Roman empire into
the Latin and Greek languages. To these we may add a third
distinction for the body of the natives in Syria, and especially
in Egypt, the use of their ancient dialects, by secluding them
from the commerce of mankind, checked the improvements of those
barbarians. 42 The slothful effeminacy of the former exposed them
to the contempt, the sullen ferociousness of the latter excited
the aversion, of the conquerors. 43 Those nations had submitted
to the Roman power, but they seldom desired or deserved the
freedom of the city: and it was remarked, that more than two
hundred and thirty years elapsed after the ruin of the Ptolemies,
before an Egyptian was admitted into the senate of Rome. 44

\footnotetext[37]{See Plin. Hist. Natur. iii. 5. Augustin. de
Civitate Dei, xix 7 Lipsius de Pronunciatione Linguæ Latinæ, c.
3.}

\footnotetext[38]{Apuleius and Augustin will answer for Africa;
Strabo for Spain and Gaul; Tacitus, in the life of Agricola, for
Britain; and Velleius Paterculus, for Pannonia. To them we may
add the language of the Inscriptions. * Note: Mr. Hallam contests
this assertion as regards Britain. “Nor did the Romans ever
establish their language—I know not whether they wished to do
so—in this island, as we perceive by that stubborn British tongue
which has survived two conquests.” In his note, Mr. Hallam
examines the passage from Tacitus (Agric. xxi.) to which Gibbon
refers. It merely asserts the progress of Latin studies among the
higher orders. (Midd. Ages, iii. 314.) Probably it was a kind of
court language, and that of public affairs and prevailed in the
Roman colonies.—M.}

\footnotetext[39]{The Celtic was preserved in the mountains of Wales,
Cornwall, and Armorica. We may observe, that Apuleius reproaches
an African youth, who lived among the populace, with the use of
the Punic; whilst he had almost forgot Greek, and neither could
nor would speak Latin, (Apolog. p. 596.) The greater part of St.
Austin’s congregations were strangers to the Punic.}

\footnotetext[40]{Spain alone produced Columella, the Senecas, Lucan,
Martial, and Quintilian.}

\footnotetext[41]{There is not, I believe, from Dionysius to Libanus,
a single Greek critic who mentions Virgil or Horace. They seem
ignorant that the Romans had any good writers.}

\footnotetext[42]{The curious reader may see in Dupin, (Bibliotheque
Ecclesiastique, tom. xix. p. 1, c. 8,) how much the use of the
Syriac and Egyptian languages was still preserved.}

\footnotetext[43]{See Juvenal, Sat. iii. and xv. Ammian. Marcellin.
xxii. 16.}

\footnotetext[44]{Dion Cassius, l. lxxvii. p. 1275. The first
instance happened under the reign of Septimius Severus.}

It is a just though trite observation, that victorious Rome was
herself subdued by the arts of Greece. Those immortal writers who
still command the admiration of modern Europe, soon became the
favorite object of study and imitation in Italy and the western
provinces. But the elegant amusements of the Romans were not
suffered to interfere with their sound maxims of policy. Whilst
they acknowledged the charms of the Greek, they asserted the
dignity of the Latin tongue, and the exclusive use of the latter
was inflexibly maintained in the administration of civil as well
as military government. 45 The two languages exercised at the
same time their separate jurisdiction throughout the empire: the
former, as the natural idiom of science; the latter, as the legal
dialect of public transactions. Those who united letters with
business were equally conversant with both; and it was almost
impossible, in any province, to find a Roman subject, of a
liberal education, who was at once a stranger to the Greek and to
the Latin language.

\footnotetext[45]{See Valerius Maximus, l. ii. c. 2, n. 2. The
emperor Claudius disfranchised an eminent Grecian for not
understanding Latin. He was probably in some public office.
Suetonius in Claud. c. 16. * Note: Causes seem to have been
pleaded, even in the senate, in both languages. Val. Max. \textit{loc.
cit}. Dion. l. lvii. c. 15.—M}

It was by such institutions that the nations of the empire
insensibly melted away into the Roman name and people. But there
still remained, in the centre of every province and of every
family, an unhappy condition of men who endured the weight,
without sharing the benefits, of society. In the free states of
antiquity, the domestic slaves were exposed to the wanton rigor
of despotism. The perfect settlement of the Roman empire was
preceded by ages of violence and rapine. The slaves consisted,
for the most part, of barbarian captives, 451 taken in thousands
by the chance of war, purchased at a vile price, 46 accustomed to
a life of independence, and impatient to break and to revenge
their fetters. Against such internal enemies, whose desperate
insurrections had more than once reduced the republic to the
brink of destruction, 47 the most severe 471 regulations, 48 and
the most cruel treatment, seemed almost justified by the great
law of self-preservation. But when the principal nations of
Europe, Asia, and Africa were united under the laws of one
sovereign, the source of foreign supplies flowed with much less
abundance, and the Romans were reduced to the milder but more
tedious method of propagation. 481 In their numerous families,
and particularly in their country estates, they encouraged the
marriage of their slaves. 482 The sentiments of nature, the
habits of education, and the possession of a dependent species of
property, contributed to alleviate the hardships of servitude. 49
The existence of a slave became an object of greater value, and
though his happiness still depended on the temper and
circumstances of the master, the humanity of the latter, instead
of being restrained by fear, was encouraged by the sense of his
own interest. The progress of manners was accelerated by the
virtue or policy of the emperors; and by the edicts of Hadrian
and the Antonines, the protection of the laws was extended to the
most abject part of mankind. The jurisdiction of life and death
over the slaves, a power long exercised and often abused, was
taken out of private hands, and reserved to the magistrates
alone. The subterraneous prisons were abolished; and, upon a just
complaint of intolerable treatment, the injured slave obtained
either his deliverance, or a less cruel master. 50

\footnotetext[451]{It was this which rendered the wars so sanguinary,
and the battles so obstinate. The immortal Robertson, in an
excellent discourse on the state of the world at the period of
the establishment of Christianity, has traced a picture of the
melancholy effects of slavery, in which we find all the depth of
his views and the strength of his mind. I shall oppose
successively some passages to the reflections of Gibbon. The
reader will see, not without interest, the truths which Gibbon
appears to have mistaken or voluntarily neglected, developed by
one of the best of modern historians. It is important to call
them to mind here, in order to establish the facts and their
consequences with accuracy. I shall more than once have occasion
to employ, for this purpose, the discourse of Robertson.
“Captives taken in war were, in all probability, the first
persons subjected to perpetual servitude; and, when the
necessities or luxury of mankind increased the demand for slaves,
every new war recruited their number, by reducing the vanquished
to that wretched condition. Hence proceeded the fierce and
desperate spirit with which wars were carried on among ancient
nations. While chains and slavery were the certain lot of the
conquered, battles were fought, and towns defended with a rage
and obstinacy which nothing but horror at such a fate could have
inspired; but, putting an end to the cruel institution of
slavery, Christianity extended its mild influences to the
practice of war, and that barbarous art, softened by its humane
spirit, ceased to be so destructive. Secure, in every event, of
personal liberty, the resistance of the vanquished became less
obstinate, and the triumph of the victor less cruel. Thus
humanity was introduced into the exercise of war, with which it
appears to be almost incompatible; and it is to the merciful
maxims of Christianity, much more than to any other cause, that
we must ascribe the little ferocity and bloodshed which accompany
modern victories.”—G.}

\footnotetext[46]{In the camp of Lucullus, an ox sold for a drachma,
and a slave for four drachmæ, or about three shillings. Plutarch.
in Lucull. p. 580. * Note: Above 100,000 prisoners were taken in
the Jewish war.—G. Hist. of Jews, iii. 71. According to a
tradition preserved by S. Jerom, after the insurrection in the
time of Hadrian, they were sold as cheap as horse. Ibid. 124.
Compare Blair on Roman Slavery, p. 19.—M., and Dureau de la
blalle, Economie Politique des Romains, l. i. c. 15. But I cannot
think that this writer has made out his case as to the common
price of an agricultural slave being from 2000 to 2500 francs,
(80l. to 100l.) He has overlooked the passages which show the
ordinary prices, (i. e. Hor. Sat. ii. vii. 45,) and argued from
extraordinary and exceptional cases.—M. 1845.}

\footnotetext[47]{Diodorus Siculus in Eclog. Hist. l. xxxiv. and
xxxvi. Florus, iii. 19, 20.}

\footnotetext[471]{The following is the example: we shall see whether
the word “severe” is here in its place. “At the time in which L.
Domitius was prætor in Sicily, a slave killed a wild boar of
extraordinary size. The prætor, struck by the dexterity and
courage of the man, desired to see him. The poor wretch, highly
gratified with the distinction, came to present himself before
the prætor, in hopes, no doubt, of praise and reward; but
Domitius, on learning that he had only a javelin to attack and
kill the boar, ordered him to be instantly crucified, under the
barbarous pretext that the law prohibited the use of this weapon,
as of all others, to slaves.” Perhaps the cruelty of Domitius is
less astonishing than the indifference with which the Roman
orator relates this circumstance, which affects him so little
that he thus expresses himself: “Durum hoc fortasse videatur,
neque ego in ullam partem disputo.” “This may appear harsh, nor
do I give any opinion on the subject.” And it is the same orator
who exclaims in the same oration, “Facinus est cruciare civem
Romanum; scelus verberare; prope parricidium necare: quid dicam
in crucem tollere?” “It is a crime to imprison a Roman citizen;
wickedness to scourge; next to parricide to put to death, what
shall I call it to crucify?”

In general, this passage of Gibbon on slavery, is full, not only
of blamable indifference, but of an exaggeration of impartiality
which resembles dishonesty. He endeavors to extenuate all that is
appalling in the condition and treatment of the slaves; he would
make us consider those cruelties as possibly “justified by
necessity.” He then describes, with minute accuracy, the
slightest mitigations of their deplorable condition; he
attributes to the virtue or the policy of the emperors the
progressive amelioration in the lot of the slaves; and he passes
over in silence the most influential cause, that which, after
rendering the slaves less miserable, has contributed at length
entirely to enfranchise them from their sufferings and their
chains,—Christianity. It would be easy to accumulate the most
frightful, the most agonizing details, of the manner in which the
Romans treated their slaves; whole works have been devoted to the
description. I content myself with referring to them. Some
reflections of Robertson, taken from the discourse already
quoted, will make us feel that Gibbon, in tracing the mitigation
of the condition of the slaves, up to a period little later than
that which witnessed the establishment of Christianity in the
world, could not have avoided the acknowledgment of the influence
of that beneficent cause, if he had not already determined not to
speak of it.

“Upon establishing despotic government in the Roman empire,
domestic tyranny rose, in a short time, to an astonishing height.
In that rank soil, every vice, which power nourishes in the
great, or oppression engenders in the mean, thrived and grew up
apace. * * * It is not the authority of any single detached
precept in the gospel, but the spirit and genius of the Christian
religion, more powerful than any particular command, which hath
abolished the practice of slavery throughout the world. The
temper which Christianity inspired was mild and gentle; and the
doctrines it taught added such dignity and lustre to human
nature, as rescued it from the dishonorable servitude into which
it was sunk.”

It is in vain, then, that Gibbon pretends to attribute solely to
the desire of keeping up the number of slaves, the milder conduct
which the Romans began to adopt in their favor at the time of the
emperors. This cause had hitherto acted in an opposite direction;
how came it on a sudden to have a different influence? “The
masters,” he says, “encouraged the marriage of their slaves; * *
* the sentiments of nature, the habits of education, contributed
to alleviate the hardships of servitude.” The children of slaves
were the property of their master, who could dispose of or
alienate them like the rest of his property. Is it in such a
situation, with such notions, that the sentiments of nature
unfold themselves, or habits of education become mild and
peaceful? We must not attribute to causes inadequate or
altogether without force, effects which require to explain them a
reference to more influential causes; and even if these slighter
causes had in effect a manifest influence, we must not forget
that they are themselves the effect of a primary, a higher, and
more extensive cause, which, in giving to the mind and to the
character a more disinterested and more humane bias, disposed men
to second or themselves to advance, by their conduct, and by the
change of manners, the happy results which it tended to
produce.—G.

I have retained the whole of M. Guizot’s note, though, in his
zeal for the invaluable blessings of freedom and Christianity, he
has done Gibbon injustice. The condition of the slaves was
undoubtedly improved under the emperors. What a great authority
has said, “The condition of a slave is better under an arbitrary
than under a free government,” (Smith’s Wealth of Nations, iv.
7,) is, I believe, supported by the history of all ages and
nations. The protecting edicts of Hadrian and the Antonines are
historical facts, and can as little be attributed to the
influence of Christianity, as the milder language of heathen
writers, of Seneca, (particularly Ep. 47,) of Pliny, and of
Plutarch. The latter influence of Christianity is admitted by
Gibbon himself. The subject of Roman slavery has recently been
investigated with great diligence in a very modest but valuable
volume, by Wm. Blair, Esq., Edin. 1833. May we be permitted,
while on the subject, to refer to the most splendid passage
extant of Mr. Pitt’s eloquence, the description of the Roman
slave-dealer. on the shores of Britain, condemning the island to
irreclaimable barbarism, as a perpetual and prolific nursery of
slaves? Speeches, vol. ii. p. 80.

Gibbon, it should be added, was one of the first and most
consistent opponents of the African slave-trade. (See Hist. ch.
xxv. and Letters to Lor Sheffield, Misc. Works)—M.}

\footnotetext[48]{See a remarkable instance of severity in Cicero in
Verrem, v. 3.}

\footnotetext[481]{An active slave-trade, which was carried on in
many quarters, particularly the Euxine, the eastern provinces,
the coast of Africa, and British must be taken into the account.
Blair, 23—32.—M.}

\footnotetext[482]{The Romans, as well in the first ages of the
republic as later, allowed to their slaves a kind of marriage,
(contubernium: ) notwithstanding this, luxury made a greater
number of slaves in demand. The increase in their population was
not sufficient, and recourse was had to the purchase of slaves,
which was made even in the provinces of the East subject to the
Romans. It is, moreover, known that slavery is a state little
favorable to population. (See Hume’s Essay, and Malthus on
population, i. 334.—G.) The testimony of Appian (B.C. l. i. c. 7)
is decisive in favor of the rapid multiplication of the
agricultural slaves; it is confirmed by the numbers engaged in
the servile wars. Compare also Blair, p. 119; likewise Columella
l. viii.—M.}

\footnotetext[49]{See in Gruter, and the other collectors, a great
number of inscriptions addressed by slaves to their wives,
children, fellow-servants, masters, \&c. They are all most
probably of the Imperial age.}

\footnotetext[50]{See the Augustan History, and a Dissertation of M.
de Burigny, in the xxxvth volume of the Academy of Inscriptions,
upon the Roman slaves.}

Hope, the best comfort of our imperfect condition, was not denied
to the Roman slave; and if he had any opportunity of rendering
himself either useful or agreeable, he might very naturally
expect that the diligence and fidelity of a few years would be
rewarded with the inestimable gift of freedom. The benevolence of
the master was so frequently prompted by the meaner suggestions
of vanity and avarice, that the laws found it more necessary to
restrain than to encourage a profuse and undistinguishing
liberality, which might degenerate into a very dangerous abuse.
51 It was a maxim of ancient jurisprudence, that a slave had not
any country of his own; he acquired with his liberty an admission
into the political society of which his patron was a member. The
consequences of this maxim would have prostituted the privileges
of the Roman city to a mean and promiscuous multitude. Some
seasonable exceptions were therefore provided; and the honorable
distinction was confined to such slaves only as, for just causes,
and with the approbation of the magistrate, should receive a
solemn and legal manumission. Even these chosen freedmen obtained
no more than the private rights of citizens, and were rigorously
excluded from civil or military honors. Whatever might be the
merit or fortune of their sons, \textit{they} likewise were esteemed
unworthy of a seat in the senate; nor were the traces of a
servile origin allowed to be completely obliterated till the
third or fourth generation. 52 Without destroying the distinction
of ranks, a distant prospect of freedom and honors was presented,
even to those whom pride and prejudice almost disdained to number
among the human species.

\footnotetext[51]{See another Dissertation of M. de Burigny, in the
xxxviith volume, on the Roman freedmen.}

\footnotetext[52]{Spanheim, Orbis Roman. l. i. c. 16, p. 124, \&c.}
It was once proposed to discriminate the slaves by a peculiar habit;
but it was justly apprehended that there might be some danger in
acquainting them with their own numbers. 53 Without interpreting,
in their utmost strictness, the liberal appellations of legions
and myriads, 54 we may venture to pronounce, that the proportion
of slaves, who were valued as property, was more considerable
than that of servants, who can be computed only as an expense. 55
The youths of a promising genius were instructed in the arts and
sciences, and their price was ascertained by the degree of their
skill and talents. 56 Almost every profession, either liberal 57
or mechanical, might be found in the household of an opulent
senator. The ministers of pomp and sensuality were multiplied
beyond the conception of modern luxury. 58 It was more for the
interest of the merchant or manufacturer to purchase, than to
hire his workmen; and in the country, slaves were employed as the
cheapest and most laborious instruments of agriculture. To
confirm the general observation, and to display the multitude of
slaves, we might allege a variety of particular instances. It was
discovered, on a very melancholy occasion, that four hundred
slaves were maintained in a single palace of Rome. 59 The same
number of four hundred belonged to an estate which an African
widow, of a very private condition, resigned to her son, whilst
she reserved for herself a much larger share of her property. 60
A freedman, under the name of Augustus, though his fortune had
suffered great losses in the civil wars, left behind him three
thousand six hundred yoke of oxen, two hundred and fifty thousand
head of smaller cattle, and what was almost included in the
description of cattle, four thousand one hundred and sixteen
slaves. 61

\footnotetext[53]{Seneca de Clementia, l. i. c. 24. The original is
much stronger, “Quantum periculum immineret si servi nostri
numerare nos cœpissent.”}

\footnotetext[54]{See Pliny (Hist. Natur. l. xxxiii.) and Athenæus
(Deipnosophist. l. vi. p. 272.) The latter boldly asserts, that
he knew very many Romans who possessed, not for use, but
ostentation, ten and even twenty thousand slaves.}

\footnotetext[55]{In Paris there are not more than 43,000 domestics
of every sort, and not a twelfth part of the inhabitants.
Messange, Recherches sui la Population, p. 186.}

\footnotetext[56]{A learned slave sold for many hundred pounds
sterling: Atticus always bred and taught them himself. Cornel.
Nepos in Vit. c. 13, [on the prices of slaves. Blair, 149.]—M.}

\footnotetext[57]{Many of the Roman physicians were slaves. See Dr.
Middleton’s Dissertation and Defence.}

\footnotetext[58]{Their ranks and offices are very copiously
enumerated by Pignorius de Servis.}

\footnotetext[59]{Tacit. Annal. xiv. 43. They were all executed for
not preventing their master’s murder. * Note: The remarkable
speech of Cassius shows the proud feelings of the Roman
aristocracy on this subject.—M}

\footnotetext[60]{Apuleius in Apolog. p. 548. edit. Delphin}

\footnotetext[61]{Plin. Hist. Natur. l. xxxiii. 47.}

The number of subjects who acknowledged the laws of Rome, of
citizens, of provincials, and of slaves, cannot now be fixed with
such a degree of accuracy, as the importance of the object would
deserve. We are informed, that when the Emperor Claudius
exercised the office of censor, he took an account of six
millions nine hundred and forty-five thousand Roman citizens,
who, with the proportion of women and children, must have
amounted to about twenty millions of souls. The multitude of
subjects of an inferior rank was uncertain and fluctuating. But,
after weighing with attention every circumstance which could
influence the balance, it seems probable that there existed, in
the time of Claudius, about twice as many provincials as there
were citizens, of either sex, and of every age; and that the
slaves were at least equal in number to the free inhabitants of
the Roman world.611 The total amount of this imperfect
calculation would rise to about one hundred and twenty millions
of persons; a degree of population which possibly exceeds that of
modern Europe, 62 and forms the most numerous society that has
ever been united under the same system of government.

\footnotetext[611]{According to Robertson, there were twice as many slaves as free
citizens.—G. Mr. Blair (p. 15) estimates three slaves to one
freeman, between the conquest of Greece, B.C. 146, and the reign
of Alexander Severus, A. D. 222, 235. The proportion was probably
larger in Italy than in the provinces.—M. On the other hand,
Zumpt, in his Dissertation quoted below, (p. 86,) asserts it to
be a gross error in Gibbon to reckon the number of slaves equal
to that of the free population. The luxury and magnificence of
the great, (he observes,) at the commencement of the empire, must
not be taken as the groundwork of calculations for the whole
Roman world. “The agricultural laborer, and the artisan, in
Spain, Gaul, Britain, Syria, and Egypt, maintained himself, as in
the present day, by his own labor and that of his household,
without possessing a single slave.” The latter part of my note
was intended to suggest this consideration. Yet so completely was
slavery rooted in the social system, both in the east and the
west, that in the great diffusion of wealth at this time, every
one, I doubt not, who could afford a domestic slave, kept one;
and generally, the number of slaves was in proportion to the
wealth. I do not believe that the cultivation of the soil by
slaves was confined to Italy; the holders of large estates in the
provinces would probably, either from choice or necessity, adopt
the same mode of cultivation. The latifundia, says Pliny, had
ruined Italy, and had begun to ruin the provinces. Slaves were no
doubt employed in agricultural labor to a great extent in Sicily,
and were the estates of those six enormous landholders who were
said to have possessed the whole province of Africa, cultivated
altogether by free coloni? Whatever may have been the case in the
rural districts, in the towns and cities the household duties
were almost entirely discharged by slaves, and vast numbers
belonged to the public establishments. I do not, however, differ
so far from Zumpt, and from M. Dureau de la Malle, as to adopt
the higher and bolder estimate of Robertson and Mr. Blair, rather
than the more cautious suggestions of Gibbon. I would reduce
rather than increase the proportion of the slave population. The
very ingenious and elaborate calculations of the French writer,
by which he deduces the amount of the population from the produce
and consumption of corn in Italy, appear to me neither precise
nor satisfactory bases for such complicated political arithmetic.
I am least satisfied with his views as to the population of the
city of Rome; but this point will be more fitly reserved for a
note on the thirty-first chapter of Gibbon. The work, however, of
M. Dureau de la Malle is very curious and full on some of the
minuter points of Roman statistics.—M. 1845.}

\footnotetext[62]{Compute twenty millions in France, twenty-two in
Germany, four in Hungary, ten in Italy with its islands, eight in
Great Britain and Ireland, eight in Spain and Portugal, ten or
twelve in the European Russia, six in Poland, six in Greece and
Turkey, four in Sweden, three in Denmark and Norway, four in the
Low Countries. The whole would amount to one hundred and five, or
one hundred and seven millions. See Voltaire, de l’Histoire
Generale. * Note: The present population of Europe is estimated
at 227,700,000. Malts Bran, Geogr. Trans edit. 1832 See details
in the different volumes Another authority, (Almanach de Gotha,)
quoted in a recent English publication, gives the following
details:—

France, 32,897,521 Germany, (including Hungary, Prussian and
Austrian Poland,) 56,136,213 Italy, 20,548,616 Great Britain and
Ireland, 24,062,947 Spain and Portugal, 13,953,959. 3,144,000
Russia, including Poland, 44,220,600 Cracow, 128,480 Turkey,
(including Pachalic of Dschesair,) 9,545,300 Greece, 637,700
Ionian Islands, 208,100 Sweden and Norway, 3,914,963 Denmark,
2,012,998 Belgium, 3,533,538 Holland, 2,444,550 Switzerland,
985,000. Total, 219,344,116

Since the publication of my first annotated edition of Gibbon,
the subject of the population of the Roman empire has been
investigated by two writers of great industry and learning; Mons.
Dureau de la Malle, in his Economie Politique des Romains, liv.
ii. c. 1. to 8, and M. Zumpt, in a dissertation printed in the
Transactions of the Berlin Academy, 1840. M. Dureau de la Malle
confines his inquiry almost entirely to the city of Rome, and
Roman Italy. Zumpt examines at greater length the axiom, which he
supposes to have been assumed by Gibbon as unquestionable, “that
Italy and the Roman world was never so populous as in the time of
the Antonines.” Though this probably was Gibbon’s opinion, he has
not stated it so peremptorily as asserted by Mr. Zumpt. It had
before been expressly laid down by Hume, and his statement was
controverted by Wallace and by Malthus. Gibbon says (p. 84) that
there is no reason to believe the country (of Italy) less
populous in the age of the Antonines, than in that of Romulus;
and Zumpt acknowledges that we have no satisfactory knowledge of
the state of Italy at that early age. Zumpt, in my opinion with
some reason, takes the period just before the first Punic war, as
that in which Roman Italy (all south of the Rubicon) was most
populous. From that time, the numbers began to diminish, at first
from the enormous waste of life out of the free population in the
foreign, and afterwards in the civil wars; from the cultivation
of the soil by slaves; towards the close of the republic, from
the repugnance to marriage, which resisted alike the dread of
legal punishment and the offer of legal immunity and privilege;
and from the depravity of manners, which interfered with the
procreation, the birth, and the rearing of children. The
arguments and the authorities of Zumpt are equally conclusive as
to the decline of population in Greece. Still the details, which
he himself adduces as to the prosperity and populousness of Asia
Minor, and the whole of the Roman East, with the advancement of
the European provinces, especially Gaul, Spain, and Britain, in
civilization, and therefore in populousness, (for I have no
confidence in the vast numbers sometimes assigned to the
barbarous inhabitants of these countries,) may, I think, fairly
compensate for any deduction to be made from Gibbon’s general
estimate on account of Greece and Italy. Gibbon himself
acknowledges his own estimate to be vague and conjectural; and I
may venture to recommend the dissertation of Zumpt as deserving
respectful consideration.—M 1815.}

