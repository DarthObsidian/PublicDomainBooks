\section{Part \thesection.}
\thispagestyle{simple}

The virtues and the reputation of the new emperors justified the
most sanguine hopes of the Romans. The various nature of their
talents seemed to appropriate to each his peculiar department of
peace and war, without leaving room for jealous emulation.
Balbinus was an admired orator, a poet of distinguished fame, and
a wise magistrate, who had exercised with innocence and applause
the civil jurisdiction in almost all the interior provinces of
the empire. His birth was noble,\footnotemark[28] his fortune affluent, his
manners liberal and affable. In him the love of pleasure was
corrected by a sense of dignity, nor had the habits of ease
deprived him of a capacity for business. The mind of Maximus was
formed in a rougher mould. By his valor and abilities he had
raised himself from the meanest origin to the first employments
of the state and army. His victories over the Sarmatians and the
Germans, the austerity of his life, and the rigid impartiality of
his justice, while he was a Præfect of the city, commanded the
esteem of a people whose affections were engaged in favor of the
more amiable Balbinus. The two colleagues had both been consuls,
(Balbinus had twice enjoyed that honorable office,) both had been
named among the twenty lieutenants of the senate; and since the
one was sixty and the other seventy-four years old,\footnotemark[29] they had
both attained the full maturity of age and experience.

\footnotetext[28]{He was descended from Cornelius Balbus, a noble
Spaniard, and the adopted son of Theophanes, the Greek historian.
Balbus obtained the freedom of Rome by the favor of Pompey, and
preserved it by the eloquence of Cicero. (See Orat. pro Cornel.
Balbo.) The friendship of Cæsar, (to whom he rendered the most
important secret services in the civil war) raised him to the
consulship and the pontificate, honors never yet possessed by a
stranger. The nephew of this Balbus triumphed over the
Garamantes. See Dictionnaire de Bayle, au mot Balbus, where he
distinguishes the several persons of that name, and rectifies,
with his usual accuracy, the mistakes of former writers
concerning them.}

\footnotetext[29]{Zonaras, l. xii. p. 622. But little dependence is
to be had on the authority of a modern Greek, so grossly ignorant
of the history of the third century, that he creates several
imaginary emperors, and confounds those who really existed.}

After the senate had conferred on Maximus and Balbinus an equal
portion of the consular and tribunitian powers, the title of
Fathers of their country, and the joint office of Supreme
Pontiff, they ascended to the Capitol to return thanks to the
gods, protectors of Rome.\footnotemark[30] The solemn rites of sacrifice were
disturbed by a sedition of the people. The licentious multitude
neither loved the rigid Maximus, nor did they sufficiently fear
the mild and humane Balbinus. Their increasing numbers surrounded
the temple of Jupiter; with obstinate clamors they asserted their
inherent right of consenting to the election of their sovereign;
and demanded, with an apparent moderation, that, besides the two
emperors, chosen by the senate, a third should be added of the
family of the Gordians, as a just return of gratitude to those
princes who had sacrificed their lives for the republic. At the
head of the city-guards, and the youth of the equestrian order,
Maximus and Balbinus attempted to cut their way through the
seditious multitude. The multitude, armed with sticks and stones,
drove them back into the Capitol. It is prudent to yield when the
contest, whatever may be the issue of it, must be fatal to both
parties. A boy, only thirteen years of age, the grandson of the
elder, and nephew\footnotemark[301] of the younger Gordian, was produced to the
people, invested with the ornaments and title of Cæsar. The
tumult was appeased by this easy condescension; and the two
emperors, as soon as they had been peaceably acknowledged in
Rome, prepared to defend Italy against the common enemy.

\footnotetext[30]{Herodian, l. vii. p. 256, supposes that the senate
was at first convoked in the Capitol, and is very eloquent on the
occasion. The Augustar History p. 116, seems much more
authentic.}

\footnotetext[301]{According to some, the son.—G.}

Whilst in Rome and Africa, revolutions succeeded each other with
such amazing rapidity, that the mind of Maximin was agitated by
the most furious passions. He is said to have received the news
of the rebellion of the Gordians, and of the decree of the senate
against him, not with the temper of a man, but the rage of a wild
beast; which, as it could not discharge itself on the distant
senate, threatened the life of his son, of his friends, and of
all who ventured to approach his person. The grateful
intelligence of the death of the Gordians was quickly followed by
the assurance that the senate, laying aside all hopes of pardon
or accommodation, had substituted in their room two emperors,
with whose merit he could not be unacquainted. Revenge was the
only consolation left to Maximin, and revenge could only be
obtained by arms. The strength of the legions had been assembled
by Alexander from all parts of the empire. Three successful
campaigns against the Germans and the Sarmatians, had raised
their fame, confirmed their discipline, and even increased their
numbers, by filling the ranks with the flower of the barbarian
youth. The life of Maximin had been spent in war, and the candid
severity of history cannot refuse him the valor of a soldier, or
even the abilities of an experienced general.\footnotemark[31] It might
naturally be expected, that a prince of such a character, instead
of suffering the rebellion to gain stability by delay, should
immediately have marched from the banks of the Danube to those of
the Tyber, and that his victorious army, instigated by contempt
for the senate, and eager to gather the spoils of Italy, should
have burned with impatience to finish the easy and lucrative
conquest. Yet as far as we can trust to the obscure chronology of
that period,\footnotemark[32] it appears that the operations of some foreign
war deferred the Italian expedition till the ensuing spring. From
the prudent conduct of Maximin, we may learn that the savage
features of his character have been exaggerated by the pencil of
party, that his passions, however impetuous, submitted to the
force of reason, and that the barbarian possessed something of
the generous spirit of Sylla, who subdued the enemies of Rome
before he suffered himself to revenge his private injuries.\footnotemark[33]

\footnotetext[31]{In Herodian, l. vii. p. 249, and in the Augustan
History, we have three several orations of Maximin to his army,
on the rebellion of Africa and Rome: M. de Tillemont has very
justly observed that they neither agree with each other nor with
truth. Histoire des Empereurs, tom. iii. p. 799.}

\footnotetext[32]{The carelessness of the writers of that age, leaves
us in a singular perplexity. 1. We know that Maximus and Balbinus
were killed during the Capitoline games. Herodian, l. viii. p.
285. The authority of Censorinus (de Die Natali, c. 18) enables
us to fix those games with certainty to the year 238, but leaves
us in ignorance of the month or day. 2. The election of Gordian
by the senate is fixed with equal certainty to the 27th of May;
but we are at a loss to discover whether it was in the same or
the preceding year. Tillemont and Muratori, who maintain the two
opposite opinions, bring into the field a desultory troop of
authorities, conjectures and probabilities. The one seems to draw
out, the other to contract the series of events between those
periods, more than can be well reconciled to reason and history.
Yet it is necessary to choose between them. Note: Eckhel has more
recently treated these chronological questions with a perspicuity
which gives great probability to his conclusions. Setting aside
all the historians, whose contradictions are irreconcilable, he
has only consulted the medals, and has arranged the events before
us in the following order:— Maximin, A. U. 990, after having
conquered the Germans, reenters Pannonia, establishes his winter
quarters at Sirmium, and prepares himself to make war against the
people of the North. In the year 991, in the cal ends of January,
commences his fourth tribunate. The Gordians are chosen emperors
in Africa, probably at the beginning of the month of March. The
senate confirms this election with joy, and declares Maximin the
enemy of Rome. Five days after he had heard of this revolt,
Maximin sets out from Sirmium on his march to Italy. These events
took place about the beginning of April; a little after, the
Gordians are slain in Africa by Capellianus, procurator of
Mauritania. The senate, in its alarm, names as emperors Balbus
and Maximus Pupianus, and intrusts the latter with the war
against Maximin. Maximin is stopped on his road near Aquileia, by
the want of provisions, and by the melting of the snows: he
begins the siege of Aquileia at the end of April. Pupianus
assembles his army at Ravenna. Maximin and his son are
assassinated by the soldiers enraged at the resistance of
Aquileia: and this was probably in the middle of May. Pupianus
returns to Rome, and assumes the government with Balbinus; they
are assassinated towards the end of July Gordian the younger
ascends the throne. Eckhel de Doct. Vol vii 295.—G.}

\footnotetext[33]{Velleius Paterculus, l. ii. c. 24. The president de
Montesquieu (in his dialogue between Sylla and Eucrates)
expresses the sentiments of the dictator in a spirited, and even
a sublime manner.}

When the troops of Maximin, advancing in excellent order, arrived
at the foot of the Julian Alps, they were terrified by the
silence and desolation that reigned on the frontiers of Italy.
The villages and open towns had been abandoned on their approach
by the inhabitants, the cattle was driven away, the provisions
removed or destroyed, the bridges broken down, nor was any thing
left which could afford either shelter or subsistence to an
invader. Such had been the wise orders of the generals of the
senate: whose design was to protract the war, to ruin the army of
Maximin by the slow operation of famine, and to consume his
strength in the sieges of the principal cities of Italy, which
they had plentifully stored with men and provisions from the
deserted country. Aquileia received and withstood the first shock
of the invasion. The streams that issue from the head of the
Hadriatic Gulf, swelled by the melting of the winter snows,\footnotemark[34]
opposed an unexpected obstacle to the arms of Maximin. At length,
on a singular bridge, constructed with art and difficulty, of
large hogsheads, he transported his army to the opposite bank,
rooted up the beautiful vineyards in the neighborhood of
Aquileia, demolished the suburbs, and employed the timber of the
buildings in the engines and towers, with which on every side he
attacked the city. The walls, fallen to decay during the security
of a long peace, had been hastily repaired on this sudden
emergency: but the firmest defence of Aquileia consisted in the
constancy of the citizens; all ranks of whom, instead of being
dismayed, were animated by the extreme danger, and their
knowledge of the tyrant’s unrelenting temper. Their courage was
supported and directed by Crispinus and Menophilus, two of the
twenty lieutenants of the senate, who, with a small body of
regular troops, had thrown themselves into the besieged place.
The army of Maximin was repulsed in repeated attacks, his
machines destroyed by showers of artificial fire; and the
generous enthusiasm of the Aquileians was exalted into a
confidence of success, by the opinion that Belenus, their tutelar
deity, combated in person in the defence of his distressed
worshippers.\footnotemark[35]

\footnotetext[34]{Muratori (Annali d’ Italia, tom. ii. p. 294) thinks
the melting of the snows suits better with the months of June or
July, than with those of February. The opinion of a man who
passed his life between the Alps and the Apennines, is
undoubtedly of great weight; yet I observe, 1. That the long
winter, of which Muratori takes advantage, is to be found only in
the Latin version, and not in the Greek text of Herodian. 2. That
the vicissitudes of suns and rains, to which the soldiers of
Maximin were exposed, (Herodian, l. viii. p. 277,) denote the
spring rather than the summer. We may observe, likewise, that
these several streams, as they melted into one, composed the
Timavus, so poetically (in every sense of the word) described by
Virgil. They are about twelve miles to the east of Aquileia. See
Cluver. Italia Antiqua, tom. i. p. 189, \&c.}

\footnotetext[35]{Herodian, l. viii. p. 272. The Celtic deity was
supposed to be Apollo, and received under that name the thanks of
the senate. A temple was likewise built to Venus the Bald, in
honor of the women of Aquileia, who had given up their hair to
make ropes for the military engines.}

The emperor Maximus, who had advanced as far as Ravenna, to
secure that important place, and to hasten the military
preparations, beheld the event of the war in the more faithful
mirror of reason and policy. He was too sensible, that a single
town could not resist the persevering efforts of a great army;
and he dreaded, lest the enemy, tired with the obstinate
resistance of Aquileia, should on a sudden relinquish the
fruitless siege, and march directly towards Rome. The fate of the
empire and the cause of freedom must then be committed to the
chance of a battle; and what arms could he oppose to the veteran
legions of the Rhine and Danube? Some troops newly levied among
the generous but enervated youth of Italy; and a body of German
auxiliaries, on whose firmness, in the hour of trial, it was
dangerous to depend. In the midst of these just alarms, the
stroke of domestic conspiracy punished the crimes of Maximin, and
delivered Rome and the senate from the calamities that would
surely have attended the victory of an enraged barbarian.

The people of Aquileia had scarcely experienced any of the common
miseries of a siege; their magazines were plentifully supplied,
and several fountains within the walls assured them of an
inexhaustible resource of fresh water. The soldiers of Maximin
were, on the contrary, exposed to the inclemency of the season,
the contagion of disease, and the horrors of famine. The open
country was ruined, the rivers filled with the slain, and
polluted with blood. A spirit of despair and disaffection began
to diffuse itself among the troops; and as they were cut off from
all intelligence, they easily believed that the whole empire had
embraced the cause of the senate, and that they were left as
devoted victims to perish under the impregnable walls of
Aquileia. The fierce temper of the tyrant was exasperated by
disappointments, which he imputed to the cowardice of his army;
and his wanton and ill-timed cruelty, instead of striking terror,
inspired hatred, and a just desire of revenge. A party of
Prætorian guards, who trembled for their wives and children in
the camp of Alba, near Rome, executed the sentence of the senate.
Maximin, abandoned by his guards, was slain in his tent, with his
son (whom he had associated to the honors of the purple),
Anulinus the præfect, and the principal ministers of his tyranny.\footnotemark[36]
The sight of their heads, borne on the point of spears,
convinced the citizens of Aquileia that the siege was at an end;
the gates of the city were thrown open, a liberal market was
provided for the hungry troops of Maximin, and the whole army
joined in solemn protestations of fidelity to the senate and the
people of Rome, and to their lawful emperors Maximus and
Balbinus. Such was the deserved fate of a brutal savage,
destitute, as he has generally been represented, of every
sentiment that distinguishes a civilized, or even a human being.
The body was suited to the soul. The stature of Maximin exceeded
the measure of eight feet, and circumstances almost incredible
are related of his matchless strength and appetite.\footnotemark[37] Had he
lived in a less enlightened age, tradition and poetry might well
have described him as one of those monstrous giants, whose
supernatural power was constantly exerted for the destruction of
mankind.

\footnotetext[36]{Herodian, l. viii. p. 279. Hist. August. p. 146.
The duration of Maximin’s reign has not been defined with much
accuracy, except by Eutropius, who allows him three years and a
few days, (l. ix. 1;) we may depend on the integrity of the text,
as the Latin original is checked by the Greek version of
Pæanius.}

\footnotetext[37]{Eight Roman feet and one third, which are equal to
above eight English feet, as the two measures are to each other
in the proportion of 967 to 1000. See Graves’s discourse on the
Roman foot. We are told that Maximin could drink in a day an
amphora (or about seven gallons) of wine, and eat thirty or forty
pounds of meat. He could move a loaded wagon, break a horse’s leg
with his fist, crumble stones in his hand, and tear up small
trees by the roots. See his life in the Augustan History.}

It is easier to conceive than to describe the universal joy of
the Roman world on the fall of the tyrant, the news of which is
said to have been carried in four days from Aquileia to Rome. The
return of Maximus was a triumphal procession; his colleague and
young Gordian went out to meet him, and the three princes made
their entry into the capital, attended by the ambassadors of
almost all the cities of Italy, saluted with the splendid
offerings of gratitude and superstition, and received with the
unfeigned acclamations of the senate and people, who persuaded
themselves that a golden age would succeed to an age of iron.\footnotemark[38]
The conduct of the two emperors corresponded with these
expectations. They administered justice in person; and the rigor
of the one was tempered by the other’s clemency. The oppressive
taxes with which Maximin had loaded the rights of inheritance and
succession, were repealed, or at least moderated. Discipline was
revived, and with the advice of the senate many wise laws were
enacted by their imperial ministers, who endeavored to restore a
civil constitution on the ruins of military tyranny. “What reward
may we expect for delivering Rome from a monster?” was the
question asked by Maximus, in a moment of freedom and confidence.

Balbinus answered it without hesitation—“The love of the senate,
of the people, and of all mankind.” “Alas!” replied his more
penetrating colleague — “alas! I dread the hatred of the soldiers,
and the fatal effects of their resentment.”\footnotemark[39] His apprehensions
were but too well justified by the event.

\footnotetext[38]{See the congratulatory letter of Claudius Julianus,
the consul to the two emperors, in the Augustan History.}

\footnotetext[39]{Hist. August. p. 171.}

Whilst Maximus was preparing to defend Italy against the common
foe, Balbinus, who remained at Rome, had been engaged in scenes
of blood and intestine discord. Distrust and jealousy reigned in
the senate; and even in the temples where they assembled, every
senator carried either open or concealed arms. In the midst of
their deliberations, two veterans of the guards, actuated either
by curiosity or a sinister motive, audaciously thrust themselves
into the house, and advanced by degrees beyond the altar of
Victory. Gallicanus, a consular, and Mæcenas, a Prætorian
senator, viewed with indignation their insolent intrusion:
drawing their daggers, they laid the spies (for such they deemed
them) dead at the foot of the altar, and then, advancing to the
door of the senate, imprudently exhorted the multitude to
massacre the Prætorians, as the secret adherents of the tyrant.
Those who escaped the first fury of the tumult took refuge in the
camp, which they defended with superior advantage against the
reiterated attacks of the people, assisted by the numerous bands
of gladiators, the property of opulent nobles. The civil war
lasted many days, with infinite loss and confusion on both sides.
When the pipes were broken that supplied the camp with water, the
Prætorians were reduced to intolerable distress; but in their
turn they made desperate sallies into the city, set fire to a
great number of houses, and filled the streets with the blood of
the inhabitants. The emperor Balbinus attempted, by ineffectual
edicts and precarious truces, to reconcile the factions at Rome.
But their animosity, though smothered for a while, burnt with
redoubled violence. The soldiers, detesting the senate and the
people, despised the weakness of a prince, who wanted either the
spirit or the power to command the obedience of his subjects.\footnotemark[40]

\footnotetext[40]{Herodian, l. viii. p. 258.}

After the tyrant’s death, his formidable army had acknowledged,
from necessity rather than from choice, the authority of Maximus,
who transported himself without delay to the camp before
Aquileia. As soon as he had received their oath of fidelity, he
addressed them in terms full of mildness and moderation;
lamented, rather than arraigned the wild disorders of the times,
and assured the soldiers, that of all their past conduct the
senate would remember only their generous desertion of the
tyrant, and their voluntary return to their duty. Maximus
enforced his exhortations by a liberal donative, purified the
camp by a solemn sacrifice of expiation, and then dismissed the
legions to their several provinces, impressed, as he hoped, with
a lively sense of gratitude and obedience.\footnotemark[41] But nothing could
reconcile the haughty spirit of the Prætorians. They attended the
emperors on the memorable day of their public entry into Rome;
but amidst the general acclamations, the sullen, dejected
countenance of the guards sufficiently declared that they
considered themselves as the object, rather than the partners, of
the triumph. When the whole body was united in their camp, those
who had served under Maximin, and those who had remained at Rome,
insensibly communicated to each other their complaints and
apprehensions. The emperors chosen by the army had perished with
ignominy; those elected by the senate were seated on the throne.\footnotemark[42]
The long discord between the civil and military powers was
decided by a war, in which the former had obtained a complete
victory. The soldiers must now learn a new doctrine of submission
to the senate; and whatever clemency was affected by that politic
assembly, they dreaded a slow revenge, colored by the name of
discipline, and justified by fair pretences of the public good.
But their fate was still in their own hands; and if they had
courage to despise the vain terrors of an impotent republic, it
was easy to convince the world, that those who were masters of
the arms, were masters of the authority, of the state.

\footnotetext[41]{Herodian, l. viii. p. 213.}

\footnotetext[42]{The observation had been made imprudently enough in
the acclamations of the senate, and with regard to the soldiers
it carried the appearance of a wanton insult. Hist. August. p.
170.}

When the senate elected two princes, it is probable that, besides
the declared reason of providing for the various emergencies of
peace and war, they were actuated by the secret desire of
weakening by division the despotism of the supreme magistrate.
Their policy was effectual, but it proved fatal both to their
emperors and to themselves. The jealousy of power was soon
exasperated by the difference of character. Maximus despised
Balbinus as a luxurious noble, and was in his turn disdained by
his colleague as an obscure soldier. Their silent discord was
understood rather than seen;\footnotemark[43] but the mutual consciousness
prevented them from uniting in any vigorous measures of defence
against their common enemies of the Prætorian camp. The whole
city was employed in the Capitoline games, and the emperors were
left almost alone in the palace. On a sudden, they were alarmed
by the approach of a troop of desperate assassins. Ignorant of
each other’s situation or designs (for they already occupied very
distant apartments), afraid to give or to receive assistance,
they wasted the important moments in idle debates and fruitless
recriminations. The arrival of the guards put an end to the vain
strife. They seized on these emperors of the senate, for such
they called them with malicious contempt, stripped them of their
garments, and dragged them in insolent triumph through the
streets of Rome, with the design of inflicting a slow and cruel
death on these unfortunate princes. The fear of a rescue from the
faithful Germans of the Imperial guards shortened their tortures;
and their bodies, mangled with a thousand wounds, were left
exposed to the insults or to the pity of the populace.\footnotemark[44]

\footnotetext[43]{Discordiæ tacitæ, et quæ intelligerentur potius
quam viderentur. \textit{Hist. August}. p. 170. This well-chosen
expression is probably stolen from some better writer.}

\footnotetext[44]{Herodian, l. viii. p. 287, 288.}

In the space of a few months, six princes had been cut off by the
sword. Gordian, who had already received the title of Cæsar, was
the only person that occurred to the soldiers as proper to fill
the vacant throne.\footnotemark[45] They carried him to the camp, and
unanimously saluted him Augustus and Emperor. His name was dear
to the senate and people; his tender age promised a long impunity
of military license; and the submission of Rome and the provinces
to the choice of the Prætorian guards saved the republic, at the
expense indeed of its freedom and dignity, from the horrors of a
new civil war in the heart of the capital.\footnotemark[46]

\footnotetext[45]{Quia non alius erat in præsenti, is the expression
of the Augustan History.}

\footnotetext[46]{Quintus Curtius (l. x. c. 9,) pays an elegant
compliment to the emperor of the day, for having, by his happy
accession, extinguished so many firebrands, sheathed so many
swords, and put an end to the evils of a divided government.
After weighing with attention every word of the passage, I am of
opinion, that it suits better with the elevation of Gordian, than
with any other period of the Roman history. In that case, it may
serve to decide the age of Quintus Curtius. Those who place him
under the first Cæsars, argue from the purity of his style but
are embarrassed by the silence of Quintilian, in his accurate
list of Roman historians. * Note: This conjecture of Gibbon is
without foundation. Many passages in the work of Quintus Curtius
clearly place him at an earlier period. Thus, in speaking of the
Parthians, he says, Hinc in Parthicum perventum est, tunc
ignobilem gentem: nunc caput omnium qui post Euphratem et Tigrim
amnes siti Rubro mari terminantur. The Parthian empire had this
extent only in the first age of the vulgar æra: to that age,
therefore, must be assigned the date of Quintus Curtius. Although
the critics (says M. de Sainte Croix) have multiplied conjectures
on this subject, most of them have ended by adopting the opinion
which places Quintus Curtius under the reign of Claudius. See
Just. Lips. ad Ann. Tac. ii. 20. Michel le Tellier Præf. in Curt.
Tillemont Hist. des Emp. i. p. 251. Du Bos Reflections sur la
Poesie, 2d Partie. Tiraboschi Storia della, Lett. Ital. ii. 149.
Examen. crit. des Historiens d’Alexandre, 2d ed. p. 104, 849,
850.—G. ——This interminable question seems as much perplexed as
ever. The first argument of M. Guizot is a strong one, except
that Parthian is often used by later writers for Persian.
Cunzius, in his preface to an edition published at Helmstadt,
(1802,) maintains the opinion of Bagnolo, which assigns Q.
Curtius to the time of Constantine the Great. Schmieder, in his
edit. Gotting. 1803, sums up in this sentence, ætatem Curtii
ignorari pala mest.—M.}

As the third Gordian was only nineteen years of age at the time
of his death, the history of his life, were it known to us with
greater accuracy than it really is, would contain little more
than the account of his education, and the conduct of the
ministers, who by turns abused or guided the simplicity of his
unexperienced youth. Immediately after his accession, he fell
into the hands of his mother’s eunuchs, that pernicious vermin of
the East, who, since the days of Elagabalus, had infested the
Roman palace. By the artful conspiracy of these wretches, an
impenetrable veil was drawn between an innocent prince and his
oppressed subjects, the virtuous disposition of Gordian was
deceived, and the honors of the empire sold without his
knowledge, though in a very public manner, to the most worthless
of mankind. We are ignorant by what fortunate accident the
emperor escaped from this ignominious slavery, and devolved his
confidence on a minister, whose wise counsels had no object
except the glory of his sovereign and the happiness of the
people. It should seem that love and learning introduced
Misitheus to the favor of Gordian. The young prince married the
daughter of his master of rhetoric, and promoted his
father-in-law to the first offices of the empire. Two admirable
letters that passed between them are still extant. The minister,
with the conscious dignity of virtue, congratulates Gordian that
he is delivered from the tyranny of the eunuchs,\footnotemark[47] and still
more that he is sensible of his deliverance. The emperor
acknowledges, with an amiable confusion, the errors of his past
conduct; and laments, with singular propriety, the misfortune of
a monarch from whom a venal tribe of courtiers perpetually labor
to conceal the truth.\footnotemark[48]

\footnotetext[47]{Hist. August. p. 161. From some hints in the two
letters, I should expect that the eunuchs were not expelled the
palace without some degree of gentle violence, and that the young
Gordian rather approved of, than consented to, their disgrace.}

\footnotetext[48]{Duxit uxorem filiam Misithei, quem causa eloquentiæ
dignum parentela sua putavit; et præfectum statim fecit; post
quod, non puerile jam et contemptibile videbatur imperium.}

The life of Misitheus had been spent in the profession of
letters, not of arms; yet such was the versatile genius of that
great man, that, when he was appointed Prætorian Præfect, he
discharged the military duties of his place with vigor and
ability. The Persians had invaded Mesopotamia, and threatened
Antioch. By the persuasion of his father-in-law, the young
emperor quitted the luxury of Rome, opened, for the last time
recorded in history, the temple of Janus, and marched in person
into the East. On his approach, with a great army, the Persians
withdrew their garrisons from the cities which they had already
taken, and retired from the Euphrates to the Tigris. Gordian
enjoyed the pleasure of announcing to the senate the first
success of his arms, which he ascribed, with a becoming modesty
and gratitude, to the wisdom of his father and Præfect. During
the whole expedition, Misitheus watched over the safety and
discipline of the army; whilst he prevented their dangerous
murmurs by maintaining a regular plenty in the camp, and by
establishing ample magazines of vinegar, bacon, straw, barley,
and wheat in all the cities of the frontier.\footnotemark[49] But the
prosperity of Gordian expired with Misitheus, who died of a flux,
not without very strong suspicions of poison. Philip, his
successor in the præfecture, was an Arab by birth, and
consequently, in the earlier part of his life, a robber by
profession. His rise from so obscure a station to the first
dignities of the empire, seems to prove that he was a bold and
able leader. But his boldness prompted him to aspire to the
throne, and his abilities were employed to supplant, not to
serve, his indulgent master. The minds of the soldiers were
irritated by an artificial scarcity, created by his contrivance
in the camp; and the distress of the army was attributed to the
youth and incapacity of the prince. It is not in our power to
trace the successive steps of the secret conspiracy and open
sedition, which were at length fatal to Gordian. A sepulchral
monument was erected to his memory on the spot\footnotemark[50] where he was
killed, near the conflux of the Euphrates with the little river
Aboras.\footnotemark[51] The fortunate Philip, raised to the empire by the
votes of the soldiers, found a ready obedience from the senate
and the provinces.\footnotemark[52]

\footnotetext[49]{Hist. August. p. 162. Aurelius Victor. Porphyrius
in Vit Plotin. ap. Fabricium, Biblioth. Græc. l. iv. c. 36. The
philosopher Plotinus accompanied the army, prompted by the love
of knowledge, and by the hope of penetrating as far as India.}

\footnotetext[50]{About twenty miles from the little town of
Circesium, on the frontier of the two empires. * Note: Now
Kerkesia; placed in the angle formed by the juncture of the
Chaboras, or al Khabour, with the Euphrates. This situation
appeared advantageous to Diocletian, that he raised
fortifications to make it the but wark of the empire on the side
of Mesopotamia. D’Anville. Geog. Anc. ii. 196.—G. It is the
Carchemish of the Old Testament, 2 Chron. xxxv. 20. ler. xlvi.
2.—M.}

\footnotetext[51]{The inscription (which contained a very singular
pun) was erased by the order of Licinius, who claimed some degree
of relationship to Philip, (Hist. August. p. 166;) but the
tumulus, or mound of earth which formed the sepulchre, still
subsisted in the time of Julian. See Ammian Marcellin. xxiii. 5.}

\footnotetext[52]{Aurelius Victor. Eutrop. ix. 2. Orosius, vii. 20.
Ammianus Marcellinus, xxiii. 5. Zosimus, l. i. p. 19. Philip, who
was a native of Bostra, was about forty years of age. * Note: Now
Bosra. It was once the metropolis of a province named Arabia, and
the chief city of Auranitis, of which the name is preserved in
Beled Hauran, the limits of which meet the desert. D’Anville.
Geog. Anc. ii. 188. According to Victor, (in Cæsar.,) Philip was
a native of Tracbonitis another province of Arabia.—G.}

We cannot forbear transcribing the ingenious, though somewhat
fanciful description, which a celebrated writer of our own times
has traced of the military government of the Roman empire. What
in that age was called the Roman empire, was only an irregular
republic, not unlike the aristocracy\footnotemark[53] of Algiers,\footnotemark[54] where the
militia, possessed of the sovereignty, creates and deposes a
magistrate, who is styled a Dey. Perhaps, indeed, it may be laid
down as a general rule, that a military government is, in some
respects, more republican than monarchical. Nor can it be said
that the soldiers only partook of the government by their
disobedience and rebellions. The speeches made to them by the
emperors, were they not at length of the same nature as those
formerly pronounced to the people by the consuls and the
tribunes? And although the armies had no regular place or forms
of assembly; though their debates were short, their action
sudden, and their resolves seldom the result of cool reflection,
did they not dispose, with absolute sway, of the public fortune?
What was the emperor, except the minister of a violent
government, elected for the private benefit of the soldiers?

\footnotetext[53]{Can the epithet of Aristocracy be applied, with any
propriety, to the government of Algiers? Every military
government floats between two extremes of absolute monarchy and
wild democracy.}

\footnotetext[54]{The military republic of the Mamelukes in Egypt
would have afforded M. de Montesquieu (see Considerations sur la
Grandeur et la Decadence des Romains, c. 16) a juster and more
noble parallel.}

“When the army had elected Philip, who was Prætorian præfect to
the third Gordian, the latter demanded that he might remain sole
emperor; he was unable to obtain it. He requested that the power
might be equally divided between them; the army would not listen
to his speech. He consented to be degraded to the rank of Cæsar;
the favor was refused him. He desired, at least, he might be
appointed Prætorian præfect; his prayer was rejected. Finally, he
pleaded for his life. The army, in these several judgments,
exercised the supreme magistracy.” According to the historian,
whose doubtful narrative the President De Montesquieu has
adopted, Philip, who, during the whole transaction, had preserved
a sullen silence, was inclined to spare the innocent life of his
benefactor; till, recollecting that his innocence might excite a
dangerous compassion in the Roman world, he commanded, without
regard to his suppliant cries, that he should be seized,
stripped, and led away to instant death. After a moment’s pause,
the inhuman sentence was executed.\footnotemark[55]

\footnotetext[55]{The Augustan History (p. 163, 164) cannot, in this
instance, be reconciled with itself or with probability. How
could Philip condemn his predecessor, and yet consecrate his
memory? How could he order his public execution, and yet, in his
letters to the senate, exculpate himself from the guilt of his
death? Philip, though an ambitious usurper, was by no means a mad
tyrant. Some chronological difficulties have likewise been
discovered by the nice eyes of Tillemont and Muratori, in this
supposed association of Philip to the empire. * Note: Wenck
endeavors to reconcile these discrepancies. He supposes that
Gordian was led away, and died a natural death in prison. This is
directly contrary to the statement of Capitolinus and of Zosimus,
whom he adduces in support of his theory. He is more successful
in his precedents of usurpers deifying the victims of their
ambition. Sit divus, dummodo non sit vivus.—M.}

