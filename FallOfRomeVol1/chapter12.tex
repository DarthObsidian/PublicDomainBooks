\chapter{Reigns Of Tacitus, Probus, Carus And His Sons.}
\section{Part \thesection.}

\textit{Conduct Of The Army And Senate After The Death Of Aurelian.
— Reigns Of Tacitus, Probus, Carus, And His Sons.}
\vspace{\onelineskip}

Such was the unhappy condition of the Roman emperors, that,
whatever might be their conduct, their fate was commonly the
same. A life of pleasure or virtue, of severity or mildness, of
indolence or glory, alike led to an untimely grave; and almost
every reign is closed by the same disgusting repetition of
treason and murder. The death of Aurelian, however, is remarkable
by its extraordinary consequences. The legions admired, lamented,
and revenged their victorious chief. The artifice of his
perfidious secretary was discovered and punished.

The deluded conspirators attended the funeral of their injured
sovereign, with sincere or well-feigned contrition, and submitted
to the unanimous resolution of the military order, which was
signified by the following epistle: “The brave and fortunate
armies to the senate and people of Rome.—The crime of one man,
and the error of many, have deprived us of the late emperor
Aurelian. May it please you, venerable lords and fathers! to
place him in the number of the gods, and to appoint a successor
whom your judgment shall declare worthy of the Imperial purple!
None of those whose guilt or misfortune have contributed to our
loss, shall ever reign over us.”\textsuperscript{1} The Roman senators heard,
without surprise, that another emperor had been assassinated in
his camp; they secretly rejoiced in the fall of Aurelian; but the
modest and dutiful address of the legions, when it was
communicated in full assembly by the consul, diffused the most
pleasing astonishment. Such honors as fear and perhaps esteem
could extort, they liberally poured forth on the memory of their
deceased sovereign. Such acknowledgments as gratitude could
inspire, they returned to the faithful armies of the republic,
who entertained so just a sense of the legal authority of the
senate in the choice of an emperor. Yet, notwithstanding this
flattering appeal, the most prudent of the assembly declined
exposing their safety and dignity to the caprice of an armed
multitude. The strength of the legions was, indeed, a pledge of
their sincerity, since those who may command are seldom reduced
to the necessity of dissembling; but could it naturally be
expected, that a hasty repentance would correct the inveterate
habits of fourscore years? Should the soldiers relapse into their
accustomed seditions, their insolence might disgrace the majesty
of the senate, and prove fatal to the object of its choice.
Motives like these dictated a decree, by which the election of a
new emperor was referred to the suffrage of the military order.

\pagenote[1]{Vopiscus in Hist. August. p. 222. Aurelius Victor
mentions a formal deputation from the troops to the senate.}

The contention that ensued is one of the best attested, but most
improbable events in the history of mankind.\textsuperscript{2} The troops, as if
satiated with the exercise of power, again conjured the senate to
invest one of its own body with the Imperial purple. The senate
still persisted in its refusal; the army in its request. The
reciprocal offer was pressed and rejected at least three times,
and, whilst the obstinate modesty of either party was resolved to
receive a master from the hands of the other, eight months
insensibly elapsed; an amazing period of tranquil anarchy, during
which the Roman world remained without a sovereign, without a
usurper, and without a sedition.\textsuperscript{201} The generals and magistrates
appointed by Aurelian continued to execute their ordinary
functions; and it is observed, that a proconsul of Asia was the
only considerable person removed from his office in the whole
course of the interregnum.

\pagenote[2]{Vopiscus, our principal authority, wrote at Rome,
sixteen years only after the death of Aurelian; and, besides the
recent notoriety of the facts, constantly draws his materials
from the Journals of the Senate, and the original papers of the
Ulpian library. Zosimus and Zonaras appear as ignorant of this
transaction as they were in general of the Roman constitution.}

\pagenote[201]{The interregnum could not be more than seven
months; Aurelian was assassinated in the middle of March, the
year of Rome 1028. Tacitus was elected the 25th September in the
same year.—G.}

An event somewhat similar, but much less authentic, is supposed
to have happened after the death of Romulus, who, in his life and
character, bore some affinity with Aurelian. The throne was
vacant during twelve months, till the election of a Sabine
philosopher, and the public peace was guarded in the same manner,
by the union of the several orders of the state. But, in the time
of Numa and Romulus, the arms of the people were controlled by
the authority of the Patricians; and the balance of freedom was
easily preserved in a small and virtuous community.\textsuperscript{3} The decline
of the Roman state, far different from its infancy, was attended
with every circumstance that could banish from an interregnum the
prospect of obedience and harmony: an immense and tumultuous
capital, a wide extent of empire, the servile equality of
despotism, an army of four hundred thousand mercenaries, and the
experience of frequent revolutions. Yet, notwithstanding all
these temptations, the discipline and memory of Aurelian still
restrained the seditious temper of the troops, as well as the
fatal ambition of their leaders. The flower of the legions
maintained their stations on the banks of the Bosphorus, and the
Imperial standard awed the less powerful camps of Rome and of the
provinces. A generous though transient enthusiasm seemed to
animate the military order; and we may hope that a few real
patriots cultivated the returning friendship of the army and the
senate as the only expedient capable of restoring the republic to
its ancient beauty and vigor.

\pagenote[3]{Liv. i. 17 Dionys. Halicarn. l. ii. p. 115. Plutarch
in Numa, p. 60. The first of these writers relates the story like
an orator, the second like a lawyer, and the third like a
moralist, and none of them probably without some intermixture of
fable.}

On the twenty-fifth of September, near eight months after the
murder of Aurelian, the consul convoked an assembly of the
senate, and reported the doubtful and dangerous situation of the
empire. He slightly insinuated, that the precarious loyalty of
the soldiers depended on the chance of every hour, and of every
accident; but he represented, with the most convincing eloquence,
the various dangers that might attend any further delay in the
choice of an emperor. Intelligence, he said, was already
received, that the Germans had passed the Rhine, and occupied
some of the strongest and most opulent cities of Gaul. The
ambition of the Persian king kept the East in perpetual alarms;
Egypt, Africa, and Illyricum, were exposed to foreign and
domestic arms, and the levity of Syria would prefer even a female
sceptre to the sanctity of the Roman laws. The consul, then
addressing himself to Tacitus, the first of the senators,\textsuperscript{4}
required his opinion on the important subject of a proper
candidate for the vacant throne.

\pagenote[4]{Vopiscus (in Hist. August p. 227) calls him “primæ
sententia consularis;” and soon afterwards Princeps senatus. It
is natural to suppose, that the monarchs of Rome, disdaining that
humble title, resigned it to the most ancient of the senators.}

If we can prefer personal merit to accidental greatness, we shall
esteem the birth of Tacitus more truly noble than that of kings.
He claimed his descent from the philosophic historian whose
writings will instruct the last generations of mankind.\textsuperscript{5} The
senator Tacitus was then seventy-five years of age.\textsuperscript{6} The long
period of his innocent life was adorned with wealth and honors.
He had twice been invested with the consular dignity,\textsuperscript{7} and
enjoyed with elegance and sobriety his ample patrimony of between
two and three millions sterling.\textsuperscript{8} The experience of so many
princes, whom he had esteemed or endured, from the vain follies
of Elagabalus to the useful rigor of Aurelian, taught him to form
a just estimate of the duties, the dangers, and the temptations
of their sublime station. From the assiduous study of his
immortal ancestor, he derived the knowledge of the Roman
constitution, and of human nature.\textsuperscript{9} The voice of the people had
already named Tacitus as the citizen the most worthy of empire.
The ungrateful rumor reached his ears, and induced him to seek
the retirement of one of his villas in Campania. He had passed
two months in the delightful privacy of Baiæ, when he reluctantly
obeyed the summons of the consul to resume his honorable place in
the senate, and to assist the republic with his counsels on this
important occasion.

\pagenote[5]{The only objection to this genealogy is, that the
historian was named Cornelius, the emperor, Claudius. But under
the lower empire, surnames were extremely various and uncertain.}

\pagenote[6]{Zonaras, l. xii. p. 637. The Alexandrian Chronicle,
by an obvious mistake, transfers that age to Aurelian.}

\pagenote[7]{In the year 273, he was ordinary consul. But he must
have been Suffectus many years before, and most probably under
Valerian.}

\pagenote[8]{Bis millies octingenties. Vopiscus in Hist. August
p. 229. This sum, according to the old standard, was equivalent
to eight hundred and forty thousand Roman pounds of silver, each
of the value of three pounds sterling. But in the age of Tacitus,
the coin had lost much of its weight and purity.}

\pagenote[9]{After his accession, he gave orders that ten copies
of the historian should be annually transcribed and placed in the
public libraries. The Roman libraries have long since perished,
and the most valuable part of Tacitus was preserved in a single
Ms., and discovered in a monastery of Westphalia. See Bayle,
Dictionnaire, Art. Tacite, and Lipsius ad Annal. ii. 9.}

He arose to speak, when from every quarter of the house, he was
saluted with the names of Augustus and emperor. “Tacitus
Augustus, the gods preserve thee! we choose thee for our
sovereign; to thy care we intrust the republic and the world.
Accept the empire from the authority of the senate. It is due to
thy rank, to thy conduct, to thy manners.” As soon as the tumult
of acclamations subsided, Tacitus attempted to decline the
dangerous honor, and to express his wonder, that they should
elect his age and infirmities to succeed the martial vigor of
Aurelian. “Are these limbs, conscript fathers! fitted to sustain
the weight of armor, or to practise the exercises of the camp?
The variety of climates, and the hardships of a military life,
would soon oppress a feeble constitution, which subsists only by
the most tender management. My exhausted strength scarcely
enables me to discharge the duty of a senator; how insufficient
would it prove to the arduous labors of war and government! Can
you hope, that the legions will respect a weak old man, whose
days have been spent in the shade of peace and retirement? Can
you desire that I should ever find reason to regret the favorable
opinion of the senate?”\textsuperscript{10}

\pagenote[10]{Vopiscus in Hist. August. p. 227.}

The reluctance of Tacitus (and it might possibly be sincere) was
encountered by the affectionate obstinacy of the senate. Five
hundred voices repeated at once, in eloquent confusion, that the
greatest of the Roman princes, Numa, Trajan, Hadrian, and the
Antonines, had ascended the throne in a very advanced season of
life; that the mind, not the body, a sovereign, not a soldier,
was the object of their choice; and that they expected from him
no more than to guide by his wisdom the valor of the legions.
These pressing though tumultuary instances were seconded by a
more regular oration of Metius Falconius, the next on the
consular bench to Tacitus himself. He reminded the assembly of
the evils which Rome had endured from the vices of headstrong and
capricious youths, congratulated them on the election of a
virtuous and experienced senator, and, with a manly, though
perhaps a selfish, freedom, exhorted Tacitus to remember the
reasons of his elevation, and to seek a successor, not in his own
family, but in the republic. The speech of Falconius was enforced
by a general acclamation. The emperor elect submitted to the
authority of his country, and received the voluntary homage of
his equals. The judgment of the senate was confirmed by the
consent of the Roman people and of the Prætorian guards.\textsuperscript{11}

\pagenote[11]{Hist. August. p. 228. Tacitus addressed the
Prætorians by the appellation of sanctissimi milites, and the
people by that of sacratissim. Quirites.}

The administration of Tacitus was not unworthy of his life and
principles. A grateful servant of the senate, he considered that
national council as the author, and himself as the subject, of
the laws.\textsuperscript{12} He studied to heal the wounds which Imperial pride,
civil discord, and military violence, had inflicted on the
constitution, and to restore, at least, the image of the ancient
republic, as it had been preserved by the policy of Augustus, and
the virtues of Trajan and the Antonines. It may not be useless to
recapitulate some of the most important prerogatives which the
senate appeared to have regained by the election of Tacitus.\textsuperscript{13}
1. To invest one of their body, under the title of emperor, with
the general command of the armies, and the government of the
frontier provinces. 2. To determine the list, or, as it was then
styled, the College of Consuls. They were twelve in number, who,
in successive pairs, each, during the space of two months, filled
the year, and represented the dignity of that ancient office. The
authority of the senate, in the nomination of the consuls, was
exercised with such independent freedom, that no regard was paid
to an irregular request of the emperor in favor of his brother
Florianus. “The senate,” exclaimed Tacitus, with the honest
transport of a patriot, “understand the character of a prince
whom they have chosen.” 3. To appoint the proconsuls and
presidents of the provinces, and to confer on all the magistrates
their civil jurisdiction. 4. To receive appeals through the
intermediate office of the præfect of the city from all the
tribunals of the empire. 5. To give force and validity, by their
decrees, to such as they should approve of the emperor’s edicts.
6. To these several branches of authority we may add some
inspection over the finances, since, even in the stern reign of
Aurelian, it was in their power to divert a part of the revenue
from the public service.\textsuperscript{14}

\pagenote[12]{In his manumissions he never exceeded the number of
a hundred, as limited by the Caninian law, which was enacted
under Augustus, and at length repealed by Justinian. See Casaubon
ad locum Vopisci.}

\pagenote[13]{See the lives of Tacitus, Florianus, and Probus, in
the Augustan History; we may be well assured, that whatever the
soldier gave the senator had already given.}

\pagenote[14]{Vopiscus in Hist. August. p. 216. The passage is
perfectly clear, both Casaubon and Salmasius wish to correct it.}

Circular epistles were sent, without delay, to all the principal
cities of the empire, Treves, Milan, Aquileia, Thessalonica,
Corinth, Athens, Antioch, Alexandria, and Carthage, to claim
their obedience, and to inform them of the happy revolution,
which had restored the Roman senate to its ancient dignity. Two
of these epistles are still extant. We likewise possess two very
singular fragments of the private correspondence of the senators
on this occasion. They discover the most excessive joy, and the
most unbounded hopes. “Cast away your indolence,” it is thus that
one of the senators addresses his friend, “emerge from your
retirements of Baiæ and Puteoli. Give yourself to the city, to
the senate. Rome flourishes, the whole republic flourishes.
Thanks to the Roman army, to an army truly Roman; at length we
have recovered our just authority, the end of all our desires. We
hear appeals, we appoint proconsuls, we create emperors; perhaps
too we may restrain them—to the wise a word is sufficient.”\textsuperscript{15}
These lofty expectations were, however, soon disappointed; nor,
indeed, was it possible that the armies and the provinces should
long obey the luxurious and unwarlike nobles of Rome. On the
slightest touch, the unsupported fabric of their pride and power
fell to the ground. The expiring senate displayed a sudden
lustre, blazed for a moment, and was extinguished forever.

\pagenote[15]{Vopiscus in Hist. August. p. 230, 232, 233. The
senators celebrated the happy restoration with hecatombs and
public rejoicings.}

All that had yet passed at Rome was no more than a theatrical
representation, unless it was ratified by the more substantial
power of the legions. Leaving the senators to enjoy their dream
of freedom and ambition, Tacitus proceeded to the Thracian camp,
and was there, by the Prætorian præfect, presented to the
assembled troops, as the prince whom they themselves had
demanded, and whom the senate had bestowed. As soon as the
præfect was silent, the emperor addressed himself to the soldiers
with eloquence and propriety. He gratified their avarice by a
liberal distribution of treasure, under the names of pay and
donative. He engaged their esteem by a spirited declaration, that
although his age might disable him from the performance of
military exploits, his counsels should never be unworthy of a
Roman general, the successor of the brave Aurelian.\textsuperscript{16}

\pagenote[16]{Hist. August. p. 228.}

Whilst the deceased emperor was making preparations for a second
expedition into the East, he had negotiated with the Alani,\textsuperscript{161} a
Scythian people, who pitched their tents in the neighborhood of
the Lake Mæotis. Those barbarians, allured by presents and
subsidies, had promised to invade Persia with a numerous body of
light cavalry. They were faithful to their engagements; but when
they arrived on the Roman frontier, Aurelian was already dead,
the design of the Persian war was at least suspended, and the
generals, who, during the interregnum, exercised a doubtful
authority, were unprepared either to receive or to oppose them.
Provoked by such treatment, which they considered as trifling and
perfidious, the Alani had recourse to their own valor for their
payment and revenge; and as they moved with the usual swiftness
of Tartars, they had soon spread themselves over the provinces of
Pontus, Cappadocia, Cilicia, and Galatia. The legions, who from
the opposite shores of the Bosphorus could almost distinguish the
flames of the cities and villages, impatiently urged their
general to lead them against the invaders. The conduct of Tacitus
was suitable to his age and station. He convinced the barbarians
of the faith, as well as the power, of the empire. Great numbers
of the Alani, appeased by the punctual discharge of the
engagements which Aurelian had contracted with them, relinquished
their booty and captives, and quietly retreated to their own
deserts, beyond the Phasis. Against the remainder, who refused
peace, the Roman emperor waged, in person, a successful war.
Seconded by an army of brave and experienced veterans, in a few
weeks he delivered the provinces of Asia from the terror of the
Scythian invasion.\textsuperscript{17}

\pagenote[161]{On the Alani, see ch. xxvi. note 55.—M.}

\pagenote[17]{Vopiscus in Hist. August. p. 230. Zosimus, l. i. p.
57. Zonaras, l. xii. p. 637. Two passages in the life of Probus
(p. 236, 238) convince me, that these Scythian invaders of Pontus
were Alani. If we may believe Zosimus, (l. i. p. 58,) Florianus
pursued them as far as the Cimmerian Bosphorus. But he had
scarcely time for so long and difficult an expedition.}

But the glory and life of Tacitus were of short duration.
Transported, in the depth of winter, from the soft retirement of
Campania to the foot of Mount Caucasus, he sunk under the
unaccustomed hardships of a military life. The fatigues of the
body were aggravated by the cares of the mind. For a while, the
angry and selfish passions of the soldiers had been suspended by
the enthusiasm of public virtue. They soon broke out with
redoubled violence, and raged in the camp, and even in the tent
of the aged emperor. His mild and amiable character served only
to inspire contempt, and he was incessantly tormented with
factions which he could not assuage, and by demands which it was
impossible to satisfy. Whatever flattering expectations he had
conceived of reconciling the public disorders, Tacitus soon was
convinced that the licentiousness of the army disdained the
feeble restraint of laws, and his last hour was hastened by
anguish and disappointment. It may be doubtful whether the
soldiers imbrued their hands in the blood of this innocent
prince.\textsuperscript{18} It is certain that their insolence was the cause of
his death. He expired at Tyana in Cappadocia, after a reign of
only six months and about twenty days.\textsuperscript{19}

\pagenote[18]{Eutropius and Aurelius Victor only say that he
died; Victor Junior adds, that it was of a fever. Zosimus and
Zonaras affirm, that he was killed by the soldiers. Vopiscus
mentions both accounts, and seems to hesitate. Yet surely these
jarring opinions are easily reconciled.}

\pagenote[19]{According to the two Victors, he reigned exactly
two hundred days.}

The eyes of Tacitus were scarcely closed, before his brother
Florianus showed himself unworthy to reign, by the hasty
usurpation of the purple, without expecting the approbation of
the senate. The reverence for the Roman constitution, which yet
influenced the camp and the provinces, was sufficiently strong to
dispose them to censure, but not to provoke them to oppose, the
precipitate ambition of Florianus. The discontent would have
evaporated in idle murmurs, had not the general of the East, the
heroic Probus, boldly declared himself the avenger of the senate.

The contest, however, was still unequal; nor could the most able
leader, at the head of the effeminate troops of Egypt and Syria,
encounter, with any hopes of victory, the legions of Europe,
whose irresistible strength appeared to support the brother of
Tacitus. But the fortune and activity of Probus triumphed over
every obstacle. The hardy veterans of his rival, accustomed to
cold climates, sickened and consumed away in the sultry heats of
Cilicia, where the summer proved remarkably unwholesome. Their
numbers were diminished by frequent desertion; the passes of the
mountains were feebly defended; Tarsus opened its gates; and the
soldiers of Florianus, when they had permitted him to enjoy the
Imperial title about three months, delivered the empire from
civil war by the easy sacrifice of a prince whom they despised.\textsuperscript{20}

\pagenote[20]{Hist. August, p. 231. Zosimus, l. i. p. 58, 59.
Zonaras, l. xii. p. 637. Aurelius Victor says, that Probus
assumed the empire in Illyricum; an opinion which (though adopted
by a very learned man) would throw that period of history into
inextricable confusion.}

The perpetual revolutions of the throne had so perfectly erased
every notion of hereditary title, that the family of an
unfortunate emperor was incapable of exciting the jealousy of his
successors. The children of Tacitus and Florianus were permitted
to descend into a private station, and to mingle with the general
mass of the people. Their poverty indeed became an additional
safeguard to their innocence. When Tacitus was elected by the
senate, he resigned his ample patrimony to the public service;\textsuperscript{21}
an act of generosity specious in appearance, but which evidently
disclosed his intention of transmitting the empire to his
descendants. The only consolation of their fallen state was the
remembrance of transient greatness, and a distant hope, the child
of a flattering prophecy, that at the end of a thousand years, a
monarch of the race of Tacitus should arise, the protector of the
senate, the restorer of Rome, and the conqueror of the whole
earth.\textsuperscript{22}

\pagenote[21]{Hist. August. p. 229}

\pagenote[22]{He was to send judges to the Parthians, Persians,
and Sarmatians, a president to Taprobani, and a proconsul to the
Roman island, (supposed by Casaubon and Salmasius to mean
Britain.) Such a history as mine (says Vopiscus with proper
modesty) will not subsist a thousand years, to expose or justify
the prediction.}

The peasants of Illyricum, who had already given Claudius and
Aurelian to the sinking empire, had an equal right to glory in
the elevation of Probus.\textsuperscript{23} Above twenty years before, the
emperor Valerian, with his usual penetration, had discovered the
rising merit of the young soldier, on whom he conferred the rank
of tribune, long before the age prescribed by the military
regulations. The tribune soon justified his choice, by a victory
over a great body of Sarmatians, in which he saved the life of a
near relation of Valerian; and deserved to receive from the
emperor’s hand the collars, bracelets, spears, and banners, the
mural and the civic crown, and all the honorable rewards reserved
by ancient Rome for successful valor. The third, and afterwards
the tenth, legion were intrusted to the command of Probus, who,
in every step of his promotion, showed himself superior to the
station which he filled. Africa and Pontus, the Rhine, the
Danube, the Euphrates, and the Nile, by turns afforded him the
most splendid occasions of displaying his personal prowess and
his conduct in war. Aurelian was indebted for the honest courage
with which he often checked the cruelty of his master. Tacitus,
who desired by the abilities of his generals to supply his own
deficiency of military talents, named him commander-in-chief of
all the eastern provinces, with five times the usual salary, the
promise of the consulship, and the hope of a triumph. When Probus
ascended the Imperial throne, he was about forty-four years of
age;\textsuperscript{24} in the full possession of his fame, of the love of the
army, and of a mature vigor of mind and body.

\pagenote[23]{For the private life of Probus, see Vopiscus in
Hist. August p. 234—237}

\pagenote[24]{According to the Alexandrian chronicle, he was
fifty at the time of his death.}

His acknowledged merit, and the success of his arms against
Florianus, left him without an enemy or a competitor. Yet, if we
may credit his own professions, very far from being desirous of
the empire, he had accepted it with the most sincere reluctance.
“But it is no longer in my power,” says Probus, in a private
letter, “to lay down a title so full of envy and of danger. I
must continue to personate the character which the soldiers have
imposed upon me.”\textsuperscript{25} His dutiful address to the senate displayed
the sentiments, or at least the language, of a Roman patriot:
“When you elected one of your order, conscript fathers! to
succeed the emperor Aurelian, you acted in a manner suitable to
your justice and wisdom. For you are the legal sovereigns of the
world, and the power which you derive from your ancestors will
descend to your posterity. Happy would it have been, if
Florianus, instead of usurping the purple of his brother, like a
private inheritance, had expected what your majesty might
determine, either in his favor, or in that of any other person.
The prudent soldiers have punished his rashness. To me they have
offered the title of Augustus. But I submit to your clemency my
pretensions and my merits.”\textsuperscript{26} When this respectful epistle was
read by the consul, the senators were unable to disguise their
satisfaction, that Probus should condescend thus numbly to
solicit a sceptre which he already possessed. They celebrated
with the warmest gratitude his virtues, his exploits, and above
all his moderation. A decree immediately passed, without a
dissenting voice, to ratify the election of the eastern armies,
and to confer on their chief all the several branches of the
Imperial dignity: the names of Cæsar and Augustus, the title of
Father of his country, the right of making in the same day three
motions in the senate,\textsuperscript{27} the office of Pontifex Maximus, the
tribunitian power, and the proconsular command; a mode of
investiture, which, though it seemed to multiply the authority of
the emperor, expressed the constitution of the ancient republic.
The reign of Probus corresponded with this fair beginning. The
senate was permitted to direct the civil administration of the
empire. Their faithful general asserted the honor of the Roman
arms, and often laid at their feet crowns of gold and barbaric
trophies, the fruits of his numerous victories.\textsuperscript{28} Yet, whilst he
gratified their vanity, he must secretly have despised their
indolence and weakness. Though it was every moment in their power
to repeal the disgraceful edict of Gallienus, the proud
successors of the Scipios patiently acquiesced in their exclusion
from all military employments. They soon experienced, that those
who refuse the sword must renounce the sceptre.

\pagenote[25]{This letter was addressed to the Prætorian præfect,
whom (on condition of his good behavior) he promised to continue
in his great office. See Hist. August. p. 237.}

\pagenote[26]{Vopiscus in Hist. August. p. 237. The date of the
letter is assuredly faulty. Instead of Nen. Februar. we may read
Non August.}

\pagenote[27]{Hist. August. p. 238. It is odd that the senate
should treat Probus less favorably than Marcus Antoninus. That
prince had received, even before the death of Pius, Jus quintoe
relationis. See Capitolin. in Hist. August. p. 24.}

\pagenote[28]{See the dutiful letter of Probus to the senate,
after his German victories. Hist. August. p. 239.}

\section{Part \thesection.}

The strength of Aurelian had crushed on every side the enemies of
Rome. After his death they seemed to revive with an increase of
fury and of numbers. They were again vanquished by the active
vigor of Probus, who, in a short reign of about six years,\textsuperscript{29}
equalled the fame of ancient heroes, and restored peace and order
to every province of the Roman world. The dangerous frontier of
Rhætia he so firmly secured, that he left it without the
suspicion of an enemy. He broke the wandering power of the
Sarmatian tribes, and by the terror of his arms compelled those
barbarians to relinquish their spoil. The Gothic nation courted
the alliance of so warlike an emperor.\textsuperscript{30} He attacked the
Isaurians in their mountains, besieged and took several of their
strongest castles,\textsuperscript{31} and flattered himself that he had forever
suppressed a domestic foe, whose independence so deeply wounded
the majesty of the empire. The troubles excited by the usurper
Firmus in the Upper Egypt had never been perfectly appeased, and
the cities of Ptolemais and Coptos, fortified by the alliance of
the Blemmyes, still maintained an obscure rebellion. The
chastisement of those cities, and of their auxiliaries the
savages of the South, is said to have alarmed the court of
Persia,\textsuperscript{32} and the Great King sued in vain for the friendship of
Probus. Most of the exploits which distinguished his reign were
achieved by the personal valor and conduct of the emperor,
insomuch that the writer of his life expresses some amazement
how, in so short a time, a single man could be present in so many
distant wars. The remaining actions he intrusted to the care of
his lieutenants, the judicious choice of whom forms no
inconsiderable part of his glory. Carus, Diocletian, Maximian,
Constantius, Galerius, Asclepiodatus, Annibalianus, and a crowd
of other chiefs, who afterwards ascended or supported the throne,
were trained to arms in the severe school of Aurelian and Probus.\textsuperscript{33}

\pagenote[29]{The date and duration of the reign of Probus are
very correctly ascertained by Cardinal Noris in his learned work,
De Epochis Syro-Macedonum, p. 96—105. A passage of Eusebius
connects the second year of Probus with the æras of several of
the Syrian cities.}

\pagenote[30]{Vopiscus in Hist. August. p. 239.}

\pagenote[31]{Zosimus (l. i. p. 62—65) tells us a very long and
trifling story of Lycius, the Isaurian robber.}

\pagenote[32]{Zosim. l. i. p. 65. Vopiscus in Hist. August. p.
239, 240. But it seems incredible that the defeat of the savages
of Æthiopia could affect the Persian monarch.}

\pagenote[33]{Besides these well-known chiefs, several others are
named by Vopiscus, (Hist. August. p. 241,) whose actions have not
reached knowledge.}

But the most important service which Probus rendered to the
republic was the deliverance of Gaul, and the recovery of seventy
flourishing cities oppressed by the barbarians of Germany, who,
since the death of Aurelian, had ravaged that great province with
impunity.\textsuperscript{34} Among the various multitude of those fierce invaders
we may distinguish, with some degree of clearness, three great
armies, or rather nations, successively vanquished by the valor
of Probus. He drove back the Franks into their morasses; a
descriptive circumstance from whence we may infer, that the
confederacy known by the manly appellation of \textit{Free}, already
occupied the flat maritime country, intersected and almost
overflown by the stagnating waters of the Rhine, and that several
tribes of the Frisians and Batavians had acceded to their
alliance. He vanquished the Burgundians, a considerable people of
the Vandalic race.\textsuperscript{341} They had wandered in quest of booty from
the banks of the Oder to those of the Seine. They esteemed
themselves sufficiently fortunate to purchase, by the restitution
of all their booty, the permission of an undisturbed retreat.
They attempted to elude that article of the treaty. Their
punishment was immediate and terrible.\textsuperscript{35} But of all the invaders
of Gaul, the most formidable were the Lygians, a distant people,
who reigned over a wide domain on the frontiers of Poland and
Silesia.\textsuperscript{36} In the Lygian nation, the Arii held the first rank by
their numbers and fierceness. “The Arii” (it is thus that they
are described by the energy of Tacitus) “study to improve by art
and circumstances the innate terrors of their barbarism. Their
shields are black, their bodies are painted black. They choose
for the combat the darkest hour of the night. Their host
advances, covered as it were with a funeral shade;\textsuperscript{37} nor do they
often find an enemy capable of sustaining so strange and infernal
an aspect. Of all our senses, the eyes are the first vanquished
in battle.”\textsuperscript{38} Yet the arms and discipline of the Romans easily
discomfited these horrid phantoms. The Lygii were defeated in a
general engagement, and Semno, the most renowned of their chiefs,
fell alive into the hands of Probus. That prudent emperor,
unwilling to reduce a brave people to despair, granted them an
honorable capitulation, and permitted them to return in safety to
their native country. But the losses which they suffered in the
march, the battle, and the retreat, broke the power of the
nation: nor is the Lygian name ever repeated in the history
either of Germany or of the empire. The deliverance of Gaul is
reported to have cost the lives of four hundred thousand of the
invaders; a work of labor to the Romans, and of expense to the
emperor, who gave a piece of gold for the head of every
barbarian.\textsuperscript{39} But as the fame of warriors is built on the
destruction of human kind, we may naturally suspect that the
sanguinary account was multiplied by the avarice of the soldiers,
and accepted without any very severe examination by the liberal
vanity of Probus.

\pagenote[34]{See the Cæsars of Julian, and Hist. August. p. 238,
240, 241.}

\pagenote[341]{It was only under the emperors Diocletian and
Maximian, that the Burgundians, in concert with the Alemanni,
invaded the interior of Gaul; under the reign of Probus, they did
no more than pass the river which separated them from the Roman
Empire: they were repelled. Gatterer presumes that this river was
the Danube; a passage in Zosimus appears to me rather to indicate
the Rhine. Zos. l. i. p. 37, edit H. Etienne, 1581.—G. On the
origin of the Burgundians may be consulted Malte Brun, Geogr vi.
p. 396, (edit. 1831,) who observes that all the remains of the
Burgundian language indicate that they spoke a Gothic
dialect.—M.}

\pagenote[35]{Zosimus, l. i. p. 62. Hist. August. p. 240. But the
latter supposes the punishment inflicted with the consent of
their kings: if so, it was partial, like the offence.}

\pagenote[36]{See Cluver. Germania Antiqua, l. iii. Ptolemy
places in their country the city of Calisia, probably Calish in
Silesia. * Note: Luden (vol ii. 501) supposes that these have
been erroneously identified with the Lygii of Tacitus. Perhaps
one fertile source of mistakes has been, that the Romans have
turned appellations into national names. Malte Brun observes of
the Lygii, “that their name appears Sclavonian, and signifies
‘inhabitants of plains;’ they are probably the Lieches of the
middle ages, and the ancestors of the Poles. We find among the
Arii the worship of the two twin gods known in the Sclavian
mythology.” Malte Brun, vol. i. p. 278, (edit. 1831.)—M. But
compare Schafarik, Slawische Alterthumer, 1, p. 406. They were of
German or Keltish descent, occupying the Wendish (or Slavian)
district, Luhy.—M. 1845.}

\pagenote[37]{Feralis umbra, is the expression of Tacitus: it is
surely a very bold one.}

\pagenote[38]{Tacit. Germania, (c. 43.)}

\pagenote[39]{Vopiscus in Hist. August. p. 238}

Since the expedition of Maximin, the Roman generals had confined
their ambition to a defensive war against the nations of Germany,
who perpetually pressed on the frontiers of the empire. The more
daring Probus pursued his Gallic victories, passed the Rhine, and
displayed his invincible eagles on the banks of the Elbe and the
Neckar. He was fully convinced that nothing could reconcile the
minds of the barbarians to peace, unless they experienced, in
their own country, the calamities of war. Germany, exhausted by
the ill success of the last emigration, was astonished by his
presence. Nine of the most considerable princes repaired to his
camp, and fell prostrate at his feet. Such a treaty was humbly
received by the Germans, as it pleased the conqueror to dictate.
He exacted a strict restitution of the effects and captives which
they had carried away from the provinces; and obliged their own
magistrates to punish the more obstinate robbers who presumed to
detain any part of the spoil. A considerable tribute of corn,
cattle, and horses, the only wealth of barbarians, was reserved
for the use of the garrisons which Probus established on the
limits of their territory. He even entertained some thoughts of
compelling the Germans to relinquish the exercise of arms, and to
trust their differences to the justice, their safety to the
power, of Rome. To accomplish these salutary ends, the constant
residence of an Imperial governor, supported by a numerous army,
was indispensably requisite. Probus therefore judged it more
expedient to defer the execution of so great a design; which was
indeed rather of specious than solid utility.\textsuperscript{40} Had Germany been
reduced into the state of a province, the Romans, with immense
labor and expense, would have acquired only a more extensive
boundary to defend against the fiercer and more active barbarians
of Scythia.

\pagenote[40]{Hist. August. 238, 239. Vopiscus quotes a letter
from the emperor to the senate, in which he mentions his design
of reducing Germany into a province.}

Instead of reducing the warlike natives of Germany to the
condition of subjects, Probus contented himself with the humble
expedient of raising a bulwark against their inroads. The country
which now forms the circle of Swabia had been left desert in the
age of Augustus by the emigration of its ancient inhabitants.\textsuperscript{41}
The fertility of the soil soon attracted a new colony from the
adjacent provinces of Gaul. Crowds of adventurers, of a roving
temper and of desperate fortunes, occupied the doubtful
possession, and acknowledged, by the payment of tithes, the
majesty of the empire.\textsuperscript{42} To protect these new subjects, a line
of frontier garrisons was gradually extended from the Rhine to
the Danube. About the reign of Hadrian, when that mode of defence
began to be practised, these garrisons were connected and covered
by a strong intrenchment of trees and palisades. In the place of
so rude a bulwark, the emperor Probus constructed a stone wall of
a considerable height, and strengthened it by towers at
convenient distances. From the neighborhood of Neustadt and
Ratisbon on the Danube, it stretched across hills, valleys,
rivers, and morasses, as far as Wimpfen on the Neckar, and at
length terminated on the banks of the Rhine, after a winding
course of near two hundred miles.\textsuperscript{43} This important barrier,
uniting the two mighty streams that protected the provinces of
Europe, seemed to fill up the vacant space through which the
barbarians, and particularly the Alemanni, could penetrate with
the greatest facility into the heart of the empire. But the
experience of the world, from China to Britain, has exposed the
vain attempt of fortifying any extensive tract of country.\textsuperscript{44} An
active enemy, who can select and vary his points of attack, must,
in the end, discover some feeble spot, or some unguarded moment.
The strength, as well as the attention, of the defenders is
divided; and such are the blind effects of terror on the firmest
troops, that a line broken in a single place is almost instantly
deserted. The fate of the wall which Probus erected may confirm
the general observation. Within a few years after his death, it
was overthrown by the Alemanni. Its scattered ruins, universally
ascribed to the power of the Dæmon, now serve only to excite the
wonder of the Swabian peasant.

\pagenote[41]{Strabo, l. vii. According to Valleius Paterculus,
(ii. 108,) Maroboduus led his Marcomanni into Bohemia; Cluverius
(German. Antiq. iii. 8) proves that it was from Swabia.}

\pagenote[42]{These settlers, from the payment of tithes, were
denominated Decunates. Tacit. Germania, c. 29}

\pagenote[43]{See notes de l’Abbé de la Bleterie a la Germanie de
Tacite, p. 183. His account of the wall is chiefly borrowed (as
he says himself) from the Alsatia Illustrata of Schoepflin.}

\pagenote[44]{See Recherches sur les Chinois et les Egyptiens,
tom. ii. p. 81—102. The anonymous author is well acquainted with
the globe in general, and with Germany in particular: with regard
to the latter, he quotes a work of M. Hanselman; but he seems to
confound the wall of Probus, designed against the Alemanni, with
the fortification of the Mattiaci, constructed in the
neighborhood of Frankfort against the Catti. * Note: De Pauw is
well known to have been the author of this work, as of the
Recherches sur les Americains before quoted. The judgment of M.
Remusat on this writer is in a very different, I fear a juster
tone. Quand au lieu de rechercher, d’examiner, d’etudier, on se
borne, comme cet ecrivain, a juger a prononcer, a decider, sans
connoitre ni l’histoire. ni les langues, sans recourir aux
sources, sans meme se douter de leur existence, on peut en
imposer pendant quelque temps a des lecteurs prevenus ou peu
instruits; mais le mepris qui ne manque guere de succeder a cet
engouement fait bientot justice de ces assertions hazardees, et
elles retombent dans l’oubli d’autant plus promptement, qu’elles
ont ete posees avec plus de confiance. Sur les l angues Tartares,
p. 231.—M.}

Among the useful conditions of peace imposed by Probus on the
vanquished nations of Germany, was the obligation of supplying
the Roman army with sixteen thousand recruits, the bravest and
most robust of their youth. The emperor dispersed them through
all the provinces, and distributed this dangerous reënforcement,
in small bands of fifty or sixty each, among the national troops;
judiciously observing, that the aid which the republic derived
from the barbarians should be felt but not seen.\textsuperscript{45} Their aid was
now become necessary. The feeble elegance of Italy and the
internal provinces could no longer support the weight of arms.
The hardy frontiers of the Rhine and Danube still produced minds
and bodies equal to the labors of the camp; but a perpetual
series of wars had gradually diminished their numbers. The
infrequency of marriage, and the ruin of agriculture, affected
the principles of population, and not only destroyed the strength
of the present, but intercepted the hope of future, generations.
The wisdom of Probus embraced a great and beneficial plan of
replenishing the exhausted frontiers, by new colonies of captive
or fugitive barbarians, on whom he bestowed lands, cattle,
instruments of husbandry, and every encouragement that might
engage them to educate a race of soldiers for the service of the
republic. Into Britain, and most probably into Cambridgeshire,\textsuperscript{46}
he transported a considerable body of Vandals. The impossibility
of an escape reconciled them to their situation, and in the
subsequent troubles of that island, they approved themselves the
most faithful servants of the state.\textsuperscript{47} Great numbers of Franks
and Gepidæ were settled on the banks of the Danube and the Rhine.
A hundred thousand Bastarnæ, expelled from their own country,
cheerfully accepted an establishment in Thrace, and soon imbibed
the manners and sentiments of Roman subjects. \textsuperscript{48} But the
expectations of Probus were too often disappointed. The
impatience and idleness of the barbarians could ill brook the
slow labors of agriculture. Their unconquerable love of freedom,
rising against despotism, provoked them into hasty rebellions,
alike fatal to themselves and to the provinces;\textsuperscript{49} nor could
these artificial supplies, however repeated by succeeding
emperors, restore the important limit of Gaul and Illyricum to
its ancient and native vigor.

\pagenote[45]{He distributed about fifty or sixty barbarians to a
Numerus, as it was then called, a corps with whose established
number we are not exactly acquainted.}

\pagenote[46]{Camden’s Britannia, Introduction, p. 136; but he
speaks from a very doubtful conjecture.}

\pagenote[47]{Zosimus, l. i. p. 62. According to Vopiscus,
another body of Vandals was less faithful.}

\pagenote[48]{Footnote 48: Hist. August. p. 240. They were
probably expelled by the Goths. Zosim. l. i. p. 66.}

\pagenote[49]{Hist. August. p. 240.}

Of all the barbarians who abandoned their new settlements, and
disturbed the public tranquillity, a very small number returned
to their own country. For a short season they might wander in
arms through the empire; but in the end they were surely
destroyed by the power of a warlike emperor. The successful
rashness of a party of Franks was attended, however, with such
memorable consequences, that it ought not to be passed unnoticed.
They had been established by Probus, on the sea-coast of Pontus,
with a view of strengthening the frontier against the inroads of
the Alani. A fleet stationed in one of the harbors of the Euxine
fell into the hands of the Franks; and they resolved, through
unknown seas, to explore their way from the mouth of the Phasis
to that of the Rhine. They easily escaped through the Bosphorus
and the Hellespont, and cruising along the Mediterranean,
indulged their appetite for revenge and plunder by frequent
descents on the unsuspecting shores of Asia, Greece, and Africa.
The opulent city of Syracuse, in whose port the navies of Athens
and Carthage had formerly been sunk, was sacked by a handful of
barbarians, who massacred the greatest part of the trembling
inhabitants. From the island of Sicily the Franks proceeded to
the columns of Hercules, trusted themselves to the ocean, coasted
round Spain and Gaul, and steering their triumphant course
through the British Channel, at length finished their surprising
voyage, by landing in safety on the Batavian or Frisian shores.\textsuperscript{50}
The example of their success, instructing their countrymen to
conceive the advantages and to despise the dangers of the sea,
pointed out to their enterprising spirit a new road to wealth and
glory.

\pagenote[50]{Panegyr. Vet. v. 18. Zosimus, l. i. p. 66.}

Notwithstanding the vigilance and activity of Probus, it was
almost impossible that he could at once contain in obedience
every part of his wide-extended dominions. The barbarians, who
broke their chains, had seized the favorable opportunity of a
domestic war. When the emperor marched to the relief of Gaul, he
devolved the command of the East on Saturninus. That general, a
man of merit and experience, was driven into rebellion by the
absence of his sovereign, the levity of the Alexandrian people,
the pressing instances of his friends, and his own fears; but
from the moment of his elevation, he never entertained a hope of
empire, or even of life. “Alas!” he said, “the republic has lost
a useful servant, and the rashness of an hour has destroyed the
services of many years. You know not,” continued he, “the misery
of sovereign power; a sword is perpetually suspended over our
head. We dread our very guards, we distrust our companions. The
choice of action or of repose is no longer in our disposition,
nor is there any age, or character, or conduct, that can protect
us from the censure of envy. In thus exalting me to the throne,
you have doomed me to a life of cares, and to an untimely fate.
The only consolation which remains is the assurance that I shall
not fall alone.”\textsuperscript{51} But as the former part of his prediction was
verified by the victory, so the latter was disappointed by the
clemency, of Probus. That amiable prince attempted even to save
the unhappy Saturninus from the fury of the soldiers. He had more
than once solicited the usurper himself to place some confidence
in the mercy of a sovereign who so highly esteemed his character,
that he had punished, as a malicious informer, the first who
related the improbable news of his disaffection.\textsuperscript{52} Saturninus
might, perhaps, have embraced the generous offer, had he not been
restrained by the obstinate distrust of his adherents. Their
guilt was deeper, and their hopes more sanguine, than those of
their experienced leader.

\pagenote[51]{Vopiscus in Hist. August. p. 245, 246. The
unfortunate orator had studied rhetoric at Carthage; and was
therefore more probably a Moor (Zosim. l. i. p. 60) than a Gaul,
as Vopiscus calls him.}

\pagenote[52]{Zonaras, l. xii. p. 638.}

The revolt of Saturninus was scarcely extinguished in the East,
before new troubles were excited in the West, by the rebellion of
Bonosus and Proculus, in Gaul. The most distinguished merit of
those two officers was their respective prowess, of the one in
the combats of Bacchus, of the other in those of Venus,\textsuperscript{53} yet
neither of them was destitute of courage and capacity, and both
sustained, with honor, the august character which the fear of
punishment had engaged them to assume, till they sunk at length
beneath the superior genius of Probus. He used the victory with
his accustomed moderation, and spared the fortune, as well as the
lives of their innocent families.\textsuperscript{54}

\pagenote[53]{A very surprising instance is recorded of the
prowess of Proculus. He had taken one hundred Sarmatian virgins.
The rest of the story he must relate in his own language: “Ex his
una necte decem inivi; omnes tamen, quod in me erat, mulieres
intra dies quindecim reddidi.” Vopiscus in Hist. August. p. 246.}

\pagenote[54]{Proculus, who was a native of Albengue, on the
Genoese coast armed two thousand of his own slaves. His riches
were great, but they were acquired by robbery. It was afterwards
a saying of his family, sibi non placere esse vel principes vel
latrones. Vopiscus in Hist. August. p. 247.}

The arms of Probus had now suppressed all the foreign and
domestic enemies of the state. His mild but steady administration
confirmed the re-ëstablishment of the public tranquillity; nor
was there left in the provinces a hostile barbarian, a tyrant, or
even a robber, to revive the memory of past disorders. It was
time that the emperor should revisit Rome, and celebrate his own
glory and the general happiness. The triumph due to the valor of
Probus was conducted with a magnificence suitable to his fortune,
and the people, who had so lately admired the trophies of
Aurelian, gazed with equal pleasure on those of his heroic
successor.\textsuperscript{55} We cannot, on this occasion, forget the desperate
courage of about fourscore gladiators, reserved, with near six
hundred others, for the inhuman sports of the amphitheatre.
Disdaining to shed their blood for the amusement of the populace,
they killed their keepers, broke from the place of their
confinement, and filled the streets of Rome with blood and
confusion. After an obstinate resistance, they were overpowered
and cut in pieces by the regular forces; but they obtained at
least an honorable death, and the satisfaction of a just revenge.\textsuperscript{56}

\pagenote[55]{Hist. August. p. 240.}

\pagenote[56]{Zosim. l. i. p. 66.}

The military discipline which reigned in the camps of Probus was
less cruel than that of Aurelian, but it was equally rigid and
exact. The latter had punished the irregularities of the soldiers
with unrelenting severity, the former prevented them by employing
the legions in constant and useful labors. When Probus commanded
in Egypt, he executed many considerable works for the splendor
and benefit of that rich country. The navigation of the Nile, so
important to Rome itself, was improved; and temples, buildings,
porticos, and palaces, were constructed by the hands of the
soldiers, who acted by turns as architects, as engineers, and as
husbandmen.\textsuperscript{57} It was reported of Hannibal, that, in order to
preserve his troops from the dangerous temptations of idleness,
he had obliged them to form large plantations of olive-trees
along the coast of Africa.\textsuperscript{58} From a similar principle, Probus
exercised his legions in covering with rich vineyards the hills
of Gaul and Pannonia, and two considerable spots are described,
which were entirely dug and planted by military labor.\textsuperscript{59} One of
these, known under the name of Mount Almo, was situated near
Sirmium, the country where Probus was born, for which he ever
retained a partial affection, and whose gratitude he endeavored
to secure, by converting into tillage a large and unhealthy tract
of marshy ground. An army thus employed constituted perhaps the
most useful, as well as the bravest, portion of Roman subjects.

\pagenote[57]{Hist. August. p. 236.}

\pagenote[58]{Aurel. Victor. in Prob. But the policy of Hannibal,
unnoticed by any more ancient writer, is irreconcilable with the
history of his life. He left Africa when he was nine years old,
returned to it when he was forty-five, and immediately lost his
army in the decisive battle of Zama. Livilus, xxx. 37.}

\pagenote[59]{Hist. August. p. 240. Eutrop. ix. 17. Aurel.
Victor. in Prob. Victor Junior. He revoked the prohibition of
Domitian, and granted a general permission of planting vines to
the Gauls, the Britons, and the Pannonians.}

But in the prosecution of a favorite scheme, the best of men,
satisfied with the rectitude of their intentions, are subject to
forget the bounds of moderation; nor did Probus himself
sufficiently consult the patience and disposition of his fierce
legionaries.\textsuperscript{60} The dangers of the military profession seem only
to be compensated by a life of pleasure and idleness; but if the
duties of the soldier are incessantly aggravated by the labors of
the peasant, he will at last sink under the intolerable burden,
or shake it off with indignation. The imprudence of Probus is
said to have inflamed the discontent of his troops. More
attentive to the interests of mankind than to those of the army,
he expressed the vain hope, that, by the establishment of
universal peace, he should soon abolish the necessity of a
standing and mercenary force.\textsuperscript{61} The unguarded expression proved
fatal to him. In one of the hottest days of summer, as he
severely urged the unwholesome labor of draining the marshes of
Sirmium, the soldiers, impatient of fatigue, on a sudden threw
down their tools, grasped their arms, and broke out into a
furious mutiny. The emperor, conscious of his danger, took refuge
in a lofty tower, constructed for the purpose of surveying the
progress of the work.\textsuperscript{62} The tower was instantly forced, and a
thousand swords were plunged at once into the bosom of the
unfortunate Probus. The rage of the troops subsided as soon as it
had been gratified. They then lamented their fatal rashness,
forgot the severity of the emperor whom they had massacred, and
hastened to perpetuate, by an honorable monument, the memory of
his virtues and victories.\textsuperscript{63}

\pagenote[60]{Julian bestows a severe, and indeed excessive,
censure on the rigor of Probus, who, as he thinks, almost
deserved his fate.}

\pagenote[61]{Vopiscus in Hist. August. p. 241. He lavishes on
this idle hope a large stock of very foolish eloquence.}

\pagenote[62]{Turris ferrata. It seems to have been a movable
tower, and cased with iron.}

\pagenote[63]{Probus, et vere probus situs est; Victor omnium
gentium Barbararum; victor etiam tyrannorum.}

When the legions had indulged their grief and repentance for the
death of Probus, their unanimous consent declared Carus, his
Prætorian præfect, the most deserving of the Imperial throne.
Every circumstance that relates to this prince appears of a mixed
and doubtful nature. He gloried in the title of Roman Citizen;
and affected to compare the purity of \textit{his} blood with the
foreign and even barbarous origin of the preceding emperors; yet
the most inquisitive of his contemporaries, very far from
admitting his claim, have variously deduced his own birth, or
that of his parents, from Illyricum, from Gaul, or from Africa.\textsuperscript{64}
Though a soldier, he had received a learned education; though
a senator, he was invested with the first dignity of the army;
and in an age when the civil and military professions began to be
irrecoverably separated from each other, they were united in the
person of Carus. Notwithstanding the severe justice which he
exercised against the assassins of Probus, to whose favor and
esteem he was highly indebted, he could not escape the suspicion
of being accessory to a deed from whence he derived the principal
advantage. He enjoyed, at least before his elevation, an
acknowledged character of virtue and abilities;\textsuperscript{65} but his
austere temper insensibly degenerated into moroseness and
cruelty; and the imperfect writers of his life almost hesitate
whether they shall not rank him in the number of Roman tyrants.\textsuperscript{66}
When Carus assumed the purple, he was about sixty years of
age, and his two sons, Carinus and Numerian had already attained
the season of manhood.\textsuperscript{67}

\pagenote[64]{Yet all this may be conciliated. He was born at
Narbonne in Illyricum, confounded by Eutropius with the more
famous city of that name in Gaul. His father might be an African,
and his mother a noble Roman. Carus himself was educated in the
capital. See Scaliger Animadversion. ad Euseb. Chron. p. 241.}

\pagenote[65]{Probus had requested of the senate an equestrian
statue and a marble palace, at the public expense, as a just
recompense of the singular merit of Carus. Vopiscus in Hist.
August. p. 249.}

\pagenote[66]{Vopiscus in Hist. August. p. 242, 249. Julian
excludes the emperor Carus and both his sons from the banquet of
the Cæsars.}

\pagenote[67]{John Malala, tom. i. p. 401. But the authority of
that ignorant Greek is very slight. He ridiculously derives from
Carus the city of Carrhæ, and the province of Caria, the latter
of which is mentioned by Homer.}

The authority of the senate expired with Probus; nor was the
repentance of the soldiers displayed by the same dutiful regard
for the civil power, which they had testified after the
unfortunate death of Aurelian. The election of Carus was decided
without expecting the approbation of the senate, and the new
emperor contented himself with announcing, in a cold and stately
epistle, that he had ascended the vacant throne.\textsuperscript{68} A behavior so
very opposite to that of his amiable predecessor afforded no
favorable presage of the new reign: and the Romans, deprived of
power and freedom, asserted their privilege of licentious
murmurs.\textsuperscript{69} The voice of congratulation and flattery was not,
however, silent; and we may still peruse, with pleasure and
contempt, an eclogue, which was composed on the accession of the
emperor Carus. Two shepherds, avoiding the noontide heat, retire
into the cave of Faunus. On a spreading beech they discover some
recent characters. The rural deity had described, in prophetic
verses, the felicity promised to the empire under the reign of so
great a prince. Faunus hails the approach of that hero, who,
receiving on his shoulders the sinking weight of the Roman world,
shall extinguish war and faction, and once again restore the
innocence and security of the golden age.\textsuperscript{70}

\pagenote[68]{Hist. August. p. 249. Carus congratulated the
senate, that one of their own order was made emperor.}

\pagenote[69]{Hist. August. p. 242.}

\pagenote[70]{See the first eclogue of Calphurnius. The design of
it is preferes by Fontenelle to that of Virgil’s Pollio. See tom.
iii. p. 148.}

It is more than probable, that these elegant trifles never
reached the ears of a veteran general, who, with the consent of
the legions, was preparing to execute the long-suspended design
of the Persian war. Before his departure for this distant
expedition, Carus conferred on his two sons, Carinus and
Numerian, the title of Cæsar, and investing the former with
almost an equal share of the Imperial power, directed the young
prince first to suppress some troubles which had arisen in Gaul,
and afterwards to fix the seat of his residence at Rome, and to
assume the government of the Western provinces.\textsuperscript{71} The safety of
Illyricum was confirmed by a memorable defeat of the Sarmatians;
sixteen thousand of those barbarians remained on the field of
battle, and the number of captives amounted to twenty thousand.
The old emperor, animated with the fame and prospect of victory,
pursued his march, in the midst of winter, through the countries
of Thrace and Asia Minor, and at length, with his younger son,
Numerian, arrived on the confines of the Persian monarchy. There,
encamping on the summit of a lofty mountain, he pointed out to
his troops the opulence and luxury of the enemy whom they were
about to invade.

\pagenote[71]{Hist. August. p. 353. Eutropius, ix. 18. Pagi.
Annal.}

The successor of Artaxerxes,\textsuperscript{711} Varanes, or Bahram, though he
had subdued the Segestans, one of the most warlike nations of
Upper Asia,\textsuperscript{72} was alarmed at the approach of the Romans, and
endeavored to retard their progress by a negotiation of peace.\textsuperscript{721}

His ambassadors entered the camp about sunset, at the time when
the troops were satisfying their hunger with a frugal repast. The
Persians expressed their desire of being introduced to the
presence of the Roman emperor. They were at length conducted to a
soldier, who was seated on the grass. A piece of stale bacon and
a few hard peas composed his supper. A coarse woollen garment of
purple was the only circumstance that announced his dignity. The
conference was conducted with the same disregard of courtly
elegance. Carus, taking off a cap which he wore to conceal his
baldness, assured the ambassadors, that, unless their master
acknowledged the superiority of Rome, he would speedily render
Persia as naked of trees as his own head was destitute of hair.\textsuperscript{73}
Notwithstanding some traces of art and preparation, we may
discover in this scene the manners of Carus, and the severe
simplicity which the martial princes, who succeeded Gallienus,
had already restored in the Roman camps. The ministers of the
Great King trembled and retired.

\pagenote[711]{Three monarchs had intervened, Sapor, (Shahpour,)
Hormisdas, (Hormooz,) Varanes; Baharam the First.—M.}

\pagenote[72]{Agathias, l. iv. p. 135. We find one of his sayings
in the Bibliotheque Orientale of M. d’Herbelot. “The definition
of humanity includes all other virtues.”}

\pagenote[721]{The manner in which his life was saved by the
Chief Pontiff from a conspiracy of his nobles, is as remarkable
as his saying. “By the advice (of the Pontiff) all the nobles
absented themselves from court. The king wandered through his
palace alone. He saw no one; all was silence around. He became
alarmed and distressed. At last the Chief Pontiff appeared, and
bowed his head in apparent misery, but spoke not a word. The king
entreated him to declare what had happened. The virtuous man
boldly related all that had passed, and conjured Bahram, in the
name of his glorious ancestors, to change his conduct and save
himself from destruction. The king was much moved, professed
himself most penitent, and said he was resolved his future life
should prove his sincerity. The overjoyed High Priest, delighted
at this success, made a signal, at which all the nobles and
attendants were in an instant, as if by magic, in their usual
places. The monarch now perceived that only one opinion prevailed
on his past conduct. He repeated therefore to his nobles all he
had said to the Chief Pontiff, and his future reign was unstained
by cruelty or oppression.” Malcolm’s Persia,—M.}

\pagenote[73]{Synesius tells this story of Carinus; and it is
much more natural to understand it of Carus, than (as Petavius
and Tillemont choose to do) of Probus.}

The threats of Carus were not without effect. He ravaged
Mesopotamia, cut in pieces whatever opposed his passage, made
himself master of the great cities of Seleucia and Ctesiphon,
(which seemed to have surrendered without resistance,) and
carried his victorious arms beyond the Tigris.\textsuperscript{74} He had seized
the favorable moment for an invasion. The Persian councils were
distracted by domestic factions, and the greater part of their
forces were detained on the frontiers of India. Rome and the East
received with transport the news of such important advantages.
Flattery and hope painted, in the most lively colors, the fall of
Persia, the conquest of Arabia, the submission of Egypt, and a
lasting deliverance from the inroads of the Scythian nations.\textsuperscript{75}
But the reign of Carus was destined to expose the vanity of
predictions. They were scarcely uttered before they were
contradicted by his death; an event attended with such ambiguous
circumstances, that it may be related in a letter from his own
secretary to the præfect of the city. “Carus,” says he, “our
dearest emperor, was confined by sickness to his bed, when a
furious tempest arose in the camp. The darkness which overspread
the sky was so thick, that we could no longer distinguish each
other; and the incessant flashes of lightning took from us the
knowledge of all that passed in the general confusion.
Immediately after the most violent clap of thunder, we heard a
sudden cry that the emperor was dead; and it soon appeared, that
his chamberlains, in a rage of grief, had set fire to the royal
pavilion; a circumstance which gave rise to the report that Carus
was killed by lightning. But, as far as we have been able to
investigate the truth, his death was the natural effect of his
disorder.”\textsuperscript{76}

\pagenote[74]{Vopiscus in Hist. August. p. 250. Eutropius, ix.
18. The two Victors.}

\pagenote[75]{To the Persian victory of Carus I refer the
dialogue of the Philopatris, which has so long been an object of
dispute among the learned. But to explain and justify my opinion,
would require a dissertation. Note: Niebuhr, in the new edition
of the Byzantine Historians, (vol. x.) has boldly assigned the
Philopatris to the tenth century, and to the reign of Nicephorus
Phocas. An opinion so decisively pronounced by Niebuhr and
favorably received by Hase, the learned editor of Leo Diaconus,
commands respectful consideration. But the whole tone of the work
appears to me altogether inconsistent with any period in which
philosophy did not stand, as it were, on some ground of equality
with Christianity. The doctrine of the Trinity is sarcastically
introduced rather as the strange doctrine of a new religion, than
the established tenet of a faith universally prevalent. The
argument, adopted from Solanus, concerning the formula of the
procession of the Holy Ghost, is utterly worthless, as it is a
mere quotation in the words of the Gospel of St. John, xv. 26.
The only argument of any value is the historic one, from the
allusion to the recent violation of many virgins in the Island of
Crete. But neither is the language of Niebuhr quite accurate, nor
his reference to the Acroases of Theodosius satisfactory. When,
then, could this occurrence take place? Why not in the
devastation of the island by the Gothic pirates, during the reign
of Claudius. Hist. Aug. in Claud. p. 814. edit. Var. Lugd. Bat
1661.—M.}

\pagenote[76]{Hist. August. p. 250. Yet Eutropius, Festus, Rufus,
the two Victors, Jerome, Sidonius Apollinaris, Syncellus, and
Zonaras, all ascribe the death of Carus to lightning.}

\section{Part \thesection.}

The vacancy of the throne was not productive of any disturbance.
The ambition of the aspiring generals was checked by their
natural fears, and young Numerian, with his absent brother
Carinus, were unanimously acknowledged as Roman emperors.

The public expected that the successor of Carus would pursue his
father’s footsteps, and, without allowing the Persians to recover
from their consternation, would advance sword in hand to the
palaces of Susa and Ecbatana.\textsuperscript{77} But the legions, however strong
in numbers and discipline, were dismayed by the most abject
superstition. Notwithstanding all the arts that were practised to
disguise the manner of the late emperor’s death, it was found
impossible to remove the opinion of the multitude, and the power
of opinion is irresistible. Places or persons struck with
lightning were considered by the ancients with pious horror, as
singularly devoted to the wrath of Heaven.\textsuperscript{78} An oracle was
remembered, which marked the River Tigris as the fatal boundary
of the Roman arms. The troops, terrified with the fate of Carus
and with their own danger, called aloud on young Numerian to obey
the will of the gods, and to lead them away from this
inauspicious scene of war. The feeble emperor was unable to
subdue their obstinate prejudice, and the Persians wondered at
the unexpected retreat of a victorious enemy.\textsuperscript{79}

\pagenote[77]{See Nemesian. Cynegeticon, v. 71, \&c.}

\pagenote[78]{See Festus and his commentators on the word
Scribonianum. Places struck by lightning were surrounded with a
wall; things were buried with mysterious ceremony.}

\pagenote[79]{Vopiscus in Hist. August. p. 250. Aurelius Victor
seems to believe the prediction, and to approve the retreat.}

The intelligence of the mysterious fate of the late emperor was
soon carried from the frontiers of Persia to Rome; and the
senate, as well as the provinces, congratulated the accession of
the sons of Carus. These fortunate youths were strangers,
however, to that conscious superiority, either of birth or of
merit, which can alone render the possession of a throne easy,
and, as it were, natural. Born and educated in a private station,
the election of their father raised them at once to the rank of
princes; and his death, which happened about sixteen months
afterwards, left them the unexpected legacy of a vast empire. To
sustain with temper this rapid elevation, an uncommon share of
virtue and prudence was requisite; and Carinus, the elder of the
brothers, was more than commonly deficient in those qualities. In
the Gallic war he discovered some degree of personal courage;\textsuperscript{80}
but from the moment of his arrival at Rome, he abandoned himself
to the luxury of the capital, and to the abuse of his fortune. He
was soft, yet cruel; devoted to pleasure, but destitute of taste;
and though exquisitely susceptible of vanity, indifferent to the
public esteem. In the course of a few months, he successively
married and divorced nine wives, most of whom he left pregnant;
and notwithstanding this legal inconstancy, found time to indulge
such a variety of irregular appetites, as brought dishonor on
himself and on the noblest houses of Rome. He beheld with
inveterate hatred all those who might remember his former
obscurity, or censure his present conduct. He banished, or put to
death, the friends and counsellors whom his father had placed
about him, to guide his inexperienced youth; and he persecuted
with the meanest revenge his school-fellows and companions who
had not sufficiently respected the latent majesty of the emperor.

With the senators, Carinus affected a lofty and regal demeanor,
frequently declaring, that he designed to distribute their
estates among the populace of Rome. From the dregs of that
populace he selected his favorites, and even his ministers. The
palace, and even the Imperial table, were filled with singers,
dancers, prostitutes, and all the various retinue of vice and
folly. One of his doorkeepers\textsuperscript{81} he intrusted with the government
of the city. In the room of the Prætorian præfect, whom he put to
death, Carinus substituted one of the ministers of his looser
pleasures. Another, who possessed the same, or even a more
infamous, title to favor, was invested with the consulship. A
confidential secretary, who had acquired uncommon skill in the
art of forgery, delivered the indolent emperor, with his own
consent from the irksome duty of signing his name.

\pagenote[80]{Nemesian. Cynegeticon, v 69. He was a contemporary,
but a poet.}

\pagenote[81]{Cancellarius. This word, so humble in its origin,
has, by a singular fortune, risen into the title of the first
great office of state in the monarchies of Europe. See Casaubon
and Salmasius, ad Hist. August, p. 253.}

When the emperor Carus undertook the Persian war, he was induced,
by motives of affection as well as policy, to secure the fortunes
of his family, by leaving in the hands of his eldest son the
armies and provinces of the West. The intelligence which he soon
received of the conduct of Carinus filled him with shame and
regret; nor had he concealed his resolution of satisfying the
republic by a severe act of justice, and of adopting, in the
place of an unworthy son, the brave and virtuous Constantius, who
at that time was governor of Dalmatia. But the elevation of
Constantius was for a while deferred; and as soon as the father’s
death had released Carinus from the control of fear or decency,
he displayed to the Romans the extravagancies of Elagabalus,
aggravated by the cruelty of Domitian.\textsuperscript{82}

\pagenote[82]{Vopiscus in Hist. August. p. 253, 254. Eutropius,
x. 19. Vic to Junior. The reign of Diocletian indeed was so long
and prosperous, that it must have been very unfavorable to the
reputation of Carinus.}

The only merit of the administration of Carinus that history
could record, or poetry celebrate, was the uncommon splendor with
which, in his own and his brother’s name, he exhibited the Roman
games of the theatre, the circus, and the amphitheatre. More than
twenty years afterwards, when the courtiers of Diocletian
represented to their frugal sovereign the fame and popularity of
his munificent predecessor, he acknowledged that the reign of
Carinus had indeed been a reign of pleasure.\textsuperscript{83} But this vain
prodigality, which the prudence of Diocletian might justly
despise, was enjoyed with surprise and transport by the Roman
people. The oldest of the citizens, recollecting the spectacles
of former days, the triumphal pomp of Probus or Aurelian, and the
secular games of the emperor Philip, acknowledged that they were
all surpassed by the superior magnificence of Carinus.\textsuperscript{84}

\pagenote[83]{Vopiscus in Hist. August. p. 254. He calls him
Carus, but the sense is sufficiently obvious, and the words were
often confounded.}

\pagenote[84]{See Calphurnius, Eclog. vii. 43. We may observe,
that the spectacles of Probus were still recent, and that the
poet is seconded by the historian.}

The spectacles of Carinus may therefore be best illustrated by
the observation of some particulars, which history has
condescended to relate concerning those of his predecessors. If
we confine ourselves solely to the hunting of wild beasts,
however we may censure the vanity of the design or the cruelty of
the execution, we are obliged to confess that neither before nor
since the time of the Romans so much art and expense have ever
been lavished for the amusement of the people.\textsuperscript{85} By the order of
Probus, a great quantity of large trees, torn up by the roots,
were transplanted into the midst of the circus. The spacious and
shady forest was immediately filled with a thousand ostriches, a
thousand stags, a thousand fallow deer, and a thousand wild
boars; and all this variety of game was abandoned to the riotous
impetuosity of the multitude. The tragedy of the succeeding day
consisted in the massacre of a hundred lions, an equal number of
lionesses, two hundred leopards, and three hundred bears.\textsuperscript{86} The
collection prepared by the younger Gordian for his triumph, and
which his successor exhibited in the secular games, was less
remarkable by the number than by the singularity of the animals.
Twenty zebras displayed their elegant forms and variegated beauty
to the eyes of the Roman people.\textsuperscript{87} Ten elks, and as many
camelopards, the loftiest and most harmless creatures that wander
over the plains of Sarmatia and Æthiopia, were contrasted with
thirty African hyænas and ten Indian tigers, the most implacable
savages of the torrid zone. The unoffending strength with which
Nature has endowed the greater quadrupeds was admired in the
rhinoceros, the hippopotamus of the Nile,\textsuperscript{88} and a majestic troop
of thirty-two elephants.\textsuperscript{89} While the populace gazed with stupid
wonder on the splendid show, the naturalist might indeed observe
the figure and properties of so many different species,
transported from every part of the ancient world into the
amphitheatre of Rome. But this accidental benefit, which science
might derive from folly, is surely insufficient to justify such a
wanton abuse of the public riches. There occurs, however, a
single instance in the first Punic war, in which the senate
wisely connected this amusement of the multitude with the
interest of the state. A considerable number of elephants, taken
in the defeat of the Carthaginian army, were driven through the
circus by a few slaves, armed only with blunt javelins.\textsuperscript{90} The
useful spectacle served to impress the Roman soldier with a just
contempt for those unwieldy animals; and he no longer dreaded to
encounter them in the ranks of war.

\pagenote[85]{The philosopher Montaigne (Essais, l. iii. 6) gives
a very just and lively view of Roman magnificence in these
spectacles.}

\pagenote[86]{Vopiscus in Hist. August. p. 240.}

\pagenote[87]{They are called Onagri; but the number is too
inconsiderable for mere wild asses. Cuper (de Elephantis
Exercitat. ii. 7) has proved from Oppian, Dion, and an anonymous
Greek, that zebras had been seen at Rome. They were brought from
some island of the ocean, perhaps Madagascar.}

\pagenote[88]{Carinus gave a hippopotamus, (see Calphurn. Eclog.
vi. 66.) In the latter spectacles, I do not recollect any
crocodiles, of which Augustus once exhibited thirty-six. Dion
Cassius, l. lv. p. 781.}

\pagenote[89]{Capitolin. in Hist. August. p. 164, 165. We are not
acquainted with the animals which he calls archeleontes; some
read argoleontes others agrioleontes: both corrections are very
nugatory}

\pagenote[90]{Plin. Hist. Natur. viii. 6, from the annals of
Piso.}

The hunting or exhibition of wild beasts was conducted with a
magnificence suitable to a people who styled themselves the
masters of the world; nor was the edifice appropriated to that
entertainment less expressive of Roman greatness. Posterity
admires, and will long admire, the awful remains of the
amphitheatre of Titus, which so well deserved the epithet of
Colossal.\textsuperscript{91} It was a building of an elliptic figure, five
hundred and sixty-four feet in length, and four hundred and
sixty-seven in breadth, founded on fourscore arches, and rising,
with four successive orders of architecture, to the height of one
hundred and forty feet.\textsuperscript{92} The outside of the edifice was
encrusted with marble, and decorated with statues. The slopes of
the vast concave, which formed the inside, were filled and
surrounded with sixty or eighty rows of seats of marble likewise,
covered with cushions, and capable of receiving with ease about
fourscore thousand spectators.\textsuperscript{93} Sixty-four \textit{vomitories} (for by
that name the doors were very aptly distinguished) poured forth
the immense multitude; and the entrances, passages, and
staircases were contrived with such exquisite skill, that each
person, whether of the senatorial, the equestrian, or the
plebeian order, arrived at his destined place without trouble or
confusion.\textsuperscript{94} Nothing was omitted, which, in any respect, could
be subservient to the convenience and pleasure of the spectators.

They were protected from the sun and rain by an ample canopy,
occasionally drawn over their heads. The air was continally
refreshed by the playing of fountains, and profusely impregnated
by the grateful scent of aromatics. In the centre of the edifice,
the \textit{arena}, or stage, was strewed with the finest sand, and
successively assumed the most different forms. At one moment it
seemed to rise out of the earth, like the garden of the
Hesperides, and was afterwards broken into the rocks and caverns
of Thrace. The subterraneous pipes conveyed an inexhaustible
supply of water; and what had just before appeared a level plain,
might be suddenly converted into a wide lake, covered with armed
vessels, and replenished with the monsters of the deep.\textsuperscript{95} In the
decoration of these scenes, the Roman emperors displayed their
wealth and liberality; and we read on various occasions that the
whole furniture of the amphitheatre consisted either of silver,
or of gold, or of amber.\textsuperscript{96} The poet who describes the games of
Carinus, in the character of a shepherd, attracted to the capital
by the fame of their magnificence, affirms that the nets designed
as a defence against the wild beasts were of gold wire; that the
porticos were gilded; and that the \textit{belt} or circle which divided
the several ranks of spectators from each other was studded with
a precious mosaic of beautiful stones.\textsuperscript{97}

\pagenote[91]{See Maffei, Verona Illustrata, p. iv. l. i. c. 2.}

\pagenote[92]{Maffei, l. ii. c. 2. The height was very much
exaggerated by the ancients. It reached almost to the heavens,
according to Calphurnius, (Eclog. vii. 23,) and surpassed the ken
of human sight, according to Ammianus Marcellinus (xvi. 10.) Yet
how trifling to the great pyramid of Egypt, which rises 500 feet
perpendicular}

\pagenote[93]{According to different copies of Victor, we read
77,000, or 87,000 spectators; but Maffei (l. ii. c. 12) finds
room on the open seats for no more than 34,000. The remainder
were contained in the upper covered galleries.}

\pagenote[94]{See Maffei, l. ii. c. 5—12. He treats the very
difficult subject with all possible clearness, and like an
architect, as well as an antiquarian.}

\pagenote[95]{Calphurn. Eclog vii. 64, 73. These lines are
curious, and the whole eclogue has been of infinite use to
Maffei. Calphurnius, as well as Martial, (see his first book,)
was a poet; but when they described the amphitheatre, they both
wrote from their own senses, and to those of the Romans.}

\pagenote[96]{Consult Plin. Hist. Natur. xxxiii. 16, xxxvii. 11.}

\pagenote[97]{Balteus en gemmis, en inlita porticus auro Certatim
radiant, \&c. Calphurn. vii.}

In the midst of this glittering pageantry, the emperor Carinus,
secure of his fortune, enjoyed the acclamations of the people,
the flattery of his courtiers, and the songs of the poets, who,
for want of a more essential merit, were reduced to celebrate the
divine graces of his person.\textsuperscript{98} In the same hour, but at the
distance of nine hundred miles from Rome, his brother expired;
and a sudden revolution transferred into the hands of a stranger
the sceptre of the house of Carus.\textsuperscript{99}

\pagenote[98]{Et Martis vultus et Apollinis esse putavi, says
Calphurnius; but John Malala, who had perhaps seen pictures of
Carinus, describes him as thick, short, and white, tom. i. p.
403.}

\pagenote[99]{With regard to the time when these Roman games were
celebrated, Scaliger, Salmasius, and Cuper have given themselves
a great deal of trouble to perplex a very clear subject.}

The sons of Carus never saw each other after their father’s
death. The arrangements which their new situation required were
probably deferred till the return of the younger brother to Rome,
where a triumph was decreed to the young emperors for the
glorious success of the Persian war.\textsuperscript{100} It is uncertain whether
they intended to divide between them the administration, or the
provinces, of the empire; but it is very unlikely that their
union would have proved of any long duration. The jealousy of
power must have been inflamed by the opposition of characters. In
the most corrupt of times, Carinus was unworthy to live: Numerian
deserved to reign in a happier period. His affable manners and
gentle virtues secured him, as soon as they became known, the
regard and affections of the public. He possessed the elegant
accomplishments of a poet and orator, which dignify as well as
adorn the humblest and the most exalted station. His eloquence,
however it was applauded by the senate, was formed not so much on
the model of Cicero, as on that of the modern declaimers; but in
an age very far from being destitute of poetical merit, he
contended for the prize with the most celebrated of his
contemporaries, and still remained the friend of his rivals; a
circumstance which evinces either the goodness of his heart, or
the superiority of his genius.\textsuperscript{101} But the talents of Numerian
were rather of the contemplative than of the active kind. When
his father’s elevation reluctantly forced him from the shade of
retirement, neither his temper nor his pursuits had qualified him
for the command of armies. His constitution was destroyed by the
hardships of the Persian war; and he had contracted, from the
heat of the climate,\textsuperscript{102} such a weakness in his eyes, as obliged
him, in the course of a long retreat, to confine himself to the
solitude and darkness of a tent or litter.

The administration of all affairs, civil as well as military, was
devolved on Arrius Aper, the Prætorian præfect, who to the power
of his important office added the honor of being father-in-law to
Numerian. The Imperial pavilion was strictly guarded by his most
trusty adherents; and during many days, Aper delivered to the
army the supposed mandates of their invisible sovereign.\textsuperscript{103}

\pagenote[100]{Nemesianus (in the Cynegeticon) seems to
anticipate in his fancy that auspicious day.}

\pagenote[101]{He won all the crowns from Nemesianus, with whom
he vied in didactic poetry. The senate erected a statue to the
son of Carus, with a very ambiguous inscription, “To the most
powerful of orators.” See Vopiscus in Hist. August. p. 251.}

\pagenote[102]{A more natural cause, at least, than that assigned
by Vopiscus, (Hist. August. p. 251,) incessantly weeping for his
father’s death.}

\pagenote[103]{In the Persian war, Aper was suspected of a design
to betray Carus. Hist. August. p. 250.}

It was not till eight months after the death of Carus, that the
Roman army, returning by slow marches from the banks of the
Tigris, arrived on those of the Thracian Bosphorus. The legions
halted at Chalcedon in Asia, while the court passed over to
Heraclea, on the European side of the Propontis.\textsuperscript{104} But a report
soon circulated through the camp, at first in secret whispers,
and at length in loud clamors, of the emperor’s death, and of the
presumption of his ambitious minister, who still exercised the
sovereign power in the name of a prince who was no more. The
impatience of the soldiers could not long support a state of
suspense. With rude curiosity they broke into the Imperial tent,
and discovered only the corpse of Numerian.\textsuperscript{105} The gradual
decline of his health might have induced them to believe that his
death was natural; but the concealment was interpreted as an
evidence of guilt, and the measures which Aper had taken to
secure his election became the immediate occasion of his ruin.
Yet, even in the transport of their rage and grief, the troops
observed a regular proceeding, which proves how firmly discipline
had been reëstablished by the martial successors of Gallienus. A
general assembly of the army was appointed to be held at
Chalcedon, whither Aper was transported in chains, as a prisoner
and a criminal. A vacant tribunal was erected in the midst of the
camp, and the generals and tribunes formed a great military
council. They soon announced to the multitude that their choice
had fallen on Diocletian, commander of the domestics or
body-guards, as the person the most capable of revenging and
succeeding their beloved emperor. The future fortunes of the
candidate depended on the chance or conduct of the present hour.
Conscious that the station which he had filled exposed him to
some suspicions, Diocletian ascended the tribunal, and raising
his eyes towards the Sun, made a solemn profession of his own
innocence, in the presence of that all-seeing Deity.\textsuperscript{106} Then,
assuming the tone of a sovereign and a judge, he commanded that
Aper should be brought in chains to the foot of the tribunal.
“This man,” said he, “is the murderer of Numerian;” and without
giving him time to enter on a dangerous justification, drew his
sword, and buried it in the breast of the unfortunate præfect. A
charge supported by such decisive proof was admitted without
contradiction, and the legions, with repeated acclamations,
acknowledged the justice and authority of the emperor Diocletian.\textsuperscript{107}

\pagenote[104]{We are obliged to the Alexandrian Chronicle, p.
274, for the knowledge of the time and place where Diocletian was
elected emperor.}

\pagenote[105]{Hist. August. p. 251. Eutrop. ix. 88. Hieronym. in
Chron. According to these judicious writers, the death of
Numerian was discovered by the stench of his dead body. Could no
aromatics be found in the Imperial household?}

\pagenote[106]{Aurel. Victor. Eutropius, ix. 20. Hieronym. in
Chron.}

\pagenote[107]{Vopiscus in Hist. August. p. 252. The reason why
Diocletian killed Aper, (a wild boar,) was founded on a prophecy
and a pun, as foolish as they are well known.}

Before we enter upon the memorable reign of that prince, it will
be proper to punish and dismiss the unworthy brother of Numerian.
Carinus possessed arms and treasures sufficient to support his
legal title to the empire. But his personal vices overbalanced
every advantage of birth and situation. The most faithful
servants of the father despised the incapacity, and dreaded the
cruel arrogance, of the son. The hearts of the people were
engaged in favor of his rival, and even the senate was inclined
to prefer a usurper to a tyrant. The arts of Diocletian inflamed
the general discontent; and the winter was employed in secret
intrigues, and open preparations for a civil war. In the spring,
the forces of the East and of the West encountered each other in
the plains of Margus, a small city of Mæsia, in the neighborhood
of the Danube.\textsuperscript{108} The troops, so lately returned from the
Persian war, had acquired their glory at the expense of health
and numbers; nor were they in a condition to contend with the
unexhausted strength of the legions of Europe. Their ranks were
broken, and, for a moment, Diocletian despaired of the purple and
of life. But the advantage which Carinus had obtained by the
valor of his soldiers, he quickly lost by the infidelity of his
officers. A tribune, whose wife he had seduced, seized the
opportunity of revenge, and, by a single blow, extinguished civil
discord in the blood of the adulterer.\textsuperscript{109}

\pagenote[108]{Eutropius marks its situation very accurately; it
was between the Mons Aureus and Viminiacum. M. d’Anville
(Geographic Ancienne, tom. i. p. 304) places Margus at Kastolatz
in Servia, a little below Belgrade and Semendria. * Note:
Kullieza—Eton Atlas—M.}

\pagenote[109]{Hist. August. p. 254. Eutropius, ix. 20. Aurelius
Victor et Epitome}

