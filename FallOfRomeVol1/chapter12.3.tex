\section{Part \thesection.}
\thispagestyle{simple}

The vacancy of the throne was not productive of any disturbance.
The ambition of the aspiring generals was checked by their
natural fears, and young Numerian, with his absent brother
Carinus, were unanimously acknowledged as Roman emperors.

The public expected that the successor of Carus would pursue his
father’s footsteps, and, without allowing the Persians to recover
from their consternation, would advance sword in hand to the
palaces of Susa and Ecbatana.\footnotemark[77] But the legions, however strong
in numbers and discipline, were dismayed by the most abject
superstition. Notwithstanding all the arts that were practised to
disguise the manner of the late emperor’s death, it was found
impossible to remove the opinion of the multitude, and the power
of opinion is irresistible. Places or persons struck with
lightning were considered by the ancients with pious horror, as
singularly devoted to the wrath of Heaven.\footnotemark[78] An oracle was
remembered, which marked the River Tigris as the fatal boundary
of the Roman arms. The troops, terrified with the fate of Carus
and with their own danger, called aloud on young Numerian to obey
the will of the gods, and to lead them away from this
inauspicious scene of war. The feeble emperor was unable to
subdue their obstinate prejudice, and the Persians wondered at
the unexpected retreat of a victorious enemy.\footnotemark[79]

\footnotetext[77]{See Nemesian. Cynegeticon, v. 71, \&c.}

\footnotetext[78]{See Festus and his commentators on the word
Scribonianum. Places struck by lightning were surrounded with a
wall; things were buried with mysterious ceremony.}

\footnotetext[79]{Vopiscus in Hist. August. p. 250. Aurelius Victor
seems to believe the prediction, and to approve the retreat.}

The intelligence of the mysterious fate of the late emperor was
soon carried from the frontiers of Persia to Rome; and the
senate, as well as the provinces, congratulated the accession of
the sons of Carus. These fortunate youths were strangers,
however, to that conscious superiority, either of birth or of
merit, which can alone render the possession of a throne easy,
and, as it were, natural. Born and educated in a private station,
the election of their father raised them at once to the rank of
princes; and his death, which happened about sixteen months
afterwards, left them the unexpected legacy of a vast empire. To
sustain with temper this rapid elevation, an uncommon share of
virtue and prudence was requisite; and Carinus, the elder of the
brothers, was more than commonly deficient in those qualities. In
the Gallic war he discovered some degree of personal courage;\footnotemark[80]
but from the moment of his arrival at Rome, he abandoned himself
to the luxury of the capital, and to the abuse of his fortune. He
was soft, yet cruel; devoted to pleasure, but destitute of taste;
and though exquisitely susceptible of vanity, indifferent to the
public esteem. In the course of a few months, he successively
married and divorced nine wives, most of whom he left pregnant;
and notwithstanding this legal inconstancy, found time to indulge
such a variety of irregular appetites, as brought dishonor on
himself and on the noblest houses of Rome. He beheld with
inveterate hatred all those who might remember his former
obscurity, or censure his present conduct. He banished, or put to
death, the friends and counsellors whom his father had placed
about him, to guide his inexperienced youth; and he persecuted
with the meanest revenge his school-fellows and companions who
had not sufficiently respected the latent majesty of the emperor.

With the senators, Carinus affected a lofty and regal demeanor,
frequently declaring, that he designed to distribute their
estates among the populace of Rome. From the dregs of that
populace he selected his favorites, and even his ministers. The
palace, and even the Imperial table, were filled with singers,
dancers, prostitutes, and all the various retinue of vice and
folly. One of his doorkeepers\footnotemark[81] he intrusted with the government
of the city. In the room of the Prætorian præfect, whom he put to
death, Carinus substituted one of the ministers of his looser
pleasures. Another, who possessed the same, or even a more
infamous, title to favor, was invested with the consulship. A
confidential secretary, who had acquired uncommon skill in the
art of forgery, delivered the indolent emperor, with his own
consent from the irksome duty of signing his name.

\footnotetext[80]{Nemesian. Cynegeticon, v 69. He was a contemporary,
but a poet.}

\footnotetext[81]{Cancellarius. This word, so humble in its origin,
has, by a singular fortune, risen into the title of the first
great office of state in the monarchies of Europe. See Casaubon
and Salmasius, ad Hist. August, p. 253.}

When the emperor Carus undertook the Persian war, he was induced,
by motives of affection as well as policy, to secure the fortunes
of his family, by leaving in the hands of his eldest son the
armies and provinces of the West. The intelligence which he soon
received of the conduct of Carinus filled him with shame and
regret; nor had he concealed his resolution of satisfying the
republic by a severe act of justice, and of adopting, in the
place of an unworthy son, the brave and virtuous Constantius, who
at that time was governor of Dalmatia. But the elevation of
Constantius was for a while deferred; and as soon as the father’s
death had released Carinus from the control of fear or decency,
he displayed to the Romans the extravagancies of Elagabalus,
aggravated by the cruelty of Domitian.\footnotemark[82]

\footnotetext[82]{Vopiscus in Hist. August. p. 253, 254. Eutropius,
x. 19. Vic to Junior. The reign of Diocletian indeed was so long
and prosperous, that it must have been very unfavorable to the
reputation of Carinus.}

The only merit of the administration of Carinus that history
could record, or poetry celebrate, was the uncommon splendor with
which, in his own and his brother’s name, he exhibited the Roman
games of the theatre, the circus, and the amphitheatre. More than
twenty years afterwards, when the courtiers of Diocletian
represented to their frugal sovereign the fame and popularity of
his munificent predecessor, he acknowledged that the reign of
Carinus had indeed been a reign of pleasure.\footnotemark[83] But this vain
prodigality, which the prudence of Diocletian might justly
despise, was enjoyed with surprise and transport by the Roman
people. The oldest of the citizens, recollecting the spectacles
of former days, the triumphal pomp of Probus or Aurelian, and the
secular games of the emperor Philip, acknowledged that they were
all surpassed by the superior magnificence of Carinus.\footnotemark[84]

\footnotetext[83]{Vopiscus in Hist. August. p. 254. He calls him
Carus, but the sense is sufficiently obvious, and the words were
often confounded.}

\footnotetext[84]{See Calphurnius, Eclog. vii. 43. We may observe,
that the spectacles of Probus were still recent, and that the
poet is seconded by the historian.}

The spectacles of Carinus may therefore be best illustrated by
the observation of some particulars, which history has
condescended to relate concerning those of his predecessors. If
we confine ourselves solely to the hunting of wild beasts,
however we may censure the vanity of the design or the cruelty of
the execution, we are obliged to confess that neither before nor
since the time of the Romans so much art and expense have ever
been lavished for the amusement of the people.\footnotemark[85] By the order of
Probus, a great quantity of large trees, torn up by the roots,
were transplanted into the midst of the circus. The spacious and
shady forest was immediately filled with a thousand ostriches, a
thousand stags, a thousand fallow deer, and a thousand wild
boars; and all this variety of game was abandoned to the riotous
impetuosity of the multitude. The tragedy of the succeeding day
consisted in the massacre of a hundred lions, an equal number of
lionesses, two hundred leopards, and three hundred bears.\footnotemark[86] The
collection prepared by the younger Gordian for his triumph, and
which his successor exhibited in the secular games, was less
remarkable by the number than by the singularity of the animals.
Twenty zebras displayed their elegant forms and variegated beauty
to the eyes of the Roman people.\footnotemark[87] Ten elks, and as many
camelopards, the loftiest and most harmless creatures that wander
over the plains of Sarmatia and Æthiopia, were contrasted with
thirty African hyænas and ten Indian tigers, the most implacable
savages of the torrid zone. The unoffending strength with which
Nature has endowed the greater quadrupeds was admired in the
rhinoceros, the hippopotamus of the Nile,\footnotemark[88] and a majestic troop
of thirty-two elephants.\footnotemark[89] While the populace gazed with stupid
wonder on the splendid show, the naturalist might indeed observe
the figure and properties of so many different species,
transported from every part of the ancient world into the
amphitheatre of Rome. But this accidental benefit, which science
might derive from folly, is surely insufficient to justify such a
wanton abuse of the public riches. There occurs, however, a
single instance in the first Punic war, in which the senate
wisely connected this amusement of the multitude with the
interest of the state. A considerable number of elephants, taken
in the defeat of the Carthaginian army, were driven through the
circus by a few slaves, armed only with blunt javelins.\footnotemark[90] The
useful spectacle served to impress the Roman soldier with a just
contempt for those unwieldy animals; and he no longer dreaded to
encounter them in the ranks of war.

\footnotetext[85]{The philosopher Montaigne (Essais, l. iii. 6) gives
a very just and lively view of Roman magnificence in these
spectacles.}

\footnotetext[86]{Vopiscus in Hist. August. p. 240.}

\footnotetext[87]{They are called Onagri; but the number is too
inconsiderable for mere wild asses. Cuper (de Elephantis
Exercitat. ii. 7) has proved from Oppian, Dion, and an anonymous
Greek, that zebras had been seen at Rome. They were brought from
some island of the ocean, perhaps Madagascar.}

\footnotetext[88]{Carinus gave a hippopotamus, (see Calphurn. Eclog.
vi. 66.) In the latter spectacles, I do not recollect any
crocodiles, of which Augustus once exhibited thirty-six. Dion
Cassius, l. lv. p. 781.}

\footnotetext[89]{Capitolin. in Hist. August. p. 164, 165. We are not
acquainted with the animals which he calls archeleontes; some
read argoleontes others agrioleontes: both corrections are very
nugatory}

\footnotetext[90]{Plin. Hist. Natur. viii. 6, from the annals of
Piso.}

The hunting or exhibition of wild beasts was conducted with a
magnificence suitable to a people who styled themselves the
masters of the world; nor was the edifice appropriated to that
entertainment less expressive of Roman greatness. Posterity
admires, and will long admire, the awful remains of the
amphitheatre of Titus, which so well deserved the epithet of
Colossal.\footnotemark[91] It was a building of an elliptic figure, five
hundred and sixty-four feet in length, and four hundred and
sixty-seven in breadth, founded on fourscore arches, and rising,
with four successive orders of architecture, to the height of one
hundred and forty feet.\footnotemark[92] The outside of the edifice was
encrusted with marble, and decorated with statues. The slopes of
the vast concave, which formed the inside, were filled and
surrounded with sixty or eighty rows of seats of marble likewise,
covered with cushions, and capable of receiving with ease about
fourscore thousand spectators.\footnotemark[93] Sixty-four \textit{vomitories} (for by
that name the doors were very aptly distinguished) poured forth
the immense multitude; and the entrances, passages, and
staircases were contrived with such exquisite skill, that each
person, whether of the senatorial, the equestrian, or the
plebeian order, arrived at his destined place without trouble or
confusion.\footnotemark[94] Nothing was omitted, which, in any respect, could
be subservient to the convenience and pleasure of the spectators.

They were protected from the sun and rain by an ample canopy,
occasionally drawn over their heads. The air was continally
refreshed by the playing of fountains, and profusely impregnated
by the grateful scent of aromatics. In the centre of the edifice,
the \textit{arena}, or stage, was strewed with the finest sand, and
successively assumed the most different forms. At one moment it
seemed to rise out of the earth, like the garden of the
Hesperides, and was afterwards broken into the rocks and caverns
of Thrace. The subterraneous pipes conveyed an inexhaustible
supply of water; and what had just before appeared a level plain,
might be suddenly converted into a wide lake, covered with armed
vessels, and replenished with the monsters of the deep.\footnotemark[95] In the
decoration of these scenes, the Roman emperors displayed their
wealth and liberality; and we read on various occasions that the
whole furniture of the amphitheatre consisted either of silver,
or of gold, or of amber.\footnotemark[96] The poet who describes the games of
Carinus, in the character of a shepherd, attracted to the capital
by the fame of their magnificence, affirms that the nets designed
as a defence against the wild beasts were of gold wire; that the
porticos were gilded; and that the \textit{belt} or circle which divided
the several ranks of spectators from each other was studded with
a precious mosaic of beautiful stones.\footnotemark[97]

\footnotetext[91]{See Maffei, Verona Illustrata, p. iv. l. i. c. 2.}

\footnotetext[92]{Maffei, l. ii. c. 2. The height was very much
exaggerated by the ancients. It reached almost to the heavens,
according to Calphurnius, (Eclog. vii. 23,) and surpassed the ken
of human sight, according to Ammianus Marcellinus (xvi. 10.) Yet
how trifling to the great pyramid of Egypt, which rises 500 feet
perpendicular}

\footnotetext[93]{According to different copies of Victor, we read
77,000, or 87,000 spectators; but Maffei (l. ii. c. 12) finds
room on the open seats for no more than 34,000. The remainder
were contained in the upper covered galleries.}

\footnotetext[94]{See Maffei, l. ii. c. 5—12. He treats the very
difficult subject with all possible clearness, and like an
architect, as well as an antiquarian.}

\footnotetext[95]{Calphurn. Eclog vii. 64, 73. These lines are
curious, and the whole eclogue has been of infinite use to
Maffei. Calphurnius, as well as Martial, (see his first book,)
was a poet; but when they described the amphitheatre, they both
wrote from their own senses, and to those of the Romans.}

\footnotetext[96]{Consult Plin. Hist. Natur. xxxiii. 16, xxxvii. 11.}

\footnotetext[97]{Balteus en gemmis, en inlita porticus auro Certatim
radiant, \&c. Calphurn. vii.}

In the midst of this glittering pageantry, the emperor Carinus,
secure of his fortune, enjoyed the acclamations of the people,
the flattery of his courtiers, and the songs of the poets, who,
for want of a more essential merit, were reduced to celebrate the
divine graces of his person.\footnotemark[98] In the same hour, but at the
distance of nine hundred miles from Rome, his brother expired;
and a sudden revolution transferred into the hands of a stranger
the sceptre of the house of Carus.\footnotemark[99]

\footnotetext[98]{Et Martis vultus et Apollinis esse putavi, says
Calphurnius; but John Malala, who had perhaps seen pictures of
Carinus, describes him as thick, short, and white, tom. i. p.
403.}

\footnotetext[99]{With regard to the time when these Roman games were
celebrated, Scaliger, Salmasius, and Cuper have given themselves
a great deal of trouble to perplex a very clear subject.}

The sons of Carus never saw each other after their father’s
death. The arrangements which their new situation required were
probably deferred till the return of the younger brother to Rome,
where a triumph was decreed to the young emperors for the
glorious success of the Persian war.\footnotemark[100] It is uncertain whether
they intended to divide between them the administration, or the
provinces, of the empire; but it is very unlikely that their
union would have proved of any long duration. The jealousy of
power must have been inflamed by the opposition of characters. In
the most corrupt of times, Carinus was unworthy to live: Numerian
deserved to reign in a happier period. His affable manners and
gentle virtues secured him, as soon as they became known, the
regard and affections of the public. He possessed the elegant
accomplishments of a poet and orator, which dignify as well as
adorn the humblest and the most exalted station. His eloquence,
however it was applauded by the senate, was formed not so much on
the model of Cicero, as on that of the modern declaimers; but in
an age very far from being destitute of poetical merit, he
contended for the prize with the most celebrated of his
contemporaries, and still remained the friend of his rivals; a
circumstance which evinces either the goodness of his heart, or
the superiority of his genius.\footnotemark[101] But the talents of Numerian
were rather of the contemplative than of the active kind. When
his father’s elevation reluctantly forced him from the shade of
retirement, neither his temper nor his pursuits had qualified him
for the command of armies. His constitution was destroyed by the
hardships of the Persian war; and he had contracted, from the
heat of the climate,\footnotemark[102] such a weakness in his eyes, as obliged
him, in the course of a long retreat, to confine himself to the
solitude and darkness of a tent or litter.

The administration of all affairs, civil as well as military, was
devolved on Arrius Aper, the Prætorian præfect, who to the power
of his important office added the honor of being father-in-law to
Numerian. The Imperial pavilion was strictly guarded by his most
trusty adherents; and during many days, Aper delivered to the
army the supposed mandates of their invisible sovereign.\footnotemark[103]

\footnotetext[100]{Nemesianus (in the Cynegeticon) seems to
anticipate in his fancy that auspicious day.}

\footnotetext[101]{He won all the crowns from Nemesianus, with whom
he vied in didactic poetry. The senate erected a statue to the
son of Carus, with a very ambiguous inscription, “To the most
powerful of orators.” See Vopiscus in Hist. August. p. 251.}

\footnotetext[102]{A more natural cause, at least, than that assigned
by Vopiscus, (Hist. August. p. 251,) incessantly weeping for his
father’s death.}

\footnotetext[103]{In the Persian war, Aper was suspected of a design
to betray Carus. Hist. August. p. 250.}

It was not till eight months after the death of Carus, that the
Roman army, returning by slow marches from the banks of the
Tigris, arrived on those of the Thracian Bosphorus. The legions
halted at Chalcedon in Asia, while the court passed over to
Heraclea, on the European side of the Propontis.\footnotemark[104] But a report
soon circulated through the camp, at first in secret whispers,
and at length in loud clamors, of the emperor’s death, and of the
presumption of his ambitious minister, who still exercised the
sovereign power in the name of a prince who was no more. The
impatience of the soldiers could not long support a state of
suspense. With rude curiosity they broke into the Imperial tent,
and discovered only the corpse of Numerian.\footnotemark[105] The gradual
decline of his health might have induced them to believe that his
death was natural; but the concealment was interpreted as an
evidence of guilt, and the measures which Aper had taken to
secure his election became the immediate occasion of his ruin.
Yet, even in the transport of their rage and grief, the troops
observed a regular proceeding, which proves how firmly discipline
had been reëstablished by the martial successors of Gallienus. A
general assembly of the army was appointed to be held at
Chalcedon, whither Aper was transported in chains, as a prisoner
and a criminal. A vacant tribunal was erected in the midst of the
camp, and the generals and tribunes formed a great military
council. They soon announced to the multitude that their choice
had fallen on Diocletian, commander of the domestics or
body-guards, as the person the most capable of revenging and
succeeding their beloved emperor. The future fortunes of the
candidate depended on the chance or conduct of the present hour.
Conscious that the station which he had filled exposed him to
some suspicions, Diocletian ascended the tribunal, and raising
his eyes towards the Sun, made a solemn profession of his own
innocence, in the presence of that all-seeing Deity.\footnotemark[106] Then,
assuming the tone of a sovereign and a judge, he commanded that
Aper should be brought in chains to the foot of the tribunal.
“This man,” said he, “is the murderer of Numerian;” and without
giving him time to enter on a dangerous justification, drew his
sword, and buried it in the breast of the unfortunate præfect. A
charge supported by such decisive proof was admitted without
contradiction, and the legions, with repeated acclamations,
acknowledged the justice and authority of the emperor Diocletian.\footnotemark[107]

\footnotetext[104]{We are obliged to the Alexandrian Chronicle, p.
274, for the knowledge of the time and place where Diocletian was
elected emperor.}

\footnotetext[105]{Hist. August. p. 251. Eutrop. ix. 88. Hieronym. in
Chron. According to these judicious writers, the death of
Numerian was discovered by the stench of his dead body. Could no
aromatics be found in the Imperial household?}

\footnotetext[106]{Aurel. Victor. Eutropius, ix. 20. Hieronym. in
Chron.}

\footnotetext[107]{Vopiscus in Hist. August. p. 252. The reason why
Diocletian killed Aper, (a wild boar,) was founded on a prophecy
and a pun, as foolish as they are well known.}

Before we enter upon the memorable reign of that prince, it will
be proper to punish and dismiss the unworthy brother of Numerian.
Carinus possessed arms and treasures sufficient to support his
legal title to the empire. But his personal vices overbalanced
every advantage of birth and situation. The most faithful
servants of the father despised the incapacity, and dreaded the
cruel arrogance, of the son. The hearts of the people were
engaged in favor of his rival, and even the senate was inclined
to prefer a usurper to a tyrant. The arts of Diocletian inflamed
the general discontent; and the winter was employed in secret
intrigues, and open preparations for a civil war. In the spring,
the forces of the East and of the West encountered each other in
the plains of Margus, a small city of Mæsia, in the neighborhood
of the Danube.\footnotemark[108] The troops, so lately returned from the
Persian war, had acquired their glory at the expense of health
and numbers; nor were they in a condition to contend with the
unexhausted strength of the legions of Europe. Their ranks were
broken, and, for a moment, Diocletian despaired of the purple and
of life. But the advantage which Carinus had obtained by the
valor of his soldiers, he quickly lost by the infidelity of his
officers. A tribune, whose wife he had seduced, seized the
opportunity of revenge, and, by a single blow, extinguished civil
discord in the blood of the adulterer.\footnotemark[109]

\footnotetext[108]{Eutropius marks its situation very accurately; it
was between the Mons Aureus and Viminiacum. M. d’Anville
(Geographic Ancienne, tom. i. p. 304) places Margus at Kastolatz
in Servia, a little below Belgrade and Semendria. * Note:
Kullieza—Eton Atlas—M.}

\footnotetext[109]{Hist. August. p. 254. Eutropius, ix. 20. Aurelius
Victor et Epitome}

