\chapter{Tyranny Of Maximin, Rebellion, Civil Wars, Death Of Maximin.}
\section{Part \thesection.}

\textit{The Elevation And Tyranny Of Maximin. — Rebellion In Africa And
Italy, Under The Authority Of The Senate. — Civil Wars And
Seditions. — Violent Deaths Of Maximin And His Son, Of Maximus And
Balbinus, And Of The Three Gordians. — Usurpation And Secular Games
Of Philip.}
\vspace{\onelineskip}

Of the various forms of government which have prevailed in the
world, an hereditary monarchy seems to present the fairest scope
for ridicule. Is it possible to relate without an indignant
smile, that, on the father’s decease, the property of a nation,
like that of a drove of oxen, descends to his infant son, as yet
unknown to mankind and to himself; and that the bravest warriors
and the wisest statesmen, relinquishing their natural right to
empire, approach the royal cradle with bended knees and
protestations of inviolable fidelity? Satire and declamation may
paint these obvious topics in the most dazzling colors, but our
more serious thoughts will respect a useful prejudice, that
establishes a rule of succession, independent of the passions of
mankind; and we shall cheerfully acquiesce in any expedient which
deprives the multitude of the dangerous, and indeed the ideal,
power of giving themselves a master.

In the cool shade of retirement, we may easily devise imaginary
forms of government, in which the sceptre shall be constantly
bestowed on the most worthy, by the free and incorrupt suffrage
of the whole community. Experience overturns these airy fabrics,
and teaches us, that in a large society, the election of a
monarch can never devolve to the wisest, or to the most numerous
part of the people. The army is the only order of men
sufficiently united to concur in the same sentiments, and
powerful enough to impose them on the rest of their
fellow-citizens; but the temper of soldiers, habituated at once
to violence and to slavery, renders them very unfit guardians of
a legal, or even a civil constitution. Justice, humanity, or
political wisdom, are qualities they are too little acquainted
with in themselves, to appreciate them in others. Valor will
acquire their esteem, and liberality will purchase their
suffrage; but the first of these merits is often lodged in the
most savage breasts; the latter can only exert itself at the
expense of the public; and both may be turned against the
possessor of the throne, by the ambition of a daring rival.

The superior prerogative of birth, when it has obtained the
sanction of time and popular opinion, is the plainest and least
invidious of all distinctions among mankind. The acknowledged
right extinguishes the hopes of faction, and the conscious
security disarms the cruelty of the monarch. To the firm
establishment of this idea we owe the peaceful succession and
mild administration of European monarchies. To the defect of it
we must attribute the frequent civil wars, through which an
Asiatic despot is obliged to cut his way to the throne of his
fathers. Yet, even in the East, the sphere of contention is
usually limited to the princes of the reigning house, and as soon
as the more fortunate competitor has removed his brethren by the
sword and the bowstring, he no longer entertains any jealousy of
his meaner subjects. But the Roman empire, after the authority of
the senate had sunk into contempt, was a vast scene of confusion.
The royal, and even noble, families of the provinces had long
since been led in triumph before the car of the haughty
republicans. The ancient families of Rome had successively fallen
beneath the tyranny of the Cæsars; and whilst those princes were
shackled by the forms of a commonwealth, and disappointed by the
repeated failure of their posterity,\textsuperscript{1} it was impossible that any
idea of hereditary succession should have taken root in the minds
of their subjects. The right to the throne, which none could
claim from birth, every one assumed from merit. The daring hopes
of ambition were set loose from the salutary restraints of law
and prejudice; and the meanest of mankind might, without folly,
entertain a hope of being raised by valor and fortune to a rank
in the army, in which a single crime would enable him to wrest
the sceptre of the world from his feeble and unpopular master.
After the murder of Alexander Severus, and the elevation of
Maximin, no emperor could think himself safe upon the throne, and
every barbarian peasant of the frontier might aspire to that
august, but dangerous station.

\pagenote[1]{There had been no example of three successive
generations on the throne; only three instances of sons who
succeeded their fathers. The marriages of the Cæsars
(notwithstanding the permission, and the frequent practice of
divorces) were generally unfruitful.}

About thirty-two years before that event, the emperor Severus,
returning from an eastern expedition, halted in Thrace, to
celebrate, with military games, the birthday of his younger son,
Geta. The country flocked in crowds to behold their sovereign,
and a young barbarian of gigantic stature earnestly solicited, in
his rude dialect, that he might be allowed to contend for the
prize of wrestling. As the pride of discipline would have been
disgraced in the overthrow of a Roman soldier by a Thracian
peasant, he was matched with the stoutest followers of the camp,
sixteen of whom he successively laid on the ground. His victory
was rewarded by some trifling gifts, and a permission to enlist
in the troops. The next day, the happy barbarian was
distinguished above a crowd of recruits, dancing and exulting
after the fashion of his country. As soon as he perceived that he
had attracted the emperor’s notice, he instantly ran up to his
horse, and followed him on foot, without the least appearance of
fatigue, in a long and rapid career. “Thracian,” said Severus
with astonishment, “art thou disposed to wrestle after thy race?”
“Most willingly, sir,” replied the unwearied youth; and, almost
in a breath, overthrew seven of the strongest soldiers in the
army. A gold collar was the prize of his matchless vigor and
activity, and he was immediately appointed to serve in the
horseguards who always attended on the person of the sovereign.\textsuperscript{2}

\pagenote[2]{Hist. August p. 138.}

Maximin, for that was his name, though born on the territories of
the empire, descended from a mixed race of barbarians. His father
was a Goth, and his mother of the nation of the Alani. He
displayed on every occasion a valor equal to his strength; and
his native fierceness was soon tempered or disguised by the
knowledge of the world. Under the reign of Severus and his son,
he obtained the rank of centurion, with the favor and esteem of
both those princes, the former of whom was an excellent judge of
merit. Gratitude forbade Maximin to serve under the assassin of
Caracalla. Honor taught him to decline the effeminate insults of
Elagabalus. On the accession of Alexander he returned to court,
and was placed by that prince in a station useful to the service,
and honorable to himself. The fourth legion, to which he was
appointed tribune, soon became, under his care, the best
disciplined of the whole army. With the general applause of the
soldiers, who bestowed on their favorite hero the names of Ajax
and Hercules, he was successively promoted to the first military
command;\textsuperscript{3} and had not he still retained too much of his savage
origin, the emperor might perhaps have given his own sister in
marriage to the son of Maximin.\textsuperscript{4}

\pagenote[3]{Hist. August. p. 140. Herodian, l. vi. p. 223.
Aurelius Victor. By comparing these authors, it should seem that
Maximin had the particular command of the Tribellian horse, with
the general commission of disciplining the recruits of the whole
army. His biographer ought to have marked, with more care, his
exploits, and the successive steps of his military promotions.}

\pagenote[4]{See the original letter of Alexander Severus, Hist.
August. p. 149.}

Instead of securing his fidelity, these favors served only to
inflame the ambition of the Thracian peasant, who deemed his
fortune inadequate to his merit, as long as he was constrained to
acknowledge a superior. Though a stranger to real wisdom, he was
not devoid of a selfish cunning, which showed him that the
emperor had lost the affection of the army, and taught him to
improve their discontent to his own advantage. It is easy for
faction and calumny to shed their poison on the administration of
the best of princes, and to accuse even their virtues by artfully
confounding them with those vices to which they bear the nearest
affinity. The troops listened with pleasure to the emissaries of
Maximin. They blushed at their own ignominious patience, which,
during thirteen years, had supported the vexatious discipline
imposed by an effeminate Syrian, the timid slave of his mother
and of the senate. It was time, they cried, to cast away that
useless phantom of the civil power, and to elect for their prince
and general a real soldier, educated in camps, exercised in war,
who would assert the glory, and distribute among his companions
the treasures, of the empire. A great army was at that time
assembled on the banks of the Rhine, under the command of the
emperor himself, who, almost immediately after his return from
the Persian war, had been obliged to march against the barbarians
of Germany. The important care of training and reviewing the new
levies was intrusted to Maximin. One day, as he entered the field
of exercise, the troops, either from a sudden impulse, or a
formed conspiracy, saluted him emperor, silenced by their loud
acclamations his obstinate refusal, and hastened to consummate
their rebellion by the murder of Alexander Severus.

The circumstances of his death are variously related. The
writers, who suppose that he died in ignorance of the ingratitude
and ambition of Maximin affirm that, after taking a frugal repast
in the sight of the army, he retired to sleep, and that, about
the seventh hour of the day, a part of his own guards broke into
the imperial tent, and, with many wounds, assassinated their
virtuous and unsuspecting prince.\textsuperscript{5} If we credit another, and
indeed a more probable account, Maximin was invested with the
purple by a numerous detachment, at the distance of several miles
from the head-quarters; and he trusted for success rather to the
secret wishes than to the public declarations of the great army.
Alexander had sufficient time to awaken a faint sense of loyalty
among the troops; but their reluctant professions of fidelity
quickly vanished on the appearance of Maximin, who declared
himself the friend and advocate of the military order, and was
unanimously acknowledged emperor of the Romans by the applauding
legions. The son of Mamæa, betrayed and deserted, withdrew into
his tent, desirous at least to conceal his approaching fate from
the insults of the multitude. He was soon followed by a tribune
and some centurions, the ministers of death; but instead of
receiving with manly resolution the inevitable stroke, his
unavailing cries and entreaties disgraced the last moments of his
life, and converted into contempt some portion of the just pity
which his innocence and misfortunes must inspire. His mother,
Mamæa, whose pride and avarice he loudly accused as the cause of
his ruin, perished with her son. The most faithful of his friends
were sacrificed to the first fury of the soldiers. Others were
reserved for the more deliberate cruelty of the usurper; and
those who experienced the mildest treatment, were stripped of
their employments, and ignominiously driven from the court and
army.\textsuperscript{6}

\pagenote[5]{Hist. August. p. 135. I have softened some of the
most improbable circumstances of this wretched biographer. From
his ill-worded narration, it should seem that the prince’s
buffoon having accidentally entered the tent, and awakened the
slumbering monarch, the fear of punishment urged him to persuade
the disaffected soldiers to commit the murder.}

\pagenote[6]{Herodian, l. vi. 223-227.}

The former tyrants, Caligula and Nero, Commodus, and Caracalla,
were all dissolute and unexperienced youths,\textsuperscript{7} educated in the
purple, and corrupted by the pride of empire, the luxury of Rome,
and the perfidious voice of flattery. The cruelty of Maximin was
derived from a different source, the fear of contempt. Though he
depended on the attachment of the soldiers, who loved him for
virtues like their own, he was conscious that his mean and
barbarian origin, his savage appearance, and his total ignorance
of the arts and institutions of civil life,\textsuperscript{8} formed a very
unfavorable contrast with the amiable manners of the unhappy
Alexander. He remembered, that, in his humbler fortune, he had
often waited before the door of the haughty nobles of Rome, and
had been denied admittance by the insolence of their slaves. He
recollected too the friendship of a few who had relieved his
poverty, and assisted his rising hopes. But those who had
spurned, and those who had protected, the Thracian, were guilty
of the same crime, the knowledge of his original obscurity. For
this crime many were put to death; and by the execution of
several of his benefactors, Maximin published, in characters of
blood, the indelible history of his baseness and ingratitude.\textsuperscript{9}

\pagenote[7]{Caligula, the eldest of the four, was only
twenty-five years of age when he ascended the throne; Caracalla
was twenty-three, Commodus nineteen, and Nero no more than
seventeen.}

\pagenote[8]{It appears that he was totally ignorant of the Greek
language; which, from its universal use in conversation and
letters, was an essential part of every liberal education.}

\pagenote[9]{Hist. August. p. 141. Herodian, l. vii. p. 237. The
latter of these historians has been most unjustly censured for
sparing the vices of Maximin.}

The dark and sanguinary soul of the tyrant was open to every
suspicion against those among his subjects who were the most
distinguished by their birth or merit. Whenever he was alarmed
with the sound of treason, his cruelty was unbounded and
unrelenting. A conspiracy against his life was either discovered
or imagined, and Magnus, a consular senator, was named as the
principal author of it. Without a witness, without a trial, and
without an opportunity of defence, Magnus, with four thousand of
his supposed accomplices, was put to death. Italy and the whole
empire were infested with innumerable spies and informers. On the
slightest accusation, the first of the Roman nobles, who had
governed provinces, commanded armies, and been adorned with the
consular and triumphal ornaments, were chained on the public
carriages, and hurried away to the emperor’s presence.
Confiscation, exile, or simple death, were esteemed uncommon
instances of his lenity. Some of the unfortunate sufferers he
ordered to be sewed up in the hides of slaughtered animals,
others to be exposed to wild beasts, others again to be beaten to
death with clubs. During the three years of his reign, he
disdained to visit either Rome or Italy. His camp, occasionally
removed from the banks of the Rhine to those of the Danube, was
the seat of his stern despotism, which trampled on every
principle of law and justice, and was supported by the avowed
power of the sword.\textsuperscript{10} No man of noble birth, elegant
accomplishments, or knowledge of civil business, was suffered
near his person; and the court of a Roman emperor revived the
idea of those ancient chiefs of slaves and gladiators, whose
savage power had left a deep impression of terror and
detestation.\textsuperscript{11}

\pagenote[10]{The wife of Maximin, by insinuating wise counsels
with female gentleness, sometimes brought back the tyrant to the
way of truth and humanity. See Ammianus Marcellinus, l. xiv. c.
l, where he alludes to the fact which he had more fully related
under the reign of the Gordians. We may collect from the medals,
that Paullina was the name of this benevolent empress; and from
the title of Diva, that she died before Maximin. (Valesius ad
loc. cit. Ammian.) Spanheim de U. et P. N. tom. ii. p. 300. Note:
If we may believe Syrcellus and Zonaras, in was Maximin himself
who ordered her death—G}

\pagenote[11]{He was compared to Spartacus and Athenio. Hist.
August p. 141.}

As long as the cruelty of Maximin was confined to the illustrious
senators, or even to the bold adventurers, who in the court or
army expose themselves to the caprice of fortune, the body of the
people viewed their sufferings with indifference, or perhaps with
pleasure. But the tyrant’s avarice, stimulated by the insatiate
desires of the soldiers, at length attacked the public property.
Every city of the empire was possessed of an independent revenue,
destined to purchase corn for the multitude, and to supply the
expenses of the games and entertainments. By a single act of
authority, the whole mass of wealth was at once confiscated for
the use of the Imperial treasury. The temples were stripped of
their most valuable offerings of gold and silver, and the statues
of gods, heroes, and emperors, were melted down and coined into
money. These impious orders could not be executed without tumults
and massacres, as in many places the people chose rather to die
in the defence of their altars, than to behold in the midst of
peace their cities exposed to the rapine and cruelty of war. The
soldiers themselves, among whom this sacrilegious plunder was
distributed, received it with a blush; and hardened as they were
in acts of violence, they dreaded the just reproaches of their
friends and relations. Throughout the Roman world a general cry
of indignation was heard, imploring vengeance on the common enemy
of human kind; and at length, by an act of private oppression, a
peaceful and unarmed province was driven into rebellion against
him.\textsuperscript{12}

\pagenote[12]{Herodian, l. vii. p. 238. Zosim. l. i. p. 15.}

The procurator of Africa was a servant worthy of such a master,
who considered the fines and confiscations of the rich as one of
the most fruitful branches of the Imperial revenue. An iniquitous
sentence had been pronounced against some opulent youths of that
country, the execution of which would have stripped them of far
the greater part of their patrimony. In this extremity, a
resolution that must either complete or prevent their ruin, was
dictated by despair. A respite of three days, obtained with
difficulty from the rapacious treasurer, was employed in
collecting from their estates a great number of slaves and
peasants blindly devoted to the commands of their lords, and
armed with the rustic weapons of clubs and axes. The leaders of
the conspiracy, as they were admitted to the audience of the
procurator, stabbed him with the daggers concealed under their
garments, and, by the assistance of their tumultuary train,
seized on the little town of Thysdrus,\textsuperscript{13} and erected the
standard of rebellion against the sovereign of the Roman empire.
They rested their hopes on the hatred of mankind against Maximin,
and they judiciously resolved to oppose to that detested tyrant
an emperor whose mild virtues had already acquired the love and
esteem of the Romans, and whose authority over the province would
give weight and stability to the enterprise. Gordianus, their
proconsul, and the object of their choice, refused, with
unfeigned reluctance, the dangerous honor, and begged with tears,
that they would suffer him to terminate in peace a long and
innocent life, without staining his feeble age with civil blood.
Their menaces compelled him to accept the Imperial purple, his
only refuge, indeed, against the jealous cruelty of Maximin;
since, according to the reasoning of tyrants, those who have been
esteemed worthy of the throne deserve death, and those who
deliberate have already rebelled.\textsuperscript{14}

\pagenote[13]{In the fertile territory of Byzacium, one hundred
and fifty miles to the south of Carthage. This city was
decorated, probably by the Gordians, with the title of colony,
and with a fine amphitheatre, which is still in a very perfect
state. See Intinerar. Wesseling, p. 59; and Shaw’s Travels, p.
117.}

\pagenote[14]{Herodian, l. vii. p. 239. Hist. August. p. 153.}

The family of Gordianus was one of the most illustrious of the
Roman senate. On the father’s side he was descended from the
Gracchi; on his mother’s, from the emperor Trajan. A great estate
enabled him to support the dignity of his birth, and in the
enjoyment of it, he displayed an elegant taste and beneficent
disposition. The palace in Rome, formerly inhabited by the great
Pompey, had been, during several generations, in the possession
of Gordian’s family.\textsuperscript{15} It was distinguished by ancient trophies
of naval victories, and decorated with the works of modern
painting. His villa on the road to Præneste was celebrated for
baths of singular beauty and extent, for three stately rooms of a
hundred feet in length, and for a magnificent portico, supported
by two hundred columns of the four most curious and costly sorts
of marble.\textsuperscript{16} The public shows exhibited at his expense, and in
which the people were entertained with many hundreds of wild
beasts and gladiators,\textsuperscript{17} seem to surpass the fortune of a
subject; and whilst the liberality of other magistrates was
confined to a few solemn festivals at Rome, the magnificence of
Gordian was repeated, when he was ædile, every month in the year,
and extended, during his consulship, to the principal cities of
Italy. He was twice elevated to the last-mentioned dignity, by
Caracalla and by Alexander; for he possessed the uncommon talent
of acquiring the esteem of virtuous princes, without alarming the
jealousy of tyrants. His long life was innocently spent in the
study of letters and the peaceful honors of Rome; and, till he
was named proconsul of Africa by the voice of the senate and the
approbation of Alexander,\textsuperscript{18} he appears prudently to have
declined the command of armies and the government of provinces.\textsuperscript{181}
As long as that emperor lived, Africa was happy under the
administration of his worthy representative: after the barbarous
Maximin had usurped the throne, Gordianus alleviated the miseries
which he was unable to prevent. When he reluctantly accepted the
purple, he was above fourscore years old; a last and valuable
remains of the happy age of the Antonines, whose virtues he
revived in his own conduct, and celebrated in an elegant poem of
thirty books. With the venerable proconsul, his son, who had
accompanied him into Africa as his lieutenant, was likewise
declared emperor. His manners were less pure, but his character
was equally amiable with that of his father. Twenty-two
acknowledged concubines, and a library of sixty-two thousand
volumes, attested the variety of his inclinations; and from the
productions which he left behind him, it appears that the former
as well as the latter were designed for use rather than for
ostentation.\textsuperscript{19} The Roman people acknowledged in the features of
the younger Gordian the resemblance of Scipio Africanus,\textsuperscript{191}
recollected with pleasure that his mother was the granddaughter
of Antoninus Pius, and rested the public hope on those latent
virtues which had hitherto, as they fondly imagined, lain
concealed in the luxurious indolence of private life.

\pagenote[15]{Hist. Aug. p. 152. The celebrated house of Pompey
in carinis was usurped by Marc Antony, and consequently became,
after the Triumvir’s death, a part of the Imperial domain. The
emperor Trajan allowed, and even encouraged, the rich senators to
purchase those magnificent and useless places, (Plin. Panegyric.
c. 50;) and it may seem probable, that, on this occasion,
Pompey’s house came into the possession of Gordian’s
great-grandfather.}

\pagenote[16]{The Claudian, the Numidian, the Carystian, and the
Synnadian. The colors of Roman marbles have been faintly
described and imperfectly distinguished. It appears, however,
that the Carystian was a sea-green, and that the marble of
Synnada was white mixed with oval spots of purple. See Salmasius
ad Hist. August. p. 164.}

\pagenote[17]{Hist. August. p. 151, 152. He sometimes gave five
hundred pair of gladiators, never less than one hundred and
fifty. He once gave for the use of the circus one hundred
Sicilian, and as many Cappæcian Cappadecian horses. The animals
designed for hunting were chiefly bears, boars, bulls, stags,
elks, wild asses, \&c. Elephants and lions seem to have been
appropriated to Imperial magnificence.}

\pagenote[18]{See the original letter, in the Augustan History,
p. 152, which at once shows Alexander’s respect for the authority
of the senate, and his esteem for the proconsul appointed by that
assembly.}

\pagenote[181]{Herodian expressly says that he had administered
many provinces, lib. vii. 10.—W.}

\pagenote[19]{By each of his concubines, the younger Gordian left
three or four children. His literary productions, though less
numerous, were by no means contemptible.}

\pagenote[191]{Not the personal likeness, but the family descent
from the Scipiod.—W.}

As soon as the Gordians had appeased the first tumult of a
popular election, they removed their court to Carthage. They were
received with the acclamations of the Africans, who honored their
virtues, and who, since the visit of Hadrian, had never beheld
the majesty of a Roman emperor. But these vain acclamations
neither strengthened nor confirmed the title of the Gordians.
They were induced by principle, as well as interest, to solicit
the approbation of the senate; and a deputation of the noblest
provincials was sent, without delay, to Rome, to relate and
justify the conduct of their countrymen, who, having long
suffered with patience, were at length resolved to act with
vigor. The letters of the new princes were modest and respectful,
excusing the necessity which had obliged them to accept the
Imperial title; but submitting their election and their fate to
the supreme judgment of the senate.\textsuperscript{20}

\pagenote[20]{Herodian, l. vii. p. 243. Hist. August. p. 144.}

The inclinations of the senate were neither doubtful nor divided.
The birth and noble alliances of the Gordians had intimately
connected them with the most illustrious houses of Rome. Their
fortune had created many dependants in that assembly, their merit
had acquired many friends. Their mild administration opened the
flattering prospect of the restoration, not only of the civil but
even of the republican government. The terror of military
violence, which had first obliged the senate to forget the murder
of Alexander, and to ratify the election of a barbarian peasant,\textsuperscript{21}
now produced a contrary effect, and provoked them to assert
the injured rights of freedom and humanity. The hatred of Maximin
towards the senate was declared and implacable; the tamest
submission had not appeased his fury, the most cautious innocence
would not remove his suspicions; and even the care of their own
safety urged them to share the fortune of an enterprise, of which
(if unsuccessful) they were sure to be the first victims. These
considerations, and perhaps others of a more private nature, were
debated in a previous conference of the consuls and the
magistrates. As soon as their resolution was decided, they
convoked in the temple of Castor the whole body of the senate,
according to an ancient form of secrecy,\textsuperscript{22} calculated to awaken
their attention, and to conceal their decrees. “Conscript
fathers,” said the consul Syllanus, “the two Gordians, both of
consular dignity, the one your proconsul, the other your
lieutenant, have been declared emperors by the general consent of
Africa. Let us return thanks,” he boldly continued, “to the youth
of Thysdrus; let us return thanks to the faithful people of
Carthage, our generous deliverers from a horrid monster—Why do
you hear me thus coolly, thus timidly? Why do you cast those
anxious looks on each other? Why hesitate? Maximin is a public
enemy! may his enmity soon expire with him, and may we long enjoy
the prudence and felicity of Gordian the father, the valor and
constancy of Gordian the son!”\textsuperscript{23} The noble ardor of the consul
revived the languid spirit of the senate. By a unanimous decree,
the election of the Gordians was ratified, Maximin, his son, and
his adherents, were pronounced enemies of their country, and
liberal rewards were offered to whomsoever had the courage and
good fortune to destroy them.

\pagenote[21]{Quod. tamen patres dum periculosum existimant;
inermes armato esistere approbaverunt. —Aurelius Victor.}

\pagenote[22]{Even the servants of the house, the scribes, \&c.,
were excluded, and their office was filled by the senators
themselves. We are obliged to the Augustan History. p. 159, for
preserving this curious example of the old discipline of the
commonwealth.}

\pagenote[23]{This spirited speech, translated from the Augustan
historian, p. 156, seems transcribed by him from the origina
registers of the senate}

During the emperor’s absence, a detachment of the Prætorian
guards remained at Rome, to protect, or rather to command, the
capital. The præfect Vitalianus had signalized his fidelity to
Maximin, by the alacrity with which he had obeyed, and even
prevented the cruel mandates of the tyrant. His death alone could
rescue the authority of the senate, and the lives of the senators
from a state of danger and suspense. Before their resolves had
transpired, a quæstor and some tribunes were commissioned to take
his devoted life. They executed the order with equal boldness and
success; and, with their bloody daggers in their hands, ran
through the streets, proclaiming to the people and the soldiers
the news of the happy revolution. The enthusiasm of liberty was
seconded by the promise of a large donative, in lands and money;
the statues of Maximin were thrown down; the capital of the
empire acknowledged, with transport, the authority of the two
Gordians and the senate;\textsuperscript{24} and the example of Rome was followed
by the rest of Italy.

\pagenote[24]{Herodian, l. vii. p. 244}

A new spirit had arisen in that assembly, whose long patience had
been insulted by wanton despotism and military license. The
senate assumed the reins of government, and, with a calm
intrepidity, prepared to vindicate by arms the cause of freedom.
Among the consular senators recommended by their merit and
services to the favor of the emperor Alexander, it was easy to
select twenty, not unequal to the command of an army, and the
conduct of a war. To these was the defence of Italy intrusted.
Each was appointed to act in his respective department,
authorized to enroll and discipline the Italian youth; and
instructed to fortify the ports and highways, against the
impending invasion of Maximin. A number of deputies, chosen from
the most illustrious of the senatorian and equestrian orders,
were despatched at the same time to the governors of the several
provinces, earnestly conjuring them to fly to the assistance of
their country, and to remind the nations of their ancient ties of
friendship with the Roman senate and people. The general respect
with which these deputies were received, and the zeal of Italy
and the provinces in favor of the senate, sufficiently prove that
the subjects of Maximin were reduced to that uncommon distress,
in which the body of the people has more to fear from oppression
than from resistance. The consciousness of that melancholy truth,
inspires a degree of persevering fury, seldom to be found in
those civil wars which are artificially supported for the benefit
of a few factious and designing leaders.\textsuperscript{25}

\pagenote[25]{Herodian, l. vii. p. 247, l. viii. p. 277. Hist.
August. p 156-158.}

For while the cause of the Gordians was embraced with such
diffusive ardor, the Gordians themselves were no more. The feeble
court of Carthage was alarmed by the rapid approach of
Capelianus, governor of Mauritania, who, with a small band of
veterans, and a fierce host of barbarians, attacked a faithful,
but unwarlike province. The younger Gordian sallied out to meet
the enemy at the head of a few guards, and a numerous
undisciplined multitude, educated in the peaceful luxury of
Carthage. His useless valor served only to procure him an
honorable death on the field of battle. His aged father, whose
reign had not exceeded thirty-six days, put an end to his life on
the first news of the defeat. Carthage, destitute of defence,
opened her gates to the conqueror, and Africa was exposed to the
rapacious cruelty of a slave, obliged to satisfy his unrelenting
master with a large account of blood and treasure.\textsuperscript{26}

\pagenote[26]{Herodian, l. vii. p. 254. Hist. August. p. 150-160.
We may observe, that one month and six days, for the reign of
Gordian, is a just correction of Casaubon and Panvinius, instead
of the absurd reading of one year and six months. See Commentar.
p. 193. Zosimus relates, l. i. p. 17, that the two Gordians
perished by a tempest in the midst of their navigation. A strange
ignorance of history, or a strange abuse of metaphors!}

The fate of the Gordians filled Rome with just but unexpected
terror. The senate, convoked in the temple of Concord, affected
to transact the common business of the day; and seemed to
decline, with trembling anxiety, the consideration of their own
and the public danger. A silent consternation prevailed in the
assembly, till a senator, of the name and family of Trajan,
awakened his brethren from their fatal lethargy. He represented
to them that the choice of cautious, dilatory measures had been
long since out of their power; that Maximin, implacable by
nature, and exasperated by injuries, was advancing towards Italy,
at the head of the military force of the empire; and that their
only remaining alternative was either to meet him bravely in the
field, or tamely to expect the tortures and ignominious death
reserved for unsuccessful rebellion. “We have lost,” continued
he, “two excellent princes; but unless we desert ourselves, the
hopes of the republic have not perished with the Gordians. Many
are the senators whose virtues have deserved, and whose abilities
would sustain, the Imperial dignity. Let us elect two emperors,
one of whom may conduct the war against the public enemy, whilst
his colleague remains at Rome to direct the civil administration.
I cheerfully expose myself to the danger and envy of the
nomination, and give my vote in favor of Maximus and Balbinus.
Ratify my choice, conscript fathers, or appoint in their place,
others more worthy of the empire.” The general apprehension
silenced the whispers of jealousy; the merit of the candidates
was universally acknowledged; and the house resounded with the
sincere acclamations of “Long life and victory to the emperors
Maximus and Balbinus. You are happy in the judgment of the
senate; may the republic be happy under your administration!”\textsuperscript{27}

\pagenote[27]{See the Augustan History, p. 166, from the
registers of the senate; the date is confessedly faulty but the
coincidence of the Apollinatian games enables us to correct it.}

\section{Part \thesection.}

The virtues and the reputation of the new emperors justified the
most sanguine hopes of the Romans. The various nature of their
talents seemed to appropriate to each his peculiar department of
peace and war, without leaving room for jealous emulation.
Balbinus was an admired orator, a poet of distinguished fame, and
a wise magistrate, who had exercised with innocence and applause
the civil jurisdiction in almost all the interior provinces of
the empire. His birth was noble,\textsuperscript{28} his fortune affluent, his
manners liberal and affable. In him the love of pleasure was
corrected by a sense of dignity, nor had the habits of ease
deprived him of a capacity for business. The mind of Maximus was
formed in a rougher mould. By his valor and abilities he had
raised himself from the meanest origin to the first employments
of the state and army. His victories over the Sarmatians and the
Germans, the austerity of his life, and the rigid impartiality of
his justice, while he was a Præfect of the city, commanded the
esteem of a people whose affections were engaged in favor of the
more amiable Balbinus. The two colleagues had both been consuls,
(Balbinus had twice enjoyed that honorable office,) both had been
named among the twenty lieutenants of the senate; and since the
one was sixty and the other seventy-four years old,\textsuperscript{29} they had
both attained the full maturity of age and experience.

\pagenote[28]{He was descended from Cornelius Balbus, a noble
Spaniard, and the adopted son of Theophanes, the Greek historian.
Balbus obtained the freedom of Rome by the favor of Pompey, and
preserved it by the eloquence of Cicero. (See Orat. pro Cornel.
Balbo.) The friendship of Cæsar, (to whom he rendered the most
important secret services in the civil war) raised him to the
consulship and the pontificate, honors never yet possessed by a
stranger. The nephew of this Balbus triumphed over the
Garamantes. See Dictionnaire de Bayle, au mot Balbus, where he
distinguishes the several persons of that name, and rectifies,
with his usual accuracy, the mistakes of former writers
concerning them.}

\pagenote[29]{Zonaras, l. xii. p. 622. But little dependence is
to be had on the authority of a modern Greek, so grossly ignorant
of the history of the third century, that he creates several
imaginary emperors, and confounds those who really existed.}

After the senate had conferred on Maximus and Balbinus an equal
portion of the consular and tribunitian powers, the title of
Fathers of their country, and the joint office of Supreme
Pontiff, they ascended to the Capitol to return thanks to the
gods, protectors of Rome.\textsuperscript{30} The solemn rites of sacrifice were
disturbed by a sedition of the people. The licentious multitude
neither loved the rigid Maximus, nor did they sufficiently fear
the mild and humane Balbinus. Their increasing numbers surrounded
the temple of Jupiter; with obstinate clamors they asserted their
inherent right of consenting to the election of their sovereign;
and demanded, with an apparent moderation, that, besides the two
emperors, chosen by the senate, a third should be added of the
family of the Gordians, as a just return of gratitude to those
princes who had sacrificed their lives for the republic. At the
head of the city-guards, and the youth of the equestrian order,
Maximus and Balbinus attempted to cut their way through the
seditious multitude. The multitude, armed with sticks and stones,
drove them back into the Capitol. It is prudent to yield when the
contest, whatever may be the issue of it, must be fatal to both
parties. A boy, only thirteen years of age, the grandson of the
elder, and nephew\textsuperscript{301} of the younger Gordian, was produced to the
people, invested with the ornaments and title of Cæsar. The
tumult was appeased by this easy condescension; and the two
emperors, as soon as they had been peaceably acknowledged in
Rome, prepared to defend Italy against the common enemy.

\pagenote[30]{Herodian, l. vii. p. 256, supposes that the senate
was at first convoked in the Capitol, and is very eloquent on the
occasion. The Augustar History p. 116, seems much more
authentic.}

\pagenote[301]{According to some, the son.—G.}

Whilst in Rome and Africa, revolutions succeeded each other with
such amazing rapidity, that the mind of Maximin was agitated by
the most furious passions. He is said to have received the news
of the rebellion of the Gordians, and of the decree of the senate
against him, not with the temper of a man, but the rage of a wild
beast; which, as it could not discharge itself on the distant
senate, threatened the life of his son, of his friends, and of
all who ventured to approach his person. The grateful
intelligence of the death of the Gordians was quickly followed by
the assurance that the senate, laying aside all hopes of pardon
or accommodation, had substituted in their room two emperors,
with whose merit he could not be unacquainted. Revenge was the
only consolation left to Maximin, and revenge could only be
obtained by arms. The strength of the legions had been assembled
by Alexander from all parts of the empire. Three successful
campaigns against the Germans and the Sarmatians, had raised
their fame, confirmed their discipline, and even increased their
numbers, by filling the ranks with the flower of the barbarian
youth. The life of Maximin had been spent in war, and the candid
severity of history cannot refuse him the valor of a soldier, or
even the abilities of an experienced general.\textsuperscript{31} It might
naturally be expected, that a prince of such a character, instead
of suffering the rebellion to gain stability by delay, should
immediately have marched from the banks of the Danube to those of
the Tyber, and that his victorious army, instigated by contempt
for the senate, and eager to gather the spoils of Italy, should
have burned with impatience to finish the easy and lucrative
conquest. Yet as far as we can trust to the obscure chronology of
that period,\textsuperscript{32} it appears that the operations of some foreign
war deferred the Italian expedition till the ensuing spring. From
the prudent conduct of Maximin, we may learn that the savage
features of his character have been exaggerated by the pencil of
party, that his passions, however impetuous, submitted to the
force of reason, and that the barbarian possessed something of
the generous spirit of Sylla, who subdued the enemies of Rome
before he suffered himself to revenge his private injuries.\textsuperscript{33}

\pagenote[31]{In Herodian, l. vii. p. 249, and in the Augustan
History, we have three several orations of Maximin to his army,
on the rebellion of Africa and Rome: M. de Tillemont has very
justly observed that they neither agree with each other nor with
truth. Histoire des Empereurs, tom. iii. p. 799.}

\pagenote[32]{The carelessness of the writers of that age, leaves
us in a singular perplexity. 1. We know that Maximus and Balbinus
were killed during the Capitoline games. Herodian, l. viii. p.
285. The authority of Censorinus (de Die Natali, c. 18) enables
us to fix those games with certainty to the year 238, but leaves
us in ignorance of the month or day. 2. The election of Gordian
by the senate is fixed with equal certainty to the 27th of May;
but we are at a loss to discover whether it was in the same or
the preceding year. Tillemont and Muratori, who maintain the two
opposite opinions, bring into the field a desultory troop of
authorities, conjectures and probabilities. The one seems to draw
out, the other to contract the series of events between those
periods, more than can be well reconciled to reason and history.
Yet it is necessary to choose between them. Note: Eckhel has more
recently treated these chronological questions with a perspicuity
which gives great probability to his conclusions. Setting aside
all the historians, whose contradictions are irreconcilable, he
has only consulted the medals, and has arranged the events before
us in the following order:— Maximin, A. U. 990, after having
conquered the Germans, reenters Pannonia, establishes his winter
quarters at Sirmium, and prepares himself to make war against the
people of the North. In the year 991, in the cal ends of January,
commences his fourth tribunate. The Gordians are chosen emperors
in Africa, probably at the beginning of the month of March. The
senate confirms this election with joy, and declares Maximin the
enemy of Rome. Five days after he had heard of this revolt,
Maximin sets out from Sirmium on his march to Italy. These events
took place about the beginning of April; a little after, the
Gordians are slain in Africa by Capellianus, procurator of
Mauritania. The senate, in its alarm, names as emperors Balbus
and Maximus Pupianus, and intrusts the latter with the war
against Maximin. Maximin is stopped on his road near Aquileia, by
the want of provisions, and by the melting of the snows: he
begins the siege of Aquileia at the end of April. Pupianus
assembles his army at Ravenna. Maximin and his son are
assassinated by the soldiers enraged at the resistance of
Aquileia: and this was probably in the middle of May. Pupianus
returns to Rome, and assumes the government with Balbinus; they
are assassinated towards the end of July Gordian the younger
ascends the throne. Eckhel de Doct. Vol vii 295.—G.}

\pagenote[33]{Velleius Paterculus, l. ii. c. 24. The president de
Montesquieu (in his dialogue between Sylla and Eucrates)
expresses the sentiments of the dictator in a spirited, and even
a sublime manner.}

When the troops of Maximin, advancing in excellent order, arrived
at the foot of the Julian Alps, they were terrified by the
silence and desolation that reigned on the frontiers of Italy.
The villages and open towns had been abandoned on their approach
by the inhabitants, the cattle was driven away, the provisions
removed or destroyed, the bridges broken down, nor was any thing
left which could afford either shelter or subsistence to an
invader. Such had been the wise orders of the generals of the
senate: whose design was to protract the war, to ruin the army of
Maximin by the slow operation of famine, and to consume his
strength in the sieges of the principal cities of Italy, which
they had plentifully stored with men and provisions from the
deserted country. Aquileia received and withstood the first shock
of the invasion. The streams that issue from the head of the
Hadriatic Gulf, swelled by the melting of the winter snows,\textsuperscript{34}
opposed an unexpected obstacle to the arms of Maximin. At length,
on a singular bridge, constructed with art and difficulty, of
large hogsheads, he transported his army to the opposite bank,
rooted up the beautiful vineyards in the neighborhood of
Aquileia, demolished the suburbs, and employed the timber of the
buildings in the engines and towers, with which on every side he
attacked the city. The walls, fallen to decay during the security
of a long peace, had been hastily repaired on this sudden
emergency: but the firmest defence of Aquileia consisted in the
constancy of the citizens; all ranks of whom, instead of being
dismayed, were animated by the extreme danger, and their
knowledge of the tyrant’s unrelenting temper. Their courage was
supported and directed by Crispinus and Menophilus, two of the
twenty lieutenants of the senate, who, with a small body of
regular troops, had thrown themselves into the besieged place.
The army of Maximin was repulsed in repeated attacks, his
machines destroyed by showers of artificial fire; and the
generous enthusiasm of the Aquileians was exalted into a
confidence of success, by the opinion that Belenus, their tutelar
deity, combated in person in the defence of his distressed
worshippers.\textsuperscript{35}

\pagenote[34]{Muratori (Annali d’ Italia, tom. ii. p. 294) thinks
the melting of the snows suits better with the months of June or
July, than with those of February. The opinion of a man who
passed his life between the Alps and the Apennines, is
undoubtedly of great weight; yet I observe, 1. That the long
winter, of which Muratori takes advantage, is to be found only in
the Latin version, and not in the Greek text of Herodian. 2. That
the vicissitudes of suns and rains, to which the soldiers of
Maximin were exposed, (Herodian, l. viii. p. 277,) denote the
spring rather than the summer. We may observe, likewise, that
these several streams, as they melted into one, composed the
Timavus, so poetically (in every sense of the word) described by
Virgil. They are about twelve miles to the east of Aquileia. See
Cluver. Italia Antiqua, tom. i. p. 189, \&c.}

\pagenote[35]{Herodian, l. viii. p. 272. The Celtic deity was
supposed to be Apollo, and received under that name the thanks of
the senate. A temple was likewise built to Venus the Bald, in
honor of the women of Aquileia, who had given up their hair to
make ropes for the military engines.}

The emperor Maximus, who had advanced as far as Ravenna, to
secure that important place, and to hasten the military
preparations, beheld the event of the war in the more faithful
mirror of reason and policy. He was too sensible, that a single
town could not resist the persevering efforts of a great army;
and he dreaded, lest the enemy, tired with the obstinate
resistance of Aquileia, should on a sudden relinquish the
fruitless siege, and march directly towards Rome. The fate of the
empire and the cause of freedom must then be committed to the
chance of a battle; and what arms could he oppose to the veteran
legions of the Rhine and Danube? Some troops newly levied among
the generous but enervated youth of Italy; and a body of German
auxiliaries, on whose firmness, in the hour of trial, it was
dangerous to depend. In the midst of these just alarms, the
stroke of domestic conspiracy punished the crimes of Maximin, and
delivered Rome and the senate from the calamities that would
surely have attended the victory of an enraged barbarian.

The people of Aquileia had scarcely experienced any of the common
miseries of a siege; their magazines were plentifully supplied,
and several fountains within the walls assured them of an
inexhaustible resource of fresh water. The soldiers of Maximin
were, on the contrary, exposed to the inclemency of the season,
the contagion of disease, and the horrors of famine. The open
country was ruined, the rivers filled with the slain, and
polluted with blood. A spirit of despair and disaffection began
to diffuse itself among the troops; and as they were cut off from
all intelligence, they easily believed that the whole empire had
embraced the cause of the senate, and that they were left as
devoted victims to perish under the impregnable walls of
Aquileia. The fierce temper of the tyrant was exasperated by
disappointments, which he imputed to the cowardice of his army;
and his wanton and ill-timed cruelty, instead of striking terror,
inspired hatred, and a just desire of revenge. A party of
Prætorian guards, who trembled for their wives and children in
the camp of Alba, near Rome, executed the sentence of the senate.
Maximin, abandoned by his guards, was slain in his tent, with his
son (whom he had associated to the honors of the purple),
Anulinus the præfect, and the principal ministers of his tyranny.\textsuperscript{36}
The sight of their heads, borne on the point of spears,
convinced the citizens of Aquileia that the siege was at an end;
the gates of the city were thrown open, a liberal market was
provided for the hungry troops of Maximin, and the whole army
joined in solemn protestations of fidelity to the senate and the
people of Rome, and to their lawful emperors Maximus and
Balbinus. Such was the deserved fate of a brutal savage,
destitute, as he has generally been represented, of every
sentiment that distinguishes a civilized, or even a human being.
The body was suited to the soul. The stature of Maximin exceeded
the measure of eight feet, and circumstances almost incredible
are related of his matchless strength and appetite.\textsuperscript{37} Had he
lived in a less enlightened age, tradition and poetry might well
have described him as one of those monstrous giants, whose
supernatural power was constantly exerted for the destruction of
mankind.

\pagenote[36]{Herodian, l. viii. p. 279. Hist. August. p. 146.
The duration of Maximin’s reign has not been defined with much
accuracy, except by Eutropius, who allows him three years and a
few days, (l. ix. 1;) we may depend on the integrity of the text,
as the Latin original is checked by the Greek version of
Pæanius.}

\pagenote[37]{Eight Roman feet and one third, which are equal to
above eight English feet, as the two measures are to each other
in the proportion of 967 to 1000. See Graves’s discourse on the
Roman foot. We are told that Maximin could drink in a day an
amphora (or about seven gallons) of wine, and eat thirty or forty
pounds of meat. He could move a loaded wagon, break a horse’s leg
with his fist, crumble stones in his hand, and tear up small
trees by the roots. See his life in the Augustan History.}

It is easier to conceive than to describe the universal joy of
the Roman world on the fall of the tyrant, the news of which is
said to have been carried in four days from Aquileia to Rome. The
return of Maximus was a triumphal procession; his colleague and
young Gordian went out to meet him, and the three princes made
their entry into the capital, attended by the ambassadors of
almost all the cities of Italy, saluted with the splendid
offerings of gratitude and superstition, and received with the
unfeigned acclamations of the senate and people, who persuaded
themselves that a golden age would succeed to an age of iron.\textsuperscript{38}
The conduct of the two emperors corresponded with these
expectations. They administered justice in person; and the rigor
of the one was tempered by the other’s clemency. The oppressive
taxes with which Maximin had loaded the rights of inheritance and
succession, were repealed, or at least moderated. Discipline was
revived, and with the advice of the senate many wise laws were
enacted by their imperial ministers, who endeavored to restore a
civil constitution on the ruins of military tyranny. “What reward
may we expect for delivering Rome from a monster?” was the
question asked by Maximus, in a moment of freedom and confidence.

Balbinus answered it without hesitation—“The love of the senate,
of the people, and of all mankind.” “Alas!” replied his more
penetrating colleague — “alas! I dread the hatred of the soldiers,
and the fatal effects of their resentment.”\textsuperscript{39} His apprehensions
were but too well justified by the event.

\pagenote[38]{See the congratulatory letter of Claudius Julianus,
the consul to the two emperors, in the Augustan History.}

\pagenote[39]{Hist. August. p. 171.}

Whilst Maximus was preparing to defend Italy against the common
foe, Balbinus, who remained at Rome, had been engaged in scenes
of blood and intestine discord. Distrust and jealousy reigned in
the senate; and even in the temples where they assembled, every
senator carried either open or concealed arms. In the midst of
their deliberations, two veterans of the guards, actuated either
by curiosity or a sinister motive, audaciously thrust themselves
into the house, and advanced by degrees beyond the altar of
Victory. Gallicanus, a consular, and Mæcenas, a Prætorian
senator, viewed with indignation their insolent intrusion:
drawing their daggers, they laid the spies (for such they deemed
them) dead at the foot of the altar, and then, advancing to the
door of the senate, imprudently exhorted the multitude to
massacre the Prætorians, as the secret adherents of the tyrant.
Those who escaped the first fury of the tumult took refuge in the
camp, which they defended with superior advantage against the
reiterated attacks of the people, assisted by the numerous bands
of gladiators, the property of opulent nobles. The civil war
lasted many days, with infinite loss and confusion on both sides.
When the pipes were broken that supplied the camp with water, the
Prætorians were reduced to intolerable distress; but in their
turn they made desperate sallies into the city, set fire to a
great number of houses, and filled the streets with the blood of
the inhabitants. The emperor Balbinus attempted, by ineffectual
edicts and precarious truces, to reconcile the factions at Rome.
But their animosity, though smothered for a while, burnt with
redoubled violence. The soldiers, detesting the senate and the
people, despised the weakness of a prince, who wanted either the
spirit or the power to command the obedience of his subjects.\textsuperscript{40}

\pagenote[40]{Herodian, l. viii. p. 258.}

After the tyrant’s death, his formidable army had acknowledged,
from necessity rather than from choice, the authority of Maximus,
who transported himself without delay to the camp before
Aquileia. As soon as he had received their oath of fidelity, he
addressed them in terms full of mildness and moderation;
lamented, rather than arraigned the wild disorders of the times,
and assured the soldiers, that of all their past conduct the
senate would remember only their generous desertion of the
tyrant, and their voluntary return to their duty. Maximus
enforced his exhortations by a liberal donative, purified the
camp by a solemn sacrifice of expiation, and then dismissed the
legions to their several provinces, impressed, as he hoped, with
a lively sense of gratitude and obedience.\textsuperscript{41} But nothing could
reconcile the haughty spirit of the Prætorians. They attended the
emperors on the memorable day of their public entry into Rome;
but amidst the general acclamations, the sullen, dejected
countenance of the guards sufficiently declared that they
considered themselves as the object, rather than the partners, of
the triumph. When the whole body was united in their camp, those
who had served under Maximin, and those who had remained at Rome,
insensibly communicated to each other their complaints and
apprehensions. The emperors chosen by the army had perished with
ignominy; those elected by the senate were seated on the throne.\textsuperscript{42}
The long discord between the civil and military powers was
decided by a war, in which the former had obtained a complete
victory. The soldiers must now learn a new doctrine of submission
to the senate; and whatever clemency was affected by that politic
assembly, they dreaded a slow revenge, colored by the name of
discipline, and justified by fair pretences of the public good.
But their fate was still in their own hands; and if they had
courage to despise the vain terrors of an impotent republic, it
was easy to convince the world, that those who were masters of
the arms, were masters of the authority, of the state.

\pagenote[41]{Herodian, l. viii. p. 213.}

\pagenote[42]{The observation had been made imprudently enough in
the acclamations of the senate, and with regard to the soldiers
it carried the appearance of a wanton insult. Hist. August. p.
170.}

When the senate elected two princes, it is probable that, besides
the declared reason of providing for the various emergencies of
peace and war, they were actuated by the secret desire of
weakening by division the despotism of the supreme magistrate.
Their policy was effectual, but it proved fatal both to their
emperors and to themselves. The jealousy of power was soon
exasperated by the difference of character. Maximus despised
Balbinus as a luxurious noble, and was in his turn disdained by
his colleague as an obscure soldier. Their silent discord was
understood rather than seen;\textsuperscript{43} but the mutual consciousness
prevented them from uniting in any vigorous measures of defence
against their common enemies of the Prætorian camp. The whole
city was employed in the Capitoline games, and the emperors were
left almost alone in the palace. On a sudden, they were alarmed
by the approach of a troop of desperate assassins. Ignorant of
each other’s situation or designs (for they already occupied very
distant apartments), afraid to give or to receive assistance,
they wasted the important moments in idle debates and fruitless
recriminations. The arrival of the guards put an end to the vain
strife. They seized on these emperors of the senate, for such
they called them with malicious contempt, stripped them of their
garments, and dragged them in insolent triumph through the
streets of Rome, with the design of inflicting a slow and cruel
death on these unfortunate princes. The fear of a rescue from the
faithful Germans of the Imperial guards shortened their tortures;
and their bodies, mangled with a thousand wounds, were left
exposed to the insults or to the pity of the populace.\textsuperscript{44}

\pagenote[43]{Discordiæ tacitæ, et quæ intelligerentur potius
quam viderentur. \textit{Hist. August}. p. 170. This well-chosen
expression is probably stolen from some better writer.}

\pagenote[44]{Herodian, l. viii. p. 287, 288.}

In the space of a few months, six princes had been cut off by the
sword. Gordian, who had already received the title of Cæsar, was
the only person that occurred to the soldiers as proper to fill
the vacant throne.\textsuperscript{45} They carried him to the camp, and
unanimously saluted him Augustus and Emperor. His name was dear
to the senate and people; his tender age promised a long impunity
of military license; and the submission of Rome and the provinces
to the choice of the Prætorian guards saved the republic, at the
expense indeed of its freedom and dignity, from the horrors of a
new civil war in the heart of the capital.\textsuperscript{46}

\pagenote[45]{Quia non alius erat in præsenti, is the expression
of the Augustan History.}

\pagenote[46]{Quintus Curtius (l. x. c. 9,) pays an elegant
compliment to the emperor of the day, for having, by his happy
accession, extinguished so many firebrands, sheathed so many
swords, and put an end to the evils of a divided government.
After weighing with attention every word of the passage, I am of
opinion, that it suits better with the elevation of Gordian, than
with any other period of the Roman history. In that case, it may
serve to decide the age of Quintus Curtius. Those who place him
under the first Cæsars, argue from the purity of his style but
are embarrassed by the silence of Quintilian, in his accurate
list of Roman historians. * Note: This conjecture of Gibbon is
without foundation. Many passages in the work of Quintus Curtius
clearly place him at an earlier period. Thus, in speaking of the
Parthians, he says, Hinc in Parthicum perventum est, tunc
ignobilem gentem: nunc caput omnium qui post Euphratem et Tigrim
amnes siti Rubro mari terminantur. The Parthian empire had this
extent only in the first age of the vulgar æra: to that age,
therefore, must be assigned the date of Quintus Curtius. Although
the critics (says M. de Sainte Croix) have multiplied conjectures
on this subject, most of them have ended by adopting the opinion
which places Quintus Curtius under the reign of Claudius. See
Just. Lips. ad Ann. Tac. ii. 20. Michel le Tellier Præf. in Curt.
Tillemont Hist. des Emp. i. p. 251. Du Bos Reflections sur la
Poesie, 2d Partie. Tiraboschi Storia della, Lett. Ital. ii. 149.
Examen. crit. des Historiens d’Alexandre, 2d ed. p. 104, 849,
850.—G. ——This interminable question seems as much perplexed as
ever. The first argument of M. Guizot is a strong one, except
that Parthian is often used by later writers for Persian.
Cunzius, in his preface to an edition published at Helmstadt,
(1802,) maintains the opinion of Bagnolo, which assigns Q.
Curtius to the time of Constantine the Great. Schmieder, in his
edit. Gotting. 1803, sums up in this sentence, ætatem Curtii
ignorari pala mest.—M.}

As the third Gordian was only nineteen years of age at the time
of his death, the history of his life, were it known to us with
greater accuracy than it really is, would contain little more
than the account of his education, and the conduct of the
ministers, who by turns abused or guided the simplicity of his
unexperienced youth. Immediately after his accession, he fell
into the hands of his mother’s eunuchs, that pernicious vermin of
the East, who, since the days of Elagabalus, had infested the
Roman palace. By the artful conspiracy of these wretches, an
impenetrable veil was drawn between an innocent prince and his
oppressed subjects, the virtuous disposition of Gordian was
deceived, and the honors of the empire sold without his
knowledge, though in a very public manner, to the most worthless
of mankind. We are ignorant by what fortunate accident the
emperor escaped from this ignominious slavery, and devolved his
confidence on a minister, whose wise counsels had no object
except the glory of his sovereign and the happiness of the
people. It should seem that love and learning introduced
Misitheus to the favor of Gordian. The young prince married the
daughter of his master of rhetoric, and promoted his
father-in-law to the first offices of the empire. Two admirable
letters that passed between them are still extant. The minister,
with the conscious dignity of virtue, congratulates Gordian that
he is delivered from the tyranny of the eunuchs,\textsuperscript{47} and still
more that he is sensible of his deliverance. The emperor
acknowledges, with an amiable confusion, the errors of his past
conduct; and laments, with singular propriety, the misfortune of
a monarch from whom a venal tribe of courtiers perpetually labor
to conceal the truth.\textsuperscript{48}

\pagenote[47]{Hist. August. p. 161. From some hints in the two
letters, I should expect that the eunuchs were not expelled the
palace without some degree of gentle violence, and that the young
Gordian rather approved of, than consented to, their disgrace.}

\pagenote[48]{Duxit uxorem filiam Misithei, quem causa eloquentiæ
dignum parentela sua putavit; et præfectum statim fecit; post
quod, non puerile jam et contemptibile videbatur imperium.}

The life of Misitheus had been spent in the profession of
letters, not of arms; yet such was the versatile genius of that
great man, that, when he was appointed Prætorian Præfect, he
discharged the military duties of his place with vigor and
ability. The Persians had invaded Mesopotamia, and threatened
Antioch. By the persuasion of his father-in-law, the young
emperor quitted the luxury of Rome, opened, for the last time
recorded in history, the temple of Janus, and marched in person
into the East. On his approach, with a great army, the Persians
withdrew their garrisons from the cities which they had already
taken, and retired from the Euphrates to the Tigris. Gordian
enjoyed the pleasure of announcing to the senate the first
success of his arms, which he ascribed, with a becoming modesty
and gratitude, to the wisdom of his father and Præfect. During
the whole expedition, Misitheus watched over the safety and
discipline of the army; whilst he prevented their dangerous
murmurs by maintaining a regular plenty in the camp, and by
establishing ample magazines of vinegar, bacon, straw, barley,
and wheat in all the cities of the frontier.\textsuperscript{49} But the
prosperity of Gordian expired with Misitheus, who died of a flux,
not without very strong suspicions of poison. Philip, his
successor in the præfecture, was an Arab by birth, and
consequently, in the earlier part of his life, a robber by
profession. His rise from so obscure a station to the first
dignities of the empire, seems to prove that he was a bold and
able leader. But his boldness prompted him to aspire to the
throne, and his abilities were employed to supplant, not to
serve, his indulgent master. The minds of the soldiers were
irritated by an artificial scarcity, created by his contrivance
in the camp; and the distress of the army was attributed to the
youth and incapacity of the prince. It is not in our power to
trace the successive steps of the secret conspiracy and open
sedition, which were at length fatal to Gordian. A sepulchral
monument was erected to his memory on the spot\textsuperscript{50} where he was
killed, near the conflux of the Euphrates with the little river
Aboras.\textsuperscript{51} The fortunate Philip, raised to the empire by the
votes of the soldiers, found a ready obedience from the senate
and the provinces.\textsuperscript{52}

\pagenote[49]{Hist. August. p. 162. Aurelius Victor. Porphyrius
in Vit Plotin. ap. Fabricium, Biblioth. Græc. l. iv. c. 36. The
philosopher Plotinus accompanied the army, prompted by the love
of knowledge, and by the hope of penetrating as far as India.}

\pagenote[50]{About twenty miles from the little town of
Circesium, on the frontier of the two empires. * Note: Now
Kerkesia; placed in the angle formed by the juncture of the
Chaboras, or al Khabour, with the Euphrates. This situation
appeared advantageous to Diocletian, that he raised
fortifications to make it the but wark of the empire on the side
of Mesopotamia. D’Anville. Geog. Anc. ii. 196.—G. It is the
Carchemish of the Old Testament, 2 Chron. xxxv. 20. ler. xlvi.
2.—M.}

\pagenote[51]{The inscription (which contained a very singular
pun) was erased by the order of Licinius, who claimed some degree
of relationship to Philip, (Hist. August. p. 166;) but the
tumulus, or mound of earth which formed the sepulchre, still
subsisted in the time of Julian. See Ammian Marcellin. xxiii. 5.}

\pagenote[52]{Aurelius Victor. Eutrop. ix. 2. Orosius, vii. 20.
Ammianus Marcellinus, xxiii. 5. Zosimus, l. i. p. 19. Philip, who
was a native of Bostra, was about forty years of age. * Note: Now
Bosra. It was once the metropolis of a province named Arabia, and
the chief city of Auranitis, of which the name is preserved in
Beled Hauran, the limits of which meet the desert. D’Anville.
Geog. Anc. ii. 188. According to Victor, (in Cæsar.,) Philip was
a native of Tracbonitis another province of Arabia.—G.}

We cannot forbear transcribing the ingenious, though somewhat
fanciful description, which a celebrated writer of our own times
has traced of the military government of the Roman empire. What
in that age was called the Roman empire, was only an irregular
republic, not unlike the aristocracy\textsuperscript{53} of Algiers,\textsuperscript{54} where the
militia, possessed of the sovereignty, creates and deposes a
magistrate, who is styled a Dey. Perhaps, indeed, it may be laid
down as a general rule, that a military government is, in some
respects, more republican than monarchical. Nor can it be said
that the soldiers only partook of the government by their
disobedience and rebellions. The speeches made to them by the
emperors, were they not at length of the same nature as those
formerly pronounced to the people by the consuls and the
tribunes? And although the armies had no regular place or forms
of assembly; though their debates were short, their action
sudden, and their resolves seldom the result of cool reflection,
did they not dispose, with absolute sway, of the public fortune?
What was the emperor, except the minister of a violent
government, elected for the private benefit of the soldiers?

\pagenote[53]{Can the epithet of Aristocracy be applied, with any
propriety, to the government of Algiers? Every military
government floats between two extremes of absolute monarchy and
wild democracy.}

\pagenote[54]{The military republic of the Mamelukes in Egypt
would have afforded M. de Montesquieu (see Considerations sur la
Grandeur et la Decadence des Romains, c. 16) a juster and more
noble parallel.}

“When the army had elected Philip, who was Prætorian præfect to
the third Gordian, the latter demanded that he might remain sole
emperor; he was unable to obtain it. He requested that the power
might be equally divided between them; the army would not listen
to his speech. He consented to be degraded to the rank of Cæsar;
the favor was refused him. He desired, at least, he might be
appointed Prætorian præfect; his prayer was rejected. Finally, he
pleaded for his life. The army, in these several judgments,
exercised the supreme magistracy.” According to the historian,
whose doubtful narrative the President De Montesquieu has
adopted, Philip, who, during the whole transaction, had preserved
a sullen silence, was inclined to spare the innocent life of his
benefactor; till, recollecting that his innocence might excite a
dangerous compassion in the Roman world, he commanded, without
regard to his suppliant cries, that he should be seized,
stripped, and led away to instant death. After a moment’s pause,
the inhuman sentence was executed.\textsuperscript{55}

\pagenote[55]{The Augustan History (p. 163, 164) cannot, in this
instance, be reconciled with itself or with probability. How
could Philip condemn his predecessor, and yet consecrate his
memory? How could he order his public execution, and yet, in his
letters to the senate, exculpate himself from the guilt of his
death? Philip, though an ambitious usurper, was by no means a mad
tyrant. Some chronological difficulties have likewise been
discovered by the nice eyes of Tillemont and Muratori, in this
supposed association of Philip to the empire. * Note: Wenck
endeavors to reconcile these discrepancies. He supposes that
Gordian was led away, and died a natural death in prison. This is
directly contrary to the statement of Capitolinus and of Zosimus,
whom he adduces in support of his theory. He is more successful
in his precedents of usurpers deifying the victims of their
ambition. Sit divus, dummodo non sit vivus.—M.}

\section{Part \thesection.}

On his return from the East to Rome, Philip, desirous of
obliterating the memory of his crimes, and of captivating the
affections of the people, solemnized the secular games with
infinite pomp and magnificence. Since their institution or
revival by Augustus,\textsuperscript{56} they had been celebrated by Claudius, by
Domitian, and by Severus, and were now renewed the fifth time, on
the accomplishment of the full period of a thousand years from
the foundation of Rome. Every circumstance of the secular games
was skillfully adapted to inspire the superstitious mind with
deep and solemn reverence. The long interval between them\textsuperscript{57}
exceeded the term of human life; and as none of the spectators
had already seen them, none could flatter themselves with the
expectation of beholding them a second time. The mystic
sacrifices were performed, during three nights, on the banks of
the Tyber; and the Campus Martius resounded with music and
dances, and was illuminated with innumerable lamps and torches.
Slaves and strangers were excluded from any participation in
these national ceremonies. A chorus of twenty-seven youths, and
as many virgins, of noble families, and whose parents were both
alive, implored the propitious gods in favor of the present, and
for the hope of the rising generation; requesting, in religious
hymns, that according to the faith of their ancient oracles, they
would still maintain the virtue, the felicity, and the empire of
the Roman people. The magnificence of Philip’s shows and
entertainments dazzled the eyes of the multitude. The devout were
employed in the rites of superstition, whilst the reflecting few
revolved in their anxious minds the past history and the future
fate of the empire.\textsuperscript{58}

\pagenote[56]{The account of the last supposed celebration,
though in an enlightened period of history, was so very doubtful
and obscure, that the alternative seems not doubtful. When the
popish jubilees, the copy of the secular games, were invented by
Boniface VII., the crafty pope pretended that he only revived an
ancient institution. See M. le Chais, Lettres sur les Jubiles.}

\pagenote[57]{Either of a hundred or a hundred and ten years.
Varro and Livy adopted the former opinion, but the infallible
authority of the Sybil consecrated the latter, (Censorinus de Die
Natal. c. 17.) The emperors Claudius and Philip, however, did not
treat the oracle with implicit respect.}

\pagenote[58]{The idea of the secular games is best understood
from the poem of Horace, and the description of Zosimus, 1. l.
ii. p. 167, \&c.}

Since Romulus, with a small band of shepherds
and outlaws, fortified himself on the hills near the Tyber, ten
centuries had already elapsed.\textsuperscript{59} During the four first ages, the
Romans, in the laborious school of poverty, had acquired the
virtues of war and government: by the vigorous exertion of those
virtues, and by the assistance of fortune, they had obtained, in
the course of the three succeeding centuries, an absolute empire
over many countries of Europe, Asia, and Africa. The last three
hundred years had been consumed in apparent prosperity and
internal decline. The nation of soldiers, magistrates, and
legislators, who composed the thirty-five tribes of the Roman
people, were dissolved into the common mass of mankind, and
confounded with the millions of servile provincials, who had
received the name, without adopting the spirit, of Romans. A
mercenary army, levied among the subjects and barbarians of the
frontier, was the only order of men who preserved and abused
their independence. By their tumultuary election, a Syrian, a
Goth, or an Arab, was exalted to the throne of Rome, and invested
with despotic power over the conquests and over the country of
the Scipios.

\pagenote[59]{The received calculation of Varro assigns to the
foundation of Rome an æra that corresponds with the 754th year
before Christ. But so little is the chronology of Rome to be
depended on, in the more early ages, that Sir Isaac Newton has
brought the same event as low as the year 627 (Compare Niebuhr
vol. i. p. 271.—M.)}

The limits of the Roman empire still extended from the Western
Ocean to the Tigris, and from Mount Atlas to the Rhine and the
Danube. To the undiscerning eye of the vulgar, Philip appeared a
monarch no less powerful than Hadrian or Augustus had formerly
been. The form was still the same, but the animating health and
vigor were fled. The industry of the people was discouraged and
exhausted by a long series of oppression. The discipline of the
legions, which alone, after the extinction of every other virtue,
had propped the greatness of the state, was corrupted by the
ambition, or relaxed by the weakness, of the emperors. The
strength of the frontiers, which had always consisted in arms
rather than in fortifications, was insensibly undermined; and the
fairest provinces were left exposed to the rapaciousness or
ambition of the barbarians, who soon discovered the decline of
the Roman empire.

