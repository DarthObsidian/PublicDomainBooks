\chapter{Six Emperors At The Same Time, Reunion Of The Empire.}
\section{Part \thesection.}

\textit{Troubles After The Abdication Of Diocletian. — Death Of
Constantius. — Elevation Of Constantine And Maxen Tius. — Six
Emperors At The Same Time.—Death Of Maximian And
Galerius. — Victories Of Constantine Over Maxentius And
Licinus. — Reunion Of The Empire Under The Authority Of Constantine.}
\vspace{\onelineskip}

The balance of power established by Diocletian subsisted no
longer than while it was sustained by the firm and dexterous hand
of the founder. It required such a fortunate mixture of different
tempers and abilities as could scarcely be found or even expected
a second time; two emperors without jealousy, two Cæsars without
ambition, and the same general interest invariably pursued by
four independent princes. The abdication of Diocletian and
Maximian was succeeded by eighteen years of discord and
confusion. The empire was afflicted by five civil wars; and the
remainder of the time was not so much a state of tranquillity as
a suspension of arms between several hostile monarchs, who,
viewing each other with an eye of fear and hatred, strove to
increase their respective forces at the expense of their
subjects.

As soon as Diocletian and Maximian had resigned the purple, their
station, according to the rules of the new constitution, was
filled by the two Cæsars, Constantius and Galerius, who
immediately assumed the title of Augustus.\textsuperscript{1}

\pagenote[1]{M. de Montesquieu (Considerations sur la Grandeur et
La Decadence des Romains, c. 17) supposes, on the authority of
Orosius and Eusebius, that, on this occasion, the empire, for the
first time, was really divided into two parts. It is difficult,
however, to discover in what respect the plan of Galerius
differed from that of Diocletian.}

The honors of seniority and precedence were allowed to the former
of those princes, and he continued under a new appellation to
administer his ancient department of Gaul, Spain, and Britain.

The government of those ample provinces was sufficient to
exercise his talents and to satisfy his ambition. Clemency,
temperance, and moderation, distinguished the amiable character
of Constantius, and his fortunate subjects had frequently
occasion to compare the virtues of their sovereign with the
passions of Maximian, and even with the arts of Diocletian.\textsuperscript{2}
Instead of imitating their eastern pride and magnificence,
Constantius preserved the modesty of a Roman prince. He declared,
with unaffected sincerity, that his most valued treasure was in
the hearts of his people, and that, whenever the dignity of the
throne, or the danger of the state, required any extraordinary
supply, he could depend with confidence on their gratitude and
liberality.\textsuperscript{3} The provincials of Gaul, Spain, and Britain,
sensible of his worth, and of their own happiness, reflected with
anxiety on the declining health of the emperor Constantius, and
the tender age of his numerous family, the issue of his second
marriage with the daughter of Maximian.

\pagenote[2]{Hic non modo amabilis, sed etiam venerabilis Gallis
fuit; præcipuc quod Diocletiani suspectam prudentiam, et
Maximiani sanguinariam violentiam imperio ejus evaserant. Eutrop.
Breviar. x. i.}

\pagenote[3]{Divitiis Provincialium (mel. provinciarum) ac
privatorum studens, fisci commoda non admodum affectans;
ducensque melius publicas opes a privatis haberi, quam intra unum
claustrum reservari. Id. ibid. He carried this maxim so far, that
whenever he gave an entertainment, he was obliged to borrow a
service of plate.}

The stern temper of Galerius was cast in a very different mould;
and while he commanded the esteem of his subjects, he seldom
condescended to solicit their affections. His fame in arms, and,
above all, the success of the Persian war, had elated his haughty
mind, which was naturally impatient of a superior, or even of an
equal. If it were possible to rely on the partial testimony of an
injudicious writer, we might ascribe the abdication of Diocletian
to the menaces of Galerius, and relate the particulars of a
\textit{private} conversation between the two princes, in which the
former discovered as much pusillanimity as the latter displayed
ingratitude and arrogance.\textsuperscript{4} But these obscure anecdotes are
sufficiently refuted by an impartial view of the character and
conduct of Diocletian. Whatever might otherwise have been his
intentions, if he had apprehended any danger from the violence of
Galerius, his good sense would have instructed him to prevent the
ignominious contest; and as he had held the sceptre with glory,
he would have resigned it without disgrace.

\pagenote[4]{Lactantius de Mort. Persecutor. c. 18. Were the
particulars of this conference more consistent with truth and
decency, we might still ask how they came to the knowledge of an
obscure rhetorician. But there are many historians who put us in
mind of the admirable saying of the great Conde to Cardinal de
Retz: “Ces coquins nous font parlor et agir, comme ils auroient
fait eux-memes a notre place.” * Note: This attack upon
Lactantius is unfounded. Lactantius was so far from having been
an obscure rhetorician, that he had taught rhetoric publicly, and
with the greatest success, first in Africa, and afterwards in
Nicomedia. His reputation obtained him the esteem of Constantine,
who invited him to his court, and intrusted to him the education
of his son Crispus. The facts which he relates took place during
his own time; he cannot be accused of dishonesty or imposture.
Satis me vixisse arbitrabor et officium hominis implesse si labor
meus aliquos homines, ab erroribus iberatos, ad iter coeleste
direxerit. De Opif. Dei, cap. 20. The eloquence of Lactantius has
caused him to be called the Christian Cicero. Annon Gent.—G.
——Yet no unprejudiced person can read this coarse and particular
private conversation of the two emperors, without assenting to
the justice of Gibbon’s severe sentence. But the authorship of
the treatise is by no means certain. The fame of Lactantius for
eloquence as well as for truth, would suffer no loss if it should
be adjudged to some more “obscure rhetorician.” Manso, in his
Leben Constantins des Grossen, concurs on this point with Gibbon
Beylage, iv. —M.}

After the elevation of Constantius and Galerius to the rank of
\textit{Augusti}, two new \textit{Cæsars} were required to supply their place,
and to complete the system of the Imperial government. Diocletian
was sincerely desirous of withdrawing himself from the world; he
considered Galerius, who had married his daughter, as the firmest
support of his family and of the empire; and he consented,
without reluctance, that his successor should assume the merit as
well as the envy of the important nomination. It was fixed
without consulting the interest or inclination of the princes of
the West. Each of them had a son who was arrived at the age of
manhood, and who might have been deemed the most natural
candidates for the vacant honor. But the impotent resentment of
Maximian was no longer to be dreaded; and the moderate
Constantius, though he might despise the dangers, was humanely
apprehensive of the calamities, of civil war. The two persons
whom Galerius promoted to the rank of Cæsar were much better
suited to serve the views of his ambition; and their principal
recommendation seems to have consisted in the want of merit or
personal consequence. The first of these was Daza, or, as he was
afterwards called, Maximin, whose mother was the sister of
Galerius. The unexperienced youth still betrayed, by his manners
and language, his rustic education, when, to his own
astonishment, as well as that of the world, he was invested by
Diocletian with the purple, exalted to the dignity of Cæsar, and
intrusted with the sovereign command of Egypt and Syria.\textsuperscript{5} At the
same time, Severus, a faithful servant, addicted to pleasure, but
not incapable of business, was sent to Milan, to receive, from
the reluctant hands of Maximian, the Cæsarian ornaments, and the
possession of Italy and Africa. According to the forms of the
constitution, Severus acknowledged the supremacy of the western
emperor; but he was absolutely devoted to the commands of his
benefactor Galerius, who, reserving to himself the intermediate
countries from the confines of Italy to those of Syria, firmly
established his power over three fourths of the monarchy. In the
full confidence that the approaching death of Constantius would
leave him sole master of the Roman world, we are assured that he
had arranged in his mind a long succession of future princes, and
that he meditated his own retreat from public life, after he
should have accomplished a glorious reign of about twenty years. \textsuperscript{6} \textsuperscript{7}

\pagenote[5]{Sublatus nuper a pecoribus et silvis (says
Lactantius de M. P. c. 19) statim Scutarius, continuo Protector,
mox Tribunus, postridie Cæsar, accepit Orientem. Aurelius Victor
is too liberal in giving him the whole portion of Diocletian.}

\pagenote[6]{His diligence and fidelity are acknowledged even by
Lactantius, de M. P. c. 18.}

\pagenote[7]{These schemes, however, rest only on the very
doubtful authority of Lactantius de M. P. c. 20.}

But within less than eighteen months, two unexpected revolutions
overturned the ambitious schemes of Galerius. The hopes of
uniting the western provinces to his empire were disappointed by
the elevation of Constantine, whilst Italy and Africa were lost
by the successful revolt of Maxentius.

I. The fame of Constantine has rendered posterity attentive to
the most minute circumstances of his life and actions. The place
of his birth, as well as the condition of his mother Helena, have
been the subject, not only of literary, but of national disputes.
Notwithstanding the recent tradition, which assigns for her
father a British king,\textsuperscript{8} we are obliged to confess, that Helena
was the daughter of an innkeeper; but at the same time, we may
defend the legality of her marriage, against those who have
represented her as the concubine of Constantius.\textsuperscript{9} The great
Constantine was most probably born at Naissus, in Dacia;\textsuperscript{10} and
it is not surprising that, in a family and province distinguished
only by the profession of arms, the youth should discover very
little inclination to improve his mind by the acquisition of
knowledge.\textsuperscript{11} He was about eighteen years of age when his father
was promoted to the rank of Cæsar; but that fortunate event was
attended with his mother’s divorce; and the splendor of an
Imperial alliance reduced the son of Helena to a state of
disgrace and humiliation. Instead of following Constantius in the
West, he remained in the service of Diocletian, signalized his
valor in the wars of Egypt and Persia, and gradually rose to the
honorable station of a tribune of the first order. The figure of
Constantine was tall and majestic; he was dexterous in all his
exercises, intrepid in war, affable in peace; in his whole
conduct, the active spirit of youth was tempered by habitual
prudence; and while his mind was engrossed by ambition, he
appeared cold and insensible to the allurements of pleasure. The
favor of the people and soldiers, who had named him as a worthy
candidate for the rank of Cæsar, served only to exasperate the
jealousy of Galerius; and though prudence might restrain him from
exercising any open violence, an absolute monarch is seldom at a
loss how to execute a sure and secret revenge.\textsuperscript{12} Every hour
increased the danger of Constantine, and the anxiety of his
father, who, by repeated letters, expressed the warmest desire of
embracing his son. For some time the policy of Galerius supplied
him with delays and excuses; but it was impossible long to refuse
so natural a request of his associate, without maintaining his
refusal by arms. The permission of the journey was reluctantly
granted, and whatever precautions the emperor might have taken to
intercept a return, the consequences of which he, with so much
reason, apprehended, they were effectually disappointed by the
incredible diligence of Constantine.\textsuperscript{13} Leaving the palace of
Nicomedia in the night, he travelled post through Bithynia,
Thrace, Dacia, Pannonia, Italy, and Gaul, and, amidst the joyful
acclamations of the people, reached the port of Boulogne in the
very moment when his father was preparing to embark for Britain. \textsuperscript{14}

\pagenote[8]{This tradition, unknown to the contemporaries of
Constantine was invented in the darkness of monestaries, was
embellished by Jeffrey of Monmouth, and the writers of the xiith
century, has been defended by our antiquarians of the last age,
and is seriously related in the ponderous History of England,
compiled by Mr. Carte, (vol. i. p. 147.) He transports, however,
the kingdom of Coil, the imaginary father of Helena, from Essex
to the wall of Antoninus.}

\pagenote[9]{Eutropius (x. 2) expresses, in a few words, the real
truth, and the occasion of the error “ex obscuriori matrimonio
ejus filius.” Zosimus (l. ii. p. 78) eagerly seized the most
unfavorable report, and is followed by Orosius, (vii. 25,) whose
authority is oddly enough overlooked by the indefatigable, but
partial Tillemont. By insisting on the divorce of Helena,
Diocletian acknowledged her marriage.}

\pagenote[10]{There are three opinions with regard to the place
of Constantine’s birth. 1. Our English antiquarians were used to
dwell with rapture on the words of his panegyrist, “Britannias
illic oriendo nobiles fecisti.” But this celebrated passage may
be referred with as much propriety to the accession, as to the
nativity of Constantine. 2. Some of the modern Greeks have
ascribed the honor of his birth to Drepanum, a town on the Gulf
of Nicomedia, (Cellarius, tom. ii. p. 174,) which Constantine
dignified with the name of Helenopolis, and Justinian adorned
with many splendid buildings, (Procop. de Edificiis, v. 2.) It is
indeed probable enough, that Helena’s father kept an inn at
Drepanum, and that Constantius might lodge there when he returned
from a Persian embassy, in the reign of Aurelian. But in the
wandering life of a soldier, the place of his marriage, and the
places where his children are born, have very little connection
with each other. 3. The claim of Naissus is supported by the
anonymous writer, published at the end of Ammianus, p. 710, and
who in general copied very good materials; and it is confirmed by
Julius Firmicus, (de Astrologia, l. i. c. 4,) who flourished
under the reign of Constantine himself. Some objections have been
raised against the integrity of the text, and the application of
the passage of Firmicus but the former is established by the best
Mss., and the latter is very ably defended by Lipsius de
Magnitudine Romana, l. iv. c. 11, et Supplement.}

\pagenote[11]{Literis minus instructus. Anonym. ad Ammian. p.
710.}

\pagenote[12]{Galerius, or perhaps his own courage, exposed him
to single combat with a Sarmatian, (Anonym. p. 710,) and with a
monstrous lion. See Praxagoras apud Photium, p. 63. Praxagoras,
an Athenian philosopher, had written a life of Constantine in two
books, which are now lost. He was a contemporary.}

\pagenote[13]{Zosimus, l. ii. p. 78, 79. Lactantius de M. P. c.
24. The former tells a very foolish story, that Constantine
caused all the post-horses which he had used to be hamstrung.
Such a bloody execution, without preventing a pursuit, would have
scattered suspicions, and might have stopped his journey. * Note:
Zosimus is not the only writer who tells this story. The younger
Victor confirms it. Ad frustrandos insequentes, publica jumenta,
quaqua iter ageret, interficiens. Aurelius Victor de Cæsar says
the same thing, G. as also the Anonymus Valesii.— M. ——Manso,
(Leben Constantins,) p. 18, observes that the story has been
exaggerated; he took this precaution during the first stage of
his journey.—M.}

\pagenote[14]{Anonym. p. 710. Panegyr. Veter. vii. 4. But
Zosimus, l. ii. p. 79, Eusebius de Vit. Constant. l. i. c. 21,
and Lactantius de M. P. c. 24. suppose, with less accuracy, that
he found his father on his death-bed.}

The British expedition, and an easy victory over the barbarians
of Caledonia, were the last exploits of the reign of Constantius.
He ended his life in the Imperial palace of York, fifteen months
after he had received the title of Augustus, and almost fourteen
years and a half after he had been promoted to the rank of Cæsar.
His death was immediately succeeded by the elevation of
Constantine. The ideas of inheritance and succession are so very
familiar, that the generality of mankind consider them as founded
not only in reason but in nature itself. Our imagination readily
transfers the same principles from private property to public
dominion: and whenever a virtuous father leaves behind him a son
whose merit seems to justify the esteem, or even the hopes, of
the people, the joint influence of prejudice and of affection
operates with irresistible weight. The flower of the western
armies had followed Constantius into Britain, and the national
troops were reënforced by a numerous body of Alemanni, who obeyed
the orders of Crocus, one of their hereditary chieftains.\textsuperscript{15} The
opinion of their own importance, and the assurance that Britain,
Gaul, and Spain would acquiesce in their nomination, were
diligently inculcated to the legions by the adherents of
Constantine. The soldiers were asked, whether they could hesitate
a moment between the honor of placing at their head the worthy
son of their beloved emperor, and the ignominy of tamely
expecting the arrival of some obscure stranger, on whom it might
please the sovereign of Asia to bestow the armies and provinces
of the West. It was insinuated to them, that gratitude and
liberality held a distinguished place among the virtues of
Constantine; nor did that artful prince show himself to the
troops, till they were prepared to salute him with the names of
Augustus and Emperor. The throne was the object of his desires;
and had he been less actuated by ambition, it was his only means
of safety. He was well acquainted with the character and
sentiments of Galerius, and sufficiently apprised, that if he
wished to live he must determine to reign. The decent and even
obstinate resistance which he chose to affect,\textsuperscript{16} was contrived
to justify his usurpation; nor did he yield to the acclamations
of the army, till he had provided the proper materials for a
letter, which he immediately despatched to the emperor of the
East. Constantine informed him of the melancholy event of his
father’s death, modestly asserted his natural claim to the
succession, and respectfully lamented, that the affectionate
violence of his troops had not permitted him to solicit the
Imperial purple in the regular and constitutional manner. The
first emotions of Galerius were those of surprise,
disappointment, and rage; and as he could seldom restrain his
passions, he loudly threatened, that he would commit to the
flames both the letter and the messenger. But his resentment
insensibly subsided; and when he recollected the doubtful chance
of war, when he had weighed the character and strength of his
adversary, he consented to embrace the honorable accommodation
which the prudence of Constantine had left open to him. Without
either condemning or ratifying the choice of the British army,
Galerius accepted the son of his deceased colleague as the
sovereign of the provinces beyond the Alps; but he gave him only
the title of Cæsar, and the fourth rank among the Roman princes,
whilst he conferred the vacant place of Augustus on his favorite
Severus. The apparent harmony of the empire was still preserved,
and Constantine, who already possessed the substance, expected,
without impatience, an opportunity of obtaining the honors, of
supreme power.\textsuperscript{17}

\pagenote[15]{Cunctis qui aderant, annitentibus, sed præcipue
Croco (alii Eroco) [Erich?] Alamannorum Rege, auxilii gratia
Constantium comitato, imperium capit. Victor Junior, c. 41. This
is perhaps the first instance of a barbarian king, who assisted
the Roman arms with an independent body of his own subjects. The
practice grew familiar and at last became fatal.}

\pagenote[16]{His panegyrist Eumenius (vii. 8) ventures to affirm
in the presence of Constantine, that he put spurs to his horse,
and tried, but in vain, to escape from the hands of his
soldiers.}

\pagenote[17]{Lactantius de M. P. c. 25. Eumenius (vii. 8.) gives
a rhetorical turn to the whole transaction.}

The children of Constantius by his second marriage were six in
number, three of either sex, and whose Imperial descent might
have solicited a preference over the meaner extraction of the son
of Helena. But Constantine was in the thirty-second year of his
age, in the full vigor both of mind and body, at the time when
the eldest of his brothers could not possibly be more than
thirteen years old. His claim of superior merit had been allowed
and ratified by the dying emperor.\textsuperscript{18} In his last moments
Constantius bequeathed to his eldest son the care of the safety
as well as greatness of the family; conjuring him to assume both
the authority and the sentiments of a father with regard to the
children of Theodora. Their liberal education, advantageous
marriages, the secure dignity of their lives, and the first
honors of the state with which they were invested, attest the
fraternal affection of Constantine; and as those princes
possessed a mild and grateful disposition, they submitted without
reluctance to the superiority of his genius and fortune.\textsuperscript{19}

\pagenote[18]{The choice of Constantine, by his dying father,
which is warranted by reason, and insinuated by Eumenius, seems
to be confirmed by the most unexceptionable authority, the
concurring evidence of Lactantius (de M. P. c. 24) and of
Libanius, (Oratio i.,) of Eusebius (in Vit. Constantin. l. i. c.
18, 21) and of Julian, (Oratio i)}

\pagenote[19]{Of the three sisters of Constantine, Constantia
married the emperor Licinius, Anastasia the Cæsar Bassianus, and
Eutropia the consul Nepotianus. The three brothers were,
Dalmatius, Julius Constantius, and Annibalianus, of whom we shall
have occasion to speak hereafter.}

II. The ambitious spirit of Galerius was scarcely reconciled to
the disappointment of his views upon the Gallic provinces, before
the unexpected loss of Italy wounded his pride as well as power
in a still more sensible part. The long absence of the emperors
had filled Rome with discontent and indignation; and the people
gradually discovered, that the preference given to Nicomedia and
Milan was not to be ascribed to the particular inclination of
Diocletian, but to the permanent form of government which he had
instituted. It was in vain that, a few months after his
abdication, his successors dedicated, under his name, those
magnificent baths, whose ruins still supply the ground as well as
the materials for so many churches and convents.\textsuperscript{20} The
tranquility of those elegant recesses of ease and luxury was
disturbed by the impatient murmurs of the Romans, and a report
was insensibly circulated, that the sums expended in erecting
those buildings would soon be required at their hands. About that
time the avarice of Galerius, or perhaps the exigencies of the
state, had induced him to make a very strict and rigorous
inquisition into the property of his subjects, for the purpose of
a general taxation, both on their lands and on their persons. A
very minute survey appears to have been taken of their real
estates; and wherever there was the slightest suspicion of
concealment, torture was very freely employed to obtain a sincere
declaration of their personal wealth.\textsuperscript{21} The privileges which had
exalted Italy above the rank of the provinces were no longer
regarded:\textsuperscript{211} and the officers of the revenue already began to
number the Roman people, and to settle the proportion of the new
taxes. Even when the spirit of freedom had been utterly
extinguished, the tamest subjects have sometimes ventured to
resist an unprecedented invasion of their property; but on this
occasion the injury was aggravated by the insult, and the sense
of private interest was quickened by that of national honor. The
conquest of Macedonia, as we have already observed, had delivered
the Roman people from the weight of personal taxes.

Though they had experienced every form of despotism, they had now
enjoyed that exemption near five hundred years; nor could they
patiently brook the insolence of an Illyrian peasant, who, from
his distant residence in Asia, presumed to number Rome among the
tributary cities of his empire. The rising fury of the people was
encouraged by the authority, or at least the connivance, of the
senate; and the feeble remains of the Prætorian guards, who had
reason to apprehend their own dissolution, embraced so honorable
a pretence, and declared their readiness to draw their swords in
the service of their oppressed country. It was the wish, and it
soon became the hope, of every citizen, that after expelling from
Italy their foreign tyrants, they should elect a prince who, by
the place of his residence, and by his maxims of government,
might once more deserve the title of Roman emperor. The name, as
well as the situation, of Maxentius determined in his favor the
popular enthusiasm.

\pagenote[20]{See Gruter. Inscrip. p. 178. The six princes are
all mentioned, Diocletian and Maximian as the senior Augusti, and
fathers of the emperors. They jointly dedicate, for the use of
their own Romans, this magnificent edifice. The architects have
delineated the ruins of these Thermoe, and the antiquarians,
particularly Donatus and Nardini, have ascertained the ground
which they covered. One of the great rooms is now the Carthusian
church; and even one of the porter’s lodges is sufficient to form
another church, which belongs to the Feuillans.}

\pagenote[21]{See Lactantius de M. P. c. 26, 31. }

\pagenote[211]{Saviguy, in his memoir on Roman taxation, (Mem.
Berl. Academ. 1822, 1823, p. 5,) dates from this period the
abolition of the Jus Italicum. He quotes a remarkable passage of
Aurelius Victor. Hinc denique parti Italiæ invec tum tributorum
ingens malum. Aur. Vict. c. 39. It was a necessary consequence of
the division of the empire: it became impossible to maintain a
second court and executive, and leave so large and fruitful a
part of the territory exempt from contribution.—M.}

Maxentius was the son of the emperor Maximian, and he had married
the daughter of Galerius. His birth and alliance seemed to offer
him the fairest promise of succeeding to the empire; but his
vices and incapacity procured him the same exclusion from the
dignity of Cæsar, which Constantine had deserved by a dangerous
superiority of merit. The policy of Galerius preferred such
associates as would never disgrace the choice, nor dispute the
commands, of their benefactor. An obscure stranger was therefore
raised to the throne of Italy, and the son of the late emperor of
the West was left to enjoy the luxury of a private fortune in a
villa a few miles distant from the capital. The gloomy passions
of his soul, shame, vexation, and rage, were inflamed by envy on
the news of Constantine’s success; but the hopes of Maxentius
revived with the public discontent, and he was easily persuaded
to unite his personal injury and pretensions with the cause of
the Roman people. Two Prætorian tribunes and a commissary of
provisions undertook the management of the conspiracy; and as
every order of men was actuated by the same spirit, the immediate
event was neither doubtful nor difficult. The præfect of the
city, and a few magistrates, who maintained their fidelity to
Severus, were massacred by the guards; and Maxentius, invested
with the Imperial ornaments, was acknowledged by the applauding
senate and people as the protector of the Roman freedom and
dignity. It is uncertain whether Maximian was previously
acquainted with the conspiracy; but as soon as the standard of
rebellion was erected at Rome, the old emperor broke from the
retirement where the authority of Diocletian had condemned him to
pass a life of melancholy and solitude, and concealed his
returning ambition under the disguise of paternal tenderness. At
the request of his son and of the senate, he condescended to
reassume the purple. His ancient dignity, his experience, and his
fame in arms, added strength as well as reputation to the party
of Maxentius.\textsuperscript{22}

\pagenote[22]{The sixth Panegyric represents the conduct of
Maximian in the most favorable light, and the ambiguous
expression of Aurelius Victor, “retractante diu,” may signify
either that he contrived, or that he opposed, the conspiracy. See
Zosimus, l. ii. p. 79, and Lactantius de M. P. c. 26.}

According to the advice, or rather the orders, of his colleague,
the emperor Severus immediately hastened to Rome, in the full
confidence, that, by his unexpected celerity, he should easily
suppress the tumult of an unwarlike populace, commanded by a
licentious youth. But he found on his arrival the gates of the
city shut against him, the walls filled with men and arms, an
experienced general at the head of the rebels, and his own troops
without spirit or affection. A large body of Moors deserted to
the enemy, allured by the promise of a large donative; and, if it
be true that they had been levied by Maximian in his African war,
preferring the natural feelings of gratitude to the artificial
ties of allegiance. Anulinus, the Prætorian præfect, declared
himself in favor of Maxentius, and drew after him the most
considerable part of the troops, accustomed to obey his commands.

Rome, according to the expression of an orator, recalled her
armies; and the unfortunate Severus, destitute of force and of
counsel, retired, or rather fled, with precipitation, to Ravenna.

Here he might for some time have been safe. The fortifications of
Ravenna were able to resist the attempts, and the morasses that
surrounded the town were sufficient to prevent the approach, of
the Italian army. The sea, which Severus commanded with a
powerful fleet, secured him an inexhaustible supply of
provisions, and gave a free entrance to the legions, which, on
the return of spring, would advance to his assistance from
Illyricum and the East. Maximian, who conducted the siege in
person, was soon convinced that he might waste his time and his
army in the fruitless enterprise, and that he had nothing to hope
either from force or famine. With an art more suitable to the
character of Diocletian than to his own, he directed his attack,
not so much against the walls of Ravenna, as against the mind of
Severus. The treachery which he had experienced disposed that
unhappy prince to distrust the most sincere of his friends and
adherents. The emissaries of Maximian easily persuaded his
credulity, that a conspiracy was formed to betray the town, and
prevailed upon his fears not to expose himself to the discretion
of an irritated conqueror, but to accept the faith of an
honorable capitulation. He was at first received with humanity
and treated with respect. Maximian conducted the captive emperor
to Rome, and gave him the most solemn assurances that he had
secured his life by the resignation of the purple. But Severus
could obtain only an easy death and an Imperial funeral. When the
sentence was signified to him, the manner of executing it was
left to his own choice; he preferred the favorite mode of the
ancients, that of opening his veins; and as soon as he expired,
his body was carried to the sepulchre which had been constructed
for the family of Gallienus.\textsuperscript{23}

\pagenote[23]{The circumstances of this war, and the death of
Severus, are very doubtfully and variously told in our ancient
fragments, (see Tillemont, Hist. des Empereurs, tom. iv. part i.
p. 555.) I have endeavored to extract from them a consistent and
probable narration. * Note: Manso justly observes that two
totally different narratives might be formed, almost upon equal
authority. Beylage, iv.—M.}

\section{Part \thesection.}

Though the characters of Constantine and Maxentius had very
little affinity with each other, their situation and interest
were the same; and prudence seemed to require that they should
unite their forces against the common enemy. Notwithstanding the
superiority of his age and dignity, the indefatigable Maximian
passed the Alps, and, courting a personal interview with the
sovereign of Gaul, carried with him his daughter Fausta as the
pledge of the new alliance. The marriage was celebrated at Arles
with every circumstance of magnificence; and the ancient
colleague of Diocletian, who again asserted his claim to the
Western empire, conferred on his son-in-law and ally the title of
Augustus. By consenting to receive that honor from Maximian,
Constantine seemed to embrace the cause of Rome and of the
senate; but his professions were ambiguous, and his assistance
slow and ineffectual. He considered with attention the
approaching contest between the masters of Italy and the emperor
of the East, and was prepared to consult his own safety or
ambition in the event of the war.\textsuperscript{24}

\pagenote[24]{The sixth Panegyric was pronounced to celebrate the
elevation of Constantine; but the prudent orator avoids the
mention either of Galerius or of Maxentius. He introduces only
one slight allusion to the actual troubles, and to the majesty of
Rome. * Note: Compare Manso, Beylage, iv. p. 302. Gibbon’s
account is at least as probable as that of his critic.—M.}

The importance of the occasion called for the presence and
abilities of Galerius. At the head of a powerful army, collected
from Illyricum and the East, he entered Italy, resolved to
revenge the death of Severus, and to chastise the rebellious
Romans; or, as he expressed his intentions, in the furious
language of a barbarian, to extirpate the senate, and to destroy
the people by the sword. But the skill of Maximian had concerted
a prudent system of defence. The invader found every place
hostile, fortified, and inaccessible; and though he forced his
way as far as Narni, within sixty miles of Rome, his dominion in
Italy was confined to the narrow limits of his camp. Sensible of
the increasing difficulties of his enterprise, the haughty
Galerius made the first advances towards a reconciliation, and
despatched two of his most considerable officers to tempt the
Roman princes by the offer of a conference, and the declaration
of his paternal regard for Maxentius, who might obtain much more
from his liberality than he could hope from the doubtful chance
of war.\textsuperscript{25} The offers of Galerius were rejected with firmness,
his perfidious friendship refused with contempt, and it was not
long before he discovered, that, unless he provided for his
safety by a timely retreat, he had some reason to apprehend the
fate of Severus. The wealth which the Romans defended against his
rapacious tyranny, they freely contributed for his destruction.
The name of Maximian, the popular arts of his son, the secret
distribution of large sums, and the promise of still more liberal
rewards, checked the ardor and corrupted the fidelity of the
Illyrian legions; and when Galerius at length gave the signal of
the retreat, it was with some difficulty that he could prevail on
his veterans not to desert a banner which had so often conducted
them to victory and honor. A contemporary writer assigns two
other causes for the failure of the expedition; but they are both
of such a nature, that a cautious historian will scarcely venture
to adopt them. We are told that Galerius, who had formed a very
imperfect notion of the greatness of Rome by the cities of the
East with which he was acquainted, found his forces inadequate to
the siege of that immense capital.

But the extent of a city serves only to render it more accessible
to the enemy: Rome had long since been accustomed to submit on
the approach of a conqueror; nor could the temporary enthusiasm
of the people have long contended against the discipline and
valor of the legions. We are likewise informed that the legions
themselves were struck with horror and remorse, and that those
pious sons of the republic refused to violate the sanctity of
their venerable parent.\textsuperscript{26} But when we recollect with how much
ease, in the more ancient civil wars, the zeal of party and the
habits of military obedience had converted the native citizens of
Rome into her most implacable enemies, we shall be inclined to
distrust this extreme delicacy of strangers and barbarians, who
had never beheld Italy till they entered it in a hostile manner.
Had they not been restrained by motives of a more interested
nature, they would probably have answered Galerius in the words
of Cæsar’s veterans: “If our general wishes to lead us to the
banks of the Tyber, we are prepared to trace out his camp.
Whatsoever walls he has determined to level with the ground, our
hands are ready to work the engines: nor shall we hesitate,
should the name of the devoted city be Rome itself.” These are
indeed the expressions of a poet; but of a poet who has been
distinguished, and even censured, for his strict adherence to the
truth of history.\textsuperscript{27}

\pagenote[25]{With regard to this negotiation, see the fragments
of an anonymous historian, published by Valesius at the end of
his edition of Ammianus Marcellinus, p. 711. These fragments have
furnished with several curious, and, as it should seem, authentic
anecdotes.}

\pagenote[26]{Lactantius de M. P. c. 28. The former of these
reasons is probably taken from Virgil’s Shepherd: “Illam * * *
ego huic notra similem, Meliboee, putavi,” \&c. Lactantius
delights in these poetical illusions.}

\pagenote[27]{Castra super Tusci si ponere Tybridis undas; (\textit{jubeas})\\
Hesperios audax veniam metator in agros.\\
Tu quoscunque voles in planum effundere muros,\\
His aries actus disperget saxa lacertis;\\
Illa licet penitus tolli quam jusseris urbem\\
Roma sit.\\
Lucan. Pharsal. i. 381.}

The legions of Galerius exhibited a very melancholy proof of
their disposition, by the ravages which they committed in their
retreat. They murdered, they ravished, they plundered, they drove
away the flocks and herds of the Italians; they burnt the
villages through which they passed, and they endeavored to
destroy the country which it had not been in their power to
subdue. During the whole march, Maxentius hung on their rear, but
he very prudently declined a general engagement with those brave
and desperate veterans. His father had undertaken a second
journey into Gaul, with the hope of persuading Constantine, who
had assembled an army on the frontier, to join in the pursuit,
and to complete the victory. But the actions of Constantine were
guided by reason, and not by resentment. He persisted in the wise
resolution of maintaining a balance of power in the divided
empire, and he no longer hated Galerius, when that aspiring
prince had ceased to be an object of terror.\textsuperscript{28}

\pagenote[28]{Lactantius de M. P. c. 27. Zosim. l. ii. p. 82. The
latter, that Constantine, in his interview with Maximian, had
promised to declare war against Galerius.}

The mind of Galerius was the most susceptible of the sterner
passions, but it was not, however, incapable of a sincere and
lasting friendship. Licinius, whose manners as well as character
were not unlike his own, seems to have engaged both his affection
and esteem. Their intimacy had commenced in the happier period
perhaps of their youth and obscurity. It had been cemented by the
freedom and dangers of a military life; they had advanced almost
by equal steps through the successive honors of the service; and
as soon as Galerius was invested with the Imperial dignity, he
seems to have conceived the design of raising his companion to
the same rank with himself. During the short period of his
prosperity, he considered the rank of Cæsar as unworthy of the
age and merit of Licinius, and rather chose to reserve for him
the place of Constantius, and the empire of the West. While the
emperor was employed in the Italian war, he intrusted his friend
with the defence of the Danube; and immediately after his return
from that unfortunate expedition, he invested Licinius with the
vacant purple of Severus, resigning to his immediate command the
provinces of Illyricum.\textsuperscript{29} The news of his promotion was no
sooner carried into the East, than Maximin, who governed, or
rather oppressed, the countries of Egypt and Syria, betrayed his
envy and discontent, disdained the inferior name of Cæsar, and,
notwithstanding the prayers as well as arguments of Galerius,
exacted, almost by violence, the equal title of Augustus.\textsuperscript{30} For
the first, and indeed for the last time, the Roman world was
administered by six emperors. In the West, Constantine and
Maxentius affected to reverence their father Maximian. In the
East, Licinius and Maximin honored with more real consideration
their benefactor Galerius. The opposition of interest, and the
memory of a recent war, divided the empire into two great hostile
powers; but their mutual fears produced an apparent tranquillity,
and even a feigned reconciliation, till the death of the elder
princes, of Maximian, and more particularly of Galerius, gave a
new direction to the views and passions of their surviving
associates.

\pagenote[29]{M. de Tillemont (Hist. des Empereurs, tom. iv. part
i. p. 559) has proved that Licinius, without passing through the
intermediate rank of Cæsar, was declared Augustus, the 11th of
November, A. D. 307, after the return of Galerius from Italy.}

\pagenote[30]{Lactantius de M. P. c. 32. When Galerius declared
Licinius Augustus with himself, he tried to satisfy his younger
associates, by inventing for Constantine and Maximin (not
Maxentius; see Baluze, p. 81) the new title of sons of the
Augusti. But when Maximin acquainted him that he had been saluted
Augustus by the army, Galerius was obliged to acknowledge him as
well as Constantine, as equal associates in the Imperial
dignity.}

When Maximian had reluctantly abdicated the empire, the venal
orators of the times applauded his philosophic moderation. When
his ambition excited, or at least encouraged, a civil war, they
returned thanks to his generous patriotism, and gently censured
that love of ease and retirement which had withdrawn him from the
public service.\textsuperscript{31} But it was impossible that minds like those of
Maximian and his son could long possess in harmony an undivided
power. Maxentius considered himself as the legal sovereign of
Italy, elected by the Roman senate and people; nor would he
endure the control of his father, who arrogantly declared that by
\textit{his} name and abilities the rash youth had been established on
the throne. The cause was solemnly pleaded before the Prætorian
guards; and those troops, who dreaded the severity of the old
emperor, espoused the party of Maxentius.\textsuperscript{32} The life and freedom
of Maximian were, however, respected, and he retired from Italy
into Illyricum, affecting to lament his past conduct, and
secretly contriving new mischiefs. But Galerius, who was well
acquainted with his character, soon obliged him to leave his
dominions, and the last refuge of the disappointed Maximian was
the court of his son-in-law Constantine.\textsuperscript{33} He was received with
respect by that artful prince, and with the appearance of filial
tenderness by the empress Fausta. That he might remove every
suspicion, he resigned the Imperial purple a second time,\textsuperscript{34}
professing himself at length convinced of the vanity of greatness
and ambition. Had he persevered in this resolution, he might have
ended his life with less dignity, indeed, than in his first
retirement, yet, however, with comfort and reputation. But the
near prospect of a throne brought back to his remembrance the
state from whence he was fallen, and he resolved, by a desperate
effort, either to reign or to perish. An incursion of the Franks
had summoned Constantine, with a part of his army, to the banks
of the Rhine; the remainder of the troops were stationed in the
southern provinces of Gaul, which lay exposed to the enterprises
of the Italian emperor, and a considerable treasure was deposited
in the city of Arles. Maximian either craftily invented, or
easily credited, a vain report of the death of Constantine.
Without hesitation he ascended the throne, seized the treasure,
and scattering it with his accustomed profusion among the
soldiers, endeavored to awake in their minds the memory of his
ancient dignity and exploits. Before he could establish his
authority, or finish the negotiation which he appears to have
entered into with his son Maxentius, the celerity of Constantine
defeated all his hopes. On the first news of his perfidy and
ingratitude, that prince returned by rapid marches from the Rhine
to the Saone, embarked on the last-mentioned river at Chalons,
and, at Lyons trusting himself to the rapidity of the Rhone,
arrived at the gates of Arles with a military force which it was
impossible for Maximian to resist, and which scarcely permitted
him to take refuge in the neighboring city of Marseilles. The
narrow neck of land which joined that place to the continent was
fortified against the besiegers, whilst the sea was open, either
for the escape of Maximian, or for the succor of Maxentius, if
the latter should choose to disguise his invasion of Gaul under
the honorable pretence of defending a distressed, or, as he might
allege, an injured father. Apprehensive of the fatal consequences
of delay, Constantine gave orders for an immediate assault; but
the scaling-ladders were found too short for the height of the
walls, and Marseilles might have sustained as long a siege as it
formerly did against the arms of Cæsar, if the garrison,
conscious either of their fault or of their danger, had not
purchased their pardon by delivering up the city and the person
of Maximian. A secret but irrevocable sentence of death was
pronounced against the usurper; he obtained only the same favor
which he had indulged to Severus, and it was published to the
world, that, oppressed by the remorse of his repeated crimes, he
strangled himself with his own hands. After he had lost the
assistance, and disdained the moderate counsels, of Diocletian,
the second period of his active life was a series of public
calamities and personal mortifications, which were terminated, in
about three years, by an ignominious death. He deserved his fate;
but we should find more reason to applaud the humanity of
Constantine, if he had spared an old man, the benefactor of his
father, and the father of his wife. During the whole of this
melancholy transaction, it appears that Fausta sacrificed the
sentiments of nature to her conjugal duties.\textsuperscript{35}

\pagenote[31]{See Panegyr. Vet. vi. 9. Audi doloris nostri
liberam vocem, \&c. The whole passage is imagined with artful
flattery, and expressed with an easy flow of eloquence.}

\pagenote[32]{Lactantius de M. P. c. 28. Zosim. l. ii. p. 82. A
report was spread, that Maxentius was the son of some obscure
Syrian, and had been substituted by the wife of Maximian as her
own child. See Aurelius Victor, Anonym. Valesian, and Panegyr.
Vet. ix. 3, 4.}

\pagenote[33]{Ab urbe pulsum, ab Italia fugatum, ab Illyrico
repudiatum, provinciis, tuis copiis, tuo palatio recepisti.
Eumen. in Panegyr Vet. vii. 14.}

\pagenote[34]{Lactantius de M. P. c. 29. Yet, after the
resignation of the purple, Constantine still continued to
Maximian the pomp and honors of the Imperial dignity; and on all
public occasions gave the right hand place to his father-in-law.
Panegyr. Vet. viii. 15.}

\pagenote[35]{Zosim. l. ii. p. 82. Eumenius in Panegyr. Vet. vii.
16—21. The latter of these has undoubtedly represented the whole
affair in the most favorable light for his sovereign. Yet even
from this partial narrative we may conclude, that the repeated
clemency of Constantine, and the reiterated treasons of Maximian,
as they are described by Lactantius, (de M. P. c. 29, 30,) and
copied by the moderns, are destitute of any historical
foundation. Note: Yet some pagan authors relate and confirm them.
Aurelius Victor speaking of Maximin, says, cumque specie officii,
dolis compositis, Constantinum generum tentaret acerbe, jure
tamen interierat. Aur. Vict. de Cæsar l. p. 623. Eutropius also
says, inde ad Gallias profectus est (Maximianus) composito
tamquam a filio esset expulsus, ut Constantino genero jun
geretur: moliens tamen Constantinum, reperta occasione,
interficere, dedit justissimo exitu. Eutrop. x. p. 661. (Anon.
Gent.)—G. —— These writers hardly confirm more than Gibbon
admits; he denies the repeated clemency of Constantine, and the
reiterated treasons of Maximian Compare Manso, p. 302.—M.}

The last years of Galerius were less shameful and unfortunate;
and though he had filled with more glory the subordinate station
of Cæsar than the superior rank of Augustus, he preserved, till
the moment of his death, the first place among the princes of the
Roman world. He survived his retreat from Italy about four years;
and wisely relinquishing his views of universal empire, he
devoted the remainder of his life to the enjoyment of pleasure,
and to the execution of some works of public utility, among which
we may distinguish the discharging into the Danube the
superfluous waters of the Lake Pelso, and the cutting down the
immense forests that encompassed it; an operation worthy of a
monarch, since it gave an extensive country to the agriculture of
his Pannonian subjects.\textsuperscript{36} His death was occasioned by a very
painful and lingering disorder. His body, swelled by an
intemperate course of life to an unwieldy corpulence, was covered
with ulcers, and devoured by innumerable swarms of those insects
which have given their name to a most loathsome disease;\textsuperscript{37} but
as Galerius had offended a very zealous and powerful party among
his subjects, his sufferings, instead of exciting their
compassion, have been celebrated as the visible effects of divine
justice.\textsuperscript{38} He had no sooner expired in his palace of Nicomedia,
than the two emperors who were indebted for their purple to his
favors, began to collect their forces, with the intention either
of disputing, or of dividing, the dominions which he had left
without a master. They were persuaded, however, to desist from
the former design, and to agree in the latter. The provinces of
Asia fell to the share of Maximin, and those of Europe augmented
the portion of Licinius. The Hellespont and the Thracian
Bosphorus formed their mutual boundary, and the banks of those
narrow seas, which flowed in the midst of the Roman world, were
covered with soldiers, with arms, and with fortifications. The
deaths of Maximian and of Galerius reduced the number of emperors
to four. The sense of their true interest soon connected Licinius
and Constantine; a secret alliance was concluded between Maximin
and Maxentius, and their unhappy subjects expected with terror
the bloody consequences of their inevitable dissensions, which
were no longer restrained by the fear or the respect which they
had entertained for Galerius.\textsuperscript{39}

\pagenote[36]{Aurelius Victor, c. 40. But that lake was situated
on the upper Pannonia, near the borders of Noricum; and the
province of Valeria (a name which the wife of Galerius gave to
the drained country) undoubtedly lay between the Drave and the
Danube, (Sextus Rufus, c. 9.) I should therefore suspect that
Victor has confounded the Lake Pelso with the Volocean marshes,
or, as they are now called, the Lake Sabaton. It is placed in the
heart of Valeria, and its present extent is not less than twelve
Hungarian miles (about seventy English) in length, and two in
breadth. See Severini Pannonia, l. i. c. 9.}

\pagenote[37]{Lactantius (de M. P. c. 33) and Eusebius (l. viii.
c. 16) describe the symptoms and progress of his disorder with
singular accuracy and apparent pleasure.}

\pagenote[38]{If any (like the late Dr. Jortin, Remarks on
Ecclesiastical History, vol. ii. p. 307—356) still delight in
recording the wonderful deaths of the persecutors, I would
recommend to their perusal an admirable passage of Grotius (Hist.
l. vii. p. 332) concerning the last illness of Philip II. of
Spain.}

\pagenote[39]{See Eusebius, l. ix. 6, 10. Lactantius de M. P. c.
36. Zosimus is less exact, and evidently confounds Maximian with
Maximin.}

Among so many crimes and misfortunes, occasioned by the passions
of the Roman princes, there is some pleasure in discovering a
single action which may be ascribed to their virtue. In the sixth
year of his reign, Constantine visited the city of Autun, and
generously remitted the arrears of tribute, reducing at the same
time the proportion of their assessment from twenty-five to
eighteen thousand heads, subject to the real and personal
capitation.\textsuperscript{40} Yet even this indulgence affords the most
unquestionable proof of the public misery. This tax was so
extremely oppressive, either in itself or in the mode of
collecting it, that whilst the revenue was increased by
extortion, it was diminished by despair: a considerable part of
the territory of Autun was left uncultivated; and great numbers
of the provincials rather chose to live as exiles and outlaws,
than to support the weight of civil society. It is but too
probable, that the bountiful emperor relieved, by a partial act
of liberality, one among the many evils which he had caused by
his general maxims of administration. But even those maxims were
less the effect of choice than of necessity. And if we except the
death of Maximian, the reign of Constantine in Gaul seems to have
been the most innocent and even virtuous period of his life.

The provinces were protected by his presence from the inroads of
the barbarians, who either dreaded or experienced his active
valor. After a signal victory over the Franks and Alemanni,
several of their princes were exposed by his order to the wild
beasts in the amphitheatre of Treves, and the people seem to have
enjoyed the spectacle, without discovering, in such a treatment
of royal captives, any thing that was repugnant to the laws of
nations or of humanity.\textsuperscript{41}

\pagenote[40]{See the viiith Panegyr., in which Eumenius
displays, in the presence of Constantine, the misery and the
gratitude of the city of Autun.}

\pagenote[41]{Eutropius, x. 3. Panegyr. Veter. vii. 10, 11, 12. A
great number of the French youth were likewise exposed to the
same cruel and ignominious death Yet the panegyric assumes
something of an apologetic tone. Te vero Constantine,
quantumlibet oderint hoses, dum perhorrescant. Hæc est enim vera
virtus, ut non ament et quiescant. The orator appeals to the
ancient ideal of the republic.—M.}

The virtues of Constantine were rendered more illustrious by the
vices of Maxentius. Whilst the Gallic provinces enjoyed as much
happiness as the condition of the times was capable of receiving,
Italy and Africa groaned under the dominion of a tyrant, as
contemptible as he was odious. The zeal of flattery and faction
has indeed too frequently sacrificed the reputation of the
vanquished to the glory of their successful rivals; but even
those writers who have revealed, with the most freedom and
pleasure, the faults of Constantine, unanimously confess that
Maxentius was cruel, rapacious, and profligate.\textsuperscript{42} He had the
good fortune to suppress a slight rebellion in Africa. The
governor and a few adherents had been guilty; the province
suffered for their crime. The flourishing cities of Cirtha and
Carthage, and the whole extent of that fertile country, were
wasted by fire and sword. The abuse of victory was followed by
the abuse of law and justice. A formidable army of sycophants and
delators invaded Africa; the rich and the noble were easily
convicted of a connection with the rebels; and those among them
who experienced the emperor’s clemency, were only punished by the
confiscation of their estates.\textsuperscript{43} So signal a victory was
celebrated by a magnificent triumph, and Maxentius exposed to the
eyes of the people the spoils and captives of a Roman province.
The state of the capital was no less deserving of compassion than
that of Africa. The wealth of Rome supplied an inexhaustible fund
for his vain and prodigal expenses, and the ministers of his
revenue were skilled in the arts of rapine. It was under his
reign that the method of exacting a \textit{free gift} from the senators
was first invented; and as the sum was insensibly increased, the
pretences of levying it, a victory, a birth, a marriage, or an
imperial consulship, were proportionably multiplied.\textsuperscript{44} Maxentius
had imbibed the same implacable aversion to the senate, which had
characterized most of the former tyrants of Rome; nor was it
possible for his ungrateful temper to forgive the generous
fidelity which had raised him to the throne, and supported him
against all his enemies. The lives of the senators were exposed
to his jealous suspicions, the dishonor of their wives and
daughters heightened the gratification of his sensual passions. \textsuperscript{45}
It may be presumed that an Imperial lover was seldom reduced
to sigh in vain; but whenever persuasion proved ineffectual, he
had recourse to violence; and there remains \textit{one} memorable
example of a noble matron, who preserved her chastity by a
voluntary death. The soldiers were the only order of men whom he
appeared to respect, or studied to please. He filled Rome and
Italy with armed troops, connived at their tumults, suffered them
with impunity to plunder, and even to massacre, the defenceless
people;\textsuperscript{46} and indulging them in the same licentiousness which
their emperor enjoyed, Maxentius often bestowed on his military
favorites the splendid villa, or the beautiful wife, of a
senator. A prince of such a character, alike incapable of
governing, either in peace or in war, might purchase the support,
but he could never obtain the esteem, of the army. Yet his pride
was equal to his other vices. Whilst he passed his indolent life
either within the walls of his palace, or in the neighboring
gardens of Sallust, he was repeatedly heard to declare, that \textit{he
alone} was emperor, and that the other princes were no more than
his lieutenants, on whom he had devolved the defence of the
frontier provinces, that he might enjoy without interruption the
elegant luxury of the capital. Rome, which had so long regretted
the absence, lamented, during the six years of his reign, the
presence of her sovereign.\textsuperscript{47}

\pagenote[42]{Julian excludes Maxentius from the banquet of the
Cæsars with abhorrence and contempt; and Zosimus (l. ii. p. 85)
accuses him of every kind of cruelty and profligacy.}

\pagenote[43]{Zosimus, l. ii. p. 83—85. Aurelius Victor.}

\pagenote[44]{The passage of Aurelius Victor should be read in
the following manner: Primus instituto pessimo, munerum specie,
Patres Oratores que pecuniam conferre prodigenti sibi cogeret.}

\pagenote[45]{Panegyr. Vet. ix. 3. Euseb. Hist Eccles. viii. 14,
et in Vit. Constant i. 33, 34. Rufinus, c. 17. The virtuous
matron who stabbed herself to escape the violence of Maxentius,
was a Christian, wife to the præfect of the city, and her name
was Sophronia. It still remains a question among the casuists,
whether, on such occasions, suicide is justifiable.}

\pagenote[46]{Prætorianis cædem vulgi quondam annueret, is the
vague expression of Aurelius Victor. See more particular, though
somewhat different, accounts of a tumult and massacre which
happened at Rome, in Eusebius, (l. viii. c. 14,) and in Zosimus,
(l. ii. p. 84.)}

\pagenote[47]{See, in the Panegyrics, (ix. 14,) a lively
description of the indolence and vain pride of Maxentius. In
another place the orator observes that the riches which Rome had
accumulated in a period of 1060 years, were lavished by the
tyrant on his mercenary bands; redemptis ad civile latrocinium
manibus in gesserat.}

Though Constantine might view the conduct of Maxentius with
abhorrence, and the situation of the Romans with compassion, we
have no reason to presume that he would have taken up arms to
punish the one or to relieve the other. But the tyrant of Italy
rashly ventured to provoke a formidable enemy, whose ambition had
been hitherto restrained by considerations of prudence, rather
than by principles of justice.\textsuperscript{48} After the death of Maximian,
his titles, according to the established custom, had been erased,
and his statues thrown down with ignominy. His son, who had
persecuted and deserted him when alive, effected to display the
most pious regard for his memory, and gave orders that a similar
treatment should be immediately inflicted on all the statues that
had been erected in Italy and Africa to the honor of Constantine.

That wise prince, who sincerely wished to decline a war, with the
difficulty and importance of which he was sufficiently
acquainted, at first dissembled the insult, and sought for
redress by the milder expedient of negotiation, till he was
convinced that the hostile and ambitious designs of the Italian
emperor made it necessary for him to arm in his own defence.
Maxentius, who openly avowed his pretensions to the whole
monarchy of the West, had already prepared a very considerable
force to invade the Gallic provinces on the side of Rhætia; and
though he could not expect any assistance from Licinius, he was
flattered with the hope that the legions of Illyricum, allured by
his presents and promises, would desert the standard of that
prince, and unanimously declare themselves his soldiers and
subjects.\textsuperscript{49} Constantine no longer hesitated. He had deliberated
with caution, he acted with vigor. He gave a private audience to
the ambassadors, who, in the name of the senate and people,
conjured him to deliver Rome from a detested tyrant; and without
regarding the timid remonstrances of his council, he resolved to
prevent the enemy, and to carry the war into the heart of Italy. \textsuperscript{50}

\pagenote[48]{After the victory of Constantine, it was
universally allowed, that the motive of delivering the republic
from a detested tyrant, would, at any time, have justified his
expedition into Italy. Euseb in Vi’. Constantin. l. i. c. 26.
Panegyr. Vet. ix. 2.}

\pagenote[49]{Zosimus, l. ii. p. 84, 85. Nazarius in Panegyr. x.
7—13.}

\pagenote[50]{See Panegyr. Vet. ix. 2. Omnibus fere tuis
Comitibus et Ducibus non solum tacite mussantibus, sed etiam
aperte timentibus; contra consilia hominum, contra Haruspicum
monita, ipse per temet liberandæ arbis tempus venisse sentires.
The embassy of the Romans is mentioned only by Zonaras, (l.
xiii.,) and by Cedrenus, (in Compend. Hist. p. 370;) but those
modern Greeks had the opportunity of consulting many writers
which have since been lost, among which we may reckon the life of
Constantine by Praxagoras. Photius (p. 63) has made a short
extract from that historical work.}

The enterprise was as full of danger as of glory; and the
unsuccessful event of two former invasions was sufficient to
inspire the most serious apprehensions. The veteran troops, who
revered the name of Maximian, had embraced in both those wars the
party of his son, and were now restrained by a sense of honor, as
well as of interest, from entertaining an idea of a second
desertion. Maxentius, who considered the Prætorian guards as the
firmest defence of his throne, had increased them to their
ancient establishment; and they composed, including the rest of
the Italians who were enlisted into his service, a formidable
body of fourscore thousand men. Forty thousand Moors and
Carthaginians had been raised since the reduction of Africa. Even
Sicily furnished its proportion of troops; and the armies of
Maxentius amounted to one hundred and seventy thousand foot and
eighteen thousand horse. The wealth of Italy supplied the
expenses of the war; and the adjacent provinces were exhausted,
to form immense magazines of corn and every other kind of
provisions.

The whole force of Constantine consisted of ninety thousand foot
and eight thousand horse;\textsuperscript{51} and as the defence of the Rhine
required an extraordinary attention during the absence of the
emperor, it was not in his power to employ above half his troops
in the Italian expedition, unless he sacrificed the public safety
to his private quarrel.\textsuperscript{52} At the head of about forty thousand
soldiers he marched to encounter an enemy whose numbers were at
least four times superior to his own. But the armies of Rome,
placed at a secure distance from danger, were enervated by
indulgence and luxury. Habituated to the baths and theatres of
Rome, they took the field with reluctance, and were chiefly
composed of veterans who had almost forgotten, or of new levies
who had never acquired, the use of arms and the practice of war.
The hardy legions of Gaul had long defended the frontiers of the
empire against the barbarians of the North; and in the
performance of that laborious service, their valor was exercised
and their discipline confirmed. There appeared the same
difference between the leaders as between the armies. Caprice or
flattery had tempted Maxentius with the hopes of conquest; but
these aspiring hopes soon gave way to the habits of pleasure and
the consciousness of his inexperience. The intrepid mind of
Constantine had been trained from his earliest youth to war, to
action, and to military command.

\pagenote[51]{Zosimus (l. ii. p. 86) has given us this curious
account of the forces on both sides. He makes no mention of any
naval armaments, though we are assured (Panegyr. Vet. ix. 25)
that the war was carried on by sea as well as by land; and that
the fleet of Constantine took possession of Sardinia, Corsica,
and the ports of Italy.}

\pagenote[52]{Panegyr. Vet. ix. 3. It is not surprising that the
orator should diminish the numbers with which his sovereign
achieved the conquest of Italy; but it appears somewhat singular
that he should esteem the tyrant’s army at no more than 100,000
men.}

\section{Part \thesection.}

When Hannibal marched from Gaul into Italy, he was obliged, first
to discover, and then to open, a way over mountains, and through
savage nations, that had never yielded a passage to a regular
army.\textsuperscript{53} The Alps were then guarded by nature, they are now
fortified by art. Citadels, constructed with no less skill than
labor and expense, command every avenue into the plain, and on
that side render Italy almost inaccessible to the enemies of the
king of Sardinia.\textsuperscript{54} But in the course of the intermediate
period, the generals, who have attempted the passage, have seldom
experienced any difficulty or resistance. In the age of
Constantine, the peasants of the mountains were civilized and
obedient subjects; the country was plentifully stocked with
provisions, and the stupendous highways, which the Romans had
carried over the Alps, opened several communications between Gaul
and Italy.\textsuperscript{55} Constantine preferred the road of the Cottian Alps,
or, as it is now called, of Mount Cenis, and led his troops with
such active diligence, that he descended into the plain of
Piedmont before the court of Maxentius had received any certain
intelligence of his departure from the banks of the Rhine. The
city of Susa, however, which is situated at the foot of Mount
Cenis, was surrounded with walls, and provided with a garrison
sufficiently numerous to check the progress of an invader; but
the impatience of Constantine’s troops disdained the tedious
forms of a siege. The same day that they appeared before Susa,
they applied fire to the gates, and ladders to the walls; and
mounting to the assault amidst a shower of stones and arrows,
they entered the place sword in hand, and cut in pieces the
greatest part of the garrison. The flames were extinguished by
the care of Constantine, and the remains of Susa preserved from
total destruction. About forty miles from thence, a more severe
contest awaited him. A numerous army of Italians was assembled
under the lieutenants of Maxentius, in the plains of Turin. Its
principal strength consisted in a species of heavy cavalry, which
the Romans, since the decline of their discipline, had borrowed
from the nations of the East. The horses, as well as the men,
were clothed in complete armor, the joints of which were artfully
adapted to the motions of their bodies. The aspect of this
cavalry was formidable, their weight almost irresistible; and as,
on this occasion, their generals had drawn them up in a compact
column or wedge, with a sharp point, and with spreading flanks,
they flattered themselves that they could easily break and
trample down the army of Constantine. They might, perhaps, have
succeeded in their design, had not their experienced adversary
embraced the same method of defence, which in similar
circumstances had been practised by Aurelian. The skilful
evolutions of Constantine divided and baffled this massy column
of cavalry. The troops of Maxentius fled in confusion towards
Turin; and as the gates of the city were shut against them, very
few escaped the sword of the victorious pursuers. By this
important service, Turin deserved to experience the clemency and
even favor of the conqueror. He made his entry into the Imperial
palace of Milan, and almost all the cities of Italy between the
Alps and the Po not only acknowledged the power, but embraced
with zeal the party, of Constantine.\textsuperscript{56}

\pagenote[53]{The three principal passages of the Alps between
Gaul and Italy, are those of Mount St. Bernard, Mount Cenis, and
Mount Genevre. Tradition, and a resemblance of names, (Alpes
Penninoe,) had assigned the first of these for the march of
Hannibal, (see Simler de Alpibus.) The Chevalier de Folard
(Polyp. tom. iv.) and M. d’Anville have led him over Mount
Genevre. But notwithstanding the authority of an experienced
officer and a learned geographer, the pretensions of Mount Cenis
are supported in a specious, not to say a convincing, manner, by
M. Grosley. Observations sur l’Italie, tom. i. p. 40, \&c. ——The
dissertation of Messrs. Cramer and Wickham has clearly shown that
the Little St. Bernard must claim the honor of Hannibal’s
passage. Mr. Long (London, 1831) has added some sensible
corrections re Hannibal’s march to the Alps.—M}

\pagenote[54]{La Brunette near Suse, Demont, Exiles,
Fenestrelles, Coni, \&c.}

\pagenote[55]{See Ammian. Marcellin. xv. 10. His description of
the roads over the Alps is clear, lively, and accurate.}

\pagenote[56]{Zosimus as well as Eusebius hasten from the passage
of the Alps to the decisive action near Rome. We must apply to
the two Panegyrics for the intermediate actions of Constantine.}

From Milan to Rome, the Æmilian and Flaminian highways offered an
easy march of about four hundred miles; but though Constantine
was impatient to encounter the tyrant, he prudently directed his
operations against another army of Italians, who, by their
strength and position, might either oppose his progress, or, in
case of a misfortune, might intercept his retreat. Ruricius
Pompeianus, a general distinguished by his valor and ability, had
under his command the city of Verona, and all the troops that
were stationed in the province of Venetia. As soon as he was
informed that Constantine was advancing towards him, he detached
a large body of cavalry, which was defeated in an engagement near
Brescia, and pursued by the Gallic legions as far as the gates of
Verona. The necessity, the importance, and the difficulties of
the siege of Verona, immediately presented themselves to the
sagacious mind of Constantine.\textsuperscript{57} The city was accessible only by
a narrow peninsula towards the west, as the other three sides
were surrounded by the Adige, a rapid river, which covered the
province of Venetia, from whence the besieged derived an
inexhaustible supply of men and provisions. It was not without
great difficulty, and after several fruitless attempts, that
Constantine found means to pass the river at some distance above
the city, and in a place where the torrent was less violent. He
then encompassed Verona with strong lines, pushed his attacks
with prudent vigor, and repelled a desperate sally of Pompeianus.
That intrepid general, when he had used every means of defence
that the strength of the place or that of the garrison could
afford, secretly escaped from Verona, anxious not for his own,
but for the public safety. With indefatigable diligence he soon
collected an army sufficient either to meet Constantine in the
field, or to attack him if he obstinately remained within his
lines. The emperor, attentive to the motions, and informed of the
approach of so formidable an enemy, left a part of his legions to
continue the operations of the siege, whilst, at the head of
those troops on whose valor and fidelity he more particularly
depended, he advanced in person to engage the general of
Maxentius. The army of Gaul was drawn up in two lines, according
to the usual practice of war; but their experienced leader,
perceiving that the numbers of the Italians far exceeded his own,
suddenly changed his disposition, and, reducing the second,
extended the front of his first line to a just proportion with
that of the enemy. Such evolutions, which only veteran troops can
execute without confusion in a moment of danger, commonly prove
decisive; but as this engagement began towards the close of the
day, and was contested with great obstinacy during the whole
night, there was less room for the conduct of the generals than
for the courage of the soldiers. The return of light displayed
the victory of Constantine, and a field of carnage covered with
many thousands of the vanquished Italians. Their general,
Pompeianus, was found among the slain; Verona immediately
surrendered at discretion, and the garrison was made prisoners of
war.\textsuperscript{58} When the officers of the victorious army congratulated
their master on this important success, they ventured to add some
respectful complaints, of such a nature, however, as the most
jealous monarchs will listen to without displeasure. They
represented to Constantine, that, not contented with all the
duties of a commander, he had exposed his own person with an
excess of valor which almost degenerated into rashness; and they
conjured him for the future to pay more regard to the
preservation of a life in which the safety of Rome and of the
empire was involved.\textsuperscript{59}

\pagenote[57]{The Marquis Maffei has examined the siege and
battle of Verona with that degree of attention and accuracy which
was due to a memorable action that happened in his native
country. The fortifications of that city, constructed by
Gallienus, were less extensive than the modern walls, and the
amphitheatre was not included within their circumference. See
Verona Illustrata, part i. p. 142 150.}

\pagenote[58]{They wanted chains for so great a multitude of
captives; and the whole council was at a loss; but the sagacious
conqueror imagined the happy expedient of converting into fetters
the swords of the vanquished. Panegyr. Vet. ix. 11.}

\pagenote[59]{Panegyr. Vet. ix. 11.}

While Constantine signalized his conduct and valor in the field,
the sovereign of Italy appeared insensible of the calamities and
danger of a civil war which reigned in the heart of his
dominions. Pleasure was still the only business of Maxentius.
Concealing, or at least attempting to conceal, from the public
knowledge the misfortunes of his arms,\textsuperscript{60} he indulged himself in
a vain confidence which deferred the remedies of the approaching
evil, without deferring the evil itself.\textsuperscript{61} The rapid progress of
Constantine\textsuperscript{62} was scarcely sufficient to awaken him from his
fatal security; he flattered himself, that his well-known
liberality, and the majesty of the Roman name, which had already
delivered him from two invasions, would dissipate with the same
facility the rebellious army of Gaul. The officers of experience
and ability, who had served under the banners of Maximian, were
at length compelled to inform his effeminate son of the imminent
danger to which he was reduced; and, with a freedom that at once
surprised and convinced him, to urge the necessity of preventing
his ruin by a vigorous exertion of his remaining power. The
resources of Maxentius, both of men and money, were still
considerable. The Prætorian guards felt how strongly their own
interest and safety were connected with his cause; and a third
army was soon collected, more numerous than those which had been
lost in the battles of Turin and Verona. It was far from the
intention of the emperor to lead his troops in person. A stranger
to the exercises of war, he trembled at the apprehension of so
dangerous a contest; and as fear is commonly superstitious, he
listened with melancholy attention to the rumors of omens and
presages which seemed to menace his life and empire. Shame at
length supplied the place of courage, and forced him to take the
field. He was unable to sustain the contempt of the Roman people.
The circus resounded with their indignant clamors, and they
tumultuously besieged the gates of the palace, reproaching the
pusillanimity of their indolent sovereign, and celebrating the
heroic spirit of Constantine.\textsuperscript{63} Before Maxentius left Rome, he
consulted the Sibylline books. The guardians of these ancient
oracles were as well versed in the arts of this world as they
were ignorant of the secrets of fate; and they returned him a
very prudent answer, which might adapt itself to the event, and
secure their reputation, whatever should be the chance of arms. \textsuperscript{64}

\pagenote[60]{Literas calamitatum suarum indices supprimebat.
Panegyr Vet. ix. 15.}

\pagenote[61]{Remedia malorum potius quam mala differebat, is the
fine censure which Tacitus passes on the supine indolence of
Vitellius.}

\pagenote[62]{The Marquis Maffei has made it extremely probable
that Constantine was still at Verona, the 1st of September, A.D.
312, and that the memorable æra of the indications was dated from
his conquest of the Cisalpine Gaul.}

\pagenote[63]{See Panegyr. Vet. xi. 16. Lactantius de M. P. c.
44.}

\pagenote[64]{Illo die hostem Romanorum esse periturum. The
vanquished became of course the enemy of Rome.}

The celerity of Constantine’s march has been compared to the
rapid conquest of Italy by the first of the Cæsars; nor is the
flattering parallel repugnant to the truth of history, since no
more than fifty-eight days elapsed between the surrender of
Verona and the final decision of the war. Constantine had always
apprehended that the tyrant would consult the dictates of fear,
and perhaps of prudence; and that, instead of risking his last
hopes in a general engagement, he would shut himself up within
the walls of Rome. His ample magazines secured him against the
danger of famine; and as the situation of Constantine admitted
not of delay, he might have been reduced to the sad necessity of
destroying with fire and sword the Imperial city, the noblest
reward of his victory, and the deliverance of which had been the
motive, or rather indeed the pretence, of the civil war.\textsuperscript{65} It
was with equal surprise and pleasure, that on his arrival at a
place called Saxa Rubra, about nine miles from Rome,\textsuperscript{66} he
discovered the army of Maxentius prepared to give him battle.\textsuperscript{67}
Their long front filled a very spacious plain, and their deep
array reached to the banks of the Tyber, which covered their
rear, and forbade their retreat. We are informed, and we may
believe, that Constantine disposed his troops with consummate
skill, and that he chose for himself the post of honor and
danger. Distinguished by the splendor of his arms, he charged in
person the cavalry of his rival; and his irresistible attack
determined the fortune of the day. The cavalry of Maxentius was
principally composed either of unwieldy cuirassiers, or of light
Moors and Numidians. They yielded to the vigor of the Gallic
horse, which possessed more activity than the one, more firmness
than the other. The defeat of the two wings left the infantry
without any protection on its flanks, and the undisciplined
Italians fled without reluctance from the standard of a tyrant
whom they had always hated, and whom they no longer feared. The
Prætorians, conscious that their offences were beyond the reach
of mercy, were animated by revenge and despair. Notwithstanding
their repeated efforts, those brave veterans were unable to
recover the victory: they obtained, however, an honorable death;
and it was observed that their bodies covered the same ground
which had been occupied by their ranks.\textsuperscript{68} The confusion then
became general, and the dismayed troops of Maxentius, pursued by
an implacable enemy, rushed by thousands into the deep and rapid
stream of the Tyber. The emperor himself attempted to escape back
into the city over the Milvian bridge; but the crowds which
pressed together through that narrow passage forced him into the
river, where he was immediately drowned by the weight of his
armor.\textsuperscript{69} His body, which had sunk very deep into the mud, was
found with some difficulty the next day. The sight of his head,
when it was exposed to the eyes of the people, convinced them of
their deliverance, and admonished them to receive with
acclamations of loyalty and gratitude the fortunate Constantine,
who thus achieved by his valor and ability the most splendid
enterprise of his life.\textsuperscript{70}

\pagenote[65]{See Panegyr. Vet. ix. 16, x. 27. The former of
these orators magnifies the hoards of corn, which Maxentius had
collected from Africa and the Islands. And yet, if there is any
truth in the scarcity mentioned by Eusebius, (in Vit. Constantin.
l. i. c. 36,) the Imperial granaries must have been open only to
the soldiers.}

\pagenote[66]{Maxentius... tandem urbe in Saxa Rubra, millia
ferme novem ægerrime progressus. Aurelius Victor. See Cellarius
Geograph. Antiq. tom. i. p. 463. Saxa Rubra was in the
neighborhood of the Cremera, a trifling rivulet, illustrated by
the valor and glorious death of the three hundred Fabii.}

\pagenote[67]{The post which Maxentius had taken, with the Tyber
in his rear is very clearly described by the two Panegyrists, ix.
16, x. 28.}

\pagenote[68]{Exceptis latrocinii illius primis auctoribus, qui
desperata venia ocum quem pugnæ sumpserant texere corporibus.
Panegyr. Vet 17.}

\pagenote[69]{A very idle rumor soon prevailed, that Maxentius,
who had not taken any precaution for his own retreat, had
contrived a very artful snare to destroy the army of the
pursuers; but that the wooden bridge, which was to have been
loosened on the approach of Constantine, unluckily broke down
under the weight of the flying Italians. M. de Tillemont (Hist.
des Empereurs, tom. iv. part i. p. 576) very seriously examines
whether, in contradiction to common sense, the testimony of
Eusebius and Zosimus ought to prevail over the silence of
Lactantius, Nazarius, and the anonymous, but contemporary orator,
who composed the ninth Panegyric. * Note: Manso (Beylage, vi.)
examines the question, and adduces two manifest allusions to the
bridge, from the Life of Constantine by Praxagoras, and from
Libanius. Is it not very probable that such a bridge was thrown
over the river to facilitate the advance, and to secure the
retreat, of the army of Maxentius? In case of defeat, orders were
given for destroying it, in order to check the pursuit: it broke
down accidentally, or in the confusion was destroyed, as has not
unfrequently been the case, before the proper time.—M.}

\pagenote[70]{Zosimus, l. ii. p. 86-88, and the two Panegyrics,
the former of which was pronounced a few months afterwards,
afford the clearest notion of this great battle. Lactantius,
Eusebius, and even the Epitomes, supply several useful hints.}

In the use of victory, Constantine neither deserved the praise of
clemency, nor incurred the censure of immoderate rigor.\textsuperscript{71} He
inflicted the same treatment to which a defeat would have exposed
his own person and family, put to death the two sons of the
tyrant, and carefully extirpated his whole race. The most
distinguished adherents of Maxentius must have expected to share
his fate, as they had shared his prosperity and his crimes; but
when the Roman people loudly demanded a greater number of
victims, the conqueror resisted, with firmness and humanity,
those servile clamors, which were dictated by flattery as well as
by resentment. Informers were punished and discouraged; the
innocent, who had suffered under the late tyranny, were recalled
from exile, and restored to their estates. A general act of
oblivion quieted the minds and settled the property of the
people, both in Italy and in Africa.\textsuperscript{72} The first time that
Constantine honored the senate with his presence, he
recapitulated his own services and exploits in a modest oration,
assured that illustrious order of his sincere regard, and
promised to reëstablish its ancient dignity and privileges. The
grateful senate repaid these unmeaning professions by the empty
titles of honor, which it was yet in their power to bestow; and
without presuming to ratify the authority of Constantine, they
passed a decree to assign him the first rank among the three
\textit{Augusti} who governed the Roman world.\textsuperscript{73} Games and festivals
were instituted to preserve the fame of his victory, and several
edifices, raised at the expense of Maxentius, were dedicated to
the honor of his successful rival. The triumphal arch of
Constantine still remains a melancholy proof of the decline of
the arts, and a singular testimony of the meanest vanity. As it
was not possible to find in the capital of the empire a sculptor
who was capable of adorning that public monument, the arch of
Trajan, without any respect either for his memory or for the
rules of propriety, was stripped of its most elegant figures. The
difference of times and persons, of actions and characters, was
totally disregarded. The Parthian captives appear prostrate at
the feet of a prince who never carried his arms beyond the
Euphrates; and curious antiquarians can still discover the head
of Trajan on the trophies of Constantine. The new ornaments which
it was necessary to introduce between the vacancies of ancient
sculpture are executed in the rudest and most unskilful manner.\textsuperscript{74}

\pagenote[71]{Zosimus, the enemy of Constantine, allows (l. ii.
p. 88) that only a few of the friends of Maxentius were put to
death; but we may remark the expressive passage of Nazarius,
(Panegyr. Vet. x. 6.) Omnibus qui labefactari statum ejus
poterant cum stirpe deletis. The other orator (Panegyr. Vet. ix.
20, 21) contents himself with observing, that Constantine, when
he entered Rome, did not imitate the cruel massacres of Cinna, of
Marius, or of Sylla. * Note: This may refer to the son or sons of
Maxentius.—M.}

\pagenote[72]{See the two Panegyrics, and the laws of this and
the ensuing year, in the Theodosian Code.}

\pagenote[73]{Panegyr. Vet. ix. 20. Lactantius de M. P. c. 44.
Maximin, who was confessedly the eldest Cæsar, claimed, with some
show of reason, the first rank among the Augusti.}

\pagenote[74]{Adhuc cuncta opera quæ magnifice construxerat,
urbis fanum atque basilicam, Flavii meritis patres sacravere.
Aurelius Victor. With regard to the theft of Trajan’s trophies,
consult Flaminius Vacca, apud Montfaucon, Diarium Italicum, p.
250, and l’Antiquite Expliquee of the latter, tom. iv. p. 171.}

The final abolition of the Prætorian guards was a measure of
prudence as well as of revenge. Those haughty troops, whose
numbers and privileges had been restored, and even augmented, by
Maxentius, were forever suppressed by Constantine. Their
fortified camp was destroyed, and the few Prætorians who had
escaped the fury of the sword were dispersed among the legions,
and banished to the frontiers of the empire, where they might be
serviceable without again becoming dangerous.\textsuperscript{75} By suppressing
the troops which were usually stationed in Rome, Constantine gave
the fatal blow to the dignity of the senate and people, and the
disarmed capital was exposed without protection to the insults or
neglect of its distant master. We may observe, that in this last
effort to preserve their expiring freedom, the Romans, from the
apprehension of a tribute, had raised Maxentius to the throne. He
exacted that tribute from the senate under the name of a free
gift. They implored the assistance of Constantine. He vanquished
the tyrant, and converted the free gift into a perpetual tax. The
senators, according to the declaration which was required of
their property, were divided into several classes. The most
opulent paid annually eight pounds of gold, the next class paid
four, the last two, and those whose poverty might have claimed an
exemption, were assessed, however, at seven pieces of gold.
Besides the regular members of the senate, their sons, their
descendants, and even their relations, enjoyed the vain
privileges, and supported the heavy burdens, of the senatorial
order; nor will it any longer excite our surprise, that
Constantine should be attentive to increase the number of persons
who were included under so useful a description.\textsuperscript{76} After the
defeat of Maxentius, the victorious emperor passed no more than
two or three months in Rome, which he visited twice during the
remainder of his life, to celebrate the solemn festivals of the
tenth and of the twentieth years of his reign. Constantine was
almost perpetually in motion, to exercise the legions, or to
inspect the state of the provinces. Treves, Milan, Aquileia,
Sirmium, Naissus, and Thessalonica, were the occasional places of
his residence, till he founded a new Rome on the confines of
Europe and Asia.\textsuperscript{77}

\pagenote[75]{Prætoriæ legiones ac subsidia factionibus aptiora
quam urbi Romæ, sublata penitus; simul arma atque usus indumenti
militaris Aurelius Victor. Zosimus (l. ii. p. 89) mentions this
fact as an historian, and it is very pompously celebrated in the
ninth Panegyric.}

\pagenote[76]{Ex omnibus provinciis optimates viros Curiæ tuæ
pigneraveris ut Senatus dignitas.... ex totius Orbis flore
consisteret. Nazarius in Panegyr. Vet x. 35. The word
pigneraveris might almost seem maliciously chosen. Concerning the
senatorial tax, see Zosimus, l. ii. p. 115, the second title of
the sixth book of the Theodosian Code, with Godefroy’s
Commentary, and Memoires de l’Academic des Inscriptions, tom.
xxviii. p. 726.}

\pagenote[77]{From the Theodosian Code, we may now begin to trace
the motions of the emperors; but the dates both of time and place
have frequently been altered by the carelessness of
transcribers.}

Before Constantine marched into Italy, he had secured the
friendship, or at least the neutrality, of Licinius, the Illyrian
emperor. He had promised his sister Constantia in marriage to
that prince; but the celebration of the nuptials was deferred
till after the conclusion of the war, and the interview of the
two emperors at Milan, which was appointed for that purpose,
appeared to cement the union of their families and interests.\textsuperscript{78}
In the midst of the public festivity they were suddenly obliged
to take leave of each other. An inroad of the Franks summoned
Constantine to the Rhine, and the hostile approach of the
sovereign of Asia demanded the immediate presence of Licinius.
Maximin had been the secret ally of Maxentius, and without being
discouraged by his fate, he resolved to try the fortune of a
civil war. He moved out of Syria, towards the frontiers of
Bithynia, in the depth of winter. The season was severe and
tempestuous; great numbers of men as well as horses perished in
the snow; and as the roads were broken up by incessant rains, he
was obliged to leave behind him a considerable part of the heavy
baggage, which was unable to follow the rapidity of his forced
marches. By this extraordinary effort of diligence, he arrived
with a harassed but formidable army, on the banks of the Thracian
Bosphorus before the lieutenants of Licinius were apprised of his
hostile intentions. Byzantium surrendered to the power of
Maximin, after a siege of eleven days. He was detained some days
under the walls of Heraclea; and he had no sooner taken
possession of that city than he was alarmed by the intelligence
that Licinius had pitched his camp at the distance of only
eighteen miles. After a fruitless negotiation, in which the two
princes attempted to seduce the fidelity of each other’s
adherents, they had recourse to arms. The emperor of the East
commanded a disciplined and veteran army of above seventy
thousand men; and Licinius, who had collected about thirty
thousand Illyrians, was at first oppressed by the superiority of
numbers. His military skill, and the firmness of his troops,
restored the day, and obtained a decisive victory. The incredible
speed which Maximin exerted in his flight is much more celebrated
than his prowess in the battle. Twenty-four hours afterwards he
was seen, pale, trembling, and without his Imperial ornaments, at
Nicomedia, one hundred and sixty miles from the place of his
defeat. The wealth of Asia was yet unexhausted; and though the
flower of his veterans had fallen in the late action, he had
still power, if he could obtain time, to draw very numerous
levies from Syria and Egypt. But he survived his misfortune only
three or four months. His death, which happened at Tarsus, was
variously ascribed to despair, to poison, and to the divine
justice. As Maximin was alike destitute of abilities and of
virtue, he was lamented neither by the people nor by the
soldiers. The provinces of the East, delivered from the terrors
of civil war, cheerfully acknowledged the authority of Licinius. \textsuperscript{79}

\pagenote[78]{Zosimus (l. ii. p. 89) observes, that before the
war the sister of Constantine had been betrothed to Licinius.
According to the younger Victor, Diocletian was invited to the
nuptials; but having ventured to plead his age and infirmities,
he received a second letter, filled with reproaches for his
supposed partiality to the cause of Maxentius and Maximin.}

\pagenote[79]{Zosimus mentions the defeat and death of Maximin as
ordinary events; but Lactantius expatiates on them, (de M. P. c.
45-50,) ascribing them to the miraculous interposition of Heaven.
Licinius at that time was one of the protectors of the church.}

The vanquished emperor left behind him two children, a boy of
about eight, and a girl of about seven, years old. Their
inoffensive age might have excited compassion; but the compassion
of Licinius was a very feeble resource, nor did it restrain him
from \textit{extinguishing} the name and memory of his adversary. The
death of Severianus will admit of less excuse, as it was dictated
neither by revenge nor by policy. The conqueror had never
received any injury from the father of that unhappy youth, and
the short and obscure reign of Severus, in a distant part of the
empire, was already forgotten. But the execution of Candidianus
was an act of the blackest cruelty and ingratitude. He was the
natural son of Galerius, the friend and benefactor of Licinius.
The prudent father had judged him too young to sustain the weight
of a diadem; but he hoped that, under the protection of princes
who were indebted to his favor for the Imperial purple,
Candidianus might pass a secure and honorable life. He was now
advancing towards the twentieth year of his age, and the royalty
of his birth, though unsupported either by merit or ambition, was
sufficient to exasperate the jealous mind of Licinius.\textsuperscript{80} To
these innocent and illustrious victims of his tyranny, we must
add the wife and daughter of the emperor Diocletian. When that
prince conferred on Galerius the title of Cæsar, he had given him
in marriage his daughter Valeria, whose melancholy adventures
might furnish a very singular subject for tragedy. She had
fulfilled and even surpassed the duties of a wife. As she had not
any children herself, she condescended to adopt the illegitimate
son of her husband, and invariably displayed towards the unhappy
Candidianus the tenderness and anxiety of a real mother. After
the death of Galerius, her ample possessions provoked the
avarice, and her personal attractions excited the desires, of his
successor, Maximin.\textsuperscript{81} He had a wife still alive; but divorce was
permitted by the Roman law, and the fierce passions of the tyrant
demanded an immediate gratification. The answer of Valeria was
such as became the daughter and widow of emperors; but it was
tempered by the prudence which her defenceless condition
compelled her to observe. She represented to the persons whom
Maximin had employed on this occasion, “that even if honor could
permit a woman of her character and dignity to entertain a
thought of second nuptials, decency at least must forbid her to
listen to his addresses at a time when the ashes of her husband
and his benefactor were still warm, and while the sorrows of her
mind were still expressed by her mourning garments. She ventured
to declare, that she could place very little confidence in the
professions of a man whose cruel inconstancy was capable of
repudiating a faithful and affectionate wife.”\textsuperscript{82} On this
repulse, the love of Maximin was converted into fury; and as
witnesses and judges were always at his disposal, it was easy for
him to cover his fury with an appearance of legal proceedings,
and to assault the reputation as well as the happiness of
Valeria. Her estates were confiscated, her eunuchs and domestics
devoted to the most inhuman tortures; and several innocent and
respectable matrons, who were honored with her friendship,
suffered death, on a false accusation of adultery. The empress
herself, together with her mother Prisca, was condemned to exile;
and as they were ignominiously hurried from place to place before
they were confined to a sequestered village in the deserts of
Syria, they exposed their shame and distress to the provinces of
the East, which, during thirty years, had respected their august
dignity. Diocletian made several ineffectual efforts to alleviate
the misfortunes of his daughter; and, as the last return that he
expected for the Imperial purple, which he had conferred upon
Maximin, he entreated that Valeria might be permitted to share
his retirement of Salona, and to close the eyes of her afflicted
father.\textsuperscript{83} He entreated; but as he could no longer threaten, his
prayers were received with coldness and disdain; and the pride of
Maximin was gratified, in treating Diocletian as a suppliant, and
his daughter as a criminal. The death of Maximin seemed to assure
the empresses of a favorable alteration in their fortune. The
public disorders relaxed the vigilance of their guard, and they
easily found means to escape from the place of their exile, and
to repair, though with some precaution, and in disguise, to the
court of Licinius. His behavior, in the first days of his reign,
and the honorable reception which he gave to young Candidianus,
inspired Valeria with a secret satisfaction, both on her own
account and on that of her adopted son. But these grateful
prospects were soon succeeded by horror and astonishment; and the
bloody executions which stained the palace of Nicomedia
sufficiently convinced her that the throne of Maximin was filled
by a tyrant more inhuman than himself. Valeria consulted her
safety by a hasty flight, and, still accompanied by her mother
Prisca, they wandered above fifteen months\textsuperscript{84} through the
provinces, concealed in the disguise of plebeian habits. They
were at length discovered at Thessalonica; and as the sentence of
their death was already pronounced, they were immediately
beheaded, and their bodies thrown into the sea. The people gazed
on the melancholy spectacle; but their grief and indignation were
suppressed by the terrors of a military guard. Such was the
unworthy fate of the wife and daughter of Diocletian. We lament
their misfortunes, we cannot discover their crimes; and whatever
idea we may justly entertain of the cruelty of Licinius, it
remains a matter of surprise that he was not contented with some
more secret and decent method of revenge.\textsuperscript{85}

\pagenote[80]{Lactantius de M. P. c. 50. Aurelius Victor touches
on the different conduct of Licinius, and of Constantine, in the
use of victory.}

\pagenote[81]{The sensual appetites of Maximin were gratified at
the expense of his subjects. His eunuchs, who forced away wives
and virgins, examined their naked charms with anxious curiosity,
lest any part of their body should be found unworthy of the royal
embraces. Coyness and disdain were considered as treason, and the
obstinate fair one was condemned to be drowned. A custom was
gradually introduced, that no person should marry a wife without
the permission of the emperor, “ut ipse in omnibus nuptiis
prægustator esset.” Lactantius de M. P. c. 38.}

\pagenote[82]{Lactantius de M. P. c. 39.}

\pagenote[83]{Diocletian at last sent cognatum suum, quendam
militarem æ potentem virum, to intercede in favor of his
daughter, (Lactantius de M. P. c. 41.) We are not sufficiently
acquainted with the history of these times to point out the
person who was employed.}

\pagenote[84]{Valeria quoque per varias provincias quindecim
mensibus plebeio cultu pervagata. Lactantius de M. P. c. 51.
There is some doubt whether we should compute the fifteen months
from the moment of her exile, or from that of her escape. The
expression of parvagata seems to denote the latter; but in that
case we must suppose that the treatise of Lactantius was written
after the first civil war between Licinius and Constantine. See
Cuper, p. 254.}

\pagenote[85]{Ita illis pudicitia et conditio exitio fuit.
Lactantius de M. P. c. 51. He relates the misfortunes of the
innocent wife and daughter of Discletian with a very natural
mixture of pity and exultation.}

The Roman world was now divided between Constantine and Licinius,
the former of whom was master of the West, and the latter of the
East. It might perhaps have been expected that the conquerors,
fatigued with civil war, and connected by a private as well as
public alliance, would have renounced, or at least would have
suspended, any further designs of ambition. And yet a year had
scarcely elapsed after the death of Maximin, before the
victorious emperors turned their arms against each other. The
genius, the success, and the aspiring temper of Constantine, may
seem to mark him out as the aggressor; but the perfidious
character of Licinius justifies the most unfavorable suspicions,
and by the faint light which history reflects on this
transaction,\textsuperscript{86} we may discover a conspiracy fomented by his arts
against the authority of his colleague. Constantine had lately
given his sister Anastasia in marriage to Bassianus, a man of a
considerable family and fortune, and had elevated his new kinsman
to the rank of Cæsar. According to the system of government
instituted by Diocletian, Italy, and perhaps Africa, were
designed for his department in the empire. But the performance of
the promised favor was either attended with so much delay, or
accompanied with so many unequal conditions, that the fidelity of
Bassianus was alienated rather than secured by the honorable
distinction which he had obtained. His nomination had been
ratified by the consent of Licinius; and that artful prince, by
the means of his emissaries, soon contrived to enter into a
secret and dangerous correspondence with the new Cæsar, to
irritate his discontents, and to urge him to the rash enterprise
of extorting by violence what he might in vain solicit from the
justice of Constantine. But the vigilant emperor discovered the
conspiracy before it was ripe for execution; and after solemnly
renouncing the alliance of Bassianus, despoiled him of the
purple, and inflicted the deserved punishment on his treason and
ingratitude. The haughty refusal of Licinius, when he was
required to deliver up the criminals who had taken refuge in his
dominions, confirmed the suspicions already entertained of his
perfidy; and the indignities offered at Æmona, on the frontiers
of Italy, to the statues of Constantine, became the signal of
discord between the two princes.\textsuperscript{87}

\pagenote[86]{The curious reader, who consults the Valesian
fragment, p. 713, will probably accuse me of giving a bold and
licentious paraphrase; but if he considers it with attention, he
will acknowledge that my interpretation is probable and
consistent.}

\pagenote[87]{The situation of Æmona, or, as it is now called,
Laybach, in Carniola, (D’Anville, Geographie Ancienne, tom. i. p.
187,) may suggest a conjecture. As it lay to the north-east of
the Julian Alps, that important territory became a natural object
of dispute between the sovereigns of Italy and of Illyricum.}

The first battle was fought near Cibalis, a city of Pannonia,
situated on the River Save, about fifty miles above Sirmium.\textsuperscript{88}
From the inconsiderable forces which in this important contest
two such powerful monarchs brought into the field, it may be
inferred that the one was suddenly provoked, and that the other
was unexpectedly surprised. The emperor of the West had only
twenty thousand, and the sovereign of the East no more than five
and thirty thousand, men. The inferiority of number was, however,
compensated by the advantage of the ground. Constantine had taken
post in a defile about half a mile in breadth, between a steep
hill and a deep morass, and in that situation he steadily
expected and repulsed the first attack of the enemy. He pursued
his success, and advanced into the plain. But the veteran legions
of Illyricum rallied under the standard of a leader who had been
trained to arms in the school of Probus and Diocletian. The
missile weapons on both sides were soon exhausted; the two
armies, with equal valor, rushed to a closer engagement of swords
and spears, and the doubtful contest had already lasted from the
dawn of the day to a late hour of the evening, when the right
wing, which Constantine led in person, made a vigorous and
decisive charge. The judicious retreat of Licinius saved the
remainder of his troops from a total defeat; but when he computed
his loss, which amounted to more than twenty thousand men, he
thought it unsafe to pass the night in the presence of an active
and victorious enemy. Abandoning his camp and magazines, he
marched away with secrecy and diligence at the head of the
greatest part of his cavalry, and was soon removed beyond the
danger of a pursuit. His diligence preserved his wife, his son,
and his treasures, which he had deposited at Sirmium. Licinius
passed through that city, and breaking down the bridge on the
Save, hastened to collect a new army in Dacia and Thrace. In his
flight he bestowed the precarious title of Cæsar on Valens, his
general of the Illyrian frontier.\textsuperscript{89}

\pagenote[88]{Cibalis or Cibalæ (whose name is still preserved in
the obscure ruins of Swilei) was situated about fifty miles from
Sirmium, the capital of Illyricum, and about one hundred from
Taurunum, or Belgrade, and the conflux of the Danube and the
Save. The Roman garrisons and cities on those rivers are finely
illustrated by M. d’Anville in a memoir inserted in l’Academie
des Inscriptions, tom. xxviii.}

\pagenote[89]{Zosimus (l. ii. p. 90, 91) gives a very particular
account of this battle; but the descriptions of Zosimus are
rhetorical rather than military}

\section{Part \thesection.}

The plain of Mardia in Thrace was the theatre of a second battle
no less obstinate and bloody than the former. The troops on both
sides displayed the same valor and discipline; and the victory
was once more decided by the superior abilities of Constantine,
who directed a body of five thousand men to gain an advantageous
height, from whence, during the heat of the action, they attacked
the rear of the enemy, and made a very considerable slaughter.
The troops of Licinius, however, presenting a double front, still
maintained their ground, till the approach of night put an end to
the combat, and secured their retreat towards the mountains of
Macedonia.\textsuperscript{90} The loss of two battles, and of his bravest
veterans, reduced the fierce spirit of Licinius to sue for peace.
His ambassador Mistrianus was admitted to the audience of
Constantine: he expatiated on the common topics of moderation and
humanity, which are so familiar to the eloquence of the
vanquished; represented in the most insinuating language, that
the event of the war was still doubtful, whilst its inevitable
calamities were alike pernicious to both the contending parties;
and declared that he was authorized to propose a lasting and
honorable peace in the name of the \textit{two} emperors his masters.
Constantine received the mention of Valens with indignation and
contempt. “It was not for such a purpose,” he sternly replied,
“that we have advanced from the shores of the western ocean in an
uninterrupted course of combats and victories, that, after
rejecting an ungrateful kinsman, we should accept for our
colleague a contemptible slave. The abdication of Valens is the
first article of the treaty.”\textsuperscript{91} It was necessary to accept this
humiliating condition; and the unhappy Valens, after a reign of a
few days, was deprived of the purple and of his life. As soon as
this obstacle was removed, the tranquillity of the Roman world
was easily restored. The successive defeats of Licinius had
ruined his forces, but they had displayed his courage and
abilities. His situation was almost desperate, but the efforts of
despair are sometimes formidable, and the good sense of
Constantine preferred a great and certain advantage to a third
trial of the chance of arms. He consented to leave his rival, or,
as he again styled Licinius, his friend and brother, in the
possession of Thrace, Asia Minor, Syria, and Egypt; but the
provinces of Pannonia, Dalmatia, Dacia, Macedonia, and Greece,
were yielded to the Western empire, and the dominions of
Constantine now extended from the confines of Caledonia to the
extremity of Peloponnesus. It was stipulated by the same treaty,
that three royal youths, the sons of emperors, should be called
to the hopes of the succession. Crispus and the young Constantine
were soon afterwards declared Cæsars in the West, while the
younger Licinius was invested with the same dignity in the East.
In this double proportion of honors, the conqueror asserted the
superiority of his arms and power.\textsuperscript{92}

\pagenote[90]{Zosimus, l. ii. p. 92, 93. Anonym. Valesian. p.
713. The Epitomes furnish some circumstances; but they frequently
confound the two wars between Licinius and Constantine.}

\pagenote[91]{Petrus Patricius in Excerpt. Legat. p. 27. If it
should be thought that signifies more properly a son-in-law, we
might conjecture that Constantine, assuming the name as well as
the duties of a father, had adopted his younger brothers and
sisters, the children of Theodora. But in the best authors
sometimes signifies a husband, sometimes a father-in-law, and
sometimes a kinsman in general. See Spanheim, Observat. ad
Julian. Orat. i. p. 72.}

\pagenote[92]{Zosimus, l. ii. p. 93. Anonym. Valesian. p. 713.
Eutropius, x. v. Aurelius Victor, Euseb. in Chron. Sozomen, l. i.
c. 2. Four of these writers affirm that the promotion of the
Cæsars was an article of the treaty. It is, however, certain,
that the younger Constantine and Licinius were not yet born; and
it is highly probable that the promotion was made the 1st of
March, A. D. 317. The treaty had probably stipulated that the two
Cæsars might be created by the western, and one only by the
eastern emperor; but each of them reserved to himself the choice
of the persons.}

The reconciliation of Constantine and Licinius, though it was
imbittered by resentment and jealousy, by the remembrance of
recent injuries, and by the apprehension of future dangers,
maintained, however, above eight years, the tranquility of the
Roman world. As a very regular series of the Imperial laws
commences about this period, it would not be difficult to
transcribe the civil regulations which employed the leisure of
Constantine. But the most important of his institutions are
intimately connected with the new system of policy and religion,
which was not perfectly established till the last and peaceful
years of his reign. There are many of his laws, which, as far as
they concern the rights and property of individuals, and the
practice of the bar, are more properly referred to the private
than to the public jurisprudence of the empire; and he published
many edicts of so local and temporary a nature, that they would
ill deserve the notice of a general history. Two laws, however,
may be selected from the crowd; the one for its importance, the
other for its singularity; the former for its remarkable
benevolence, the latter for its excessive severity. 1. The horrid
practice, so familiar to the ancients, of exposing or murdering
their new-born infants, was become every day more frequent in the
provinces, and especially in Italy. It was the effect of
distress; and the distress was principally occasioned by the
intolerant burden of taxes, and by the vexatious as well as cruel
prosecutions of the officers of the revenue against their
insolvent debtors. The less opulent or less industrious part of
mankind, instead of rejoicing in an increase of family, deemed it
an act of paternal tenderness to release their children from the
impending miseries of a life which they themselves were unable to
support. The humanity of Constantine, moved, perhaps, by some
recent and extraordinary instances of despair, engaged him to
address an edict to all the cities of Italy, and afterwards of
Africa, directing immediate and sufficient relief to be given to
those parents who should produce before the magistrates the
children whom their own poverty would not allow them to educate.
But the promise was too liberal, and the provision too vague, to
effect any general or permanent benefit.\textsuperscript{93} The law, though it
may merit some praise, served rather to display than to alleviate
the public distress. It still remains an authentic monument to
contradict and confound those venal orators, who were too well
satisfied with their own situation to discover either vice or
misery under the government of a generous sovereign.\textsuperscript{94} 2. The
laws of Constantine against rapes were dictated with very little
indulgence for the most amiable weaknesses of human nature; since
the description of that crime was applied not only to the brutal
violence which compelled, but even to the gentle seduction which
might persuade, an unmarried woman, under the age of twenty-five,
to leave the house of her parents. “The successful ravisher was
punished with death;” and as if simple death was inadequate to
the enormity of his guilt, he was either burnt alive, or torn in
pieces by wild beasts in the amphitheatre. The virgin’s
declaration, that she had been carried away with her own consent,
instead of saving her lover, exposed her to share his fate. The
duty of a public prosecution was intrusted to the parents of the
guilty or unfortunate maid; and if the sentiments of nature
prevailed on them to dissemble the injury, and to repair by a
subsequent marriage the honor of their family, they were
themselves punished by exile and confiscation. The slaves,
whether male or female, who were convicted of having been
accessory to rape or seduction, were burnt alive, or put to death
by the ingenious torture of pouring down their throats a quantity
of melted lead. As the crime was of a public kind, the accusation
was permitted even to strangers.\textsuperscript{9401}

\pagenote[9401]{This explanation appears to me little probable.
Godefroy has made a much more happy conjecture, supported by all
the historical circumstances which relate to this edict. It was
published the 12th of May, A. D. 315. at Naissus in Pannonia, the
birthplace of Constantine. The 8th of October, in that year,
Constantine gained the victory of Cibalis over Licinius. He was
yet uncertain as to the fate of the war: the Christians, no
doubt, whom he favored, had prophesied his victory. Lactantius,
then preceptor of Crispus, had just written his work upon
Christianity, (his Divine Institutes;) he had dedicated it to
Constantine. In this book he had inveighed with great force
against infanticide, and the exposure of infants, (l. vi. c. 20.)
Is it not probable that Constantine had read this work, that he
had conversed on the subject with Lactantius, that he was moved,
among other things, by the passage to which I have referred, and
in the first transport of his enthusiasm, he published the edict
in question? The whole of the edict bears the character of
precipitation, of excitement, (entrainement,) rather than of
deliberate reflection—the extent of the promises, the
indefiniteness of the means, of the conditions, and of the time
during which the parents might have a right to the succor of the
state. Is there not reason to believe that the humanity of
Constantine was excited by the influence of Lactantius, by that
of the principles of Christianity, and of the Christians
themselves, already in high esteem with the emperor, rather than
by some “extraordinary instances of despair”? * * * See
Hegewisch, Essai Hist. sur les Finances Romaines. The edict for
Africa was not published till 322: of that we may say in truth
that its origin was in the misery of the times. Africa had
suffered much from the cruelty of Maxentius. Constantine says
expressly, that he had learned that parents, under the pressure
of distress, were there selling their children. This decree is
more distinct, more maturely deliberated than the former; the
succor which was to be given to the parents, and the source from
which it was to be derived, are determined. (Code Theod. l. xi.
tit. 27, c 2.) If the direct utility of these laws may not have
been very extensive, they had at least the great and happy effect
of establishing a decisive opposition between the principles of
the government and those which, at this time, had prevailed among
the subjects of the empire.—G.}

The commencement of the action was not limited to any term of
years, and the consequences of the sentence were extended to the
innocent offspring of such an irregular union.\textsuperscript{95} But whenever
the offence inspires less horror than the punishment, the rigor
of penal law is obliged to give way to the common feelings of
mankind. The most odious parts of this edict were softened or
repealed in the subsequent reigns;\textsuperscript{96} and even Constantine
himself very frequently alleviated, by partial acts of mercy, the
stern temper of his general institutions. Such, indeed, was the
singular humor of that emperor, who showed himself as indulgent,
and even remiss, in the execution of his laws, as he was severe,
and even cruel, in the enacting of them. It is scarcely possible
to observe a more decisive symptom of weakness, either in the
character of the prince, or in the constitution of the
government.\textsuperscript{97}

\pagenote[93]{Codex Theodosian. l. xi. tit. 27, tom. iv. p. 188,
with Godefroy’s observations. See likewise l. v. tit. 7, 8.}

\pagenote[94]{Omnia foris placita, domi prospera, annonæ
ubertate, fructuum copia, \&c. Panegyr. Vet. x. 38. This oration
of Nazarius was pronounced on the day of the Quinquennalia of the
Cæsars, the 1st of March, A. D. 321.}

\pagenote[95]{See the edict of Constantine, addressed to the
Roman people, in the Theodosian Code, l. ix. tit. 24, tom. iii.
p. 189.}

\pagenote[96]{His son very fairly assigns the true reason of the
repeal: “Na sub specie atrocioris judicii aliqua in ulciscendo
crimine dilatio næ ceretur.” Cod. Theod. tom. iii. p. 193}

\pagenote[97]{Eusebius (in Vita Constant. l. iii. c. 1) chooses
to affirm, that in the reign of this hero, the sword of justice
hung idle in the hands of the magistrates. Eusebius himself, (l.
iv. c. 29, 54,) and the Theodosian Code, will inform us that this
excessive lenity was not owing to the want either of atrocious
criminals or of penal laws.}

The civil administration was sometimes interrupted by the
military defence of the empire. Crispus, a youth of the most
amiable character, who had received with the title of Cæsar the
command of the Rhine, distinguished his conduct, as well as
valor, in several victories over the Franks and Alemanni, and
taught the barbarians of that frontier to dread the eldest son of
Constantine, and the grandson of Constantius.\textsuperscript{98} The emperor
himself had assumed the more difficult and important province of
the Danube. The Goths, who in the time of Claudius and Aurelian
had felt the weight of the Roman arms, respected the power of the
empire, even in the midst of its intestine divisions. But the
strength of that warlike nation was now restored by a peace of
near fifty years; a new generation had arisen, who no longer
remembered the misfortunes of ancient days; the Sarmatians of the
Lake Mæotis followed the Gothic standard either as subjects or as
allies, and their united force was poured upon the countries of
Illyricum. Campona, Margus, and Benonia,\textsuperscript{982} appear to have been
the scenes of several memorable sieges and battles;\textsuperscript{99} and though
Constantine encountered a very obstinate resistance, he prevailed
at length in the contest, and the Goths were compelled to
purchased an ignominious retreat, by restoring the booty and
prisoners which they had taken. Nor was this advantage sufficient
to satisfy the indignation of the emperor. He resolved to
chastise as well as to repulse the insolent barbarians who had
dared to invade the territories of Rome. At the head of his
legions he passed the Danube, after repairing the bridge which
had been constructed by Trajan, penetrated into the strongest
recesses of Dacia,\textsuperscript{100} and when he had inflicted a severe
revenge, condescended to give peace to the suppliant Goths, on
condition that, as often as they were required, they should
supply his armies with a body of forty thousand soldiers.\textsuperscript{101}
Exploits like these were no doubt honorable to Constantine, and
beneficial to the state; but it may surely be questioned, whether
they can justify the exaggerated assertion of Eusebius, that ALL
SCYTHIA, as far as the extremity of the North, divided as it was
into so many names and nations of the most various and savage
manners, had been added by his victorious arms to the Roman
empire.\textsuperscript{102}

\pagenote[98]{Nazarius in Panegyr. Vet. x. The victory of Crispus
over the Alemanni is expressed on some medals. * Note: Other
medals are extant, the legends of which commemorate the success
of Constantine over the Sarmatians and other barbarous nations,
Sarmatia Devicta. Victoria Gothica. Debellatori Gentium
Barbarorum. Exuperator Omnium Gentium. St. Martin, note on Le
Beau, i. 148.—M.}

\pagenote[982]{Campona, Old Buda in Hungary; Margus, Benonia,
Widdin, in Mæsia—G and M.}

\pagenote[99]{See Zosimus, l. ii. p. 93, 94; though the narrative
of that historian is neither clear nor consistent. The Panegyric
of Optatianus (c. 23) mentions the alliance of the Sarmatians
with the Carpi and Getæ, and points out the several fields of
battle. It is supposed that the Sarmatian games, celebrated in
the month of November, derived their origin from the success of
this war.}

\pagenote[100]{In the Cæsars of Julian, (p. 329. Commentaire de
Spanheim, p. 252.) Constantine boasts, that he had recovered the
province (Dacia) which Trajan had subdued. But it is insinuated
by Silenus, that the conquests of Constantine were like the
gardens of Adonis, which fade and wither almost the moment they
appear.}

\pagenote[101]{Jornandes de Rebus Geticis, c. 21. I know not
whether we may entirely depend on his authority. Such an alliance
has a very recent air, and scarcely is suited to the maxims of
the beginning of the fourth century.}

\pagenote[102]{Eusebius in Vit. Constantin. l. i. c. 8. This
passage, however, is taken from a general declamation on the
greatness of Constantine, and not from any particular account of
the Gothic war.}

In this exalted state of glory, it was impossible that
Constantine should any longer endure a partner in the empire.
Confiding in the superiority of his genius and military power, he
determined, without any previous injury, to exert them for the
destruction of Licinius, whose advanced age and unpopular vices
seemed to offer a very easy conquest.\textsuperscript{103} But the old emperor,
awakened by the approaching danger, deceived the expectations of
his friends, as well as of his enemies. Calling forth that spirit
and those abilities by which he had deserved the friendship of
Galerius and the Imperial purple, he prepared himself for the
contest, collected the forces of the East, and soon filled the
plains of Hadrianople with his troops, and the straits of the
Hellespont with his fleet. The army consisted of one hundred and
fifty thousand foot, and fifteen thousand horse; and as the
cavalry was drawn, for the most part, from Phrygia and
Cappadocia, we may conceive a more favorable opinion of the
beauty of the horses, than of the courage and dexterity of their
riders. The fleet was composed of three hundred and fifty galleys
of three ranks of oars. A hundred and thirty of these were
furnished by Egypt and the adjacent coast of Africa. A hundred
and ten sailed from the ports of Phœnicia and the isle of Cyprus;
and the maritime countries of Bithynia, Ionia, and Caria were
likewise obliged to provide a hundred and ten galleys. The troops
of Constantine were ordered to a rendezvous at Thessalonica; they
amounted to above a hundred and twenty thousand horse and foot. \textsuperscript{104}
Their emperor was satisfied with their martial appearance,
and his army contained more soldiers, though fewer men, than that
of his eastern competitor. The legions of Constantine were levied
in the warlike provinces of Europe; action had confirmed their
discipline, victory had elevated their hopes, and there were
among them a great number of veterans, who, after seventeen
glorious campaigns under the same leader, prepared themselves to
deserve an honorable dismission by a last effort of their valor. \textsuperscript{105}
But the naval preparations of Constantine were in every
respect much inferior to those of Licinius. The maritime cities
of Greece sent their respective quotas of men and ships to the
celebrated harbor of Piræus, and their united forces consisted of
no more than two hundred small vessels—a very feeble armament, if
it is compared with those formidable fleets which were equipped
and maintained by the republic of Athens during the Peloponnesian
war.\textsuperscript{106} Since Italy was no longer the seat of government, the
naval establishments of Misenum and Ravenna had been gradually
neglected; and as the shipping and mariners of the empire were
supported by commerce rather than by war, it was natural that
they should the most abound in the industrious provinces of Egypt
and Asia. It is only surprising that the eastern emperor, who
possessed so great a superiority at sea, should have neglected
the opportunity of carrying an offensive war into the centre of
his rival’s dominions.

\pagenote[103]{Constantinus tamen, vir ingens, et omnia efficere
nitens quæ animo præparasset, simul principatum totius urbis
affectans, Licinio bellum intulit. Eutropius, x. 5. Zosimus, l.
ii. p 89. The reasons which they have assigned for the first
civil war, may, with more propriety, be applied to the second.}

\pagenote[104]{Zosimus, l. ii. p. 94, 95.}

\pagenote[105]{Constantine was very attentive to the privileges
and comforts of his fellow-veterans, (Conveterani,) as he now
began to style them. See the Theodosian Code, l. vii. tit. 10,
tom. ii. p. 419, 429.}

\pagenote[106]{Whilst the Athenians maintained the empire of the
sea, their fleet consisted of three, and afterwards of four,
hundred galleys of three ranks of oars, all completely equipped
and ready for immediate service. The arsenal in the port of
Piræus had cost the republic a thousand talents, about two
hundred and sixteen thousand pounds. See Thucydides de Bel.
Pelopon. l. ii. c. 13, and Meursius de Fortuna Attica, c. 19.}

Instead of embracing such an active resolution, which might have
changed the whole face of the war, the prudent Licinius expected
the approach of his rival in a camp near Hadrianople, which he
had fortified with an anxious care that betrayed his apprehension
of the event. Constantine directed his march from Thessalonica
towards that part of Thrace, till he found himself stopped by the
broad and rapid stream of the Hebrus, and discovered the numerous
army of Licinius, which filled the steep ascent of the hill, from
the river to the city of Hadrianople. Many days were spent in
doubtful and distant skirmishes; but at length the obstacles of
the passage and of the attack were removed by the intrepid
conduct of Constantine. In this place we might relate a wonderful
exploit of Constantine, which, though it can scarcely be
paralleled either in poetry or romance, is celebrated, not by a
venal orator devoted to his fortune, but by an historian, the
partial enemy of his fame. We are assured that the valiant
emperor threw himself into the River Hebrus, accompanied only by
\textit{twelve} horsemen, and that by the effort or terror of his
invincible arm, he broke, slaughtered, and put to flight a host
of a hundred and fifty thousand men. The credulity of Zosimus
prevailed so strongly over his passion, that among the events of
the memorable battle of Hadrianople, he seems to have selected
and embellished, not the most important, but the most marvellous.
The valor and danger of Constantine are attested by a slight
wound which he received in the thigh; but it may be discovered
even from an imperfect narration, and perhaps a corrupted text,
that the victory was obtained no less by the conduct of the
general than by the courage of the hero; that a body of five
thousand archers marched round to occupy a thick wood in the rear
of the enemy, whose attention was diverted by the construction of
a bridge, and that Licinius, perplexed by so many artful
evolutions, was reluctantly drawn from his advantageous post to
combat on equal ground on the plain. The contest was no longer
equal. His confused multitude of new levies was easily vanquished
by the experienced veterans of the West. Thirty-four thousand men
are reported to have been slain. The fortified camp of Licinius
was taken by assault the evening of the battle; the greater part
of the fugitives, who had retired to the mountains, surrendered
themselves the next day to the discretion of the conqueror; and
his rival, who could no longer keep the field, confined himself
within the walls of Byzantium.\textsuperscript{107}

\pagenote[107]{Zosimus, l. ii. p. 95, 96. This great battle is
described in the Valesian fragment, (p. 714,) in a clear though
concise manner. “Licinius vero circum Hadrianopolin maximo
exercitu latera ardui montis impleverat; illuc toto agmine
Constantinus inflexit. Cum bellum terra marique traheretur,
quamvis per arduum suis nitentibus, attamen disciplina militari
et felicitate, Constantinus Licinu confusum et sine ordine
agentem vicit exercitum; leviter femore sau ciatus.”}

The siege of Byzantium, which was immediately undertaken by
Constantine, was attended with great labor and uncertainty. In
the late civil wars, the fortifications of that place, so justly
considered as the key of Europe and Asia, had been repaired and
strengthened; and as long as Licinius remained master of the sea,
the garrison was much less exposed to the danger of famine than
the army of the besiegers. The naval commanders of Constantine
were summoned to his camp, and received his positive orders to
force the passage of the Hellespont, as the fleet of Licinius,
instead of seeking and destroying their feeble enemy, continued
inactive in those narrow straits, where its superiority of
numbers was of little use or advantage. Crispus, the emperor’s
eldest son, was intrusted with the execution of this daring
enterprise, which he performed with so much courage and success,
that he deserved the esteem, and most probably excited the
jealousy, of his father. The engagement lasted two days; and in
the evening of the first, the contending fleets, after a
considerable and mutual loss, retired into their respective
harbors of Europe and Asia. The second day, about noon, a strong
south wind\textsuperscript{108} sprang up, which carried the vessels of Crispus
against the enemy; and as the casual advantage was improved by
his skilful intrepidity, he soon obtained a complete victory. A
hundred and thirty vessels were destroyed, five thousand men were
slain, and Amandus, the admiral of the Asiatic fleet, escaped
with the utmost difficulty to the shores of Chalcedon. As soon as
the Hellespont was open, a plentiful convoy of provisions flowed
into the camp of Constantine, who had already advanced the
operations of the siege. He constructed artificial mounds of
earth of an equal height with the ramparts of Byzantium. The
lofty towers which were erected on that foundation galled the
besieged with large stones and darts from the military engines,
and the battering rams had shaken the walls in several places. If
Licinius persisted much longer in the defence, he exposed himself
to be involved in the ruin of the place. Before he was
surrounded, he prudently removed his person and treasures to
Chalcedon in Asia; and as he was always desirous of associating
companions to the hopes and dangers of his fortune, he now
bestowed the title of Cæsar on Martinianus, who exercised one of
the most important offices of the empire.\textsuperscript{109}

\pagenote[108]{Zosimus, l. ii. p. 97, 98. The current always sets
out of the Hellespont; and when it is assisted by a north wind,
no vessel can attempt the passage. A south wind renders the force
of the current almost imperceptible. See Tournefort’s Voyage au
Levant, Let. xi.}

\pagenote[109]{Aurelius Victor. Zosimus, l. ii. p. 93. According
to the latter, Martinianus was Magister Officiorum, (he uses the
Latin appellation in Greek.) Some medals seem to intimate, that
during his short reign he received the title of Augustus.}

Such were still the resources, and such the abilities, of
Licinius, that, after so many successive defeats, he collected in
Bithynia a new army of fifty or sixty thousand men, while the
activity of Constantine was employed in the siege of Byzantium.
The vigilant emperor did not, however, neglect the last struggles
of his antagonist. A considerable part of his victorious army was
transported over the Bosphorus in small vessels, and the decisive
engagement was fought soon after their landing on the heights of
Chrysopolis, or, as it is now called, of Scutari. The troops of
Licinius, though they were lately raised, ill armed, and worse
disciplined, made head against their conquerors with fruitless
but desperate valor, till a total defeat, and a slaughter of five
and twenty thousand men, irretrievably determined the fate of
their leader.\textsuperscript{110} He retired to Nicomedia, rather with the view
of gaining some time for negotiation, than with the hope of any
effectual defence. Constantia, his wife, and the sister of
Constantine, interceded with her brother in favor of her husband,
and obtained from his policy, rather than from his compassion, a
solemn promise, confirmed by an oath, that after the sacrifice of
Martinianus, and the resignation of the purple, Licinius himself
should be permitted to pass the remainder of this life in peace
and affluence. The behavior of Constantia, and her relation to
the contending parties, naturally recalls the remembrance of that
virtuous matron who was the sister of Augustus, and the wife of
Antony. But the temper of mankind was altered, and it was no
longer esteemed infamous for a Roman to survive his honor and
independence. Licinius solicited and accepted the pardon of his
offences, laid himself and his purple at the feet of his \textit{lord}
and \textit{master}, was raised from the ground with insulting pity, was
admitted the same day to the Imperial banquet, and soon
afterwards was sent away to Thessalonica, which had been chosen
for the place of his confinement.\textsuperscript{111} His confinement was soon
terminated by death, and it is doubtful whether a tumult of the
soldiers, or a decree of the senate, was suggested as the motive
for his execution. According to the rules of tyranny, he was
accused of forming a conspiracy, and of holding a treasonable
correspondence with the barbarians; but as he was never
convicted, either by his own conduct or by any legal evidence, we
may perhaps be allowed, from his weakness, to presume his
innocence.\textsuperscript{112} The memory of Licinius was branded with infamy,
his statues were thrown down, and by a hasty edict, of such
mischievous tendency that it was almost immediately corrected,
all his laws, and all the judicial proceedings of his reign, were
at once abolished.\textsuperscript{113} By this victory of Constantine, the Roman
world was again united under the authority of one emperor,
thirty-seven years after Diocletian had divided his power and
provinces with his associate Maximian.

\pagenote[110]{Eusebius (in Vita Constantin. I. ii. c. 16, 17)
ascribes this decisive victory to the pious prayers of the
emperor. The Valesian fragment (p. 714) mentions a body of Gothic
auxiliaries, under their chief Aliquaca, who adhered to the party
of Licinius.}

\pagenote[111]{Zosimus, l. ii. p. 102. Victor Junior in Epitome.
Anonym. Valesian. p. 714.}

\pagenote[112]{Contra religionem sacramenti Thessalonicæ privatus
occisus est. Eutropius, x. 6; and his evidence is confirmed by
Jerome (in Chronic.) as well as by Zosimus, l. ii. p. 102. The
Valesian writer is the only one who mentions the soldiers, and it
is Zonaras alone who calls in the assistance of the senate.
Eusebius prudently slides over this delicate transaction. But
Sozomen, a century afterwards, ventures to assert the treasonable
practices of Licinius.}

\pagenote[113]{See the Theodosian Code, l. xv. tit. 15, tom. v. p
404, 405. These edicts of Constantine betray a degree of passion
and precipitation very unbecoming the character of a lawgiver.}

The successive steps of the elevation of Constantine, from his
first assuming the purple at York, to the resignation of
Licinius, at Nicomedia, have been related with some minuteness
and precision, not only as the events are in themselves both
interesting and important, but still more, as they contributed to
the decline of the empire by the expense of blood and treasure,
and by the perpetual increase, as well of the taxes, as of the
military establishment. The foundation of Constantinople, and the
establishment of the Christian religion, were the immediate and
memorable consequences of this revolution.

