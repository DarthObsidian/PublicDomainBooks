\chapter{The Extent Of The Empire In The Age Of The Antonines.}
\section{Part \thesection.}
\begin{center}
\textbf{\large Introduction.}
\end{center}

\textit{The Extent And Military Force Of The Empire In The Age Of The Antonines.}
\vspace{\onelineskip}

In the second century of the Christian Æra, the empire of Rome
comprehended the fairest part of the earth, and the most
civilized portion of mankind. The frontiers of that extensive
monarchy were guarded by ancient renown and disciplined valor.
The gentle but powerful influence of laws and manners had
gradually cemented the union of the provinces. Their peaceful
inhabitants enjoyed and abused the advantages of wealth and
luxury. The image of a free constitution was preserved with
decent reverence: the Roman senate appeared to possess the
sovereign authority, and devolved on the emperors all the
executive powers of government. During a happy period of more
than fourscore years, the public administration was conducted by
the virtue and abilities of Nerva, Trajan, Hadrian, and the two
Antonines. It is the design of this, and of the two succeeding
chapters, to describe the prosperous condition of their empire;
and afterwards, from the death of Marcus Antoninus, to deduce the
most important circumstances of its decline and fall; a
revolution which will ever be remembered, and is still felt by
the nations of the earth.

The principal conquests of the Romans were achieved under the
republic; and the emperors, for the most part, were satisfied
with preserving those dominions which had been acquired by the
policy of the senate, the active emulations of the consuls, and
the martial enthusiasm of the people. The seven first centuries
were filled with a rapid succession of triumphs; but it was
reserved for Augustus to relinquish the ambitious design of
subduing the whole earth, and to introduce a spirit of moderation
into the public councils. Inclined to peace by his temper and
situation, it was easy for him to discover that Rome, in her
present exalted situation, had much less to hope than to fear
from the chance of arms; and that, in the prosecution of remote
wars, the undertaking became every day more difficult, the event
more doubtful, and the possession more precarious, and less
beneficial. The experience of Augustus added weight to these
salutary reflections, and effectually convinced him that, by the
prudent vigor of his counsels, it would be easy to secure every
concession which the safety or the dignity of Rome might require
from the most formidable barbarians. Instead of exposing his
person and his legions to the arrows of the Parthians, he
obtained, by an honorable treaty, the restitution of the
standards and prisoners which had been taken in the defeat of
Crassus.\textsuperscript{1}

\pagenote[1]{Dion Cassius, (l. liv. p. 736,) with the annotations
of Reimar, who has collected all that Roman vanity has left upon
the subject. The marble of Ancyra, on which Augustus recorded his
own exploits, asserted that \textit{he compelled} the Parthians to
restore the ensigns of Crassus.}

His generals, in the early part of his reign, attempted the
reduction of Ethiopia and Arabia Felix. They marched near a
thousand miles to the south of the tropic; but the heat of the
climate soon repelled the invaders, and protected the un-warlike
natives of those sequestered regions.\textsuperscript{2c} The northern countries
of Europe scarcely deserved the expense and labor of conquest.
The forests and morasses of Germany were filled with a hardy race
of barbarians, who despised life when it was separated from
freedom; and though, on the first attack, they seemed to yield to
the weight of the Roman power, they soon, by a signal act of
despair, regained their independence, and reminded Augustus of
the vicissitude of fortune.\textsuperscript{3a} On the death of that emperor, his
testament was publicly read in the senate. He bequeathed, as a
valuable legacy to his successors, the advice of confining the
empire within those limits which nature seemed to have placed as
its permanent bulwarks and boundaries: on the west, the Atlantic
Ocean; the Rhine and Danube on the north; the Euphrates on the
east; and towards the south, the sandy deserts of Arabia and
Africa.\textsuperscript{4a}

\pagenote[2c]{Strabo, (l. xvi. p. 780,) Pliny the elder, (Hist.
Natur. l. vi. c. 32, 35, [28, 29,]) and Dion Cassius, (l. liii.
p. 723, and l. liv. p. 734,) have left us very curious details
concerning these wars. The Romans made themselves masters of
Mariaba, or Merab, a city of Arabia Felix, well known to the
Orientals. (See Abulfeda and the Nubian geography, p. 52) They
were arrived within three days’ journey of the spice country, the
rich object of their invasion.\\
Note: It is this city of Merab that the Arabs say was the
residence of Belkis, queen of Saba, who desired to see Solomon. A
dam, by which the waters collected in its neighborhood were kept
back, having been swept away, the sudden inundation destroyed
this city, of which, nevertheless, vestiges remain. It bordered
on a country called Adramout, where a particular aromatic plant
grows: it is for this reason that we real in the history of the
Roman expedition, that they were arrived within three days’
journey of the spice country.—G. Compare \textit{Malte-Brun, Geogr}.
Eng. trans. vol. ii. p. 215. The period of this flood has been
copiously discussed by Reiske, (\textit{Program. de vetustâ Epochâ
Arabum, rupturâ cataractæ Merabensis}.) Add. Johannsen, \textit{Hist.
Yemanæ}, p. 282. Bonn, 1828; and see Gibbon, note 16. to Chap.
L.—M.\\
Note: Two, according to Strabo. The detailed account of Strabo
makes the invaders fail before Marsuabæ this cannot be the same
place as Mariaba. Ukert observes, that Ælius Gallus would not
have failed for want of water before Mariaba. (See M. Guizot’s
note above.) “Either, therefore, they were different places, or
Strabo is mistaken.” (Ukert, \textit{Geographie der Griechen und Römer},
vol. i. p. 181.) Strabo, indeed, mentions Mariaba distinct from
Marsuabæ. Gibbon has followed Pliny in reckoning Mariaba among
the conquests of Gallus. There can be little doubt that he is
wrong, as Gallus did not approach the capital of Sabæa. Compare
the note of the Oxford editor of Strabo.—M.}

\pagenote[3a]{By the slaughter of Varus and his three legions.
See the first book of the Annals of Tacitus. Sueton. in August.
c. 23, and Velleius Paterculus, l. ii. c. 117, \&c. Augustus did
not receive the melancholy news with all the temper and firmness
that might have been expected from his character.}

\pagenote[4a]{Tacit. Annal. l. ii. Dion Cassius, l. lvi. p. 833,
and the speech of Augustus himself, in Julian’s Cæsars. It
receives great light from the learned notes of his French
translator, M. Spanheim.}

Happily for the repose of mankind, the moderate system
recommended by the wisdom of Augustus, was adopted by the fears
and vices of his immediate successors. Engaged in the pursuit of
pleasure, or in the exercise of tyranny, the first Cæsars seldom
showed themselves to the armies, or to the provinces; nor were
they disposed to suffer, that those triumphs which \textit{their}
indolence neglected, should be usurped by the conduct and valor
of their lieutenants. The military fame of a subject was
considered as an insolent invasion of the Imperial prerogative;
and it became the duty, as well as interest, of every Roman
general, to guard the frontiers intrusted to his care, without
aspiring to conquests which might have proved no less fatal to
himself than to the vanquished barbarians.\textsuperscript{5}

\pagenote[5]{Germanicus, Suetonius Paulinus, and Agricola were
checked and recalled in the course of their victories. Corbulo
was put to death. Military merit, as it is admirably expressed by
Tacitus, was, in the strictest sense of the word, \textit{imperatoria
virtus}.}

The only accession which the Roman empire received, during the
first century of the Christian Æra, was the province of Britain.
In this single instance, the successors of Cæsar and Augustus
were persuaded to follow the example of the former, rather than
the precept of the latter. The proximity of its situation to the
coast of Gaul seemed to invite their arms; the pleasing though
doubtful intelligence of a pearl fishery attracted their avarice;\textsuperscript{6}
and as Britain was viewed in the light of a distinct and
insulated world, the conquest scarcely formed any exception to
the general system of continental measures. After a war of about
forty years, undertaken by the most stupid,\textsuperscript{7} maintained by the
most dissolute, and terminated by the most timid of all the
emperors, the far greater part of the island submitted to the
Roman yoke.\textsuperscript{8} The various tribes of Britain possessed valor
without conduct, and the love of freedom without the spirit of
union. They took up arms with savage fierceness; they laid them
down, or turned them against each other, with wild inconsistency;
and while they fought singly, they were successively subdued.
Neither the fortitude of Caractacus, nor the despair of Boadicea,
nor the fanaticism of the Druids, could avert the slavery of
their country, or resist the steady progress of the Imperial
generals, who maintained the national glory, when the throne was
disgraced by the weakest, or the most vicious of mankind. At the
very time when Domitian, confined to his palace, felt the terrors
which he inspired, his legions, under the command of the virtuous
Agricola, defeated the collected force of the Caledonians, at the
foot of the Grampian Hills; and his fleets, venturing to explore
an unknown and dangerous navigation, displayed the Roman arms
round every part of the island. The conquest of Britain was
considered as already achieved; and it was the design of Agricola
to complete and insure his success, by the easy reduction of
Ireland, for which, in his opinion, one legion and a few
auxiliaries were sufficient.\textsuperscript{9} The western isle might be improved
into a valuable possession, and the Britons would wear their
chains with the less reluctance, if the prospect and example of
freedom were on every side removed from before their eyes.

\pagenote[6]{Cæsar himself conceals that ignoble motive; but it
is mentioned by Suetonius, c. 47. The British pearls proved,
however, of little value, on account of their dark and livid
color. Tacitus observes, with reason, (in Agricola, c. 12,) that
it was an inherent defect. “Ego facilius crediderim, naturam
margaritis deesse quam nobis avaritiam.”}

\pagenote[7]{Claudius, Nero, and Domitian. A hope is expressed by
Pomponius Mela, l. iii. c. 6, (he wrote under Claudius,) that, by
the success of the Roman arms, the island and its savage
inhabitants would soon be better known. It is amusing enough to
peruse such passages in the midst of London.}

\pagenote[8]{See the admirable abridgment given by Tacitus, in
the life of Agricola, and copiously, though perhaps not
completely, illustrated by our own antiquarians, Camden and
Horsley.}

\pagenote[9]{The Irish writers, jealous of their national honor,
are extremely provoked on this occasion, both with Tacitus and
with Agricola.}

But the superior merit of Agricola soon occasioned his removal
from the government of Britain; and forever disappointed this
rational, though extensive scheme of conquest. Before his
departure, the prudent general had provided for security as well
as for dominion. He had observed, that the island is almost
divided into two unequal parts by the opposite gulfs, or, as they
are now called, the Friths of Scotland. Across the narrow
interval of about forty miles, he had drawn a line of military
stations, which was afterwards fortified, in the reign of
Antoninus Pius, by a turf rampart, erected on foundations of
stone.\textsuperscript{10} This wall of Antoninus, at a small distance beyond the
modern cities of Edinburgh and Glasgow, was fixed as the limit of
the Roman province. The native Caledonians preserved, in the
northern extremity of the island, their wild independence, for
which they were not less indebted to their poverty than to their
valor. Their incursions were frequently repelled and chastised;
but their country was never subdued.\textsuperscript{11} The masters of the
fairest and most wealthy climates of the globe turned with
contempt from gloomy hills, assailed by the winter tempest, from
lakes concealed in a blue mist, and from cold and lonely heaths,
over which the deer of the forest were chased by a troop of naked
barbarians.\textsuperscript{12}

\pagenote[10]{See Horsley’s Britannia Romana, l. i. c. 10. Note:
Agricola fortified the line from Dumbarton to Edinburgh,
consequently within Scotland. The emperor Hadrian, during his
residence in Britain, about the year 121, caused a rampart of
earth to be raised between Newcastle and Carlisle. Antoninus
Pius, having gained new victories over the Caledonians, by the
ability of his general, Lollius, Urbicus, caused a new rampart of
earth to be constructed between Edinburgh and Dumbarton. Lastly,
Septimius Severus caused a wall of stone to be built parallel to
the rampart of Hadrian, and on the same locality. See John
Warburton’s Vallum Romanum, or the History and Antiquities of the
Roman Wall. London, 1754, 4to.—W. See likewise a good note on the
Roman wall in Lingard’s History of England, vol. i. p. 40, 4to
edit—M.}

\pagenote[11]{The poet Buchanan celebrates with elegance and
spirit (see his Sylvæ, v.) the unviolated independence of his
native country. But, if the single testimony of Richard of
Cirencester was sufficient to create a Roman province of
Vespasiana to the north of the wall, that independence would be
reduced within very narrow limits.}

\pagenote[12]{See Appian (in Proœm.) and the uniform imagery of
Ossian’s Poems, which, according to every hypothesis, were
composed by a native Caledonian.}

Such was the state of the Roman frontiers, and such the maxims of
Imperial policy, from the death of Augustus to the accession of
Trajan. That virtuous and active prince had received the
education of a soldier, and possessed the talents of a general.\textsuperscript{13}
The peaceful system of his predecessors was interrupted by
scenes of war and conquest; and the legions, after a long
interval, beheld a military emperor at their head. The first
exploits of Trajan were against the Dacians, the most warlike of
men, who dwelt beyond the Danube, and who, during the reign of
Domitian, had insulted, with impunity, the Majesty of Rome.\textsuperscript{14} To
the strength and fierceness of barbarians they added a contempt
for life, which was derived from a warm persuasion of the
immortality and transmigration of the soul.\textsuperscript{15} Decebalus, the
Dacian king, approved himself a rival not unworthy of Trajan; nor
did he despair of his own and the public fortune, till, by the
confession of his enemies, he had exhausted every resource both
of valor and policy.\textsuperscript{16} This memorable war, with a very short
suspension of hostilities, lasted five years; and as the emperor
could exert, without control, the whole force of the state, it
was terminated by an absolute submission of the barbarians.\textsuperscript{17}
The new province of Dacia, which formed a second exception to the
precept of Augustus, was about thirteen hundred miles in
circumference. Its natural boundaries were the Niester, the Teyss
or Tibiscus, the Lower Danube, and the Euxine Sea. The vestiges
of a military road may still be traced from the banks of the
Danube to the neighborhood of Bender, a place famous in modern
history, and the actual frontier of the Turkish and Russian
empires.\textsuperscript{18}

\pagenote[13]{See Pliny’s Panegyric, which seems founded on
facts.}

\pagenote[14]{Dion Cassius, l. lxvii.}

\pagenote[15]{Herodotus, l. iv. c. 94. Julian in the Cæsars, with
Spanheims observations.}

\pagenote[16]{Plin. Epist. viii. 9.}

\pagenote[17]{Dion Cassius, l. lxviii. p. 1123, 1131. Julian in
Cæsaribus Eutropius, viii. 2, 6. Aurelius Victor in Epitome.}

\pagenote[18]{See a Memoir of M. d’Anville, on the Province of
Dacia, in the Academie des Inscriptions, tom. xxviii. p.
444—468.}

Trajan was ambitious of fame; and as long as mankind shall
continue to bestow more liberal applause on their destroyers than
on their benefactors, the thirst of military glory will ever be
the vice of the most exalted characters. The praises of
Alexander, transmitted by a succession of poets and historians,
had kindled a dangerous emulation in the mind of Trajan. Like
him, the Roman emperor undertook an expedition against the
nations of the East; but he lamented with a sigh, that his
advanced age scarcely left him any hopes of equalling the renown
of the son of Philip.\textsuperscript{19} Yet the success of Trajan, however
transient, was rapid and specious. The degenerate Parthians,
broken by intestine discord, fled before his arms. He descended
the River Tigris in triumph, from the mountains of Armenia to the
Persian Gulf. He enjoyed the honor of being the first, as he was
the last, of the Roman generals, who ever navigated that remote
sea. His fleets ravaged the coast of Arabia; and Trajan vainly
flattered himself that he was approaching towards the confines of
India.\textsuperscript{20} Every day the astonished senate received the
intelligence of new names and new nations, that acknowledged his
sway. They were informed that the kings of Bosphorus, Colchos,
Iberia, Albania, Osrhoene, and even the Parthian monarch himself,
had accepted their diadems from the hands of the emperor; that
the independent tribes of the Median and Carduchian hills had
implored his protection; and that the rich countries of Armenia,
Mesopotamia, and Assyria, were reduced into the state of
provinces.\textsuperscript{21} But the death of Trajan soon clouded the splendid
prospect; and it was justly to be dreaded, that so many distant
nations would throw off the unaccustomed yoke, when they were no
longer restrained by the powerful hand which had imposed it.

\pagenote[19]{Trajan’s sentiments are represented in a very just
and lively manner in the Cæsars of Julian.}

\pagenote[20]{Eutropius and Sextus Rufus have endeavored to
perpetuate the illusion. See a very sensible dissertation of M.
Freret in the Académie des Inscriptions, tom. xxi. p. 55.}

\pagenote[21]{Dion Cassius, l. lxviii.; and the Abbreviators.}

\section{Part \thesection.}

It was an ancient tradition, that when the Capitol was founded by
one of the Roman kings, the god Terminus (who presided over
boundaries, and was represented, according to the fashion of that
age, by a large stone) alone, among all the inferior deities,
refused to yield his place to Jupiter himself. A favorable
inference was drawn from his obstinacy, which was interpreted by
the augurs as a sure presage that the boundaries of the Roman
power would never recede.\textsuperscript{22} During many ages, the prediction, as
it is usual, contributed to its own accomplishment. But though
Terminus had resisted the Majesty of Jupiter, he submitted to the
authority of the emperor Hadrian.\textsuperscript{23} The resignation of all the
eastern conquests of Trajan was the first measure of his reign.
He restored to the Parthians the election of an independent
sovereign; withdrew the Roman garrisons from the provinces of
Armenia, Mesopotamia, and Assyria; and, in compliance with the
precept of Augustus, once more established the Euphrates as the
frontier of the empire.\textsuperscript{24} Censure, which arraigns the public
actions and the private motives of princes, has ascribed to envy,
a conduct which might be attributed to the prudence and
moderation of Hadrian. The various character of that emperor,
capable, by turns, of the meanest and the most generous
sentiments, may afford some color to the suspicion. It was,
however, scarcely in his power to place the superiority of his
predecessor in a more conspicuous light, than by thus confessing
himself unequal to the task of defending the conquests of Trajan.

\pagenote[22]{Ovid. Fast. l. ii. ver. 667. See Livy, and
Dionysius of Halicarnassus, under the reign of Tarquin.}

\pagenote[23]{St. Augustin is highly delighted with the proof of
the weakness of Terminus, and the vanity of the Augurs. See De
Civitate Dei, iv. 29. * Note: The turn of Gibbon’s sentence is
Augustin’s: “Plus Hadrianum regem hominum, quam regem Deorum
timuisse videatur.”—M}

\pagenote[24]{See the Augustan History, p. 5, Jerome’s Chronicle,
and all the Epitomizers. It is somewhat surprising, that this
memorable event should be omitted by Dion, or rather by
Xiphilin.}

The martial and ambitious spirit of Trajan formed a very singular
contrast with the moderation of his successor. The restless
activity of Hadrian was not less remarkable when compared with
the gentle repose of Antoninus Pius. The life of the former was
almost a perpetual journey; and as he possessed the various
talents of the soldier, the statesman, and the scholar, he
gratified his curiosity in the discharge of his duty.

Careless of the difference of seasons and of climates, he marched
on foot, and bare-headed, over the snows of Caledonia, and the
sultry plains of the Upper Egypt; nor was there a province of the
empire which, in the course of his reign, was not honored with
the presence of the monarch.\textsuperscript{25} But the tranquil life of
Antoninus Pius was spent in the bosom of Italy, and, during the
twenty-three years that he directed the public administration,
the longest journeys of that amiable prince extended no farther
than from his palace in Rome to the retirement of his Lanuvian
villa.\textsuperscript{26}

\pagenote[25]{Dion, l. lxix. p. 1158. Hist. August. p. 5, 8. If
all our historians were lost, medals, inscriptions, and other
monuments, would be sufficient to record the onecolumonecolumnf Hadrian.
Note: The journeys of Hadrian are traced in a note on Solvet’s
translation of Hegewisch, Essai sur l’Epoque de Histoire Romaine
la plus heureuse pour Genre Humain Paris, 1834, p. 123.—M.}

\pagenote[26]{See the Augustan History and the Epitomes.}

Notwithstanding this difference in their personal conduct, the
general system of Augustus was equally adopted and uniformly
pursued by Hadrian and by the two Antonines. They persisted in
the design of maintaining the dignity of the empire, without
attempting to enlarge its limits. By every honorable expedient
they invited the friendship of the barbarians; and endeavored to
convince mankind that the Roman power, raised above the
temptation of conquest, was actuated only by the love of order
and justice. During a long period of forty-three years, their
virtuous labors were crowned with success; and if we except a few
slight hostilities, that served to exercise the legions of the
frontier, the reigns of Hadrian and Antoninus Pius offer the fair
prospect of universal peace.\textsuperscript{27} The Roman name was revered among
the most remote nations of the earth. The fiercest barbarians
frequently submitted their differences to the arbitration of the
emperor; and we are informed by a contemporary historian that he
had seen ambassadors who were refused the honor which they came
to solicit of being admitted into the rank of subjects.\textsuperscript{28}

\pagenote[27]{We must, however, remember, that in the time of
Hadrian, a rebellion of the Jews raged with religious fury,
though only in a single province. Pausanias (l. viii. c. 43)
mentions two necessary and successful wars, conducted by the
generals of Pius: 1st. Against the wandering Moors, who were
driven into the solitudes of Atlas. 2d. Against the Brigantes of
Britain, who had invaded the Roman province. Both these wars
(with several other hostilities) are mentioned in the Augustan
History, p. 19.}

\pagenote[28]{Appian of Alexandria, in the preface to his History
of the Roman Wars.}

The terror of the Roman arms added weight and dignity to the
moderation of the emperors. They preserved peace by a constant
preparation for war; and while justice regulated their conduct,
they announced to the nations on their confines, that they were
as little disposed to endure, as to offer an injury. The military
strength, which it had been sufficient for Hadrian and the elder
Antoninus to display, was exerted against the Parthians and the
Germans by the emperor Marcus. The hostilities of the barbarians
provoked the resentment of that philosophic monarch, and, in the
prosecution of a just defence, Marcus and his generals obtained
many signal victories, both on the Euphrates and on the Danube.\textsuperscript{29}
The military establishment of the Roman empire, which thus
assured either its tranquillity or success, will now become the
proper and important object of our attention.

\pagenote[29]{Dion, l. lxxi. Hist. August. in Marco. The Parthian
victories gave birth to a crowd of contemptible historians, whose
memory has been rescued from oblivion and exposed to ridicule, in
a very lively piece of criticism of Lucian.}

In the purer ages of the commonwealth, the use of arms was
reserved for those ranks of citizens who had a country to love, a
property to defend, and some share in enacting those laws, which
it was their interest as well as duty to maintain. But in
proportion as the public freedom was lost in extent of conquest,
war was gradually improved into an art, and degraded into a
trade.\textsuperscript{30} The legions themselves, even at the time when they were
recruited in the most distant provinces, were supposed to consist
of Roman citizens. That distinction was generally considered,
either as a legal qualification or as a proper recompense for the
soldier; but a more serious regard was paid to the essential
merit of age, strength, and military stature.\textsuperscript{31} In all levies, a
just preference was given to the climates of the North over those
of the South: the race of men born to the exercise of arms was
sought for in the country rather than in cities; and it was very
reasonably presumed, that the hardy occupations of smiths,
carpenters, and huntsmen, would supply more vigor and resolution
than the sedentary trades which are employed in the service of
luxury.\textsuperscript{32} After every qualification of property had been laid
aside, the armies of the Roman emperors were still commanded, for
the most part, by officers of liberal birth and education; but
the common soldiers, like the mercenary troops of modern Europe,
were drawn from the meanest, and very frequently from the most
profligate, of mankind.

\pagenote[30]{The poorest rank of soldiers possessed above forty
pounds sterling, (Dionys. Halicarn. iv. 17,) a very high
qualification at a time when money was so scarce, that an ounce
of silver was equivalent to seventy pounds weight of brass. The
populace, excluded by the ancient constitution, were
indiscriminately admitted by Marius. See Sallust. de Bell.
Jugurth. c. 91. * Note: On the uncertainty of all these
estimates, and the difficulty of fixing the relative value of
brass and silver, compare Niebuhr, vol. i. p. 473, \&c. Eng.
trans. p. 452. According to Niebuhr, the relative disproportion
in value, between the two metals, arose, in a great degree from
the abundance of brass or copper.—M. Compare also Dureau ‘de la
Malle Economie Politique des Romains especially L. l. c. ix.—M.
1845.}

\pagenote[31]{Cæsar formed his legion Alauda of Gauls and
strangers; but it was during the license of civil war; and after
the victory, he gave them the freedom of the city for their
reward.}

\pagenote[32]{See Vegetius, de Re Militari, l. i. c. 2—7.}

That public virtue, which among the ancients was denominated
patriotism, is derived from a strong sense of our own interest in
the preservation and prosperity of the free government of which
we are members. Such a sentiment, which had rendered the legions
of the republic almost invincible, could make but a very feeble
impression on the mercenary servants of a despotic prince; and it
became necessary to supply that defect by other motives, of a
different, but not less forcible nature—honor and religion. The
peasant, or mechanic, imbibed the useful prejudice that he was
advanced to the more dignified profession of arms, in which his
rank and reputation would depend on his own valor; and that,
although the prowess of a private soldier must often escape the
notice of fame, his own behavior might sometimes confer glory or
disgrace on the company, the legion, or even the army, to whose
honors he was associated. On his first entrance into the service,
an oath was administered to him with every circumstance of
solemnity. He promised never to desert his standard, to submit
his own will to the commands of his leaders, and to sacrifice his
life for the safety of the emperor and the empire.\textsuperscript{33} The
attachment of the Roman troops to their standards was inspired by
the united influence of religion and of honor. The golden eagle,
which glittered in the front of the legion, was the object of
their fondest devotion; nor was it esteemed less impious than it
was ignominious, to abandon that sacred ensign in the hour of
danger.\textsuperscript{34} These motives, which derived their strength from the
imagination, were enforced by fears and hopes of a more
substantial kind. Regular pay, occasional donatives, and a stated
recompense, after the appointed time of service, alleviated the
hardships of the military life,\textsuperscript{35} whilst, on the other hand, it
was impossible for cowardice or disobedience to escape the
severest punishment. The centurions were authorized to chastise
with blows, the generals had a right to punish with death; and it
was an inflexible maxim of Roman discipline, that a good soldier
should dread his officers far more than the enemy. From such
laudable arts did the valor of the Imperial troops receive a
degree of firmness and docility unattainable by the impetuous and
irregular passions of barbarians.

\pagenote[33]{The oath of service and fidelity to the emperor was
annually renewed by the troops on the first of January.}

\pagenote[34]{Tacitus calls the Roman eagles, Bellorum Deos. They
were placed in a chapel in the camp, and with the other deities
received the religious worship of the troops. * Note: See also
Dio. Cass. xl. c. 18. —M.}

\pagenote[35]{See Gronovius de Pecunia vetere, l. iii. p. 120,
\&c. The emperor Domitian raised the annual stipend of the
legionaries to twelve pieces of gold, which, in his time, was
equivalent to about ten of our guineas. This pay, somewhat higher
than our own, had been, and was afterwards, gradually increased,
according to the progress of wealth and military government.
After twenty years’ service, the veteran received three thousand
denarii, (about one hundred pounds sterling,) or a proportionable
allowance of land. The pay and advantages of the guards were, in
general, about double those of the legions.}

And yet so sensible were the Romans of the imperfection of valor
without skill and practice, that, in their language, the name of
an army was borrowed from the word which signified exercise.\textsuperscript{36}
Military exercises were the important and unremitted object of
their discipline. The recruits and young soldiers were constantly
trained, both in the morning and in the evening, nor was age or
knowledge allowed to excuse the veterans from the daily
repetition of what they had completely learnt. Large sheds were
erected in the winter-quarters of the troops, that their useful
labors might not receive any interruption from the most
tempestuous weather; and it was carefully observed, that the arms
destined to this imitation of war, should be of double the weight
which was required in real action.\textsuperscript{37} It is not the purpose of
this work to enter into any minute description of the Roman
exercises. We shall only remark, that they comprehended whatever
could add strength to the body, activity to the limbs, or grace
to the motions. The soldiers were diligently instructed to march,
to run, to leap, to swim, to carry heavy burdens, to handle every
species of arms that was used either for offence or for defence,
either in distant engagement or in a closer onset; to form a
variety of evolutions; and to move to the sound of flutes in the
Pyrrhic or martial dance.\textsuperscript{38} In the midst of peace, the Roman
troops familiarized themselves with the practice of war; and it
is prettily remarked by an ancient historian who had fought
against them, that the effusion of blood was the only
circumstance which distinguished a field of battle from a field
of exercise.\textsuperscript{39} It was the policy of the ablest generals, and
even of the emperors themselves, to encourage these military
studies by their presence and example; and we are informed that
Hadrian, as well as Trajan, frequently condescended to instruct
the unexperienced soldiers, to reward the diligent, and sometimes
to dispute with them the prize of superior strength or dexterity.\textsuperscript{40}
Under the reigns of those princes, the science of tactics was
cultivated with success; and as long as the empire retained any
vigor, their military instructions were respected as the most
perfect model of Roman discipline.

\pagenote[36]{\textit{Exercitus ab exercitando}, Varro de Lingua Latina,
l. iv. Cicero in Tusculan. l. ii. 37. 15. There is room for a
very interesting work, which should lay open the connection
between the languages and manners of nations. * Note I am not
aware of the existence, at present, of such a work; but the
profound observations of the late William von Humboldt, in the
introduction to his posthumously published Essay on the Language
of the Island of Java, (uber die Kawi-sprache, Berlin, 1836,) may
cause regret that this task was not completed by that
accomplished and universal scholar.—M.}

\pagenote[37]{Vegatius, l. ii. and the rest of his first book.}

\pagenote[38]{The Pyrrhic dance is extremely well illustrated by
M. le Beau, in the Academie des Inscriptions, tom. xxxv. p. 262,
\&c. That learned academician, in a series of memoirs, has
collected all the passages of the ancients that relate to the
Roman legion.}

\pagenote[39]{Joseph. de Bell. Judaico, l. iii. c. 5. We are
indebted to this Jew for some very curious details of Roman
discipline.}

\pagenote[40]{Plin. Panegyr. c. 13. Life of Hadrian, in the
Augustan History.}

Nine centuries of war had gradually introduced into the service
many alterations and improvements. The legions, as they are
described by Polybius,\textsuperscript{41} in the time of the Punic wars, differed
very materially from those which achieved the victories of Cæsar,
or defended the monarchy of Hadrian and the Antonines. The
constitution of the Imperial legion may be described in a few
words.\textsuperscript{42} The heavy-armed infantry, which composed its principal
strength,\textsuperscript{43} was divided into ten cohorts, and fifty-five
companies, under the orders of a correspondent number of tribunes
and centurions. The first cohort, which always claimed the post
of honor and the custody of the eagle, was formed of eleven
hundred and five soldiers, the most approved for valor and
fidelity. The remaining nine cohorts consisted each of five
hundred and fifty-five; and the whole body of legionary infantry
amounted to six thousand one hundred men. Their arms were
uniform, and admirably adapted to the nature of their service: an
open helmet, with a lofty crest; a breastplate, or coat of mail;
greaves on their legs, and an ample buckler on their left arm.
The buckler was of an oblong and concave figure, four feet in
length, and two and a half in breadth, framed of a light wood,
covered with a bull’s hide, and strongly guarded with plates of
brass. Besides a lighter spear, the legionary soldier grasped in
his right hand the formidable \textit{pilum}, a ponderous javelin, whose
utmost length was about six feet, and which was terminated by a
massy triangular point of steel of eighteen inches.\textsuperscript{44} This
instrument was indeed much inferior to our modern fire-arms;
since it was exhausted by a single discharge, at the distance of
only ten or twelve paces. Yet when it was launched by a firm and
skilful hand, there was not any cavalry that durst venture within
its reach, nor any shield or corselet that could sustain the
impetuosity of its weight. As soon as the Roman had darted his
\textit{pilum}, he drew his sword, and rushed forwards to close with the
enemy. His sword was a short well-tempered Spanish blade, that
carried a double edge, and was alike suited to the purpose of
striking or of pushing; but the soldier was always instructed to
prefer the latter use of his weapon, as his own body remained
less exposed, whilst he inflicted a more dangerous wound on his
adversary.\textsuperscript{45} The legion was usually drawn up eight deep; and the
regular distance of three feet was left between the files as well
as ranks.\textsuperscript{46} A body of troops, habituated to preserve this open
order, in a long front and a rapid charge, found themselves
prepared to execute every disposition which the circumstances of
war, or the skill of their leader, might suggest. The soldier
possessed a free space for his arms and motions, and sufficient
intervals were allowed, through which seasonable reinforcements
might be introduced to the relief of the exhausted combatants.\textsuperscript{47}
The tactics of the Greeks and Macedonians were formed on very
different principles. The strength of the phalanx depended on
sixteen ranks of long pikes, wedged together in the closest
array.\textsuperscript{48} But it was soon discovered by reflection, as well as by
the event, that the strength of the phalanx was unable to contend
with the activity of the legion.\textsuperscript{49}

\pagenote[41]{See an admirable digression on the Roman
discipline, in the sixth book of his History.}

\pagenote[42]{Vegetius de Re Militari, l. ii. c. 4, \&c.
Considerable part of his very perplexed abridgment was taken from
the regulations of Trajan and Hadrian; and the legion, as he
describes it, cannot suit any other age of the Roman empire.}

\pagenote[43]{Vegetius de Re Militari, l. ii. c. 1. In the purer
age of Cæsar and Cicero, the word miles was almost confined to
the infantry. Under the lower empire, and the times of chivalry,
it was appropriated almost as exclusively to the men at arms, who
fought on horseback.}

\pagenote[44]{In the time of Polybius and Dionysius of
Halicarnassus, (l. v. c. 45,) the steel point of the pilum seems
to have been much longer. In the time of Vegetius, it was reduced
to a foot, or even nine inches. I have chosen a medium.}

\pagenote[45]{For the legionary arms, see Lipsius de Militia
Romana, l. iii. c. 2—7.}

\pagenote[46]{See the beautiful comparison of Virgil, Georgic ii.
v. 279.}

\pagenote[47]{M. Guichard, Memoires Militaires, tom. i. c. 4, and
Nouveaux Memoires, tom. i. p. 293—311, has treated the subject
like a scholar and an officer.}

\pagenote[48]{See Arrian’s Tactics. With the true partiality of a
Greek, Arrian rather chose to describe the phalanx, of which he
had read, than the legions which he had commanded.}

\pagenote[49]{Polyb. l. xvii. (xviii. 9.)}

The cavalry, without which the force of the legion would have
remained imperfect, was divided into ten troops or squadrons; the
first, as the companion of the first cohort, consisted of a
hundred and thirty-two men; whilst each of the other nine
amounted only to sixty-six. The entire establishment formed a
regiment, if we may use the modern expression, of seven hundred
and twenty-six horse, naturally connected with its respective
legion, but occasionally separated to act in the line, and to
compose a part of the wings of the army. \textsuperscript{50} The cavalry of the
emperors was no longer composed, like that of the ancient
republic, of the noblest youths of Rome and Italy, who, by
performing their military service on horseback, prepared
themselves for the offices of senator and consul; and solicited,
by deeds of valor, the future suffrages of their countrymen.\textsuperscript{51}
Since the alteration of manners and government, the most wealthy
of the equestrian order were engaged in the administration of
justice, and of the revenue;\textsuperscript{52} and whenever they embraced the
profession of arms, they were immediately intrusted with a troop
of horse, or a cohort of foot.\textsuperscript{53} Trajan and Hadrian formed their
cavalry from the same provinces, and the same class of their
subjects, which recruited the ranks of the legion. The horses
were bred, for the most part, in Spain or Cappadocia. The Roman
troopers despised the complete armor with which the cavalry of
the East was encumbered. \textit{Their} more useful arms consisted in a
helmet, an oblong shield, light boots, and a coat of mail. A
javelin, and a long broad sword, were their principal weapons of
offence. The use of lances and of iron maces they seem to have
borrowed from the barbarians.\textsuperscript{54}

\pagenote[50]{Veget. de Re Militari, l. ii. c. 6. His positive
testimony, which might be supported by circumstantial evidence,
ought surely to silence those critics who refuse the Imperial
legion its proper body of cavalry. Note: See also Joseph. B. J.
iii. vi. 2.—M.}

\pagenote[51]{See Livy almost throughout, particularly xlii. 61.}

\pagenote[52]{Plin. Hist. Natur. xxxiii. 2. The true sense of
that very curious passage was first discovered and illustrated by
M. de Beaufort, Republique Romaine, l. ii. c. 2.}

\pagenote[53]{As in the instance of Horace and Agricola. This
appears to have been a defect in the Roman discipline; which
Hadrian endeavored to remedy by ascertaining the legal age of a
tribune. * Note: These details are not altogether accurate.
Although, in the latter days of the republic, and under the first
emperors, the young Roman nobles obtained the command of a
squadron or a cohort with greater facility than in the former
times, they never obtained it without passing through a tolerably
long military service. Usually they served first in the prætorian
cohort, which was intrusted with the guard of the general: they
were received into the companionship (contubernium) of some
superior officer, and were there formed for duty. Thus Julius
Cæsar, though sprung from a great family, served first as
contubernalis under the prætor, M. Thermus, and later under
Servilius the Isaurian. (Suet. Jul. 2, 5. Plut. in Par. p. 516.
Ed. Froben.) The example of Horace, which Gibbon adduces to prove
that young knights were made tribunes immediately on entering the
service, proves nothing. In the first place, Horace was not a
knight; he was the son of a freedman of Venusia, in Apulia, who
exercised the humble office of coactor exauctionum, (collector of
payments at auctions.) (Sat. i. vi. 45, or 86.) Moreover, when
the poet was made tribune, Brutus, whose army was nearly entirely
composed of Orientals, gave this title to all the Romans of
consideration who joined him. The emperors were still less
difficult in their choice; the number of tribunes was augmented;
the title and honors were conferred on persons whom they wished
to attack to the court. Augustus conferred on the sons of
senators, sometimes the tribunate, sometimes the command of a
squadron. Claudius gave to the knights who entered into the
service, first the command of a cohort of auxiliaries, later that
of a squadron, and at length, for the first time, the tribunate.
(Suet in Claud. with the notes of Ernesti.) The abuses that arose
caused by the edict of Hadrian, which fixed the age at which that
honor could be attained. (Spart. in Had. \&c.) This edict was
subsequently obeyed; for the emperor Valerian, in a letter
addressed to Mulvius Gallinnus, prætorian præfect, excuses
himself for having violated it in favor of the young Probus
afterwards emperor, on whom he had conferred the tribunate at an
earlier age on account of his rare talents. (Vopisc. in Prob.
iv.)—W. and G. Agricola, though already invested with the title
of tribune, was contubernalis in Britain with Suetonius Paulinus.
Tac. Agr. v.—M.}

\pagenote[54]{See Arrian’s Tactics.}

The safety and honor of the empire was principally intrusted to
the legions, but the policy of Rome condescended to adopt every
useful instrument of war. Considerable levies were regularly made
among the provincials, who had not yet deserved the honorable
distinction of Romans. Many dependent princes and communities,
dispersed round the frontiers, were permitted, for a while, to
hold their freedom and security by the tenure of military
service.\textsuperscript{55} Even select troops of hostile barbarians were
frequently compelled or persuaded to consume their dangerous
valor in remote climates, and for the benefit of the state.\textsuperscript{56}
All these were included under the general name of auxiliaries;
and howsoever they might vary according to the difference of
times and circumstances, their numbers were seldom much inferior
to those of the legions themselves.\textsuperscript{57} Among the auxiliaries, the
bravest and most faithful bands were placed under the command of
præfects and centurions, and severely trained in the arts of
Roman discipline; but the far greater part retained those arms,
to which the nature of their country, or their early habits of
life, more peculiarly adapted them. By this institution, each
legion, to whom a certain proportion of auxiliaries was allotted,
contained within itself every species of lighter troops, and of
missile weapons; and was capable of encountering every nation,
with the advantages of its respective arms and discipline.\textsuperscript{58} Nor
was the legion destitute of what, in modern language, would be
styled a train of artillery. It consisted in ten military engines
of the largest, and fifty-five of a smaller size; but all of
which, either in an oblique or horizontal manner, discharged
stones and darts with irresistible violence.\textsuperscript{59}

\pagenote[55]{Such, in particular, was the state of the
Batavians. Tacit. Germania, c. 29.}

\pagenote[56]{Marcus Antoninus obliged the vanquished Quadi and
Marcomanni to supply him with a large body of troops, which he
immediately sent into Britain. Dion Cassius, l. lxxi. (c. 16.)}

\pagenote[57]{Tacit. Annal. iv. 5. Those who fix a regular
proportion of as many foot, and twice as many horse, confound the
auxiliaries of the emperors with the Italian allies of the
republic.}

\pagenote[58]{Vegetius, ii. 2. Arrian, in his order of march and
battle against the Alani.}

\pagenote[59]{The subject of the ancient machines is treated with
great knowledge and ingenuity by the Chevalier Folard, (Polybe,
tom. ii. p. 233-290.) He prefers them in many respects to our
modern cannon and mortars. We may observe, that the use of them
in the field gradually became more prevalent, in proportion as
personal valor and military skill declined with the Roman empire.
When men were no longer found, their place was supplied by
machines. See Vegetius, ii. 25. Arrian.}

\section{Part \thesection.}

The camp of a Roman legion presented the appearance of a
fortified city.\textsuperscript{60} As soon as the space was marked out, the
pioneers carefully levelled the ground, and removed every
impediment that might interrupt its perfect regularity. Its form
was an exact quadrangle; and we may calculate, that a square of
about seven hundred yards was sufficient for the encampment of
twenty thousand Romans; though a similar number of our own troops
would expose to the enemy a front of more than treble that
extent. In the midst of the camp, the prætorium, or general’s
quarters, rose above the others; the cavalry, the infantry, and
the auxiliaries occupied their respective stations; the streets
were broad and perfectly straight, and a vacant space of two
hundred feet was left on all sides between the tents and the
rampart. The rampart itself was usually twelve feet high, armed
with a line of strong and intricate palisades, and defended by a
ditch of twelve feet in depth as well as in breadth. This
important labor was performed by the hands of the legionaries
themselves; to whom the use of the spade and the pickaxe was no
less familiar than that of the sword or \textit{pilum}. Active valor may
often be the present of nature; but such patient diligence can be
the fruit only of habit and discipline.\textsuperscript{61}

\pagenote[60]{Vegetius finishes his second book, and the
description of the legion, with the following emphatic
words:—“Universa quæ in quoque belli genere necessaria esse
creduntur, secum legio debet ubique portare, ut in quovis loco
fixerit castra, armatam faciat civitatem.”}

\pagenote[61]{For the Roman Castrametation, see Polybius, l. vi.
with Lipsius de Militia Romana, Joseph. de Bell. Jud. l. iii. c.
5. Vegetius, i. 21—25, iii. 9, and Memoires de Guichard, tom. i.
c. 1.}

Whenever the trumpet gave the signal of departure, the camp was
almost instantly broke up, and the troops fell into their ranks
without delay or confusion. Besides their arms, which the
legionaries scarcely considered as an encumbrance, they were
laden with their kitchen furniture, the instruments of
fortification, and the provision of many days.\textsuperscript{62} Under this
weight, which would oppress the delicacy of a modern soldier,
they were trained by a regular step to advance, in about six
hours, near twenty miles.\textsuperscript{63} On the appearance of an enemy, they
threw aside their baggage, and by easy and rapid evolutions
converted the column of march into an order of battle.\textsuperscript{64} The
slingers and archers skirmished in the front; the auxiliaries
formed the first line, and were seconded or sustained by the
strength of the legions; the cavalry covered the flanks, and the
military engines were placed in the rear.

\pagenote[62]{Cicero in Tusculan. ii. 37, [15.]—Joseph. de Bell.
Jud. l. iii. 5, Frontinus, iv. 1.}

\pagenote[63]{Vegetius, i. 9. See Memoires de l’Academie des
Inscriptions, tom. xxv. p. 187.}

\pagenote[64]{See those evolutions admirably well explained by M.
Guichard Nouveaux Memoires, tom. i. p. 141—234.}

Such were the arts of war, by which the Roman emperors defended
their extensive conquests, and preserved a military spirit, at a
time when every other virtue was oppressed by luxury and
despotism. If, in the consideration of their armies, we pass from
their discipline to their numbers, we shall not find it easy to
define them with any tolerable accuracy. We may compute, however,
that the legion, which was itself a body of six thousand eight
hundred and thirty-one Romans, might, with its attendant
auxiliaries, amount to about twelve thousand five hundred men.
The peace establishment of Hadrian and his successors was
composed of no less than thirty of these formidable brigades; and
most probably formed a standing force of three hundred and
seventy-five thousand men. Instead of being confined within the
walls of fortified cities, which the Romans considered as the
refuge of weakness or pusillanimity, the legions were encamped on
the banks of the great rivers, and along the frontiers of the
barbarians. As their stations, for the most part, remained fixed
and permanent, we may venture to describe the distribution of the
troops. Three legions were sufficient for Britain. The principal
strength lay upon the Rhine and Danube, and consisted of sixteen
legions, in the following proportions: two in the Lower, and
three in the Upper Germany; one in Rhætia, one in Noricum, four
in Pannonia, three in Mæsia, and two in Dacia. The defence of the
Euphrates was intrusted to eight legions, six of whom were
planted in Syria, and the other two in Cappadocia. With regard to
Egypt, Africa, and Spain, as they were far removed from any
important scene of war, a single legion maintained the domestic
tranquillity of each of those great provinces. Even Italy was not
left destitute of a military force. Above twenty thousand chosen
soldiers, distinguished by the titles of City Cohorts and
Prætorian Guards, watched over the safety of the monarch and the
capital. As the authors of almost every revolution that
distracted the empire, the Prætorians will, very soon, and very
loudly, demand our attention; but, in their arms and
institutions, we cannot find any circumstance which discriminated
them from the legions, unless it were a more splendid appearance,
and a less rigid discipline.\textsuperscript{65}

\pagenote[65]{Tacitus (Annal. iv. 5) has given us a state of the
legions under Tiberius; and Dion Cassius (l. lv. p. 794) under
Alexander Severus. I have endeavored to fix on the proper medium
between these two periods. See likewise Lipsius de Magnitudine
Romana, l. i. c. 4, 5.}

The navy maintained by the emperors might seem inadequate to
their greatness; but it was fully sufficient for every useful
purpose of government. The ambition of the Romans was confined to
the land; nor was that warlike people ever actuated by the
enterprising spirit which had prompted the navigators of Tyre, of
Carthage, and even of Marseilles, to enlarge the bounds of the
world, and to explore the most remote coasts of the ocean. To the
Romans the ocean remained an object of terror rather than of
curiosity;\textsuperscript{66} the whole extent of the Mediterranean, after the
destruction of Carthage, and the extirpation of the pirates, was
included within their provinces. The policy of the emperors was
directed only to preserve the peaceful dominion of that sea, and
to protect the commerce of their subjects. With these moderate
views, Augustus stationed two permanent fleets in the most
convenient ports of Italy, the one at Ravenna, on the Adriatic,
the other at Misenum, in the Bay of Naples. Experience seems at
length to have convinced the ancients, that as soon as their
galleys exceeded two, or at the most three ranks of oars, they
were suited rather for vain pomp than for real service. Augustus
himself, in the victory of Actium, had seen the superiority of
his own light frigates (they were called Liburnians) over the
lofty but unwieldy castles of his rival.\textsuperscript{67} Of these Liburnians
he composed the two fleets of Ravenna and Misenum, destined to
command, the one the eastern, the other the western division of
the Mediterranean; and to each of the squadrons he attached a
body of several thousand marines. Besides these two ports, which
may be considered as the principal seats of the Roman navy, a
very considerable force was stationed at Frejus, on the coast of
Provence, and the Euxine was guarded by forty ships, and three
thousand soldiers. To all these we add the fleet which preserved
the communication between Gaul and Britain, and a great number of
vessels constantly maintained on the Rhine and Danube, to harass
the country, or to intercept the passage of the barbarians.\textsuperscript{68} If
we review this general state of the Imperial forces; of the
cavalry as well as infantry; of the legions, the auxiliaries, the
guards, and the navy; the most liberal computation will not allow
us to fix the entire establishment by sea and by land at more
than four hundred and fifty thousand men: a military power,
which, however formidable it may seem, was equalled by a monarch
of the last century, whose kingdom was confined within a single
province of the Roman empire.\textsuperscript{69}

\pagenote[66]{The Romans tried to disguise, by the pretence of
religious awe their ignorance and terror. See Tacit. Germania, c.
34.}

\pagenote[67]{Plutarch, in Marc. Anton. [c. 67.] And yet, if we
may credit Orosius, these monstrous castles were no more than ten
feet above the water, vi. 19.}

\pagenote[68]{See Lipsius, de Magnitud. Rom. l. i. c. 5. The
sixteen last chapters of Vegetius relate to naval affairs.}

\pagenote[69]{Voltaire, Siecle de Louis XIV. c. 29. It must,
however, be remembered, that France still feels that
extraordinary effort.}

We have attempted to explain the spirit which moderated, and the
strength which supported, the power of Hadrian and the Antonines.
We shall now endeavor, with clearness and precision, to describe
the provinces once united under their sway, but, at present,
divided into so many independent and hostile states. Spain, the
western extremity of the empire, of Europe, and of the ancient
world, has, in every age, invariably preserved the same natural
limits; the Pyrenæan Mountains, the Mediterranean, and the
Atlantic Ocean. That great peninsula, at present so unequally
divided between two sovereigns, was distributed by Augustus into
three provinces, Lusitania, Bætica, and Tarraconensis. The
kingdom of Portugal now fills the place of the warlike country of
the Lusitanians; and the loss sustained by the former on the side
of the East, is compensated by an accession of territory towards
the North. The confines of Grenada and Andalusia correspond with
those of ancient Bætica. The remainder of Spain, Gallicia, and
the Asturias, Biscay, and Navarre, Leon, and the two Castiles,
Murcia, Valencia, Catalonia, and Arragon, all contributed to form
the third and most considerable of the Roman governments, which,
from the name of its capital, was styled the province of
Tarragona.\textsuperscript{70} Of the native barbarians, the Celtiberians were the
most powerful, as the Cantabrians and Asturians proved the most
obstinate. Confident in the strength of their mountains, they
were the last who submitted to the arms of Rome, and the first
who threw off the yoke of the Arabs.

\pagenote[70]{See Strabo, l. ii. It is natural enough to suppose,
that Arragon is derived from Tarraconensis, and several moderns
who have written in Latin use those words as synonymous. It is,
however, certain, that the Arragon, a little stream which falls
from the Pyrenees into the Ebro, first gave its name to a
country, and gradually to a kingdom. See d’Anville, Geographie du
Moyen Age, p. 181.}

Ancient Gaul, as it contained the whole country between the
Pyrenees, the Alps, the Rhine, and the Ocean, was of greater
extent than modern France. To the dominions of that powerful
monarchy, with its recent acquisitions of Alsace and Lorraine, we
must add the duchy of Savoy, the cantons of Switzerland, the four
electorates of the Rhine, and the territories of Liege,
Luxemburgh, Hainault, Flanders, and Brabant. When Augustus gave
laws to the conquests of his father, he introduced a division of
Gaul, equally adapted to the progress of the legions, to the
course of the rivers, and to the principal national distinctions,
which had comprehended above a hundred independent states.\textsuperscript{71} The
sea-coast of the Mediterranean, Languedoc, Provence, and
Dauphiné, received their provincial appellation from the colony
of Narbonne. The government of Aquitaine was extended from the
Pyrenees to the Loire. The country between the Loire and the
Seine was styled the Celtic Gaul, and soon borrowed a new
denomination from the celebrated colony of Lugdunum, or Lyons.
The Belgic lay beyond the Seine, and in more ancient times had
been bounded only by the Rhine; but a little before the age of
Cæsar, the Germans, abusing their superiority of valor, had
occupied a considerable portion of the Belgic territory. The
Roman conquerors very eagerly embraced so flattering a
circumstance, and the Gallic frontier of the Rhine, from Basil to
Leyden, received the pompous names of the Upper and the Lower
Germany.\textsuperscript{72} Such, under the reign of the Antonines, were the six
provinces of Gaul; the Narbonnese, Aquitaine, the Celtic, or
Lyonnese, the Belgic, and the two Germanies.

\pagenote[71]{One hundred and fifteen \textit{cities} appear in the
Notitia of Gaul; and it is well known that this appellation was
applied not only to the capital town, but to the whole territory
of each state. But Plutarch and Appian increase the number of
tribes to three or four hundred.}

\pagenote[72]{D’Anville. Notice de l’Ancienne Gaule.}

We have already had occasion to mention the conquest of Britain,
and to fix the boundary of the Roman Province in this island. It
comprehended all England, Wales, and the Lowlands of Scotland, as
far as the Friths of Dumbarton and Edinburgh. Before Britain lost
her freedom, the country was irregularly divided between thirty
tribes of barbarians, of whom the most considerable were the
Belgæ in the West, the Brigantes in the North, the Silures in
South Wales, and the Iceni in Norfolk and Suffolk.\textsuperscript{73} As far as
we can either trace or credit the resemblance of manners and
language, Spain, Gaul, and Britain were peopled by the same hardy
race of savages. Before they yielded to the Roman arms, they
often disputed the field, and often renewed the contest. After
their submission, they constituted the western division of the
European provinces, which extended from the columns of Hercules
to the wall of Antoninus, and from the mouth of the Tagus to the
sources of the Rhine and Danube.

\pagenote[73]{Whittaker’s History of Manchester, vol. i. c. 3.}
Before the Roman conquest, the country which is now called
Lombardy, was not considered as a part of Italy. It had been
occupied by a powerful colony of Gauls, who, settling themselves
along the banks of the Po, from Piedmont to Romagna, carried
their arms and diffused their name from the Alps to the Apennine.

The Ligurians dwelt on the rocky coast which now forms the
republic of Genoa. Venice was yet unborn; but the territories of
that state, which lie to the east of the Adige, were inhabited by
the Venetians.\textsuperscript{74} The middle part of the peninsula, that now
composes the duchy of Tuscany and the ecclesiastical state, was
the ancient seat of the Etruscans and Umbrians; to the former of
whom Italy was indebted for the first rudiments of civilized
life.\textsuperscript{75} The Tyber rolled at the foot of the seven hills of Rome,
and the country of the Sabines, the Latins, and the Volsci, from
that river to the frontiers of Naples, was the theatre of her
infant victories. On that celebrated ground the first consuls
deserved triumphs, their successors adorned villas, and \textit{their}
posterity have erected convents.\textsuperscript{76} Capua and Campania possessed
the immediate territory of Naples; the rest of the kingdom was
inhabited by many warlike nations, the Marsi, the Samnites, the
Apulians, and the Lucanians; and the sea-coasts had been covered
by the flourishing colonies of the Greeks. We may remark, that
when Augustus divided Italy into eleven regions, the little
province of Istria was annexed to that seat of Roman sovereignty.\textsuperscript{77}

\pagenote[74]{The Italian Veneti, though often confounded with
the Gauls, were more probably of Illyrian origin. See M. Freret,
Mémoires de l’Académie des Inscriptions, tom. xviii. * Note: Or
Liburnian, according to Niebuhr. Vol. i. p. 172.—M.}

\pagenote[75]{See Maffei Verona illustrata, l. i. * Note: Add
Niebuhr, vol. i., and Otfried Müller, \textit{die Etrusker}, which
contains much that is known, and much that is conjectured, about
this remarkable people. Also Micali, Storia degli antichi popoli
Italiani. Florence, 1832—M.}

\pagenote[76]{The first contrast was observed by the ancients.
See Florus, i. 11. The second must strike every modern
traveller.}

\pagenote[77]{Pliny (Hist. Natur. l. iii.) follows the division
of Italy by Augustus.}

The European provinces of Rome were protected by the course of
the Rhine and the Danube. The latter of those mighty streams,
which rises at the distance of only thirty miles from the former,
flows above thirteen hundred miles, for the most part to the
south-east, collects the tribute of sixty navigable rivers, and
is, at length, through six mouths, received into the Euxine,
which appears scarcely equal to such an accession of waters.\textsuperscript{78}
The provinces of the Danube soon acquired the general appellation
of Illyricum, or the Illyrian frontier,\textsuperscript{79} and were esteemed the
most warlike of the empire; but they deserve to be more
particularly considered under the names of Rhætia, Noricum,
Pannonia, Dalmatia, Dacia, Mæsia, Thrace, Macedonia, and Greece.

\pagenote[78]{Tournefort, Voyages en Grece et Asie Mineure,
lettre xviii.}

\pagenote[79]{The name of Illyricum originally belonged to the
sea-coast of the Adriatic, and was gradually extended by the
Romans from the Alps to the Euxine Sea. See Severini Pannonia, l.
i. c. 3.}

The province of Rhætia, which soon extinguished the name of the
Vindelicians, extended from the summit of the Alps to the banks
of the Danube; from its source, as far as its conflux with the
Inn. The greatest part of the flat country is subject to the
elector of Bavaria; the city of Augsburg is protected by the
constitution of the German empire; the Grisons are safe in their
mountains, and the country of Tirol is ranked among the numerous
provinces of the house of Austria.

The wide extent of territory which is included between the Inn,
the Danube, and the Save,—Austria, Styria, Carinthia, Carniola,
the Lower Hungary, and Sclavonia,—was known to the ancients under
the names of Noricum and Pannonia. In their original state of
independence, their fierce inhabitants were intimately connected.
Under the Roman government they were frequently united, and they
still remain the patrimony of a single family. They now contain
the residence of a German prince, who styles himself Emperor of
the Romans, and form the centre, as well as strength, of the
Austrian power. It may not be improper to observe, that if we
except Bohemia, Moravia, the northern skirts of Austria, and a
part of Hungary between the Teyss and the Danube, all the other
dominions of the House of Austria were comprised within the
limits of the Roman Empire.

Dalmatia, to which the name of Illyricum more properly belonged,
was a long, but narrow tract, between the Save and the Adriatic.
The best part of the sea-coast, which still retains its ancient
appellation, is a province of the Venetian state, and the seat of
the little republic of Ragusa. The inland parts have assumed the
Sclavonian names of Croatia and Bosnia; the former obeys an
Austrian governor, the latter a Turkish pacha; but the whole
country is still infested by tribes of barbarians, whose savage
independence irregularly marks the doubtful limit of the
Christian and Mahometan power.\textsuperscript{80}

\pagenote[80]{A Venetian traveller, the Abbate Fortis, has lately
given us some account of those very obscure countries. But the
geography and antiquities of the western Illyricum can be
expected only from the munificence of the emperor, its
sovereign.}

After the Danube had received the waters of the Teyss and the
Save, it acquired, at least among the Greeks, the name of Ister.\textsuperscript{81}
It formerly divided Mæsia and Dacia, the latter of which, as
we have already seen, was a conquest of Trajan, and the only
province beyond the river. If we inquire into the present state
of those countries, we shall find that, on the left hand of the
Danube, Temeswar and Transylvania have been annexed, after many
revolutions, to the crown of Hungary; whilst the principalities
of Moldavia and Wallachia acknowledge the supremacy of the
Ottoman Porte. On the right hand of the Danube, Mæsia, which,
during the middle ages, was broken into the barbarian kingdoms of
Servia and Bulgaria, is again united in Turkish slavery.

\pagenote[81]{The Save rises near the confines of \textit{Istria}, and
was considered by the more early Greeks as the principal stream
of the Danube.}

The appellation of Roumelia, which is still bestowed by the Turks
on the extensive countries of Thrace, Macedonia, and Greece,
preserves the memory of their ancient state under the Roman
empire. In the time of the Antonines, the martial regions of
Thrace, from the mountains of Hæmus and Rhodope, to the Bosphorus
and the Hellespont, had assumed the form of a province.
Notwithstanding the change of masters and of religion, the new
city of Rome, founded by Constantine on the banks of the
Bosphorus, has ever since remained the capital of a great
monarchy. The kingdom of Macedonia, which, under the reign of
Alexander, gave laws to Asia, derived more solid advantages from
the policy of the two Philips; and with its dependencies of
Epirus and Thessaly, extended from the Ægean to the Ionian Sea.
When we reflect on the fame of Thebes and Argos, of Sparta and
Athens, we can scarcely persuade ourselves, that so many immortal
republics of ancient Greece were lost in a single province of the
Roman empire, which, from the superior influence of the Achæan
league, was usually denominated the province of Achaia.

Such was the state of Europe under the Roman emperors. The
provinces of Asia, without excepting the transient conquests of
Trajan, are all comprehended within the limits of the Turkish
power. But, instead of following the arbitrary divisions of
despotism and ignorance, it will be safer for us, as well as more
agreeable, to observe the indelible characters of nature. The
name of Asia Minor is attributed with some propriety to the
peninsula, which, confined betwixt the Euxine and the
Mediterranean, advances from the Euphrates towards Europe. The
most extensive and flourishing district, westward of Mount Taurus
and the River Halys, was dignified by the Romans with the
exclusive title of Asia. The jurisdiction of that province
extended over the ancient monarchies of Troy, Lydia, and Phrygia,
the maritime countries of the Pamphylians, Lycians, and Carians,
and the Grecian colonies of Ionia, which equalled in arts, though
not in arms, the glory of their parent. The kingdoms of Bithynia
and Pontus possessed the northern side of the peninsula from
Constantinople to Trebizond. On the opposite side, the province
of Cilicia was terminated by the mountains of Syria: the inland
country, separated from the Roman Asia by the River Halys, and
from Armenia by the Euphrates, had once formed the independent
kingdom of Cappadocia. In this place we may observe, that the
northern shores of the Euxine, beyond Trebizond in Asia, and
beyond the Danube in Europe, acknowledged the sovereignty of the
emperors, and received at their hands either tributary princes or
Roman garrisons. Budzak, Crim Tartary, Circassia, and Mingrelia,
are the modern appellations of those savage countries.\textsuperscript{82}

\pagenote[82]{See the Periplus of Arrian. He examined the coasts
of the Euxine, when he was governor of Cappadocia.}

Under the successors of Alexander, Syria was the seat of the
Seleucidæ, who reigned over Upper Asia, till the successful
revolt of the Parthians confined their dominions between the
Euphrates and the Mediterranean. When Syria became subject to the
Romans, it formed the eastern frontier of their empire: nor did
that province, in its utmost latitude, know any other bounds than
the mountains of Cappadocia to the north, and towards the south,
the confines of Egypt, and the Red Sea. Phœnicia and Palestine
were sometimes annexed to, and sometimes separated from, the
jurisdiction of Syria. The former of these was a narrow and rocky
coast; the latter was a territory scarcely superior to Wales,
either in fertility or extent.\textsuperscript{821} Yet Phœnicia and Palestine
will forever live in the memory of mankind; since America, as
well as Europe, has received letters from the one, and religion
from the other.\textsuperscript{83} A sandy desert, alike destitute of wood and
water, skirts along the doubtful confine of Syria, from the
Euphrates to the Red Sea. The wandering life of the Arabs was
inseparably connected with their independence; and wherever, on
some spots less barren than the rest, they ventured to for many
settled habitations, they soon became subjects to the Roman
empire.\textsuperscript{84}

\pagenote[821]{This comparison is exaggerated, with the
intention, no doubt, of attacking the authority of the Bible,
which boasts of the fertility of Palestine. Gibbon’s only
authorities were that of Strabo (l. xvi. 1104) and the present
state of the country. But Strabo only speaks of the neighborhood
of Jerusalem, which he calls barren and arid to the extent of
sixty stadia round the city: in other parts he gives a favorable
testimony to the fertility of many parts of Palestine: thus he
says, “Near Jericho there is a grove of palms, and a country of a
hundred stadia, full of springs, and well peopled.” Moreover,
Strabo had never seen Palestine; he spoke only after reports,
which may be as inaccurate as those according to which he has
composed that description of Germany, in which Gluverius has
detected so many errors. (Gluv. Germ. iii. 1.) Finally, his
testimony is contradicted and refuted by that of other ancient
authors, and by medals. Tacitus says, in speaking of Palestine,
“The inhabitants are healthy and robust; the rains moderate; the
soil fertile.” (Hist. v. 6.) Ammianus Macellinus says also, “The
last of the Syrias is Palestine, a country of considerable
extent, abounding in clean and well-cultivated land, and
containing some fine cities, none of which yields to the other;
but, as it were, being on a parallel, are rivals.”—xiv. 8. See
also the historian Josephus, Hist. vi. 1. Procopius of Cæserea,
who lived in the sixth century, says that Chosroes, king of
Persia, had a great desire to make himself master of Palestine,
\textit{on account of its} extraordinary fertility, its opulence, and
the great number of its inhabitants. The Saracens thought the
same, and were afraid that Omar. when he went to Jerusalem,
charmed with the fertility of the soil and the purity of the air,
would never return to Medina. (Ockley, Hist. of Sarac. i. 232.)
The importance attached by the Romans to the conquest of
Palestine, and the obstacles they encountered, prove also the
richness and population of the country. Vespasian and Titus
caused medals to be struck with trophies, in which Palestine is
represented by a female under a palm-tree, to signify the
richness of he country, with this legend: \textit{Judæa capta}. Other
medals also indicate this fertility; for instance, that of Herod
holding a bunch of grapes, and that of the young Agrippa
displaying fruit. As to the present state of he country, one
perceives that it is not fair to draw any inference against its
ancient fertility: the disasters through which it has passed, the
government to which it is subject, the disposition of the
inhabitants, explain sufficiently the wild and uncultivated
appearance of the land, where, nevertheless, fertile and
cultivated districts are still found, according to the testimony
of travellers; among others, of Shaw, Maundrel, La Rocque, \&c.—G.
The Abbé Guénée, in his \textit{Lettres de quelques Juifs à Mons. de
Voltaire}, has exhausted the subject of the fertility of
Palestine; for Voltaire had likewise indulged in sarcasm on this
subject. Gibbon was assailed on this point, not, indeed, by Mr.
Davis, who, he slyly insinuates, was prevented by his patriotism
as a Welshman from resenting the comparison with Wales, but by
other writers. In his Vindication, he first established the
correctness of his measurement of Palestine, which he estimates
as 7600 square English miles, while Wales is about 7011. As to
fertility, he proceeds in the following dexterously composed and
splendid passage: “The emperor Frederick II., the enemy and the
victim of the clergy, is accused of saying, after his return from
his crusade, that the God of the Jews would have despised his
promised land, if he had once seen the fruitful realms of Sicily
and Naples.” (See Giannone, Istor. Civ. del R. di Napoli, ii.
245.) This raillery, which malice has, perhaps, falsely imputed
to Frederick, is inconsistent with truth and piety; yet it must
be confessed that the soil of Palestine does not contain that
inexhaustible, and, as it were, spontaneous principle of
fertility, which, under the most unfavorable circumstances, has
covered with rich harvests the banks of the Nile, the fields of
Sicily, or the plains of Poland. The Jordan is the only navigable
river of Palestine: a considerable part of the narrow space is
occupied, or rather lost, in the \textit{Dead Sea} whose horrid aspect
inspires every sensation of disgust, and countenances every tale
of horror. The districts which border on Arabia partake of the
sandy quality of the adjacent desert. The face of the country,
except the sea-coast, and the valley of the Jordan, is covered
with mountains, which appear, for the most part, as naked and
barren rocks; and in the neighborhood of Jerusalem, there is a
real scarcity of the two elements of earth and water. (See
Maundrel’s Travels, p. 65, and Reland’s Palestin. i. 238, 395.)
These disadvantages, which now operate in their fullest extent,
were formerly corrected by the labors of a numerous people, and
the active protection of a wise government. The hills were
clothed with rich beds of artificial mould, the rain was
collected in vast cisterns, a supply of fresh water was conveyed
by pipes and aqueducts to the dry lands. The breed of cattle was
encouraged in those parts which were not adapted for tillage, and
almost every spot was compelled to yield some production for the
use of the inhabitants.\\
Pater ispe colendi Haud facilem esse viam voluit, primusque par
artem Movit agros; curis acuens mortalia corda, Nec torpere gravi
passus sua Regna veterno. Gibbon, Misc. Works, iv. 540.\\
But Gibbon has here eluded the question about the land “flowing
with milk and honey.” He is describing Judæa only, without
comprehending Galilee, or the rich pastures beyond the Jordan,
even now proverbial for their flocks and herds. (See Burckhardt’s
Travels, and Hist of Jews, i. 178.) The following is believed to
be a fair statement: “The extraordinary fertility of the whole
country must be taken into the account. No part was waste; very
little was occupied by unprofitable wood; the more fertile hills
were cultivated in artificial terraces, others were hung with
orchards of fruit trees the more rocky and barren districts were
covered with vineyards.” Even in the present day, the wars and
misgovernment of ages have not exhausted the natural richness of
the soil. “Galilee,” says Malte Brun, “would be a paradise were
it inhabited by an industrious people under an enlightened
government. No land could be less dependent on foreign
importation; it bore within itself every thing that could be
necessary for the subsistence and comfort of a simple
agricultural people. The climate was healthy, the seasons
regular; the former rains, which fell about October, after the
vintage, prepared the ground for the seed; that latter, which
prevailed during March and the beginning of April, made it grow
rapidly. Directly the rains ceased, the grain ripened with still
greater rapidity, and was gathered in before the end of May. The
summer months were dry and very hot, but the nights cool and
refreshed by copious dews. In September, the vintage was
gathered. Grain of all kinds, wheat, barley, millet, zea, and
other sorts, grew in abundance; the wheat commonly yielded thirty
for one. Besides the vine and the olive, the almond, the date,
figs of many kinds, the orange, the pomegranate, and many other
fruit trees, flourished in the greatest luxuriance. Great
quantity of honey was collected. The balm-tree, which produced
the opobalsamum, a great object of trade, was probably introduced
from Arabia, in the time of Solomon. It flourished about Jericho
and in Gilead.”—Milman’s Hist. of Jews. i. 177.—M.}

\pagenote[83]{The progress of religion is well known. The use of
letter was introduced among the savages of Europe about fifteen
hundred years before Christ; and the Europeans carried them to
America about fifteen centuries after the Christian Æra. But in a
period of three thousand years, the Phœnician alphabet received
considerable alterations, as it passed through the hands of the
Greeks and Romans.}

\pagenote[84]{Dion Cassius, lib. lxviii. p. 1131.}

The geographers of antiquity have frequently hesitated to what
portion of the globe they should ascribe Egypt.\textsuperscript{85} By its
situation that celebrated kingdom is included within the immense
peninsula of Africa; but it is accessible only on the side of
Asia, whose revolutions, in almost every period of history, Egypt
has humbly obeyed. A Roman præfect was seated on the splendid
throne of the Ptolemies; and the iron sceptre of the Mamelukes is
now in the hands of a Turkish pacha. The Nile flows down the
country, above five hundred miles from the tropic of Cancer to
the Mediterranean, and marks on either side the extent of
fertility by the measure of its inundations. Cyrene, situate
towards the west, and along the sea-coast, was first a Greek
colony, afterwards a province of Egypt, and is now lost in the
desert of Barca.\textsuperscript{851}

\pagenote[85]{Ptolemy and Strabo, with the modern geographers,
fix the Isthmus of Suez as the boundary of Asia and Africa.
Dionysius, Mela, Pliny, Sallust, Hirtius, and Solinus, have
preferred for that purpose the western branch of the Nile, or
even the great Catabathmus, or descent, which last would assign
to Asia, not only Egypt, but part of Libya.}

\pagenote[851]{The French editor has a long and unnecessary note
on the History of Cyrene. For the present state of that coast and
country, the volume of Captain Beechey is full of interesting
details. Egypt, now an independent and improving kingdom,
appears, under the enterprising rule of Mahommed Ali, likely to
revenge its former oppression upon the decrepit power of the
Turkish empire.—M. —This note was written in 1838. The future
destiny of Egypt is an important problem, only to be solved by
time. This observation will also apply to the new French colony
in Algiers.—M. 1845.}

From Cyrene to the ocean, the coast of Africa extends above
fifteen hundred miles; yet so closely is it pressed between the
Mediterranean and the Sahara, or sandy desert, that its breadth
seldom exceeds fourscore or a hundred miles. The eastern division
was considered by the Romans as the more peculiar and proper
province of Africa. Till the arrival of the Phœnician colonies,
that fertile country was inhabited by the Libyans, the most
savage of mankind. Under the immediate jurisdiction of Carthage,
it became the centre of commerce and empire; but the republic of
Carthage is now degenerated into the feeble and disorderly states
of Tripoli and Tunis. The military government of Algiers
oppresses the wide extent of Numidia, as it was once united under
Massinissa and Jugurtha; but in the time of Augustus, the limits
of Numidia were contracted; and, at least, two thirds of the
country acquiesced in the name of Mauritania, with the epithet of
Cæsariensis. The genuine Mauritania, or country of the Moors,
which, from the ancient city of Tingi, or Tangier, was
distinguished by the appellation of Tingitana, is represented by
the modern kingdom of Fez. Salle, on the Ocean, so infamous at
present for its piratical depredations, was noticed by the
Romans, as the extreme object of their power, and almost of their
geography. A city of their foundation may still be discovered
near Mequinez, the residence of the barbarian whom we condescend
to style the Emperor of Morocco; but it does not appear, that his
more southern dominions, Morocco itself, and Segelmessa, were
ever comprehended within the Roman province. The western parts of
Africa are intersected by the branches of Mount Atlas, a name so
idly celebrated by the fancy of poets;\textsuperscript{86} but which is now
diffused over the immense ocean that rolls between the ancient
and the new continent.\textsuperscript{87}

\pagenote[86]{The long range, moderate height, and gentle
declivity of Mount Atlas, (see Shaw’s Travels, p. 5,) are very
unlike a solitary mountain which rears its head into the clouds,
and seems to support the heavens. The peak of Teneriff, on the
contrary, rises a league and a half above the surface of the sea;
and, as it was frequently visited by the Phœnicians, might engage
the notice of the Greek poets. See Buffon, Histoire Naturelle,
tom. i. p. 312. Histoire des Voyages, tom. ii.}

\pagenote[87]{M. de Voltaire, tom. xiv. p. 297, unsupported by
either fact or probability, has generously bestowed the Canary
Islands on the Roman empire.}

Having now finished the circuit of the Roman empire, we may
observe, that Africa is divided from Spain by a narrow strait of
about twelve miles, through which the Atlantic flows into the
Mediterranean. The columns of Hercules, so famous among the
ancients, were two mountains which seemed to have been torn
asunder by some convulsion of the elements; and at the foot of
the European mountain, the fortress of Gibraltar is now seated.
The whole extent of the Mediterranean Sea, its coasts and its
islands, were comprised within the Roman dominion. Of the larger
islands, the two Baleares, which derive their name of Majorca and
Minorca from their respective size, are subject at present, the
former to Spain, the latter to Great Britain.\textsuperscript{871} It is easier to
deplore the fate, than to describe the actual condition, of
Corsica.\textsuperscript{872} Two Italian sovereigns assume a regal title from
Sardinia and Sicily. Crete, or Candia, with Cyprus, and most of
the smaller islands of Greece and Asia, have been subdued by the
Turkish arms, whilst the little rock of Malta defies their power,
and has emerged, under the government of its military Order, into
fame and opulence.\textsuperscript{873}

\pagenote[871]{Minorca was lost to Great Britain in 1782. Ann.
Register for that year.—M.}

\pagenote[872]{The gallant struggles of the Corsicans for their
independence, under Paoli, were brought to a close in the year
1769. This volume was published in 1776. See Botta, Storia
d’Italia, vol. xiv.—M.}

\pagenote[873]{Malta, it need scarcely be said, is now in the
possession of the English. We have not, however, thought it
necessary to notice every change in the political state of the
world, since the time of Gibbon.—M}

This long enumeration of provinces, whose broken fragments have
formed so many powerful kingdoms, might almost induce us to
forgive the vanity or ignorance of the ancients. Dazzled with the
extensive sway, the irresistible strength, and the real or
affected moderation of the emperors, they permitted themselves to
despise, and sometimes to forget, the outlying countries which
had been left in the enjoyment of a barbarous independence; and
they gradually usurped the license of confounding the Roman
monarchy with the globe of the earth.\textsuperscript{88} But the temper, as well
as knowledge, of a modern historian, require a more sober and
accurate language. He may impress a juster image of the greatness
of Rome, by observing that the empire was above two thousand
miles in breadth, from the wall of Antoninus and the northern
limits of Dacia, to Mount Atlas and the tropic of Cancer; that it
extended in length more than three thousand miles from the
Western Ocean to the Euphrates; that it was situated in the
finest part of the Temperate Zone, between the twenty-fourth and
fifty-sixth degrees of northern latitude; and that it was
supposed to contain above sixteen hundred thousand square miles,
for the most part of fertile and well-cultivated land.\textsuperscript{89}

\pagenote[88]{Bergier, Hist. des Grands Chemins, l. iii. c. 1, 2,
3, 4, a very useful collection.}

\pagenote[89]{See Templeman’s Survey of the Globe; but I distrust
both the Doctor’s learning and his maps.}

