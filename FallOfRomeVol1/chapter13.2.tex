\section{Part \thesection.}
\thispagestyle{simple}

Britain had none but domestic enemies to dread; and as long as
the governors preserved their fidelity, and the troops their
discipline, the incursions of the naked savages of Scotland or
Ireland could never materially affect the safety of the province.

The peace of the continent, and the defence of the principal
rivers which bounded the empire, were objects of far greater
difficulty and importance. The policy of Diocletian, which
inspired the councils of his associates, provided for the public
tranquility, by encouraging a spirit of dissension among the
barbarians, and by strengthening the fortifications of the Roman
limit. In the East he fixed a line of camps from Egypt to the
Persian dominions, and for every camp, he instituted an adequate
number of stationary troops, commanded by their respective
officers, and supplied with every kind of arms, from the new
arsenals which he had formed at Antioch, Emesa, and Damascus.\footnotemark[32]
Nor was the precaution of the emperor less watchful against the
well-known valor of the barbarians of Europe. From the mouth of
the Rhine to that of the Danube, the ancient camps, towns, and
citidels, were diligently reëstablished, and, in the most exposed
places, new ones were skilfully constructed: the strictest
vigilance was introduced among the garrisons of the frontier, and
every expedient was practised that could render the long chain of
fortifications firm and impenetrable.\footnotemark[33] A barrier so respectable
was seldom violated, and the barbarians often turned against each
other their disappointed rage. The Goths, the Vandals, the
Gepidæ, the Burgundians, the Alemanni, wasted each other’s
strength by destructive hostilities: and whosoever vanquished,
they vanquished the enemies of Rome. The subjects of Diocletian
enjoyed the bloody spectacle, and congratulated each other, that
the mischiefs of civil war were now experienced only by the
barbarians.\footnotemark[34]

\footnotetext[32]{John Malala, in Chron, Antiochen. tom. i. p. 408,
409.}

\footnotetext[33]{Zosim. l. i. p. 3. That partial historian seems to
celebrate the vigilance of Diocletian with a design of exposing
the negligence of Constantine; we may, however, listen to an
orator: “Nam quid ego alarum et cohortium castra percenseam, toto
Rheni et Istri et Euphraus limite restituta.” Panegyr. Vet. iv.
18.}

\footnotetext[34]{Ruunt omnes in sanguinem suum populi, quibus ron
contigilesse Romanis, obstinatæque feritatis poenas nunc sponte
persolvunt. Panegyr. Vet. iii. 16. Mamertinus illustrates the
fact by the example of almost all the nations in the world.}

Notwithstanding the policy of Diocletian, it was impossible to
maintain an equal and undisturbed tranquillity during a reign of
twenty years, and along a frontier of many hundred miles.
Sometimes the barbarians suspended their domestic animosities,
and the relaxed vigilance of the garrisons sometimes gave a
passage to their strength or dexterity. Whenever the provinces
were invaded, Diocletian conducted himself with that calm dignity
which he always affected or possessed; reserved his presence for
such occasions as were worthy of his interposition, never exposed
his person or reputation to any unnecessary danger, insured his
success by every means that prudence could suggest, and
displayed, with ostentation, the consequences of his victory. In
wars of a more difficult nature, and more doubtful event, he
employed the rough valor of Maximian; and that faithful soldier
was content to ascribe his own victories to the wise counsels and
auspicious influence of his benefactor. But after the adoption of
the two Cæsars, the emperors themselves, retiring to a less
laborious scene of action, devolved on their adopted sons the
defence of the Danube and of the Rhine. The vigilant Galerius was
never reduced to the necessity of vanquishing an army of
barbarians on the Roman territory. \footnotemark[35] The brave and active
Constantius delivered Gaul from a very furious inroad of the
Alemanni; and his victories of Langres and Vindonissa appear to
have been actions of considerable danger and merit. As he
traversed the open country with a feeble guard, he was
encompassed on a sudden by the superior multitude of the enemy.
He retreated with difficulty towards Langres; but, in the general
consternation, the citizens refused to open their gates, and the
wounded prince was drawn up the wall by the means of a rope. But,
on the news of his distress, the Roman troops hastened from all
sides to his relief, and before the evening he had satisfied his
honor and revenge by the slaughter of six thousand Alemanni.\footnotemark[36]
From the monuments of those times, the obscure traces of several
other victories over the barbarians of Sarmatia and Germany might
possibly be collected; but the tedious search would not be
rewarded either with amusement or with instruction.

\footnotetext[35]{He complained, though not with the strictest truth,
“Jam fluxisse annos quindecim in quibus, in Illyrico, ad ripam
Danubii relegatus cum gentibus barbaris luctaret.” Lactant. de M.
P. c. 18.}

\footnotetext[36]{In the Greek text of Eusebius, we read six
thousand, a number which I have preferred to the sixty thousand
of Jerome, Orosius Eutropius, and his Greek translator Pæanius.}

The conduct which the emperor Probus had adopted in the disposal
of the vanquished was imitated by Diocletian and his associates.
The captive barbarians, exchanging death for slavery, were
distributed among the provincials, and assigned to those
districts (in Gaul, the territories of Amiens, Beauvais, Cambray,
Treves, Langres, and Troyes, are particularly specified)\footnotemark[37] which
had been depopulated by the calamities of war. They were usefully
employed as shepherds and husbandmen, but were denied the
exercise of arms, except when it was found expedient to enroll
them in the military service. Nor did the emperors refuse the
property of lands, with a less servile tenure, to such of the
barbarians as solicited the protection of Rome. They granted a
settlement to several colonies of the Carpi, the Bastarnæ, and
the Sarmatians; and, by a dangerous indulgence, permitted them in
some measure to retain their national manners and independence.\footnotemark[38]
Among the provincials, it was a subject of flattering
exultation, that the barbarian, so lately an object of terror,
now cultivated their lands, drove their cattle to the neighboring
fair, and contributed by his labor to the public plenty. They
congratulated their masters on the powerful accession of subjects
and soldiers; but they forgot to observe, that multitudes of
secret enemies, insolent from favor, or desperate from
oppression, were introduced into the heart of the empire.\footnotemark[39]

\footnotetext[37]{Panegyr. Vet. vii. 21.}

\footnotetext[38]{There was a settlement of the Sarmatians in the
neighborhood of Treves, which seems to have been deserted by
those lazy barbarians. Ausonius speaks of them in his Mosella:——
“Unde iter ingrediens nemorosa per avia solum, Et nulla humani
spectans vestigia cultus; ........ Arvaque Sauromatum nuper
metata colonis.”}

\footnotetext[39]{There was a town of the Carpi in the Lower Mæsia.
See the rhetorical exultation of Eumenius.}

While the Cæsars exercised their valor on the banks of the Rhine
and Danube, the presence of the emperors was required on the
southern confines of the Roman world. From the Nile to Mount
Atlas, Africa was in arms. A confederacy of five Moorish nations
issued from their deserts to invade the peaceful provinces.\footnotemark[40]
Julian had assumed the purple at Carthage.\footnotemark[41] Achilleus at
Alexandria, and even the Blemmyes, renewed, or rather continued,
their incursions into the Upper Egypt. Scarcely any circumstances
have been preserved of the exploits of Maximian in the western
parts of Africa; but it appears, by the event, that the progress
of his arms was rapid and decisive, that he vanquished the
fiercest barbarians of Mauritania, and that he removed them from
the mountains, whose inaccessible strength had inspired their
inhabitants with a lawless confidence, and habituated them to a
life of rapine and violence.\footnotemark[42] Diocletian, on his side, opened
the campaign in Egypt by the siege of Alexandria, cut off the
aqueducts which conveyed the waters of the Nile into every
quarter of that immense city,\footnotemark[43] and rendering his camp
impregnable to the sallies of the besieged multitude, he pushed
his reiterated attacks with caution and vigor. After a siege of
eight months, Alexandria, wasted by the sword and by fire,
implored the clemency of the conqueror, but it experienced the
full extent of his severity. Many thousands of the citizens
perished in a promiscuous slaughter, and there were few obnoxious
persons in Egypt who escaped a sentence either of death or at
least of exile.\footnotemark[44] The fate of Busiris and of Coptos was still
more melancholy than that of Alexandria: those proud cities, the
former distinguished by its antiquity, the latter enriched by the
passage of the Indian trade, were utterly destroyed by the arms
and by the severe order of Diocletian.\footnotemark[45] The character of the
Egyptian nation, insensible to kindness, but extremely
susceptible of fear, could alone justify this excessive rigor.
The seditions of Alexandria had often affected the tranquillity
and subsistence of Rome itself. Since the usurpation of Firmus,
the province of Upper Egypt, incessantly relapsing into
rebellion, had embraced the alliance of the savages of Æthiopia.
The number of the Blemmyes, scattered between the Island of Meroe
and the Red Sea, was very inconsiderable, their disposition was
unwarlike, their weapons rude and inoffensive.\footnotemark[46] Yet in the
public disorders, these barbarians, whom antiquity, shocked with
the deformity of their figure, had almost excluded from the human
species, presumed to rank themselves among the enemies of Rome.\footnotemark[47]
Such had been the unworthy allies of the Egyptians; and while
the attention of the state was engaged in more serious wars,
their vexations inroads might again harass the repose of the
province. With a view of opposing to the Blemmyes a suitable
adversary, Diocletian persuaded the Nobatæ, or people of Nubia,
to remove from their ancient habitations in the deserts of Libya,
and resigned to them an extensive but unprofitable territory
above Syene and the cataracts of the Nile, with the stipulation,
that they should ever respect and guard the frontier of the
empire. The treaty long subsisted; and till the establishment of
Christianity introduced stricter notions of religious worship, it
was annually ratified by a solemn sacrifice in the isle of
Elephantine, in which the Romans, as well as the barbarians,
adored the same visible or invisible powers of the universe.\footnotemark[48]

\footnotetext[40]{Scaliger (Animadvers. ad Euseb. p. 243) decides, in
his usual manner, that the Quinque gentiani, or five African
nations, were the five great cities, the Pentapolis of the
inoffensive province of Cyrene.}

\footnotetext[41]{After his defeat, Julian stabbed himself with a
dagger, and immediately leaped into the flames. Victor in
Epitome.}

\footnotetext[42]{Tu ferocissimos Mauritaniæ populos inaccessis
montium jugis et naturali munitione fidentes, expugnasti,
recepisti, transtulisti. Panegyr Vet. vi. 8.}

\footnotetext[43]{See the description of Alexandria, in Hirtius de
Bel. Alexandrin c. 5.}

\footnotetext[44]{Eutrop. ix. 24. Orosius, vii. 25. John Malala in
Chron. Antioch. p. 409, 410. Yet Eumenius assures us, that Egypt
was pacified by the clemency of Diocletian.}

\footnotetext[45]{Eusebius (in Chron.) places their destruction
several years sooner and at a time when Egypt itself was in a
state of rebellion against the Romans.}

\footnotetext[46]{Strabo, l. xvii. p. 172. Pomponius Mela, l. i. c.
4. His words are curious: “Intra, si credere libet vix, homines
magisque semiferi Ægipanes, et Blemmyes, et Satyri.”}

\footnotetext[47]{Ausus sese inserere fortunæ et provocare arma
Romana.}

\footnotetext[48]{See Procopius de Bell. Persic. l. i. c. 19. Note:
Compare, on the epoch of the final extirpation of the rites of
Paganism from the Isle of Philæ, (Elephantine,) which subsisted
till the edict of Theodosius, in the sixth century, a
dissertation of M. Letronne, on certain Greek inscriptions. The
dissertation contains some very interesting observations on the
conduct and policy of Diocletian in Egypt. Mater pour l’Hist. du
Christianisme en Egypte, Nubie et Abyssinie, Paris 1817—M.}

At the same time that Diocletian chastised the past crimes of the
Egyptians, he provided for their future safety and happiness by
many wise regulations, which were confirmed and enforced under
the succeeding reigns.\footnotemark[49] One very remarkable edict which he
published, instead of being condemned as the effect of jealous
tyranny, deserves to be applauded as an act of prudence and
humanity. He caused a diligent inquiry to be made “for all the
ancient books which treated of the admirable art of making gold
and silver, and without pity, committed them to the flames;
apprehensive, as we are assumed, lest the opulence of the
Egyptians should inspire them with confidence to rebel against
the empire.”\footnotemark[50] But if Diocletian had been convinced of the
reality of that valuable art, far from extinguishing the memory,
he would have converted the operation of it to the benefit of the
public revenue. It is much more likely, that his good sense
discovered to him the folly of such magnificent pretensions, and
that he was desirous of preserving the reason and fortunes of his
subjects from the mischievous pursuit. It may be remarked, that
these ancient books, so liberally ascribed to Pythagoras, to
Solomon, or to Hermes, were the pious frauds of more recent
adepts. The Greeks were inattentive either to the use or to the
abuse of chemistry. In that immense register, where Pliny has
deposited the discoveries, the arts, and the errors of mankind,
there is not the least mention of the transmutation of metals;
and the persecution of Diocletian is the first authentic event in
the history of alchemy. The conquest of Egypt by the Arabs
diffused that vain science over the globe. Congenial to the
avarice of the human heart, it was studied in China as in Europe,
with equal eagerness, and with equal success. The darkness of the
middle ages insured a favorable reception to every tale of
wonder, and the revival of learning gave new vigor to hope, and
suggested more specious arts of deception. Philosophy, with the
aid of experience, has at length banished the study of alchemy;
and the present age, however desirous of riches, is content to
seek them by the humbler means of commerce and industry.\footnotemark[51]

\footnotetext[49]{He fixed the public allowance of corn, for the
people of Alexandria, at two millions of medimni; about four
hundred thousand quarters. Chron. Paschal. p. 276 Procop. Hist.
Arcan. c. 26.}

\footnotetext[50]{John Antioch, in Excerp. Valesian. p. 834. Suidas
in Diocletian.}

\footnotetext[51]{See a short history and confutation of Alchemy, in
the works of that philosophical compiler, La Mothe le Vayer, tom.
i. p. 32—353.}

The reduction of Egypt was immediately followed by the Persian
war. It was reserved for the reign of Diocletian to vanquish that
powerful nation, and to extort a confession from the successors
of Artaxerxes, of the superior majesty of the Roman empire.

We have observed, under the reign of Valerian, that Armenia was
subdued by the perfidy and the arms of the Persians, and that,
after the assassination of Chosroes, his son Tiridates, the
infant heir of the monarchy, was saved by the fidelity of his
friends, and educated under the protection of the emperors.
Tiridates derived from his exile such advantages as he could
never have obtained on the throne of Armenia; the early knowledge
of adversity, of mankind, and of the Roman discipline. He
signalized his youth by deeds of valor, and displayed a matchless
dexterity, as well as strength, in every martial exercise, and
even in the less honorable contests of the Olympian games.\footnotemark[52]
Those qualities were more nobly exerted in the defence of his
benefactor Licinius.\footnotemark[53] That officer, in the sedition which
occasioned the death of Probus, was exposed to the most imminent
danger, and the enraged soldiers were forcing their way into his
tent, when they were checked by the single arm of the Armenian
prince. The gratitude of Tiridates contributed soon afterwards to
his restoration. Licinius was in every station the friend and
companion of Galerius, and the merit of Galerius, long before he
was raised to the dignity of Cæsar, had been known and esteemed
by Diocletian. In the third year of that emperor’s reign
Tiridates was invested with the kingdom of Armenia. The justice
of the measure was not less evident than its expediency. It was
time to rescue from the usurpation of the Persian monarch an
important territory, which, since the reign of Nero, had been
always granted under the protection of the empire to a younger
branch of the house of Arsaces.\footnotemark[54]

\footnotetext[52]{See the education and strength of Tiridates in the
Armenian history of Moses of Chorene, l. ii. c. 76. He could
seize two wild bulls by the horns, and break them off with his
hands.}

\footnotetext[53]{If we give credit to the younger Victor, who
supposes that in the year 323 Licinius was only sixty years of
age, he could scarcely be the same person as the patron of
Tiridates; but we know from much better authority, (Euseb. Hist.
Ecclesiast. l. x. c. 8,) that Licinius was at that time in the
last period of old age: sixteen years before, he is represented
with gray hairs, and as the contemporary of Galerius. See
Lactant. c. 32. Licinius was probably born about the year 250.}

\footnotetext[54]{See the sixty-second and sixty-third books of Dion
Cassius.}

When Tiridates appeared on the frontiers of Armenia, he was
received with an unfeigned transport of joy and loyalty. During
twenty-six years, the country had experienced the real and
imaginary hardships of a foreign yoke. The Persian monarchs
adorned their new conquest with magnificent buildings; but those
monuments had been erected at the expense of the people, and were
abhorred as badges of slavery. The apprehension of a revolt had
inspired the most rigorous precautions: oppression had been
aggravated by insult, and the consciousness of the public hatred
had been productive of every measure that could render it still
more implacable. We have already remarked the intolerant spirit
of the Magian religion. The statues of the deified kings of
Armenia, and the sacred images of the sun and moon, were broke in
pieces by the zeal of the conqueror; and the perpetual fire of
Ormuzd was kindled and preserved upon an altar erected on the
summit of Mount Bagavan.\footnotemark[55] It was natural, that a people
exasperated by so many injuries, should arm with zeal in the
cause of their independence, their religion, and their hereditary
sovereign. The torrent bore down every obstacle, and the Persian
garrisons retreated before its fury. The nobles of Armenia flew
to the standard of Tiridates, all alleging their past merit,
offering their future service, and soliciting from the new king
those honors and rewards from which they had been excluded with
disdain under the foreign government.\footnotemark[56] The command of the army
was bestowed on Artavasdes, whose father had saved the infancy of
Tiridates, and whose family had been massacred for that generous
action. The brother of Artavasdes obtained the government of a
province. One of the first military dignities was conferred on
the satrap Otas, a man of singular temperance and fortitude, who
presented to the king his sister\footnotemark[57] and a considerable treasure,
both of which, in a sequestered fortress, Otas had preserved from
violation. Among the Armenian nobles appeared an ally, whose
fortunes are too remarkable to pass unnoticed. His name was
Mamgo,\footnotemark[571] his origin was Scythian, and the horde which
acknowledge his authority had encamped a very few years before on
the skirts of the Chinese empire,\footnotemark[58] which at that time extended
as far as the neighborhood of Sogdiana.\footnotemark[59] Having incurred the
displeasure of his master, Mamgo, with his followers, retired to
the banks of the Oxus, and implored the protection of Sapor. The
emperor of China claimed the fugitive, and alleged the rights of
sovereignty. The Persian monarch pleaded the laws of hospitality,
and with some difficulty avoided a war, by the promise that he
would banish Mamgo to the uttermost parts of the West, a
punishment, as he described it, not less dreadful than death
itself. Armenia was chosen for the place of exile, and a large
district was assigned to the Scythian horde, on which they might
feed their flocks and herds, and remove their encampment from one
place to another, according to the different seasons of the year.

They were employed to repel the invasion of Tiridates; but their
leader, after weighing the obligations and injuries which he had
received from the Persian monarch, resolved to abandon his party.

The Armenian prince, who was well acquainted with the merit as
well as power of Mamgo, treated him with distinguished respect;
and, by admitting him into his confidence, acquired a brave and
faithful servant, who contributed very effectually to his
restoration.\footnotemark[60]

\footnotetext[55]{Moses of Chorene. Hist. Armen. l. ii. c. 74. The
statues had been erected by Valarsaces, who reigned in Armenia
about 130 years before Christ, and was the first king of the
family of Arsaces, (see Moses, Hist. Armen. l. ii. 2, 3.) The
deification of the Arsacides is mentioned by Justin, (xli. 5,)
and by Ammianus Marcellinus, (xxiii. 6.)}

\footnotetext[56]{The Armenian nobility was numerous and powerful.
Moses mentions many families which were distinguished under the
reign of Valarsaces, (l. ii. 7,) and which still subsisted in his
own time, about the middle of the fifth century. See the preface
of his Editors.}

\footnotetext[57]{She was named Chosroiduchta, and had not the os
patulum like other women. (Hist. Armen. l. ii. c. 79.) I do not
understand the expression. * Note: Os patulum signifies merely a
large and widely opening mouth. Ovid (Metam. xv. 513) says,
speaking of the monster who attacked Hippolytus, patulo partem
maris evomit ore. Probably a wide mouth was a common defect among
the Armenian women.—G.}

\footnotetext[571]{Mamgo (according to M. St. Martin, note to Le
Beau. ii. 213) belonged to the imperial race of Hon, who had
filled the throne of China for four hundred years. Dethroned by
the usurping race of Wei, Mamgo found a hospitable reception in
Persia in the reign of Ardeschir. The emperor of china having
demanded the surrender of the fugitive and his partisans, Sapor,
then king, threatened with war both by Rome and China, counselled
Mamgo to retire into Armenia. “I have expelled him from my
dominions, (he answered the Chinese ambassador;) I have banished
him to the extremity of the earth, where the sun sets; I have
dismissed him to certain death.” Compare Mem. sur l’Armenie, ii.
25.—M.}

\footnotetext[58]{In the Armenian history, (l. ii. 78,) as well as in
the Geography, (p. 367,) China is called Zenia, or Zenastan. It
is characterized by the production of silk, by the opulence of
the natives, and by their love of peace, above all the other
nations of the earth. * Note: See St. Martin, Mem. sur l’Armenie,
i. 304.}

\footnotetext[59]{Vou-ti, the first emperor of the seventh dynasty,
who then reigned in China, had political transactions with
Fergana, a province of Sogdiana, and is said to have received a
Roman embassy, (Histoire des Huns, tom. i. p. 38.) In those ages
the Chinese kept a garrison at Kashgar, and one of their
generals, about the time of Trajan, marched as far as the Caspian
Sea. With regard to the intercourse between China and the Western
countries, a curious memoir of M. de Guignes may be consulted, in
the Academie des Inscriptions, tom. xxii. p. 355. * Note: The
Chinese Annals mention, under the ninth year of Yan-hi, which
corresponds with the year 166 J. C., an embassy which arrived
from Tathsin, and was sent by a prince called An-thun, who can be
no other than Marcus Aurelius Antoninus, who then ruled over the
Romans. St. Martin, Mem. sur l’Armænic. ii. 30. See also
Klaproth, Tableaux Historiques de l’Asie, p. 69. The embassy came
by Jy-nan, Tonquin.—M.}

\footnotetext[60]{See Hist. Armen. l. ii. c. 81.}

For a while, fortune appeared to favor the enterprising valor of
Tiridates. He not only expelled the enemies of his family and
country from the whole extent of Armenia, but in the prosecution
of his revenge he carried his arms, or at least his incursions,
into the heart of Assyria. The historian, who has preserved the
name of Tiridates from oblivion, celebrates, with a degree of
national enthusiasm, his personal prowess: and, in the true
spirit of eastern romance, describes the giants and the elephants
that fell beneath his invincible arm. It is from other
information that we discover the distracted state of the Persian
monarchy, to which the king of Armenia was indebted for some part
of his advantages. The throne was disputed by the ambition of
contending brothers; and Hormuz, after exerting without success
the strength of his own party, had recourse to the dangerous
assistance of the barbarians who inhabited the banks of the
Caspian Sea.\footnotemark[61] The civil war was, however, soon terminated,
either by a victor or by a reconciliation; and Narses, who was
universally acknowledged as king of Persia, directed his whole
force against the foreign enemy. The contest then became too
unequal; nor was the valor of the hero able to withstand the
power of the monarch. Tiridates, a second time expelled from the
throne of Armenia, once more took refuge in the court of the
emperors.\footnotemark[611] Narses soon reëstablished his authority over the
revolted province; and loudly complaining of the protection
afforded by the Romans to rebels and fugitives, aspired to the
conquest of the East.\footnotemark[62]

\footnotetext[61]{Ipsos Persas ipsumque Regem ascitis Saccis, et
Russis, et Gellis, petit frater Ormies. Panegyric. Vet. iii. 1.
The Saccæ were a nation of wandering Scythians, who encamped
towards the sources of the Oxus and the Jaxartes. The Gelli where
the inhabitants of Ghilan, along the Caspian Sea, and who so
long, under the name of Dilemines, infested the Persian monarchy.
See d’Herbelot, Bibliotheque}

\footnotetext[611]{M St. Martin represents this differently. Le roi
de Perse * * * profits d’un voyage que Tiridate avoit fait a Rome
pour attaquer ce royaume. This reads like the evasion of the
national historians to disguise the fact discreditable to their
hero. See Mem. sur l’Armenie, i. 304.—M.}

\footnotetext[62]{Moses of Chorene takes no notice of this second
revolution, which I have been obliged to collect from a passage
of Ammianus Marcellinus, (l. xxiii. c. 5.) Lactantius speaks of
the ambition of Narses: “Concitatus domesticis exemplis avi sui
Saporis ad occupandum orientem magnis copiis inhiabat.” De Mort.
Persecut. c. 9.}

Neither prudence nor honor could permit the emperors to forsake
the cause of the Armenian king, and it was resolved to exert the
force of the empire in the Persian war. Diocletian, with the calm
dignity which he constantly assumed, fixed his own station in the
city of Antioch, from whence he prepared and directed the
military operations.\footnotemark[63] The conduct of the legions was intrusted
to the intrepid valor of Galerius, who, for that important
purpose, was removed from the banks of the Danube to those of the
Euphrates. The armies soon encountered each other in the plains
of Mesopotamia, and two battles were fought with various and
doubtful success; but the third engagement was of a more decisive
nature; and the Roman army received a total overthrow, which is
attributed to the rashness of Galerius, who, with an
inconsiderable body of troops, attacked the innumerable host of
the Persians.\footnotemark[64] But the consideration of the country that was
the scene of action, may suggest another reason for his defeat.
The same ground on which Galerius was vanquished, had been
rendered memorable by the death of Crassus, and the slaughter of
ten legions. It was a plain of more than sixty miles, which
extended from the hills of Carrhæ to the Euphrates; a smooth and
barren surface of sandy desert, without a hillock, without a
tree, and without a spring of fresh water.\footnotemark[65] The steady infantry
of the Romans, fainting with heat and thirst, could neither hope
for victory if they preserved their ranks, nor break their ranks
without exposing themselves to the most imminent danger. In this
situation they were gradually encompassed by the superior
numbers, harassed by the rapid evolutions, and destroyed by the
arrows of the barbarian cavalry.

The king of Armenia had signalized his valor in the battle, and
acquired personal glory by the public misfortune. He was pursued
as far as the Euphrates; his horse was wounded, and it appeared
impossible for him to escape the victorious enemy. In this
extremity Tiridates embraced the only refuge which appeared
before him: he dismounted and plunged into the stream. His armor
was heavy, the river very deep, and at those parts at least half
a mile in breadth;\footnotemark[66] yet such was his strength and dexterity,
that he reached in safety the opposite bank.\footnotemark[67] With regard to
the Roman general, we are ignorant of the circumstances of his
escape; but when he returned to Antioch, Diocletian received him,
not with the tenderness of a friend and colleague, but with the
indignation of an offended sovereign. The haughtiest of men,
clothed in his purple, but humbled by the sense of his fault and
misfortune, was obliged to follow the emperor’s chariot above a
mile on foot, and to exhibit, before the whole court, the
spectacle of his disgrace.\footnotemark[68]

\footnotetext[63]{We may readily believe, that Lactantius ascribes to
cowardice the conduct of Diocletian. Julian, in his oration,
says, that he remained with all the forces of the empire; a very
hyperbolical expression.}

\footnotetext[64]{Our five abbreviators, Eutropius, Festus, the two
Victors, and Orosius, all relate the last and great battle; but
Orosius is the only one who speaks of the two former.}

\footnotetext[65]{The nature of the country is finely described by
Plutarch, in the life of Crassus; and by Xenophon, in the first
book of the Anabasis}

\footnotetext[66]{See Foster’s Dissertation in the second volume of
the translation of the Anabasis by Spelman; which I will venture
to recommend as one of the best versions extant.}

\footnotetext[67]{Hist. Armen. l. ii. c. 76. I have transferred this
exploit of Tiridates from an imaginary defeat to the real one of
Galerius.}

\footnotetext[68]{Ammian. Marcellin. l. xiv. The mile, in the hands
of Eutropoius, (ix. 24,) of Festus (c. 25,) and of Orosius, (vii
25), easily increased to several miles}

As soon as Diocletian had indulged his private resentment, and
asserted the majesty of supreme power, he yielded to the
submissive entreaties of the Cæsar, and permitted him to retrieve
his own honor, as well as that of the Roman arms. In the room of
the unwarlike troops of Asia, which had most probably served in
the first expedition, a second army was drawn from the veterans
and new levies of the Illyrian frontier, and a considerable body
of Gothic auxiliaries were taken into the Imperial pay.\footnotemark[69] At the
head of a chosen army of twenty-five thousand men, Galerius again
passed the Euphrates; but, instead of exposing his legions in the
open plains of Mesopotamia he advanced through the mountains of
Armenia, where he found the inhabitants devoted to his cause, and
the country as favorable to the operations of infantry as it was
inconvenient for the motions of cavalry.\footnotemark[70] Adversity had
confirmed the Roman discipline, while the barbarians, elated by
success, were become so negligent and remiss, that in the moment
when they least expected it, they were surprised by the active
conduct of Galerius, who, attended only by two horsemen, had with
his own eyes secretly examined the state and position of their
camp. A surprise, especially in the night time, was for the most
part fatal to a Persian army. “Their horses were tied, and
generally shackled, to prevent their running away; and if an
alarm happened, a Persian had his housing to fix, his horse to
bridle, and his corselet to put on, before he could mount.”\footnotemark[71] On
this occasion, the impetuous attack of Galerius spread disorder
and dismay over the camp of the barbarians. A slight resistance
was followed by a dreadful carnage, and, in the general
confusion, the wounded monarch (for Narses commanded his armies
in person) fled towards the deserts of Media. His sumptuous
tents, and those of his satraps, afforded an immense booty to the
conqueror; and an incident is mentioned, which proves the rustic
but martial ignorance of the legions in the elegant superfluities
of life. A bag of shining leather, filled with pearls, fell into
the hands of a private soldier; he carefully preserved the bag,
but he threw away its contents, judging that whatever was of no
use could not possibly be of any value.\footnotemark[72] The principal loss of
Narses was of a much more affecting nature. Several of his wives,
his sisters, and children, who had attended the army, were made
captives in the defeat. But though the character of Galerius had
in general very little affinity with that of Alexander, he
imitated, after his victory, the amiable behavior of the
Macedonian towards the family of Darius. The wives and children
of Narses were protected from violence and rapine, conveyed to a
place of safety, and treated with every mark of respect and
tenderness, that was due from a generous enemy to their age,
their sex, and their royal dignity.\footnotemark[73]

\footnotetext[69]{Aurelius Victor. Jornandes de Rebus Geticis, c.
21.}

\footnotetext[70]{Aurelius Victor says, “Per Armeniam in hostes
contendit, quæ fermo sola, seu facilior vincendi via est.” He
followed the conduct of Trajan, and the idea of Julius Cæsar.}

\footnotetext[71]{Xenophon’s Anabasis, l. iii. For that reason the
Persian cavalry encamped sixty stadia from the enemy.}

\footnotetext[72]{The story is told by Ammianus, l. xxii. Instead of
saccum, some read scutum.}

\footnotetext[73]{The Persians confessed the Roman superiority in
morals as well as in arms. Eutrop. ix. 24. But this respect and
gratitude of enemies is very seldom to be found in their own
accounts.}

