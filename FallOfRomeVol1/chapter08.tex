\chapter{State Of Persia And Restoration Of The Monarchy.}
\section{Part \thesection.}

\textit{Of The State Of Persia After The Restoration Of The Monarchy By Artaxerxes.}
\vspace{\onelineskip}

Whenever Tacitus indulges himself in those beautiful episodes, in
which he relates some domestic transaction of the Germans or of
the Parthians, his principal object is to relieve the attention
of the reader from a uniform scene of vice and misery. From the
reign of Augustus to the time of Alexander Severus, the enemies
of Rome were in her bosom—the tyrants and the soldiers; and her
prosperity had a very distant and feeble interest in the
revolutions that might happen beyond the Rhine and the Euphrates.
But when the military order had levelled, in wild anarchy, the
power of the prince, the laws of the senate, and even the
discipline of the camp, the barbarians of the North and of the
East, who had long hovered on the frontier, boldly attacked the
provinces of a declining monarchy. Their vexatious inroads were
changed into formidable irruptions, and, after a long vicissitude
of mutual calamities, many tribes of the victorious invaders
established themselves in the provinces of the Roman Empire. To
obtain a clearer knowledge of these great events, we shall
endeavor to form a previous idea of the character, forces, and
designs of those nations who avenged the cause of Hannibal and
Mithridates.

In the more early ages of the world, whilst the forest that
covered Europe afforded a retreat to a few wandering savages, the
inhabitants of Asia were already collected into populous cities,
and reduced under extensive empires the seat of the arts, of
luxury, and of despotism. The Assyrians reigned over the East,\textsuperscript{1}
till the sceptre of Ninus and Semiramis dropped from the hands of
their enervated successors. The Medes and the Babylonians divided
their power, and were themselves swallowed up in the monarchy of
the Persians, whose arms could not be confined within the narrow
limits of Asia. Followed, as it is said, by two millions of
\textit{men}, Xerxes, the descendant of Cyrus, invaded Greece.

Thirty thousand \textit{soldiers}, under the command of Alexander, the
son of Philip, who was intrusted by the Greeks with their glory
and revenge, were sufficient to subdue Persia. The princes of the
house of Seleucus usurped and lost the Macedonian command over
the East. About the same time, that, by an ignominious treaty,
they resigned to the Romans the country on this side Mount Tarus,
they were driven by the Parthians,\textsuperscript{1001} an obscure horde of
Scythian origin, from all the provinces of Upper Asia. The
formidable power of the Parthians, which spread from India to the
frontiers of Syria, was in its turn subverted by Ardshir, or
Artaxerxes; the founder of a new dynasty, which, under the name
of Sassanides, governed Persia till the invasion of the Arabs.
This great revolution, whose fatal influence was soon experienced
by the Romans, happened in the fourth year of Alexander Severus,
two hundred and twenty-six years after the Christian era.\textsuperscript{2} \textsuperscript{201}

\pagenote[1]{An ancient chronologist, quoted by Valleius
Paterculus, (l. i. c. 6,) observes, that the Assyrians, the
Medes, the Persians, and the Macedonians, reigned over Asia one
thousand nine hundred and ninety-five years, from the accession
of Ninus to the defeat of Antiochus by the Romans. As the latter
of these great events happened 289 years before Christ, the
former may be placed 2184 years before the same æra. The
Astronomical Observations, found at Babylon, by Alexander, went
fifty years higher.}

\pagenote[1001]{The Parthians were a tribe of the Indo-Germanic
branch which dwelt on the south-east of the Caspian, and belonged
to the same race as the Getæ, the Massagetæ, and other nations,
confounded by the ancients under the vague denomination of
Scythians. Klaproth, Tableaux Hist. d l’Asie, p. 40. Strabo (p.
747) calls the Parthians Carduchi, i.e., the inhabitants of
Curdistan.—M.}

\pagenote[2]{In the five hundred and thirty-eighth year of the
æra of Seleucus. See Agathias, l. ii. p. 63. This great event
(such is the carelessness of the Orientals) is placed by
Eutychius as high as the tenth year of Commodus, and by Moses of
Chorene as low as the reign of Philip. Ammianus Marcellinus has
so servilely copied (xxiii. 6) his ancient materials, which are
indeed very good, that he describes the family of the Arsacides
as still seated on the Persian throne in the middle of the fourth
century.}

\pagenote[201]{The Persian History, if the poetry of the Shah
Nameh, the Book of Kings, may deserve that name mentions four
dynasties from the earliest ages to the invasion of the Saracens.
The Shah Nameh was composed with the view of perpetuating the
remains of the original Persian records or traditions which had
survived the Saracenic invasion. The task was undertaken by the
poet Dukiki, and afterwards, under the patronage of Mahmood of
Ghazni, completed by Ferdusi. The first of these dynasties is
that of Kaiomors, as Sir W. Jones observes, the dark and fabulous
period; the second, that of the Kaianian, the heroic and
poetical, in which the earned have discovered some curious, and
imagined some fanciful, analogies with the Jewish, the Greek, and
the Roman accounts of the eastern world. See, on the Shah Nameh,
Translation by Goerres, with Von Hammer’s Review, Vienna Jahrbuch
von Lit. 17, 75, 77. Malcolm’s Persia, 8vo. ed. i. 503. Macan’s
Preface to his Critical Edition of the Shah Nameh. On the early
Persian History, a very sensible abstract of various opinions in
Malcolm’s Hist. of Persian.—M.}

Artaxerxes had served with great reputation in the armies of
Artaban, the last king of the Parthians, and it appears that he
was driven into exile and rebellion by royal ingratitude, the
customary reward for superior merit. His birth was obscure, and
the obscurity equally gave room to the aspersions of his enemies,
and the flattery of his adherents. If we credit the scandal of
the former, Artaxerxes sprang from the illegitimate commerce of a
tanner’s wife with a common soldier.\textsuperscript{3} The latter represent him
as descended from a branch of the ancient kings of Persian,
though time and misfortune had gradually reduced his ancestors to
the humble station of private citizens.\textsuperscript{4} As the lineal heir of
the monarchy, he asserted his right to the throne, and challenged
the noble task of delivering the Persians from the oppression
under which they groaned above five centuries since the death of
Darius. The Parthians were defeated in three great battles.\textsuperscript{401}
In the last of these their king Artaban was slain, and the spirit
of the nation was forever broken.\textsuperscript{5} The authority of Artaxerxes
was solemnly acknowledged in a great assembly held at Balch in
Khorasan.\textsuperscript{501} Two younger branches of the royal house of Arsaces
were confounded among the prostrate satraps. A third, more
mindful of ancient grandeur than of present necessity, attempted
to retire, with a numerous train of vessels, towards their
kinsman, the king of Armenia; but this little army of deserters
was intercepted, and cut off, by the vigilance of the conqueror,\textsuperscript{6}
who boldly assumed the double diadem, and the title of King of
Kings, which had been enjoyed by his predecessor. But these
pompous titles, instead of gratifying the vanity of the Persian,
served only to admonish him of his duty, and to inflame in his
soul the ambition of restoring in their full splendor, the
religion and empire of Cyrus.

\pagenote[3]{The tanner’s name was Babec; the soldier’s, Sassan:
from the former Artaxerxes obtained the surname of Babegan, from
the latter all his descendants have been styled Sassanides.}

\pagenote[4]{D’Herbelot, Bibliotheque Orientale, Ardshir.}

\pagenote[401]{In the plain of Hoormuz, the son of Babek was
hailed in the field with the proud title of Shahan Shah, king of
kings—a name ever since assumed by the sovereigns of Persia.
Malcolm, i. 71.—M.}

\pagenote[5]{Dion Cassius, l. lxxx. Herodian, l. vi. p. 207.
Abulpharagins Dynast. p. 80.}

\pagenote[501]{See the Persian account of the rise of Ardeschir
Babegan in Malcolm l 69.—M.}

\pagenote[6]{See Moses Chorenensis, l. ii. c. 65—71.}

I. During the long servitude of Persia under the Macedonian and
the Parthian yoke, the nations of Europe and Asia had mutually
adopted and corrupted each other’s superstitions. The Arsacides,
indeed, practised the worship of the Magi; but they disgraced and
polluted it with a various mixture of foreign idolatry.\textsuperscript{601} The
memory of Zoroaster, the ancient prophet and philosopher of the
Persians,\textsuperscript{7} was still revered in the East; but the obsolete and
mysterious language, in which the Zendavesta was composed,\textsuperscript{8}
opened a field of dispute to seventy sects, who variously
explained the fundamental doctrines of their religion, and were
all indifferently devided by a crowd of infidels, who rejected
the divine mission and miracles of the prophet. To suppress the
idolaters, reunite the schismatics, and confute the unbelievers,
by the infallible decision of a general council, the pious
Artaxerxes summoned the Magi from all parts of his dominions.
These priests, who had so long sighed in contempt and obscurity
obeyed the welcome summons; and, on the appointed day, appeared,
to the number of about eighty thousand. But as the debates of so
tumultuous an assembly could not have been directed by the
authority of reason, or influenced by the art of policy, the
Persian synod was reduced, by successive operations, to forty
thousand, to four thousand, to four hundred, to forty, and at
last to seven Magi, the most respected for their learning and
piety. One of these, Erdaviraph, a young but holy prelate,
received from the hands of his brethren three cups of
soporiferous wine. He drank them off, and instantly fell into a
long and profound sleep. As soon as he waked, he related to the
king and to the believing multitude, his journey to heaven, and
his intimate conferences with the Deity. Every doubt was silenced
by this supernatural evidence; and the articles of the faith of
Zoroaster were fixed with equal authority and precision.\textsuperscript{9} A
short delineation of that celebrated system will be found useful,
not only to display the character of the Persian nation, but to
illustrate many of their most important transactions, both in
peace and war, with the Roman empire.\textsuperscript{10}

\pagenote[601]{Silvestre de Sacy (Antiquites de la Perse) had
proved the neglect of the Zoroastrian religion under the Parthian
kings.—M.}

\pagenote[7]{Hyde and Prideaux, working up the Persian legends
and their own conjectures into a very agreeable story, represent
Zoroaster as a contemporary of Darius Hystaspes. But it is
sufficient to observe, that the Greek writers, who lived almost
in the age of Darius, agree in placing the æra of Zoroaster many
hundred, or even thousand, years before their own time. The
judicious criticisms of Mr. Moyle perceived, and maintained
against his uncle, Dr. Prideaux, the antiquity of the Persian
prophet. See his work, vol. ii. * Note: There are three leading
theories concerning the age of Zoroaster: 1. That which assigns
him to an age of great and almost indefinite antiquity—it is that
of Moyle, adopted by Gibbon, Volney, Recherches sur l’Histoire,
ii. 2. Rhode, also, (die Heilige Sage, \&c.,) in a very ingenious
and ably-developed theory, throws the Bactrian prophet far back
into antiquity 2. Foucher, (Mem. de l’Acad. xxvii. 253,) Tychsen,
(in Com. Soc. Gott. ii. 112), Heeren, (ldeen. i. 459,) and
recently Holty, identify the Gushtasp of the Persian mythological
history with Cyaxares the First, the king of the Medes, and
consider the religion to be Median in its origin. M. Guizot
considers this opinion most probable, note in loc. 3. Hyde,
Prideaux, Anquetil du Perron, Kleuker, Herder, Goerres,
(Mythen-Geschichte,) Von Hammer. (Wien. Jahrbuch, vol. ix.,)
Malcolm, (i. 528,) De Guigniaut, (Relig. de l’Antiq. 2d part,
vol. iii.,) Klaproth, (Tableaux de l’Asie, p. 21,) make Gushtasp
Darius Hystaspes, and Zoroaster his contemporary. The silence of
Herodotus appears the great objection to this theory. Some
writers, as M. Foucher (resting, as M. Guizot observes, on the
doubtful authority of Pliny,) make more than one Zoroaster, and
so attempt to reconcile the conflicting theories.— M.}

\pagenote[8]{That ancient idiom was called the Zend. The language
of the commentary, the Pehlvi, though much more modern, has
ceased many ages ago to be a living tongue. This fact alone (if
it is allowed as authentic) sufficiently warrants the antiquity
of those writings which M d’Anquetil has brought into Europe, and
translated into French. * Note: Zend signifies life, living. The
word means, either the collection of the canonical books of the
followers of Zoroaster, or the language itself in which they are
written. They are the books that contain the word of life whether
the language was originally called Zend, or whether it was so
called from the contents of the books. Avesta means word, oracle,
revelation: this term is not the title of a particular work, but
of the collection of the books of Zoroaster, as the revelation of
Ormuzd. This collection is sometimes called Zendavesta, sometimes
briefly Zend. The Zend was the ancient language of Media, as is
proved by its affinity with the dialects of Armenia and Georgia;
it was already a dead language under the Arsacides in the country
which was the scene of the events recorded in the Zendavesta.
Some critics, among others Richardson and Sir W. Jones, have
called in question the antiquity of these books. The former
pretended that Zend had never been a written or spoken language,
but had been invented in the later times by the Magi, for the
purposes of their art; but Kleuker, in the dissertations which he
added to those of Anquetil and the Abbé Foucher, has proved that
the Zend was a living and spoken language.—G. Sir W. Jones
appears to have abandoned his doubts, on discovering the affinity
between the Zend and the Sanskrit. Since the time of Kleuker,
this question has been investigated by many learned scholars. Sir
W. Jones, Leyden, (Asiat. Research. x. 283,) and Mr. Erskine,
(Bombay Trans. ii. 299,) consider it a derivative from the
Sanskrit. The antiquity of the Zendavesta has likewise been
asserted by Rask, the great Danish linguist, who, according to
Malcolm, brought back from the East fresh transcripts and
additions to those published by Anquetil. According to Rask, the
Zend and Sanskrit are sister dialects; the one the parent of the
Persian, the other of the Indian family of languages.—G. and
M.——But the subject is more satisfactorily illustrated in Bopp’s
comparative Grammar of the Sanscrit, Zend, Greek, Latin,
Lithuanian, Gothic, and German languages. Berlin. 1833-5.
According to Bopp, the Zend is, in some respects, of a more
remarkable structure than the Sanskrit. Parts of the Zendavesta
have been published in the original, by M. Bournouf, at Paris,
and M. Ol. shausen, in Hamburg.—M.——The Pehlvi was the language
of the countries bordering on Assyria, and probably of Assyria
itself. Pehlvi signifies valor, heroism; the Pehlvi, therefore,
was the language of the ancient heroes and kings of Persia, the
valiant. (Mr. Erskine prefers the derivation from Pehla, a
border.—M.) It contains a number of Aramaic roots. Anquetil
considered it formed from the Zend. Kleuker does not adopt this
opinion. The Pehlvi, he says, is much more flowing, and less
overcharged with vowels, than the Zend. The books of Zoroaster,
first written in Zend, were afterwards translated into Pehlvi and
Parsi. The Pehlvi had fallen into disuse under the dynasty of the
Sassanides, but the learned still wrote it. The Parsi, the
dialect of Pars or Farristan, was then prevailing dialect.
Kleuker, Anhang zum Zend Avesta, 2, ii. part i. p. 158, part ii.
31.—G.——Mr. Erskine (Bombay Transactions) considers the existing
Zendavesta to have been compiled in the time of Ardeschir
Babegan.—M.}

\pagenote[9]{Hyde de Religione veterum Pers. c. 21.}

\pagenote[10]{I have principally drawn this account from the
Zendavesta of M. d’Anquetil, and the Sadder, subjoined to Dr.
Hyde’s treatise. It must, however, be confessed, that the studied
obscurity of a prophet, the figurative style of the East, and the
deceitful medium of a French or Latin version may have betrayed
us into error and heresy, in this abridgment of Persian theology.
* Note: It is to be regretted that Gibbon followed the
post-Mahometan Sadder of Hyde.—M.}

The great and fundamental article of the system was the
celebrated doctrine of the two principles; a bold and injudicious
attempt of Eastern philosophy to reconcile the existence of moral
and physical evil with the attributes of a beneficent Creator and
Governor of the world. The first and original Being, in whom, or
by whom, the universe exists, is denominated in the writings of
Zoroaster, \textit{Time without bounds};\textsuperscript{1001a} but it must be confessed,
that this infinite substance seems rather a metaphysical
abstraction of the mind than a real object endowed with
self-consciousness, or possessed of moral perfections. From
either the blind or the intelligent operation of this infinite
Time, which bears but too near an affinity with the chaos of the
Greeks, the two secondary but active principles of the universe
were from all eternity produced, Ormusd and Ahriman, each of them
possessed of the powers of creation, but each disposed, by his
invariable nature, to exercise them with different designs.\textsuperscript{1002}
The principle of good is eternally aborbed in light; the
principle of evil eternally buried in darkness. The wise
benevolence of Ormusd formed man capable of virtue, and
abundantly provided his fair habitation with the materials of
happiness. By his vigilant providence, the motion of the planets,
the order of the seasons, and the temperate mixture of the
elements, are preserved. But the malice of Ahriman has long since
pierced \textit{Ormusd’s egg;} or, in other words, has violated the
harmony of his works. Since that fatal eruption, the most minute
articles of good and evil are intimately intermingled and
agitated together; the rankest poisons spring up amidst the most
salutary plants; deluges, earthquakes, and conflagrations attest
the conflict of Nature, and the little world of man is
perpetually shaken by vice and misfortune. Whilst the rest of
human kind are led away captives in the chains of their infernal
enemy, the faithful Persian alone reserves his religious
adoration for his friend and protector Ormusd, and fights under
his banner of light, in the full confidence that he shall, in the
last day, share the glory of his triumph. At that decisive
period, the enlightened wisdom of goodness will render the power
of Ormusd superior to the furious malice of his rival. Ahriman
and his followers, disarmed and subdued, will sink into their
native darkness; and virtue will maintain the eternal peace and
harmony of the universe.\textsuperscript{11} \textsuperscript{1101}

\pagenote[1001a]{Zeruane Akerene, so translated by Anquetil and
Kleuker. There is a dissertation of Foucher on this subject, Mem.
de l’Acad. des Inscr. t. xxix. According to Bohlen (das alte
Indien) it is the Sanskrit Sarvan Akaranam, the Uncreated Whole;
or, according to Fred. Schlegel, Sarvan Akharyam the Uncreate
Indivisible.—M.}

\pagenote[1002]{This is an error. Ahriman was not forced by his
invariable nature to do evil; the Zendavesta expressly recognizes
(see the Izeschne) that he was born good, that in his origin he
was light; envy rendered him evil; he became jealous of the power
and attributes of Ormuzd; then light was changed into darkness,
and Ahriman was precipitated into the abyss. See the Abridgment
of the Doctrine of the Ancient Persians, by Anquetil, c. ii
Section 2.—G.}

\pagenote[11]{The modern Parsees (and in some degree the Sadder)
exalt Ormusd into the first and omnipotent cause, whilst they
degrade Ahriman into an inferior but rebellious spirit. Their
desire of pleasing the Mahometans may have contributed to refine
their theological systems.}

\pagenote[1101]{According to the Zendavesta, Ahriman will not be
annihilated or precipitated forever into darkness: at the
resurrection of the dead he will be entirely defeated by Ormuzd,
his power will be destroyed, his kingdom overthrown to its
foundations, he will himself be purified in torrents of melting
metal; he will change his heart and his will, become holy,
heavenly establish in his dominions the law and word of Ormuzd,
unite himself with him in everlasting friendship, and both will
sing hymns in honor of the Great Eternal. See Anquetil’s
Abridgment. Kleuker, Anhang part iii. p 85, 36; and the Izeschne,
one of the books of the Zendavesta. According to the Sadder
Bun-Dehesch, a more modern work, Ahriman is to be annihilated:
but this is contrary to the text itself of the Zendavesta, and to
the idea its author gives of the kingdom of Eternity, after the
twelve thousand years assigned to the contest between Good and
Evil.—G.}

\section{Part \thesection.}

The theology of Zoroaster was darkly comprehended by foreigners,
and even by the far greater number of his disciples; but the most
careless observers were struck with the philosophic simplicity of
the Persian worship. “That people,” said Herodotus,\textsuperscript{12} “rejects
the use of temples, of altars, and of statues, and smiles at the
folly of those nations who imagine that the gods are sprung from,
or bear any affinity with, the human nature. The tops of the
highest mountains are the places chosen for sacrifices. Hymns and
prayers are the principal worship; the Supreme God, who fills the
wide circle of heaven, is the object to whom they are addressed.”
Yet, at the same time, in the true spirit of a polytheist, he
accuseth them of adoring Earth, Water, Fire, the Winds, and the
Sun and Moon. But the Persians of every age have denied the
charge, and explained the equivocal conduct, which might appear
to give a color to it. The elements, and more particularly Fire,
Light, and the Sun, whom they called Mithra, \textsuperscript{1201} were the
objects of their religious reverence because they considered them
as the purest symbols, the noblest productions, and the most
powerful agents of the Divine Power and Nature.\textsuperscript{13}

\pagenote[12]{Herodotus, l. i. c. 131. But Dr. Prideaux thinks,
with reason, that the use of temples was afterwards permitted in
the Magian religion. Note: The Pyræa, or fire temples of the
Zoroastrians, (observes Kleuker, Persica, p. 16,) were only to be
found in Media or Aderbidjan, provinces into which Herodotus did
not penetrate.—M.}

\pagenote[1201]{Among the Persians Mithra is not the Sun:
Anquetil has contested and triumphantly refuted the opinion of
those who confound them, and it is evidently contrary to the text
of the Zendavesta. Mithra is the first of the genii, or jzeds,
created by Ormuzd; it is he who watches over all nature. Hence
arose the misapprehension of some of the Greeks, who have said
that Mithra was the summus deus of the Persians: he has a
thousand ears and ten thousand eyes. The Chaldeans appear to have
assigned him a higher rank than the Persians. It is he who
bestows upon the earth the light of the sun. The sun. named Khor,
(brightness,) is thus an inferior genius, who, with many other
genii, bears a part in the functions of Mithra. These assistant
genii to another genius are called his kamkars; but in the
Zendavesta they are never confounded. On the days sacred to a
particular genius, the Persian ought to recite, not only the
prayers addressed to him, but those also which are addressed to
his kamkars; thus the hymn or iescht of Mithra is recited on the
day of the sun, (Khor,) and vice versa. It is probably this which
has sometimes caused them to be confounded; but Anquetil had
himself exposed this error, which Kleuker, and all who have
studied the Zendavesta, have noticed. See viii. Diss. of
Anquetil. Kleuker’s Anhang, part iii. p. 132.—G. M. Guizot is
unquestionably right, according to the pure and original doctrine
of the Zend. The Mithriac worship, which was so extensively
propagated in the West, and in which Mithra and the sun were
perpetually confounded, seems to have been formed from a fusion
of Zoroastrianism and Chaldaism, or the Syrian worship of the
sun. An excellent abstract of the question, with references to
the works of the chief modern writers on his curious subject, De
Sacy, Kleuker, Von Hammer, \&c., may be found in De Guigniaut’s
translation of Kreuzer. Relig. d’Antiquite, notes viii. ix. to
book ii. vol. i. 2d part, page 728.—M.}

\pagenote[13]{Hyde de Relig. Pers. c. 8. Notwithstanding all
their distinctions and protestations, which seem sincere enough,
their tyrants, the Mahometans, have constantly stigmatized them
as idolatrous worshippers of the fire.}

Every mode of religion, to make a deep and lasting impression on
the human mind, must exercise our obedience, by enjoining
practices of devotion, for which we can assign no reason; and
must acquire our esteem, by inculcating moral duties analogous to
the dictates of our own hearts. The religion of Zoroaster was
abundantly provided with the former and possessed a sufficient
portion of the latter. At the age of puberty, the faithful
Persian was invested with a mysterious girdle, the badge of the
divine protection; and from that moment all the actions of his
life, even the most indifferent, or the most necessary, were
sanctified by their peculiar prayers, ejaculations, or
genuflections; the omission of which, under any circumstances,
was a grievous sin, not inferior in guilt to the violation of the
moral duties. The moral duties, however, of justice, mercy,
liberality, \&c., were in their turn required of the disciple of
Zoroaster, who wished to escape the persecution of Ahriman, and
to live with Ormusd in a blissful eternity, where the degree of
felicity will be exactly proportioned to the degree of virtue and
piety.\textsuperscript{14}

\pagenote[14]{See the Sadder, the smallest part of which consists
of moral precepts. The ceremonies enjoined are infinite and
trifling. Fifteen genuflections, prayers, \&c., were required
whenever the devout Persian cut his nails or made water; or as
often as he put on the sacred girdle Sadder, Art. 14, 50, 60. *
Note: Zoroaster exacted much less ceremonial observance, than at
a later period, the priests of his doctrines. This is the
progress of all religions the worship, simple in its origin, is
gradually overloaded with minute superstitions. The maxim of the
Zendavesta, on the relative merit of sowing the earth and of
prayers, quoted below by Gibbon, proves that Zoroaster did not
attach too much importance to these observances. Thus it is not
from the Zendavesta that Gibbon derives the proof of his
allegation, but from the Sadder, a much later work.—G}

But there are some remarkable instances in which Zoroaster lays
aside the prophet, assumes the legislator, and discovers a
liberal concern for private and public happiness, seldom to be
found among the grovelling or visionary schemes of superstition.
Fasting and celibacy, the common means of purchasing the divine
favor, he condemns with abhorrence as a criminal rejection of the
best gifts of Providence. The saint, in the Magian religion, is
obliged to beget children, to plant useful trees, to destroy
noxious animals, to convey water to the dry lands of Persia, and
to work out his salvation by pursuing all the labors of
agriculture.\textsuperscript{1401} We may quote from the Zendavesta a wise and
benevolent maxim, which compensates for many an absurdity. “He
who sows the ground with care and diligence acquires a greater
stock of religious merit than he could gain by the repetition of
ten thousand prayers.”\textsuperscript{15} In the spring of every year a festival
was celebrated, destined to represent the primitive equality, and
the present connection, of mankind. The stately kings of Persia,
exchanging their vain pomp for more genuine greatness, freely
mingled with the humblest but most useful of their subjects. On
that day the husbandmen were admitted, without distinction, to
the table of the king and his satraps. The monarch accepted their
petitions, inquired into their grievances, and conversed with
them on the most equal terms. “From your labors,” was he
accustomed to say, (and to say with truth, if not with
sincerity,) “from your labors we receive our subsistence; you
derive your tranquillity from our vigilance: since, therefore, we
are mutually necessary to each other, let us live together like
brothers in concord and love.”\textsuperscript{16} Such a festival must indeed
have degenerated, in a wealthy and despotic empire, into a
theatrical representation; but it was at least a comedy well
worthy of a royal audience, and which might sometimes imprint a
salutary lesson on the mind of a young prince.

\pagenote[1401]{See, on Zoroaster’s encouragement of agriculture,
the ingenious remarks of Heeren, Ideen, vol. i. p. 449, \&c., and
Rhode, Heilige Sage, p. 517—M.}

\pagenote[15]{Zendavesta, tom. i. p. 224, and Precis du Systeme
de Zoroastre, tom. iii.}

\pagenote[16]{Hyde de Religione Persarum, c. 19.}

Had Zoroaster, in all his institutions, invariably supported this
exalted character, his name would deserve a place with those of
Numa and Confucius, and his system would be justly entitled to
all the applause, which it has pleased some of our divines, and
even some of our philosophers, to bestow on it. But in that
motley composition, dictated by reason and passion, by enthusiasm
and by selfish motives, some useful and sublime truths were
disgraced by a mixture of the most abject and dangerous
superstition. The Magi, or sacerdotal order, were extremely
numerous, since, as we have already seen, fourscore thousand of
them were convened in a general council. Their forces were
multiplied by discipline. A regular hierarchy was diffused
through all the provinces of Persia; and the Archimagus, who
resided at Balch, was respected as the visible head of the
church, and the lawful successor of Zoroaster.\textsuperscript{17} The property of
the Magi was very considerable. Besides the less invidious
possession of a large tract of the most fertile lands of Media,\textsuperscript{18}
they levied a general tax on the fortunes and the industry of
the Persians.\textsuperscript{19} “Though your good works,” says the interested
prophet, “exceed in number the leaves of the trees, the drops of
rain, the stars in the heaven, or the sands on the sea-shore,
they will all be unprofitable to you, unless they are accepted by
the \textit{destour}, or priest. To obtain the acceptation of this guide
to salvation, you must faithfully pay him \textit{tithes} of all you
possess, of your goods, of your lands, and of your money. If the
destour be satisfied, your soul will escape hell tortures; you
will secure praise in this world and happiness in the next. For
the destours are the teachers of religion; they know all things,
and they deliver all men.”\textsuperscript{20} \textsuperscript{201a}

\pagenote[17]{Hyde de Religione Persarum, c. 28. Both Hyde and
Prideaux affect to apply to the Magian the terms consecrated to
the Christian hierarchy.}

\pagenote[18]{Ammian. Marcellin. xxiii. 6. He informs us (as far
as we may credit him) of two curious particulars: 1. That the
Magi derived some of their most secret doctrines from the Indian
Brachmans; and 2. That they were a tribe, or family, as well as
order.}

\pagenote[19]{The divine institution of tithes exhibits a
singular instance of conformity between the law of Zoroaster and
that of Moses. Those who cannot otherwise account for it, may
suppose, if they please that the Magi of the latter times
inserted so useful an interpolation into the writings of their
prophet.}

\pagenote[20]{Sadder, Art. viii.}

\pagenote[201a]{The passage quoted by Gibbon is not taken from
the writings of Zoroaster, but from the Sadder, a work, as has
been before said, much later than the books which form the
Zendavesta. and written by a Magus for popular use; what it
contains, therefore, cannot be attributed to Zoroaster. It is
remarkable that Gibbon should fall into this error, for Hyde
himself does not ascribe the Sadder to Zoroaster; he remarks that
it is written inverse, while Zoroaster always wrote in prose.
Hyde, i. p. 27. Whatever may be the case as to the latter
assertion, for which there appears little foundation, it is
unquestionable that the Sadder is of much later date. The Abbé
Foucher does not even believe it to be an extract from the works
of Zoroaster. See his Diss. before quoted. Mem. de l’Acad. des
Ins. t. xxvii.—G. Perhaps it is rash to speak of any part of the
Zendavesta as the writing of Zoroaster, though it may be a
genuine representation of his. As to the Sadder, Hyde (in Præf.)
considered it not above 200 years old. It is manifestly
post-Mahometan. See Art. xxv. on fasting.—M.}

These convenient maxims of reverence and implicit faith were
doubtless imprinted with care on the tender minds of youth; since
the Magi were the masters of education in Persia, and to their
hands the children even of the royal family were intrusted.\textsuperscript{21}
The Persian priests, who were of a speculative genius, preserved
and investigated the secrets of Oriental philosophy; and
acquired, either by superior knowledge, or superior art, the
reputation of being well versed in some occult sciences, which
have derived their appellation from the Magi.\textsuperscript{22} Those of more
active dispositions mixed with the world in courts and cities;
and it is observed, that the administration of Artaxerxes was in
a great measure directed by the counsels of the sacerdotal order,
whose dignity, either from policy or devotion, that prince
restored to its ancient splendor.\textsuperscript{23}

\pagenote[21]{Plato in Alcibiad.}

\pagenote[22]{Pliny (Hist. Natur. l. xxx. c. 1) observes, that
magic held mankind by the triple chain of religion, of physic,
and of astronomy.}

\pagenote[23]{Agathias, l. iv. p. 134.}

The first counsel of the Magi was agreeable to the unsociable
genius of their faith,\textsuperscript{24} to the practice of ancient kings,\textsuperscript{25}
and even to the example of their legislator, who had fallen a
victim to a religious war, excited by his own intolerant zeal.\textsuperscript{26}
By an edict of Artaxerxes, the exercise of every worship, except
that of Zoroaster, was severely prohibited. The temples of the
Parthians, and the statues of their deified monarchs, were thrown
down with ignominy.\textsuperscript{27} The sword of Aristotle (such was the name
given by the Orientals to the polytheism and philosophy of the
Greeks) was easily broken; \textsuperscript{28} the flames of persecution soon
reached the more stubborn Jews and Christians;\textsuperscript{29} nor did they
spare the heretics of their own nation and religion. The majesty
of Ormusd, who was jealous of a rival, was seconded by the
despotism of Artaxerxes, who could not suffer a rebel; and the
schismatics within his vast empire were soon reduced to the
inconsiderable number of eighty thousand.\textsuperscript{30} \textsuperscript{301} This spirit of
persecution reflects dishonor on the religion of Zoroaster; but
as it was not productive of any civil commotion, it served to
strengthen the new monarchy, by uniting all the various
inhabitants of Persia in the bands of religious zeal. \textsuperscript{302}

\pagenote[24]{Mr. Hume, in the Natural History of Religion,
sagaciously remarks, that the most refined and philosophic sects
are constantly the most intolerant. * Note: Hume’s comparison is
rather between theism and polytheism. In India, in Greece, and in
modern Europe, philosophic religion has looked down with
contemptuous toleration on the superstitions of the vulgar.—M.}

\pagenote[25]{Cicero de Legibus, ii. 10. Xerxes, by the advice of
the Magi, destroyed the temples of Greece.}

\pagenote[26]{Hyde de Relig. Persar. c. 23, 24. D’Herbelot,
Bibliotheque Orientale, Zurdusht. Life of Zoroaster in tom. ii.
of the Zendavesta.}

\pagenote[27]{Compare Moses of Chorene, l. ii. c. 74, with
Ammian. Marcel lin. xxiii. 6. Hereafter I shall make use of these
passages.}

\pagenote[28]{Rabbi Abraham, in the Tarikh Schickard, p. 108,
109.}

\pagenote[29]{Basnage, Histoire des Juifs, l. viii. c. 3.
Sozomen, l. ii. c. 1 Manes, who suffered an ignominious death,
may be deemed a Magian as well as a Christian heretic.}

\pagenote[30]{Hyde de Religione Persar. c. 21.}

\pagenote[301]{It is incorrect to attribute these persecutions to
Artaxerxes. The Jews were held in honor by him, and their schools
flourished during his reign. Compare Jost, Geschichte der
Isræliter, b. xv. 5, with Basnage. Sapor was forced by the people
to temporary severities; but their real persecution did not begin
till the reigns of Yezdigerd and Kobad. Hist. of Jews, iii. 236.
According to Sozomen, i. viii., Sapor first persecuted the
Christians. Manes was put to death by Varanes the First, A. D.
277. Beausobre, Hist. de Man. i. 209.—M.}

\pagenote[302]{In the testament of Ardischer in Ferdusi, the poet
assigns these sentiments to the dying king, as he addresses his
son: Never forget that as a king, you are at once the protector
of religion and of your country. Consider the altar and the
throne as inseparable; they must always sustain each other.
Malcolm’s Persia. i. 74—M}

II. Artaxerxes, by his valor and conduct, had wrested the sceptre
of the East from the ancient royal family of Parthia. There still
remained the more difficult task of establishing, throughout the
vast extent of Persia, a uniform and vigorous administration. The
weak indulgence of the Arsacides had resigned to their sons and
brothers the principal provinces, and the greatest offices of the
kingdom in the nature of hereditary possessions. The \textit{vitaxæ}, or
eighteen most powerful satraps, were permitted to assume the
regal title; and the vain pride of the monarch was delighted with
a nominal dominion over so many vassal kings. Even tribes of
barbarians in their mountains, and the Greek cities of Upper
Asia,\textsuperscript{31} within their walls, scarcely acknowledged, or seldom
obeyed. any superior; and the Parthian empire exhibited, under
other names, a lively image of the feudal system\textsuperscript{32} which has
since prevailed in Europe. But the active victor, at the head of
a numerous and disciplined army, visited in person every province
of Persia. The defeat of the boldest rebels, and the reduction of
the strongest fortifications,\textsuperscript{33} diffused the terror of his arms,
and prepared the way for the peaceful reception of his authority.
An obstinate resistance was fatal to the chiefs; but their
followers were treated with lenity.\textsuperscript{34} A cheerful submission was
rewarded with honors and riches, but the prudent Artaxerxes,
suffering no person except himself to assume the title of king,
abolished every intermediate power between the throne and the
people. His kingdom, nearly equal in extent to modern Persia,
was, on every side, bounded by the sea, or by great rivers; by
the Euphrates, the Tigris, the Araxes, the Oxus, and the Indus,
by the Caspian Sea, and the Gulf of Persia.\textsuperscript{35} That country was
computed to contain, in the last century, five hundred and
fifty-four cities, sixty thousand villages, and about forty
millions of souls.\textsuperscript{36} If we compare the administration of the
house of Sassan with that of the house of Sefi, the political
influence of the Magian with that of the Mahometan religion, we
shall probably infer, that the kingdom of Artaxerxes contained at
least as great a number of cities, villages, and inhabitants. But
it must likewise be confessed, that in every age the want of
harbors on the sea-coast, and the scarcity of fresh water in the
inland provinces, have been very unfavorable to the commerce and
agriculture of the Persians; who, in the calculation of their
numbers, seem to have indulged one of the meanest, though most
common, artifices of national vanity.

\pagenote[31]{These colonies were extremely numerous. Seleucus
Nicator founded thirty-nine cities, all named from himself, or
some of his relations, (see Appian in Syriac. p. 124.) The æra of
Seleucus (still in use among the eastern Christians) appears as
late as the year 508, of Christ 196, on the medals of the Greek
cities within the Parthian empire. See Moyle’s works, vol. i. p.
273, \&c., and M. Freret, Mem. de l’Academie, tom. xix.}

\pagenote[32]{The modern Persians distinguish that period as the
dynasty of the kings of the nations. See Plin. Hist. Nat. vi.
25.}

\pagenote[33]{Eutychius (tom. i. p. 367, 371, 375) relates the
siege of the island of Mesene in the Tigris, with some
circumstances not unlike the story of Nysus and Scylla.}

\pagenote[34]{Agathias, ii. 64, [and iv. p. 260.] The princes of
Segestan de fended their independence during many years. As
romances generally transport to an ancient period the events of
their own time, it is not impossible that the fabulous exploits
of Rustan, Prince of Segestan, many have been grafted on this
real history.}

\pagenote[35]{We can scarcely attribute to the Persian monarchy
the sea-coast of Gedrosia or Macran, which extends along the
Indian Ocean from Cape Jask (the promontory Capella) to Cape
Goadel. In the time of Alexander, and probably many ages
afterwards, it was thinly inhabited by a savage people of
Icthyophagi, or Fishermen, who knew no arts, who acknowledged no
master, and who were divided by in-hospitable deserts from the
rest of the world. (See Arrian de Reb. Indicis.) In the twelfth
century, the little town of Taiz (supposed by M. d’Anville to be
the Teza of Ptolemy) was peopled and enriched by the resort of
the Arabian merchants. (See Geographia Nubiens, p. 58, and
d’Anville, Geographie Ancienne, tom. ii. p. 283.) In the last
age, the whole country was divided between three princes, one
Mahometan and two Idolaters, who maintained their independence
against the successors of Shah Abbas. (Voyages de Tavernier, part
i. l. v. p. 635.)}

\pagenote[36]{Chardin, tom. iii c 1 2, 3.}

As soon as the ambitious mind of Artaxerxes had triumphed ever
the resistance of his vassals, he began to threaten the
neighboring states, who, during the long slumber of his
predecessors, had insulted Persia with impunity. He obtained some
easy victories over the wild Scythians and the effeminate
Indians; but the Romans were an enemy, who, by their past
injuries and present power, deserved the utmost efforts of his
arms. A forty years’ tranquillity, the fruit of valor and
moderation, had succeeded the victories of Trajan. During the
period that elapsed from the accession of Marcus to the reign of
Alexander, the Roman and the Parthian empires were twice engaged
in war; and although the whole strength of the Arsacides
contended with a part only of the forces of Rome, the event was
most commonly in favor of the latter. Macrinus, indeed, prompted
by his precarious situation and pusillanimous temper, purchased a
peace at the expense of near two millions of our money;\textsuperscript{37} but
the generals of Marcus, the emperor Severus, and his son, erected
many trophies in Armenia, Mesopotamia, and Assyria. Among their
exploits, the imperfect relation of which would have unseasonably
interrupted the more important series of domestic revolutions, we
shall only mention the repeated calamities of the two great
cities of Seleucia and Ctesiphon.

\pagenote[37]{Dion, l. xxviii. p. 1335.}

Seleucia, on the western bank of the Tigris, about forty-five
miles to the north of ancient Babylon, was the capital of the
Macedonian conquests in Upper Asia.\textsuperscript{38} Many ages after the fall
of their empire, Seleucia retained the genuine characters of a
Grecian colony, arts, military virtue, and the love of freedom.
The independent republic was governed by a senate of three
hundred nobles; the people consisted of six hundred thousand
citizens; the walls were strong, and as long as concord prevailed
among the several orders of the state, they viewed with contempt
the power of the Parthian: but the madness of faction was
sometimes provoked to implore the dangerous aid of the common
enemy, who was posted almost at the gates of the colony.\textsuperscript{39} The
Parthian monarchs, like the Mogul sovereigns of Hindostan,
delighted in the pastoral life of their Scythian ancestors; and
the Imperial camp was frequently pitched in the plain of
Ctesiphon, on the eastern bank of the Tigris, at the distance of
only three miles from Seleucia.\textsuperscript{40} The innumerable attendants on
luxury and despotism resorted to the court, and the little
village of Ctesiphon insensibly swelled into a great city.\textsuperscript{41}
Under the reign of Marcus, the Roman generals penetrated as far
as Ctesiphon and Seleucia. They were received as friends by the
Greek colony; they attacked as enemies the seat of the Parthian
kings; yet both cities experienced the same treatment. The sack
and conflagration of Seleucia, with the massacre of three hundred
thousand of the inhabitants, tarnished the glory of the Roman
triumph.\textsuperscript{42} Seleucia, already exhausted by the neighborhood of a
too powerful rival, sunk under the fatal blow; but Ctesiphon, in
about thirty-three years, had sufficiently recovered its strength
to maintain an obstinate siege against the emperor Severus. The
city was, however, taken by assault; the king, who defended it in
person, escaped with precipitation; a hundred thousand captives,
and a rich booty, rewarded the fatigues of the Roman soldiers.\textsuperscript{43}
Notwithstanding these misfortunes, Ctesiphon succeeded to Babylon
and to Seleucia, as one of the great capitals of the East. In
summer, the monarch of Persia enjoyed at Ecbatana the cool
breezes of the mountains of Media; but the mildness of the
climate engaged him to prefer Ctesiphon for his winter residence.

\pagenote[38]{For the precise situation of Babylon, Seleucia,
Ctesiphon, Moiain, and Bagdad, cities often confounded with each
other, see an excellent Geographical Tract of M. d’Anville, in
Mem. de l’Academie, tom. xxx.}

\pagenote[39]{Tacit. Annal. xi. 42. Plin. Hist. Nat. vi. 26.}

\pagenote[40]{This may be inferred from Strabo, l. xvi. p. 743.}

\pagenote[41]{That most curious traveller, Bernier, who followed
the camp of Aurengzebe from Delhi to Cashmir, describes with
great accuracy the immense moving city. The guard of cavalry
consisted of 35,000 men, that of infantry of 10,000. It was
computed that the camp contained 150,000 horses, mules, and
elephants; 50,000 camels, 50,000 oxen, and between 300,000 and
400,000 persons. Almost all Delhi followed the court, whose
magnificence supported its industry.}

\pagenote[42]{Dion, l. lxxi. p. 1178. Hist. August. p. 38.
Eutrop. viii. 10 Euseb. in Chronic. Quadratus (quoted in the
Augustan History) attempted to vindicate the Romans by alleging
that the citizens of Seleucia had first violated their faith.}

\pagenote[43]{Dion, l. lxxv. p. 1263. Herodian, l. iii. p. 120.
Hist. August. p. 70.}

From these successful inroads the Romans derived no real or
lasting benefit; nor did they attempt to preserve such distant
conquests, separated from the provinces of the empire by a large
tract of intermediate desert. The reduction of the kingdom of
Osrhoene was an acquisition of less splendor indeed, but of a far
more solid advantage. That little state occupied the northern and
most fertile part of Mesopotamia, between the Euphrates and the
Tigris. Edessa, its capital, was situated about twenty miles
beyond the former of those rivers; and the inhabitants, since the
time of Alexander, were a mixed race of Greeks, Arabs, Syrians,
and Armenians.\textsuperscript{44} The feeble sovereigns of Osrhoene, placed on
the dangerous verge of two contending empires, were attached from
inclination to the Parthian cause; but the superior power of Rome
exacted from them a reluctant homage, which is still attested by
their medals. After the conclusion of the Parthian war under
Marcus, it was judged prudent to secure some substantial pledges
of their doubtful fidelity. Forts were constructed in several
parts of the country, and a Roman garrison was fixed in the
strong town of Nisibis. During the troubles that followed the
death of Commodus, the princes of Osrhoene attempted to shake off
the yoke; but the stern policy of Severus confirmed their
dependence,\textsuperscript{45} and the perfidy of Caracalla completed the easy
conquest. Abgarus, the last king of Edessa, was sent in chains to
Rome, his dominions reduced into a province, and his capital
dignified with the rank of colony; and thus the Romans, about ten
years before the fall of the Parthian monarchy, obtained a firm
and permanent establishment beyond the Euphrates.\textsuperscript{46}

\pagenote[44]{The polished citizens of Antioch called those of
Edessa mixed barbarians. It was, however, some praise, that of
the three dialects of the Syriac, the purest and most elegant
(the Aramæan) was spoken at Edessa. This remark M. Bayer (Hist.
Edess. p 5) has borrowed from George of Malatia, a Syrian
writer.}

\pagenote[45]{Dion, l. lxxv. p. 1248, 1249, 1250. M. Bayer has
neglected to use this most important passage.}

\pagenote[46]{This kingdom, from Osrhoes, who gave a new name to
the country, to the last Abgarus, had lasted 353 years. See the
learned work of M. Bayer, Historia Osrhoena et Edessena.}

Prudence as well as glory might have justified a war on the side
of Artaxerxes, had his views been confined to the defence or
acquisition of a useful frontier. but the ambitious Persian
openly avowed a far more extensive design of conquest; and he
thought himself able to support his lofty pretensions by the arms
of reason as well as by those of power. Cyrus, he alleged, had
first subdued, and his successors had for a long time possessed,
the whole extent of Asia, as far as the Propontis and the Ægean
Sea; the provinces of Caria and Ionia, under their empire, had
been governed by Persian satraps, and all Egypt, to the confines
of Æthiopia, had acknowledged their sovereignty.\textsuperscript{47} Their rights
had been suspended, but not destroyed, by a long usurpation; and
as soon as he received the Persian diadem, which birth and
successful valor had placed upon his head, the first great duty
of his station called upon him to restore the ancient limits and
splendor of the monarchy. The Great King, therefore, (such was
the haughty style of his embassies to the emperor Alexander,)
commanded the Romans instantly to depart from all the provinces
of his ancestors, and, yielding to the Persians the empire of
Asia, to content themselves with the undisturbed possession of
Europe. This haughty mandate was delivered by four hundred of the
tallest and most beautiful of the Persians; who, by their fine
horses, splendid arms, and rich apparel, displayed the pride and
greatness of their master.\textsuperscript{48} Such an embassy was much less an
offer of negotiation than a declaration of war. Both Alexander
Severus and Artaxerxes, collecting the military force of the
Roman and Persian monarchies, resolved in this important contest
to lead their armies in person.

\pagenote[47]{Xenophon, in the preface to the Cyropædia, gives a
clear and magnificent idea of the extent of the empire of Cyrus.
Herodotus (l. iii. c. 79, \&c.) enters into a curious and
particular description of the twenty great Satrapies into which
the Persian empire was divided by Darius Hystaspes.}

\pagenote[48]{Herodian, vi. 209, 212.}

If we credit what should seem the most authentic of all records,
an oration, still extant, and delivered by the emperor himself to
the senate, we must allow that the victory of Alexander Severus
was not inferior to any of those formerly obtained over the
Persians by the son of Philip. The army of the Great King
consisted of one hundred and twenty thousand horse, clothed in
complete armor of steel; of seven hundred elephants, with towers
filled with archers on their backs, and of eighteen hundred
chariots armed with scythes. This formidable host, the like of
which is not to be found in eastern history, and has scarcely
been imagined in eastern romance,\textsuperscript{49} was discomfited in a great
battle, in which the Roman Alexander proved himself an intrepid
soldier and a skilful general. The Great King fled before his
valor; an immense booty, and the conquest of Mesopotamia, were
the immediate fruits of this signal victory. Such are the
circumstances of this ostentatious and improbable relation,
dictated, as it too plainly appears, by the vanity of the
monarch, adorned by the unblushing servility of his flatterers,
and received without contradiction by a distant and obsequious
senate.\textsuperscript{50} Far from being inclined to believe that the arms of
Alexander obtained any memorable advantage over the Persians, we
are induced to suspect that all this blaze of imaginary glory was
designed to conceal some real disgrace.

\pagenote[49]{There were two hundred scythed chariots at the
battle of Arbela, in the host of Darius. In the vast army of
Tigranes, which was vanquished by Lucullus, seventeen thousand
horse only were completely armed. Antiochus brought fifty-four
elephants into the field against the Romans: by his frequent wars
and negotiations with the princes of India, he had once collected
a hundred and fifty of those great animals; but it may be
questioned whether the most powerful monarch of Hindostan evci
formed a line of battle of seven hundred elephants. Instead of
three or four thousand elephants, which the Great Mogul was
supposed to possess, Tavernier (Voyages, part ii. l. i. p. 198)
discovered, by a more accurate inquiry, that he had only five
hundred for his baggage, and eighty or ninety for the service of
war. The Greeks have varied with regard to the number which Porus
brought into the field; but Quintus Curtius, (viii. 13,) in this
instance judicious and moderate, is contented with eighty-five
elephants, distinguished by their size and strength. In Siam,
where these animals are the most numerous and the most esteemed,
eighteen elephants are allowed as a sufficient proportion for
each of the nine brigades into which a just army is divided. The
whole number, of one hundred and sixty-two elephants of war, may
sometimes be doubled. Hist. des Voyages, tom. ix. p. 260. * Note:
Compare Gibbon’s note 10 to ch. lvii—M.}

\pagenote[50]{Hist. August. p. 133. * Note: See M. Guizot’s note,
p. 267. According to the Persian authorities Ardeschir extended
his conquests to the Euphrates. Malcolm i. 71.—M.}

Our suspicions are confirmed by the authority of a contemporary
historian, who mentions the virtues of Alexander with respect,
and his faults with candor. He describes the judicious plan which
had been formed for the conduct of the war. Three Roman armies
were destined to invade Persia at the same time, and by different
roads. But the operations of the campaign, though wisely
concerted, were not executed either with ability or success. The
first of these armies, as soon as it had entered the marshy
plains of Babylon, towards the artificial conflux of the
Euphrates and the Tigris,\textsuperscript{51} was encompassed by the superior
numbers, and destroyed by the arrows of the enemy. The alliance
of Chosroes, king of Armenia,\textsuperscript{52} and the long tract of
mountainous country, in which the Persian cavalry was of little
service, opened a secure entrance into the heart of Media, to the
second of the Roman armies. These brave troops laid waste the
adjacent provinces, and by several successful actions against
Artaxerxes, gave a faint color to the emperor’s vanity. But the
retreat of this victorious army was imprudent, or at least
unfortunate. In repassing the mountains, great numbers of
soldiers perished by the badness of the roads, and the severity
of the winter season. It had been resolved, that whilst these two
great detachments penetrated into the opposite extremes of the
Persian dominions, the main body, under the command of Alexander
himself, should support their attack, by invading the centre of
the kingdom. But the unexperienced youth, influenced by his
mother’s counsels, and perhaps by his own fears, deserted the
bravest troops, and the fairest prospect of victory; and after
consuming in Mesopotamia an inactive and inglorious summer, he
led back to Antioch an army diminished by sickness, and provoked
by disappointment. The behavior of Artaxerxes had been very
different. Flying with rapidity from the hills of Media to the
marshes of the Euphrates, he had everywhere opposed the invaders
in person; and in either fortune had united with the ablest
conduct the most undaunted resolution. But in several obstinate
engagements against the veteran legions of Rome, the Persian
monarch had lost the flower of his troops. Even his victories had
weakened his power. The favorable opportunities of the absence of
Alexander, and of the confusions that followed that emperor’s
death, presented themselves in vain to his ambition. Instead of
expelling the Romans, as he pretended, from the continent of
Asia, he found himself unable to wrest from their hands the
little province of Mesopotamia.\textsuperscript{53}

\pagenote[51]{M. de Tillemont has already observed, that
Herodian’s geography is somewhat confused.}

\pagenote[52]{Moses of Chorene (Hist. Armen. l. ii. c. 71)
illustrates this invasion of Media, by asserting that Chosroes,
king of Armenia, defeated Artaxerxes, and pursued him to the
confines of India. The exploits of Chosroes have been magnified;
and he acted as a dependent ally to the Romans.}

\pagenote[53]{For the account of this war, see Herodian, l. vi.
p. 209, 212. The old abbreviators and modern compilers have
blindly followed the Augustan History.}

The reign of Artaxerxes, which, from the last defeat of the
Parthians, lasted only fourteen years, forms a memorable æra in
the history of the East, and even in that of Rome. His character
seems to have been marked by those bold and commanding features,
that generally distinguish the princes who conquer, from those
who inherit, an empire. Till the last period of the Persian
monarchy, his code of laws was respected as the groundwork of
their civil and religious policy.\textsuperscript{54} Several of his sayings are
preserved. One of them in particular discovers a deep insight
into the constitution of government. “The authority of the
prince,” said Artaxerxes, “must be defended by a military force;
that force can only be maintained by taxes; all taxes must, at
last, fall upon agriculture; and agriculture can never flourish
except under the protection of justice and moderation.”\textsuperscript{55}
Artaxerxes bequeathed his new empire, and his ambitious designs
against the Romans, to Sapor, a son not unworthy of his great
father; but those designs were too extensive for the power of
Persia, and served only to involve both nations in a long series
of destructive wars and reciprocal calamities.

\pagenote[54]{Eutychius, tom. ii. p. 180, vers. Pocock. The great
Chosroes Noushirwan sent the code of Artaxerxes to all his
satraps, as the invariable rule of their conduct.}

\pagenote[55]{D’Herbelot, Bibliotheque Orientale, au mot Ardshir.
We may observe, that after an ancient period of fables, and a
long interval of darkness, the modern histories of Persia begin
to assume an air of truth with the dynasty of Sassanides. Compare
Malcolm, i. 79.—M.}

The Persians, long since civilized and corrupted, were very far
from possessing the martial independence, and the intrepid
hardiness, both of mind and body, which have rendered the
northern barbarians masters of the world. The science of war,
that constituted the more rational force of Greece and Rome, as
it now does of Europe, never made any considerable progress in
the East. Those disciplined evolutions which harmonize and
animate a confused multitude, were unknown to the Persians. They
were equally unskilled in the arts of constructing, besieging, or
defending regular fortifications. They trusted more to their
numbers than to their courage; more to their courage than to
their discipline. The infantry was a half-armed, spiritless crowd
of peasants, levied in haste by the allurements of plunder, and
as easily dispersed by a victory as by a defeat. The monarch and
his nobles transported into the camp the pride and luxury of the
seraglio. Their military operations were impeded by a useless
train of women, eunuchs, horses, and camels; and in the midst of
a successful campaign, the Persian host was often separated or
destroyed by an unexpected famine.\textsuperscript{56}

\pagenote[56]{Herodian, l. vi. p. 214. Ammianus Marcellinus, l.
xxiii. c. 6. Some differences may be observed between the two
historians, the natural effects of the changes produced by a
century and a half.}

But the nobles of Persia, in the bosom of luxury and despotism,
preserved a strong sense of personal gallantry and national
honor. From the age of seven years they were taught to speak
truth, to shoot with the bow, and to ride; and it was universally
confessed that in the two last of these arts they had made a more
than common proficiency.\textsuperscript{57} The most distinguished youth were
educated under the monarch’s eye, practised their exercises in
the gate of his palace, and were severely trained up to the
habits of temperance and obedience, in their long and laborious
parties of hunting. In every province, the satrap maintained a
like school of military virtue. The Persian nobles (so natural is
the idea of feudal tenures) received from the king’s bounty lands
and houses, on the condition of their service in war. They were
ready on the first summons to mount on horseback, with a martial
and splendid train of followers, and to join the numerous bodies
of guards, who were carefully selected from among the most robust
slaves, and the bravest adventurers of Asia. These armies, both
of light and of heavy cavalry, equally formidable by the
impetuosity of their charge and the rapidity of their motions,
threatened, as an impending cloud, the eastern provinces of the
declining empire of Rome.\textsuperscript{58}

\pagenote[57]{The Persians are still the most skilful horsemen,
and their horses the finest in the East.}

\pagenote[58]{From Herodotus, Xenophon, Herodian, Ammianus,
Chardin, \&c., I have extracted such probable accounts of the
Persian nobility, as seem either common to every age, or
particular to that of the Sassanides.}

