\section{Part \thesection.}
\thispagestyle{simple}

The Goths were now in possession of the Ukraine, a country of
considerable extent and uncommon fertility, intersected with
navigable rivers, which, from either side, discharge themselves
into the Borysthenes; and interspersed with large and lofty
forests of oaks. The plenty of game and fish, the innumerable
bee-hives deposited in the hollow of old trees, and in the
cavities of rocks, and forming, even in that rude age, a valuable
branch of commerce, the size of the cattle, the temperature of
the air, the aptness of the soil for every species of grain, and
the luxuriancy of the vegetation, all displayed the liberality of
Nature, and tempted the industry of man.\footnotemark[28] But the Goths
withstood all these temptations, and still adhered to a life of
idleness, of poverty, and of rapine.

\footnotetext[28]{Genealogical History of the Tartars, p. 593. Mr.
Bell (vol. ii. p 379) traversed the Ukraine, in his journey from
Petersburgh to Constantinople. The modern face of the country is
a just representation of the ancient, since, in the hands of the
Cossacks, it still remains in a state of nature.}

The Scythian hordes, which, towards the east, bordered on the new
settlements of the Goths, presented nothing to their arms, except
the doubtful chance of an unprofitable victory. But the prospect
of the Roman territories was far more alluring; and the fields of
Dacia were covered with rich harvests, sown by the hands of an
industrious, and exposed to be gathered by those of a warlike,
people. It is probable that the conquests of Trajan, maintained
by his successors, less for any real advantage than for ideal
dignity, had contributed to weaken the empire on that side. The
new and unsettled province of Dacia was neither strong enough to
resist, nor rich enough to satiate, the rapaciousness of the
barbarians. As long as the remote banks of the Niester were
considered as the boundary of the Roman power, the fortifications
of the Lower Danube were more carelessly guarded, and the
inhabitants of Mæsia lived in supine security, fondly conceiving
themselves at an inaccessible distance from any barbarian
invaders. The irruptions of the Goths, under the reign of Philip,
fatally convinced them of their mistake. The king, or leader, of
that fierce nation, traversed with contempt the province of
Dacia, and passed both the Niester and the Danube without
encountering any opposition capable of retarding his progress.
The relaxed discipline of the Roman troops betrayed the most
important posts, where they were stationed, and the fear of
deserved punishment induced great numbers of them to enlist under
the Gothic standard. The various multitude of barbarians
appeared, at length, under the walls of Marcianopolis, a city
built by Trajan in honor of his sister, and at that time the
capital of the second Mæsia.\footnotemark[29] The inhabitants consented to
ransom their lives and property by the payment of a large sum of
money, and the invaders retreated back into their deserts,
animated, rather than satisfied, with the first success of their
arms against an opulent but feeble country. Intelligence was soon
transmitted to the emperor Decius, that Cniva, king of the Goths,
had passed the Danube a second time, with more considerable
forces; that his numerous detachments scattered devastation over
the province of Mæsia, whilst the main body of the army,
consisting of seventy thousand Germans and Sarmatians, a force
equal to the most daring achievements, required the presence of
the Roman monarch, and the exertion of his military power.

\footnotetext[29]{In the sixteenth chapter of Jornandes, instead of
secundo Mæsiam we may venture to substitute secundam, the second
Mæsia, of which Marcianopolis was certainly the capital. (See
Hierocles de Provinciis, and Wesseling ad locum, p. 636.
Itinerar.) It is surprising how this palpable error of the scribe
should escape the judicious correction of Grotius. Note: Luden
has observed that Jornandes mentions two passages over the
Danube; this relates to the second irruption into Mæsia.
Geschichte des T V. ii. p. 448.—M.}

Decius found the Goths engaged before Nicopolis, one of the many
monuments of Trajan’s victories.\footnotemark[30] On his approach they raised
the siege, but with a design only of marching away to a conquest
of greater importance, the siege of Philippopolis, a city of
Thrace, founded by the father of Alexander, near the foot of
Mount Hæmus.\footnotemark[31] Decius followed them through a difficult country,
and by forced marches; but when he imagined himself at a
considerable distance from the rear of the Goths, Cniva turned
with rapid fury on his pursuers. The camp of the Romans was
surprised and pillaged, and, for the first time, their emperor
fled in disorder before a troop of half-armed barbarians. After a
long resistance, Philoppopolis, destitute of succor, was taken by
storm. A hundred thousand persons are reported to have been
massacred in the sack of that great city.\footnotemark[32] Many prisoners of
consequence became a valuable accession to the spoil; and
Priscus, a brother of the late emperor Philip, blushed not to
assume the purple, under the protection of the barbarous enemies
of Rome.\footnotemark[33] The time, however, consumed in that tedious siege,
enabled Decius to revive the courage, restore the discipline, and
recruit the numbers of his troops. He intercepted several parties
of Carpi, and other Germans, who were hastening to share the
victory of their countrymen,\footnotemark[34] intrusted the passes of the
mountains to officers of approved valor and fidelity,\footnotemark[35] repaired
and strengthened the fortifications of the Danube, and exerted
his utmost vigilance to oppose either the progress or the retreat
of the Goths. Encouraged by the return of fortune, he anxiously
waited for an opportunity to retrieve, by a great and decisive
blow, his own glory, and that of the Roman arms.\footnotemark[36]

\footnotetext[30]{The place is still called Nicop. D’Anville,
Geographie Ancienne, tom. i. p. 307. The little stream, on whose
banks it stood, falls into the Danube.}

\footnotetext[31]{Stephan. Byzant. de Urbibus, p. 740. Wesseling,
Itinerar. p. 136. Zonaras, by an odd mistake, ascribes the
foundation of Philippopolis to the immediate predecessor of
Decius. * Note: Now Philippopolis or Philiba; its situation among
the hills caused it to be also called Trimontium. D’Anville,
Geog. Anc. i. 295.—G.}

\footnotetext[32]{Ammian. xxxi. 5.}

\footnotetext[33]{Aurel. Victor. c. 29.}

\footnotetext[34]{Victoriæ Carpicæ, on some medals of Decius,
insinuate these advantages.}

\footnotetext[35]{Claudius (who afterwards reigned with so much
glory) was posted in the pass of Thermopylæ with 200 Dardanians,
100 heavy and 160 light horse, 60 Cretan archers, and 1000
well-armed recruits. See an original letter from the emperor to
his officer, in the Augustan History, p. 200.}

\footnotetext[36]{Jornandes, c. 16—18. Zosimus, l. i. p. 22. In the
general account of this war, it is easy to discover the opposite
prejudices of the Gothic and the Grecian writer. In carelessness
alone they are alike.}

At the same time when Decius was struggling with the violence of
the tempest, his mind, calm and deliberate amidst the tumult of
war, investigated the more general causes that, since the age of
the Antonines, had so impetuously urged the decline of the Roman
greatness. He soon discovered that it was impossible to replace
that greatness on a permanent basis without restoring public
virtue, ancient principles and manners, and the oppressed majesty
of the laws. To execute this noble but arduous design, he first
resolved to revive the obsolete office of censor; an office
which, as long as it had subsisted in its pristine integrity, had
so much contributed to the perpetuity of the state,\footnotemark[37] till it
was usurped and gradually neglected by the Cæsars.\footnotemark[38] Conscious
that the favor of the sovereign may confer power, but that the
esteem of the people can alone bestow authority, he submitted the
choice of the censor to the unbiased voice of the senate. By
their unanimous votes, or rather acclamations, Valerian, who was
afterwards emperor, and who then served with distinction in the
army of Decius, was declared the most worthy of that exalted
honor. As soon as the decree of the senate was transmitted to the
emperor, he assembled a great council in his camp, and before the
investiture of the censor elect, he apprised him of the
difficulty and importance of his great office. “Happy Valerian,”
said the prince to his distinguished subject, “happy in the
general approbation of the senate and of the Roman republic!
Accept the censorship of mankind; and judge of our manners. You
will select those who deserve to continue members of the senate;
you will restore the equestrian order to its ancient splendor;
you will improve the revenue, yet moderate the public burdens.
You will distinguish into regular classes the various and
infinite multitude of citizens, and accurately view the military
strength, the wealth, the virtue, and the resources of Rome. Your
decisions shall obtain the force of laws. The army, the palace,
the ministers of justice, and the great officers of the empire,
are all subject to your tribunal. None are exempted, excepting
only the ordinary consuls,\footnotemark[39] the præfect of the city, the king
of the sacrifices, and (as long as she preserves her chastity
inviolate) the eldest of the vestal virgins. Even these few, who
may not dread the severity, will anxiously solicit the esteem, of
the Roman censor.”\footnotemark[40]

\footnotetext[37]{Montesquieu, Grandeur et Decadence des Romains, c.
viii. He illustrates the nature and use of the censorship with
his usual ingenuity, and with uncommon precision.}

\footnotetext[38]{Vespasian and Titus were the last censors, (Pliny,
Hist. Natur vii. 49. Censorinus de Die Natali.) The modesty of
Trajan refused an honor which he deserved, and his example became
a law to the Antonines. See Pliny’s Panegyric, c. 45 and 60.}

\footnotetext[39]{Yet in spite of his exemption, Pompey appeared
before that tribunal during his consulship. The occasion, indeed,
was equally singular and honorable. Plutarch in Pomp. p. 630.}

\footnotetext[40]{See the original speech in the Augustan Hist. p.
173-174.}

A magistrate, invested with such extensive powers, would have
appeared not so much the minister, as the colleague of his
sovereign.\footnotemark[41] Valerian justly dreaded an elevation so full of
envy and of suspicion. He modestly argued the alarming greatness
of the trust, his own insufficiency, and the incurable corruption
of the times. He artfully insinuated, that the office of censor
was inseparable from the Imperial dignity, and that the feeble
hands of a subject were unequal to the support of such an immense
weight of cares and of power.\footnotemark[42] The approaching event of war
soon put an end to the prosecution of a project so specious, but
so impracticable; and whilst it preserved Valerian from the
danger, saved the emperor Decius from the disappointment, which
would most probably have attended it. A censor may maintain, he
can never restore, the morals of a state. It is impossible for
such a magistrate to exert his authority with benefit, or even
with effect, unless he is supported by a quick sense of honor and
virtue in the minds of the people, by a decent reverence for the
public opinion, and by a train of useful prejudices combating on
the side of national manners. In a period when these principles
are annihilated, the censorial jurisdiction must either sink into
empty pageantry, or be converted into a partial instrument of
vexatious oppression.\footnotemark[43] It was easier to vanquish the Goths than
to eradicate the public vices; yet even in the first of these
enterprises, Decius lost his army and his life.

\footnotetext[41]{This transaction might deceive Zonaras, who
supposes that Valerian was actually declared the colleague of
Decius, l. xii. p. 625.}

\footnotetext[42]{Hist. August. p. 174. The emperor’s reply is
omitted.}

\footnotetext[43]{Such as the attempts of Augustus towards a
reformation of manness. Tacit. Annal. iii. 24.}

The Goths were now, on every side, surrounded and pursued by the
Roman arms. The flower of their troops had perished in the long
siege of Philippopolis, and the exhausted country could no longer
afford subsistence for the remaining multitude of licentious
barbarians. Reduced to this extremity, the Goths would gladly
have purchased, by the surrender of all their booty and
prisoners, the permission of an undisturbed retreat. But the
emperor, confident of victory, and resolving, by the chastisement
of these invaders, to strike a salutary terror into the nations
of the North, refused to listen to any terms of accommodation.
The high-spirited barbarians preferred death to slavery. An
obscure town of Mæsia, called Forum Terebronii,\footnotemark[44] was the scene
of the battle. The Gothic army was drawn up in three lines, and
either from choice or accident, the front of the third line was
covered by a morass. In the beginning of the action, the son of
Decius, a youth of the fairest hopes, and already associated to
the honors of the purple, was slain by an arrow, in the sight of
his afflicted father; who, summoning all his fortitude,
admonished the dismayed troops, that the loss of a single soldier
was of little importance to the republic.\footnotemark[45] The conflict was
terrible; it was the combat of despair against grief and rage.
The first line of the Goths at length gave way in disorder; the
second, advancing to sustain it, shared its fate; and the third
only remained entire, prepared to dispute the passage of the
morass, which was imprudently attempted by the presumption of the
enemy. “Here the fortune of the day turned, and all things became
adverse to the Romans; the place deep with ooze, sinking under
those who stood, slippery to such as advanced; their armor heavy,
the waters deep; nor could they wield, in that uneasy situation,
their weighty javelins. The barbarians, on the contrary, were
inured to encounter in the bogs, their persons tall, their spears
long, such as could wound at a distance.”\footnotemark[46] In this morass the
Roman army, after an ineffectual struggle, was irrecoverably
lost; nor could the body of the emperor ever be found.\footnotemark[47] Such
was the fate of Decius, in the fiftieth year of his age; an
accomplished prince, active in war and affable in peace;\footnotemark[48] who,
together with his son, has deserved to be compared, both in life
and death, with the brightest examples of ancient virtue.\footnotemark[49]

\footnotetext[44]{Tillemont, Histoire des Empereurs, tom. iii. p.
598. As Zosimus and some of his followers mistake the Danube for
the Tanais, they place the field of battle in the plains of
Scythia.}

\footnotetext[45]{Aurelius Victor allows two distinct actions for the
deaths of the two Decii; but I have preferred the account of
Jornandes.}

\footnotetext[46]{I have ventured to copy from Tacitus (Annal. i. 64)
the picture of a similar engagement between a Roman army and a
German tribe.}

\footnotetext[47]{Jornandes, c. 18. Zosimus, l. i. p. 22, [c. 23.]
Zonaras, l. xii. p. 627. Aurelius Victor.}

\footnotetext[48]{The Decii were killed before the end of the year
two hundred and fifty-one, since the new princes took possession
of the consulship on the ensuing calends of January.}

\footnotetext[49]{Hist. August. p. 223, gives them a very honorable
place among the small number of good emperors who reigned between
Augustus and Diocletian.}

This fatal blow humbled, for a very little time, the insolence of
the legions. They appeared to have patiently expected, and
submissively obeyed, the decree of the senate which regulated the
succession to the throne. From a just regard for the memory of
Decius, the Imperial title was conferred on Hostilianus, his only
surviving son; but an equal rank, with more effectual power, was
granted to Gallus, whose experience and ability seemed equal to
the great trust of guardian to the young prince and the
distressed empire.\footnotemark[50] The first care of the new emperor was to
deliver the Illyrian provinces from the intolerable weight of the
victorious Goths. He consented to leave in their hands the rich
fruits of their invasion, an immense booty, and what was still
more disgraceful, a great number of prisoners of the highest
merit and quality. He plentifully supplied their camp with every
conveniency that could assuage their angry spirits or facilitate
their so much wished-for departure; and he even promised to pay
them annually a large sum of gold, on condition they should never
afterwards infest the Roman territories by their incursions.\footnotemark[51]

\footnotetext[50]{Hæc ubi Patres comperere.. .. decernunt. Victor in
Cæsaribus.}

\footnotetext[51]{Zonaras, l. xii. p. 628.}

In the age of the Scipios, the most opulent kings of the earth,
who courted the protection of the victorious commonwealth, were
gratified with such trifling presents as could only derive a
value from the hand that bestowed them; an ivory chair, a coarse
garment of purple, an inconsiderable piece of plate, or a
quantity of copper coin.\footnotemark[52] After the wealth of nations had
centred in Rome, the emperors displayed their greatness, and even
their policy, by the regular exercise of a steady and moderate
liberality towards the allies of the state. They relieved the
poverty of the barbarians, honored their merit, and recompensed
their fidelity. These voluntary marks of bounty were understood
to flow, not from the fears, but merely from the generosity or
the gratitude of the Romans; and whilst presents and subsidies
were liberally distributed among friends and suppliants, they
were sternly refused to such as claimed them as a debt.\footnotemark[53] But
this stipulation, of an annual payment to a victorious enemy,
appeared without disguise in the light of an ignominious tribute;
the minds of the Romans were not yet accustomed to accept such
unequal laws from a tribe of barbarians; and the prince, who by a
necessary concession had probably saved his country, became the
object of the general contempt and aversion. The death of
Hostiliamus, though it happened in the midst of a raging
pestilence, was interpreted as the personal crime of Gallus;\footnotemark[54]
and even the defeat of the later emperor was ascribed by the
voice of suspicion to the perfidious counsels of his hated
successor.\footnotemark[55] The tranquillity which the empire enjoyed during
the first year of his administration,\footnotemark[56] served rather to inflame
than to appease the public discontent; and as soon as the
apprehensions of war were removed, the infamy of the peace was
more deeply and more sensibly felt.

\footnotetext[52]{A \textit{Sella}, a \textit{Toga}, and a golden \textit{Patera} of five
pounds weight, were accepted with joy and gratitude by the
wealthy king of Egypt. (Livy, xxvii. 4.) \textit{Quina millia Æris}, a
weight of copper, in value about eighteen pounds sterling, was
the usual present made to foreign are ambassadors. (Livy, xxxi.
9.)}

\footnotetext[53]{See the firmness of a Roman general so late as the
time of Alexander Severus, in the Excerpta Legationum, p. 25,
edit. Louvre.}

\footnotetext[54]{For the plague, see Jornandes, c. 19, and Victor in
Cæsaribus.}

\footnotetext[55]{These improbable accusations are alleged by
Zosimus, l. i. p. 28, 24.}

\footnotetext[56]{Jornandes, c. 19. The Gothic writer at least
observed the peace which his victorious countrymen had sworn to
Gallus.}

But the Romans were irritated to a still higher degree, when they
discovered that they had not even secured their repose, though at
the expense of their honor. The dangerous secret of the wealth
and weakness of the empire had been revealed to the world. New
swarms of barbarians, encouraged by the success, and not
conceiving themselves bound by the obligation of their brethren,
spread devastation though the Illyrian provinces, and terror as
far as the gates of Rome. The defence of the monarchy, which
seemed abandoned by the pusillanimous emperor, was assumed by
Æmilianus, governor of Pannonia and Mæsia; who rallied the
scattered forces, and revived the fainting spirits of the troops.
The barbarians were unexpectedly attacked, routed, chased, and
pursued beyond the Danube. The victorious leader distributed as a
donative the money collected for the tribute, and the
acclamations of the soldiers proclaimed him emperor on the field
of battle.\footnotemark[57] Gallus, who, careless of the general welfare,
indulged himself in the pleasures of Italy, was almost in the
same instant informed of the success, of the revolt, and of the
rapid approach of his aspiring lieutenant. He advanced to meet
him as far as the plains of Spoleto. When the armies came in
sight of each other, the soldiers of Gallus compared the
ignominious conduct of their sovereign with the glory of his
rival. They admired the valor of Æmilianus; they were attracted
by his liberality, for he offered a considerable increase of pay
to all deserters.\footnotemark[58] The murder of Gallus, and of his son
Volusianus, put an end to the civil war; and the senate gave a
legal sanction to the rights of conquest. The letters of
Æmilianus to that assembly displayed a mixture of moderation and
vanity. He assured them, that he should resign to their wisdom
the civil administration; and, contenting himself with the
quality of their general, would in a short time assert the glory
of Rome, and deliver the empire from all the barbarians both of
the North and of the East.\footnotemark[59] His pride was flattered by the
applause of the senate; and medals are still extant, representing
him with the name and attributes of Hercules the Victor, and Mars
the Avenger.\footnotemark[60]

\footnotetext[57]{Zosimus, l. i. p. 25, 26.}

\footnotetext[58]{Victor in Cæsaribus.}

\footnotetext[59]{Zonaras, l. xii. p. 628.}

\footnotetext[60]{Banduri Numismata, p. 94.}

If the new monarch possessed the abilities, he wanted the time,
necessary to fulfil these splendid promises. Less than four
months intervened between his victory and his fall.\footnotemark[61] He had
vanquished Gallus: he sunk under the weight of a competitor more
formidable than Gallus. That unfortunate prince had sent
Valerian, already distinguished by the honorable title of censor,
to bring the legions of Gaul and Germany\footnotemark[62] to his aid. Valerian
executed that commission with zeal and fidelity; and as he
arrived too late to save his sovereign, he resolved to revenge
him. The troops of Æmilianus, who still lay encamped in the
plains of Spoleto, were awed by the sanctity of his character,
but much more by the superior strength of his army; and as they
were now become as incapable of personal attachment as they had
always been of constitutional principle, they readily imbrued
their hands in the blood of a prince who so lately had been the
object of their partial choice. The guilt was theirs,\footnotemark[621] but the
advantage of it was Valerian’s; who obtained the possession of
the throne by the means indeed of a civil war, but with a degree
of innocence singular in that age of revolutions; since he owed
neither gratitude nor allegiance to his predecessor, whom he
dethroned.

\footnotetext[61]{Eutropius, l. ix. c. 6, says tertio mense. Eusebio
this emperor.}

\footnotetext[62]{Zosimus, l. i. p. 28. Eutropius and Victor station
Valerian’s army in Rhætia.}

\footnotetext[621]{Aurelius Victor says that Æmilianus died of a
natural disorder. Tropius, in speaking of his death, does not say
that he was assassinated—G.}

Valerian was about sixty years of age\footnotemark[63] when he was invested
with the purple, not by the caprice of the populace, or the
clamors of the army, but by the unanimous voice of the Roman
world. In his gradual ascent through the honors of the state, he
had deserved the favor of virtuous princes, and had declared
himself the enemy of tyrants.\footnotemark[64] His noble birth, his mild but
unblemished manners, his learning, prudence, and experience, were
revered by the senate and people; and if mankind (according to
the observation of an ancient writer) had been left at liberty to
choose a master, their choice would most assuredly have fallen on
Valerian.\footnotemark[65] Perhaps the merit of this emperor was inadequate to
his reputation; perhaps his abilities, or at least his spirit,
were affected by the languor and coldness of old age. The
consciousness of his decline engaged him to share the throne with
a younger and more active associate;\footnotemark[66] the emergency of the
times demanded a general no less than a prince; and the
experience of the Roman censor might have directed him where to
bestow the Imperial purple, as the reward of military merit. But
instead of making a judicious choice, which would have confirmed
his reign and endeared his memory, Valerian, consulting only the
dictates of affection or vanity, immediately invested with the
supreme honors his son Gallienus, a youth whose effeminate vices
had been hitherto concealed by the obscurity of a private
station. The joint government of the father and the son subsisted
about seven, and the sole administration of Gallienus continued
about eight, years. But the whole period was one uninterrupted
series of confusion and calamity. As the Roman empire was at the
same time, and on every side, attacked by the blind fury of
foreign invaders, and the wild ambition of domestic usurpers, we
shall consult order and perspicuity, by pursuing, not so much the
doubtful arrangement of dates, as the more natural distribution
of subjects. The most dangerous enemies of Rome, during the
reigns of Valerian and Gallienus, were, 1. The Franks; 2. The
Alemanni; 3. The Goths; and, 4. The Persians. Under these general
appellations, we may comprehend the adventures of less
considerable tribes, whose obscure and uncouth names would only
serve to oppress the memory and perplex the attention of the
reader.

\footnotetext[63]{He was about seventy at the time of his accession,
or, as it is more probable, of his death. Hist. August. p. 173.
Tillemont, Hist. des Empereurs, tom. iii. p. 893, note 1.}

\footnotetext[64]{Inimicus tyrannorum. Hist. August. p. 173. In the
glorious struggle of the senate against Maximin, Valerian acted a
very spirited part. Hist. August. p. 156.}

\footnotetext[65]{According to the distinction of Victor, he seems to
have received the title of Imperator from the army, and that of
Augustus from the senate.}

\footnotetext[66]{From Victor and from the medals, Tillemont (tom.
iii. p. 710) very justly infers, that Gallienus was associated to
the empire about the month of August of the year 253.}

I. As the posterity of the Franks compose one of the greatest and
most enlightened nations of Europe, the powers of learning and
ingenuity have been exhausted in the discovery of their
unlettered ancestors. To the tales of credulity have succeeded
the systems of fancy. Every passage has been sifted, every spot
has been surveyed, that might possibly reveal some faint traces
of their origin. It has been supposed that Pannonia,\footnotemark[67] that
Gaul, that the northern parts of Germany,\footnotemark[68] gave birth to that
celebrated colony of warriors. At length the most rational
critics, rejecting the fictitious emigrations of ideal
conquerors, have acquiesced in a sentiment whose simplicity
persuades us of its truth.\footnotemark[69] They suppose, that about the year
two hundred and forty,\footnotemark[70] a new confederacy was formed under the
name of Franks, by the old inhabitants of the Lower Rhine and the
Weser.\footnotemark[701] The present circle of Westphalia, the Landgraviate of
Hesse, and the duchies of Brunswick and Luneburg, were the
ancient seat of the Chauci who, in their inaccessible morasses,
defied the Roman arms;\footnotemark[71] of the Cherusci, proud of the fame of
Arminius; of the Catti, formidable by their firm and intrepid
infantry; and of several other tribes of inferior power and
renown.\footnotemark[72] The love of liberty was the ruling passion of these
Germans; the enjoyment of it their best treasure; the word that
expressed that enjoyment the most pleasing to their ear. They
deserved, they assumed, they maintained the honorable epithet of
Franks, or Freemen; which concealed, though it did not
extinguish, the peculiar names of the several states of the
confederacy.\footnotemark[73] Tacit consent, and mutual advantage, dictated the
first laws of the union; it was gradually cemented by habit and
experience. The league of the Franks may admit of some comparison
with the Helvetic body; in which every canton, retaining its
independent sovereignty, consults with its brethren in the common
cause, without acknowledging the authority of any supreme head or
representative assembly.\footnotemark[74] But the principle of the two
confederacies was extremely different. A peace of two hundred
years has rewarded the wise and honest policy of the Swiss. An
inconstant spirit, the thirst of rapine, and a disregard to the
most solemn treaties, disgraced the character of the Franks.

\footnotetext[67]{Various systems have been formed to explain a
difficult passage in Gregory of Tours, l. ii. c. 9.}

\footnotetext[68]{The Geographer of Ravenna, i. 11, by mentioning
Mauringania, on the confines of Denmark, as the ancient seat of
the Franks, gave birth to an ingenious system of Leibritz.}

\footnotetext[69]{See Cluver. Germania Antiqua, l. iii. c. 20. M.
Freret, in the Memoires de l’Academie des Inscriptions, tom.
xviii.}

\footnotetext[70]{Most probably under the reign of Gordian, from an
accidental circumstance fully canvassed by Tillemont, tom. iii.
p. 710, 1181.}

\footnotetext[701]{The confederation of the Franks appears to have
been formed, 1. Of the Chauci. 2. Of the Sicambri, the
inhabitants of the duchy of Berg. 3. Of the Attuarii, to the
north of the Sicambri, in the principality of Waldeck, between
the Dimel and the Eder. 4. Of the Bructeri, on the banks of the
Lippe, and in the Hartz. 5. Of the Chamavii, the Gambrivii of
Tacitua, who were established, at the time of the Frankish
confederation, in the country of the Bructeri. 6. Of the Catti,
in Hessia.—G. The Salii and Cherasci are added. Greenwood’s Hist.
of Germans, i 193.—M.}

\footnotetext[71]{Plin. Hist. Natur. xvi. l. The Panegyrists
frequently allude to the morasses of the Franks.}

\footnotetext[72]{Tacit. Germania, c. 30, 37.}

\footnotetext[73]{In a subsequent period, most of those old names are
occasionally mentioned. See some vestiges of them in Cluver.
Germ. Antiq. l. iii.}

\footnotetext[74]{Simler de Republica Helvet. cum notis Fuselin.}

