\chapter{Destruction Of Paganism.}

\textit{Final Destruction Of Paganism. — Introduction Of The Worship Of
Saints, And Relics, Among The Christians.}

The ruin of Paganism, in the age of Theodosius, is perhaps the
only example of the total extirpation of any ancient and popular
superstition; and may therefore deserve to be considered as a
singular event in the history of the human mind. The Christians,
more especially the clergy, had impatiently supported the prudent
delays of Constantine, and the equal toleration of the elder
Valentinian; nor could they deem their conquest perfect or
secure, as long as their adversaries were permitted to exist. The
influence which Ambrose and his brethren had acquired over the
youth of Gratian, and the piety of Theodosius, was employed to
infuse the maxims of persecution into the breasts of their
Imperial proselytes. Two specious principles of religious
jurisprudence were established, from whence they deduced a direct
and rigorous conclusion, against the subjects of the empire who
still adhered to the ceremonies of their ancestors: that the
magistrate is, in some measure, guilty of the crimes which he
neglects to prohibit, or to punish; and, that the idolatrous
worship of fabulous deities, and real daemons, is the most
abominable crime against the supreme majesty of the Creator. The
laws of Moses, and the examples of Jewish history,\textsuperscript{1} were
hastily, perhaps erroneously, applied, by the clergy, to the mild
and universal reign of Christianity.\textsuperscript{2} The zeal of the emperors
was excited to vindicate their own honor, and that of the Deity:
and the temples of the Roman world were subverted, about sixty
years after the conversion of Constantine.

\pagenote[1]{St. Ambrose (tom. ii. de Obit. Theodos. p. 1208)
expressly praises and recommends the zeal of Josiah in the
destruction of idolatry The language of Julius Firmicus Maternus
on the same subject (de Errore Profan. Relig. p. 467, edit.
Gronov.) is piously inhuman. Nec filio jubet (the Mosaic Law)
parci, nec fratri, et per amatam conjugera gladium vindicem
ducit, \&c.}

\pagenote[2]{Bayle (tom. ii. p. 406, in his Commentaire
Philosophique) justifies, and limits, these intolerant laws by
the temporal reign of Jehovah over the Jews. The attempt is
laudable.}

From the age of Numa to the reign of Gratian, the Romans
preserved the regular succession of the several colleges of the
sacerdotal order.\textsuperscript{3} Fifteen Pontiffs exercised their supreme
jurisdiction over all things, and persons, that were consecrated
to the service of the gods; and the various questions which
perpetually arose in a loose and traditionary system, were
submitted to the judgment of their holy tribunal. Fifteen grave
and learned Augurs observed the face of the heavens, and
prescribed the actions of heroes, according to the flight of
birds. Fifteen keepers of the Sibylline books (their name of
Quindecemvirs was derived from their number) occasionally
consulted the history of future, and, as it should seem, of
contingent, events. Six Vestals devoted their virginity to the
guard of the sacred fire, and of the unknown pledges of the
duration of Rome; which no mortal had been suffered to behold
with impunity.\textsuperscript{4} Seven Epulos prepared the table of the gods,
conducted the solemn procession, and regulated the ceremonies of
the annual festival. The three Flamens of Jupiter, of Mars, and
of Quirinus, were considered as the peculiar ministers of the
three most powerful deities, who watched over the fate of Rome
and of the universe. The King of the Sacrifices represented the
person of Numa, and of his successors, in the religious
functions, which could be performed only by royal hands. The
confraternities of the Salians, the Lupercals, \&c., practised
such rites as might extort a smile of contempt from every
reasonable man, with a lively confidence of recommending
themselves to the favor of the immortal gods. The authority,
which the Roman priests had formerly obtained in the counsels of
the republic, was gradually abolished by the establishment of
monarchy, and the removal of the seat of empire. But the dignity
of their sacred character was still protected by the laws, and
manners of their country; and they still continued, more
especially the college of pontiffs, to exercise in the capital,
and sometimes in the provinces, the rights of their
ecclesiastical and civil jurisdiction. Their robes of purple,
chariotz of state, and sumptuous entertainments, attracted the
admiration of the people; and they received, from the consecrated
lands, and the public revenue, an ample stipend, which liberally
supported the splendor of the priesthood, and all the expenses of
the religious worship of the state. As the service of the altar
was not incompatible with the command of armies, the Romans,
after their consulships and triumphs, aspired to the place of
pontiff, or of augur; the seats of Cicero\textsuperscript{5} and Pompey were
filled, in the fourth century, by the most illustrious members of
the senate; and the dignity of their birth reflected additional
splendor on their sacerdotal character. The fifteen priests, who
composed the college of pontiffs, enjoyed a more distinguished
rank as the companions of their sovereign; and the Christian
emperors condescended to accept the robe and ensigns, which were
appropriated to the office of supreme pontiff. But when Gratian
ascended the throne, more scrupulous or more enlightened, he
sternly rejected those profane symbols;\textsuperscript{6} applied to the service
of the state, or of the church, the revenues of the priests and
vestals; abolished their honors and immunities; and dissolved the
ancient fabric of Roman superstition, which was supported by the
opinions and habits of eleven hundred years. Paganism was still
the constitutional religion of the senate. The hall, or temple,
in which they assembled, was adorned by the statue and altar of
Victory;\textsuperscript{7} a majestic female standing on a globe, with flowing
garments, expanded wings, and a crown of laurel in her
outstretched hand.\textsuperscript{8} The senators were sworn on the altar of the
goddess to observe the laws of the emperor and of the empire: and
a solemn offering of wine and incense was the ordinary prelude of
their public deliberations.\textsuperscript{9} The removal of this ancient
monument was the only injury which Constantius had offered to the
superstition of the Romans. The altar of Victory was again
restored by Julian, tolerated by Valentinian, and once more
banished from the senate by the zeal of Gratian.\textsuperscript{10} But the
emperor yet spared the statues of the gods which were exposed to
the public veneration: four hundred and twenty-four temples, or
chapels, still remained to satisfy the devotion of the people;
and in every quarter of Rome the delicacy of the Christians was
offended by the fumes of idolatrous sacrifice.\textsuperscript{11}

\pagenote[3]{See the outlines of the Roman hierarchy in Cicero,
(de Legibus, ii. 7, 8,) Livy, (i. 20,) Dionysius
Halicarnassensis, (l. ii. p. 119-129, edit. Hudson,) Beaufort,
(Republique Romaine, tom. i. p. 1-90,) and Moyle, (vol. i. p.
10-55.) The last is the work of an English whig, as well as of a
Roman antiquary.}

\pagenote[4]{These mystic, and perhaps imaginary, symbols have
given birth to various fables and conjectures. It seems probable,
that the Palladium was a small statue (three cubits and a half
high) of Minerva, with a lance and distaff; that it was usually
enclosed in a seria, or barrel; and that a similar barrel was
placed by its side to disconcert curiosity, or sacrilege. See
Mezeriac (Comment. sur les Epitres d’Ovide, tom i. p. 60—66) and
Lipsius, (tom. iii. p. 610 de Vesta, \&c. c 10.)}

\pagenote[5]{Cicero frankly (ad Atticum, l. ii. Epist. 5) or
indirectly (ad Familiar. l. xv. Epist. 4) confesses that the
Augurate is the supreme object of his wishes. Pliny is proud to
tread in the footsteps of Cicero, (l. iv. Epist. 8,) and the
chain of tradition might be continued from history and marbles.}

\pagenote[6]{Zosimus, l. iv. p. 249, 250. I have suppressed the
foolish pun about Pontifex and Maximus.}

\pagenote[7]{This statue was transported from Tarentum to Rome,
placed in the Curia Julia by Caesar, and decorated by Augustus
with the spoils of Egypt.}

\pagenote[8]{Prudentius (l. ii. in initio) has drawn a very
awkward portrait of Victory; but the curious reader will obtain
more satisfaction from Montfaucon’s Antiquities, (tom. i. p.
341.)}

\pagenote[9]{See Suetonius (in August. c. 35) and the Exordium of
Pliny’s Panegyric.}

\pagenote[10]{These facts are mutually allowed by the two
advocates, Symmachus and Ambrose.}

\pagenote[11]{The Notitia Urbis, more recent than Constantine,
does not find one Christian church worthy to be named among the
edifices of the city. Ambrose (tom. ii. Epist. xvii. p. 825)
deplores the public scandals of Rome, which continually offended
the eyes, the ears, and the nostrils of the faithful.}

But the Christians formed the least numerous party in the senate
of Rome:\textsuperscript{12} and it was only by their absence, that they could
express their dissent from the legal, though profane, acts of a
Pagan majority. In that assembly, the dying embers of freedom
were, for a moment, revived and inflamed by the breath of
fanaticism. Four respectable deputations were successively voted
to the Imperial court,\textsuperscript{13} to represent the grievances of the
priesthood and the senate, and to solicit the restoration of the
altar of Victory. The conduct of this important business was
intrusted to the eloquent Symmachus,\textsuperscript{14} a wealthy and noble
senator, who united the sacred characters of pontiff and augur
with the civil dignities of proconsul of Africa and præfect of
the city. The breast of Symmachus was animated by the warmest
zeal for the cause of expiring Paganism; and his religious
antagonists lamented the abuse of his genius, and the inefficacy
of his moral virtues.\textsuperscript{15} The orator, whose petition is extant to
the emperor Valentinian, was conscious of the difficulty and
danger of the office which he had assumed. He cautiously avoids
every topic which might appear to reflect on the religion of his
sovereign; humbly declares, that prayers and entreaties are his
only arms; and artfully draws his arguments from the schools of
rhetoric, rather than from those of philosophy. Symmachus
endeavors to seduce the imagination of a young prince, by
displaying the attributes of the goddess of victory; he
insinuates, that the confiscation of the revenues, which were
consecrated to the service of the gods, was a measure unworthy of
his liberal and disinterested character; and he maintains, that
the Roman sacrifices would be deprived of their force and energy,
if they were no longer celebrated at the expense, as well as in
the name, of the republic. Even scepticism is made to supply an
apology for superstition. The great and incomprehensible secret
of the universe eludes the inquiry of man. Where reason cannot
instruct, custom may be permitted to guide; and every nation
seems to consult the dictates of prudence, by a faithful
attachment to those rites and opinions, which have received the
sanction of ages. If those ages have been crowned with glory and
prosperity, if the devout people have frequently obtained the
blessings which they have solicited at the altars of the gods, it
must appear still more advisable to persist in the same salutary
practice; and not to risk the unknown perils that may attend any
rash innovations. The test of antiquity and success was applied
with singular advantage to the religion of Numa; and Rome
herself, the celestial genius that presided over the fates of the
city, is introduced by the orator to plead her own cause before
the tribunal of the emperors. “Most excellent princes,” says the
venerable matron, “fathers of your country! pity and respect my
age, which has hitherto flowed in an uninterrupted course of
piety. Since I do not repent, permit me to continue in the
practice of my ancient rites. Since I am born free, allow me to
enjoy my domestic institutions. This religion has reduced the
world under my laws. These rites have repelled Hannibal from the
city, and the Gauls from the Capitol. Were my gray hairs reserved
for such intolerable disgrace? I am ignorant of the new system
that I am required to adopt; but I am well assured, that the
correction of old age is always an ungrateful and ignominious
office.”\textsuperscript{16} The fears of the people supplied what the discretion
of the orator had suppressed; and the calamities, which
afflicted, or threatened, the declining empire, were unanimously
imputed, by the Pagans, to the new religion of Christ and of
Constantine.

\pagenote[12]{Ambrose repeatedly affirms, in contradiction to
common sense (Moyle’s Works, vol. ii. p. 147,) that the
Christians had a majority in the senate.}

\pagenote[13]{The first (A.D. 382) to Gratian, who refused them
audience; the second (A.D. 384) to Valentinian, when the field
was disputed by Symmachus and Ambrose; the third (A.D. 388) to
Theodosius; and the fourth (A.D. 392) to Valentinian. Lardner
(Heathen Testimonies, vol. iv. p. 372-399) fairly represents the
whole transaction.}

\pagenote[14]{Symmachus, who was invested with all the civil and
sacerdotal honors, represented the emperor under the two
characters of Pontifex Maximus, and Princeps Senatus. See the
proud inscription at the head of his works. * Note: Mr. Beugnot
has made it doubtful whether Symmachus was more than Pontifex
Major. Destruction du Paganisme, vol. i. p. 459.—M.}

\pagenote[15]{As if any one, says Prudentius (in Symmach. i. 639)
should dig in the mud with an instrument of gold and ivory. Even
saints, and polemic saints, treat this adversary with respect and
civility.}

\pagenote[16]{See the fifty-fourth Epistle of the tenth book of
Symmachus. In the form and disposition of his ten books of
Epistles, he imitated the younger Pliny; whose rich and florid
style he was supposed, by his friends, to equal or excel,
(Macrob. Saturnal. l. v. c. i.) But the luxcriancy of Symmachus
consists of barren leaves, without fruits, and even without
flowers. Few facts, and few sentiments, can be extracted from his
verbose correspondence.}

But the hopes of Symmachus were repeatedly baffled by the firm
and dexterous opposition of the archbishop of Milan, who
fortified the emperors against the fallacious eloquence of the
advocate of Rome. In this controversy, Ambrose condescends to
speak the language of a philosopher, and to ask, with some
contempt, why it should be thought necessary to introduce an
imaginary and invisible power, as the cause of those victories,
which were sufficiently explained by the valor and discipline of
the legions. He justly derides the absurd reverence for
antiquity, which could only tend to discourage the improvements
of art, and to replunge the human race into their original
barbarism. From thence, gradually rising to a more lofty and
theological tone, he pronounces, that Christianity alone is the
doctrine of truth and salvation; and that every mode of
Polytheism conducts its deluded votaries, through the paths of
error, to the abyss of eternal perdition.\textsuperscript{17} Arguments like
these, when they were suggested by a favorite bishop, had power
to prevent the restoration of the altar of Victory; but the same
arguments fell, with much more energy and effect, from the mouth
of a conqueror; and the gods of antiquity were dragged in triumph
at the chariot-wheels of Theodosius.\textsuperscript{18} In a full meeting of the
senate, the emperor proposed, according to the forms of the
republic, the important question, Whether the worship of Jupiter,
or that of Christ, should be the religion of the Romans.\textsuperscript{1811} The
liberty of suffrages, which he affected to allow, was destroyed
by the hopes and fears that his presence inspired; and the
arbitrary exile of Symmachus was a recent admonition, that it
might be dangerous to oppose the wishes of the monarch. On a
regular division of the senate, Jupiter was condemned and
degraded by the sense of a very large majority; and it is rather
surprising, that any members should be found bold enough to
declare, by their speeches and votes, that they were still
attached to the interest of an abdicated deity.\textsuperscript{19} The hasty
conversion of the senate must be attributed either to
supernatural or to sordid motives; and many of these reluctant
proselytes betrayed, on every favorable occasion, their secret
disposition to throw aside the mask of odious dissimulation. But
they were gradually fixed in the new religion, as the cause of
the ancient became more hopeless; they yielded to the authority
of the emperor, to the fashion of the times, and to the
entreaties of their wives and children,\textsuperscript{20} who were instigated
and governed by the clergy of Rome and the monks of the East. The
edifying example of the Anician family was soon imitated by the
rest of the nobility: the Bassi, the Paullini, the Gracchi,
embraced the Christian religion; and “the luminaries of the
world, the venerable assembly of Catos (such are the high-flown
expressions of Prudentius) were impatient to strip themselves of
their pontifical garment; to cast the skin of the old serpent; to
assume the snowy robes of baptismal innocence, and to humble the
pride of the consular fasces before tombs of the martyrs.”\textsuperscript{21} The
citizens, who subsisted by their own industry, and the populace,
who were supported by the public liberality, filled the churches
of the Lateran, and Vatican, with an incessant throng of devout
proselytes. The decrees of the senate, which proscribed the
worship of idols, were ratified by the general consent of the
Romans;\textsuperscript{22} the splendor of the Capitol was defaced, and the
solitary temples were abandoned to ruin and contempt.\textsuperscript{23} Rome
submitted to the yoke of the Gospel; and the vanquished provinces
had not yet lost their reverence for the name and authority of
Rome.\textsuperscript{2311}

\pagenote[17]{See Ambrose, (tom. ii. Epist. xvii. xviii. p.
825-833.) The former of these epistles is a short caution; the
latter is a formal reply of the petition or libel of Symmachus.
The same ideas are more copiously expressed in the poetry, if it
may deserve that name, of Prudentius; who composed his two books
against Symmachus (A.D. 404) while that senator was still alive.
It is whimsical enough that Montesquieu (Considerations, \&c. c.
xix. tom. iii. p. 487) should overlook the two professed
antagonists of Symmachus, and amuse himself with descanting on
the more remote and indirect confutations of Orosius, St.
Augustin, and Salvian.}

\pagenote[18]{See Prudentius (in Symmach. l. i. 545, \&c.) The
Christian agrees with the Pagan Zosimus (l. iv. p. 283) in
placing this visit of Theodosius after the second civil war,
gemini bis victor caede Tyranni, (l. i. 410.) But the time and
circumstances are better suited to his first triumph.}

\pagenote[1811]{M. Beugnot (in his Histoire de la Destruction du
Paganisme en Occident, i. p. 483-488) questions, altogether, the
truth of this statement. It is very remarkable that Zosimus and
Prudentius concur in asserting the fact of the question being
solemnly deliberated by the senate, though with directly opposite
results. Zosimus declares that the majority of the assembly
adhered to the ancient religion of Rome; Gibbon has adopted the
authority of Prudentius, who, as a Latin writer, though a poet,
deserves more credit than the Greek historian. Both concur in
placing this scene after the second triumph of Theodosius; but it
has been almost demonstrated (and Gibbon—see the preceding
note—seems to have acknowledged this) by Pagi and Tillemont, that
Theodosius did not visit Rome after the defeat of Eugenius. M.
Beugnot urges, with much force, the improbability that the
Christian emperor would submit such a question to the senate,
whose authority was nearly obsolete, except on one occasion,
which was almost hailed as an epoch in the restoration of her
ancient privileges. The silence of Ambrose and of Jerom on an
event so striking, and redounding so much to the honor of
Christianity, is of considerable weight. M. Beugnot would ascribe
the whole scene to the poetic imagination of Prudentius; but I
must observe, that, however Prudentius is sometimes elevated by
the grandeur of his subject to vivid and eloquent language, this
flight of invention would be so much bolder and more vigorous
than usual with this poet, that I cannot but suppose there must
have been some foundation for the story, though it may have been
exaggerated by the poet, or misrepresented by the historian.—M}

\pagenote[19]{Prudentius, after proving that the sense of the
senate is declared by a legal majority, proceeds to say, (609,
\&c.)—

Adspice quam pleno subsellia nostra Senatu Decernant infame Jovis
pulvinar, et omne Idolum longe purgata ex urbe fugandum, Qua vocat
egregii sententia Principis, illuc Libera, cum pedibus, tum corde,
frequentia transit.

Zosimus ascribes to the conscript fathers a heathenish courage,
which few of them are found to possess.}

\pagenote[20]{Jerom specifies the pontiff Albinus, who was
surrounded with such a believing family of children and
grandchildren, as would have been sufficient to convert even
Jupiter himself; an extraordinary proselyted (tom. i. ad Laetam,
p. 54.)}

\pagenote[21]{

Exultare Patres videas, pulcherrima mundi Lumina; Conciliumque
senum gestire Catonum Candidiore toga niveum pietatis amictum
Sumere; et exuvias deponere pontificales.

The fancy of Prudentius is warmed and elevated by victory}

\pagenote[22]{Prudentius, after he has described the conversion
of the senate and people, asks, with some truth and confidence,

Et dubitamus adhuc Romam, tibi, Christe, dicatam In leges transisse
tuas?}

\pagenote[23]{Jerom exults in the desolation of the Capitol, and
the other temples of Rome, (tom. i. p. 54, tom. ii. p. 95.)}

\pagenote[2311]{M. Beugnot is more correct in his general
estimate of the measures enforced by Theodosius for the abolition
of Paganism. He seized (according to Zosimus) the funds bestowed
by the public for the expense of sacrifices. The public
sacrifices ceased, not because they were positively prohibited,
but because the public treasury would no longer bear the expense.
The public and the private sacrifices in the provinces, which
were not under the same regulations with those of the capital,
continued to take place. In Rome itself, many pagan ceremonies,
which were without sacrifice, remained in full force. The gods,
therefore, were invoked, the temples were frequented, the
pontificates inscribed, according to ancient usage, among the
family titles of honor; and it cannot be asserted that idolatry
was completely destroyed by Theodosius. See Beugnot, p. 491.—M.}

\section{Part \thesection.}

The filial piety of the emperors themselves engaged them to
proceed, with some caution and tenderness, in the reformation of
the eternal city. Those absolute monarchs acted with less regard
to the prejudices of the provincials. The pious labor which had
been suspended near twenty years since the death of Constantius,\textsuperscript{24}
was vigorously resumed, and finally accomplished, by the zeal
of Theodosius. Whilst that warlike prince yet struggled with the
Goths, not for the glory, but for the safety, of the republic, he
ventured to offend a considerable party of his subjects, by some
acts which might perhaps secure the protection of Heaven, but
which must seem rash and unseasonable in the eye of human
prudence. The success of his first experiments against the Pagans
encouraged the pious emperor to reiterate and enforce his edicts
of proscription: the same laws which had been originally
published in the provinces of the East, were applied, after the
defeat of Maximus, to the whole extent of the Western empire; and
every victory of the orthodox Theodosius contributed to the
triumph of the Christian and Catholic faith.\textsuperscript{25} He attacked
superstition in her most vital part, by prohibiting the use of
sacrifices, which he declared to be criminal as well as infamous;
and if the terms of his edicts more strictly condemned the
impious curiosity which examined the entrails of the victim,\textsuperscript{26}
every subsequent explanation tended to involve in the same guilt
the general practice of immolation, which essentially constituted
the religion of the Pagans. As the temples had been erected for
the purpose of sacrifice, it was the duty of a benevolent prince
to remove from his subjects the dangerous temptation of offending
against the laws which he had enacted. A special commission was
granted to Cynegius, the Prætorian præfect of the East, and
afterwards to the counts Jovius and Gaudentius, two officers of
distinguished rank in the West; by which they were directed to
shut the temples, to seize or destroy the instruments of
idolatry, to abolish the privileges of the priests, and to
confiscate the consecrated property for the benefit of the
emperor, of the church, or of the army.\textsuperscript{27} Here the desolation
might have stopped: and the naked edifices, which were no longer
employed in the service of idolatry, might have been protected
from the destructive rage of fanaticism. Many of those temples
were the most splendid and beautiful monuments of Grecian
architecture; and the emperor himself was interested not to
deface the splendor of his own cities, or to diminish the value
of his own possessions. Those stately edifices might be suffered
to remain, as so many lasting trophies of the victory of Christ.
In the decline of the arts they might be usefully converted into
magazines, manufactures, or places of public assembly: and
perhaps, when the walls of the temple had been sufficiently
purified by holy rites, the worship of the true Deity might be
allowed to expiate the ancient guilt of idolatry. But as long as
they subsisted, the Pagans fondly cherished the secret hope, that
an auspicious revolution, a second Julian, might again restore
the altars of the gods: and the earnestness with which they
addressed their unavailing prayers to the throne,\textsuperscript{28} increased
the zeal of the Christian reformers to extirpate, without mercy,
the root of superstition. The laws of the emperors exhibit some
symptoms of a milder disposition:\textsuperscript{29} but their cold and languid
efforts were insufficient to stem the torrent of enthusiasm and
rapine, which was conducted, or rather impelled, by the spiritual
rulers of the church. In Gaul, the holy Martin, bishop of Tours,\textsuperscript{30}
marched at the head of his faithful monks to destroy the
idols, the temples, and the consecrated trees of his extensive
diocese; and, in the execution of this arduous task, the prudent
reader will judge whether Martin was supported by the aid of
miraculous powers, or of carnal weapons. In Syria, the divine and
excellent Marcellus,\textsuperscript{31} as he is styled by Theodoret, a bishop
animated with apostolic fervor, resolved to level with the ground
the stately temples within the diocese of Apamea. His attack was
resisted by the skill and solidity with which the temple of
Jupiter had been constructed. The building was seated on an
eminence: on each of the four sides, the lofty roof was supported
by fifteen massy columns, sixteen feet in circumference; and the
large stone, of which they were composed, were firmly cemented
with lead and iron. The force of the strongest and sharpest tools
had been tried without effect. It was found necessary to
undermine the foundations of the columns, which fell down as soon
as the temporary wooden props had been consumed with fire; and
the difficulties of the enterprise are described under the
allegory of a black daemon, who retarded, though he could not
defeat, the operations of the Christian engineers. Elated with
victory, Marcellus took the field in person against the powers of
darkness; a numerous troop of soldiers and gladiators marched
under the episcopal banner, and he successively attacked the
villages and country temples of the diocese of Apamea. Whenever
any resistance or danger was apprehended, the champion of the
faith, whose lameness would not allow him either to fight or fly,
placed himself at a convenient distance, beyond the reach of
darts. But this prudence was the occasion of his death: he was
surprised and slain by a body of exasperated rustics; and the
synod of the province pronounced, without hesitation, that the
holy Marcellus had sacrificed his life in the cause of God. In
the support of this cause, the monks, who rushed with tumultuous
fury from the desert, distinguished themselves by their zeal and
diligence. They deserved the enmity of the Pagans; and some of
them might deserve the reproaches of avarice and intemperance; of
avarice, which they gratified with holy plunder, and of
intemperance, which they indulged at the expense of the people,
who foolishly admired their tattered garments, loud psalmody, and
artificial paleness.\textsuperscript{32} A small number of temples was protected
by the fears, the venality, the taste, or the prudence, of the
civil and ecclesiastical governors. The temple of the Celestial
Venus at Carthage, whose sacred precincts formed a circumference
of two miles, was judiciously converted into a Christian church;\textsuperscript{33}
and a similar consecration has preserved inviolate the
majestic dome of the Pantheon at Rome.\textsuperscript{34} But in almost every
province of the Roman world, an army of fanatics, without
authority, and without discipline, invaded the peaceful
inhabitants; and the ruin of the fairest structures of antiquity
still displays the ravages of those Barbarians, who alone had
time and inclination to execute such laborious destruction.

\pagenote[24]{Libanius (Orat. pro Templis, p. 10, Genev. 1634,
published by James Godefroy, and now extremely scarce) accuses
Valentinian and Valens of prohibiting sacrifices. Some partial
order may have been issued by the Eastern emperor; but the idea
of any general law is contradicted by the silence of the Code,
and the evidence of ecclesiastical history. Note: See in Reiske’s
edition of Libanius, tom. ii. p. 155. Sacrific was prohibited by
Valens, but not the offering of incense.—M.}

\pagenote[25]{See his laws in the Theodosian Code, l. xvi. tit.
x. leg. 7-11.}

\pagenote[26]{Homer’s sacrifices are not accompanied with any
inquisition of entrails, (see Feithius, Antiquitat. Homer. l. i.
c. 10, 16.) The Tuscans, who produced the first Haruspices,
subdued both the Greeks and the Romans, (Cicero de Divinatione,
ii. 23.)}

\pagenote[27]{Zosimus, l. iv. p. 245, 249. Theodoret. l. v. c.
21. Idatius in Chron. Prosper. Aquitan. l. iii. c. 38, apud
Baronium, Annal. Eccles. A.D. 389, No. 52. Libanius (pro Templis,
p. 10) labors to prove that the commands of Theodosius were not
direct and positive. * Note: Libanius appears to be the best
authority for the East, where, under Theodosius, the work of
devastation was carried on with very different degrees of
violence, according to the temper of the local authorities and of
the clergy; and more especially the neighborhood of the more
fanatican monks. Neander well observes, that the prohibition of
sacrifice would be easily misinterpreted into an authority for
the destruction of the buildings in which sacrifices were
performed. (Geschichte der Christlichen religion ii. p. 156.) An
abuse of this kind led to this remarkable oration of Libanius.
Neander, however, justly doubts whether this bold vindication or
at least exculpation, of Paganism was ever delivered before, or
even placed in the hands of the Christian emperor.—M.}

\pagenote[28]{Cod. Theodos, l. xvi. tit. x. leg. 8, 18. There is
room to believe, that this temple of Edessa, which Theodosius
wished to save for civil uses, was soon afterwards a heap of
ruins, (Libanius pro Templis, p. 26, 27, and Godefroy’s notes, p.
59.)}

\pagenote[29]{See this curious oration of Libanius pro Templis,
pronounced, or rather composed, about the year 390. I have
consulted, with advantage, Dr. Lardner’s version and remarks,
(Heathen Testimonies, vol. iv. p. 135-163.)}

\pagenote[30]{See the Life of Martin by Sulpicius Severus, c.
9-14. The saint once mistook (as Don Quixote might have done) a
harmless funeral for an idolatrous procession, and imprudently
committed a miracle.}

\pagenote[31]{Compare Sozomen, (l. vii. c. 15) with Theodoret,
(l. v. c. 21.) Between them, they relate the crusade and death of
Marcellus.}

\pagenote[32]{Libanius, pro Templis, p. 10-13. He rails at these
black-garbed men, the Christian monks, who eat more than
elephants. Poor elephants! they are temperate animals.}

\pagenote[33]{Prosper. Aquitan. l. iii. c. 38, apud Baronium;
Annal. Eccles. A.D. 389, No. 58, \&c. The temple had been shut
some time, and the access to it was overgrown with brambles.}

\pagenote[34]{Donatus, Roma Antiqua et Nova, l. iv. c. 4, p. 468.
This consecration was performed by Pope Boniface IV. I am
ignorant of the favorable circumstances which had preserved the
Pantheon above two hundred years after the reign of Theodosius.}

In this wide and various prospect of devastation, the spectator
may distinguish the ruins of the temple of Serapis, at
Alexandria.\textsuperscript{35} Serapis does not appear to have been one of the
native gods, or monsters, who sprung from the fruitful soil of
superstitious Egypt.\textsuperscript{36} The first of the Ptolemies had been
commanded, by a dream, to import the mysterious stranger from the
coast of Pontus, where he had been long adored by the inhabitants
of Sinope; but his attributes and his reign were so imperfectly
understood, that it became a subject of dispute, whether he
represented the bright orb of day, or the gloomy monarch of the
subterraneous regions.\textsuperscript{37} The Egyptians, who were obstinately
devoted to the religion of their fathers, refused to admit this
foreign deity within the walls of their cities.\textsuperscript{38} But the
obsequious priests, who were seduced by the liberality of the
Ptolemies, submitted, without resistance, to the power of the god
of Pontus: an honorable and domestic genealogy was provided; and
this fortunate usurper was introduced into the throne and bed of
Osiris,\textsuperscript{39} the husband of Isis, and the celestial monarch of
Egypt. Alexandria, which claimed his peculiar protection, gloried
in the name of the city of Serapis. His temple,\textsuperscript{40} which rivalled
the pride and magnificence of the Capitol, was erected on the
spacious summit of an artificial mount, raised one hundred steps
above the level of the adjacent parts of the city; and the
interior cavity was strongly supported by arches, and distributed
into vaults and subterraneous apartments. The consecrated
buildings were surrounded by a quadrangular portico; the stately
halls, and exquisite statues, displayed the triumph of the arts;
and the treasures of ancient learning were preserved in the
famous Alexandrian library, which had arisen with new splendor
from its ashes.\textsuperscript{41} After the edicts of Theodosius had severely
prohibited the sacrifices of the Pagans, they were still
tolerated in the city and temple of Serapis; and this singular
indulgence was imprudently ascribed to the superstitious terrors
of the Christians themselves; as if they had feared to abolish
those ancient rites, which could alone secure the inundations of
the Nile, the harvests of Egypt, and the subsistence of
Constantinople.\textsuperscript{42}

\pagenote[35]{Sophronius composed a recent and separate history,
(Jerom, in Script. Eccles. tom. i. p. 303,) which has furnished
materials to Socrates, (l. v. c. 16.) Theodoret, (l. v. c. 22,)
and Rufinus, (l. ii. c. 22.) Yet the last, who had been at
Alexandria before and after the event, may deserve the credit of
an original witness.}

\pagenote[36]{Gerard Vossius (Opera, tom. v. p. 80, and de
Idoloaltria, l. i. c. 29) strives to support the strange notion
of the Fathers; that the patriarch Joseph was adored in Egypt, as
the bull Apis, and the god Serapis. * Note: Consult du Dieu
Serapis et son Origine, par J D. Guigniaut, (the translator of
Creuzer’s Symbolique,) Paris, 1828; and in the fifth volume of
Bournouf’s translation of Tacitus.—M.}

\pagenote[37]{Origo dei nondum nostris celebrata. Aegyptiorum
antistites sic memorant, \&c., Tacit. Hist. iv. 83. The Greeks,
who had travelled into Egypt, were alike ignorant of this new
deity.}

\pagenote[38]{Macrobius, Saturnal, l. i. c. 7. Such a living fact
decisively proves his foreign extraction.}

\pagenote[39]{At Rome, Isis and Serapis were united in the same
temple. The precedency which the queen assumed, may seem to
betray her unequal alliance with the stranger of Pontus. But the
superiority of the female sex was established in Egypt as a civil
and religious institution, (Diodor. Sicul. tom. i. l. i. p. 31,
edit. Wesseling,) and the same order is observed in Plutarch’s
Treatise of Isis and Osiris; whom he identifies with Serapis.}

\pagenote[40]{Ammianus, (xxii. 16.) The Expositio totius Mundi,
(p. 8, in Hudson’s Geograph. Minor. tom. iii.,) and Rufinus, (l.
ii. c. 22,) celebrate the Serapeum, as one of the wonders of the
world.}

\pagenote[41]{See Mémoires de l’Acad. des Inscriptions, tom. ix.
p. 397-416. The old library of the Ptolemies was totally consumed
in Caesar’s Alexandrian war. Marc Antony gave the whole
collection of Pergamus (200,000 volumes) to Cleopatra, as the
foundation of the new library of Alexandria.}

\pagenote[42]{Libanius (pro Templis, p. 21) indiscreetly provokes
his Christian masters by this insulting remark.}

At that time\textsuperscript{43} the archiepiscopal throne of Alexandria was
filled by Theophilus,\textsuperscript{44} the perpetual enemy of peace and virtue;
a bold, bad man, whose hands were alternately polluted with gold
and with blood. His pious indignation was excited by the honors
of Serapis; and the insults which he offered to an ancient temple
of Bacchus,\textsuperscript{4411} convinced the Pagans that he meditated a more
important and dangerous enterprise. In the tumultuous capital of
Egypt, the slightest provocation was sufficient to inflame a
civil war. The votaries of Serapis, whose strength and numbers
were much inferior to those of their antagonists, rose in arms at
the instigation of the philosopher Olympius,\textsuperscript{45} who exhorted them
to die in the defence of the altars of the gods. These Pagan
fanatics fortified themselves in the temple, or rather fortress,
of Serapis; repelled the besiegers by daring sallies, and a
resolute defence; and, by the inhuman cruelties which they
exercised on their Christian prisoners, obtained the last
consolation of despair. The efforts of the prudent magistrate
were usefully exerted for the establishment of a truce, till the
answer of Theodosius should determine the fate of Serapis. The
two parties assembled, without arms, in the principal square; and
the Imperial rescript was publicly read. But when a sentence of
destruction against the idols of Alexandria was pronounced, the
Christians set up a shout of joy and exultation, whilst the
unfortunate Pagans, whose fury had given way to consternation,
retired with hasty and silent steps, and eluded, by their flight
or obscurity, the resentment of their enemies. Theophilus
proceeded to demolish the temple of Serapis, without any other
difficulties, than those which he found in the weight and
solidity of the materials: but these obstacles proved so
insuperable, that he was obliged to leave the foundations; and to
content himself with reducing the edifice itself to a heap of
rubbish, a part of which was soon afterwards cleared away, to
make room for a church, erected in honor of the Christian
martyrs. The valuable library of Alexandria was pillaged or
destroyed; and near twenty years afterwards, the appearance of
the empty shelves excited the regret and indignation of every
spectator, whose mind was not totally darkened by religious
prejudice.\textsuperscript{46} The compositions of ancient genius, so many of
which have irretrievably perished, might surely have been
excepted from the wreck of idolatry, for the amusement and
instruction of succeeding ages; and either the zeal or the
avarice of the archbishop,\textsuperscript{47} might have been satiated with the
rich spoils, which were the reward of his victory. While the
images and vases of gold and silver were carefully melted, and
those of a less valuable metal were contemptuously broken, and
cast into the streets, Theophilus labored to expose the frauds
and vices of the ministers of the idols; their dexterity in the
management of the loadstone; their secret methods of introducing
a human actor into a hollow statue;\textsuperscript{4711} and their scandalous
abuse of the confidence of devout husbands and unsuspecting
females.\textsuperscript{48} Charges like these may seem to deserve some degree of
credit, as they are not repugnant to the crafty and interested
spirit of superstition. But the same spirit is equally prone to
the base practice of insulting and calumniating a fallen enemy;
and our belief is naturally checked by the reflection, that it is
much less difficult to invent a fictitious story, than to support
a practical fraud. The colossal statue of Serapis\textsuperscript{49} was involved
in the ruin of his temple and religion. A great number of plates
of different metals, artificially joined together, composed the
majestic figure of the deity, who touched on either side the
walls of the sanctuary. The aspect of Serapis, his sitting
posture, and the sceptre, which he bore in his left hand, were
extremely similar to the ordinary representations of Jupiter. He
was distinguished from Jupiter by the basket, or bushel, which
was placed on his head; and by the emblematic monster which he
held in his right hand; the head and body of a serpent branching
into three tails, which were again terminated by the triple heads
of a dog, a lion, and a wolf. It was confidently affirmed, that
if any impious hand should dare to violate the majesty of the
god, the heavens and the earth would instantly return to their
original chaos. An intrepid soldier, animated by zeal, and armed
with a weighty battle-axe, ascended the ladder; and even the
Christian multitude expected, with some anxiety, the event of the
combat.\textsuperscript{50} He aimed a vigorous stroke against the cheek of
Serapis; the cheek fell to the ground; the thunder was still
silent, and both the heavens and the earth continued to preserve
their accustomed order and tranquillity. The victorious soldier
repeated his blows: the huge idol was overthrown, and broken in
pieces; and the limbs of Serapis were ignominiously dragged
through the streets of Alexandria. His mangled carcass was burnt
in the Amphitheatre, amidst the shouts of the populace; and many
persons attributed their conversion to this discovery of the
impotence of their tutelar deity. The popular modes of religion,
that propose any visible and material objects of worship, have
the advantage of adapting and familiarizing themselves to the
senses of mankind: but this advantage is counterbalanced by the
various and inevitable accidents to which the faith of the
idolater is exposed. It is scarcely possible, that, in every
disposition of mind, he should preserve his implicit reverence
for the idols, or the relics, which the naked eye, and the
profane hand, are unable to distinguish from the most common
productions of art or nature; and if, in the hour of danger,
their secret and miraculous virtue does not operate for their own
preservation, he scorns the vain apologies of his priests, and
justly derides the object, and the folly, of his superstitious
attachment.\textsuperscript{51} After the fall of Serapis, some hopes were still
entertained by the Pagans, that the Nile would refuse his annual
supply to the impious masters of Egypt; and the extraordinary
delay of the inundation seemed to announce the displeasure of the
river-god. But this delay was soon compensated by the rapid swell
of the waters. They suddenly rose to such an unusual height, as
to comfort the discontented party with the pleasing expectation
of a deluge; till the peaceful river again subsided to the
well-known and fertilizing level of sixteen cubits, or about
thirty English feet.\textsuperscript{52}

\pagenote[43]{We may choose between the date of Marcellinus (A.D.
389) or that of Prosper, ( A.D. 391.) Tillemont (Hist. des Emp.
tom. v. p. 310, 756) prefers the former, and Pagi the latter.}

\pagenote[44]{Tillemont, Mem. Eccles. tom. xi. p. 441-500. The
ambiguous situation of Theophilus—a saint, as the friend of Jerom
a devil, as the enemy of Chrysostom—produces a sort of
impartiality; yet, upon the whole, the balance is justly inclined
against him.}

\pagenote[4411]{No doubt a temple of Osiris. St. Martin, iv
398-M.}

\pagenote[45]{Lardner (Heathen Testimonies, vol. iv. p. 411) has
alleged beautiful passage from Suidas, or rather from Damascius,
which show the devout and virtuous Olympius, not in the light of
a warrior, but of a prophet.}

\pagenote[46]{Nos vidimus armaria librorum, quibus direptis,
exinanita ea a nostris hominibus, nostris temporibus memorant.
Orosius, l. vi. c. 15, p. 421, edit. Havercamp. Though a bigot,
and a controversial writer. Orosius seems to blush.}

\pagenote[47]{Eunapius, in the Lives of Antoninus and Aedesius,
execrates the sacrilegious rapine of Theophilus. Tillemont (Mem.
Eccles. tom. xiii. p. 453) quotes an epistle of Isidore of
Pelusium, which reproaches the primate with the idolatrous
worship of gold, the auri sacra fames.}

\pagenote[4711]{An English traveller, Mr. Wilkinson, has
discovered the secret of the vocal Memnon. There was a cavity in
which a person was concealed, and struck a stone, which gave a
ringing sound like brass. The Arabs, who stood below when Mr.
Wilkinson performed the miracle, described sound just as the
author of the epigram.—M.}

\pagenote[48]{Rufinus names the priest of Saturn, who, in the
character of the god, familiarly conversed with many pious ladies
of quality, till he betrayed himself, in a moment of transport,
when he could not disguise the tone of his voice. The authentic
and impartial narrative of Aeschines, (see Bayle, Dictionnaire
Critique, Scamandre,) and the adventure of Mudus, (Joseph.
Antiquitat. Judaic. l. xviii. c. 3, p. 877 edit. Havercamp,) may
prove that such amorous frauds have been practised with success.}

\pagenote[49]{See the images of Serapis, in Montfaucon, (tom. ii.
p. 297:) but the description of Macrobius (Saturnal. l. i. c. 20)
is much more picturesque and satisfactory.}

\pagenote[50]{

Sed fortes tremuere manus, motique verenda Majestate loci, si
robora sacra ferirent In sua credebant redituras membra secures.

(Lucan. iii. 429.) “Is it true,” (said Augustus to a veteran of
Italy, at whose house he supped) “that the man who gave the first
blow to the golden statue of Anaitis, was instantly deprived of
his eyes, and of his life?”—“I was that man, (replied the
clear-sighted veteran,) and you now sup on one of the legs of the
goddess.” (Plin. Hist. Natur. xxxiii. 24)}

\pagenote[51]{The history of the reformation affords frequent
examples of the sudden change from superstition to contempt.}

\pagenote[52]{Sozomen, l. vii. c. 20. I have supplied the
measure. The same standard, of the inundation, and consequently
of the cubit, has uniformly subsisted since the time of
Herodotus. See Freret, in the Mem. de l’Academie des
Inscriptions, tom. xvi. p. 344-353. Greaves’s Miscellaneous
Works, vol. i. p. 233. The Egyptian cubit is about twenty-two
inches of the English measure. * Note: Compare Wilkinson’s Thebes
and Egypt, p. 313.—M.}

The temples of the Roman empire were deserted, or destroyed; but
the ingenious superstition of the Pagans still attempted to elude
the laws of Theodosius, by which all sacrifices had been severely
prohibited. The inhabitants of the country, whose conduct was
less opposed to the eye of malicious curiosity, disguised their
religious, under the appearance of convivial, meetings. On the
days of solemn festivals, they assembled in great numbers under
the spreading shade of some consecrated trees; sheep and oxen
were slaughtered and roasted; and this rural entertainment was
sanctified by the use of incense, and by the hymns which were
sung in honor of the gods. But it was alleged, that, as no part
of the animal was made a burnt-offering, as no altar was provided
to receive the blood, and as the previous oblation of salt cakes,
and the concluding ceremony of libations, were carefully omitted,
these festal meetings did not involve the guests in the guilt, or
penalty, of an illegal sacrifice.\textsuperscript{53} Whatever might be the truth
of the facts, or the merit of the distinction,\textsuperscript{54} these vain
pretences were swept away by the last edict of Theodosius, which
inflicted a deadly wound on the superstition of the Pagans.\textsuperscript{55} \textsuperscript{5511}
This prohibitory law is expressed in the most absolute and
comprehensive terms. “It is our will and pleasure,” says the
emperor, “that none of our subjects, whether magistrates or
private citizens, however exalted or however humble may be their
rank and condition, shall presume, in any city or in any place,
to worship an inanimate idol, by the sacrifice of a guiltless
victim.” The act of sacrificing, and the practice of divination
by the entrails of the victim, are declared (without any regard
to the object of the inquiry) a crime of high treason against the
state, which can be expiated only by the death of the guilty. The
rites of Pagan superstition, which might seem less bloody and
atrocious, are abolished, as highly injurious to the truth and
honor of religion; luminaries, garlands, frankincense, and
libations of wine, are specially enumerated and condemned; and
the harmless claims of the domestic genius, of the household
gods, are included in this rigorous proscription. The use of any
of these profane and illegal ceremonies, subjects the offender to
the forfeiture of the house or estate, where they have been
performed; and if he has artfully chosen the property of another
for the scene of his impiety, he is compelled to discharge,
without delay, a heavy fine of twenty-five pounds of gold, or
more than one thousand pounds sterling. A fine, not less
considerable, is imposed on the connivance of the secret enemies
of religion, who shall neglect the duty of their respective
stations, either to reveal, or to punish, the guilt of idolatry.
Such was the persecuting spirit of the laws of Theodosius, which
were repeatedly enforced by his sons and grandsons, with the loud
and unanimous applause of the Christian world.\textsuperscript{56}

\pagenote[53]{Libanius (pro Templis, p. 15, 16, 17) pleads their
cause with gentle and insinuating rhetoric. From the earliest
age, such feasts had enlivened the country: and those of Bacchus
(Georgic. ii. 380) had produced the theatre of Athens. See
Godefroy, ad loc. Liban. and Codex Theodos. tom. vi. p. 284.}

\pagenote[54]{Honorius tolerated these rustic festivals, (A.D.
399.) “Absque ullo sacrificio, atque ulla superstitione
damnabili.” But nine years afterwards he found it necessary to
reiterate and enforce the same proviso, (Codex Theodos. l. xvi.
tit. x. leg. 17, 19.)}

\pagenote[55]{Cod. Theodos. l. xvi. tit. x. leg. 12. Jortin
(Remarks on Eccles. History, vol. iv. p. 134) censures, with
becoming asperity, the style and sentiments of this intolerant
law.}

\pagenote[5511]{Paganism maintained its ground for a considerable
time in the rural districts. Endelechius, a poet who lived at the
beginning of the fifth century, speaks of the cross as Signum
quod perhibent esse crucis Dei, Magnis qui colitur solus
inurbibus. In the middle of the same century, Maximus, bishop of
Turin, writes against the heathen deities as if their worship was
still in full vigor in the neighborhood of his city. Augustine
complains of the encouragement of the Pagan rites by heathen
landowners; and Zeno of Verona, still later, reproves the apathy
of the Christian proprietors in conniving at this abuse. (Compare
Neander, ii. p. 169.) M. Beugnot shows that this was the case
throughout the north and centre of Italy and in Sicily. But
neither of these authors has adverted to one fact, which must
have tended greatly to retard the progress of Christianity in
these quarters. It was still chiefly a slave population which
cultivated the soil; and however, in the towns, the better class
of Christians might be eager to communicate “the blessed liberty
of the gospel” to this class of mankind; however their condition
could not but be silently ameliorated by the humanizing influence
of Christianity; yet, on the whole, no doubt the servile class
would be the least fitted to receive the gospel; and its general
propagation among them would be embarrassed by many peculiar
difficulties. The rural population was probably not entirely
converted before the general establishment of the monastic
institutions. Compare Quarterly Review of Beugnot. vol lvii. p.
52—M.}

\pagenote[56]{Such a charge should not be lightly made; but it
may surely be justified by the authority of St. Augustin, who
thus addresses the Donatists: “Quis nostrum, quis vestrum non
laudat leges ab Imperatoribus datas adversus sacrificia
Paganorum? Et certe longe ibi poera severior constituta est;
illius quippe impietatis capitale supplicium est.” Epist. xciii.
No. 10, quoted by Le Clerc, (Bibliothèque Choisie, tom. viii. p.
277,) who adds some judicious reflections on the intolerance of
the victorious Christians. * Note: Yet Augustine, with laudable
inconsistency, disapproved of the forcible demolition of the
temples. “Let us first extirpate the idolatry of the hearts of
the heathen, and they will either themselves invite us or
anticipate us in the execution of this good work,” tom. v. p. 62.
Compare Neander, ii. 169, and, in p. 155, a beautiful passage
from Chrysostom against all violent means of propagating
Christianity.—M.}

\section{Part \thesection.}

In the cruel reigns of Decius and Diocletian, Christianity had
been proscribed, as a revolt from the ancient and hereditary
religion of the empire; and the unjust suspicions which were
entertained of a dark and dangerous faction, were, in some
measure, countenanced by the inseparable union and rapid
conquests of the Catholic church. But the same excuses of fear
and ignorance cannot be applied to the Christian emperors who
violated the precepts of humanity and of the Gospel. The
experience of ages had betrayed the weakness, as well as folly,
of Paganism; the light of reason and of faith had already
exposed, to the greatest part of mankind, the vanity of idols;
and the declining sect, which still adhered to their worship,
might have been permitted to enjoy, in peace and obscurity, the
religious costumes of their ancestors. Had the Pagans been
animated by the undaunted zeal which possessed the minds of the
primitive believers, the triumph of the Church must have been
stained with blood; and the martyrs of Jupiter and Apollo might
have embraced the glorious opportunity of devoting their lives
and fortunes at the foot of their altars. But such obstinate zeal
was not congenial to the loose and careless temper of Polytheism.
The violent and repeated strokes of the orthodox princes were
broken by the soft and yielding substance against which they were
directed; and the ready obedience of the Pagans protected them
from the pains and penalties of the Theodosian Code.\textsuperscript{57} Instead
of asserting, that the authority of the gods was superior to that
of the emperor, they desisted, with a plaintive murmur, from the
use of those sacred rites which their sovereign had condemned. If
they were sometimes tempted by a sally of passion, or by the
hopes of concealment, to indulge their favorite superstition,
their humble repentance disarmed the severity of the Christian
magistrate, and they seldom refused to atone for their rashness,
by submitting, with some secret reluctance, to the yoke of the
Gospel. The churches were filled with the increasing multitude of
these unworthy proselytes, who had conformed, from temporal
motives, to the reigning religion; and whilst they devoutly
imitated the postures, and recited the prayers, of the faithful,
they satisfied their conscience by the silent and sincere
invocation of the gods of antiquity.\textsuperscript{58} If the Pagans wanted
patience to suffer they wanted spirit to resist; and the
scattered myriads, who deplored the ruin of the temples, yielded,
without a contest, to the fortune of their adversaries. The
disorderly opposition\textsuperscript{59} of the peasants of Syria, and the
populace of Alexandria, to the rage of private fanaticism, was
silenced by the name and authority of the emperor. The Pagans of
the West, without contributing to the elevation of Eugenius,
disgraced, by their partial attachment, the cause and character
of the usurper. The clergy vehemently exclaimed, that he
aggravated the crime of rebellion by the guilt of apostasy; that,
by his permission, the altar of victory was again restored; and
that the idolatrous symbols of Jupiter and Hercules were
displayed in the field, against the invincible standard of the
cross. But the vain hopes of the Pagans were soon annihilated by
the defeat of Eugenius; and they were left exposed to the
resentment of the conqueror, who labored to deserve the favor of
Heaven by the extirpation of idolatry.\textsuperscript{60}

\pagenote[57]{Orosius, l. vii. c. 28, p. 537. Augustin (Enarrat.
in Psalm cxl apud Lardner, Heathen Testimonies, vol. iv. p. 458)
insults their cowardice. “Quis eorum comprehensus est in
sacrificio (cum his legibus sta prohiberentur) et non negavit?”}

\pagenote[58]{Libanius (pro Templis, p. 17, 18) mentions, without
censure the occasional conformity, and as it were theatrical
play, of these hypocrites.}

\pagenote[59]{Libanius concludes his apology (p. 32) by declaring
to the emperor, that unless he expressly warrants the destruction
of the temples, the proprietors will defend themselves and the
laws.}

\pagenote[60]{Paulinus, in Vit. Ambros. c. 26. Augustin de
Civitat. Dei, l. v. c. 26. Theodoret, l. v. c. 24.}

A nation of slaves is always prepared to applaud the clemency of
their master, who, in the abuse of absolute power, does not
proceed to the last extremes of injustice and oppression.
Theodosius might undoubtedly have proposed to his Pagan subjects
the alternative of baptism or of death; and the eloquent Libanius
has praised the moderation of a prince, who never enacted, by any
positive law, that all his subjects should immediately embrace
and practise the religion of their sovereign.\textsuperscript{61} The profession
of Christianity was not made an essential qualification for the
enjoyment of the civil rights of society, nor were any peculiar
hardships imposed on the sectaries, who credulously received the
fables of Ovid, and obstinately rejected the miracles of the
Gospel. The palace, the schools, the army, and the senate, were
filled with declared and devout Pagans; they obtained, without
distinction, the civil and military honors of the empire.\textsuperscript{6111}
Theodosius distinguished his liberal regard for virtue and genius
by the consular dignity, which he bestowed on Symmachus;\textsuperscript{62} and
by the personal friendship which he expressed to Libanius;\textsuperscript{63} and
the two eloquent apologists of Paganism were never required
either to change or to dissemble their religious opinions. The
Pagans were indulged in the most licentious freedom of speech and
writing; the historical and philosophic remains of Eunapius,
Zosimus,\textsuperscript{64} and the fanatic teachers of the school of Plato,
betray the most furious animosity, and contain the sharpest
invectives, against the sentiments and conduct of their
victorious adversaries. If these audacious libels were publicly
known, we must applaud the good sense of the Christian princes,
who viewed, with a smile of contempt, the last struggles of
superstition and despair.\textsuperscript{65} But the Imperial laws, which
prohibited the sacrifices and ceremonies of Paganism, were
rigidly executed; and every hour contributed to destroy the
influence of a religion, which was supported by custom, rather
than by argument. The devotion or the poet, or the philosopher,
may be secretly nourished by prayer, meditation, and study; but
the exercise of public worship appears to be the only solid
foundation of the religious sentiments of the people, which
derive their force from imitation and habit. The interruption of
that public exercise may consummate, in the period of a few
years, the important work of a national revolution. The memory of
theological opinions cannot long be preserved, without the
artificial helps of priests, of temples, and of books.\textsuperscript{66} The
ignorant vulgar, whose minds are still agitated by the blind
hopes and terrors of superstition, will be soon persuaded by
their superiors to direct their vows to the reigning deities of
the age; and will insensibly imbibe an ardent zeal for the
support and propagation of the new doctrine, which spiritual
hunger at first compelled them to accept. The generation that
arose in the world after the promulgation of the Imperial laws,
was attracted within the pale of the Catholic church: and so
rapid, yet so gentle, was the fall of Paganism, that only
twenty-eight years after the death of Theodosius, the faint and
minute vestiges were no longer visible to the eye of the
legislator.\textsuperscript{67}

\pagenote[61]{Libanius suggests the form of a persecuting edict,
which Theodosius might enact, (pro Templis, p. 32;) a rash joke,
and a dangerous experiment. Some princes would have taken his
advice.}

\pagenote[6111]{The most remarkable instance of this, at a much
later period, occurs in the person of Merobaudes, a general and a
poet, who flourished in the first half of the fifth century. A
statue in honor of Merobaudes was placed in the Forum of Trajan,
of which the inscription is still extant. Fragments of his poems
have been recovered by the industry and sagacity of Niebuhr. In
one passage, Merobaudes, in the genuine heathen spirit,
attributes the ruin of the empire to the abolition of Paganism,
and almost renews the old accusation of Atheism against
Christianity. He impersonates some deity, probably Discord, who
summons Bellona to take arms for the destruction of Rome; and in
a strain of fierce irony recommends to her other fatal measures,
to extirpate the gods of Rome:—

Roma, ipsique tremant furialia murmura reges. Jam superos terris
atque hospita numina pelle: Romanos populare Deos, et nullus in
aris Vestoe exoratoe fotus strue palleat ignis. Ilis instructa
dolis palatia celsa subibo; Majorum mores, et pectora prisca
fugabo Funditus; atque simul, nullo discrimine rerum, Spernantur
fortes, nec sic reverentia justis. Attica neglecto pereat facundia
Phoebo: Indignis contingat honos, et pondera rerum; Non virtus sed
casus agat; tristique cupido; Pectoribus saevi demens furor
aestuet aevi; Omniaque hoec sine mente Jovis, sine numine sumimo.

Merobaudes in Niebuhr’s edit. of the Byzantines, p. 14.—M.}

\pagenote[62]{Denique pro meritis terrestribus aequa rependens

Munera, sacricolis summos impertit honores.
Dux bonus, et certare sinit cum laude suorum, Nec pago implicitos
per debita culmina mundi Ire viros prohibet. Ipse magistratum tibi
consulis, ipse tribunal
Contulit. Prudent. in Symmach. i. 617, \&c.

Note: I have inserted some lines omitted by Gibbon.—M.}

\pagenote[63]{Libanius (pro Templis, p. 32) is proud that
Theodosius should thus distinguish a man, who even in his
presence would swear by Jupiter. Yet this presence seems to be no
more than a figure of rhetoric.}

\pagenote[64]{Zosimus, who styles himself Count and Ex-advocate
of the Treasury, reviles, with partial and indecent bigotry, the
Christian princes, and even the father of his sovereign. His work
must have been privately circulated, since it escaped the
invectives of the ecclesiastical historians prior to Evagrius,
(l. iii. c. 40-42,) who lived towards the end of the sixth
century. * Note: Heyne in his Disquisitio in Zosimum Ejusque
Fidem. places Zosimum towards the close of the fifth century.
Zosim. Heynii, p. xvii.—M.}

\pagenote[65]{Yet the Pagans of Africa complained, that the times
would not allow them to answer with freedom the City of God; nor
does St. Augustin (v. 26) deny the charge.}

\pagenote[66]{The Moors of Spain, who secretly preserved the
Mahometan religion above a century, under the tyranny of the
Inquisition, possessed the Koran, with the peculiar use of the
Arabic tongue. See the curious and honest story of their
expulsion in Geddes, (Miscellanies, vol. i. p. 1-198.)}

\pagenote[67]{Paganos qui supersunt, quanquam jam nullos esse
credamus, \&c. Cod. Theodos. l. xvi. tit. x. leg. 22, A.D. 423.
The younger Theodosius was afterwards satisfied, that his
judgment had been somewhat premature. Note: The statement of
Gibbon is much too strongly worded. M. Beugnot has traced the
vestiges of Paganism in the West, after this period, in monuments
and inscriptions with curious industry. Compare likewise note, p.
112, on the more tardy progress of Christianity in the rural
districts.—M.}

The ruin of the Pagan religion is described by the sophists as a
dreadful and amazing prodigy, which covered the earth with
darkness, and restored the ancient dominion of chaos and of
night. They relate, in solemn and pathetic strains, that the
temples were converted into sepulchres, and that the holy places,
which had been adorned by the statues of the gods, were basely
polluted by the relics of Christian martyrs. “The monks” (a race
of filthy animals, to whom Eunapius is tempted to refuse the name
of men) “are the authors of the new worship, which, in the place
of those deities who are conceived by the understanding, has
substituted the meanest and most contemptible slaves. The heads,
salted and pickled, of those infamous malefactors, who for the
multitude of their crimes have suffered a just and ignominious
death; their bodies still marked by the impression of the lash,
and the scars of those tortures which were inflicted by the
sentence of the magistrate; such” (continues Eunapius) “are the
gods which the earth produces in our days; such are the martyrs,
the supreme arbitrators of our prayers and petitions to the
Deity, whose tombs are now consecrated as the objects of the
veneration of the people.”\textsuperscript{68} Without approving the malice, it is
natural enough to share the surprise of the sophist, the
spectator of a revolution, which raised those obscure victims of
the laws of Rome to the rank of celestial and invisible
protectors of the Roman empire. The grateful respect of the
Christians for the martyrs of the faith, was exalted, by time and
victory, into religious adoration; and the most illustrious of
the saints and prophets were deservedly associated to the honors
of the martyrs. One hundred and fifty years after the glorious
deaths of St. Peter and St. Paul, the Vatican and the Ostian road
were distinguished by the tombs, or rather by the trophies, of
those spiritual heroes.\textsuperscript{69} In the age which followed the
conversion of Constantine, the emperors, the consuls, and the
generals of armies, devoutly visited the sepulchres of a
tentmaker and a fisherman;\textsuperscript{70} and their venerable bones were
deposited under the altars of Christ, on which the bishops of the
royal city continually offered the unbloody sacrifice.\textsuperscript{71} The new
capital of the Eastern world, unable to produce any ancient and
domestic trophies, was enriched by the spoils of dependent
provinces. The bodies of St. Andrew, St. Luke, and St. Timothy,
had reposed near three hundred years in the obscure graves, from
whence they were transported, in solemn pomp, to the church of
the apostles, which the magnificence of Constantine had founded
on the banks of the Thracian Bosphorus.\textsuperscript{72} About fifty years
afterwards, the same banks were honored by the presence of
Samuel, the judge and prophet of the people of Israel. His ashes,
deposited in a golden vase, and covered with a silken veil, were
delivered by the bishops into each other’s hands. The relics of
Samuel were received by the people with the same joy and
reverence which they would have shown to the living prophet; the
highways, from Palestine to the gates of Constantinople, were
filled with an uninterrupted procession; and the emperor Arcadius
himself, at the head of the most illustrious members of the
clergy and senate, advanced to meet his extraordinary guest, who
had always deserved and claimed the homage of kings.\textsuperscript{73} The
example of Rome and Constantinople confirmed the faith and
discipline of the Catholic world. The honors of the saints and
martyrs, after a feeble and ineffectual murmur of profane reason,\textsuperscript{74}
were universally established; and in the age of Ambrose and
Jerom, something was still deemed wanting to the sanctity of a
Christian church, till it had been consecrated by some portion of
holy relics, which fixed and inflamed the devotion of the
faithful.

\pagenote[68]{See Eunapius, in the Life of the sophist Aedesius;
in that of Eustathius he foretells the ruin of Paganism.}

\pagenote[69]{Caius, (apud Euseb. Hist. Eccles. l. ii. c. 25,) a
Roman presbyter, who lived in the time of Zephyrinus, (A.D.
202-219,) is an early witness of this superstitious practice.}

\pagenote[70]{Chrysostom. Quod Christus sit Deus. Tom. i. nov.
edit. No. 9. I am indebted for this quotation to Benedict the
XIVth’s pastoral letter on the Jubilee of the year 1759. See the
curious and entertaining letters of M. Chais, tom. iii.}

\pagenote[71]{Male facit ergo Romanus episcopus? qui, super
mortuorum hominum, Petri \& Pauli, secundum nos, ossa veneranda
... offeri Domino sacrificia, et tumulos eorum, Christi
arbitratur altaria. Jerom. tom. ii. advers. Vigilant. p. 183.}

\pagenote[72]{Jerom (tom. ii. p. 122) bears witness to these
translations, which are neglected by the ecclesiastical
historians. The passion of St. Andrew at Patrae is described in
an epistle from the clergy of Achaia, which Baronius (Annal.
Eccles. A.D. 60, No. 34) wishes to believe, and Tillemont is
forced to reject. St. Andrew was adopted as the spiritual founder
of Constantinople, (Mem. Eccles. tom. i. p. 317-323, 588-594.)}

\pagenote[73]{Jerom (tom. ii. p. 122) pompously describes the
translation of Samuel, which is noticed in all the chronicles of
the times.}

\pagenote[74]{The presbyter Vigilantius, the Protestant of his
age, firmly, though ineffectually, withstood the superstition of
monks, relics, saints, fasts, \&c., for which Jerom compares him
to the Hydra, Cerberus, the Centaurs, \&c., and considers him only
as the organ of the Daemon, (tom. ii. p. 120-126.) Whoever will
peruse the controversy of St. Jerom and Vigilantius, and St.
Augustin’s account of the miracles of St. Stephen, may speedily
gain some idea of the spirit of the Fathers.}

In the long period of twelve hundred years, which elapsed between
the reign of Constantine and the reformation of Luther, the
worship of saints and relics corrupted the pure and perfect
simplicity of the Christian model: and some symptoms of
degeneracy may be observed even in the first generations which
adopted and cherished this pernicious innovation.

I. The satisfactory experience, that the relics of saints were
more valuable than gold or precious stones,\textsuperscript{75} stimulated the
clergy to multiply the treasures of the church. Without much
regard for truth or probability, they invented names for
skeletons, and actions for names. The fame of the apostles, and
of the holy men who had imitated their virtues, was darkened by
religious fiction. To the invincible band of genuine and
primitive martyrs, they added myriads of imaginary heroes, who
had never existed, except in the fancy of crafty or credulous
legendaries; and there is reason to suspect, that Tours might not
be the only diocese in which the bones of a malefactor were
adored, instead of those of a saint.\textsuperscript{76} A superstitious practice,
which tended to increase the temptations of fraud, and credulity,
insensibly extinguished the light of history, and of reason, in
the Christian world.

\pagenote[75]{M. de Beausobre (Hist. du Manicheisme, tom. ii. p.
648) has applied a worldly sense to the pious observation of the
clergy of Smyrna, who carefully preserved the relics of St.
Polycarp the martyr.}

\pagenote[76]{Martin of Tours (see his Life, c. 8, by Sulpicius
Severus) extorted this confession from the mouth of the dead man.
The error is allowed to be natural; the discovery is supposed to
be miraculous. Which of the two was likely to happen most
frequently?}

II. But the progress of superstition would have been much less
rapid and victorious, if the faith of the people had not been
assisted by the seasonable aid of visions and miracles, to
ascertain the authenticity and virtue of the most suspicious
relics. In the reign of the younger Theodosius, Lucian,\textsuperscript{77} a
presbyter of Jerusalem, and the ecclesiastical minister of the
village of Caphargamala, about twenty miles from the city,
related a very singular dream, which, to remove his doubts, had
been repeated on three successive Saturdays. A venerable figure
stood before him, in the silence of the night, with a long beard,
a white robe, and a gold rod; announced himself by the name of
Gamaliel, and revealed to the astonished presbyter, that his own
corpse, with the bodies of his son Abibas, his friend Nicodemus,
and the illustrious Stephen, the first martyr of the Christian
faith, were secretly buried in the adjacent field. He added, with
some impatience, that it was time to release himself and his
companions from their obscure prison; that their appearance would
be salutary to a distressed world; and that they had made choice
of Lucian to inform the bishop of Jerusalem of their situation
and their wishes. The doubts and difficulties which still
retarded this important discovery were successively removed by
new visions; and the ground was opened by the bishop, in the
presence of an innumerable multitude. The coffins of Gamaliel, of
his son, and of his friend, were found in regular order; but when
the fourth coffin, which contained the remains of Stephen, was
shown to the light, the earth trembled, and an odor, such as that
of paradise, was smelt, which instantly cured the various
diseases of seventy-three of the assistants. The companions of
Stephen were left in their peaceful residence of Caphargamala:
but the relics of the first martyr were transported, in solemn
procession, to a church constructed in their honor on Mount Sion;
and the minute particles of those relics, a drop of blood,\textsuperscript{78} or
the scrapings of a bone, were acknowledged, in almost every
province of the Roman world, to possess a divine and miraculous
virtue. The grave and learned Augustin,\textsuperscript{79} whose understanding
scarcely admits the excuse of credulity, has attested the
innumerable prodigies which were performed in Africa by the
relics of St. Stephen; and this marvellous narrative is inserted
in the elaborate work of the City of God, which the bishop of
Hippo designed as a solid and immortal proof of the truth of
Christianity. Augustin solemnly declares, that he has selected
those miracles only which were publicly certified by the persons
who were either the objects, or the spectators, of the power of
the martyr. Many prodigies were omitted, or forgotten; and Hippo
had been less favorably treated than the other cities of the
province. And yet the bishop enumerates above seventy miracles,
of which three were resurrections from the dead, in the space of
two years, and within the limits of his own diocese.\textsuperscript{80} If we
enlarge our view to all the dioceses, and all the saints, of the
Christian world, it will not be easy to calculate the fables, and
the errors, which issued from this inexhaustible source. But we
may surely be allowed to observe, that a miracle, in that age of
superstition and credulity, lost its name and its merit, since it
could scarcely be considered as a deviation from the ordinary and
established laws of nature.

\pagenote[77]{Lucian composed in Greek his original narrative,
which has been translated by Avitus, and published by Baronius,
(Annal. Eccles. A.D. 415, No. 7-16.) The Benedictine editors of
St. Augustin have given (at the end of the work de Civitate Dei)
two several copies, with many various readings. It is the
character of falsehood to be loose and inconsistent. The most
incredible parts of the legend are smoothed and softened by
Tillemont, (Mem. Eccles. tom. ii. p. 9, \&c.)}

\pagenote[78]{A phial of St. Stephen’s blood was annually
liquefied at Naples, till he was superseded by St. Jamarius,
(Ruinart. Hist. Persecut. Vandal p. 529.)}

\pagenote[79]{Augustin composed the two-and-twenty books de
Civitate Dei in the space of thirteen years, A.D. 413-426.
Tillemont, (Mem. Eccles. tom. xiv. p. 608, \&c.) His learning is
too often borrowed, and his arguments are too often his own; but
the whole work claims the merit of a magnificent design,
vigorously, and not unskilfully, executed.}

\pagenote[80]{See Augustin de Civitat. Dei, l. xxii. c. 22, and
the Appendix, which contains two books of St. Stephen’s miracles,
by Evodius, bishop of Uzalis. Freculphus (apud Basnage, Hist. des
Juifs, tom. vii. p. 249) has preserved a Gallic or a Spanish
proverb, “Whoever pretends to have read all the miracles of St.
Stephen, he lies.”}

III. The innumerable miracles, of which the tombs of the martyrs
were the perpetual theatre, revealed to the pious believer the
actual state and constitution of the invisible world; and his
religious speculations appeared to be founded on the firm basis
of fact and experience. Whatever might be the condition of vulgar
souls, in the long interval between the dissolution and the
resurrection of their bodies, it was evident that the superior
spirits of the saints and martyrs did not consume that portion of
their existence in silent and inglorious sleep.\textsuperscript{81} It was evident
(without presuming to determine the place of their habitation, or
the nature of their felicity) that they enjoyed the lively and
active consciousness of their happiness, their virtue, and their
powers; and that they had already secured the possession of their
eternal reward. The enlargement of their intellectual faculties
surpassed the measure of the human imagination; since it was
proved by experience, that they were capable of hearing and
understanding the various petitions of their numerous votaries;
who, in the same moment of time, but in the most distant parts of
the world, invoked the name and assistance of Stephen or of
Martin.\textsuperscript{82} The confidence of their petitioners was founded on the
persuasion, that the saints, who reigned with Christ, cast an eye
of pity upon earth; that they were warmly interested in the
prosperity of the Catholic Church; and that the individuals, who
imitated the example of their faith and piety, were the peculiar
and favorite objects of their most tender regard. Sometimes,
indeed, their friendship might be influenced by considerations of
a less exalted kind: they viewed with partial affection the
places which had been consecrated by their birth, their
residence, their death, their burial, or the possession of their
relics. The meaner passions of pride, avarice, and revenge, may
be deemed unworthy of a celestial breast; yet the saints
themselves condescended to testify their grateful approbation of
the liberality of their votaries; and the sharpest bolts of
punishment were hurled against those impious wretches, who
violated their magnificent shrines, or disbelieved their
supernatural power.\textsuperscript{83} Atrocious, indeed, must have been the
guilt, and strange would have been the scepticism, of those men,
if they had obstinately resisted the proofs of a divine agency,
which the elements, the whole range of the animal creation, and
even the subtle and invisible operations of the human mind, were
compelled to obey.\textsuperscript{84} The immediate, and almost instantaneous,
effects that were supposed to follow the prayer, or the offence,
satisfied the Christians of the ample measure of favor and
authority which the saints enjoyed in the presence of the Supreme
God; and it seemed almost superfluous to inquire whether they
were continually obliged to intercede before the throne of grace;
or whether they might not be permitted to exercise, according to
the dictates of their benevolence and justice, the delegated
powers of their subordinate ministry. The imagination, which had
been raised by a painful effort to the contemplation and worship
of the Universal Cause, eagerly embraced such inferior objects of
adoration as were more proportioned to its gross conceptions and
imperfect faculties. The sublime and simple theology of the
primitive Christians was gradually corrupted; and the Monarchy of
heaven, already clouded by metaphysical subtleties, was degraded
by the introduction of a popular mythology, which tended to
restore the reign of polytheism.\textsuperscript{85}

\pagenote[81]{Burnet (de Statu Mortuorum, p. 56-84) collects the
opinions of the Fathers, as far as they assert the sleep, or
repose, of human souls till the day of judgment. He afterwards
exposes (p. 91, \&c.) the inconveniences which must arise, if they
possessed a more active and sensible existence.}

\pagenote[82]{Vigilantius placed the souls of the prophets and
martyrs, either in the bosom of Abraham, (in loco refrigerii,) or
else under the altar of God. Nec posse suis tumulis et ubi
voluerunt adesse praesentes. But Jerom (tom. ii. p. 122) sternly
refutes this blasphemy. Tu Deo leges pones? Tu apostolis vincula
injicies, ut usque ad diem judicii teneantur custodia, nec sint
cum Domino suo; de quibus scriptum est, Sequuntur Agnum quocunque
vadit. Si Agnus ubique, ergo, et hi, qui cum Agno sunt, ubique
esse credendi sunt. Et cum diabolus et daemones tote vagentur in
orbe, \&c.}

\pagenote[83]{Fleury Discours sur l’Hist. Ecclesiastique, iii p.
80.}

\pagenote[84]{At Minorca, the relics of St. Stephen converted, in
eight days, 540 Jews; with the help, indeed, of some wholesome
severities, such as burning the synagogue, driving the obstinate
infidels to starve among the rocks, \&c. See the original letter
of Severus, bishop of Minorca (ad calcem St. Augustin. de Civ.
Dei,) and the judicious remarks of Basnage, (tom. viii. p.
245-251.)}

\pagenote[85]{Mr. Hume (Essays, vol. ii. p. 434) observes, like a
philosopher, the natural flux and reflux of polytheism and
theism.}

IV. As the objects of religion were gradually reduced to the
standard of the imagination, the rites and ceremonies were
introduced that seemed most powerfully to affect the senses of
the vulgar. If, in the beginning of the fifth century,\textsuperscript{86}
Tertullian, or Lactantius,\textsuperscript{87} had been suddenly raised from the
dead, to assist at the festival of some popular saint, or martyr,\textsuperscript{88}
they would have gazed with astonishment, and indignation, on
the profane spectacle, which had succeeded to the pure and
spiritual worship of a Christian congregation. As soon as the
doors of the church were thrown open, they must have been
offended by the smoke of incense, the perfume of flowers, and the
glare of lamps and tapers, which diffused, at noonday, a gaudy,
superfluous, and, in their opinion, a sacrilegious light. If they
approached the balustrade of the altar, they made their way
through the prostrate crowd, consisting, for the most part, of
strangers and pilgrims, who resorted to the city on the vigil of
the feast; and who already felt the strong intoxication of
fanaticism, and, perhaps, of wine. Their devout kisses were
imprinted on the walls and pavement of the sacred edifice; and
their fervent prayers were directed, whatever might be the
language of their church, to the bones, the blood, or the ashes
of the saint, which were usually concealed, by a linen or silken
veil, from the eyes of the vulgar. The Christians frequented the
tombs of the martyrs, in the hope of obtaining, from their
powerful intercession, every sort of spiritual, but more
especially of temporal, blessings. They implored the preservation
of their health, or the cure of their infirmities; the
fruitfulness of their barren wives, or the safety and happiness
of their children. Whenever they undertook any distant or
dangerous journey, they requested, that the holy martyrs would be
their guides and protectors on the road; and if they returned
without having experienced any misfortune, they again hastened to
the tombs of the martyrs, to celebrate, with grateful
thanksgivings, their obligations to the memory and relics of
those heavenly patrons. The walls were hung round with symbols of
the favors which they had received; eyes, and hands, and feet, of
gold and silver: and edifying pictures, which could not long
escape the abuse of indiscreet or idolatrous devotion,
represented the image, the attributes, and the miracles of the
tutelar saint. The same uniform original spirit of superstition
might suggest, in the most distant ages and countries, the same
methods of deceiving the credulity, and of affecting the senses
of mankind:\textsuperscript{89} but it must ingenuously be confessed, that the
ministers of the Catholic church imitated the profane model,
which they were impatient to destroy. The most respectable
bishops had persuaded themselves that the ignorant rustics would
more cheerfully renounce the superstitions of Paganism, if they
found some resemblance, some compensation, in the bosom of
Christianity. The religion of Constantine achieved, in less than
a century, the final conquest of the Roman empire: but the
victors themselves were insensibly subdued by the arts of their
vanquished rivals.\textsuperscript{90} \textsuperscript{9011}

\pagenote[86]{D’Aubigne (see his own Mémoires, p. 156-160)
frankly offered, with the consent of the Huguenot ministers, to
allow the first 400 years as the rule of faith. The Cardinal du
Perron haggled for forty years more, which were indiscreetly
given. Yet neither party would have found their account in this
foolish bargain.}

\pagenote[87]{The worship practised and inculcated by Tertullian,
Lactantius Arnobius, \&c., is so extremely pure and spiritual,
that their declamations against the Pagan sometimes glance
against the Jewish, ceremonies.}

\pagenote[88]{Faustus the Manichaean accuses the Catholics of
idolatry. Vertitis idola in martyres.... quos votis similibus
colitis. M. de Beausobre, (Hist. Critique du Manicheisme, tom.
ii. p. 629-700,) a Protestant, but a philosopher, has
represented, with candor and learning, the introduction of
Christian idolatry in the fourth and fifth centuries.}

\pagenote[89]{The resemblance of superstition, which could not be
imitated, might be traced from Japan to Mexico. Warburton has
seized this idea, which he distorts, by rendering it too general
and absolute, (Divine Legation, vol. iv. p. 126, \&c.)}

\pagenote[90]{The imitation of Paganism is the subject of Dr.
Middleton’s agreeable letter from Rome. Warburton’s
animadversions obliged him to connect (vol. iii. p. 120-132,) the
history of the two religions, and to prove the antiquity of the
Christian copy.}

\pagenote[9011]{But there was always this important difference
between Christian and heathen Polytheism. In Paganism this was
the whole religion; in the darkest ages of Christianity, some,
however obscure and vague, Christian notions of future
retribution, of the life after death, lurked at the bottom, and
operated, to a certain extent, on the thoughts and feelings,
sometimes on the actions.—M.}

