\chapter{Revolt Of The Goths.}

\textit{Revolt Of The Goths. — They Plunder Greece. — Two Great Invasions Of
Italy By Alaric And Radagaisus. — They Are Repulsed By Stilicho. — The
Germans Overrun Gaul. — Usurpation Of Constantine In The
West. — Disgrace And Death Of Stilicho.}

If the subjects of Rome could be ignorant of their obligations to
the great Theodosius, they were too soon convinced, how painfully
the spirit and abilities of their deceased emperor had supported
the frail and mouldering edifice of the republic. He died in the
month of January; and before the end of the winter of the same
year, the Gothic nation was in arms.\textsuperscript{1} The Barbarian auxiliaries
erected their independent standard; and boldly avowed the hostile
designs, which they had long cherished in their ferocious minds.
Their countrymen, who had been condemned, by the conditions of
the last treaty, to a life of tranquility and labor, deserted
their farms at the first sound of the trumpet; and eagerly
resumed the weapons which they had reluctantly laid down. The
barriers of the Danube were thrown open; the savage warriors of
Scythia issued from their forests; and the uncommon severity of
the winter allowed the poet to remark, “that they rolled their
ponderous wagons over the broad and icy back of the indignant
river.”\textsuperscript{2} The unhappy natives of the provinces to the south of
the Danube submitted to the calamities, which, in the course of
twenty years, were almost grown familiar to their imagination;
and the various troops of Barbarians, who gloried in the Gothic
name, were irregularly spread from woody shores of Dalmatia, to
the walls of Constantinople.\textsuperscript{3} The interruption, or at least the
diminution, of the subsidy, which the Goths had received from the
prudent liberality of Theodosius, was the specious pretence of
their revolt: the affront was imbittered by their contempt for
the unwarlike sons of Theodosius; and their resentment was
inflamed by the weakness, or treachery, of the minister of
Arcadius. The frequent visits of Rufinus to the camp of the
Barbarians whose arms and apparel he affected to imitate, were
considered as a sufficient evidence of his guilty correspondence,
and the public enemy, from a motive either of gratitude or of
policy, was attentive, amidst the general devastation, to spare
the private estates of the unpopular præfect. The Goths, instead
of being impelled by the blind and headstrong passions of their
chiefs, were now directed by the bold and artful genius of
Alaric. That renowned leader was descended from the noble race of
the Balti;\textsuperscript{4} which yielded only to the royal dignity of the
Amali: he had solicited the command of the Roman armies; and the
Imperial court provoked him to demonstrate the folly of their
refusal, and the importance of their loss. Whatever hopes might
be entertained of the conquest of Constantinople, the judicious
general soon abandoned an impracticable enterprise. In the midst
of a divided court and a discontented people, the emperor
Arcadius was terrified by the aspect of the Gothic arms; but the
want of wisdom and valor was supplied by the strength of the
city; and the fortifications, both of the sea and land, might
securely brave the impotent and random darts of the Barbarians.
Alaric disdained to trample any longer on the prostrate and
ruined countries of Thrace and Dacia, and he resolved to seek a
plentiful harvest of fame and riches in a province which had
hitherto escaped the ravages of war.\textsuperscript{5}

\pagenote[1]{The revolt of the Goths, and the blockade of
Constantinople, are distinctly mentioned by Claudian, (in Rufin.
l. ii. 7-100,) Zosimus, (l. v. 292,) and Jornandes, (de Rebus
Geticis, c. 29.)}

\pagenote[2]{—

Alii per toga ferocis Danubii solidata ruunt; expertaque remis
Frangunt stagna rotis.

Claudian and Ovid often amuse their fancy by interchanging the
metaphors and properties of liquid water, and solid ice. Much
false wit has been expended in this easy exercise.}

\pagenote[3]{Jerom, tom. i. p. 26. He endeavors to comfort his
friend Heliodorus, bishop of Altinum, for the loss of his nephew,
Nepotian, by a curious recapitulation of all the public and
private misfortunes of the times. See Tillemont, Mem. Eccles.
tom. xii. p. 200, \&c.}

\pagenote[4]{Baltha or bold: origo mirifica, says Jornandes, (c.
29.) This illustrious race long continued to flourish in France,
in the Gothic province of Septimania, or Languedoc; under the
corrupted appellation of Boax; and a branch of that family
afterwards settled in the kingdom of Naples (Grotius in Prolegom.
ad Hist. Gothic. p. 53.) The lords of Baux, near Arles, and of
seventy-nine subordinate places, were independent of the counts
of Provence, (Longuerue, Description de la France, tom. i. p.
357).}

\pagenote[5]{Zosimus (l. v. p. 293-295) is our best guide for the
conquest of Greece: but the hints and allusion of Claudian are so
many rays of historic light.}

The character of the civil and military officers, on whom Rufinus
had devolved the government of Greece, confirmed the public
suspicion, that he had betrayed the ancient seat of freedom and
learning to the Gothic invader. The proconsul Antiochus was the
unworthy son of a respectable father; and Gerontius, who
commanded the provincial troops, was much better qualified to
execute the oppressive orders of a tyrant, than to defend, with
courage and ability, a country most remarkably fortified by the
hand of nature. Alaric had traversed, without resistance, the
plains of Macedonia and Thessaly, as far as the foot of Mount
Oeta, a steep and woody range of hills, almost impervious to his
cavalry. They stretched from east to west, to the edge of the
sea-shore; and left, between the precipice and the Malian Gulf,
an interval of three hundred feet, which, in some places, was
contracted to a road capable of admitting only a single carriage.\textsuperscript{6}
In this narrow pass of Thermopylae, where Leonidas and the
three hundred Spartans had gloriously devoted their lives, the
Goths might have been stopped, or destroyed, by a skilful
general; and perhaps the view of that sacred spot might have
kindled some sparks of military ardor in the breasts of the
degenerate Greeks. The troops which had been posted to defend the
Straits of Thermopylae, retired, as they were directed, without
attempting to disturb the secure and rapid passage of Alaric;\textsuperscript{7}
and the fertile fields of Phocis and Boeotia were instantly
covered by a deluge of Barbarians who massacred the males of an
age to bear arms, and drove away the beautiful females, with the
spoil and cattle of the flaming villages. The travellers, who
visited Greece several years afterwards, could easily discover
the deep and bloody traces of the march of the Goths; and Thebes
was less indebted for her preservation to the strength of her
seven gates, than to the eager haste of Alaric, who advanced to
occupy the city of Athens, and the important harbor of the
Piraeus. The same impatience urged him to prevent the delay and
danger of a siege, by the offer of a capitulation; and as soon as
the Athenians heard the voice of the Gothic herald, they were
easily persuaded to deliver the greatest part of their wealth, as
the ransom of the city of Minerva and its inhabitants. The treaty
was ratified by solemn oaths, and observed with mutual fidelity.
The Gothic prince, with a small and select train, was admitted
within the walls; he indulged himself in the refreshment of the
bath, accepted a splendid banquet, which was provided by the
magistrate, and affected to show that he was not ignorant of the
manners of civilized nations.\textsuperscript{8} But the whole territory of
Attica, from the promontory of Sunium to the town of Megara, was
blasted by his baleful presence; and, if we may use the
comparison of a contemporary philosopher, Athens itself resembled
the bleeding and empty skin of a slaughtered victim. The distance
between Megara and Corinth could not much exceed thirty miles;
but the bad road, an expressive name, which it still bears among
the Greeks, was, or might easily have been made, impassable for
the march of an enemy. The thick and gloomy woods of Mount
Cithaeron covered the inland country; the Scironian rocks
approached the water’s edge, and hung over the narrow and winding
path, which was confined above six miles along the sea-shore.\textsuperscript{9}
The passage of those rocks, so infamous in every age, was
terminated by the Isthmus of Corinth; and a small a body of firm
and intrepid soldiers might have successfully defended a
temporary intrenchment of five or six miles from the Ionian to
the Aegean Sea. The confidence of the cities of Peloponnesus in
their natural rampart, had tempted them to neglect the care of
their antique walls; and the avarice of the Roman governors had
exhausted and betrayed the unhappy province.\textsuperscript{10} Corinth, Argos,
Sparta, yielded without resistance to the arms of the Goths; and
the most fortunate of the inhabitants were saved, by death, from
beholding the slavery of their families and the conflagration of
their cities.\textsuperscript{11} The vases and statues were distributed among the
Barbarians, with more regard to the value of the materials, than
to the elegance of the workmanship; the female captives submitted
to the laws of war; the enjoyment of beauty was the reward of
valor; and the Greeks could not reasonably complain of an abuse
which was justified by the example of the heroic times.\textsuperscript{12} The
descendants of that extraordinary people, who had considered
valor and discipline as the walls of Sparta, no longer remembered
the generous reply of their ancestors to an invader more
formidable than Alaric. “If thou art a god, thou wilt not hurt
those who have never injured thee; if thou art a man,
advance:—and thou wilt find men equal to thyself.”\textsuperscript{13} From
Thermopylae to Sparta, the leader of the Goths pursued his
victorious march without encountering any mortal antagonists: but
one of the advocates of expiring Paganism has confidently
asserted, that the walls of Athens were guarded by the goddess
Minerva, with her formidable Aegis, and by the angry phantom of
Achilles;\textsuperscript{14} and that the conqueror was dismayed by the presence
of the hostile deities of Greece. In an age of miracles, it would
perhaps be unjust to dispute the claim of the historian Zosimus
to the common benefit: yet it cannot be dissembled, that the mind
of Alaric was ill prepared to receive, either in sleeping or
waking visions, the impressions of Greek superstition. The songs
of Homer, and the fame of Achilles, had probably never reached
the ear of the illiterate Barbarian; and the Christian faith,
which he had devoutly embraced, taught him to despise the
imaginary deities of Rome and Athens. The invasion of the Goths,
instead of vindicating the honor, contributed, at least
accidentally, to extirpate the last remains of Paganism: and the
mysteries of Ceres, which had subsisted eighteen hundred years,
did not survive the destruction of Eleusis, and the calamities of
Greece.\textsuperscript{15}

\pagenote[6]{Compare Herodotus (l. vii. c. 176) and Livy, (xxxvi.
15.) The narrow entrance of Greece was probably enlarged by each
successive ravisher.}

\pagenote[7]{He passed, says Eunapius, (in Vit. Philosoph. p. 93,
edit. Commelin, 1596,) through the straits, of Thermopylae.}

\pagenote[8]{In obedience to Jerom and Claudian, (in Rufin. l.
ii. 191,) I have mixed some darker colors in the mild
representation of Zosimus, who wished to soften the calamities of
Athens.

Nec fera Cecropias traxissent vincula matres.

Synesius (Epist. clvi. p. 272, edit. Petav.) observes, that
Athens, whose sufferings he imputes to the proconsul’s avarice,
was at that time less famous for her schools of philosophy than
for her trade of honey.}

\pagenote[9]{—

Vallata mari Scironia rupes, Et duo continuo connectens aequora
muro Isthmos. —Claudian de Bel. Getico, 188.

The Scironian rocks are described by Pausanias, (l. i. c. 44, p.
107, edit. Kuhn,) and our modern travellers, Wheeler (p. 436) and
Chandler, (p. 298.) Hadrian made the road passable for two
carriages.}

\pagenote[10]{Claudian (in Rufin. l. ii. 186, and de Bello
Getico, 611, \&c.) vaguely, though forcibly, delineates the scene
of rapine and destruction.}

\pagenote[11]{These generous lines of Homer (Odyss. l. v. 306)
were transcribed by one of the captive youths of Corinth: and the
tears of Mummius may prove that the rude conqueror, though he was
ignorant of the value of an original picture, possessed the
purest source of good taste, a benevolent heart, (Plutarch,
Symposiac. l. ix. tom. ii. p. 737, edit. Wechel.)}

\pagenote[12]{Homer perpetually describes the exemplary patience
of those female captives, who gave their charms, and even their
hearts, to the murderers of their fathers, brothers, \&c. Such a
passion (of Eriphile for Achilles) is touched with admirable
delicacy by Racine.}

\pagenote[13]{Plutarch (in Pyrrho, tom. ii. p. 474, edit. Brian)
gives the genuine answer in the Laconic dialect. Pyrrhus attacked
Sparta with 25,000 foot, 2000 horse, and 24 elephants, and the
defence of that open town is a fine comment on the laws of
Lycurgus, even in the last stage of decay.}

\pagenote[14]{Such, perhaps, as Homer (Iliad, xx. 164) had so
nobly painted him.}

\pagenote[15]{Eunapius (in Vit. Philosoph. p. 90-93) intimates
that a troop of monks betrayed Greece, and followed the Gothic
camp. * Note: The expression is curious: Vit. Max. t. i. p. 53,
edit. Boissonade.—M.}

The last hope of a people who could no longer depend on their
arms, their gods, or their sovereign, was placed in the powerful
assistance of the general of the West; and Stilicho, who had not
been permitted to repulse, advanced to chastise, the invaders of
Greece.\textsuperscript{16} A numerous fleet was equipped in the ports of Italy;
and the troops, after a short and prosperous navigation over the
Ionian Sea, were safely disembarked on the isthmus, near the
ruins of Corinth. The woody and mountainous country of Arcadia,
the fabulous residence of Pan and the Dryads, became the scene of
a long and doubtful conflict between the two generals not
unworthy of each other. The skill and perseverance of the Roman
at length prevailed; and the Goths, after sustaining a
considerable loss from disease and desertion, gradually retreated
to the lofty mountain of Pholoe, near the sources of the Peneus,
and on the frontiers of Elis; a sacred country, which had
formerly been exempted from the calamities of war.\textsuperscript{17} The camp of
the Barbarians was immediately besieged; the waters of the river\textsuperscript{18}
were diverted into another channel; and while they labored
under the intolerable pressure of thirst and hunger, a strong
line of circumvallation was formed to prevent their escape. After
these precautions, Stilicho, too confident of victory, retired to
enjoy his triumph, in the theatrical games, and lascivious
dances, of the Greeks; his soldiers, deserting their standards,
spread themselves over the country of their allies, which they
stripped of all that had been saved from the rapacious hands of
the enemy. Alaric appears to have seized the favorable moment to
execute one of those hardy enterprises, in which the abilities of
a general are displayed with more genuine lustre, than in the
tumult of a day of battle. To extricate himself from the prison
of Peloponnesus, it was necessary that he should pierce the
intrenchments which surrounded his camp; that he should perform a
difficult and dangerous march of thirty miles, as far as the Gulf
of Corinth; and that he should transport his troops, his
captives, and his spoil, over an arm of the sea, which, in the
narrow interval between Rhium and the opposite shore, is at least
half a mile in breadth.\textsuperscript{19} The operations of Alaric must have
been secret, prudent, and rapid; since the Roman general was
confounded by the intelligence, that the Goths, who had eluded
his efforts, were in full possession of the important province of
Epirus. This unfortunate delay allowed Alaric sufficient time to
conclude the treaty, which he secretly negotiated, with the
ministers of Constantinople. The apprehension of a civil war
compelled Stilicho to retire, at the haughty mandate of his
rivals, from the dominions of Arcadius; and he respected, in the
enemy of Rome, the honorable character of the ally and servant of
the emperor of the East.

\pagenote[16]{For Stilicho’s Greek war, compare the honest
narrative of Zosimus (l. v. p. 295, 296) with the curious
circumstantial flattery of Claudian, (i. Cons. Stilich. l. i.
172-186, iv. Cons. Hon. 459-487.) As the event was not glorious,
it is artfully thrown into the shade.}

\pagenote[17]{The troops who marched through Elis delivered up
their arms. This security enriched the Eleans, who were lovers of
a rural life. Riches begat pride: they disdained their privilege,
and they suffered. Polybius advises them to retire once more
within their magic circle. See a learned and judicious discourse
on the Olympic games, which Mr. West has prefixed to his
translation of Pindar.}

\pagenote[18]{Claudian (in iv. Cons. Hon. 480) alludes to the
fact without naming the river; perhaps the Alpheus, (i. Cons.
Stil. l. i. 185.)

—-Et Alpheus Geticis angustus acervis Tardior ad Siculos etiamnum
pergit amores.

Yet I should prefer the Peneus, a shallow stream in a wide and
deep bed, which runs through Elis, and falls into the sea below
Cyllene. It had been joined with the Alpheus to cleanse the
Augean stable. (Cellarius, tom. i. p. 760. Chandler’s Travels, p.
286.)}

\pagenote[19]{Strabo, l. viii. p. 517. Plin. Hist. Natur. iv. 3.
Wheeler, p. 308. Chandler, p. 275. They measured from different
points the distance between the two lands.}

A Grecian philosopher,\textsuperscript{20} who visited Constantinople soon after
the death of Theodosius, published his liberal opinions
concerning the duties of kings, and the state of the Roman
republic. Synesius observes, and deplores, the fatal abuse, which
the imprudent bounty of the late emperor had introduced into the
military service. The citizens and subjects had purchased an
exemption from the indispensable duty of defending their country;
which was supported by the arms of Barbarian mercenaries. The
fugitives of Scythia were permitted to disgrace the illustrious
dignities of the empire; their ferocious youth, who disdained the
salutary restraint of laws, were more anxious to acquire the
riches, than to imitate the arts, of a people, the object of
their contempt and hatred; and the power of the Goths was the
stone of Tantalus, perpetually suspended over the peace and
safety of the devoted state. The measures which Synesius
recommends, are the dictates of a bold and generous patriot. He
exhorts the emperor to revive the courage of his subjects, by the
example of manly virtue; to banish luxury from the court and from
the camp; to substitute, in the place of the Barbarian
mercenaries, an army of men, interested in the defence of their
laws and of their property; to force, in such a moment of public
danger, the mechanic from his shop, and the philosopher from his
school; to rouse the indolent citizen from his dream of pleasure,
and to arm, for the protection of agriculture, the hands of the
laborious husbandman. At the head of such troops, who might
deserve the name, and would display the spirit, of Romans, he
animates the son of Theodosius to encounter a race of Barbarians,
who were destitute of any real courage; and never to lay down his
arms, till he had chased them far away into the solitudes of
Scythia; or had reduced them to the state of ignominious
servitude, which the Lacedaemonians formerly imposed on the
captive Helots.\textsuperscript{21} The court of Arcadius indulged the zeal,
applauded the eloquence, and neglected the advice, of Synesius.
Perhaps the philosopher who addresses the emperor of the East in
the language of reason and virtue, which he might have used to a
Spartan king, had not condescended to form a practicable scheme,
consistent with the temper, and circumstances, of a degenerate
age. Perhaps the pride of the ministers, whose business was
seldom interrupted by reflection, might reject, as wild and
visionary, every proposal, which exceeded the measure of their
capacity, and deviated from the forms and precedents of office.
While the oration of Synesius, and the downfall of the
Barbarians, were the topics of popular conversation, an edict was
published at Constantinople, which declared the promotion of
Alaric to the rank of master-general of the Eastern Illyricum.
The Roman provincials, and the allies, who had respected the
faith of treaties, were justly indignant, that the ruin of Greece
and Epirus should be so liberally rewarded. The Gothic conqueror
was received as a lawful magistrate, in the cities which he had
so lately besieged. The fathers, whose sons he had massacred, the
husbands, whose wives he had violated, were subject to his
authority; and the success of his rebellion encouraged the
ambition of every leader of the foreign mercenaries. The use to
which Alaric applied his new command, distinguishes the firm and
judicious character of his policy. He issued his orders to the
four magazines and manufactures of offensive and defensive arms,
Margus, Ratiaria, Naissus, and Thessalonica, to provide his
troops with an extraordinary supply of shields, helmets, swords,
and spears; the unhappy provincials were compelled to forge the
instruments of their own destruction; and the Barbarians removed
the only defect which had sometimes disappointed the efforts of
their courage.\textsuperscript{22} The birth of Alaric, the glory of his past
exploits, and the confidence in his future designs, insensibly
united the body of the nation under his victorious standard; and,
with the unanimous consent of the Barbarian chieftains, the
master-general of Illyricum was elevated, according to ancient
custom, on a shield, and solemnly proclaimed king of the
Visigoths.\textsuperscript{23} Armed with this double power, seated on the verge
of the two empires, he alternately sold his deceitful promises to
the courts of Arcadius and Honorius; till he declared and
executed his resolution of invading the dominions of the West.
The provinces of Europe which belonged to the Eastern emperor,
were already exhausted; those of Asia were inaccessible; and the
strength of Constantinople had resisted his attack. But he was
tempted by the fame, the beauty, the wealth of Italy, which he
had twice visited; and he secretly aspired to plant the Gothic
standard on the walls of Rome, and to enrich his army with the
accumulated spoils of three hundred triumphs.\textsuperscript{25}

\pagenote[20]{Synesius passed three years (A.D. 397-400) at
Constantinople, as deputy from Cyrene to the emperor Arcadius. He
presented him with a crown of gold, and pronounced before him the
instructive oration de Regno, (p. 1-32, edit. Petav. Paris,
1612.) The philosopher was made bishop of Ptolemais, A.D. 410,
and died about 430. See Tillemont, Mem. Eccles. tom. xii. p. 490,
554, 683-685.}

\pagenote[21]{Synesius de Regno, p. 21-26.}

\pagenote[22]{—qui foedera rumpit

Ditatur: qui servat, eget: vastator Achivae Gentis, et Epirum
nuper populatus inultam, Praesidet Illyrico: jam, quos obsedit,
amicos Ingreditur muros; illis responsa daturus, Quorum
conjugibus potitur, natosque peremit.

Claudian in Eutrop. l. ii. 212. Alaric applauds his own policy
(de Bell Getic. 533-543) in the use which he had made of this
Illyrian jurisdiction.}

\pagenote[23]{Jornandes, c. 29, p. 651. The Gothic historian
adds, with unusual spirit, Cum suis deliberans suasit suo labore
quaerere regna, quam alienis per otium subjacere.

Discors odiisque anceps civilibus orbis, Non sua vis tutata diu,
dum foedera fallax Ludit, et alternae perjuria venditat aulae.
—-Claudian de Bell. Get. 565}

\pagenote[25]{Alpibus Italiae ruptis penetrabis ad Urbem. This
authentic prediction was announced by Alaric, or at least by
Claudian, (de Bell. Getico, 547,) seven years before the event.
But as it was not accomplished within the term which has been
rashly fixed the interpreters escaped through an ambiguous
meaning.}

The scarcity of facts,\textsuperscript{26} and the uncertainty of dates,\textsuperscript{27} oppose
our attempts to describe the circumstances of the first invasion
of Italy by the arms of Alaric. His march, perhaps from
Thessalonica, through the warlike and hostile country of
Pannonia, as far as the foot of the Julian Alps; his passage of
those mountains, which were strongly guarded by troops and
intrenchments; the siege of Aquileia, and the conquest of the
provinces of Istria and Venetia, appear to have employed a
considerable time. Unless his operations were extremely cautious
and slow, the length of the interval would suggest a probable
suspicion, that the Gothic king retreated towards the banks of
the Danube; and reenforced his army with fresh swarms of
Barbarians, before he again attempted to penetrate into the heart
of Italy. Since the public and important events escape the
diligence of the historian, he may amuse himself with
contemplating, for a moment, the influence of the arms of Alaric
on the fortunes of two obscure individuals, a presbyter of
Aquileia and a husbandman of Verona. The learned Rufinus, who was
summoned by his enemies to appear before a Roman synod,\textsuperscript{28} wisely
preferred the dangers of a besieged city; and the Barbarians, who
furiously shook the walls of Aquileia, might save him from the
cruel sentence of another heretic, who, at the request of the
same bishops, was severely whipped, and condemned to perpetual
exile on a desert island.\textsuperscript{29} The old man,\textsuperscript{30} who had passed his
simple and innocent life in the neighborhood of Verona, was a
stranger to the quarrels both of kings and of bishops; his
pleasures, his desires, his knowledge, were confined within the
little circle of his paternal farm; and a staff supported his
aged steps, on the same ground where he had sported in his
infancy. Yet even this humble and rustic felicity (which Claudian
describes with so much truth and feeling) was still exposed to
the undistinguishing rage of war. His trees, his old contemporary
trees,\textsuperscript{31} must blaze in the conflagration of the whole country; a
detachment of Gothic cavalry might sweep away his cottage and his
family; and the power of Alaric could destroy this happiness,
which he was not able either to taste or to bestow. “Fame,” says
the poet, “encircling with terror her gloomy wings, proclaimed
the march of the Barbarian army, and filled Italy with
consternation:” the apprehensions of each individual were
increased in just proportion to the measure of his fortune: and
the most timid, who had already embarked their valuable effects,
meditated their escape to the Island of Sicily, or the African
coast. The public distress was aggravated by the fears and
reproaches of superstition.\textsuperscript{32} Every hour produced some horrid
tale of strange and portentous accidents; the Pagans deplored the
neglect of omens, and the interruption of sacrifices; but the
Christians still derived some comfort from the powerful
intercession of the saints and martyrs.\textsuperscript{33}

\pagenote[26]{Our best materials are 970 verses of Claudian in
the poem on the Getic war, and the beginning of that which
celebrates the sixth consulship of Honorius. Zosimus is totally
silent; and we are reduced to such scraps, or rather crumbs, as
we can pick from Orosius and the Chronicles.}

\pagenote[27]{Notwithstanding the gross errors of Jornandes, who
confounds the Italian wars of Alaric, (c. 29,) his date of the
consulship of Stilicho and Aurelian (A.D. 400) is firm and
respectable. It is certain from Claudian (Tillemont, Hist. des
Emp. tom. v. p. 804) that the battle of Polentia was fought A.D.
403; but we cannot easily fill the interval.}

\pagenote[28]{Tantum Romanae urbis judicium fugis, ut magis
obsidionem barbaricam, quam pacatoe urbis judicium velis
sustinere. Jerom, tom. ii. p. 239. Rufinus understood his own
danger; the peaceful city was inflamed by the beldam Marcella,
and the rest of Jerom’s faction.}

\pagenote[29]{Jovinian, the enemy of fasts and of celibacy, who
was persecuted and insulted by the furious Jerom, (Jortin’s
Remarks, vol. iv. p. 104, \&c.) See the original edict of
banishment in the Theodosian Code, xvi. tit. v. leg. 43.}

\pagenote[30]{This epigram (de Sene Veronensi qui suburbium
nusquam egres sus est) is one of the earliest and most pleasing
compositions of Claudian. Cowley’s imitation (Hurd’s edition,
vol. ii. p. 241) has some natural and happy strokes: but it is
much inferior to the original portrait, which is evidently drawn
from the life.}

\pagenote[31]{

Ingentem meminit parvo qui germine quercum Aequaevumque videt
consenuisse nemus.
A neighboring wood born with himself he sees, And loves his old
contemporary trees.

In this passage, Cowley is perhaps superior to his original; and
the English poet, who was a good botanist, has concealed the oaks
under a more general expression.}

\pagenote[32]{Claudian de Bell. Get. 199-266. He may seem prolix:
but fear and superstition occupied as large a space in the minds
of the Italians.}

\pagenote[33]{From the passages of Paulinus, which Baronius has
produced, (Annal. Eccles. A.D. 403, No. 51,) it is manifest that
the general alarm had pervaded all Italy, as far as Nola in
Campania, where that famous penitent had fixed his abode.}

\section{Part \thesection.}

The emperor Honorius was distinguished, above his subjects, by
the preeminence of fear, as well as of rank. The pride and luxury
in which he was educated, had not allowed him to suspect, that
there existed on the earth any power presumptuous enough to
invade the repose of the successor of Augustus. The arts of
flattery concealed the impending danger, till Alaric approached
the palace of Milan. But when the sound of war had awakened the
young emperor, instead of flying to arms with the spirit, or even
the rashness, of his age, he eagerly listened to those timid
counsellors, who proposed to convey his sacred person, and his
faithful attendants, to some secure and distant station in the
provinces of Gaul. Stilicho alone\textsuperscript{34} had courage and authority to
resist his disgraceful measure, which would have abandoned Rome
and Italy to the Barbarians; but as the troops of the palace had
been lately detached to the Rhaetian frontier, and as the
resource of new levies was slow and precarious, the general of
the West could only promise, that if the court of Milan would
maintain their ground during his absence, he would soon return
with an army equal to the encounter of the Gothic king. Without
losing a moment, (while each moment was so important to the
public safety,) Stilicho hastily embarked on the Larian Lake,
ascended the mountains of ice and snow, amidst the severity of an
Alpine winter, and suddenly repressed, by his unexpected
presence, the enemy, who had disturbed the tranquillity of
Rhaetia.\textsuperscript{35} The Barbarians, perhaps some tribes of the Alemanni,
respected the firmness of a chief, who still assumed the language
of command; and the choice which he condescended to make, of a
select number of their bravest youth, was considered as a mark of
his esteem and favor. The cohorts, who were delivered from the
neighboring foe, diligently repaired to the Imperial standard;
and Stilicho issued his orders to the most remote troops of the
West, to advance, by rapid marches, to the defence of Honorius
and of Italy. The fortresses of the Rhine were abandoned; and the
safety of Gaul was protected only by the faith of the Germans,
and the ancient terror of the Roman name. Even the legion, which
had been stationed to guard the wall of Britain against the
Caledonians of the North, was hastily recalled;\textsuperscript{36} and a numerous
body of the cavalry of the Alani was persuaded to engage in the
service of the emperor, who anxiously expected the return of his
general. The prudence and vigor of Stilicho were conspicuous on
this occasion, which revealed, at the same time, the weakness of
the falling empire. The legions of Rome, which had long since
languished in the gradual decay of discipline and courage, were
exterminated by the Gothic and civil wars; and it was found
impossible, without exhausting and exposing the provinces, to
assemble an army for the defence of Italy.

\pagenote[34]{Solus erat Stilicho, \&c., is the exclusive
commendation which Claudian bestows, (del Bell. Get. 267,)
without condescending to except the emperor. How insignificant
must Honorius have appeared in his own court.}

\pagenote[35]{The face of the country, and the hardiness of
Stilicho, are finely described, (de Bell. Get. 340-363.)}

\pagenote[36]{

Venit et extremis legio praetenta Britannis, Quae Scoto dat frena
truci. —-De Bell. Get. 416.

Yet the most rapid march from Edinburgh, or Newcastle, to Milan,
must have required a longer space of time than Claudian seems
willing to allow for the duration of the Gothic war.}

\section{Part \thesection.}

When Stilicho seemed to abandon his sovereign in the unguarded
palace of Milan, he had probably calculated the term of his
absence, the distance of the enemy, and the obstacles that might
retard their march. He principally depended on the rivers of
Italy, the Adige, the Mincius, the Oglio, and the Addua, which,
in the winter or spring, by the fall of rains, or by the melting
of the snows, are commonly swelled into broad and impetuous
torrents.\textsuperscript{37} But the season happened to be remarkably dry: and
the Goths could traverse, without impediment, the wide and stony
beds, whose centre was faintly marked by the course of a shallow
stream. The bridge and passage of the Addua were secured by a
strong detachment of the Gothic army; and as Alaric approached
the walls, or rather the suburbs, of Milan, he enjoyed the proud
satisfaction of seeing the emperor of the Romans fly before him.
Honorius, accompanied by a feeble train of statesmen and eunuchs,
hastily retreated towards the Alps, with a design of securing his
person in the city of Arles, which had often been the royal
residence of his predecessors.\textsuperscript{3711} But Honorius\textsuperscript{38} had scarcely
passed the Po, before he was overtaken by the speed of the Gothic
cavalry;\textsuperscript{39} since the urgency of the danger compelled him to seek
a temporary shelter within the fortifications of Asta, a town of
Liguria or Piemont, situate on the banks of the Tanarus.\textsuperscript{40} The
siege of an obscure place, which contained so rich a prize, and
seemed incapable of a long resistance, was instantly formed, and
indefatigably pressed, by the king of the Goths; and the bold
declaration, which the emperor might afterwards make, that his
breast had never been susceptible of fear, did not probably
obtain much credit, even in his own court.\textsuperscript{41} In the last, and
almost hopeless extremity, after the Barbarians had already
proposed the indignity of a capitulation, the Imperial captive
was suddenly relieved by the fame, the approach, and at length
the presence, of the hero, whom he had so long expected. At the
head of a chosen and intrepid vanguard, Stilicho swam the stream
of the Addua, to gain the time which he must have lost in the
attack of the bridge; the passage of the Po was an enterprise of
much less hazard and difficulty; and the successful action, in
which he cut his way through the Gothic camp under the walls of
Asta, revived the hopes, and vindicated the honor, of Rome.
Instead of grasping the fruit of his victory, the Barbarian was
gradually invested, on every side, by the troops of the West, who
successively issued through all the passes of the Alps; his
quarters were straitened; his convoys were intercepted; and the
vigilance of the Romans prepared to form a chain of
fortifications, and to besiege the lines of the besiegers. A
military council was assembled of the long-haired chiefs of the
Gothic nation; of aged warriors, whose bodies were wrapped in
furs, and whose stern countenances were marked with honorable
wounds. They weighed the glory of persisting in their attempt
against the advantage of securing their plunder; and they
recommended the prudent measure of a seasonable retreat. In this
important debate, Alaric displayed the spirit of the conqueror of
Rome; and after he had reminded his countrymen of their
achievements and of their designs, he concluded his animating
speech by the solemn and positive assurance that he was resolved
to find in Italy either a kingdom or a grave.\textsuperscript{42}

\pagenote[37]{Every traveller must recollect the face of
Lombardy, (see Fonvenelle, tom. v. p. 279,) which is often
tormented by the capricious and irregular abundance of waters.
The Austrians, before Genoa, were encamped in the dry bed of the
Polcevera. “Ne sarebbe” (says Muratori) “mai passato per mente a
que’ buoni Alemanni, che quel picciolo torrente potesse, per cosi
dire, in un instante cangiarsi in un terribil gigante.” (Annali
d’Italia, tom. xvi. p. 443, Milan, 1752, 8vo edit.)}

\pagenote[3711]{According to Le Beau and his commentator M. St.
Martin, Honorius did not attempt to fly. Settlements were offered
to the Goths in Lombardy, and they advanced from the Po towards
the Alps to take possession of them. But it was a treacherous
stratagem of Stilicho, who surprised them while they were
reposing on the faith of this treaty. Le Beau, v. x.}

\pagenote[38]{Claudian does not clearly answer our question,
Where was Honorius himself? Yet the flight is marked by the
pursuit; and my idea of the Gothic was is justified by the
Italian critics, Sigonius (tom. P, ii. p. 369, de Imp. Occident.
l. x.) and Muratori, (Annali d’Italia. tom. iv. p. 45.)}

\pagenote[39]{One of the roads may be traced in the Itineraries,
(p. 98, 288, 294, with Wesseling’s Notes.) Asta lay some miles on
the right hand.}

\pagenote[40]{Asta, or Asti, a Roman colony, is now the capital
of a pleasant country, which, in the sixteenth century, devolved
to the dukes of Savoy, (Leandro Alberti Descrizzione d’Italia, p.
382.)}

\pagenote[41]{Nec me timor impulit ullus. He might hold this
proud language the next year at Rome, five hundred miles from the
scene of danger (vi. Cons. Hon. 449.)}

\pagenote[42]{Hanc ego vel victor regno, vel morte tenebo Victus,
humum.——The speeches (de Bell. Get. 479-549) of the Gothic
Nestor, and Achilles, are strong, characteristic, adapted to the
circumstances; and possibly not less genuine than those of Livy.}

The loose discipline of the Barbarians always exposed them to the
danger of a surprise; but, instead of choosing the dissolute
hours of riot and intemperance, Stilicho resolved to attack the
Christian Goths, whilst they were devoutly employed in
celebrating the festival of Easter.\textsuperscript{43} The execution of the
stratagem, or, as it was termed by the clergy of the sacrilege,
was intrusted to Saul, a Barbarian and a Pagan, who had served,
however, with distinguished reputation among the veteran generals
of Theodosius. The camp of the Goths, which Alaric had pitched in
the neighborhood of Pollentia,\textsuperscript{44} was thrown into confusion by
the sudden and impetuous charge of the Imperial cavalry; but, in
a few moments, the undaunted genius of their leader gave them an
order, and a field of battle; and, as soon as they had recovered
from their astonishment, the pious confidence, that the God of
the Christians would assert their cause, added new strength to
their native valor. In this engagement, which was long maintained
with equal courage and success, the chief of the Alani, whose
diminutive and savage form concealed a magnanimous soul approved
his suspected loyalty, by the zeal with which he fought, and
fell, in the service of the republic; and the fame of this
gallant Barbarian has been imperfectly preserved in the verses of
Claudian, since the poet, who celebrates his virtue, has omitted
the mention of his name. His death was followed by the flight and
dismay of the squadrons which he commanded; and the defeat of the
wing of cavalry might have decided the victory of Alaric, if
Stilicho had not immediately led the Roman and Barbarian infantry
to the attack. The skill of the general, and the bravery of the
soldiers, surmounted every obstacle. In the evening of the bloody
day, the Goths retreated from the field of battle; the
intrenchments of their camp were forced, and the scene of rapine
and slaughter made some atonement for the calamities which they
had inflicted on the subjects of the empire.\textsuperscript{45} The magnificent
spoils of Corinth and Argos enriched the veterans of the West;
the captive wife of Alaric, who had impatiently claimed his
promise of Roman jewels and Patrician handmaids,\textsuperscript{46} was reduced
to implore the mercy of the insulting foe; and many thousand
prisoners, released from the Gothic chains, dispersed through the
provinces of Italy the praises of their heroic deliverer. The
triumph of Stilicho\textsuperscript{47} was compared by the poet, and perhaps by
the public, to that of Marius; who, in the same part of Italy,
had encountered and destroyed another army of Northern
Barbarians. The huge bones, and the empty helmets, of the Cimbri
and of the Goths, would easily be confounded by succeeding
generations; and posterity might erect a common trophy to the
memory of the two most illustrious generals, who had vanquished,
on the same memorable ground, the two most formidable enemies of
Rome.\textsuperscript{48}

\pagenote[43]{Orosius (l. vii. c. 37) is shocked at the impiety
of the Romans, who attacked, on Easter Sunday, such pious
Christians. Yet, at the same time, public prayers were offered at
the shrine of St. Thomas of Edessa, for the destruction of the
Arian robber. See Tillemont (Hist des Emp. tom. v. p. 529) who
quotes a homily, which has been erroneously ascribed to St.
Chrysostom.}

\pagenote[44]{The vestiges of Pollentia are twenty-five miles to
the south-east of Turin. Urbs, in the same neighborhood, was a
royal chase of the kings of Lombardy, and a small river, which
excused the prediction, “penetrabis ad urbem,” (Cluver. Ital.
Antiq tom. i. p. 83-85.)}

\pagenote[45]{Orosius wishes, in doubtful words, to insinuate the
defeat of the Romans. “Pugnantes vicimus, victores victi sumus.”
Prosper (in Chron.) makes it an equal and bloody battle, but the
Gothic writers Cassiodorus (in Chron.) and Jornandes (de Reb.
Get. c. 29) claim a decisive victory.}

\pagenote[46]{Demens Ausonidum gemmata monilia matrum, Romanasque
alta famulas cervice petebat. De Bell. Get. 627.}

\pagenote[47]{Claudian (de Bell. Get. 580-647) and Prudentius (in
Symmach. n. 694-719) celebrate, without ambiguity, the Roman
victory of Pollentia. They are poetical and party writers; yet
some credit is due to the most suspicious witnesses, who are
checked by the recent notoriety of facts.}

\pagenote[48]{Claudian’s peroration is strong and elegant; but
the identity of the Cimbric and Gothic fields must be understood
(like Virgil’s Philippi, Georgic i. 490) according to the loose
geography of a poet. Verselle and Pollentia are sixty miles from
each other; and the latitude is still greater, if the Cimbri were
defeated in the wide and barren plain of Verona, (Maffei, Verona
Illustrata, P. i. p. 54-62.)}

The eloquence of Claudian\textsuperscript{49} has celebrated, with lavish
applause, the victory of Pollentia, one of the most glorious days
in the life of his patron; but his reluctant and partial muse
bestows more genuine praise on the character of the Gothic king.
His name is, indeed, branded with the reproachful epithets of
pirate and robber, to which the conquerors of every age are so
justly entitled; but the poet of Stilicho is compelled to
acknowledge that Alaric possessed the invincible temper of mind,
which rises superior to every misfortune, and derives new
resources from adversity. After the total defeat of his infantry,
he escaped, or rather withdrew, from the field of battle, with
the greatest part of his cavalry entire and unbroken. Without
wasting a moment to lament the irreparable loss of so many brave
companions, he left his victorious enemy to bind in chains the
captive images of a Gothic king;\textsuperscript{50} and boldly resolved to break
through the unguarded passes of the Apennine, to spread
desolation over the fruitful face of Tuscany, and to conquer or
die before the gates of Rome. The capital was saved by the active
and incessant diligence of Stilicho; but he respected the despair
of his enemy; and, instead of committing the fate of the republic
to the chance of another battle, he proposed to purchase the
absence of the Barbarians. The spirit of Alaric would have
rejected such terms, the permission of a retreat, and the offer
of a pension, with contempt and indignation; but he exercised a
limited and precarious authority over the independent chieftains
who had raised him, for their service, above the rank of his
equals; they were still less disposed to follow an unsuccessful
general, and many of them were tempted to consult their interest
by a private negotiation with the minister of Honorius. The king
submitted to the voice of his people, ratified the treaty with
the empire of the West, and repassed the Po with the remains of
the flourishing army which he had led into Italy. A considerable
part of the Roman forces still continued to attend his motions;
and Stilicho, who maintained a secret correspondence with some of
the Barbarian chiefs, was punctually apprised of the designs that
were formed in the camp and council of Alaric. The king of the
Goths, ambitious to signalize his retreat by some splendid
achievement, had resolved to occupy the important city of Verona,
which commands the principal passage of the Rhaetian Alps; and,
directing his march through the territories of those German
tribes, whose alliance would restore his exhausted strength, to
invade, on the side of the Rhine, the wealthy and unsuspecting
provinces of Gaul. Ignorant of the treason which had already
betrayed his bold and judicious enterprise, he advanced towards
the passes of the mountains, already possessed by the Imperial
troops; where he was exposed, almost at the same instant, to a
general attack in the front, on his flanks, and in the rear. In
this bloody action, at a small distance from the walls of Verona,
the loss of the Goths was not less heavy than that which they had
sustained in the defeat of Pollentia; and their valiant king, who
escaped by the swiftness of his horse, must either have been
slain or made prisoner, if the hasty rashness of the Alani had
not disappointed the measures of the Roman general. Alaric
secured the remains of his army on the adjacent rocks; and
prepared himself, with undaunted resolution, to maintain a siege
against the superior numbers of the enemy, who invested him on
all sides. But he could not oppose the destructive progress of
hunger and disease; nor was it possible for him to check the
continual desertion of his impatient and capricious Barbarians.
In this extremity he still found resources in his own courage, or
in the moderation of his adversary; and the retreat of the Gothic
king was considered as the deliverance of Italy.\textsuperscript{51} Yet the
people, and even the clergy, incapable of forming any rational
judgment of the business of peace and war, presumed to arraign
the policy of Stilicho, who so often vanquished, so often
surrounded, and so often dismissed the implacable enemy of the
republic. The first momen of the public safety is devoted to
gratitude and joy; but the second is diligently occupied by envy
and calumny.\textsuperscript{52}

\pagenote[49]{Claudian and Prudentius must be strictly examined,
to reduce the figures, and extort the historic sense, of those
poets.}

\pagenote[50]{

Et gravant en airain ses freles avantages De mes etats conquis
enchainer les images.

The practice of exposing in triumph the images of kings and
provinces was familiar to the Romans. The bust of Mithridates
himself was twelve feet high, of massy gold, (Freinshem.
Supplement. Livian. ciii. 47.)}

\pagenote[51]{The Getic war, and the sixth consulship of
Honorius, obscurely connect the events of Alaric’s retreat and
losses.}

\pagenote[52]{Taceo de Alarico... saepe visto, saepe concluso,
semperque dimisso. Orosius, l. vii. c. 37, p. 567. Claudian (vi.
Cons. Hon. 320) drops the curtain with a fine image.}

The citizens of Rome had been astonished by the approach of
Alaric; and the diligence with which they labored to restore the
walls of the capital, confessed their own fears, and the decline
of the empire. After the retreat of the Barbarians, Honorius was
directed to accept the dutiful invitation of the senate, and to
celebrate, in the Imperial city, the auspicious era of the
Gothic victory, and of his sixth consulship.\textsuperscript{53} The suburbs and
the streets, from the Milvian bridge to the Palatine mount, were
filled by the Roman people, who, in the space of a hundred years,
had only thrice been honored with the presence of their
sovereigns. While their eyes were fixed on the chariot where
Stilicho was deservedly seated by the side of his royal pupil,
they applauded the pomp of a triumph, which was not stained, like
that of Constantine, or of Theodosius, with civil blood. The
procession passed under a lofty arch, which had been purposely
erected: but in less than seven years, the Gothic conquerors of
Rome might read, if they were able to read, the superb
inscription of that monument, which attested the total defeat and
destruction of their nation.\textsuperscript{54} The emperor resided several
months in the capital, and every part of his behavior was
regulated with care to conciliate the affection of the clergy,
the senate, and the people of Rome. The clergy was edified by his
frequent visits and liberal gifts to the shrines of the apostles.
The senate, who, in the triumphal procession, had been excused
from the humiliating ceremony of preceding on foot the Imperial
chariot, was treated with the decent reverence which Stilicho
always affected for that assembly. The people was repeatedly
gratified by the attention and courtesy of Honorius in the public
games, which were celebrated on that occasion with a magnificence
not unworthy of the spectator. As soon as the appointed number of
chariot-races was concluded, the decoration of the Circus was
suddenly changed; the hunting of wild beasts afforded a various
and splendid entertainment; and the chase was succeeded by a
military dance, which seems, in the lively description of
Claudian, to present the image of a modern tournament.

\pagenote[53]{The remainder of Claudian’s poem on the sixth
consulship of Honorius, describes the journey, the triumph, and
the games, (330-660.)}

\pagenote[54]{See the inscription in Mascou’s History of the
Ancient Germans, viii. 12. The words are positive and indiscreet:
Getarum nationem in omne aevum domitam, \&c.}

In these games of Honorius, the inhuman combats of gladiators\textsuperscript{55}
polluted, for the last time, the amphitheater of Rome. The first
Christian emperor may claim the honor of the first edict which
condemned the art and amusement of shedding human blood;\textsuperscript{56} but
this benevolent law expressed the wishes of the prince, without
reforming an inveterate abuse, which degraded a civilized nation
below the condition of savage cannibals. Several hundred, perhaps
several thousand, victims were annually slaughtered in the great
cities of the empire; and the month of December, more peculiarly
devoted to the combats of gladiators, still exhibited to the eyes
of the Roman people a grateful spectacle of blood and cruelty.
Amidst the general joy of the victory of Pollentia, a Christian
poet exhorted the emperor to extirpate, by his authority, the
horrid custom which had so long resisted the voice of humanity
and religion.\textsuperscript{57} The pathetic representations of Prudentius were
less effectual than the generous boldness of Telemachus, an
Asiatic monk, whose death was more useful to mankind than his
life.\textsuperscript{58} The Romans were provoked by the interruption of their
pleasures; and the rash monk, who had descended into the arena to
separate the gladiators, was overwhelmed under a shower of
stones. But the madness of the people soon subsided; they
respected the memory of Telemachus, who had deserved the honors
of martyrdom; and they submitted, without a murmur, to the laws
of Honorius, which abolished forever the human sacrifices of the
amphitheater.\textsuperscript{5811} The citizens, who adhered to the manners of
their ancestors, might perhaps insinuate that the last remains of
a martial spirit were preserved in this school of fortitude,
which accustomed the Romans to the sight of blood, and to the
contempt of death; a vain and cruel prejudice, so nobly confuted
by the valor of ancient Greece, and of modern Europe!\textsuperscript{59}

\pagenote[55]{On the curious, though horrid, subject of the
gladiators, consult the two books of the Saturnalia of Lipsius,
who, as an antiquarian, is inclined to excuse the practice of
antiquity, (tom. iii. p. 483-545.)}

\pagenote[56]{Cod. Theodos. l. xv. tit. xii. leg. i. The
Commentary of Godefroy affords large materials (tom. v. p. 396)
for the history of gladiators.}

\pagenote[57]{See the peroration of Prudentius (in Symmach. l.
ii. 1121-1131) who had doubtless read the eloquent invective of
Lactantius, (Divin. Institut. l. vi. c. 20.) The Christian
apologists have not spared these bloody games, which were
introduced in the religious festivals of Paganism.}

\pagenote[58]{Theodoret, l. v. c. 26. I wish to believe the story
of St. Telemachus. Yet no church has been dedicated, no altar has
been erected, to the only monk who died a martyr in the cause of
humanity.}

\pagenote[5811]{Muller, in his valuable Treatise, de Genio,
moribus et luxu aevi Theodosiani, is disposed to question the
effect produced by the heroic, or rather saintly, death of
Telemachus. No prohibitory law of Honorius is to be found in the
Theodosian Code, only the old and imperfect edict of Constantine.
But Muller has produced no evidence or allusion to gladiatorial
shows after this period. The combats with wild beasts certainly
lasted till the fall of the Western empire; but the gladiatorial
combats ceased either by common consent, or by Imperial
edict.—M.}

\pagenote[59]{Crudele gladiatorum spectaculum et inhumanum
nonnullis videri solet, et haud scio an ita sit, ut nunc fit.
Cicero Tusculan. ii. 17. He faintly censures the abuse, and
warmly defends the use, of these sports; oculis nulla poterat
esse fortior contra dolorem et mortem disciplina. Seneca (epist.
vii.) shows the feelings of a man.}

The recent danger, to which the person of the emperor had been
exposed in the defenceless palace of Milan, urged him to seek a
retreat in some inaccessible fortress of Italy, where he might
securely remain, while the open country was covered by a deluge
of Barbarians. On the coast of the Adriatic, about ten or twelve
miles from the most southern of the seven mouths of the Po, the
Thessalians had founded the ancient colony of Ravenna,\textsuperscript{60} which
they afterwards resigned to the natives of Umbria. Augustus, who
had observed the opportunity of the place, prepared, at the
distance of three miles from the old town, a capacious harbor,
for the reception of two hundred and fifty ships of war. This
naval establishment, which included the arsenals and magazines,
the barracks of the troops, and the houses of the artificers,
derived its origin and name from the permanent station of the
Roman fleet; the intermediate space was soon filled with
buildings and inhabitants, and the three extensive and populous
quarters of Ravenna gradually contributed to form one of the most
important cities of Italy. The principal canal of Augustus poured
a copious stream of the waters of the Po through the midst of the
city, to the entrance of the harbor; the same waters were
introduced into the profound ditches that encompassed the walls;
they were distributed by a thousand subordinate canals, into
every part of the city, which they divided into a variety of
small islands; the communication was maintained only by the use
of boats and bridges; and the houses of Ravenna, whose appearance
may be compared to that of Venice, were raised on the foundation
of wooden piles. The adjacent country, to the distance of many
miles, was a deep and impassable morass; and the artificial
causeway, which connected Ravenna with the continent, might be
easily guarded or destroyed, on the approach of a hostile army
These morasses were interspersed, however, with vineyards: and
though the soil was exhausted by four or five crops, the town
enjoyed a more plentiful supply of wine than of fresh water.\textsuperscript{61}
The air, instead of receiving the sickly, and almost
pestilential, exhalations of low and marshy grounds, was
distinguished, like the neighborhood of Alexandria, as uncommonly
pure and salubrious; and this singular advantage was ascribed to
the regular tides of the Adriatic, which swept the canals,
interrupted the unwholesome stagnation of the waters, and
floated, every day, the vessels of the adjacent country into the
heart of Ravenna. The gradual retreat of the sea has left the
modern city at the distance of four miles from the Adriatic; and
as early as the fifth or sixth century of the Christian era, the
port of Augustus was converted into pleasant orchards; and a
lonely grove of pines covered the ground where the Roman fleet
once rode at anchor.\textsuperscript{62} Even this alteration contributed to
increase the natural strength of the place, and the shallowness
of the water was a sufficient barrier against the large ships of
the enemy. This advantageous situation was fortified by art and
labor; and in the twentieth year of his age, the emperor of the
West, anxious only for his personal safety, retired to the
perpetual confinement of the walls and morasses of Ravenna. The
example of Honorius was imitated by his feeble successors, the
Gothic kings, and afterwards the Exarchs, who occupied the throne
and palace of the emperors; and till the middle of the eight
century, Ravenna was considered as the seat of government, and
the capital of Italy.\textsuperscript{63}

\pagenote[60]{This account of Ravenna is drawn from Strabo, (l.
v. p. 327,) Pliny, (iii. 20,) Stephen of Byzantium, (sub voce, p.
651, edit. Berkel,) Claudian, (in vi. Cons. Honor. 494, \&c.,)
Sidonius Apollinaris, (l. i. epist. 5, 8,) Jornandes, (de Reb.
Get. c. 29,) Procopius (de Bell, (lothic, l. i. c. i. p. 309,
edit. Louvre,) and Cluverius, (Ital. Antiq tom i. p. 301-307.)
Yet I still want a local antiquarian and a good topographical
map.}

\pagenote[61]{Martial (Epigram iii. 56, 57) plays on the trick of
the knave, who had sold him wine instead of water; but he
seriously declares that a cistern at Ravenna is more valuable
than a vineyard. Sidonius complains that the town is destitute of
fountains and aqueducts; and ranks the want of fresh water among
the local evils, such as the croaking of frogs, the stinging of
gnats, \&c.}

\pagenote[62]{The fable of Theodore and Honoria, which Dryden has
so admirably transplanted from Boccaccio, (Giornata iii. novell.
viii.,) was acted in the wood of Chiassi, a corrupt word from
Classis, the naval station which, with the intermediate road, or
suburb the Via Caesaris, constituted the triple city of Ravenna.}

\pagenote[63]{From the year 404, the dates of the Theodosian Code
become sedentary at Constantinople and Ravenna. See Godefroy’s
Chronology of the Laws, tom. i. p. cxlviii., \&c.}

The fears of Honorius were not without foundation, nor were his
precautions without effect. While Italy rejoiced in her
deliverance from the Goths, a furious tempest was excited among
the nations of Germany, who yielded to the irresistible impulse
that appears to have been gradually communicated from the eastern
extremity of the continent of Asia. The Chinese annals, as they
have been interpreted by the learned industry of the present age,
may be usefully applied to reveal the secret and remote causes of
the fall of the Roman empire. The extensive territory to the
north of the great wall was possessed, after the flight of the
Huns, by the victorious Sienpi, who were sometimes broken into
independent tribes, and sometimes reunited under a supreme chief;
till at length, styling themselves Topa, or masters of the earth,
they acquired a more solid consistence, and a more formidable
power. The Topa soon compelled the pastoral nations of the
eastern desert to acknowledge the superiority of their arms; they
invaded China in a period of weakness and intestine discord; and
these fortunate Tartars, adopting the laws and manners of the
vanquished people, founded an Imperial dynasty, which reigned
near one hundred and sixty years over the northern provinces of
the monarchy. Some generations before they ascended the throne of
China, one of the Topa princes had enlisted in his cavalry a
slave of the name of Moko, renowned for his valor, but who was
tempted, by the fear of punishment, to desert his standard, and
to range the desert at the head of a hundred followers. This gang
of robbers and outlaws swelled into a camp, a tribe, a numerous
people, distinguished by the appellation of Geougen; and their
hereditary chieftains, the posterity of Moko the slave, assumed
their rank among the Scythian monarchs. The youth of Toulun, the
greatest of his descendants, was exercised by those misfortunes
which are the school of heroes. He bravely struggled with
adversity, broke the imperious yoke of the Topa, and became the
legislator of his nation, and the conqueror of Tartary. His
troops were distributed into regular bands of a hundred and of a
thousand men; cowards were stoned to death; the most splendid
honors were proposed as the reward of valor; and Toulun, who had
knowledge enough to despise the learning of China, adopted only
such arts and institutions as were favorable to the military
spirit of his government. His tents, which he removed in the
winter season to a more southern latitude, were pitched, during
the summer, on the fruitful banks of the Selinga. His conquests
stretched from Corea far beyond the River Irtish. He vanquished,
in the country to the north of the Caspian Sea, the nation of the
Huns; and the new title of Khan, or Cagan, expressed the fame and
power which he derived from this memorable victory.\textsuperscript{64}

\pagenote[64]{See M. de Guignes, Hist. des Huns, tom. i. p.
179-189, tom ii p. 295, 334-338.}

The chain of events is interrupted, or rather is concealed, as it
passes from the Volga to the Vistula, through the dark interval
which separates the extreme limits of the Chinese, and of the
Roman, geography. Yet the temper of the Barbarians, and the
experience of successive emigrations, sufficiently declare, that
the Huns, who were oppressed by the arms of the Geougen, soon
withdrew from the presence of an insulting victor. The countries
towards the Euxine were already occupied by their kindred tribes;
and their hasty flight, which they soon converted into a bold
attack, would more naturally be directed towards the rich and
level plains, through which the Vistula gently flows into the
Baltic Sea. The North must again have been alarmed, and agitated,
by the invasion of the Huns;\textsuperscript{6411} and the nations who retreated
before them must have pressed with incumbent weight on the
confines of Germany.\textsuperscript{65} The inhabitants of those regions, which
the ancients have assigned to the Suevi, the Vandals, and the
Burgundians, might embrace the resolution of abandoning to the
fugitives of Sarmatia their woods and morasses; or at least of
discharging their superfluous numbers on the provinces of the
Roman empire.\textsuperscript{66} About four years after the victorious Toulun had
assumed the title of Khan of the Geougen, another Barbarian, the
haughty Rhodogast, or Radagaisus,\textsuperscript{67} marched from the northern
extremities of Germany almost to the gates of Rome, and left the
remains of his army to achieve the destruction of the West. The
Vandals, the Suevi, and the Burgundians, formed the strength of
this mighty host; but the Alani, who had found a hospitable
reception in their new seats, added their active cavalry to the
heavy infantry of the Germans; and the Gothic adventurers crowded
so eagerly to the standard of Radagaisus, that by some
historians, he has been styled the King of the Goths. Twelve
thousand warriors, distinguished above the vulgar by their noble
birth, or their valiant deeds, glittered in the van;\textsuperscript{68} and the
whole multitude, which was not less than two hundred thousand
fighting men, might be increased, by the accession of women, of
children, and of slaves, to the amount of four hundred thousand
persons. This formidable emigration issued from the same coast of
the Baltic, which had poured forth the myriads of the Cimbri and
Teutones, to assault Rome and Italy in the vigor of the republic.
After the departure of those Barbarians, their native country,
which was marked by the vestiges of their greatness, long
ramparts, and gigantic moles,\textsuperscript{69} remained, during some ages, a
vast and dreary solitude; till the human species was renewed by
the powers of generation, and the vacancy was filled by the
influx of new inhabitants. The nations who now usurp an extent of
land which they are unable to cultivate, would soon be assisted
by the industrious poverty of their neighbors, if the government
of Europe did not protect the claims of dominion and property.

\pagenote[6411]{There is no authority which connects this inroad
of the Teutonic tribes with the movements of the Huns. The Huns
can hardly have reached the shores of the Baltic, and probably
the greater part of the forces of Radagaisus, particularly the
Vandals, had long occupied a more southern position.—M.}

\pagenote[65]{Procopius (de Bell. Vandal. l. i. c. iii. p. 182)
has observed an emigration from the Palus Maeotis to the north of
Germany, which he ascribes to famine. But his views of ancient
history are strangely darkened by ignorance and error.}

\pagenote[66]{Zosimus (l. v. p. 331) uses the general description
of the nations beyond the Danube and the Rhine. Their situation,
and consequently their names, are manifestly shown, even in the
various epithets which each ancient writer may have casually
added.}

\pagenote[67]{The name of Rhadagast was that of a local deity of
the Obotrites, (in Mecklenburg.) A hero might naturally assume
the appellation of his tutelar god; but it is not probable that
the Barbarians should worship an unsuccessful hero. See Mascou,
Hist. of the Germans, viii. 14. * Note: The god of war and of
hospitality with the Vends and all the Sclavonian races of
Germany bore the name of Radegast, apparently the same with
Rhadagaisus. His principal temple was at Rhetra in Mecklenburg.
It was adorned with great magnificence. The statue of the gold
was of gold. St. Martin, v. 255. A statue of Radegast, of much
coarser materials, and of the rudest workmanship, was discovered
between 1760 and 1770, with those of other Wendish deities, on
the supposed site of Rhetra. The names of the gods were cut upon
them in Runic characters. See the very curious volume on these
antiquities—Die Gottesdienstliche Alterthumer der Obotriter—Masch
and Wogen. Berlin, 1771.—M.}

\pagenote[68]{Olympiodorus (apud Photium, p. 180), uses the Greek
word which does not convey any precise idea. I suspect that they
were the princes and nobles with their faithful companions; the
knights with their squires, as they would have been styled some
centuries afterwards.}

\pagenote[69]{Tacit. de Moribus Germanorum, c. 37.}

\section{Part \thesection.}

The correspondence of nations was, in that age, so imperfect and
precarious, that the revolutions of the North might escape the
knowledge of the court of Ravenna; till the dark cloud, which was
collected along the coast of the Baltic, burst in thunder upon
the banks of the Upper Danube. The emperor of the West, if his
ministers disturbed his amusements by the news of the impending
danger, was satisfied with being the occasion, and the spectator,
of the war.\textsuperscript{70} The safety of Rome was intrusted to the counsels,
and the sword, of Stilicho; but such was the feeble and exhausted
state of the empire, that it was impossible to restore the
fortifications of the Danube, or to prevent, by a vigorous
effort, the invasion of the Germans.\textsuperscript{71} The hopes of the vigilant
minister of Honorius were confined to the defence of Italy. He
once more abandoned the provinces, recalled the troops, pressed
the new levies, which were rigorously exacted, and
pusillanimously eluded; employed the most efficacious means to
arrest, or allure, the deserters; and offered the gift of
freedom, and of two pieces of gold, to all the slaves who would
enlist.\textsuperscript{72} By these efforts he painfully collected, from the
subjects of a great empire, an army of thirty or forty thousand
men, which, in the days of Scipio or Camillus, would have been
instantly furnished by the free citizens of the territory of
Rome.\textsuperscript{73} The thirty legions of Stilicho were reenforced by a
large body of Barbarian auxiliaries; the faithful Alani were
personally attached to his service; and the troops of Huns and of
Goths, who marched under the banners of their native princes,
Huldin and Sarus, were animated by interest and resentment to
oppose the ambition of Radagaisus. The king of the confederate
Germans passed, without resistance, the Alps, the Po, and the
Apennine; leaving on one hand the inaccessible palace of
Honorius, securely buried among the marshes of Ravenna; and, on
the other, the camp of Stilicho, who had fixed his head-quarters
at Ticinum, or Pavia, but who seems to have avoided a decisive
battle, till he had assembled his distant forces. Many cities of
Italy were pillaged, or destroyed; and the siege of Florence,\textsuperscript{74}
by Radagaisus, is one of the earliest events in the history of
that celebrated republic; whose firmness checked and delayed the
unskillful fury of the Barbarians. The senate and people trembled
at their approach within a hundred and eighty miles of Rome; and
anxiously compared the danger which they had escaped, with the
new perils to which they were exposed. Alaric was a Christian and
a soldier, the leader of a disciplined army; who understood the
laws of war, who respected the sanctity of treaties, and who had
familiarly conversed with the subjects of the empire in the same
camps, and the same churches. The savage Radagaisus was a
stranger to the manners, the religion, and even the language, of
the civilized nations of the South. The fierceness of his temper
was exasperated by cruel superstition; and it was universally
believed, that he had bound himself, by a solemn vow, to reduce
the city into a heap of stones and ashes, and to sacrifice the
most illustrious of the Roman senators on the altars of those
gods who were appeased by human blood. The public danger, which
should have reconciled all domestic animosities, displayed the
incurable madness of religious faction. The oppressed votaries of
Jupiter and Mercury respected, in the implacable enemy of Rome,
the character of a devout Pagan; loudly declared, that they were
more apprehensive of the sacrifices, than of the arms, of
Radagaisus; and secretly rejoiced in the calamities of their
country, which condemned the faith of their Christian
adversaries.\textsuperscript{75} \textsuperscript{7511}

\pagenote[70]{

Cujus agendi Spectator vel causa fui, —-(Claudian, vi. Cons. Hon.
439,)

is the modest language of Honorius, in speaking of the Gothic
war, which he had seen somewhat nearer.}

\pagenote[71]{Zosimus (l. v. p. 331) transports the war, and the
victory of Stilisho, beyond the Danube. A strange error, which is
awkwardly and imperfectly cured (Tillemont, Hist. des Emp. tom.
v. p. 807.) In good policy, we must use the service of Zosimus,
without esteeming or trusting him.}

\pagenote[72]{Codex Theodos. l. vii. tit. xiii. leg. 16. The date
of this law A.D. 406. May 18 satisfies me, as it had done
Godefroy, (tom. ii. p. 387,) of the true year of the invasion of
Radagaisus. Tillemont, Pagi, and Muratori, prefer the preceding
year; but they are bound, by certain obligations of civility and
respect, to St. Paulinus of Nola.}

\pagenote[73]{Soon after Rome had been taken by the Gauls, the
senate, on a sudden emergency, armed ten legions, 3000 horse, and
42,000 foot; a force which the city could not have sent forth
under Augustus, (Livy, xi. 25.) This declaration may puzzle an
antiquary, but it is clearly explained by Montesquieu.}

\pagenote[74]{Machiavel has explained, at least as a philosopher,
the origin of Florence, which insensibly descended, for the
benefit of trade, from the rock of Faesulae to the banks of the
Arno, (Istoria Fiorentina, tom. i. p. 36. Londra, 1747.) The
triumvirs sent a colony to Florence, which, under Tiberius,
(Tacit. Annal. i. 79,) deserved the reputation and name of a
flourishing city. See Cluver. Ital. Antiq. tom. i. p. 507, \&c.}

\pagenote[75]{Yet the Jupiter of Radagaisus, who worshipped Thor
and Woden, was very different from the Olympic or Capitoline
Jove. The accommodating temper of Polytheism might unite those
various and remote deities; but the genuine Romans ahhorred the
human sacrifices of Gaul and Germany.}

\pagenote[7511]{Gibbon has rather softened the language of
Augustine as to this threatened insurrection of the Pagans, in
order to restore the prohibited rites and ceremonies of Paganism;
and their treasonable hopes that the success of Radagaisus would
be the triumph of idolatry. Compare ii. 25—M.}

Florence was reduced to the last extremity; and the fainting
courage of the citizens was supported only by the authority of
St. Ambrose; who had communicated, in a dream, the promise of a
speedy deliverance.\textsuperscript{76} On a sudden, they beheld, from their
walls, the banners of Stilicho, who advanced, with his united
force, to the relief of the faithful city; and who soon marked
that fatal spot for the grave of the Barbarian host. The apparent
contradictions of those writers who variously relate the defeat
of Radagaisus, may be reconciled without offering much violence
to their respective testimonies. Orosius and Augustin, who were
intimately connected by friendship and religion, ascribed this
miraculous victory to the providence of God, rather than to the
valor of man.\textsuperscript{77} They strictly exclude every idea of chance, or
even of bloodshed; and positively affirm, that the Romans, whose
camp was the scene of plenty and idleness, enjoyed the distress
of the Barbarians, slowly expiring on the sharp and barren ridge
of the hills of Faesulae, which rise above the city of Florence.
Their extravagant assertion that not a single soldier of the
Christian army was killed, or even wounded, may be dismissed with
silent contempt; but the rest of the narrative of Augustin and
Orosius is consistent with the state of the war, and the
character of Stilicho. Conscious that he commanded the last army
of the republic, his prudence would not expose it, in the open
field, to the headstrong fury of the Germans. The method of
surrounding the enemy with strong lines of circumvallation, which
he had twice employed against the Gothic king, was repeated on a
larger scale, and with more considerable effect. The examples of
Caesar must have been familiar to the most illiterate of the
Roman warriors; and the fortifications of Dyrrachium, which
connected twenty-four castles, by a perpetual ditch and rampart
of fifteen miles, afforded the model of an intrenchment which
might confine, and starve, the most numerous host of Barbarians.\textsuperscript{78}
The Roman troops had less degenerated from the industry, than
from the valor, of their ancestors; and if their servile and
laborious work offended the pride of the soldiers, Tuscany could
supply many thousand peasants, who would labor, though, perhaps,
they would not fight, for the salvation of their native country.
The imprisoned multitude of horses and men\textsuperscript{79} was gradually
destroyed, by famine rather than by the sword; but the Romans
were exposed, during the progress of such an extensive work, to
the frequent attacks of an impatient enemy. The despair of the
hungry Barbarians would precipitate them against the
fortifications of Stilicho; the general might sometimes indulge
the ardor of his brave auxiliaries, who eagerly pressed to
assault the camp of the Germans; and these various incidents
might produce the sharp and bloody conflicts which dignify the
narrative of Zosimus, and the Chronicles of Prosper and
Marcellinus.\textsuperscript{80} A seasonable supply of men and provisions had
been introduced into the walls of Florence, and the famished host
of Radagaisus was in its turn besieged. The proud monarch of so
many warlike nations, after the loss of his bravest warriors, was
reduced to confide either in the faith of a capitulation, or in
the clemency of Stilicho.\textsuperscript{81} But the death of the royal captive,
who was ignominiously beheaded, disgraced the triumph of Rome and
of Christianity; and the short delay of his execution was
sufficient to brand the conqueror with the guilt of cool and
deliberate cruelty.\textsuperscript{82} The famished Germans, who escaped the fury
of the auxiliaries, were sold as slaves, at the contemptible
price of as many single pieces of gold; but the difference of
food and climate swept away great numbers of those unhappy
strangers; and it was observed, that the inhuman purchasers,
instead of reaping the fruits of their labor were soon obliged to
provide the expense of their interment. Stilicho informed the
emperor and the senate of his success; and deserved, a second
time, the glorious title of Deliverer of Italy.\textsuperscript{83}

\pagenote[76]{Paulinus (in Vit. Ambros c. 50) relates this story,
which he received from the mouth of Pansophia herself, a
religious matron of Florence. Yet the archbishop soon ceased to
take an active part in the business of the world, and never
became a popular saint.}

\pagenote[77]{Augustin de Civitat. Dei, v. 23. Orosius, l. vii.
c. 37, p. 567-571. The two friends wrote in Africa, ten or twelve
years after the victory; and their authority is implicitly
followed by Isidore of Seville, (in Chron. p. 713, edit. Grot.)
How many interesting facts might Orosius have inserted in the
vacant space which is devoted to pious nonsense!}

\pagenote[78]{

Franguntur montes, planumque per ardua Caesar Ducit opus: pandit
fossas, turritaque summis Disponit castella jugis, magnoque
necessu Amplexus fines, saltus, memorosaque tesqua Et silvas,
vastaque feras indagine claudit.!

Yet the simplicity of truth (Caesar, de Bell. Civ. iii. 44) is
far greater than the amplifications of Lucan, (Pharsal. l. vi.
29-63.)}

\pagenote[79]{The rhetorical expressions of Orosius, “in arido et
aspero montis jugo;” “in unum ac parvum verticem,” are not very
suitable to the encampment of a great army. But Faesulae, only
three miles from Florence, might afford space for the
head-quarters of Radagaisus, and would be comprehended within the
circuit of the Roman lines.}

\pagenote[80]{See Zosimus, l. v. p. 331, and the Chronicles of
Prosper and Marcellinus.}

\pagenote[81]{Olympiodorus (apud Photium, p. 180) uses an
expression which would denote a strict and friendly alliance, and
render Stilicho still more criminal. The paulisper detentus,
deinde interfectus, of Orosius, is sufficiently odious. * Note:
Gibbon, by translating this passage of Olympiodorus, as if it had
been good Greek, has probably fallen into an error. The natural
order of the words is as Gibbon translates it; but it is almost
clear, refers to the Gothic chiefs, “whom Stilicho, after he had
defeated Radagaisus, attached to his army.” So in the version
corrected by Classen for Niebuhr’s edition of the Byzantines, p.
450.—M.}

\pagenote[82]{Orosius, piously inhuman, sacrifices the king and
people, Agag and the Amalekites, without a symptom of compassion.
The bloody actor is less detestable than the cool, unfeeling
historian.——Note: Considering the vow, which he was universally
believed to have made, to destroy Rome, and to sacrifice the
senators on the altars, and that he is said to have immolated his
prisoners to his gods, the execution of Radagaisus, if, as it
appears, he was taken in arms, cannot deserve Gibbon’s severe
condemnation. Mr. Herbert (notes to his poem of Attila, p. 317)
justly observes, that “Stilicho had probably authority for
hanging him on the first tree.” Marcellinus, adds Mr. Herbert,
attributes the execution to the Gothic chiefs Sarus.—M.}

\pagenote[83]{And Claudian’s muse, was she asleep? had she been
ill paid! Methinks the seventh consulship of Honorius (A.D. 407)
would have furnished the subject of a noble poem. Before it was
discovered that the state could no longer be saved, Stilicho
(after Romulus, Camillus and Marius) might have been worthily
surnamed the fourth founder of Rome.}

The fame of the victory, and more especially of the miracle, has
encouraged a vain persuasion, that the whole army, or rather
nation, of Germans, who migrated from the shores of the Baltic,
miserably perished under the walls of Florence. Such indeed was
the fate of Radagaisus himself, of his brave and faithful
companions, and of more than one third of the various multitude
of Sueves and Vandals, of Alani and Burgundians, who adhered to
the standard of their general.\textsuperscript{84} The union of such an army might
excite our surprise, but the causes of separation are obvious and
forcible; the pride of birth, the insolence of valor, the
jealousy of command, the impatience of subordination, and the
obstinate conflict of opinions, of interests, and of passions,
among so many kings and warriors, who were untaught to yield, or
to obey. After the defeat of Radagaisus, two parts of the German
host, which must have exceeded the number of one hundred thousand
men, still remained in arms, between the Apennine and the Alps,
or between the Alps and the Danube. It is uncertain whether they
attempted to revenge the death of their general; but their
irregular fury was soon diverted by the prudence and firmness of
Stilicho, who opposed their march, and facilitated their retreat;
who considered the safety of Rome and Italy as the great object
of his care, and who sacrificed, with too much indifference, the
wealth and tranquillity of the distant provinces.\textsuperscript{85} The
Barbarians acquired, from the junction of some Pannonian
deserters, the knowledge of the country, and of the roads; and
the invasion of Gaul, which Alaric had designed, was executed by
the remains of the great army of Radagaisus.\textsuperscript{86}

\pagenote[84]{A luminous passage of Prosper’s Chronicle, “In tres
partes, pes diversos principes, diversus exercitus,” reduces the
miracle of Florence and connects the history of Italy, Gaul, and
Germany.}

\pagenote[85]{Orosius and Jerom positively charge him with
instigating the in vasion. “Excitatae a Stilichone gentes,” \&c.
They must mean a directly. He saved Italy at the expense of Gaul}

\pagenote[86]{The Count de Buat is satisfied, that the Germans
who invaded Gaul were the two thirds that yet remained of the
army of Radagaisus. See the Histoire Ancienne des Peuples de
l’Europe, (tom. vii. p. 87, 121. Paris, 1772;) an elaborate work,
which I had not the advantage of perusing till the year 1777. As
early as 1771, I find the same idea expressed in a rough draught
of the present History. I have since observed a similar
intimation in Mascou, (viii. 15.) Such agreement, without mutual
communication, may add some weight to our common sentiment.}

Yet if they expected to derive any assistance from the tribes of
Germany, who inhabited the banks of the Rhine, their hopes were
disappointed. The Alemanni preserved a state of inactive
neutrality; and the Franks distinguished their zeal and courage
in the defence of the of the empire. In the rapid progress down
the Rhine, which was the first act of the administration of
Stilicho, he had applied himself, with peculiar attention, to
secure the alliance of the warlike Franks, and to remove the
irreconcilable enemies of peace and of the republic. Marcomir,
one of their kings, was publicly convicted, before the tribunal
of the Roman magistrate, of violating the faith of treaties. He
was sentenced to a mild, but distant exile, in the province of
Tuscany; and this degradation of the regal dignity was so far
from exciting the resentment of his subjects, that they punished
with death the turbulent Sunno, who attempted to revenge his
brother; and maintained a dutiful allegiance to the princes, who
were established on the throne by the choice of Stilicho.\textsuperscript{87} When
the limits of Gaul and Germany were shaken by the northern
emigration, the Franks bravely encountered the single force of
the Vandals; who, regardless of the lessons of adversity, had
again separated their troops from the standard of their Barbarian
allies. They paid the penalty of their rashness; and twenty
thousand Vandals, with their king Godigisclus, were slain in the
field of battle. The whole people must have been extirpated, if
the squadrons of the Alani, advancing to their relief, had not
trampled down the infantry of the Franks; who, after an honorable
resistance, were compelled to relinquish the unequal contest. The
victorious confederates pursued their march, and on the last day
of the year, in a season when the waters of the Rhine were most
probably frozen, they entered, without opposition, the
defenceless provinces of Gaul. This memorable passage of the
Suevi, the Vandals, the Alani, and the Burgundians, who never
afterwards retreated, may be considered as the fall of the Roman
empire in the countries beyond the Alps; and the barriers, which
had so long separated the savage and the civilized nations of the
earth, were from that fatal moment levelled with the ground.\textsuperscript{88}

\pagenote[87]{

Provincia missos Expellet citius fasces, quam Francia reges Quos
dederis.

Claudian (i. Cons. Stil. l. i. 235, \&c.) is clear and
satisfactory. These kings of France are unknown to Gregory of
Tours; but the author of the Gesta Francorum mentions both Sunno
and Marcomir, and names the latter as the father of Pharamond,
(in tom. ii. p. 543.) He seems to write from good materials,
which he did not understand.}

\pagenote[88]{See Zosimus, (l. vi. p. 373,) Orosius, (l. vii. c.
40, p. 576,) and the Chronicles. Gregory of Tours (l. ii. c. 9,
p. 165, in the second volume of the Historians of France) has
preserved a valuable fragment of Renatus Profuturus Frigeridus,
whose three names denote a Christian, a Roman subject, and a
Semi-Barbarian.}

While the peace of Germany was secured by the attachment of the
Franks, and the neutrality of the Alemanni, the subjects of Rome,
unconscious of their approaching calamities, enjoyed the state of
quiet and prosperity, which had seldom blessed the frontiers of
Gaul. Their flocks and herds were permitted to graze in the
pastures of the Barbarians; their huntsmen penetrated, without
fear or danger, into the darkest recesses of the Hercynian wood.\textsuperscript{89}
The banks of the Rhine were crowned, like those of the Tyber,
with elegant houses, and well-cultivated farms; and if a poet
descended the river, he might express his doubt, on which side
was situated the territory of the Romans.\textsuperscript{90} This scene of peace
and plenty was suddenly changed into a desert; and the prospect
of the smoking ruins could alone distinguish the solitude of
nature from the desolation of man. The flourishing city of Mentz
was surprised and destroyed; and many thousand Christians were
inhumanly massacred in the church. Worms perished after a long
and obstinate siege; Strasburgh, Spires, Rheims, Tournay, Arras,
Amiens, experienced the cruel oppression of the German yoke; and
the consuming flames of war spread from the banks of the Rhine
over the greatest part of the seventeen provinces of Gaul. That
rich and extensive country, as far as the ocean, the Alps, and
the Pyrenees, was delivered to the Barbarians, who drove before
them, in a promiscuous crowd, the bishop, the senator, and the
virgin, laden with the spoils of their houses and altars.\textsuperscript{91} The
ecclesiastics, to whom we are indebted for this vague description
of the public calamities, embraced the opportunity of exhorting
the Christians to repent of the sins which had provoked the
Divine Justice, and to renounce the perishable goods of a
wretched and deceitful world. But as the Pelagian controversy,\textsuperscript{92}
which attempts to sound the abyss of grace and predestination,
soon became the serious employment of the Latin clergy, the
Providence which had decreed, or foreseen, or permitted, such a
train of moral and natural evils, was rashly weighed in the
imperfect and fallacious balance of reason. The crimes, and the
misfortunes, of the suffering people, were presumptuously
compared with those of their ancestors; and they arraigned the
Divine Justice, which did not exempt from the common destruction
the feeble, the guiltless, the infant portion of the human
species. These idle disputants overlooked the invariable laws of
nature, which have connected peace with innocence, plenty with
industry, and safety with valor. The timid and selfish policy of
the court of Ravenna might recall the Palatine legions for the
protection of Italy; the remains of the stationary troops might
be unequal to the arduous task; and the Barbarian auxiliaries
might prefer the unbounded license of spoil to the benefits of a
moderate and regular stipend. But the provinces of Gaul were
filled with a numerous race of hardy and robust youth, who, in
the defence of their houses, their families, and their altars, if
they had dared to die, would have deserved to vanquish. The
knowledge of their native country would have enabled them to
oppose continual and insuperable obstacles to the progress of an
invader; and the deficiency of the Barbarians, in arms, as well
as in discipline, removed the only pretence which excuses the
submission of a populous country to the inferior numbers of a
veteran army. When France was invaded by Charles V., he inquired
of a prisoner, how many days Paris might be distant from the
frontier; “Perhaps twelve, but they will be days of battle:”\textsuperscript{93}
such was the gallant answer which checked the arrogance of that
ambitious prince. The subjects of Honorius, and those of Francis
I., were animated by a very different spirit; and in less than
two years, the divided troops of the savages of the Baltic, whose
numbers, were they fairly stated, would appear contemptible,
advanced, without a combat, to the foot of the Pyrenean
Mountains.

\pagenote[89]{Claudian (i. Cons. Stil. l. i. 221, \&c., l. ii.
186) describes the peace and prosperity of the Gallic frontier.
The Abbe Dubos (Hist. Critique, \&c., tom. i. p. 174) would read
Alba (a nameless rivulet of the Ardennes) instead of Albis; and
expatiates on the danger of the Gallic cattle grazing beyond the
Elbe. Foolish enough! In poetical geography, the Elbe, and the
Hercynian, signify any river, or any wood, in Germany. Claudian
is not prepared for the strict examination of our antiquaries.}

\pagenote[90]{—Germinasque viator Cum videat ripas, quae sit
Romana requirat.}

\pagenote[91]{Jerom, tom. i. p. 93. See in the 1st vol. of the
Historians of France, p. 777, 782, the proper extracts from the
Carmen de Providentil Divina, and Salvian. The anonymous poet was
himself a captive, with his bishop and fellow-citizens.}

\pagenote[92]{The Pelagian doctrine, which was first agitated
A.D. 405, was condemned, in the space of ten years, at Rome and
Carthage. St Augustin fought and conquered; but the Greek church
was favorable to his adversaries; and (what is singular enough)
the people did not take any part in a dispute which they could
not understand.}

\pagenote[93]{See the Mémoires de Guillaume du Bellay, l. vi. In
French, the original reproof is less obvious, and more pointed,
from the double sense of the word journee, which alike signifies,
a day’s travel, or a battle.}

In the early part of the reign of Honorius, the vigilance of
Stilicho had successfully guarded the remote island of Britain
from her incessant enemies of the ocean, the mountains, and the
Irish coast.\textsuperscript{94} But those restless Barbarians could not neglect
the fair opportunity of the Gothic war, when the walls and
stations of the province were stripped of the Roman troops. If
any of the legionaries were permitted to return from the Italian
expedition, their faithful report of the court and character of
Honorius must have tended to dissolve the bonds of allegiance,
and to exasperate the seditious temper of the British army. The
spirit of revolt, which had formerly disturbed the age of
Gallienus, was revived by the capricious violence of the
soldiers; and the unfortunate, perhaps the ambitious, candidates,
who were the objects of their choice, were the instruments, and
at length the victims, of their passion.\textsuperscript{95} Marcus was the first
whom they placed on the throne, as the lawful emperor of Britain
and of the West. They violated, by the hasty murder of Marcus,
the oath of fidelity which they had imposed on themselves; and
their disapprobation of his manners may seem to inscribe an
honorable epitaph on his tomb. Gratian was the next whom they
adorned with the diadem and the purple; and, at the end of four
months, Gratian experienced the fate of his predecessor. The
memory of the great Constantine, whom the British legions had
given to the church and to the empire, suggested the singular
motive of their third choice. They discovered in the ranks a
private soldier of the name of Constantine, and their impetuous
levity had already seated him on the throne, before they
perceived his incapacity to sustain the weight of that glorious
appellation.\textsuperscript{96} Yet the authority of Constantine was less
precarious, and his government was more successful, than the
transient reigns of Marcus and of Gratian. The danger of leaving
his inactive troops in those camps, which had been twice polluted
with blood and sedition, urged him to attempt the reduction of
the Western provinces. He landed at Boulogne with an
inconsiderable force; and after he had reposed himself some days,
he summoned the cities of Gaul, which had escaped the yoke of the
Barbarians, to acknowledge their lawful sovereign. They obeyed
the summons without reluctance. The neglect of the court of
Ravenna had absolved a deserted people from the duty of
allegiance; their actual distress encouraged them to accept any
circumstances of change, without apprehension, and, perhaps, with
some degree of hope; and they might flatter themselves, that the
troops, the authority, and even the name of a Roman emperor, who
fixed his residence in Gaul, would protect the unhappy country
from the rage of the Barbarians. The first successes of
Constantine against the detached parties of the Germans, were
magnified by the voice of adulation into splendid and decisive
victories; which the reunion and insolence of the enemy soon
reduced to their just value. His negotiations procured a short
and precarious truce; and if some tribes of the Barbarians were
engaged, by the liberality of his gifts and promises, to
undertake the defence of the Rhine, these expensive and uncertain
treaties, instead of restoring the pristine vigor of the Gallic
frontier, served only to disgrace the majesty of the prince, and
to exhaust what yet remained of the treasures of the republic.
Elated, however, with this imaginary triumph, the vain deliverer
of Gaul advanced into the provinces of the South, to encounter a
more pressing and personal danger. Sarus the Goth was ordered to
lay the head of the rebel at the feet of the emperor Honorius;
and the forces of Britain and Italy were unworthily consumed in
this domestic quarrel. After the loss of his two bravest
generals, Justinian and Nevigastes, the former of whom was slain
in the field of battle, the latter in a peaceful but treacherous
interview, Constantine fortified himself within the walls of
Vienna. The place was ineffectually attacked seven days; and the
Imperial army supported, in a precipitate retreat, the ignominy
of purchasing a secure passage from the freebooters and outlaws
of the Alps.\textsuperscript{97} Those mountains now separated the dominions of
two rival monarchs; and the fortifications of the double frontier
were guarded by the troops of the empire, whose arms would have
been more usefully employed to maintain the Roman limits against
the Barbarians of Germany and Scythia.

\pagenote[94]{Claudian, (i. Cons. Stil. l. ii. 250.) It is
supposed that the Scots of Ireland invaded, by sea, the whole
western coast of Britain: and some slight credit may be given
even to Nennius and the Irish traditions, (Carte’s Hist. of
England, vol. i. p. 169.) Whitaker’s Genuine History of the
Britons, p. 199. The sixty-six lives of St. Patrick, which were
extant in the ninth century, must have contained as many thousand
lies; yet we may believe, that, in one of these Irish inroads the
future apostle was led away captive, (Usher, Antiquit. Eccles
Britann. p. 431, and Tillemont, Mem. Eccles. tom. xvi. p. 45 782,
\&c.)}

\pagenote[95]{The British usurpers are taken from Zosimus, (l.
vi. p. 371-375,) Orosius, (l. vii. c. 40, p. 576, 577,)
Olympiodorus, (apud Photium, p. 180, 181,) the ecclesiastical
historians, and the Chronicles. The Latins are ignorant of
Marcus.}

\pagenote[96]{Cum in Constantino inconstantiam... execrarentur,
(Sidonius Apollinaris, l. v. epist. 9, p. 139, edit. secund.
Sirmond.) Yet Sidonius might be tempted, by so fair a pun, to
stigmatize a prince who had disgraced his grandfather.}

\pagenote[97]{Bagaudoe is the name which Zosimus applies to them;
perhaps they deserved a less odious character, (see Dubos, Hist.
Critique, tom. i. p. 203, and this History, vol. i. p. 407.) We
shall hear of them again.}

\section{Part \thesection.}

On the side of the Pyrenees, the ambition of Constantine might be
justified by the proximity of danger; but his throne was soon
established by the conquest, or rather submission, of Spain;
which yielded to the influence of regular and habitual
subordination, and received the laws and magistrates of the
Gallic præfecture. The only opposition which was made to the
authority of Constantine proceeded not so much from the powers of
government, or the spirit of the people, as from the private zeal
and interest of the family of Theodosius. Four brothers\textsuperscript{98} had
obtained, by the favor of their kinsman, the deceased emperor, an
honorable rank and ample possessions in their native country; and
the grateful youths resolved to risk those advantages in the
service of his son. After an unsuccessful effort to maintain
their ground at the head of the stationary troops of Lusitania,
they retired to their estates; where they armed and levied, at
their own expense, a considerable body of slaves and dependants,
and boldly marched to occupy the strong posts of the Pyrenean
Mountains. This domestic insurrection alarmed and perplexed the
sovereign of Gaul and Britain; and he was compelled to negotiate
with some troops of Barbarian auxiliaries, for the service of the
Spanish war. They were distinguished by the title of Honorians;\textsuperscript{99}
a name which might have reminded them of their fidelity to
their lawful sovereign; and if it should candidly be allowed that
the Scots were influenced by any partial affection for a British
prince, the Moors and the Marcomanni could be tempted only by the
profuse liberality of the usurper, who distributed among the
Barbarians the military, and even the civil, honors of Spain. The
nine bands of Honorians, which may be easily traced on the
establishment of the Western empire, could not exceed the number
of five thousand men: yet this inconsiderable force was
sufficient to terminate a war, which had threatened the power and
safety of Constantine. The rustic army of the Theodosian family
was surrounded and destroyed in the Pyrenees: two of the brothers
had the good fortune to escape by sea to Italy, or the East; the
other two, after an interval of suspense, were executed at Arles;
and if Honorius could remain insensible of the public disgrace,
he might perhaps be affected by the personal misfortunes of his
generous kinsmen. Such were the feeble arms which decided the
possession of the Western provinces of Europe, from the wall of
Antoninus to the columns of Hercules. The events of peace and war
have undoubtedly been diminished by the narrow and imperfect view
of the historians of the times, who were equally ignorant of the
causes, and of the effects, of the most important revolutions.
But the total decay of the national strength had annihilated even
the last resource of a despotic government; and the revenue of
exhausted provinces could no longer purchase the military service
of a discontented and pusillanimous people.

\pagenote[98]{Verinianus, Didymus, Theodosius, and Lagodius, who
in modern courts would be styled princes of the blood, were not
distinguished by any rank or privileges above the rest of their
fellow-subjects.}

\pagenote[99]{These Honoriani, or Honoriaci, consisted of two
bands of Scots, or Attacotti, two of Moors, two of Marcomanni,
the Victores, the Asca in, and the Gallicani, (Notitia Imperii,
sect. xxxiii. edit. Lab.) They were part of the sixty-five
Auxilia Palatina, and are properly styled by Zosimus, (l. vi.
374.)}

The poet, whose flattery has ascribed to the Roman eagle the
victories of Pollentia and Verona, pursues the hasty retreat of
Alaric, from the confines of Italy, with a horrid train of
imaginary spectres, such as might hover over an army of
Barbarians, which was almost exterminated by war, famine, and
disease.\textsuperscript{100} In the course of this unfortunate expedition, the
king of the Goths must indeed have sustained a considerable loss;
and his harassed forces required an interval of repose, to
recruit their numbers and revive their confidence. Adversity had
exercised and displayed the genius of Alaric; and the fame of his
valor invited to the Gothic standard the bravest of the Barbarian
warriors; who, from the Euxine to the Rhine, were agitated by the
desire of rapine and conquest. He had deserved the esteem, and he
soon accepted the friendship, of Stilicho himself. Renouncing the
service of the emperor of the East, Alaric concluded, with the
court of Ravenna, a treaty of peace and alliance, by which he was
declared master-general of the Roman armies throughout the
præfecture of Illyricum; as it was claimed, according to the
true and ancient limits, by the minister of Honorius.\textsuperscript{101} The
execution of the ambitious design, which was either stipulated,
or implied, in the articles of the treaty, appears to have been
suspended by the formidable irruption of Radagaisus; and the
neutrality of the Gothic king may perhaps be compared to the
indifference of Caesar, who, in the conspiracy of Catiline,
refused either to assist, or to oppose, the enemy of the
republic. After the defeat of the Vandals, Stilicho resumed his
pretensions to the provinces of the East; appointed civil
magistrates for the administration of justice, and of the
finances; and declared his impatience to lead to the gates of
Constantinople the united armies of the Romans and of the Goths.
The prudence, however, of Stilicho, his aversion to civil war,
and his perfect knowledge of the weakness of the state, may
countenance the suspicion, that domestic peace, rather than
foreign conquest, was the object of his policy; and that his
principal care was to employ the forces of Alaric at a distance
from Italy. This design could not long escape the penetration of
the Gothic king, who continued to hold a doubtful, and perhaps a
treacherous, correspondence with the rival courts; who
protracted, like a dissatisfied mercenary, his languid operations
in Thessaly and Epirus, and who soon returned to claim the
extravagant reward of his ineffectual services. From his camp
near Aemona,\textsuperscript{102} on the confines of Italy, he transmitted to the
emperor of the West a long account of promises, of expenses, and
of demands; called for immediate satisfaction, and clearly
intimated the consequences of a refusal. Yet if his conduct was
hostile, his language was decent and dutiful. He humbly professed
himself the friend of Stilicho, and the soldier of Honorius;
offered his person and his troops to march, without delay,
against the usurper of Gaul; and solicited, as a permanent
retreat for the Gothic nation, the possession of some vacant
province of the Western empire.

\pagenote[100]{

Comitatur euntem Pallor, et atra fames; et saucia lividus ora
Luctus; et inferno stridentes agmine morbi. —-Claudian in vi.
Cons. Hon. 821, \&c.}

\pagenote[101]{These dark transactions are investigated by the
Count de Bual (Hist. des Peuples de l’Europe, tom. vii. c.
iii.—viii. p. 69-206,) whose laborious accuracy may sometimes
fatigue a superficial reader.}

\pagenote[102]{See Zosimus, l. v. p. 334, 335. He interrupts his
scanty narrative to relate the fable of Aemona, and of the ship
Argo; which was drawn overland from that place to the Adriatic.
Sozomen (l. viii. c. 25, l. ix. c. 4) and Socrates (l. vii. c.
10) cast a pale and doubtful light; and Orosius (l. vii. c. 38,
p. 571) is abominably partial.}

The political and secret transactions of two statesmen, who
labored to deceive each other and the world, must forever have
been concealed in the impenetrable darkness of the cabinet, if
the debates of a popular assembly had not thrown some rays of
light on the correspondence of Alaric and Stilicho. The necessity
of finding some artificial support for a government, which, from
a principle, not of moderation, but of weakness, was reduced to
negotiate with its own subjects, had insensibly revived the
authority of the Roman senate; and the minister of Honorius
respectfully consulted the legislative council of the republic.
Stilicho assembled the senate in the palace of the Caesars;
represented, in a studied oration, the actual state of affairs;
proposed the demands of the Gothic king, and submitted to their
consideration the choice of peace or war. The senators, as if
they had been suddenly awakened from a dream of four hundred
years, appeared, on this important occasion, to be inspired by
the courage, rather than by the wisdom, of their predecessors.
They loudly declared, in regular speeches, or in tumultuary
acclamations, that it was unworthy of the majesty of Rome to
purchase a precarious and disgraceful truce from a Barbarian
king; and that, in the judgment of a magnanimous people, the
chance of ruin was always preferable to the certainty of
dishonor. The minister, whose pacific intentions were seconded
only by the voice of a few servile and venal followers, attempted
to allay the general ferment, by an apology for his own conduct,
and even for the demands of the Gothic prince. “The payment of a
subsidy, which had excited the indignation of the Romans, ought
not (such was the language of Stilicho) to be considered in the
odious light, either of a tribute, or of a ransom, extorted by
the menaces of a Barbarian enemy. Alaric had faithfully asserted
the just pretensions of the republic to the provinces which were
usurped by the Greeks of Constantinople: he modestly required the
fair and stipulated recompense of his services; and if he had
desisted from the prosecution of his enterprise, he had obeyed,
in his retreat, the peremptory, though private, letters of the
emperor himself. These contradictory orders (he would not
dissemble the errors of his own family) had been procured by the
intercession of Serena. The tender piety of his wife had been too
deeply affected by the discord of the royal brothers, the sons of
her adopted father; and the sentiments of nature had too easily
prevailed over the stern dictates of the public welfare.” These
ostensible reasons, which faintly disguise the obscure intrigues
of the palace of Ravenna, were supported by the authority of
Stilicho; and obtained, after a warm debate, the reluctant
approbation of the senate. The tumult of virtue and freedom
subsided; and the sum of four thousand pounds of gold was
granted, under the name of a subsidy, to secure the peace of
Italy, and to conciliate the friendship of the king of the Goths.
Lampadius alone, one of the most illustrious members of the
assembly, still persisted in his dissent; exclaimed, with a loud
voice, “This is not a treaty of peace, but of servitude;”\textsuperscript{103} and
escaped the danger of such bold opposition by immediately
retiring to the sanctuary of a Christian church. [See Palace Of
The Caesars]

\pagenote[103]{Zosimus, l. v. p. 338, 339. He repeats the words
of Lampadius, as they were spoke in Latin, “Non est ista pax, sed
pactio servi tutis,” and then translates them into Greek for the
benefit of his readers. * Note: From Cicero’s XIIth Philippic,
14.—M.}

But the reign of Stilicho drew towards its end; and the proud
minister might perceive the symptoms of his approaching disgrace.
The generous boldness of Lampadius had been applauded; and the
senate, so patiently resigned to a long servitude, rejected with
disdain the offer of invidious and imaginary freedom. The troops,
who still assumed the name and prerogatives of the Roman legions,
were exasperated by the partial affection of Stilicho for the
Barbarians: and the people imputed to the mischievous policy of
the minister the public misfortunes, which were the natural
consequence of their own degeneracy. Yet Stilicho might have
continued to brave the clamors of the people, and even of the
soldiers, if he could have maintained his dominion over the
feeble mind of his pupil. But the respectful attachment of
Honorius was converted into fear, suspicion, and hatred. The
crafty Olympius,\textsuperscript{104} who concealed his vices under the mask of
Christian piety, had secretly undermined the benefactor, by whose
favor he was promoted to the honorable offices of the Imperial
palace. Olympius revealed to the unsuspecting emperor, who had
attained the twenty-fifth year of his age, that he was without
weight, or authority, in his own government; and artfully alarmed
his timid and indolent disposition by a lively picture of the
designs of Stilicho, who already meditated the death of his
sovereign, with the ambitious hope of placing the diadem on the
head of his son Eucherius. The emperor was instigated, by his new
favorite, to assume the tone of independent dignity; and the
minister was astonished to find, that secret resolutions were
formed in the court and council, which were repugnant to his
interest, or to his intentions. Instead of residing in the palace
of Rome, Honorius declared that it was his pleasure to return to
the secure fortress of Ravenna. On the first intelligence of the
death of his brother Arcadius, he prepared to visit
Constantinople, and to regulate, with the authority of a
guardian, the provinces of the infant Theodosius.\textsuperscript{105} The
representation of the difficulty and expense of such a distant
expedition, checked this strange and sudden sally of active
diligence; but the dangerous project of showing the emperor to
the camp of Pavia, which was composed of the Roman troops, the
enemies of Stilicho, and his Barbarian auxiliaries, remained
fixed and unalterable. The minister was pressed, by the advice of
his confidant, Justinian, a Roman advocate, of a lively and
penetrating genius, to oppose a journey so prejudicial to his
reputation and safety. His strenuous but ineffectual efforts
confirmed the triumph of Olympius; and the prudent lawyer
withdrew himself from the impending ruin of his patron.

\pagenote[104]{He came from the coast of the Euxine, and
exercised a splendid office. His actions justify his character,
which Zosimus (l. v. p. 340) exposes with visible satisfaction.
Augustin revered the piety of Olympius, whom he styles a true son
of the church, (Baronius, Annal. Eccles, Eccles. A.D. 408, No.
19, \&c. Tillemont, Mem. Eccles. tom. xiii. p. 467, 468.) But
these praises, which the African saint so unworthily bestows,
might proceed as well from ignorance as from adulation.}

\pagenote[105]{Zosimus, l. v. p. 338, 339. Sozomen, l. ix. c. 4.
Stilicho offered to undertake the journey to Constantinople, that
he might divert Honorius from the vain attempt. The Eastern
empire would not have obeyed, and could not have been conquered.}

In the passage of the emperor through Bologna, a mutiny of the
guards was excited and appeased by the secret policy of Stilicho;
who announced his instructions to decimate the guilty, and
ascribed to his own intercession the merit of their pardon. After
this tumult, Honorius embraced, for the last time, the minister
whom he now considered as a tyrant, and proceeded on his way to
the camp of Pavia; where he was received by the loyal
acclamations of the troops who were assembled for the service of
the Gallic war. On the morning of the fourth day, he pronounced,
as he had been taught, a military oration in the presence of the
soldiers, whom the charitable visits, and artful discourses, of
Olympius had prepared to execute a dark and bloody conspiracy. At
the first signal, they massacred the friends of Stilicho, the
most illustrious officers of the empire; two Prætorian
præfects, of Gaul and of Italy; two masters-general of the
cavalry and infantry; the master of the offices; the quaestor,
the treasurer, and the count of the domestics. Many lives were
lost; many houses were plundered; the furious sedition continued
to rage till the close of the evening; and the trembling emperor,
who was seen in the streets of Pavia without his robes or diadem,
yielded to the persuasions of his favorite; condemned the memory
of the slain; and solemnly approved the innocence and fidelity of
their assassins. The intelligence of the massacre of Pavia filled
the mind of Stilicho with just and gloomy apprehensions; and he
instantly summoned, in the camp of Bologna, a council of the
confederate leaders, who were attached to his service, and would
be involved in his ruin. The impetuous voice of the assembly
called aloud for arms, and for revenge; to march, without a
moment’s delay, under the banners of a hero, whom they had so
often followed to victory; to surprise, to oppress, to extirpate
the guilty Olympius, and his degenerate Romans; and perhaps to
fix the diadem on the head of their injured general. Instead of
executing a resolution, which might have been justified by
success, Stilicho hesitated till he was irrecoverably lost. He
was still ignorant of the fate of the emperor; he distrusted the
fidelity of his own party; and he viewed with horror the fatal
consequences of arming a crowd of licentious Barbarians against
the soldiers and people of Italy. The confederates, impatient of
his timorous and doubtful delay, hastily retired, with fear and
indignation. At the hour of midnight, Sarus, a Gothic warrior,
renowned among the Barbarians themselves for his strength and
valor, suddenly invaded the camp of his benefactor, plundered the
baggage, cut in pieces the faithful Huns, who guarded his person,
and penetrated to the tent, where the minister, pensive and
sleepless, meditated on the dangers of his situation. Stilicho
escaped with difficulty from the sword of the Goths and, after
issuing a last and generous admonition to the cities of Italy, to
shut their gates against the Barbarians, his confidence, or his
despair, urged him to throw himself into Ravenna, which was
already in the absolute possession of his enemies. Olympius, who
had assumed the dominion of Honorius, was speedily informed, that
his rival had embraced, as a suppliant the altar of the Christian
church. The base and cruel disposition of the hypocrite was
incapable of pity or remorse; but he piously affected to elude,
rather than to violate, the privilege of the sanctuary. Count
Heraclian, with a troop of soldiers, appeared, at the dawn of
day, before the gates of the church of Ravenna. The bishop was
satisfied by a solemn oath, that the Imperial mandate only
directed them to secure the person of Stilicho: but as soon as
the unfortunate minister had been tempted beyond the holy
threshold, he produced the warrant for his instant execution.
Stilicho supported, with calm resignation, the injurious names of
traitor and parricide; repressed the unseasonable zeal of his
followers, who were ready to attempt an ineffectual rescue; and,
with a firmness not unworthy of the last of the Roman generals,
submitted his neck to the sword of Heraclian.\textsuperscript{106}

\pagenote[106]{Zosimus (l. v. p. 336-345) has copiously, though
not clearly, related the disgrace and death of Stilicho.
Olympiodorus, (apud Phot. p. 177.) Orosius, (l. vii. c. 38, p.
571, 572,) Sozomen, (l. ix. c. 4,) and Philostorgius, (l. xi. c.
3, l. xii. c. 2,) afford supplemental hints.}

The servile crowd of the palace, who had so long adored the
fortune of Stilicho, affected to insult his fall; and the most
distant connection with the master-general of the West, which had
so lately been a title to wealth and honors, was studiously
denied, and rigorously punished. His family, united by a triple
alliance with the family of Theodosius, might envy the condition
of the meanest peasant. The flight of his son Eucherius was
intercepted; and the death of that innocent youth soon followed
the divorce of Thermantia, who filled the place of her sister
Maria; and who, like Maria, had remained a virgin in the Imperial
bed.\textsuperscript{107} The friends of Stilicho, who had escaped the massacre of
Pavia, were persecuted by the implacable revenge of Olympius; and
the most exquisite cruelty was employed to extort the confession
of a treasonable and sacrilegious conspiracy. They died in
silence: their firmness justified the choice,\textsuperscript{108} and perhaps
absolved the innocence of their patron: and the despotic power,
which could take his life without a trial, and stigmatize his
memory without a proof, has no jurisdiction over the impartial
suffrage of posterity.\textsuperscript{109} The services of Stilicho are great and
manifest; his crimes, as they are vaguely stated in the language
of flattery and hatred, are obscure at least, and improbable.
About four months after his death, an edict was published, in the
name of Honorius, to restore the free communication of the two
empires, which had been so long interrupted by the public enemy.\textsuperscript{110}
The minister, whose fame and fortune depended on the
prosperity of the state, was accused of betraying Italy to the
Barbarians; whom he repeatedly vanquished at Pollentia, at
Verona, and before the walls of Florence. His pretended design of
placing the diadem on the head of his son Eucherius, could not
have been conducted without preparations or accomplices; and the
ambitious father would not surely have left the future emperor,
till the twentieth year of his age, in the humble station of
tribune of the notaries. Even the religion of Stilicho was
arraigned by the malice of his rival. The seasonable, and almost
miraculous, deliverance was devoutly celebrated by the applause
of the clergy; who asserted, that the restoration of idols, and
the persecution of the church, would have been the first measure
of the reign of Eucherius. The son of Stilicho, however, was
educated in the bosom of Christianity, which his father had
uniformly professed, and zealously supported.\textsuperscript{111} \textsuperscript{1111} Serena had
borrowed her magnificent necklace from the statue of Vesta;\textsuperscript{112}
and the Pagans execrated the memory of the sacrilegious minister,
by whose order the Sibylline books, the oracles of Rome, had been
committed to the flames.\textsuperscript{113} The pride and power of Stilicho
constituted his real guilt. An honorable reluctance to shed the
blood of his countrymen appears to have contributed to the
success of his unworthy rival; and it is the last humiliation of
the character of Honorius, that posterity has not condescended to
reproach him with his base ingratitude to the guardian of his
youth, and the support of his empire.

\pagenote[107]{Zosimus, l. v. p. 333. The marriage of a Christian
with two sisters, scandalizes Tillemont, (Hist. des Empereurs,
tom. v. p. 557;) who expects, in vain, that Pope Innocent I.
should have done something in the way either of censure or of
dispensation.}

\pagenote[108]{Two of his friends are honorably mentioned,
(Zosimus, l. v. p. 346:) Peter, chief of the school of notaries,
and the great chamberlain Deuterius. Stilicho had secured the
bed-chamber; and it is surprising that, under a feeble prince,
the bed-chamber was not able to secure him.}

\pagenote[109]{Orosius (l. vii. c. 38, p. 571, 572) seems to copy
the false and furious manifestos, which were dispersed through
the provinces by the new administration.}

\pagenote[110]{See the Theodosian code, l. vii. tit. xvi. leg. 1,
l. ix. tit. xlii. leg. 22. Stilicho is branded with the name of
proedo publicus, who employed his wealth, ad omnem ditandam,
inquietandamque Barbariem.}

\pagenote[111]{Augustin himself is satisfied with the effectual
laws, which Stilicho had enacted against heretics and idolaters;
and which are still extant in the Code. He only applies to
Olympius for their confirmation, (Baronius, Annal. Eccles. A.D.
408, No. 19.)}

\pagenote[112]{Zosimus, l. v. p. 351. We may observe the bad
taste of the age, in dressing their statues with such awkward
finery.}

\pagenote[113]{See Rutilius Numatianus, (Itinerar. l. ii. 41-60,)
to whom religious enthusiasm has dictated some elegant and
forcible lines. Stilicho likewise stripped the gold plates from
the doors of the Capitol, and read a prophetic sentence which was
engraven under them, (Zosimus, l. v. p. 352.) These are foolish
stories: yet the charge of impiety adds weight and credit to the
praise which Zosimus reluctantly bestows on his virtues. Note:
One particular in the extorted praise of Zosimus, deserved the
notice of the historian, as strongly opposed to the former
imputations of Zosimus himself, and indicative of he corrupt
practices of a declining age. “He had never bartered promotion in
the army for bribes, nor peculated in the supplies of provisions
for the army.” l. v. c. xxxiv.—M.}

\pagenote[1111]{[Hence, perhaps, the accusation of treachery is
countenanced by Hatilius:—

Quo magis est facinus diri Stilichonis iniquum Proditor arcani
quod fuit imperii. Romano generi dum nititur esse superstes,
Crudelis summis miscuit ima furor. Dumque timet, quicquid se
fecerat ipso timeri, Immisit Latiae barbara tela neci.  Rutil.
Itin. II. 41.—M.] Among the train of dependants whose wealth and
dignity attracted the notice of their own times, our curiosity is
excited by the celebrated name of the poet Claudian, who enjoyed
the favor of Stilicho, and was overwhelmed in the ruin of his
patron.}

Among the train of dependants whose wealth and dignity attracted
the notice of their own times, \textit{our} curiosity is excited by the
celebrated name of the poet Claudian, who enjoyed the favor of
Stilicho, and was overwhelmed in the ruin of his patron. The
titular offices of tribune and notary fixed his rank in the
Imperial court: he was indebted to the powerful intercession of
Serena for his marriage with a very rich heiress of the province
of Africa;\textsuperscript{114} and the statute of Claudian, erected in the forum
of Trajan, was a monument of the taste and liberality of the
Roman senate.\textsuperscript{115} After the praises of Stilicho became offensive
and criminal, Claudian was exposed to the enmity of a powerful
and unforgiving courtier, whom he had provoked by the insolence
of wit. He had compared, in a lively epigram, the opposite
characters of two Prætorian præfects of Italy; he contrasts the
innocent repose of a philosopher, who sometimes resigned the
hours of business to slumber, perhaps to study, with the
interesting diligence of a rapacious minister, indefatigable in
the pursuit of unjust or sacrilegious, gain. “How happy,”
continues Claudian, “how happy might it be for the people of
Italy, if Mallius could be constantly awake, and if Hadrian would
always sleep!”\textsuperscript{116} The repose of Mallius was not disturbed by
this friendly and gentle admonition; but the cruel vigilance of
Hadrian watched the opportunity of revenge, and easily obtained,
from the enemies of Stilicho, the trifling sacrifice of an
obnoxious poet. The poet concealed himself, however, during the
tumult of the revolution; and, consulting the dictates of
prudence rather than of honor, he addressed, in the form of an
epistle, a suppliant and humble recantation to the offended
præfect. He deplores, in mournful strains, the fatal
indiscretion into which he had been hurried by passion and folly;
submits to the imitation of his adversary the generous examples
of the clemency of gods, of heroes, and of lions; and expresses
his hope that the magnanimity of Hadrian will not trample on a
defenceless and contemptible foe, already humbled by disgrace and
poverty, and deeply wounded by the exile, the tortures, and the
death of his dearest friends.\textsuperscript{117} Whatever might be the success
of his prayer, or the accidents of his future life, the period of
a few years levelled in the grave the minister and the poet: but
the name of Hadrian is almost sunk in oblivion, while Claudian is
read with pleasure in every country which has retained, or
acquired, the knowledge of the Latin language. If we fairly
balance his merits and his defects, we shall acknowledge that
Claudian does not either satisfy, or silence, our reason. It
would not be easy to produce a passage that deserves the epithet
of sublime or pathetic; to select a verse that melts the heart or
enlarges the imagination. We should vainly seek, in the poems of
Claudian, the happy invention, and artificial conduct, of an
interesting fable; or the just and lively representation of the
characters and situations of real life. For the service of his
patron, he published occasional panegyrics and invectives: and
the design of these slavish compositions encouraged his
propensity to exceed the limits of truth and nature. These
imperfections, however, are compensated in some degree by the
poetical virtues of Claudian. He was endowed with the rare and
precious talent of raising the meanest, of adorning the most
barren, and of diversifying the most similar, topics: his
coloring, more especially in descriptive poetry, is soft and
splendid; and he seldom fails to display, and even to abuse, the
advantages of a cultivated understanding, a copious fancy, an
easy, and sometimes forcible, expression; and a perpetual flow of
harmonious versification. To these commendations, independent of
any accidents of time and place, we must add the peculiar merit
which Claudian derived from the unfavorable circumstances of his
birth. In the decline of arts, and of empire, a native of Egypt,\textsuperscript{118}
who had received the education of a Greek, assumed, in a
mature age, the familiar use, and absolute command, of the Latin
language;\textsuperscript{119} soared above the heads of his feeble
contemporaries; and placed himself, after an interval of three
hundred years, among the poets of ancient Rome.\textsuperscript{120}

\pagenote[114]{At the nuptials of Orpheus (a modest comparison!)
all the parts of animated nature contributed their various gifts;
and the gods themselves enriched their favorite. Claudian had
neither flocks, nor herds, nor vines, nor olives. His wealthy
bride was heiress to them all. But he carried to Africa a
recommendatory letter from Serena, his Juno, and was made happy,
(Epist. ii. ad Serenam.)}

\pagenote[115]{Claudian feels the honor like a man who deserved
it, (in praefat Bell. Get.) The original inscription, on marble,
was found at Rome, in the fifteenth century, in the house of
Pomponius Laetus. The statue of a poet, far superior to Claudian,
should have been erected, during his lifetime, by the men of
letters, his countrymen and contemporaries. It was a noble
design.}

\pagenote[116]{See Epigram xxx.

Mallius indulget somno noctesque diesque: Insomnis Pharius sacra,
profana, rapit. Omnibus, hoc, Italae gentes, exposcite votis;
Mallius ut vigilet, dormiat ut Pharius.

Hadrian was a Pharian, (of Alexandrian.) See his public life in
Godefroy, Cod. Theodos. tom. vi. p. 364. Mallius did not always
sleep. He composed some elegant dialogues on the Greek systems of
natural philosophy, (Claud, in Mall. Theodor. Cons. 61-112.)}

\pagenote[117]{See Claudian’s first Epistle. Yet, in some places,
an air of irony and indignation betrays his secret reluctance. *
Note: M. Beugnot has pointed out one remarkable characteristic of
Claudian’s poetry, and of the times—his extraordinary religious
indifference. Here is a poet writing at the actual crisis of the
complete triumph of the new religion, the visible extinction of
the old: if we may so speak, a strictly historical poet, whose
works, excepting his Mythological poem on the rape of Proserpine,
are confined to temporary subjects, and to the politics of his
own eventful day; yet, excepting in one or two small and
indifferent pieces, manifestly written by a Christian, and
interpolated among his poems, there is no allusion whatever to
the great religious strife. No one would know the existence of
Christianity at that period of the world, by reading the works of
Claudian. His panegyric and his satire preserve the same
religious impartiality; award their most lavish praise or their
bitterest invective on Christian or Pagan; he insults the fall of
Eugenius, and glories in the victories of Theodosius. Under the
child,—and Honorius never became more than a child,—Christianity
continued to inflict wounds more and more deadly on expiring
Paganism. Are the gods of Olympus agitated with apprehension at
the birth of this new enemy? They are introduced as rejoicing at
his appearance, and promising long years of glory. The whole
prophetic choir of Paganism, all the oracles throughout the
world, are summoned to predict the felicity of his reign. His
birth is compared to that of Apollo, but the narrow limits of an
island must not confine the new deity—

... Non littora nostro Sufficerent angusta Deo.

Augury and divination, the shrines of Ammon, and of Delphi, the
Persian Magi, and the Etruscan seers, the Chaldean astrologers,
the Sibyl herself, are described as still discharging their
prophetic functions, and celebrating the natal day of this
Christian prince. They are noble lines, as well as curious
illustrations of the times:

... Quae tunc documenta futuri? Quae voces avium? quanti per inane
volatus? Quis vatum discursus erat?  Tibi corniger Ammon, Et dudum
taciti rupere silentia Delphi. Te Persae cecinere Magi, te sensit
Etruscus Augur, et inspectis Babylonius horruit astris; Chaldaei
stupuere senes, Cumanaque rursus Itonuit rupes, rabidae delubra
Sibyllae. —Claud. iv. Cons. Hon. 141.

From the Quarterly Review of Beugnot. Hist. de la Paganisme en
Occident, Q. R. v. lvii. p. 61.—M.}

\pagenote[118]{National vanity has made him a Florentine, or a
Spaniard. But the first Epistle of Claudian proves him a native
of Alexandria, (Fabricius, Bibliot. Latin. tom. iii. p. 191-202,
edit. Ernest.)}

\pagenote[119]{His first Latin verses were composed during the
consulship of Probinus, A.D. 395.

Romanos bibimus primum, te consule, fontes, Et Latiae cessit
Graia Thalia togae.

Besides some Greek epigrams, which are still extant, the Latin
poet had composed, in Greek, the Antiquities of Tarsus,
Anazarbus, Berytus, Nice, \&c. It is more easy to supply the loss
of good poetry, than of authentic history.}

\pagenote[120]{Strada (Prolusion v. vi.) allows him to contend
with the five heroic poets, Lucretius, Virgil, Ovid, Lucan, and
Statius. His patron is the accomplished courtier Balthazar
Castiglione. His admirers are numerous and passionate. Yet the
rigid critics reproach the exotic weeds, or flowers, which spring
too luxuriantly in his Latian soil}

