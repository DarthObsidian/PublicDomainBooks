\chapter{Division Of Roman Empire Between Sons Of Theodosius.}
\section{Part \thesection.}

\textit{Final Division Of The Roman Empire Between The Sons Of
Theodosius. — Reign Of Arcadius And Honoriuas — Administration Of
Rufinus And Stilicho. — Revolt And Defeat Of Gildo In Africa.}

The genius of Rome expired with Theodosius; the last of the
successors of Augustus and Constantine, who appeared in the field
at the head of their armies, and whose authority was universally
acknowledged throughout the whole extent of the empire. The
memory of his virtues still continued, however, to protect the
feeble and inexperienced youth of his two sons. After the death
of their father, Arcadius and Honorius were saluted, by the
unanimous consent of mankind, as the lawful emperors of the East,
and of the West; and the oath of fidelity was eagerly taken by
every order of the state; the senates of old and new Rome, the
clergy, the magistrates, the soldiers, and the people. Arcadius,
who was then about eighteen years of age, was born in Spain, in
the humble habitation of a private family. But he received a
princely education in the palace of Constantinople; and his
inglorious life was spent in that peaceful and splendid seat of
royalty, from whence he appeared to reign over the provinces of
Thrace, Asia Minor, Syria, and Egypt, from the Lower Danube to
the confines of Persia and Æthiopia. His younger brother
Honorius, assumed, in the eleventh year of his age, the nominal
government of Italy, Africa, Gaul, Spain, and Britain; and the
troops, which guarded the frontiers of his kingdom, were opposed,
on one side, to the Caledonians, and on the other, to the Moors.
The great and martial præfecture of Illyricum was divided
between the two princes: the defence and possession of the
provinces of Noricum, Pannonia, and Dalmatia still belonged to
the Western empire; but the two large dioceses of Dacia and
Macedonia, which Gratian had intrusted to the valor of
Theodosius, were forever united to the empire of the East. The
boundary in Europe was not very different from the line which now
separates the Germans and the Turks; and the respective
advantages of territory, riches, populousness, and military
strength, were fairly balanced and compensated, in this final and
permanent division of the Roman empire. The hereditary sceptre of
the sons of Theodosius appeared to be the gift of nature, and of
their father; the generals and ministers had been accustomed to
adore the majesty of the royal infants; and the army and people
were not admonished of their rights, and of their power, by the
dangerous example of a recent election. The gradual discovery of
the weakness of Arcadius and Honorius, and the repeated
calamities of their reign, were not sufficient to obliterate the
deep and early impressions of loyalty. The subjects of Rome, who
still reverenced the persons, or rather the names, of their
sovereigns, beheld, with equal abhorrence, the rebels who
opposed, and the ministers who abused, the authority of the
throne.

Theodosius had tarnished the glory of his reign by the elevation
of Rufinus; an odious favorite, who, in an age of civil and
religious faction, has deserved, from every party, the imputation
of every crime. The strong impulse of ambition and avarice\textsuperscript{1} had
urged Rufinus to abandon his native country, an obscure corner of
Gaul,\textsuperscript{2} to advance his fortune in the capital of the East: the
talent of bold and ready elocution,\textsuperscript{3} qualified him to succeed in
the lucrative profession of the law; and his success in that
profession was a regular step to the most honorable and important
employments of the state. He was raised, by just degrees, to the
station of master of the offices. In the exercise of his various
functions, so essentially connected with the whole system of
civil government, he acquired the confidence of a monarch, who
soon discovered his diligence and capacity in business, and who
long remained ignorant of the pride, the malice, and the
covetousness of his disposition. These vices were concealed
beneath the mask of profound dissimulation;\textsuperscript{4} his passions were
subservient only to the passions of his master; yet in the horrid
massacre of Thessalonica, the cruel Rufinus inflamed the fury,
without imitating the repentance, of Theodosius. The minister,
who viewed with proud indifference the rest of mankind, never
forgave the appearance of an injury; and his personal enemies had
forfeited, in his opinion, the merit of all public services.
Promotus, the master-general of the infantry, had saved the
empire from the invasion of the Ostrogoths; but he indignantly
supported the preeminence of a rival, whose character and
profession he despised; and in the midst of a public council, the
impatient soldier was provoked to chastise with a blow the
indecent pride of the favorite. This act of violence was
represented to the emperor as an insult, which it was incumbent
on his dignity to resent. The disgrace and exile of Promotus were
signified by a peremptory order, to repair, without delay, to a
military station on the banks of the Danube; and the death of
that general (though he was slain in a skirmish with the
Barbarians) was imputed to the perfidious arts of Rufinus.\textsuperscript{5} The
sacrifice of a hero gratified his revenge; the honors of the
consulship elated his vanity; but his power was still imperfect
and precarious, as long as the important posts of præfect of the
East, and of præfect of Constantinople, were filled by Tatian,\textsuperscript{6}
and his son Proculus; whose united authority balanced, for some
time, the ambition and favor of the master of the offices. The
two præfects were accused of rapine and corruption in the
administration of the laws and finances. For the trial of these
illustrious offenders, the emperor constituted a special
commission: several judges were named to share the guilt and
reproach of injustice; but the right of pronouncing sentence was
reserved to the president alone, and that president was Rufinus
himself. The father, stripped of the præfecture of the East, was
thrown into a dungeon; but the son, conscious that few ministers
can be found innocent, where an enemy is their judge, had
secretly escaped; and Rufinus must have been satisfied with the
least obnoxious victim, if despotism had not condescended to
employ the basest and most ungenerous artifice. The prosecution
was conducted with an appearance of equity and moderation, which
flattered Tatian with the hope of a favorable event: his
confidence was fortified by the solemn assurances, and perfidious
oaths, of the president, who presumed to interpose the sacred
name of Theodosius himself; and the unhappy father was at last
persuaded to recall, by a private letter, the fugitive Proculus.
He was instantly seized, examined, condemned, and beheaded, in
one of the suburbs of Constantinople, with a precipitation which
disappointed the clemency of the emperor. Without respecting the
misfortunes of a consular senator, the cruel judges of Tatian
compelled him to behold the execution of his son: the fatal cord
was fastened round his own neck; but in the moment when he
expected. and perhaps desired, the relief of a speedy death, he
was permitted to consume the miserable remnant of his old age in
poverty and exile.\textsuperscript{7} The punishment of the two præfects might,
perhaps, be excused by the exceptionable parts of their own
conduct; the enmity of Rufinus might be palliated by the jealous
and unsociable nature of ambition. But he indulged a spirit of
revenge equally repugnant to prudence and to justice, when he
degraded their native country of Lycia from the rank of Roman
provinces; stigmatized a guiltless people with a mark of
ignominy; and declared, that the countrymen of Tatian and
Proculus should forever remain incapable of holding any
employment of honor or advantage under the Imperial government.\textsuperscript{8}
The new præfect of the East (for Rufinus instantly succeeded to
the vacant honors of his adversary) was not diverted, however, by
the most criminal pursuits, from the performance of the religious
duties, which in that age were considered as the most essential
to salvation. In the suburb of Chalcedon, surnamed the Oak, he
had built a magnificent villa; to which he devoutly added a
stately church, consecrated to the apostles St. Peter and St.
Paul, and continually sanctified by the prayers and penance of a
regular society of monks. A numerous, and almost general, synod
of the bishops of the Eastern empire, was summoned to celebrate,
at the same time, the dedication of the church, and the baptism
of the founder. This double ceremony was performed with
extraordinary pomp; and when Rufinus was purified, in the holy
font, from all the sins that he had hitherto committed, a
venerable hermit of Egypt rashly proposed himself as the sponsor
of a proud and ambitious statesman.\textsuperscript{9}

\pagenote[1]{Alecto, envious of the public felicity, convenes an
infernal synod Megaera recommends her pupil Rufinus, and excites
him to deeds of mischief, \&c. But there is as much difference
between Claudian’s fury and that of Virgil, as between the
characters of Turnus and Rufinus.}

\pagenote[2]{It is evident, (Tillemont, Hist. des Emp. tom. v. p.
770,) though De Marca is ashamed of his countryman, that Rufinus
was born at Elusa, the metropolis of Novempopulania, now a small
village of Gassony, (D’Anville, Notice de l’Ancienne Gaule, p.
289.)}

\pagenote[3]{Philostorgius, l. xi c. 3, with Godefroy’s Dissert.
p. 440.}

\pagenote[4]{A passage of Suidas is expressive of his profound
dissimulation.}

\pagenote[5]{Zosimus, l. iv. p. 272, 273.}

\pagenote[6]{Zosimus, who describes the fall of Tatian and his
son, (l. iv. p. 273, 274,) asserts their innocence; and even his
testimony may outweigh the charges of their enemies, (Cod. Theod.
tom. iv. p. 489,) who accuse them of oppressing the Curiae. The
connection of Tatian with the Arians, while he was præfect of
Egypt, (A.D. 373,) inclines Tillemont to believe that he was
guilty of every crime, (Hist. des Emp. tom. v. p. 360. Mem.
Eccles. tom vi. p. 589.)}

\pagenote[7]{—Juvenum rorantia colla Ante patrum vultus stricta
cecidere securi.

Ibat grandaevus nato moriente superstes Post trabeas exsul. —-In
Rufin. i. 248.

The facts of Zosimus explain the allusions of Claudian; but his
classic interpreters were ignorant of the fourth century. The
fatal cord, I found, with the help of Tillemont, in a sermon of
St. Asterius of Amasea.}

\pagenote[8]{This odious law is recited and repealed by Arcadius,
(A.D. 296,) on the Theodosian Code, l. ix. tit. xxxviii. leg. 9.
The sense as it is explained by Claudian, (in Rufin. i. 234,) and
Godefroy, (tom. iii. p. 279,) is perfectly clear.

—-Exscindere cives Funditus; et nomen gentis delere laborat.

The scruples of Pagi and Tillemont can arise only from their zeal
for the glory of Theodosius.}

\pagenote[9]{Ammonius.... Rufinum propriis manibus suscepit sacro
fonte mundatum. See Rosweyde’s Vitae Patrum, p. 947. Sozomen (l.
viii. c. 17) mentions the church and monastery; and Tillemont
(Mem. Eccles. tom. ix. p. 593) records this synod, in which St.
Gregory of Nyssa performed a conspicuous part.}

The character of Theodosius imposed on his minister the task of
hypocrisy, which disguised, and sometimes restrained, the abuse
of power; and Rufinus was apprehensive of disturbing the indolent
slumber of a prince still capable of exerting the abilities and
the virtue, which had raised him to the throne.\textsuperscript{10} But the
absence, and, soon afterwards, the death, of the emperor,
confirmed the absolute authority of Rufinus over the person and
dominions of Arcadius; a feeble youth, whom the imperious
præfect considered as his pupil, rather than his sovereign.
Regardless of the public opinion, he indulged his passions
without remorse, and without resistance; and his malignant and
rapacious spirit rejected every passion that might have
contributed to his own glory, or the happiness of the people. His
avarice,\textsuperscript{11} which seems to have prevailed, in his corrupt mind,
over every other sentiment, attracted the wealth of the East, by
the various arts of partial and general extortion; oppressive
taxes, scandalous bribery, immoderate fines, unjust
confiscations, forced or fictitious testaments, by which the
tyrant despoiled of their lawful inheritance the children of
strangers, or enemies; and the public sale of justice, as well as
of favor, which he instituted in the palace of Constantinople.
The ambitious candidate eagerly solicited, at the expense of the
fairest part of his patrimony, the honors and emoluments of some
provincial government; the lives and fortunes of the unhappy
people were abandoned to the most liberal purchaser; and the
public discontent was sometimes appeased by the sacrifice of an
unpopular criminal, whose punishment was profitable only to the
præfect of the East, his accomplice and his judge. If avarice
were not the blindest of the human passions, the motives of
Rufinus might excite our curiosity; and we might be tempted to
inquire with what view he violated every principle of humanity
and justice, to accumulate those immense treasures, which he
could not spend without folly, nor possess without danger.
Perhaps he vainly imagined, that he labored for the interest of
an only daughter, on whom he intended to bestow his royal pupil,
and the august rank of Empress of the East. Perhaps he deceived
himself by the opinion, that his avarice was the instrument of
his ambition. He aspired to place his fortune on a secure and
independent basis, which should no longer depend on the caprice
of the young emperor; yet he neglected to conciliate the hearts
of the soldiers and people, by the liberal distribution of those
riches, which he had acquired with so much toil, and with so much
guilt. The extreme parsimony of Rufinus left him only the
reproach and envy of ill-gotten wealth; his dependants served him
without attachment; the universal hatred of mankind was repressed
only by the influence of servile fear. The fate of Lucian
proclaimed to the East, that the præfect, whose industry was
much abated in the despatch of ordinary business, was active and
indefatigable in the pursuit of revenge. Lucian, the son of the
præfect Florentius, the oppressor of Gaul, and the enemy of
Julian, had employed a considerable part of his inheritance, the
fruit of rapine and corruption, to purchase the friendship of
Rufinus, and the high office of Count of the East. But the new
magistrate imprudently departed from the maxims of the court, and
of the times; disgraced his benefactor by the contrast of a
virtuous and temperate administration; and presumed to refuse an
act of injustice, which might have tended to the profit of the
emperor’s uncle. Arcadius was easily persuaded to resent the
supposed insult; and the præfect of the East resolved to execute
in person the cruel vengeance, which he meditated against this
ungrateful delegate of his power. He performed with incessant
speed the journey of seven or eight hundred miles, from
Constantinople to Antioch, entered the capital of Syria at the
dead of night, and spread universal consternation among a people
ignorant of his design, but not ignorant of his character. The
Count of the fifteen provinces of the East was dragged, like the
vilest malefactor, before the arbitrary tribunal of Rufinus.
Notwithstanding the clearest evidence of his integrity, which was
not impeached even by the voice of an accuser, Lucian was
condemned, almost with out a trial, to suffer a cruel and
ignominious punishment. The ministers of the tyrant, by the
orders, and in the presence, of their master, beat him on the
neck with leather thongs armed at the extremities with lead; and
when he fainted under the violence of the pain, he was removed in
a close litter, to conceal his dying agonies from the eyes of the
indignant city. No sooner had Rufinus perpetrated this inhuman
act, the sole object of his expedition, than he returned, amidst
the deep and silent curses of a trembling people, from Antioch to
Constantinople; and his diligence was accelerated by the hope of
accomplishing, without delay, the nuptials of his daughter with
the emperor of the East.\textsuperscript{12}

\pagenote[10]{Montesquieu (Esprit des Loix, l. xii. c. 12)
praises one of the laws of Theodosius addressed to the præfect
Rufinus, (l. ix. tit. iv. leg. unic.,) to discourage the
prosecution of treasonable, or sacrilegious, words. A tyrannical
statute always proves the existence of tyranny; but a laudable
edict may only contain the specious professions, or ineffectual
wishes, of the prince, or his ministers. This, I am afraid, is a
just, though mortifying, canon of criticism.}

\pagenote[11]{

—fluctibus auri Expleri sitis ista nequit— ***** Congestae
cumulantur opes; orbisque ruinas Accipit una domus.

This character (Claudian, in. Rufin. i. 184-220) is confirmed by
Jerom, a disinterested witness, (dedecus insatiabilis avaritiae,
tom. i. ad Heliodor. p. 26,) by Zosimus, (l. v. p. 286,) and by
Suidas, who copied the history of Eunapius.}

\pagenote[12]{

—Caetera segnis; Ad facinus velox; penitus regione remotas Impiger
ire vias.

This allusion of Claudian (in Rufin. i. 241) is again explained
by the circumstantial narrative of Zosimus, (l. v. p. 288, 289.)}

But Rufinus soon experienced, that a prudent minister should
constantly secure his royal captive by the strong, though
invisible chain of habit; and that the merit, and much more
easily the favor, of the absent, are obliterated in a short time
from the mind of a weak and capricious sovereign. While the
præfect satiated his revenge at Antioch, a secret conspiracy of
the favorite eunuchs, directed by the great chamberlain
Eutropius, undermined his power in the palace of Constantinople.
They discovered that Arcadius was not inclined to love the
daughter of Rufinus, who had been chosen, without his consent,
for his bride; and they contrived to substitute in her place the
fair Eudoxia, the daughter of Bauto,\textsuperscript{13} a general of the Franks
in the service of Rome; and who was educated, since the death of
her father, in the family of the sons of Promotus. The young
emperor, whose chastity had been strictly guarded by the pious
care of his tutor Arsenius,\textsuperscript{14} eagerly listened to the artful and
flattering descriptions of the charms of Eudoxia: he gazed with
impatient ardor on her picture, and he understood the necessity
of concealing his amorous designs from the knowledge of a
minister who was so deeply interested to oppose the consummation
of his happiness. Soon after the return of Rufinus, the
approaching ceremony of the royal nuptials was announced to the
people of Constantinople, who prepared to celebrate, with false
and hollow acclamations, the fortune of his daughter. A splendid
train of eunuchs and officers issued, in hymeneal pomp, from the
gates of the palace; bearing aloft the diadem, the robes, and the
inestimable ornaments, of the future empress. The solemn
procession passed through the streets of the city, which were
adorned with garlands, and filled with spectators; but when it
reached the house of the sons of Promotus, the principal eunuch
respectfully entered the mansion, invested the fair Eudoxia with
the Imperial robes, and conducted her in triumph to the palace
and bed of Arcadius.\textsuperscript{15} The secrecy and success with which this
conspiracy against Rufinus had been conducted, imprinted a mark
of indelible ridicule on the character of a minister, who had
suffered himself to be deceived, in a post where the arts of
deceit and dissimulation constitute the most distinguished merit.
He considered, with a mixture of indignation and fear, the
victory of an aspiring eunuch, who had secretly captivated the
favor of his sovereign; and the disgrace of his daughter, whose
interest was inseparably connected with his own, wounded the
tenderness, or, at least, the pride of Rufinus. At the moment
when he flattered himself that he should become the father of a
line of kings, a foreign maid, who had been educated in the house
of his implacable enemies, was introduced into the Imperial bed;
and Eudoxia soon displayed a superiority of sense and spirit, to
improve the ascendant which her beauty must acquire over the mind
of a fond and youthful husband. The emperor would soon be
instructed to hate, to fear, and to destroy the powerful subject,
whom he had injured; and the consciousness of guilt deprived
Rufinus of every hope, either of safety or comfort, in the
retirement of a private life. But he still possessed the most
effectual means of defending his dignity, and perhaps of
oppressing his enemies. The præfect still exercised an
uncontrolled authority over the civil and military government of
the East; and his treasures, if he could resolve to use them,
might be employed to procure proper instruments for the execution
of the blackest designs, that pride, ambition, and revenge could
suggest to a desperate statesman. The character of Rufinus seemed
to justify the accusations that he conspired against the person
of his sovereign, to seat himself on the vacant throne; and that
he had secretly invited the Huns and the Goths to invade the
provinces of the empire, and to increase the public confusion.
The subtle præfect, whose life had been spent in the intrigues
of the palace, opposed, with equal arms, the artful measures of
the eunuch Eutropius; but the timid soul of Rufinus was
astonished by the hostile approach of a more formidable rival, of
the great Stilicho, the general, or rather the master, of the
empire of the West.\textsuperscript{16}

\pagenote[13]{Zosimus (l. iv. p. 243) praises the valor,
prudence, and integrity of Bauto the Frank. See Tillemont, Hist.
des Empereurs, tom. v. p. 771.}

\pagenote[14]{Arsenius escaped from the palace of Constantinople,
and passed fifty-five years in rigid penance in the monasteries
of Egypt. See Tillemont, Mem. Eccles. tom. xiv. p. 676-702; and
Fleury, Hist Eccles. tom. v. p. 1, \&c.; but the latter, for want
of authentic materials, has given too much credit to the legend
of Metaphrastes.}

\pagenote[15]{This story (Zosimus, l. v. p. 290) proves that the
hymeneal rites of antiquity were still practised, without
idolatry, by the Christians of the East; and the bride was
forcibly conducted from the house of her parents to that of her
husband. Our form of marriage requires, with less delicacy, the
express and public consent of a virgin.}

\pagenote[16]{Zosimus, (l. v. p. 290,) Orosius, (l. vii. c. 37,)
and the Chronicle of Marcellinus. Claudian (in Rufin. ii. 7-100)
paints, in lively colors, the distress and guilt of the
præfect.}

The celestial gift, which Achilles obtained, and Alexander
envied, of a poet worthy to celebrate the actions of heroes has
been enjoyed by Stilicho, in a much higher degree than might have
been expected from the declining state of genius, and of art. The
muse of Claudian,\textsuperscript{17} devoted to his service, was always prepared
to stigmatize his adversaries, Rufinus, or Eutropius, with
eternal infamy; or to paint, in the most splendid colors, the
victories and virtues of a powerful benefactor. In the review of
a period indifferently supplied with authentic materials, we
cannot refuse to illustrate the annals of Honorius, from the
invectives, or the panegyrics, of a contemporary writer; but as
Claudian appears to have indulged the most ample privilege of a
poet and a courtier, some criticism will be requisite to
translate the language of fiction or exaggeration, into the truth
and simplicity of historic prose. His silence concerning the
family of Stilicho may be admitted as a proof, that his patron
was neither able, nor desirous, to boast of a long series of
illustrious progenitors; and the slight mention of his father, an
officer of Barbarian cavalry in the service of Valens, seems to
countenance the assertion, that the general, who so long
commanded the armies of Rome, was descended from the savage and
perfidious race of the Vandals.\textsuperscript{18} If Stilicho had not possessed
the external advantages of strength and stature, the most
flattering bard, in the presence of so many thousand spectators,
would have hesitated to affirm, that he surpassed the measure of
the demi-gods of antiquity; and that whenever he moved, with
lofty steps, through the streets of the capital, the astonished
crowd made room for the stranger, who displayed, in a private
condition, the awful majesty of a hero. From his earliest youth
he embraced the profession of arms; his prudence and valor were
soon distinguished in the field; the horsemen and archers of the
East admired his superior dexterity; and in each degree of his
military promotions, the public judgment always prevented and
approved the choice of the sovereign. He was named, by
Theodosius, to ratify a solemn treaty with the monarch of Persia;
he supported, during that important embassy, the dignity of the
Roman name; and after he returned to Constantinople, his merit
was rewarded by an intimate and honorable alliance with the
Imperial family. Theodosius had been prompted, by a pious motive
of fraternal affection, to adopt, for his own, the daughter of
his brother Honorius; the beauty and accomplishments of Serena\textsuperscript{19}
were universally admired by the obsequious court; and Stilicho
obtained the preference over a crowd of rivals, who ambitiously
disputed the hand of the princess, and the favor of her adopted
father.\textsuperscript{20} The assurance that the husband of Serena would be
faithful to the throne, which he was permitted to approach,
engaged the emperor to exalt the fortunes, and to employ the
abilities, of the sagacious and intrepid Stilicho. He rose,
through the successive steps of master of the horse, and count of
the domestics, to the supreme rank of master-general of all the
cavalry and infantry of the Roman, or at least of the Western,
empire;\textsuperscript{21} and his enemies confessed, that he invariably
disdained to barter for gold the rewards of merit, or to defraud
the soldiers of the pay and gratifications which they deserved or
claimed, from the liberality of the state.\textsuperscript{22} The valor and
conduct which he afterwards displayed, in the defence of Italy,
against the arms of Alaric and Radagaisus, may justify the fame
of his early achievements and in an age less attentive to the
laws of honor, or of pride, the Roman generals might yield the
preeminence of rank, to the ascendant of superior genius.\textsuperscript{23} He
lamented, and revenged, the murder of Promotus, his rival and his
friend; and the massacre of many thousands of the flying
Bastarnae is represented by the poet as a bloody sacrifice, which
the Roman Achilles offered to the manes of another Patroclus. The
virtues and victories of Stilicho deserved the hatred of Rufinus:
and the arts of calumny might have been successful if the tender
and vigilant Serena had not protected her husband against his
domestic foes, whilst he vanquished in the field the enemies of
the empire.\textsuperscript{24} Theodosius continued to support an unworthy
minister, to whose diligence he delegated the government of the
palace, and of the East; but when he marched against the tyrant
Eugenius, he associated his faithful general to the labors and
glories of the civil war; and in the last moments of his life,
the dying monarch recommended to Stilicho the care of his sons,
and of the republic.\textsuperscript{25} The ambition and the abilities of
Stilicho were not unequal to the important trust; and he claimed
the guardianship of the two empires, during the minority of
Arcadius and Honorius.\textsuperscript{26} The first measure of his
administration, or rather of his reign, displayed to the nations
the vigor and activity of a spirit worthy to command. He passed
the Alps in the depth of winter; descended the stream of the
Rhine, from the fortress of Basil to the marshes of Batavia;
reviewed the state of the garrisons; repressed the enterprises of
the Germans; and, after establishing along the banks a firm and
honorable peace, returned, with incredible speed, to the palace
of Milan.\textsuperscript{27} The person and court of Honorius were subject to the
master-general of the West; and the armies and provinces of
Europe obeyed, without hesitation, a regular authority, which was
exercised in the name of their young sovereign. Two rivals only
remained to dispute the claims, and to provoke the vengeance, of
Stilicho. Within the limits of Africa, Gildo, the Moor,
maintained a proud and dangerous independence; and the minister
of Constantinople asserted his equal reign over the emperor, and
the empire, of the East.

\pagenote[17]{Stilicho, directly or indirectly, is the perpetual
theme of Claudian. The youth and private life of the hero are
vaguely expressed in the poem on his first consulship, 35-140.}

\pagenote[18]{Vandalorum, imbellis, avarae, perfidae, et dolosae,
gentis, genere editus. Orosius, l. vii. c. 38. Jerom (tom. i. ad
Gerontiam, p. 93) call him a Semi-Barbarian.}

\pagenote[19]{Claudian, in an imperfect poem, has drawn a fair,
perhaps a flattering, portrait of Serena. That favorite niece of
Theodosius was born, as well as here sister Thermantia, in Spain;
from whence, in their earliest youth, they were honorably
conducted to the palace of Constantinople.}

\pagenote[20]{Some doubt may be entertained, whether this
adoption was legal or only metaphorical, (see Ducange, Fam.
Byzant. p. 75.) An old inscription gives Stilicho the singular
title of Pro-gener Divi Theodosius}

\pagenote[21]{Claudian (Laus Serenae, 190, 193) expresses, in
poetic language “the dilectus equorum,” and the “gemino mox idem
culmine duxit agmina.” The inscription adds, “count of the
domestics,” an important command, which Stilicho, in the height
of his grandeur, might prudently retain.}

\pagenote[22]{The beautiful lines of Claudian (in i. Cons.
Stilich. ii. 113) displays his genius; but the integrity of
Stilicho (in the military administration) is much more firmly
established by the unwilling evidence of Zosimus, (l. v. p.
345.)}

\pagenote[23]{—Si bellica moles Ingrueret, quamvis annis et jure
minori,

Cedere grandaevos equitum peditumque magistros

Adspiceres. Claudian, Laus Seren. p. 196, \&c. A modern general
would deem their submission either heroic patriotism or abject
servility.}

\pagenote[24]{Compare the poem on the first consulship (i.
95-115) with the Laus Serenoe (227-237, where it unfortunately
breaks off.) We may perceive the deep, inveterate malice of
Rufinus.}

\pagenote[25]{—Quem fratribus ipse Discedens, clypeum
defensoremque dedisti. Yet the nomination (iv. Cons. Hon. 432)
was private, (iii. Cons. Hon. 142,) cunctos discedere... jubet;
and may therefore be suspected. Zosimus and Suidas apply to
Stilicho and Rufinus the same equal title of guardians, or
procurators.}

\pagenote[26]{The Roman law distinguishes two sorts of minority,
which expired at the age of fourteen, and of twenty-five. The one
was subject to the tutor, or guardian, of the person; the other,
to the curator, or trustee, of the estate, (Heineccius,
Antiquitat. Rom. ad Jurisprudent. pertinent. l. i. tit. xxii.
xxiii. p. 218-232.) But these legal ideas were never accurately
transferred into the constitution of an elective monarchy.}

\pagenote[27]{See Claudian, (i. Cons. Stilich. i. 188-242;) but
he must allow more than fifteen days for the journey and return
between Milan and Leyden.}

\section{Part \thesection.}

The impartiality which Stilicho affected, as the common guardian
of the royal brothers, engaged him to regulate the equal division
of the arms, the jewels, and the magnificent wardrobe and
furniture of the deceased emperor.\textsuperscript{28} But the most important
object of the inheritance consisted of the numerous legions,
cohorts, and squadrons, of Romans, or Barbarians, whom the event
of the civil war had united under the standard of Theodosius. The
various multitudes of Europe and Asia, exasperated by recent
animosities, were overawed by the authority of a single man; and
the rigid discipline of Stilicho protected the lands of the
citizens from the rapine of the licentious soldier.\textsuperscript{29} Anxious,
however, and impatient, to relieve Italy from the presence of
this formidable host, which could be useful only on the frontiers
of the empire, he listened to the just requisition of the
minister of Arcadius, declared his intention of reconducting in
person the troops of the East, and dexterously employed the rumor
of a Gothic tumult to conceal his private designs of ambition and
revenge.\textsuperscript{30} The guilty soul of Rufinus was alarmed by the
approach of a warrior and a rival, whose enmity he deserved; he
computed, with increasing terror, the narrow space of his life
and greatness; and, as the last hope of safety, he interposed the
authority of the emperor Arcadius. Stilicho, who appears to have
directed his march along the sea-coast of the Adriatic, was not
far distant from the city of Thessalonica, when he received a
peremptory message, to recall the troops of the East, and to
declare, that his nearer approach would be considered, by the
Byzantine court, as an act of hostility. The prompt and
unexpected obedience of the general of the West, convinced the
vulgar of his loyalty and moderation; and, as he had already
engaged the affection of the Eastern troops, he recommended to
their zeal the execution of his bloody design, which might be
accomplished in his absence, with less danger, perhaps, and with
less reproach. Stilicho left the command of the troops of the
East to Gainas, the Goth, on whose fidelity he firmly relied,
with an assurance, at least, that the hardy Barbarians would
never be diverted from his purpose by any consideration of fear
or remorse. The soldiers were easily persuaded to punish the
enemy of Stilicho and of Rome; and such was the general hatred
which Rufinus had excited, that the fatal secret, communicated to
thousands, was faithfully preserved during the long march from
Thessalonica to the gates of Constantinople. As soon as they had
resolved his death, they condescended to flatter his pride; the
ambitious præfect was seduced to believe, that those powerful
auxiliaries might be tempted to place the diadem on his head; and
the treasures which he distributed, with a tardy and reluctant
hand, were accepted by the indignant multitude as an insult,
rather than as a gift. At the distance of a mile from the
capital, in the field of Mars, before the palace of Hebdomon, the
troops halted: and the emperor, as well as his minister,
advanced, according to ancient custom, respectfully to salute the
power which supported their throne. As Rufinus passed along the
ranks, and disguised, with studied courtesy, his innate
haughtiness, the wings insensibly wheeled from the right and
left, and enclosed the devoted victim within the circle of their
arms. Before he could reflect on the danger of his situation,
Gainas gave the signal of death; a daring and forward soldier
plunged his sword into the breast of the guilty præfect, and
Rufinus fell, groaned, and expired, at the feet of the affrighted
emperor. If the agonies of a moment could expiate the crimes of a
whole life, or if the outrages inflicted on a breathless corpse
could be the object of pity, our humanity might perhaps be
affected by the horrid circumstances which accompanied the murder
of Rufinus. His mangled body was abandoned to the brutal fury of
the populace of either sex, who hastened in crowds, from every
quarter of the city, to trample on the remains of the haughty
minister, at whose frown they had so lately trembled. His right
hand was cut off, and carried through the streets of
Constantinople, in cruel mockery, to extort contributions for the
avaricious tyrant, whose head was publicly exposed, borne aloft
on the point of a long lance.\textsuperscript{31} According to the savage maxims
of the Greek republics, his innocent family would have shared the
punishment of his crimes. The wife and daughter of Rufinus were
indebted for their safety to the influence of religion. Her
sanctuary protected them from the raging madness of the people;
and they were permitted to spend the remainder of their lives in
the exercise of Christian devotions, in the peaceful retirement
of Jerusalem.\textsuperscript{32}

\pagenote[28]{I. Cons. Stilich. ii. 88-94. Not only the robes and
diadems of the deceased emperor, but even the helmets,
sword-hilts, belts, rasses, \&c., were enriched with pearls,
emeralds, and diamonds.}

\pagenote[29]{—Tantoque remoto Principe, mutatas orbis non sensit
habenas. This high commendation (i. Cons. Stil. i. 149) may be
justified by the fears of the dying emperor, (de Bell. Gildon.
292-301;) and the peace and good order which were enjoyed after
his death, (i. Cons. Stil i. 150-168.)}

\pagenote[30]{Stilicho’s march, and the death of Rufinus, are
described by Claudian, (in Rufin. l. ii. 101-453, Zosimus, l. v.
p. 296, 297,) Sozomen (l. viii. c. 1,) Socrates, l. vi. c. 1,)
Philostorgius, (l. xi c. 3, with Godefory, p. 441,) and the
Chronicle of Marcellinus.}

\pagenote[31]{The dissection of Rufinus, which Claudian performs
with the savage coolness of an anatomist, (in Rufin. ii.
405-415,) is likewise specified by Zosimus and Jerom, (tom. i. p.
26.)}

\pagenote[32]{The Pagan Zosimus mentions their sanctuary and
pilgrimage. The sister of Rufinus, Sylvania, who passed her life
at Jerusalem, is famous in monastic history. 1. The studious
virgin had diligently, and even repeatedly, perused the
commentators on the Bible, Origen, Gregory, Basil, \&c., to the
amount of five millions of lines. 2. At the age of threescore,
she could boast, that she had never washed her hands, face, or
any part of her whole body, except the tips of her fingers to
receive the communion. See the Vitae Patrum, p. 779, 977.}

The servile poet of Stilicho applauds, with ferocious joy, this
horrid deed, which, in the execution, perhaps, of justice,
violated every law of nature and society, profaned the majesty of
the prince, and renewed the dangerous examples of military
license. The contemplation of the universal order and harmony had
satisfied Claudian of the existence of the Deity; but the
prosperous impunity of vice appeared to contradict his moral
attributes; and the fate of Rufinus was the only event which
could dispel the religious doubts of the poet.\textsuperscript{33} Such an act
might vindicate the honor of Providence, but it did not much
contribute to the happiness of the people. In less than three
months they were informed of the maxims of the new
administration, by a singular edict, which established the
exclusive right of the treasury over the spoils of Rufinus; and
silenced, under heavy penalties, the presumptuous claims of the
subjects of the Eastern empire, who had been injured by his
rapacious tyranny.\textsuperscript{34} Even Stilicho did not derive from the
murder of his rival the fruit which he had proposed; and though
he gratified his revenge, his ambition was disappointed. Under
the name of a favorite, the weakness of Arcadius required a
master, but he naturally preferred the obsequious arts of the
eunuch Eutropius, who had obtained his domestic confidence: and
the emperor contemplated, with terror and aversion, the stern
genius of a foreign warrior. Till they were divided by the
jealousy of power, the sword of Gainas, and the charms of
Eudoxia, supported the favor of the great chamberlain of the
palace: the perfidious Goth, who was appointed master-general of
the East, betrayed, without scruple, the interest of his
benefactor; and the same troops, who had so lately massacred the
enemy of Stilicho, were engaged to support, against him, the
independence of the throne of Constantinople. The favorites of
Arcadius fomented a secret and irreconcilable war against a
formidable hero, who aspired to govern, and to defend, the two
empires of Rome, and the two sons of Theodosius. They incessantly
labored, by dark and treacherous machinations, to deprive him of
the esteem of the prince, the respect of the people, and the
friendship of the Barbarians. The life of Stilicho was repeatedly
attempted by the dagger of hired assassins; and a decree was
obtained from the senate of Constantinople, to declare him an
enemy of the republic, and to confiscate his ample possessions in
the provinces of the East. At a time when the only hope of
delaying the ruin of the Roman name depended on the firm union,
and reciprocal aid, of all the nations to whom it had been
gradually communicated, the subjects of Arcadius and Honorius
were instructed, by their respective masters, to view each other
in a foreign, and even hostile, light; to rejoice in their mutual
calamities, and to embrace, as their faithful allies, the
Barbarians, whom they excited to invade the territories of their
countrymen.\textsuperscript{35} The natives of Italy affected to despise the
servile and effeminate Greeks of Byzantium, who presumed to
imitate the dress, and to usurp the dignity, of Roman senators;\textsuperscript{36}
and the Greeks had not yet forgot the sentiments of hatred and
contempt, which their polished ancestors had so long entertained
for the rude inhabitants of the West. The distinction of two
governments, which soon produced the separation of two nations,
will justify my design of suspending the series of the Byzantine
history, to prosecute, without interruption, the disgraceful, but
memorable, reign of Honorius.

\pagenote[33]{See the beautiful exordium of his invective against
Rufinus, which is curiously discussed by the sceptic Bayle,
Dictionnaire Critique, Rufin. Not. E.}

\pagenote[34]{See the Theodosian Code, l. ix. tit. xlii. leg. 14,
15. The new ministers attempted, with inconsistent avarice, to
seize the spoils of their predecessor, and to provide for their
own future security.}

\pagenote[35]{See Claudian, (i. Cons. Stilich, l. i. 275, 292,
296, l. ii. 83,) and Zosimus, (l. v. p. 302.)}

\pagenote[36]{Claudian turns the consulship of the eunuch
Eutropius into a national reflection, (l. ii. 134):—

—-Plaudentem cerne senatum, Et Byzantinos proceres Graiosque
Quirites: O patribus plebes, O digni consule patres.

It is curious to observe the first symptoms of jealousy and
schism between old and new Rome, between the Greeks and Latins.}

The prudent Stilicho, instead of persisting to force the
inclinations of a prince, and people, who rejected his
government, wisely abandoned Arcadius to his unworthy favorites;
and his reluctance to involve the two empires in a civil war
displayed the moderation of a minister, who had so often
signalized his military spirit and abilities. But if Stilicho had
any longer endured the revolt of Africa, he would have betrayed
the security of the capital, and the majesty of the Western
emperor, to the capricious insolence of a Moorish rebel. Gildo,\textsuperscript{37}
the brother of the tyrant Firmus, had preserved and obtained,
as the reward of his apparent fidelity, the immense patrimony
which was forfeited by treason: long and meritorious service, in
the armies of Rome, raised him to the dignity of a military
count; the narrow policy of the court of Theodosius had adopted
the mischievous expedient of supporting a legal government by the
interest of a powerful family; and the brother of Firmus was
invested with the command of Africa. His ambition soon usurped
the administration of justice, and of the finances, without
account, and without control; and he maintained, during a reign
of twelve years, the possession of an office, from which it was
impossible to remove him, without the danger of a civil war.
During those twelve years, the provinces of Africa groaned under
the dominion of a tyrant, who seemed to unite the unfeeling
temper of a stranger with the partial resentments of domestic
faction. The forms of law were often superseded by the use of
poison; and if the trembling guests, who were invited to the
table of Gildo, presumed to express fears, the insolent suspicion
served only to excite his fury, and he loudly summoned the
ministers of death. Gildo alternately indulged the passions of
avarice and lust;\textsuperscript{38} and if his days were terrible to the rich,
his nights were not less dreadful to husbands and parents. The
fairest of their wives and daughters were prostituted to the
embraces of the tyrant; and afterwards abandoned to a ferocious
troop of Barbarians and assassins, the black, or swarthy, natives
of the desert; whom Gildo considered as the only guardians of his
throne. In the civil war between Theodosius and Eugenius, the
count, or rather the sovereign, of Africa, maintained a haughty
and suspicious neutrality; refused to assist either of the
contending parties with troops or vessels, expected the
declaration of fortune, and reserved for the conqueror the vain
professions of his allegiance. Such professions would not have
satisfied the master of the Roman world; but the death of
Theodosius, and the weakness and discord of his sons, confirmed
the power of the Moor; who condescended, as a proof of his
moderation, to abstain from the use of the diadem, and to supply
Rome with the customary tribute, or rather subsidy, of corn. In
every division of the empire, the five provinces of Africa were
invariably assigned to the West; and Gildo had to govern that
extensive country in the name of Honorius, but his knowledge of
the character and designs of Stilicho soon engaged him to address
his homage to a more distant and feeble sovereign. The ministers
of Arcadius embraced the cause of a perfidious rebel; and the
delusive hope of adding the numerous cities of Africa to the
empire of the East, tempted them to assert a claim, which they
were incapable of supporting, either by reason or by arms.\textsuperscript{39}

\pagenote[37]{Claudian may have exaggerated the vices of Gildo;
but his Moorish extraction, his notorious actions, and the
complaints of St. Augustin, may justify the poet’s invectives.
Baronius (Annal. Eccles. A.D. 398, No. 35-56) has treated the
African rebellion with skill and learning.}

\pagenote[38]{

Instat terribilis vivis, morientibus haeres, Virginibus raptor,
thalamis obscoenus adulter. Nulla quies: oritur praeda cessante
libido, Divitibusque dies, et nox metuenda maritis. Mauris
clarissima quaeque Fastidita datur. ——De Bello Gildonico, 165,
189.

Baronius condemns, still more severely, the licentiousness of
Gildo; as his wife, his daughter, and his sister, were examples
of perfect chastity. The adulteries of the African soldiers are
checked by one of the Imperial laws.}

\pagenote[39]{Inque tuam sortem numerosas transtulit urbes.
Claudian (de Bell. Gildonico, 230-324) has touched, with
political delicacy, the intrigues of the Byzantine court, which
are likewise mentioned by Zosimus, (l. v. p. 302.)}

When Stilicho had given a firm and decisive answer to the
pretensions of the Byzantine court, he solemnly accused the
tyrant of Africa before the tribunal, which had formerly judged
the kings and nations of the earth; and the image of the republic
was revived, after a long interval, under the reign of Honorius.
The emperor transmitted an accurate and ample detail of the
complaints of the provincials, and the crimes of Gildo, to the
Roman senate; and the members of that venerable assembly were
required to pronounce the condemnation of the rebel. Their
unanimous suffrage declared him the enemy of the republic; and
the decree of the senate added a sacred and legitimate sanction
to the Roman arms.\textsuperscript{40} A people, who still remembered that their
ancestors had been the masters of the world, would have
applauded, with conscious pride, the representation of ancient
freedom; if they had not since been accustomed to prefer the
solid assurance of bread to the unsubstantial visions of liberty
and greatness. The subsistence of Rome depended on the harvests
of Africa; and it was evident, that a declaration of war would be
the signal of famine. The præfect Symmachus, who presided in the
deliberations of the senate, admonished the minister of his just
apprehension, that as soon as the revengeful Moor should prohibit
the exportation of corn, tranquility and perhaps the safety, of
the capital would be threatened by the hungry rage of a turbulent
multitude.\textsuperscript{41} The prudence of Stilicho conceived and executed,
without delay, the most effectual measure for the relief of the
Roman people. A large and seasonable supply of corn, collected in
the inland provinces of Gaul, was embarked on the rapid stream of
the Rhone, and transported, by an easy navigation, from the Rhone
to the Tyber. During the whole term of the African war, the
granaries of Rome were continually filled, her dignity was
vindicated from the humiliating dependence, and the minds of an
immense people were quieted by the calm confidence of peace and
plenty.\textsuperscript{42}

\pagenote[40]{Symmachus (l. iv. epist. 4) expresses the judicial
forms of the senate; and Claudian (i. Cons. Stilich. l. i. 325,
\&c.) seems to feel the spirit of a Roman.}

\pagenote[41]{Claudian finely displays these complaints of
Symmachus, in a speech of the goddess of Rome, before the throne
of Jupiter, (de Bell Gildon. 28-128.)}

\pagenote[42]{See Claudian (in Eutrop. l. i 401, \&c. i. Cons.
Stil. l. i. 306, \&c. i. Cons. Stilich. 91, \&c.)}

The cause of Rome, and the conduct of the African war, were
intrusted by Stilicho to a general, active and ardent to avenge
his private injuries on the head of the tyrant. The spirit of
discord which prevailed in the house of Nabal, had excited a
deadly quarrel between two of his sons, Gildo and Mascezel.\textsuperscript{43}
The usurper pursued, with implacable rage, the life of his
younger brother, whose courage and abilities he feared; and
Mascezel, oppressed by superior power, took refuge in the court
of Milan, where he soon received the cruel intelligence that his
two innocent and helpless children had been murdered by their
inhuman uncle. The affliction of the father was suspended only by
the desire of revenge. The vigilant Stilicho already prepared to
collect the naval and military force of the Western empire; and
he had resolved, if the tyrant should be able to wage an equal
and doubtful war, to march against him in person. But as Italy
required his presence, and as it might be dangerous to weaken the
defence of the frontier, he judged it more advisable, that
Mascezel should attempt this arduous adventure at the head of a
chosen body of Gallic veterans, who had lately served under the
standard of Eugenius. These troops, who were exhorted to convince
the world that they could subvert, as well as defend the throne
of a usurper, consisted of the Jovian, the Herculian, and the
Augustan legions; of the Nervian auxiliaries; of the soldiers who
displayed in their banners the symbol of a lion, and of the
troops which were distinguished by the auspicious names of
Fortunate, and Invincible. Yet such was the smallness of their
establishments, or the difficulty of recruiting, that these seven
bands,\textsuperscript{44} of high dignity and reputation in the service of Rome,
amounted to no more than five thousand effective men.\textsuperscript{45} The
fleet of galleys and transports sailed in tempestuous weather
from the port of Pisa, in Tuscany, and steered their course to
the little island of Capraria; which had borrowed that name from
the wild goats, its original inhabitants, whose place was
occupied by a new colony of a strange and savage appearance. “The
whole island (says an ingenious traveller of those times) is
filled, or rather defiled, by men who fly from the light. They
call themselves Monks, or solitaries, because they choose to live
alone, without any witnesses of their actions. They fear the
gifts of fortune, from the apprehension of losing them; and, lest
they should be miserable, they embrace a life of voluntary
wretchedness. How absurd is their choice! how perverse their
understanding! to dread the evils, without being able to support
the blessings, of the human condition. Either this melancholy
madness is the effect of disease, or exercise on their own bodies
the tortures which are inflicted on fugitive slaves by the hand
of justice.”\textsuperscript{46} Such was the contempt of a profane magistrate for
the monks as the chosen servants of God.\textsuperscript{47} Some of them were
persuaded, by his entreaties, to embark on board the fleet; and
it is observed, to the praise of the Roman general, that his days
and nights were employed in prayer, fasting, and the occupation
of singing psalms. The devout leader, who, with such a
reenforcement, appeared confident of victory, avoided the
dangerous rocks of Corsica, coasted along the eastern side of
Sardinia, and secured his ships against the violence of the south
wind, by casting anchor in the and capacious harbor of Cagliari,
at the distance of one hundred and forty miles from the African
shores.\textsuperscript{48}

\pagenote[43]{He was of a mature age; since he had formerly (A.D.
373) served against his brother Firmus (Ammian. xxix. 5.)
Claudian, who understood the court of Milan, dwells on the
injuries, rather than the merits, of Mascezel, (de Bell. Gild.
389-414.) The Moorish war was not worthy of Honorius, or
Stilicho, \&c.}

\pagenote[44]{Claudian, Bell. Gild. 415-423. The change of
discipline allowed him to use indifferently the names of Legio
Cohors, Manipulus. See Notitia Imperii, S. 38, 40.}

\pagenote[45]{Orosius (l. vii. c. 36, p. 565) qualifies this
account with an expression of doubt, (ut aiunt;) and it scarcely
coincides with Zosimus, (l. v. p. 303.) Yet Claudian, after some
declamation about Cadmus, soldiers, frankly owns that Stilicho
sent a small army lest the rebels should fly, ne timeare times,
(i. Cons. Stilich. l. i. 314 \&c.)}

\pagenote[46]{Claud. Rutil. Numatian. Itinerar. i. 439-448. He
afterwards (515-526) mentions a religious madman on the Isle of
Gorgona. For such profane remarks, Rutilius and his accomplices
are styled, by his commentator, Barthius, rabiosi canes diaboli.
Tillemont (Mem. Eccles com. xii. p. 471) more calmly observes,
that the unbelieving poet praises where he means to censure.}

\pagenote[47]{Orosius, l. vii. c. 36, p. 564. Augustin commends
two of these savage saints of the Isle of Goats, (epist. lxxxi.
apud Tillemont, Mem. Eccles. tom. xiii. p. 317, and Baronius,
Annal Eccles. A.D. 398 No. 51.)}

\pagenote[48]{Here the first book of the Gildonic war is
terminated. The rest of Claudian’s poem has been lost; and we are
ignorant how or where the army made good their landing in Afica.}

Gildo was prepared to resist the invasion with all the forces of
Africa. By the liberality of his gifts and promises, he
endeavored to secure the doubtful allegiance of the Roman
soldiers, whilst he attracted to his standard the distant tribes
of Gaetulia and Æthiopia. He proudly reviewed an army of seventy
thousand men, and boasted, with the rash presumption which is the
forerunner of disgrace, that his numerous cavalry would trample
under their horses’ feet the troops of Mascezel, and involve, in
a cloud of burning sand, the natives of the cold regions of Gaul
and Germany.\textsuperscript{49} But the Moor, who commanded the legions of
Honorius, was too well acquainted with the manners of his
countrymen, to entertain any serious apprehension of a naked and
disorderly host of Barbarians; whose left arm, instead of a
shield, was protected only by mantle; who were totally disarmed
as soon as they had darted their javelin from their right hand;
and whose horses had never been in combat. He fixed his camp of
five thousand veterans in the face of a superior enemy, and,
after the delay of three days, gave the signal of a general
engagement.\textsuperscript{50} As Mascezel advanced before the front with fair
offers of peace and pardon, he encountered one of the foremost
standard-bearers of the Africans, and, on his refusal to yield,
struck him on the arm with his sword. The arm, and the standard,
sunk under the weight of the blow; and the imaginary act of
submission was hastily repeated by all the standards of the line.
At this the disaffected cohorts proclaimed the name of their
lawful sovereign; the Barbarians, astonished by the defection of
their Roman allies, dispersed, according to their custom, in
tumultuary flight; and Mascezel obtained honors the of an easy,
and almost bloodless, victory.\textsuperscript{51} The tyrant escaped from the
field of battle to the sea-shore; and threw himself into a small
vessel, with the hope of reaching in safety some friendly port of
the empire of the East; but the obstinacy of the wind drove him
back into the harbor of Tabraca,\textsuperscript{52} which had acknowledged, with
the rest of the province, the dominion of Honorius, and the
authority of his lieutenant. The inhabitants, as a proof of their
repentance and loyalty, seized and confined the person of Gildo
in a dungeon; and his own despair saved him from the intolerable
torture of supporting the presence of an injured and victorious
brother.\textsuperscript{53} The captives and the spoils of Africa were laid at
the feet of the emperor; but Stilicho, whose moderation appeared
more conspicuous and more sincere, in the midst of prosperity,
still affected to consult the laws of the republic; and referred
to the senate and people of Rome the judgment of the most
illustrious criminals.\textsuperscript{54} Their trial was public and solemn; but
the judges, in the exercise of this obsolete and precarious
jurisdiction, were impatient to punish the African magistrates,
who had intercepted the subsistence of the Roman people. The rich
and guilty province was oppressed by the Imperial ministers, who
had a visible interest to multiply the number of the accomplices
of Gildo; and if an edict of Honorius seems to check the
malicious industry of informers, a subsequent edict, at the
distance of ten years, continues and renews the prosecution of
the offences which had been committed in the time of the general
rebellion.\textsuperscript{55} The adherents of the tyrant who escaped the first
fury of the soldiers, and the judges, might derive some
consolation from the tragic fate of his brother, who could never
obtain his pardon for the extraordinary services which he had
performed. After he had finished an important war in the space of
a single winter, Mascezel was received at the court of Milan with
loud applause, affected gratitude, and secret jealousy;\textsuperscript{56} and
his death, which, perhaps, was the effect of passage of a bridge,
the Moorish prince, who accompanied the master-general of the
West, was suddenly thrown from his horse into the river; the
officious haste of the attendants was restrained by a cruel and
perfidious smile which they observed on the countenance of
Stilicho; and while they delayed the necessary assistance, the
unfortunate Mascezel was irrecoverably drowned.\textsuperscript{57}

\pagenote[49]{Orosius must be responsible for the account. The
presumption of Gildo and his various train of Barbarians is
celebrated by Claudian, Cons. Stil. l. i. 345-355.}

\pagenote[50]{St. Ambrose, who had been dead about a year,
revealed, in a vision, the time and place of the victory.
Mascezel afterwards related his dream to Paulinus, the original
biographer of the saint, from whom it might easily pass to
Orosius.}

\pagenote[51]{Zosimus (l. v. p. 303) supposes an obstinate
combat; but the narrative of Orosius appears to conceal a real
fact, under the disguise of a miracle.}

\pagenote[52]{Tabraca lay between the two Hippos, (Cellarius,
tom. ii. p. 112; D’Anville, tom. iii. p. 84.) Orosius has
distinctly named the field of battle, but our ignorance cannot
define the precise situation.}

\pagenote[53]{The death of Gildo is expressed by Claudian (i.
Cons. Stil. 357) and his best interpreters, Zosimus and Orosius.}

\pagenote[54]{Claudian (ii. Cons. Stilich. 99-119) describes
their trial (tremuit quos Africa nuper, cernunt rostra reos,) and
applauds the restoration of the ancient constitution. It is here
that he introduces the famous sentence, so familiar to the
friends of despotism:

—-Nunquam libertas gratior exstat, Quam sub rege pio.

But the freedom which depends on royal piety, scarcely deserves
appellation}

\pagenote[55]{See the Theodosian Code, l. ix. tit. xxxix. leg. 3,
tit. xl. leg. 19.}

\pagenote[56]{Stilicho, who claimed an equal share in all the
victories of Theodosius and his son, particularly asserts, that
Africa was recovered by the wisdom of his counsels, (see an
inscription produced by Baronius.)}

\pagenote[57]{I have softened the narrative of Zosimus, which, in
its crude simplicity, is almost incredible, (l. v. p. 303.)
Orosius damns the victorious general (p. 538) for violating the
right of sanctuary.}

The joy of the African triumph was happily connected with the
nuptials of the emperor Honorius, and of his cousin Maria, the
daughter of Stilicho: and this equal and honorable alliance
seemed to invest the powerful minister with the authority of a
parent over his submissive pupil. The muse of Claudian was not
silent on this propitious day;\textsuperscript{58} he sung, in various and lively
strains, the happiness of the royal pair; and the glory of the
hero, who confirmed their union, and supported their throne. The
ancient fables of Greece, which had almost ceased to be the
object of religious faith, were saved from oblivion by the genius
of poetry. The picture of the Cyprian grove, the seat of harmony
and love; the triumphant progress of Venus over her native seas,
and the mild influence which her presence diffused in the palace
of Milan, express to every age the natural sentiments of the
heart, in the just and pleasing language of allegorical fiction.
But the amorous impatience which Claudian attributes to the young
prince,\textsuperscript{59} must excite the smiles of the court; and his beauteous
spouse (if she deserved the praise of beauty) had not much to
fear or to hope from the passions of her lover. Honorius was only
in the fourteenth year of his age; Serena, the mother of his
bride, deferred, by art of persuasion, the consummation of the
royal nuptials; Maria died a virgin, after she had been ten years
a wife; and the chastity of the emperor was secured by the
coldness, or perhaps, the debility, of his constitution.\textsuperscript{60} His
subjects, who attentively studied the character of their young
sovereign, discovered that Honorius was without passions, and
consequently without talents; and that his feeble and languid
disposition was alike incapable of discharging the duties of his
rank, or of enjoying the pleasures of his age. In his early youth
he made some progress in the exercises of riding and drawing the
bow: but he soon relinquished these fatiguing occupations, and
the amusement of feeding poultry became the serious and daily
care of the monarch of the West,\textsuperscript{61} who resigned the reins of
empire to the firm and skilful hand of his guardian Stilicho. The
experience of history will countenance the suspicion that a
prince who was born in the purple, received a worse education
than the meanest peasant of his dominions; and that the ambitious
minister suffered him to attain the age of manhood, without
attempting to excite his courage, or to enlighten his
understanding.\textsuperscript{62} The predecessors of Honorius were accustomed to
animate by their example, or at least by their presence, the
valor of the legions; and the dates of their laws attest the
perpetual activity of their motions through the provinces of the
Roman world. But the son of Theodosius passed the slumber of his
life, a captive in his palace, a stranger in his country, and the
patient, almost the indifferent, spectator of the ruin of the
Western empire, which was repeatedly attacked, and finally
subverted, by the arms of the Barbarians. In the eventful history
of a reign of twenty-eight years, it will seldom be necessary to
mention the name of the emperor Honorius.

\pagenote[58]{Claudian,as the poet laureate, composed a serious
and elaborate epithalamium of 340 lines; besides some gay
Fescennines, which were sung, in a more licentious tone, on the
wedding night.}

\pagenote[59]{

Calet obvius ire Jam princeps, tardumque cupit discedere solem.
Nobilis haud aliter sonipes.

(De Nuptiis Honor. et Mariae, and more freely in the Fescennines
112-116)

Dices, O quoties,hoc mihi dulcius Quam flavos decics vincere
Sarmatas. .... Tum victor madido prosilias toro, Nocturni referens
vulnera proelii.}

\pagenote[60]{See Zosimus, l. v. p. 333.}

\pagenote[61]{Procopius de Bell. Gothico, l. i. c. 2. I have
borrowed the general practice of Honorius, without adopting the
singular, and indeed improbable tale, which is related by the
Greek historian.}

\pagenote[62]{The lessons of Theodosius, or rather Claudian, (iv.
Cons. Honor 214-418,) might compose a fine institution for the
future prince of a great and free nation. It was far above
Honorius, and his degenerate subjects.}

