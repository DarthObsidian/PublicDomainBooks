\chapter{Civil Wars, Reign Of Theodosius.}

\textit{Death Of Gratian. — Ruin Of Arianism. — St. Ambrose. — First Civil War,
Against Maximus. — Character, Administration, And Penance Of
Theodosius. — Death Of Valentinian II. — Second Civil War, Against
Eugenius. — Death Of Theodosius.}

The fame of Gratian, before he had accomplished the twentieth
year of his age, was equal to that of the most celebrated
princes. His gentle and amiable disposition endeared him to his
private friends, the graceful affability of his manners engaged
the affection of the people: the men of letters, who enjoyed the
liberality, acknowledged the taste and eloquence, of their
sovereign; his valor and dexterity in arms were equally applauded
by the soldiers; and the clergy considered the humble piety of
Gratian as the first and most useful of his virtues. The victory
of Colmar had delivered the West from a formidable invasion; and
the grateful provinces of the East ascribed the merits of
Theodosius to the author of his greatness, and of the public
safety. Gratian survived those memorable events only four or five
years; but he survived his reputation; and, before he fell a
victim to rebellion, he had lost, in a great measure, the respect
and confidence of the Roman world.

The remarkable alteration of his character or conduct may not be
imputed to the arts of flattery, which had besieged the son of
Valentinian from his infancy; nor to the headstrong passions
which the that gentle youth appears to have escaped. A more
attentive view of the life of Gratian may perhaps suggest the
true cause of the disappointment of the public hopes. His
apparent virtues, instead of being the hardy productions of
experience and adversity, were the premature and artificial
fruits of a royal education. The anxious tenderness of his father
was continually employed to bestow on him those advantages, which
he might perhaps esteem the more highly, as he himself had been
deprived of them; and the most skilful masters of every science,
and of every art, had labored to form the mind and body of the
young prince.\textsuperscript{1} The knowledge which they painfully communicated
was displayed with ostentation, and celebrated with lavish
praise. His soft and tractable disposition received the fair
impression of their judicious precepts, and the absence of
passion might easily be mistaken for the strength of reason. His
preceptors gradually rose to the rank and consequence of
ministers of state:\textsuperscript{2} and, as they wisely dissembled their secret
authority, he seemed to act with firmness, with propriety, and
with judgment, on the most important occasions of his life and
reign. But the influence of this elaborate instruction did not
penetrate beyond the surface; and the skilful preceptors, who so
accurately guided the steps of their royal pupil, could not
infuse into his feeble and indolent character the vigorous and
independent principle of action which renders the laborious
pursuit of glory essentially necessary to the happiness, and
almost to the existence, of the hero. As soon as time and
accident had removed those faithful counsellors from the throne,
the emperor of the West insensibly descended to the level of his
natural genius; abandoned the reins of government to the
ambitious hands which were stretched forwards to grasp them; and
amused his leisure with the most frivolous gratifications. A
public sale of favor and injustice was instituted, both in the
court and in the provinces, by the worthless delegates of his
power, whose merit it was made sacrilege to question.\textsuperscript{3} The
conscience of the credulous prince was directed by saints and
bishops;\textsuperscript{4} who procured an Imperial edict to punish, as a capital
offence, the violation, the neglect, or even the ignorance, of
the divine law.\textsuperscript{5} Among the various arts which had exercised the
youth of Gratian, he had applied himself, with singular
inclination and success, to manage the horse, to draw the bow,
and to dart the javelin; and these qualifications, which might be
useful to a soldier, were prostituted to the viler purposes of
hunting. Large parks were enclosed for the Imperial pleasures,
and plentifully stocked with every species of wild beasts; and
Gratian neglected the duties, and even the dignity, of his rank,
to consume whole days in the vain display of his dexterity and
boldness in the chase. The pride and wish of the Roman emperor to
excel in an art, in which he might be surpassed by the meanest of
his slaves, reminded the numerous spectators of the examples of
Nero and Commodus, but the chaste and temperate Gratian was a
stranger to their monstrous vices; and his hands were stained
only with the blood of animals.\textsuperscript{6} The behavior of Gratian, which
degraded his character in the eyes of mankind, could not have
disturbed the security of his reign, if the army had not been
provoked to resent their peculiar injuries. As long as the young
emperor was guided by the instructions of his masters, he
professed himself the friend and pupil of the soldiers; many of
his hours were spent in the familiar conversation of the camp;
and the health, the comforts, the rewards, the honors, of his
faithful troops, appeared to be the objects of his attentive
concern. But, after Gratian more freely indulged his prevailing
taste for hunting and shooting, he naturally connected himself
with the most dexterous ministers of his favorite amusement. A
body of the Alani was received into the military and domestic
service of the palace; and the admirable skill, which they were
accustomed to display in the unbounded plains of Scythia, was
exercised, on a more narrow theatre, in the parks and enclosures
of Gaul. Gratian admired the talents and customs of these
favorite guards, to whom alone he intrusted the defence of his
person; and, as if he meant to insult the public opinion, he
frequently showed himself to the soldiers and people, with the
dress and arms, the long bow, the sounding quiver, and the fur
garments of a Scythian warrior. The unworthy spectacle of a Roman
prince, who had renounced the dress and manners of his country,
filled the minds of the legions with grief and indignation.\textsuperscript{7}
Even the Germans, so strong and formidable in the armies of the
empire, affected to disdain the strange and horrid appearance of
the savages of the North, who, in the space of a few years, had
wandered from the banks of the Volga to those of the Seine. A
loud and licentious murmur was echoed through the camps and
garrisons of the West; and as the mild indolence of Gratian
neglected to extinguish the first symptoms of discontent, the
want of love and respect was not supplied by the influence of
fear. But the subversion of an established government is always a
work of some real, and of much apparent, difficulty; and the
throne of Gratian was protected by the sanctions of custom, law,
religion, and the nice balance of the civil and military powers,
which had been established by the policy of Constantine. It is
not very important to inquire from what cause the revolt of
Britain was produced. Accident is commonly the parent of
disorder; the seeds of rebellion happened to fall on a soil which
was supposed to be more fruitful than any other in tyrants and
usurpers;\textsuperscript{8} the legions of that sequestered island had been long
famous for a spirit of presumption and arrogance;\textsuperscript{9} and the name
of Maximus was proclaimed, by the tumultuary, but unanimous
voice, both of the soldiers and of the provincials. The emperor,
or the rebel,—for this title was not yet ascertained by
fortune,—was a native of Spain, the countryman, the
fellow-soldier, and the rival of Theodosius whose elevation he
had not seen without some emotions of envy and resentment: the
events of his life had long since fixed him in Britain; and I
should not be unwilling to find some evidence for the marriage,
which he is said to have contracted with the daughter of a
wealthy lord of Caernarvonshire.\textsuperscript{10} But this provincial rank
might justly be considered as a state of exile and obscurity; and
if Maximus had obtained any civil or military office, he was not
invested with the authority either of governor or general.\textsuperscript{11} His
abilities, and even his integrity, are acknowledged by the
partial writers of the age; and the merit must indeed have been
conspicuous that could extort such a confession in favor of the
vanquished enemy of Theodosius. The discontent of Maximus might
incline him to censure the conduct of his sovereign, and to
encourage, perhaps, without any views of ambition, the murmurs of
the troops. But in the midst of the tumult, he artfully, or
modestly, refused to ascend the throne; and some credit appears
to have been given to his own positive declaration, that he was
compelled to accept the dangerous present of the Imperial purple.\textsuperscript{12}

\pagenote[1]{Valentinian was less attentive to the religion of
his son; since he intrusted the education of Gratian to Ausonius,
a professed Pagan. (Mem. de l’Academie des Inscriptions, tom. xv.
p. 125-138). The poetical fame of Ausonius condemns the taste of
his age.}

\pagenote[2]{Ausonius was successively promoted to the Prætorian
præfecture of Italy, (A.D. 377,) and of Gaul, (A.D. 378;) and
was at length invested with the consulship, (A.D. 379.) He
expressed his gratitude in a servile and insipid piece of
flattery, (Actio Gratiarum, p. 699-736,) which has survived more
worthy productions.}

\pagenote[3]{Disputare de principali judicio non oportet.
Sacrilegii enim instar est dubitare, an is dignus sit, quem
elegerit imperator. Codex Justinian, l. ix. tit. xxix. leg. 3.
This convenient law was revived and promulgated, after the death
of Gratian, by the feeble court of Milan.}

\pagenote[4]{Ambrose composed, for his instruction, a theological
treatise on the faith of the Trinity: and Tillemont, (Hist. des
Empereurs, tom. v. p. 158, 169,) ascribes to the archbishop the
merit of Gratian’s intolerant laws.}

\pagenote[5]{Qui divinae legis sanctitatem nesciendo omittunt,
aut negligende violant, et offendunt, sacrilegium committunt.
Codex Justinian. l. ix. tit. xxix. leg. 1. Theodosius indeed may
claim his share in the merit of this comprehensive law.}

\pagenote[6]{Ammianus (xxxi. 10) and the younger Victor
acknowledge the virtues of Gratian; and accuse, or rather lament,
his degenerate taste. The odious parallel of Commodus is saved by
“licet incruentus;” and perhaps Philostorgius (l. x. c. 10, and
Godefroy, p. 41) had guarded with some similar reserve, the
comparison of Nero.}

\pagenote[7]{Zosimus (l. iv. p. 247) and the younger Victor
ascribe the revolution to the favor of the Alani, and the
discontent of the Roman troops Dum exercitum negligeret, et
paucos ex Alanis, quos ingenti auro ad sa transtulerat,
anteferret veteri ac Romano militi.}

\pagenote[8]{Britannia fertilis provincia tyrannorum, is a
memorable expression, used by Jerom in the Pelagian controversy,
and variously tortured in the disputes of our national
antiquaries. The revolutions of the last age appeared to justify
the image of the sublime Bossuet, “sette ile, plus orageuse que
les mers qui l’environment.”}

\pagenote[9]{Zosimus says of the British soldiers.}

\pagenote[10]{Helena, the daughter of Eudda. Her chapel may still
be seen at Caer-segont, now Caer-narvon. (Carte’s Hist. of
England, vol. i. p. 168, from Rowland’s Mona Antiqua.) The
prudent reader may not perhaps be satisfied with such Welsh
evidence.}

\pagenote[11]{Camden (vol. i. introduct. p. ci.) appoints him
governor at Britain; and the father of our antiquities is
followed, as usual, by his blind progeny. Pacatus and Zosimus had
taken some pains to prevent this error, or fable; and I shall
protect myself by their decisive testimonies. Regali habitu
exulem suum, illi exules orbis induerunt, (in Panegyr. Vet. xii.
23,) and the Greek historian still less equivocally, (Maximus)
(l. iv. p. 248.)}

\pagenote[12]{Sulpicius Severus, Dialog. ii. 7. Orosius, l. vii.
c. 34. p. 556. They both acknowledge (Sulpicius had been his
subject) his innocence and merit. It is singular enough, that
Maximus should be less favorably treated by Zosimus, the partial
adversary of his rival.}

But there was danger likewise in refusing the empire; and from
the moment that Maximus had violated his allegiance to his lawful
sovereign, he could not hope to reign, or even to live, if he
confined his moderate ambition within the narrow limits of
Britain. He boldly and wisely resolved to prevent the designs of
Gratian; the youth of the island crowded to his standard, and he
invaded Gaul with a fleet and army, which were long afterwards
remembered, as the emigration of a considerable part of the
British nation.\textsuperscript{13} The emperor, in his peaceful residence of
Paris, was alarmed by their hostile approach; and the darts which
he idly wasted on lions and bears, might have been employed more
honorably against the rebels. But his feeble efforts announced
his degenerate spirit and desperate situation; and deprived him
of the resources, which he still might have found, in the support
of his subjects and allies. The armies of Gaul, instead of
opposing the march of Maximus, received him with joyful and loyal
acclamations; and the shame of the desertion was transferred from
the people to the prince. The troops, whose station more
immediately attached them to the service of the palace, abandoned
the standard of Gratian the first time that it was displayed in
the neighborhood of Paris. The emperor of the West fled towards
Lyons, with a train of only three hundred horse; and, in the
cities along the road, where he hoped to find refuge, or at least
a passage, he was taught, by cruel experience, that every gate is
shut against the unfortunate. Yet he might still have reached, in
safety, the dominions of his brother; and soon have returned with
the forces of Italy and the East; if he had not suffered himself
to be fatally deceived by the perfidious governor of the Lyonnese
province. Gratian was amused by protestations of doubtful
fidelity, and the hopes of a support, which could not be
effectual; till the arrival of Andragathius, the general of the
cavalry of Maximus, put an end to his suspense. That resolute
officer executed, without remorse, the orders or the intention of
the usurper. Gratian, as he rose from supper, was delivered into
the hands of the assassin: and his body was denied to the pious
and pressing entreaties of his brother Valentinian.\textsuperscript{14} The death
of the emperor was followed by that of his powerful general
Mellobaudes, the king of the Franks; who maintained, to the last
moment of his life, the ambiguous reputation, which is the just
recompense of obscure and subtle policy.\textsuperscript{15} These executions
might be necessary to the public safety: but the successful
usurper, whose power was acknowledged by all the provinces of the
West, had the merit, and the satisfaction, of boasting, that,
except those who had perished by the chance of war, his triumph
was not stained by the blood of the Romans.\textsuperscript{16}

\pagenote[13]{Archbishop Usher (Antiquat. Britan. Eccles. p. 107,
108) has diligently collected the legends of the island, and the
continent. The whole emigration consisted of 30,000 soldiers, and
100,000 plebeians, who settled in Bretagne. Their destined
brides, St. Ursula with 11,000 noble, and 60,000 plebeian,
virgins, mistook their way; landed at Cologne, and were all most
cruelly murdered by the Huns. But the plebeian sisters have been
defrauded of their equal honors; and what is still harder, John
Trithemius presumes to mention the children of these British
virgins.}

\pagenote[14]{Zosimus (l. iv. p. 248, 249) has transported the
death of Gratian from Lugdunum in Gaul (Lyons) to Singidunum in
Moesia. Some hints may be extracted from the Chronicles; some
lies may be detected in Sozomen (l. vii. c. 13) and Socrates, (l.
v. c. 11.) Ambrose is our most authentic evidence, (tom. i.
Enarrat. in Psalm lxi. p. 961, tom ii. epist. xxiv. p. 888 \&c.,
and de Obitu Valentinian Consolat. Ner. 28, p. 1182.)}

\pagenote[15]{Pacatus (xii. 28) celebrates his fidelity; while
his treachery is marked in Prosper’s Chronicle, as the cause of
the ruin of Gratian. Ambrose, who has occasion to exculpate
himself, only condemns the death of Vallio, a faithful servant of
Gratian, (tom. ii. epist. xxiv. p. 891, edit. Benedict.) * Note:
Le Beau contests the reading in the chronicle of Prosper upon
which this charge rests. Le Beau, iv. 232.—M. * Note: According
to Pacatus, the Count Vallio, who commanded the army, was carried
to Chalons to be burnt alive; but Maximus, dreading the
imputation of cruelty, caused him to be secretly strangled by his
Bretons. Macedonius also, master of the offices, suffered the
death which he merited. Le Beau, iv. 244.—M.}

\pagenote[16]{He protested, nullum ex adversariis nisi in acissie
occubu. Sulp. Jeverus in Vit. B. Martin, c. 23. The orator
Theodosius bestows reluctant, and therefore weighty, praise on
his clemency. Si cui ille, pro ceteris sceleribus suis, minus
crudelis fuisse videtur, (Panegyr. Vet. xii. 28.)}

The events of this revolution had passed in such rapid
succession, that it would have been impossible for Theodosius to
march to the relief of his benefactor, before he received the
intelligence of his defeat and death. During the season of
sincere grief, or ostentatious mourning, the Eastern emperor was
interrupted by the arrival of the principal chamberlain of
Maximus; and the choice of a venerable old man, for an office
which was usually exercised by eunuchs, announced to the court of
Constantinople the gravity and temperance of the British usurper.

The ambassador condescended to justify, or excuse, the conduct of
his master; and to protest, in specious language, that the murder
of Gratian had been perpetrated, without his knowledge or
consent, by the precipitate zeal of the soldiers. But he
proceeded, in a firm and equal tone, to offer Theodosius the
alternative of peace, or war. The speech of the ambassador
concluded with a spirited declaration, that although Maximus, as
a Roman, and as the father of his people, would choose rather to
employ his forces in the common defence of the republic, he was
armed and prepared, if his friendship should be rejected, to
dispute, in a field of battle, the empire of the world. An
immediate and peremptory answer was required; but it was
extremely difficult for Theodosius to satisfy, on this important
occasion, either the feelings of his own mind, or the
expectations of the public. The imperious voice of honor and
gratitude called aloud for revenge. From the liberality of
Gratian, he had received the Imperial diadem; his patience would
encourage the odious suspicion, that he was more deeply sensible
of former injuries, than of recent obligations; and if he
accepted the friendship, he must seem to share the guilt, of the
assassin. Even the principles of justice, and the interest of
society, would receive a fatal blow from the impunity of Maximus;
and the example of successful usurpation would tend to dissolve
the artificial fabric of government, and once more to replunge
the empire in the crimes and calamities of the preceding age.
But, as the sentiments of gratitude and honor should invariably
regulate the conduct of an individual, they may be overbalanced
in the mind of a sovereign, by the sense of superior duties; and
the maxims both of justice and humanity must permit the escape of
an atrocious criminal, if an innocent people would be involved in
the consequences of his punishment. The assassin of Gratian had
usurped, but he actually possessed, the most warlike provinces of
the empire: the East was exhausted by the misfortunes, and even
by the success, of the Gothic war; and it was seriously to be
apprehended, that, after the vital strength of the republic had
been wasted in a doubtful and destructive contest, the feeble
conqueror would remain an easy prey to the Barbarians of the
North. These weighty considerations engaged Theodosius to
dissemble his resentment, and to accept the alliance of the
tyrant. But he stipulated, that Maximus should content himself
with the possession of the countries beyond the Alps. The brother
of Gratian was confirmed and secured in the sovereignty of Italy,
Africa, and the Western Illyricum; and some honorable conditions
were inserted in the treaty, to protect the memory, and the laws,
of the deceased emperor.\textsuperscript{17} According to the custom of the age,
the images of the three Imperial colleagues were exhibited to the
veneration of the people; nor should it be lightly supposed,
that, in the moment of a solemn reconciliation, Theodosius
secretly cherished the intention of perfidy and revenge.\textsuperscript{18}

\pagenote[17]{Ambrose mentions the laws of Gratian, quas non
abrogavit hostia (tom. ii epist. xvii. p. 827.)}

\pagenote[18]{Zosimus, l. iv. p. 251, 252. We may disclaim his
odious suspicions; but we cannot reject the treaty of peace which
the friends of Theodosius have absolutely forgotten, or slightly
mentioned.}

The contempt of Gratian for the Roman soldiers had exposed him to
the fatal effects of their resentment. His profound veneration
for the Christian clergy was rewarded by the applause and
gratitude of a powerful order, which has claimed, in every age,
the privilege of dispensing honors, both on earth and in heaven.\textsuperscript{19}
The orthodox bishops bewailed his death, and their own
irreparable loss; but they were soon comforted by the discovery,
that Gratian had committed the sceptre of the East to the hands
of a prince, whose humble faith and fervent zeal, were supported
by the spirit and abilities of a more vigorous character. Among
the benefactors of the church, the fame of Constantine has been
rivalled by the glory of Theodosius. If Constantine had the
advantage of erecting the standard of the cross, the emulation of
his successor assumed the merit of subduing the Arian heresy, and
of abolishing the worship of idols in the Roman world. Theodosius
was the first of the emperors baptized in the true faith of the
Trinity. Although he was born of a Christian family, the maxims,
or at least the practice, of the age, encouraged him to delay the
ceremony of his initiation; till he was admonished of the danger
of delay, by the serious illness which threatened his life,
towards the end of the first year of his reign. Before he again
took the field against the Goths, he received the sacrament of
baptism\textsuperscript{20} from Acholius, the orthodox bishop of Thessalonica:\textsuperscript{21}
and, as the emperor ascended from the holy font, still glowing
with the warm feelings of regeneration, he dictated a solemn
edict, which proclaimed his own faith, and prescribed the
religion of his subjects. “It is our pleasure (such is the
Imperial style) that all the nations, which are governed by our
clemency and moderation, should steadfastly adhere to the
religion which was taught by St. Peter to the Romans; which
faithful tradition has preserved; and which is now professed by
the pontiff Damasus, and by Peter, bishop of Alexandria, a man of
apostolic holiness. According to the discipline of the apostles,
and the doctrine of the gospel, let us believe the sole deity of
the Father, the Son, and the Holy Ghost; under an equal majesty,
and a pious Trinity. We authorize the followers of this doctrine
to assume the title of Catholic Christians; and as we judge, that
all others are extravagant madmen, we brand them with the
infamous name of Heretics; and declare that their conventicles
shall no longer usurp the respectable appellation of churches.
Besides the condemnation of divine justice, they must expect to
suffer the severe penalties, which our authority, guided by
heavenly wisdom, shall think proper to inflict upon them.”\textsuperscript{22} The
faith of a soldier is commonly the fruit of instruction, rather
than of inquiry; but as the emperor always fixed his eyes on the
visible landmarks of orthodoxy, which he had so prudently
constituted, his religious opinions were never affected by the
specious texts, the subtle arguments, and the ambiguous creeds of
the Arian doctors. Once indeed he expressed a faint inclination
to converse with the eloquent and learned Eunomius, who lived in
retirement at a small distance from Constantinople. But the
dangerous interview was prevented by the prayers of the empress
Flaccilla, who trembled for the salvation of her husband; and the
mind of Theodosius was confirmed by a theological argument,
adapted to the rudest capacity. He had lately bestowed on his
eldest son, Arcadius, the name and honors of Augustus, and the
two princes were seated on a stately throne to receive the homage
of their subjects. A bishop, Amphilochius of Iconium, approached
the throne, and after saluting, with due reverence, the person of
his sovereign, he accosted the royal youth with the same familiar
tenderness which he might have used towards a plebeian child.
Provoked by this insolent behavior, the monarch gave orders, that
the rustic priest should be instantly driven from his presence.
But while the guards were forcing him to the door, the dexterous
polemic had time to execute his design, by exclaiming, with a
loud voice, “Such is the treatment, O emperor! which the King of
heaven has prepared for those impious men, who affect to worship
the Father, but refuse to acknowledge the equal majesty of his
divine Son.” Theodosius immediately embraced the bishop of
Iconium, and never forgot the important lesson, which he had
received from this dramatic parable.\textsuperscript{23}

\pagenote[19]{Their oracle, the archbishop of Milan, assigns to
his pupil Gratian, a high and respectable place in heaven, (tom.
ii. de Obit. Val. Consol p. 1193.)}

\pagenote[20]{For the baptism of Theodosius, see Sozomen, (l.
vii. c. 4,) Socrates, (l. v. c. 6,) and Tillemont, (Hist. des
Empereurs, tom. v. p. 728.)}

\pagenote[21]{Ascolius, or Acholius, was honored by the
friendship, and the praises, of Ambrose; who styles him murus
fidei atque sanctitatis, (tom. ii. epist. xv. p. 820;) and
afterwards celebrates his speed and diligence in running to
Constantinople, Italy, \&c., (epist. xvi. p. 822.) a virtue which
does not appertain either to a wall, or a bishop.}

\pagenote[22]{Codex Theodos. l. xvi. tit. i. leg. 2, with
Godefroy’s Commentary, tom. vi. p. 5-9. Such an edict deserved
the warmest praises of Baronius, auream sanctionem, edictum pium
et salutare.—Sic itua ad astra.}

\pagenote[23]{Sozomen, l. vii. c. 6. Theodoret, l. v. c. 16.
Tillemont is displeased (Mem. Eccles. tom. vi. p. 627, 628) with
the terms of “rustic bishop,” “obscure city.” Yet I must take
leave to think, that both Amphilochius and Iconium were objects
of inconsiderable magnitude in the Roman empire.}

\section{Part \thesection.}

Constantinople was the principal seat and fortress of Arianism;
and, in a long interval of forty years,\textsuperscript{24} the faith of the
princes and prelates, who reigned in the capital of the East, was
rejected in the purer schools of Rome and Alexandria. The
archiepiscopal throne of Macedonius, which had been polluted with
so much Christian blood, was successively filled by Eudoxus and
Damophilus. Their diocese enjoyed a free importation of vice and
error from every province of the empire; the eager pursuit of
religious controversy afforded a new occupation to the busy
idleness of the metropolis; and we may credit the assertion of an
intelligent observer, who describes, with some pleasantry, the
effects of their loquacious zeal. “This city,” says he, “is full
of mechanics and slaves, who are all of them profound
theologians; and preach in the shops, and in the streets. If you
desire a man to change a piece of silver, he informs you, wherein
the Son differs from the Father; if you ask the price of a loaf,
you are told by way of reply, that the Son is inferior to the
Father; and if you inquire, whether the bath is ready, the answer
is, that the Son was made out of nothing.”\textsuperscript{25} The heretics, of
various denominations, subsisted in peace under the protection of
the Arians of Constantinople; who endeavored to secure the
attachment of those obscure sectaries, while they abused, with
unrelenting severity, the victory which they had obtained over
the followers of the council of Nice. During the partial reigns
of Constantius and Valens, the feeble remnant of the Homoousians
was deprived of the public and private exercise of their
religion; and it has been observed, in pathetic language, that
the scattered flock was left without a shepherd to wander on the
mountains, or to be devoured by rapacious wolves.\textsuperscript{26} But, as
their zeal, instead of being subdued, derived strength and vigor
from oppression, they seized the first moments of imperfect
freedom, which they had acquired by the death of Valens, to form
themselves into a regular congregation, under the conduct of an
episcopal pastor. Two natives of Cappadocia, Basil, and Gregory
Nazianzen,\textsuperscript{27} were distinguished above all their contemporaries,\textsuperscript{28}
by the rare union of profane eloquence and of orthodox piety.

These orators, who might sometimes be compared, by themselves,
and by the public, to the most celebrated of the ancient Greeks,
were united by the ties of the strictest friendship. They had
cultivated, with equal ardor, the same liberal studies in the
schools of Athens; they had retired, with equal devotion, to the
same solitude in the deserts of Pontus; and every spark of
emulation, or envy, appeared to be totally extinguished in the
holy and ingenuous breasts of Gregory and Basil. But the
exaltation of Basil, from a private life to the archiepiscopal
throne of Caesarea, discovered to the world, and perhaps to
himself, the pride of his character; and the first favor which he
condescended to bestow on his friend, was received, and perhaps
was intended, as a cruel insult.\textsuperscript{29} Instead of employing the
superior talents of Gregory in some useful and conspicuous
station, the haughty prelate selected, among the fifty bishoprics
of his extensive province, the wretched village of Sasima,\textsuperscript{30}
without water, without verdure, without society, situate at the
junction of three highways, and frequented only by the incessant
passage of rude and clamorous wagoners. Gregory submitted with
reluctance to this humiliating exile; he was ordained bishop of
Sasima; but he solemnly protests, that he never consummated his
spiritual marriage with this disgusting bride. He afterwards
consented to undertake the government of his native church of
Nazianzus,\textsuperscript{31} of which his father had been bishop above
five-and-forty years. But as he was still conscious that he
deserved another audience, and another theatre, he accepted, with
no unworthy ambition, the honorable invitation, which was
addressed to him from the orthodox party of Constantinople. On
his arrival in the capital, Gregory was entertained in the house
of a pious and charitable kinsman; the most spacious room was
consecrated to the uses of religious worship; and the name of
Anastasia was chosen to express the resurrection of the Nicene
faith. This private conventicle was afterwards converted into a
magnificent church; and the credulity of the succeeding age was
prepared to believe the miracles and visions, which attested the
presence, or at least the protection, of the Mother of God.\textsuperscript{32}
The pulpit of the Anastasia was the scene of the labors and
triumphs of Gregory Nazianzen; and, in the space of two years, he
experienced all the spiritual adventures which constitute the
prosperous or adverse fortunes of a missionary.\textsuperscript{33} The Arians,
who were provoked by the boldness of his enterprise, represented
his doctrine, as if he had preached three distinct and equal
Deities; and the devout populace was excited to suppress, by
violence and tumult, the irregular assemblies of the Athanasian
heretics. From the cathedral of St. Sophia there issued a motley
crowd “of common beggars, who had forfeited their claim to pity;
of monks, who had the appearance of goats or satyrs; and of
women, more terrible than so many Jezebels.” The doors of the
Anastasia were broke open; much mischief was perpetrated, or
attempted, with sticks, stones, and firebrands; and as a man lost
his life in the affray, Gregory, who was summoned the next
morning before the magistrate, had the satisfaction of supposing,
that he publicly confessed the name of Christ. After he was
delivered from the fear and danger of a foreign enemy, his infant
church was disgraced and distracted by intestine faction. A
stranger who assumed the name of Maximus,\textsuperscript{34} and the cloak of a
Cynic philosopher, insinuated himself into the confidence of
Gregory; deceived and abused his favorable opinion; and forming a
secret connection with some bishops of Egypt, attempted, by a
clandestine ordination, to supplant his patron in the episcopal
seat of Constantinople. These mortifications might sometimes
tempt the Cappadocian missionary to regret his obscure solitude.
But his fatigues were rewarded by the daily increase of his fame
and his congregation; and he enjoyed the pleasure of observing,
that the greater part of his numerous audience retired from his
sermons satisfied with the eloquence of the preacher,\textsuperscript{35} or
dissatisfied with the manifold imperfections of their faith and
practice.\textsuperscript{36}

\pagenote[24]{Sozomen, l. vii. c. v. Socrates, l. v. c. 7.
Marcellin. in Chron. The account of forty years must be dated
from the election or intrusion of Eusebius, who wisely exchanged
the bishopric of Nicomedia for the throne of Constantinople.}

\pagenote[25]{See Jortin’s Remarks on Ecclesiastical History,
vol. iv. p. 71. The thirty-third Oration of Gregory Nazianzen
affords indeed some similar ideas, even some still more
ridiculous; but I have not yet found the words of this remarkable
passage, which I allege on the faith of a correct and liberal
scholar.}

\pagenote[26]{See the thirty-second Oration of Gregory Nazianzen,
and the account of his own life, which he has composed in 1800
iambics. Yet every physician is prone to exaggerate the
inveterate nature of the disease which he has cured.}

\pagenote[27]{I confess myself deeply indebted to the two lives
of Gregory Nazianzen, composed, with very different views, by
Tillemont (Mem. Eccles. tom. ix. p. 305-560, 692-731) and Le
Clerc, (Bibliothèque Universelle, tom. xviii. p. 1-128.)}

\pagenote[28]{Unless Gregory Nazianzen mistook thirty years in
his own age, he was born, as well as his friend Basil, about the
year 329. The preposterous chronology of Suidas has been
graciously received, because it removes the scandal of Gregory’s
father, a saint likewise, begetting children after he became a
bishop, (Tillemont, Mem. Eccles. tom. ix. p. 693-697.)}

\pagenote[29]{Gregory’s Poem on his own Life contains some
beautiful lines, (tom. ii. p. 8,) which burst from the heart, and
speak the pangs of injured and lost friendship. ——In the
Midsummer Night’s Dream, Helena addresses the same pathetic
complaint to her friend Hermia:—Is all the counsel that we two
have shared. The sister’s vows, \&c. Shakspeare had never read the
poems of Gregory Nazianzen; he was ignorant of the Greek
language; but his mother tongue, the language of Nature, is the
same in Cappadocia and in Britain.}

\pagenote[30]{This unfavorable portrait of Sasimae is drawn by
Gregory Nazianzen, (tom. ii. de Vita sua, p. 7, 8.) Its precise
situation, forty-nine miles from Archelais, and thirty-two from
Tyana, is fixed in the Itinerary of Antoninus, (p. 144, edit.
Wesseling.)}

\pagenote[31]{The name of Nazianzus has been immortalized by
Gregory; but his native town, under the Greek or Roman title of
Diocaesarea, (Tillemont, Mem. Eccles. tom. ix. p. 692,) is
mentioned by Pliny, (vi. 3,) Ptolemy, and Hierocles, (Itinerar.
Wesseling, p. 709). It appears to have been situate on the edge
of Isauria.}

\pagenote[32]{See Ducange, Constant. Christiana, l. iv. p. 141,
142. The Sozomen (l. vii. c. 5) is interpreted to mean the Virgin
Mary.}

\pagenote[33]{Tillemont (Mem. Eccles. tom. ix. p. 432, \&c.)
diligently collects, enlarges, and explains, the oratorical and
poetical hints of Gregory himself.}

\pagenote[34]{He pronounced an oration (tom. i. Orat. xxiii. p.
409) in his praise; but after their quarrel, the name of Maximus
was changed into that of Heron, (see Jerom, tom. i. in Catalog.
Script. Eccles. p. 301). I touch slightly on these obscure and
personal squabbles.}

\pagenote[35]{Under the modest emblem of a dream, Gregory (tom.
ii. Carmen ix. p. 78) describes his own success with some human
complacency. Yet it should seem, from his familiar conversation
with his auditor St. Jerom, (tom. i. Epist. ad Nepotian. p. 14,)
that the preacher understood the true value of popular applause.}

\pagenote[36]{Lachrymae auditorum laudes tuae sint, is the lively
and judicious advice of St. Jerom.}

The Catholics of Constantinople were animated with joyful
confidence by the baptism and edict of Theodosius; and they
impatiently waited the effects of his gracious promise. Their
hopes were speedily accomplished; and the emperor, as soon as he
had finished the operations of the campaign, made his public
entry into the capital at the head of a victorious army. The next
day after his arrival, he summoned Damophilus to his presence,
and offered that Arian prelate the hard alternative of
subscribing the Nicene creed, or of instantly resigning, to the
orthodox believers, the use and possession of the episcopal
palace, the cathedral of St. Sophia, and all the churches of
Constantinople. The zeal of Damophilus, which in a Catholic saint
would have been justly applauded, embraced, without hesitation, a
life of poverty and exile,\textsuperscript{37} and his removal was immediately
followed by the purification of the Imperial city. The Arians
might complain, with some appearance of justice, that an
inconsiderable congregation of sectaries should usurp the hundred
churches, which they were insufficient to fill; whilst the far
greater part of the people was cruelly excluded from every place
of religious worship. Theodosius was still inexorable; but as the
angels who protected the Catholic cause were only visible to the
eyes of faith, he prudently reenforced those heavenly legions
with the more effectual aid of temporal and carnal weapons; and
the church of St. Sophia was occupied by a large body of the
Imperial guards. If the mind of Gregory was susceptible of pride,
he must have felt a very lively satisfaction, when the emperor
conducted him through the streets in solemn triumph; and, with
his own hand, respectfully placed him on the archiepiscopal
throne of Constantinople. But the saint (who had not subdued the
imperfections of human virtue) was deeply affected by the
mortifying consideration, that his entrance into the fold was
that of a wolf, rather than of a shepherd; that the glittering
arms which surrounded his person, were necessary for his safety;
and that he alone was the object of the imprecations of a great
party, whom, as men and citizens, it was impossible for him to
despise. He beheld the innumerable multitude of either sex, and
of every age, who crowded the streets, the windows, and the roofs
of the houses; he heard the tumultuous voice of rage, grief,
astonishment, and despair; and Gregory fairly confesses, that on
the memorable day of his installation, the capital of the East
wore the appearance of a city taken by storm, and in the hands of
a Barbarian conqueror.\textsuperscript{38} About six weeks afterwards, Theodosius
declared his resolution of expelling from all the churches of his
dominions the bishops and their clergy who should obstinately
refuse to believe, or at least to profess, the doctrine of the
council of Nice. His lieutenant, Sapor, was armed with the ample
powers of a general law, a special commission, and a military
force;\textsuperscript{39} and this ecclesiastical revolution was conducted with
so much discretion and vigor, that the religion of the emperor
was established, without tumult or bloodshed, in all the
provinces of the East. The writings of the Arians, if they had
been permitted to exist,\textsuperscript{40} would perhaps contain the lamentable
story of the persecution, which afflicted the church under the
reign of the impious Theodosius; and the sufferings of their holy
confessors might claim the pity of the disinterested reader. Yet
there is reason to imagine, that the violence of zeal and revenge
was, in some measure, eluded by the want of resistance; and that,
in their adversity, the Arians displayed much less firmness than
had been exerted by the orthodox party under the reigns of
Constantius and Valens. The moral character and conduct of the
hostile sects appear to have been governed by the same common
principles of nature and religion: but a very material
circumstance may be discovered, which tended to distinguish the
degrees of their theological faith. Both parties, in the schools,
as well as in the temples, acknowledged and worshipped the divine
majesty of Christ; and, as we are always prone to impute our own
sentiments and passions to the Deity, it would be deemed more
prudent and respectful to exaggerate, than to circumscribe, the
adorable perfections of the Son of God. The disciple of
Athanasius exulted in the proud confidence, that he had entitled
himself to the divine favor; while the follower of Arius must
have been tormented by the secret apprehension, that he was
guilty, perhaps, of an unpardonable offence, by the scanty
praise, and parsimonious honors, which he bestowed on the Judge
of the World. The opinions of Arianism might satisfy a cold and
speculative mind: but the doctrine of the Nicene creed, most
powerfully recommended by the merits of faith and devotion, was
much better adapted to become popular and successful in a
believing age.

\pagenote[37]{Socrates (l. v. c. 7) and Sozomen (l. vii. c. 5)
relate the evangelical words and actions of Damophilus without a
word of approbation. He considered, says Socrates, that it is
difficult to resist the powerful, but it was easy, and would have
been profitable, to submit.}

\pagenote[38]{See Gregory Nazianzen, tom. ii. de Vita sua, p. 21,
22. For the sake of posterity, the bishop of Constantinople
records a stupendous prodigy. In the month of November, it was a
cloudy morning, but the sun broke forth when the procession
entered the church.}

\pagenote[39]{Of the three ecclesiastical historians, Theodoret
alone (l. v. c. 2) has mentioned this important commission of
Sapor, which Tillemont (Hist. des Empereurs, tom. v. p. 728)
judiciously removes from the reign of Gratian to that of
Theodosius.}

\pagenote[40]{I do not reckon Philostorgius, though he mentions
(l. ix. c. 19) the explosion of Damophilus. The Eunomian
historian has been carefully strained through an orthodox sieve.}

The hope, that truth and wisdom would be found in the assemblies
of the orthodox clergy, induced the emperor to convene, at
Constantinople, a synod of one hundred and fifty bishops, who
proceeded, without much difficulty or delay, to complete the
theological system which had been established in the council of
Nice. The vehement disputes of the fourth century had been
chiefly employed on the nature of the Son of God; and the various
opinions which were embraced, concerning the Second, were
extended and transferred, by a natural analogy, to the Third
person of the Trinity.\textsuperscript{41} Yet it was found, or it was thought,
necessary, by the victorious adversaries of Arianism, to explain
the ambiguous language of some respectable doctors; to confirm
the faith of the Catholics; and to condemn an unpopular and
inconsistent sect of Macedonians; who freely admitted that the
Son was consubstantial to the Father, while they were fearful of
seeming to acknowledge the existence of Three Gods. A final and
unanimous sentence was pronounced to ratify the equal Deity of
the Holy Ghost: the mysterious doctrine has been received by all
the nations, and all the churches of the Christian world; and
their grateful reverence has assigned to the bishops of
Theodosius the second rank among the general councils.\textsuperscript{42} Their
knowledge of religious truth may have been preserved by
tradition, or it may have been communicated by inspiration; but
the sober evidence of history will not allow much weight to the
personal authority of the Fathers of Constantinople. In an age
when the ecclesiastics had scandalously degenerated from the
model of apostolic purity, the most worthless and corrupt were
always the most eager to frequent, and disturb, the episcopal
assemblies. The conflict and fermentation of so many opposite
interests and tempers inflamed the passions of the bishops: and
their ruling passions were, the love of gold, and the love of
dispute. Many of the same prelates who now applauded the orthodox
piety of Theodosius, had repeatedly changed, with prudent
flexibility, their creeds and opinions; and in the various
revolutions of the church and state, the religion of their
sovereign was the rule of their obsequious faith. When the
emperor suspended his prevailing influence, the turbulent synod
was blindly impelled by the absurd or selfish motives of pride,
hatred, or resentment. The death of Meletius, which happened at
the council of Constantinople, presented the most favorable
opportunity of terminating the schism of Antioch, by suffering
his aged rival, Paulinus, peaceably to end his days in the
episcopal chair. The faith and virtues of Paulinus were
unblemished. But his cause was supported by the Western churches;
and the bishops of the synod resolved to perpetuate the mischiefs
of discord, by the hasty ordination of a perjured candidate,\textsuperscript{43}
rather than to betray the imagined dignity of the East, which had
been illustrated by the birth and death of the Son of God. Such
unjust and disorderly proceedings forced the gravest members of
the assembly to dissent and to secede; and the clamorous majority
which remained masters of the field of battle, could be compared
only to wasps or magpies, to a flight of cranes, or to a flock of
geese.\textsuperscript{44}

\pagenote[41]{Le Clerc has given a curious extract (Bibliothèque
Universelle, tom. xviii. p. 91-105) of the theological sermons
which Gregory Nazianzen pronounced at Constantinople against the
Arians, Eunomians, Macedonians, \&c. He tells the Macedonians, who
deified the Father and the Son without the Holy Ghost, that they
might as well be styled Tritheists as Ditheists. Gregory himself
was almost a Tritheist; and his monarchy of heaven resembles a
well-regulated aristocracy.}

\pagenote[42]{The first general council of Constantinople now
triumphs in the Vatican; but the popes had long hesitated, and
their hesitation perplexes, and almost staggers, the humble
Tillemont, (Mem. Eccles. tom. ix. p. 499, 500.)}

\pagenote[43]{Before the death of Meletius, six or eight of his
most popular ecclesiastics, among whom was Flavian, had abjured,
for the sake of peace, the bishopric of Antioch, (Sozomen, l.
vii. c. 3, 11. Socrates, l. v. c. v.) Tillemont thinks it his
duty to disbelieve the story; but he owns that there are many
circumstances in the life of Flavian which seem inconsistent with
the praises of Chrysostom, and the character of a saint, (Mem.
Eccles. tom. x. p. 541.)}

\pagenote[44]{Consult Gregory Nazianzen, de Vita sua, tom. ii. p.
25-28. His general and particular opinion of the clergy and their
assemblies may be seen in verse and prose, (tom. i. Orat. i. p.
33. Epist. lv. p. 814, tom. ii. Carmen x. p. 81.) Such passages
are faintly marked by Tillemont, and fairly produced by Le
Clerc.}

A suspicion may possibly arise, that so unfavorable a picture of
ecclesiastical synods has been drawn by the partial hand of some
obstinate heretic, or some malicious infidel. But the name of the
sincere historian who has conveyed this instructive lesson to the
knowledge of posterity, must silence the impotent murmurs of
superstition and bigotry. He was one of the most pious and
eloquent bishops of the age; a saint, and a doctor of the church;
the scourge of Arianism, and the pillar of the orthodox faith; a
distinguished member of the council of Constantinople, in which,
after the death of Meletius, he exercised the functions of
president; in a word—Gregory Nazianzen himself. The harsh and
ungenerous treatment which he experienced,\textsuperscript{45} instead of
derogating from the truth of his evidence, affords an additional
proof of the spirit which actuated the deliberations of the
synod. Their unanimous suffrage had confirmed the pretensions
which the bishop of Constantinople derived from the choice of the
people, and the approbation of the emperor. But Gregory soon
became the victim of malice and envy. The bishops of the East,
his strenuous adherents, provoked by his moderation in the
affairs of Antioch, abandoned him, without support, to the
adverse faction of the Egyptians; who disputed the validity of
his election, and rigorously asserted the obsolete canon, that
prohibited the licentious practice of episcopal translations. The
pride, or the humility, of Gregory prompted him to decline a
contest which might have been imputed to ambition and avarice;
and he publicly offered, not without some mixture of indignation,
to renounce the government of a church which had been restored,
and almost created, by his labors. His resignation was accepted
by the synod, and by the emperor, with more readiness than he
seems to have expected. At the time when he might have hoped to
enjoy the fruits of his victory, his episcopal throne was filled
by the senator Nectarius; and the new archbishop, accidentally
recommended by his easy temper and venerable aspect, was obliged
to delay the ceremony of his consecration, till he had previously
despatched the rites of his baptism.\textsuperscript{46} After this remarkable
experience of the ingratitude of princes and prelates, Gregory
retired once more to his obscure solitude of Cappadocia; where he
employed the remainder of his life, about eight years, in the
exercises of poetry and devotion. The title of Saint has been
added to his name: but the tenderness of his heart,\textsuperscript{47} and the
elegance of his genius, reflect a more pleasing lustre on the
memory of Gregory Nazianzen.

\pagenote[45]{See Gregory, tom. ii. de Vita sua, p. 28-31. The
fourteenth, twenty-seventh, and thirty-second Orations were
pronounced in the several stages of this business. The peroration
of the last, (tom. i. p. 528,) in which he takes a solemn leave
of men and angels, the city and the emperor, the East and the
West, \&c., is pathetic, and almost sublime.}

\pagenote[46]{The whimsical ordination of Nectarius is attested
by Sozomen, (l. vii. c. 8;) but Tillemont observes, (Mem. Eccles.
tom. ix. p. 719,) Apres tout, ce narre de Sozomene est si
honteux, pour tous ceux qu’il y mele, et surtout pour Theodose,
qu’il vaut mieux travailler a le detruire, qu’a le soutenir; an
admirable canon of criticism!}

\pagenote[47]{I can only be understood to mean, that such was his
natural temper when it was not hardened, or inflamed, by
religious zeal. From his retirement, he exhorts Nectarius to
prosecute the heretics of Constantinople.}

It was not enough that Theodosius had suppressed the insolent
reign of Arianism, or that he had abundantly revenged the
injuries which the Catholics sustained from the zeal of
Constantius and Valens. The orthodox emperor considered every
heretic as a rebel against the supreme powers of heaven and of
earth; and each of those powers might exercise their peculiar
jurisdiction over the soul and body of the guilty. The decrees of
the council of Constantinople had ascertained the true standard
of the faith; and the ecclesiastics, who governed the conscience
of Theodosius, suggested the most effectual methods of
persecution. In the space of fifteen years, he promulgated at
least fifteen severe edicts against the heretics;\textsuperscript{48} more
especially against those who rejected the doctrine of the
Trinity; and to deprive them of every hope of escape, he sternly
enacted, that if any laws or rescripts should be alleged in their
favor, the judges should consider them as the illegal productions
either of fraud or forgery. The penal statutes were directed
against the ministers, the assemblies, and the persons of the
heretics; and the passions of the legislator were expressed in
the language of declamation and invective. I. The heretical
teachers, who usurped the sacred titles of Bishops, or
Presbyters, were not only excluded from the privileges and
emoluments so liberally granted to the orthodox clergy, but they
were exposed to the heavy penalties of exile and confiscation, if
they presumed to preach the doctrine, or to practise the rites,
of their accursed sects. A fine of ten pounds of gold (above four
hundred pounds sterling) was imposed on every person who should
dare to confer, or receive, or promote, an heretical ordination:
and it was reasonably expected, that if the race of pastors could
be extinguished, their helpless flocks would be compelled, by
ignorance and hunger, to return within the pale of the Catholic
church. II. The rigorous prohibition of conventicles was
carefully extended to every possible circumstance, in which the
heretics could assemble with the intention of worshipping God and
Christ according to the dictates of their conscience. Their
religious meetings, whether public or secret, by day or by night,
in cities or in the country, were equally proscribed by the
edicts of Theodosius; and the building, or ground, which had been
used for that illegal purpose, was forfeited to the Imperial
domain. III. It was supposed, that the error of the heretics
could proceed only from the obstinate temper of their minds; and
that such a temper was a fit object of censure and punishment.
The anathemas of the church were fortified by a sort of civil
excommunication; which separated them from their fellow-citizens,
by a peculiar brand of infamy; and this declaration of the
supreme magistrate tended to justify, or at least to excuse, the
insults of a fanatic populace. The sectaries were gradually
disqualified from the possession of honorable or lucrative
employments; and Theodosius was satisfied with his own justice,
when he decreed, that, as the Eunomians distinguished the nature
of the Son from that of the Father, they should be incapable of
making their wills or of receiving any advantage from
testamentary donations. The guilt of the Manichaean heresy was
esteemed of such magnitude, that it could be expiated only by the
death of the offender; and the same capital punishment was
inflicted on the Audians, or Quartodecimans,\textsuperscript{49} who should dare
to perpetrate the atrocious crime of celebrating on an improper
day the festival of Easter. Every Roman might exercise the right
of public accusation; but the office of Inquisitors of the Faith,
a name so deservedly abhorred, was first instituted under the
reign of Theodosius. Yet we are assured, that the execution of
his penal edicts was seldom enforced; and that the pious emperor
appeared less desirous to punish, than to reclaim, or terrify,
his refractory subjects.\textsuperscript{50}

\pagenote[48]{See the Theodosian Code, l. xvi. tit. v. leg. 6—23,
with Godefroy’s commentary on each law, and his general summary,
or Paratitlon, tom vi. p. 104-110.}

\pagenote[49]{They always kept their Easter, like the Jewish
Passover, on the fourteenth day of the first moon after the
vernal equinox; and thus pertinaciously opposed the Roman Church
and Nicene synod, which had fixed Easter to a Sunday. Bingham’s
Antiquities, l. xx. c. 5, vol. ii. p. 309, fol. edit.}

\pagenote[50]{Sozomen, l. vii. c. 12.}

The theory of persecution was established by Theodosius, whose
justice and piety have been applauded by the saints: but the
practice of it, in the fullest extent, was reserved for his rival
and colleague, Maximus, the first, among the Christian princes,
who shed the blood of his Christian subjects on account of their
religious opinions. The cause of the Priscillianists,\textsuperscript{51} a recent
sect of heretics, who disturbed the provinces of Spain, was
transferred, by appeal, from the synod of Bordeaux to the
Imperial consistory of Treves; and by the sentence of the
Prætorian præfect, seven persons were tortured, condemned, and
executed. The first of these was Priscillian\textsuperscript{52} himself, bishop
of Avila, in Spain; who adorned the advantages of birth and
fortune, by the accomplishments of eloquence and learning.\textsuperscript{53} Two
presbyters, and two deacons, accompanied their beloved master in
his death, which they esteemed as a glorious martyrdom; and the
number of religious victims was completed by the execution of
Latronian, a poet, who rivalled the fame of the ancients; and of
Euchrocia, a noble matron of Bordeaux, the widow of the orator
Delphidius.\textsuperscript{54} Two bishops who had embraced the sentiments of
Priscillian, were condemned to a distant and dreary exile;\textsuperscript{55} and
some indulgence was shown to the meaner criminals, who assumed
the merit of an early repentance. If any credit could be allowed
to confessions extorted by fear or pain, and to vague reports,
the offspring of malice and credulity, the heresy of the
Priscillianists would be found to include the various
abominations of magic, of impiety, and of lewdness.\textsuperscript{56}
Priscillian, who wandered about the world in the company of his
spiritual sisters, was accused of praying stark naked in the
midst of the congregation; and it was confidently asserted, that
the effects of his criminal intercourse with the daughter of
Euchrocia had been suppressed, by means still more odious and
criminal. But an accurate, or rather a candid, inquiry will
discover, that if the Priscillianists violated the laws of
nature, it was not by the licentiousness, but by the austerity,
of their lives. They absolutely condemned the use of the
marriage-bed; and the peace of families was often disturbed by
indiscreet separations. They enjoyed, or recommended, a total
abstinence from all animal food; and their continual prayers,
fasts, and vigils, inculcated a rule of strict and perfect
devotion. The speculative tenets of the sect, concerning the
person of Christ, and the nature of the human soul, were derived
from the Gnostic and Manichaean system; and this vain philosophy,
which had been transported from Egypt to Spain, was ill adapted
to the grosser spirits of the West. The obscure disciples of
Priscillian suffered languished, and gradually disappeared: his
tenets were rejected by the clergy and people, but his death was
the subject of a long and vehement controversy; while some
arraigned, and others applauded, the justice of his sentence. It
is with pleasure that we can observe the humane inconsistency of
the most illustrious saints and bishops, Ambrose of Milan,\textsuperscript{57} and
Martin of Tours,\textsuperscript{58} who, on this occasion, asserted the cause of
toleration. They pitied the unhappy men, who had been executed at
Treves; they refused to hold communion with their episcopal
murderers; and if Martin deviated from that generous resolution,
his motives were laudable, and his repentance was exemplary. The
bishops of Tours and Milan pronounced, without hesitation, the
eternal damnation of heretics; but they were surprised, and
shocked, by the bloody image of their temporal death, and the
honest feelings of nature resisted the artificial prejudices of
theology. The humanity of Ambrose and Martin was confirmed by the
scandalous irregularity of the proceedings against Priscillian
and his adherents. The civil and ecclesiastical ministers had
transgressed the limits of their respective provinces. The
secular judge had presumed to receive an appeal, and to pronounce
a definitive sentence, in a matter of faith, and episcopal
jurisdiction. The bishops had disgraced themselves, by exercising
the functions of accusers in a criminal prosecution. The cruelty
of Ithacius,\textsuperscript{59} who beheld the tortures, and solicited the death,
of the heretics, provoked the just indignation of mankind; and
the vices of that profligate bishop were admitted as a proof,
that his zeal was instigated by the sordid motives of interest.
Since the death of Priscillian, the rude attempts of persecution
have been refined and methodized in the holy office, which
assigns their distinct parts to the ecclesiastical and secular
powers. The devoted victim is regularly delivered by the priest
to the magistrate, and by the magistrate to the executioner; and
the inexorable sentence of the church, which declares the
spiritual guilt of the offender, is expressed in the mild
language of pity and intercession.

\pagenote[51]{See the Sacred History of Sulpicius Severus, (l.
ii. p. 437-452, edit. Ludg. Bat. 1647,) a correct and original
writer. Dr. Lardner (Credibility, \&c., part ii. vol. ix. p.
256-350) has labored this article with pure learning, good sense,
and moderation. Tillemont (Mem. Eccles. tom. viii. p. 491-527)
has raked together all the dirt of the fathers; a useful
scavenger!}

\pagenote[52]{Severus Sulpicius mentions the arch-heretic with
esteem and pity Faelix profecto, si non pravo studio corrupisset
optimum ingenium prorsus multa in eo animi et corporis bona
cerneres. (Hist. Sacra, l ii. p. 439.) Even Jerom (tom. i. in
Script. Eccles. p. 302) speaks with temper of Priscillian and
Latronian.}

\pagenote[53]{The bishopric (in Old Castile) is now worth 20,000
ducats a year, (Busching’s Geography, vol. ii. p. 308,) and is
therefore much less likely to produce the author of a new
heresy.}

\pagenote[54]{Exprobrabatur mulieri viduae nimia religio, et
diligentius culta divinitas, (Pacat. in Panegyr. Vet. xii. 29.)
Such was the idea of a humane, though ignorant, polytheist.}

\pagenote[55]{One of them was sent in Sillinam insulam quae ultra
Britannianest. What must have been the ancient condition of the
rocks of Scilly? (Camden’s Britannia, vol. ii. p. 1519.)}

\pagenote[56]{The scandalous calumnies of Augustin, Pope Leo,
\&c., which Tillemont swallows like a child, and Lardner refutes
like a man, may suggest some candid suspicions in favor of the
older Gnostics.}

\pagenote[57]{Ambros. tom. ii. Epist. xxiv. p. 891.}

\pagenote[58]{In the Sacred History, and the Life of St. Martin,
Sulpicius Severus uses some caution; but he declares himself more
freely in the Dialogues, (iii. 15.) Martin was reproved, however,
by his own conscience, and by an angel; nor could he afterwards
perform miracles with so much ease.}

\pagenote[59]{The Catholic Presbyter (Sulp. Sever. l. ii. p. 448)
and the Pagan Orator (Pacat. in Panegyr. Vet. xii. 29) reprobate,
with equal indignation, the character and conduct of Ithacius.}

\section{Part \thesection.}

Among the ecclesiastics, who illustrated the reign of Theodosius,
Gregory Nazianzen was distinguished by the talents of an eloquent
preacher; the reputation of miraculous gifts added weight and
dignity to the monastic virtues of Martin of Tours;\textsuperscript{60} but the
palm of episcopal vigor and ability was justly claimed by the
intrepid Ambrose.\textsuperscript{61} He was descended from a noble family of
Romans; his father had exercised the important office of
Prætorian præfect of Gaul; and the son, after passing through
the studies of a liberal education, attained, in the regular
gradation of civil honors, the station of consular of Liguria, a
province which included the Imperial residence of Milan. At the
age of thirty-four, and before he had received the sacrament of
baptism, Ambrose, to his own surprise, and to that of the world,
was suddenly transformed from a governor to an archbishop.
Without the least mixture, as it is said, of art or intrigue, the
whole body of the people unanimously saluted him with the
episcopal title; the concord and perseverance of their
acclamations were ascribed to a praeternatural impulse; and the
reluctant magistrate was compelled to undertake a spiritual
office, for which he was not prepared by the habits and
occupations of his former life. But the active force of his
genius soon qualified him to exercise, with zeal and prudence,
the duties of his ecclesiastical jurisdiction; and while he
cheerfully renounced the vain and splendid trappings of temporal
greatness, he condescended, for the good of the church, to direct
the conscience of the emperors, and to control the administration
of the empire. Gratian loved and revered him as a father; and the
elaborate treatise on the faith of the Trinity was designed for
the instruction of the young prince. After his tragic death, at a
time when the empress Justina trembled for her own safety, and
for that of her son Valentinian, the archbishop of Milan was
despatched, on two different embassies, to the court of Treves.
He exercised, with equal firmness and dexterity, the powers of
his spiritual and political characters; and perhaps contributed,
by his authority and eloquence, to check the ambition of Maximus,
and to protect the peace of Italy.\textsuperscript{62} Ambrose had devoted his
life, and his abilities, to the service of the church. Wealth was
the object of his contempt; he had renounced his private
patrimony; and he sold, without hesitation, the consecrated
plate, for the redemption of captives. The clergy and people of
Milan were attached to their archbishop; and he deserved the
esteem, without soliciting the favor, or apprehending the
displeasure, of his feeble sovereigns.

\pagenote[60]{The Life of St. Martin, and the Dialogues
concerning his miracles contain facts adapted to the grossest
barbarism, in a style not unworthy of the Augustan age. So
natural is the alliance between good taste and good sense, that I
am always astonished by this contrast.}

\pagenote[61]{The short and superficial Life of St. Ambrose, by
his deacon Paulinus, (Appendix ad edit. Benedict. p. i.—xv.,) has
the merit of original evidence. Tillemont (Mem. Eccles. tom. x.
p. 78-306) and the Benedictine editors (p. xxxi.—lxiii.) have
labored with their usual diligence.}

\pagenote[62]{Ambrose himself (tom. ii. Epist. xxiv. p. 888—891)
gives the emperor a very spirited account of his own embassy.}

The government of Italy, and of the young emperor, naturally
devolved to his mother Justina, a woman of beauty and spirit, but
who, in the midst of an orthodox people, had the misfortune of
professing the Arian heresy, which she endeavored to instil into
the mind of her son. Justina was persuaded, that a Roman emperor
might claim, in his own dominions, the public exercise of his
religion; and she proposed to the archbishop, as a moderate and
reasonable concession, that he should resign the use of a single
church, either in the city or the suburbs of Milan. But the
conduct of Ambrose was governed by very different principles.\textsuperscript{63}
The palaces of the earth might indeed belong to Caesar; but the
churches were the houses of God; and, within the limits of his
diocese, he himself, as the lawful successor of the apostles, was
the only minister of God. The privileges of Christianity,
temporal as well as spiritual, were confined to the true
believers; and the mind of Ambrose was satisfied, that his own
theological opinions were the standard of truth and orthodoxy.
The archbishop, who refused to hold any conference, or
negotiation, with the instruments of Satan, declared, with modest
firmness, his resolution to die a martyr, rather than to yield to
the impious sacrilege; and Justina, who resented the refusal as
an act of insolence and rebellion, hastily determined to exert
the Imperial prerogative of her son. As she desired to perform
her public devotions on the approaching festival of Easter,
Ambrose was ordered to appear before the council. He obeyed the
summons with the respect of a faithful subject, but he was
followed, without his consent, by an innumerable people; they
pressed, with impetuous zeal, against the gates of the palace;
and the affrighted ministers of Valentinian, instead of
pronouncing a sentence of exile on the archbishop of Milan,
humbly requested that he would interpose his authority, to
protect the person of the emperor, and to restore the tranquility
of the capital. But the promises which Ambrose received and
communicated were soon violated by a perfidious court; and,
during six of the most solemn days, which Christian piety had set
apart for the exercise of religion, the city was agitated by the
irregular convulsions of tumult and fanaticism. The officers of
the household were directed to prepare, first, the Portian, and
afterwards, the new, Basilica, for the immediate reception of the
emperor and his mother. The splendid canopy and hangings of the
royal seat were arranged in the customary manner; but it was
found necessary to defend them. by a strong guard, from the
insults of the populace. The Arian ecclesiastics, who ventured to
show themselves in the streets, were exposed to the most imminent
danger of their lives; and Ambrose enjoyed the merit and
reputation of rescuing his personal enemies from the hands of the
enraged multitude.

\pagenote[63]{His own representation of his principles and
conduct (tom. ii. Epist. xx xxi. xxii. p. 852-880) is one of the
curious monuments of ecclesiastical antiquity. It contains two
letters to his sister Marcellina, with a petition to Valentinian
and the sermon de Basilicis non madendis.}

But while he labored to restrain the effects of their zeal, the
pathetic vehemence of his sermons continually inflamed the angry
and seditious temper of the people of Milan. The characters of
Eve, of the wife of Job, of Jezebel, of Herodias, were indecently
applied to the mother of the emperor; and her desire to obtain a
church for the Arians was compared to the most cruel persecutions
which Christianity had endured under the reign of Paganism. The
measures of the court served only to expose the magnitude of the
evil. A fine of two hundred pounds of gold was imposed on the
corporate body of merchants and manufacturers: an order was
signified, in the name of the emperor, to all the officers, and
inferior servants, of the courts of justice, that, during the
continuance of the public disorders, they should strictly confine
themselves to their houses; and the ministers of Valentinian
imprudently confessed, that the most respectable part of the
citizens of Milan was attached to the cause of their archbishop.
He was again solicited to restore peace to his country, by timely
compliance with the will of his sovereign. The reply of Ambrose
was couched in the most humble and respectful terms, which might,
however, be interpreted as a serious declaration of civil war.
“His life and fortune were in the hands of the emperor; but he
would never betray the church of Christ, or degrade the dignity
of the episcopal character. In such a cause he was prepared to
suffer whatever the malice of the daemon could inflict; and he
only wished to die in the presence of his faithful flock, and at
the foot of the altar; he had not contributed to excite, but it
was in the power of God alone to appease, the rage of the people:
he deprecated the scenes of blood and confusion which were likely
to ensue; and it was his fervent prayer, that he might not
survive to behold the ruin of a flourishing city, and perhaps the
desolation of all Italy.”\textsuperscript{64} The obstinate bigotry of Justina
would have endangered the empire of her son, if, in this contest
with the church and people of Milan, she could have depended on
the active obedience of the troops of the palace. A large body of
Goths had marched to occupy the Basilica, which was the object of
the dispute: and it might be expected from the Arian principles,
and barbarous manners, of these foreign mercenaries, that they
would not entertain any scruples in the execution of the most
sanguinary orders. They were encountered, on the sacred
threshold, by the archbishop, who, thundering against them a
sentence of excommunication, asked them, in the tone of a father
and a master, whether it was to invade the house of God, that
they had implored the hospitable protection of the republic. The
suspense of the Barbarians allowed some hours for a more
effectual negotiation; and the empress was persuaded, by the
advice of her wisest counsellors, to leave the Catholics in
possession of all the churches of Milan; and to dissemble, till a
more convenient season, her intentions of revenge. The mother of
Valentinian could never forgive the triumph of Ambrose; and the
royal youth uttered a passionate exclamation, that his own
servants were ready to betray him into the hands of an insolent
priest.

\pagenote[64]{Retz had a similar message from the queen, to
request that he would appease the tumult of Paris. It was no
longer in his power, \&c. A quoi j’ajoutai tout ce que vous pouvez
vous imaginer de respect de douleur, de regret, et de soumission,
\&c. (Mémoires, tom. i. p. 140.) Certainly I do not compare either
the causes or the men yet the coadjutor himself had some idea (p.
84) of imitating St. Ambrose}

The laws of the empire, some of which were inscribed with the
name of Valentinian, still condemned the Arian heresy, and seemed
to excuse the resistance of the Catholics. By the influence of
Justina, an edict of toleration was promulgated in all the
provinces which were subject to the court of Milan; the free
exercise of their religion was granted to those who professed the
faith of Rimini; and the emperor declared, that all persons who
should infringe this sacred and salutary constitution, should be
capitally punished, as the enemies of the public peace.\textsuperscript{65} The
character and language of the archbishop of Milan may justify the
suspicion, that his conduct soon afforded a reasonable ground, or
at least a specious pretence, to the Arian ministers; who watched
the opportunity of surprising him in some act of disobedience to
a law which he strangely represents as a law of blood and
tyranny. A sentence of easy and honorable banishment was
pronounced, which enjoined Ambrose to depart from Milan without
delay; whilst it permitted him to choose the place of his exile,
and the number of his companions. But the authority of the
saints, who have preached and practised the maxims of passive
loyalty, appeared to Ambrose of less moment than the extreme and
pressing danger of the church. He boldly refused to obey; and his
refusal was supported by the unanimous consent of his faithful
people.\textsuperscript{66} They guarded by turns the person of their archbishop;
the gates of the cathedral and the episcopal palace were strongly
secured; and the Imperial troops, who had formed the blockade,
were unwilling to risk the attack, of that impregnable fortress.
The numerous poor, who had been relieved by the liberality of
Ambrose, embraced the fair occasion of signalizing their zeal and
gratitude; and as the patience of the multitude might have been
exhausted by the length and uniformity of nocturnal vigils, he
prudently introduced into the church of Milan the useful
institution of a loud and regular psalmody. While he maintained
this arduous contest, he was instructed, by a dream, to open the
earth in a place where the remains of two martyrs, Gervasius and
Protasius,\textsuperscript{67} had been deposited above three hundred years.
Immediately under the pavement of the church two perfect
skeletons were found,\textsuperscript{68} with the heads separated from their
bodies, and a plentiful effusion of blood. The holy relics were
presented, in solemn pomp, to the veneration of the people; and
every circumstance of this fortunate discovery was admirably
adapted to promote the designs of Ambrose. The bones of the
martyrs, their blood, their garments, were supposed to contain a
healing power; and the praeternatural influence was communicated
to the most distant objects, without losing any part of its
original virtue. The extraordinary cure of a blind man,\textsuperscript{69} and
the reluctant confessions of several daemoniacs, appeared to
justify the faith and sanctity of Ambrose; and the truth of those
miracles is attested by Ambrose himself, by his secretary
Paulinus, and by his proselyte, the celebrated Augustin, who, at
that time, professed the art of rhetoric in Milan. The reason of
the present age may possibly approve the incredulity of Justina
and her Arian court; who derided the theatrical representations
which were exhibited by the contrivance, and at the expense, of
the archbishop.\textsuperscript{70} Their effect, however, on the minds of the
people, was rapid and irresistible; and the feeble sovereign of
Italy found himself unable to contend with the favorite of
Heaven. The powers likewise of the earth interposed in the
defence of Ambrose: the disinterested advice of Theodosius was
the genuine result of piety and friendship; and the mask of
religious zeal concealed the hostile and ambitious designs of the
tyrant of Gaul.\textsuperscript{71}

\pagenote[65]{Sozomen alone (l. vii. c. 13) throws this luminous
fact into a dark and perplexed narrative.}

\pagenote[66]{Excubabat pia plebs in ecclesia, mori parata cum
episcopo suo.... Nos, adhuc frigidi, excitabamur tamen civitate
attonita atque curbata. Augustin. Confession. l. ix. c. 7}

\pagenote[67]{Tillemont, Mem. Eccles. tom. ii. p. 78, 498. Many
churches in Italy, Gaul, \&c., were dedicated to these unknown
martyrs, of whom St. Gervaise seems to have been more fortunate
than his companion.}

\pagenote[68]{Invenimus mirae magnitudinis viros duos, ut prisca
aetas ferebat, tom. ii. Epist. xxii. p. 875. The size of these
skeletons was fortunately, or skillfully, suited to the popular
prejudice of the gradual decrease of the human stature, which has
prevailed in every age since the time of Homer.—Grandiaque
effossis mirabitur ossa sepulchris.}

\pagenote[69]{Ambros. tom. ii. Epist. xxii. p. 875. Augustin.
Confes, l. ix. c. 7, de Civitat. Dei, l. xxii. c. 8. Paulin. in
Vita St. Ambros. c. 14, in Append. Benedict. p. 4. The blind
man’s name was Severus; he touched the holy garment, recovered
his sight, and devoted the rest of his life (at least twenty-five
years) to the service of the church. I should recommend this
miracle to our divines, if it did not prove the worship of
relics, as well as the Nicene creed.}

\pagenote[70]{Paulin, in Tit. St. Ambros. c. 5, in Append.
Benedict. p. 5.}

\pagenote[71]{Tillemont, Mem. Eccles. tom. x. p. 190, 750. He
partially allow the mediation of Theodosius, and capriciously
rejects that of Maximus, though it is attested by Prosper,
Sozomen, and Theodoret.}

The reign of Maximus might have ended in peace and prosperity,
could he have contented himself with the possession of three
ample countries, which now constitute the three most flourishing
kingdoms of modern Europe. But the aspiring usurper, whose sordid
ambition was not dignified by the love of glory and of arms,
considered his actual forces as the instruments only of his
future greatness, and his success was the immediate cause of his
destruction. The wealth which he extorted\textsuperscript{72} from the oppressed
provinces of Gaul, Spain, and Britain, was employed in levying
and maintaining a formidable army of Barbarians, collected, for
the most part, from the fiercest nations of Germany. The conquest
of Italy was the object of his hopes and preparations: and he
secretly meditated the ruin of an innocent youth, whose
government was abhorred and despised by his Catholic subjects.
But as Maximus wished to occupy, without resistance, the passes
of the Alps, he received, with perfidious smiles, Domninus of
Syria, the ambassador of Valentinian, and pressed him to accept
the aid of a considerable body of troops, for the service of a
Pannonian war. The penetration of Ambrose had discovered the
snares of an enemy under the professions of friendship;\textsuperscript{73} but
the Syrian Domninus was corrupted, or deceived, by the liberal
favor of the court of Treves; and the council of Milan
obstinately rejected the suspicion of danger, with a blind
confidence, which was the effect, not of courage, but of fear.
The march of the auxiliaries was guided by the ambassador; and
they were admitted, without distrust, into the fortresses of the
Alps. But the crafty tyrant followed, with hasty and silent
footsteps, in the rear; and, as he diligently intercepted all
intelligence of his motions, the gleam of armor, and the dust
excited by the troops of cavalry, first announced the hostile
approach of a stranger to the gates of Milan. In this extremity,
Justina and her son might accuse their own imprudence, and the
perfidious arts of Maximus; but they wanted time, and force, and
resolution, to stand against the Gauls and Germans, either in the
field, or within the walls of a large and disaffected city.
Flight was their only hope, Aquileia their only refuge; and as
Maximus now displayed his genuine character, the brother of
Gratian might expect the same fate from the hands of the same
assassin. Maximus entered Milan in triumph; and if the wise
archbishop refused a dangerous and criminal connection with the
usurper, he might indirectly contribute to the success of his
arms, by inculcating, from the pulpit, the duty of resignation,
rather than that of resistance.\textsuperscript{74} The unfortunate Justina
reached Aquileia in safety; but she distrusted the strength of
the fortifications: she dreaded the event of a siege; and she
resolved to implore the protection of the great Theodosius, whose
power and virtue were celebrated in all the countries of the
West. A vessel was secretly provided to transport the Imperial
family; they embarked with precipitation in one of the obscure
harbors of Venetia, or Istria; traversed the whole extent of the
Adriatic and Ionian Seas; turned the extreme promontory of
Peloponnesus; and, after a long, but successful navigation,
reposed themselves in the port of Thessalonica. All the subjects
of Valentinian deserted the cause of a prince, who, by his
abdication, had absolved them from the duty of allegiance; and if
the little city of Aemona, on the verge of Italy, had not
presumed to stop the career of his inglorious victory, Maximus
would have obtained, without a struggle, the sole possession of
the Western empire.

\pagenote[72]{The modest censure of Sulpicius (Dialog. iii. 15)
inflicts a much deeper wound than the declamation of Pacatus,
(xii. 25, 26.)}

\pagenote[73]{Esto tutior adversus hominem, pacis involurco
tegentem, was the wise caution of Ambrose (tom. ii. p. 891) after
his return from his second embassy.}

\pagenote[74]{Baronius (A.D. 387, No. 63) applies to this season
of public distress some of the penitential sermons of the
archbishop.}

Instead of inviting his royal guests to take the palace of
Constantinople, Theodosius had some unknown reasons to fix their
residence at Thessalonica; but these reasons did not proceed from
contempt or indifference, as he speedily made a visit to that
city, accompanied by the greatest part of his court and senate.
After the first tender expressions of friendship and sympathy,
the pious emperor of the East gently admonished Justina, that the
guilt of heresy was sometimes punished in this world, as well as
in the next; and that the public profession of the Nicene faith
would be the most efficacious step to promote the restoration of
her son, by the satisfaction which it must occasion both on earth
and in heaven. The momentous question of peace or war was
referred, by Theodosius, to the deliberation of his council; and
the arguments which might be alleged on the side of honor and
justice, had acquired, since the death of Gratian, a considerable
degree of additional weight. The persecution of the Imperial
family, to which Theodosius himself had been indebted for his
fortune, was now aggravated by recent and repeated injuries.
Neither oaths nor treaties could restrain the boundless ambition
of Maximus; and the delay of vigorous and decisive measures,
instead of prolonging the blessings of peace, would expose the
Eastern empire to the danger of a hostile invasion. The
Barbarians, who had passed the Danube, had lately assumed the
character of soldiers and subjects, but their native fierceness
was yet untamed: and the operations of a war, which would
exercise their valor, and diminish their numbers, might tend to
relieve the provinces from an intolerable oppression.
Notwithstanding these specious and solid reasons, which were
approved by a majority of the council, Theodosius still hesitated
whether he should draw the sword in a contest which could no
longer admit any terms of reconciliation; and his magnanimous
character was not disgraced by the apprehensions which he felt
for the safety of his infant sons, and the welfare of his
exhausted people. In this moment of anxious doubt, while the fate
of the Roman world depended on the resolution of a single man,
the charms of the princess Galla most powerfully pleaded the
cause of her brother Valentinian.\textsuperscript{75} The heart of Theodosius wa
softened by the tears of beauty; his affections were insensibly
engaged by the graces of youth and innocence: the art of Justina
managed and directed the impulse of passion; and the celebration
of the royal nuptials was the assurance and signal of the civil
war. The unfeeling critics, who consider every amorous weakness
as an indelible stain on the memory of a great and orthodox
emperor, are inclined, on this occasion, to dispute the
suspicious evidence of the historian Zosimus. For my own part, I
shall frankly confess, that I am willing to find, or even to
seek, in the revolutions of the world, some traces of the mild
and tender sentiments of domestic life; and amidst the crowd of
fierce and ambitious conquerors, I can distinguish, with peculiar
complacency, a gentle hero, who may be supposed to receive his
armor from the hands of love. The alliance of the Persian king
was secured by the faith of treaties; the martial Barbarians were
persuaded to follow the standard, or to respect the frontiers, of
an active and liberal monarch; and the dominions of Theodosius,
from the Euphrates to the Adriatic, resounded with the
preparations of war both by land and sea. The skilful disposition
of the forces of the East seemed to multiply their numbers, and
distracted the attention of Maximus. He had reason to fear, that
a chosen body of troops, under the command of the intrepid
Arbogastes, would direct their march along the banks of the
Danube, and boldly penetrate through the Rhaetian provinces into
the centre of Gaul. A powerful fleet was equipped in the harbors
of Greece and Epirus, with an apparent design, that, as soon as
the passage had been opened by a naval victory, Valentinian and
his mother should land in Italy, proceed, without delay, to Rome,
and occupy the majestic seat of religion and empire. In the mean
while, Theodosius himself advanced at the head of a brave and
disciplined army, to encounter his unworthy rival, who, after the
siege of Aemona,\textsuperscript{7511} had fixed his camp in the neighborhood of
Siscia, a city of Pannonia, strongly fortified by the broad and
rapid stream of the Save.

\pagenote[75]{The flight of Valentinian, and the love of
Theodosius for his sister, are related by Zosimus, (l. iv. p.
263, 264.) Tillemont produces some weak and ambiguous evidence to
antedate the second marriage of Theodosius, (Hist. des Empereurs,
to. v. p. 740,) and consequently to refute ces contes de Zosime,
qui seroient trop contraires a la piete de Theodose.}

\pagenote[7511]{Aemonah, Laybach. Siscia Sciszek.—M.}

\section{Part \thesection.}

The veterans, who still remembered the long resistance, and
successive resources, of the tyrant Magnentius, might prepare
themselves for the labors of three bloody campaigns. But the
contest with his successor, who, like him, had usurped the throne
of the West, was easily decided in the term of two months,\textsuperscript{76} and
within the space of two hundred miles. The superior genius of the
emperor of the East might prevail over the feeble Maximus, who,
in this important crisis, showed himself destitute of military
skill, or personal courage; but the abilities of Theodosius were
seconded by the advantage which he possessed of a numerous and
active cavalry. The Huns, the Alani, and, after their example,
the Goths themselves, were formed into squadrons of archers; who
fought on horseback, and confounded the steady valor of the Gauls
and Germans, by the rapid motions of a Tartar war. After the
fatigue of a long march, in the heat of summer, they spurred
their foaming horses into the waters of the Save, swam the river
in the presence of the enemy, and instantly charged and routed
the troops who guarded the high ground on the opposite side.
Marcellinus, the tyrant’s brother, advanced to support them with
the select cohorts, which were considered as the hope and
strength of the army. The action, which had been interrupted by
the approach of night, was renewed in the morning; and, after a
sharp conflict, the surviving remnant of the bravest soldiers of
Maximus threw down their arms at the feet of the conqueror.
Without suspending his march, to receive the loyal acclamations
of the citizens of Aemona, Theodosius pressed forwards to
terminate the war by the death or captivity of his rival, who
fled before him with the diligence of fear. From the summit of
the Julian Alps, he descended with such incredible speed into the
plain of Italy, that he reached Aquileia on the evening of the
first day; and Maximus, who found himself encompassed on all
sides, had scarcely time to shut the gates of the city. But the
gates could not long resist the effort of a victorious enemy; and
the despair, the disaffection, the indifference of the soldiers
and people, hastened the downfall of the wretched Maximus. He was
dragged from his throne, rudely stripped of the Imperial
ornaments, the robe, the diadem, and the purple slippers; and
conducted, like a malefactor, to the camp and presence of
Theodosius, at a place about three miles from Aquileia. The
behavior of the emperor was not intended to insult, and he showed
disposition to pity and forgive, the tyrant of the West, who had
never been his personal enemy, and was now become the object of
his contempt. Our sympathy is the most forcibly excited by the
misfortunes to which we are exposed; and the spectacle of a proud
competitor, now prostrate at his feet, could not fail of
producing very serious and solemn thoughts in the mind of the
victorious emperor. But the feeble emotion of involuntary pity
was checked by his regard for public justice, and the memory of
Gratian; and he abandoned the victim to the pious zeal of the
soldiers, who drew him out of the Imperial presence, and
instantly separated his head from his body. The intelligence of
his defeat and death was received with sincere or well-dissembled
joy: his son Victor, on whom he had conferred the title of
Augustus, died by the order, perhaps by the hand, of the bold
Arbogastes; and all the military plans of Theodosius were
successfully executed. When he had thus terminated the civil war,
with less difficulty and bloodshed than he might naturally
expect, he employed the winter months of his residence at Milan,
to restore the state of the afflicted provinces; and early in the
spring he made, after the example of Constantine and Constantius,
his triumphal entry into the ancient capital of the Roman empire.\textsuperscript{77}

\pagenote[76]{See Godefroy’s Chronology of the Laws, Cod.
Theodos, tom l. p. cxix.}

\pagenote[77]{Besides the hints which may be gathered from
chronicles and ecclesiastical history, Zosimus (l. iv. p.
259—267,) Orosius, (l. vii. c. 35,) and Pacatus, (in Panegyr.
Vet. xii. 30-47,) supply the loose and scanty materials of this
civil war. Ambrose (tom. ii. Epist. xl. p. 952, 953) darkly
alludes to the well-known events of a magazine surprised, an
action at Petovio, a Sicilian, perhaps a naval, victory, \&c.,
Ausonius (p. 256, edit. Toll.) applauds the peculiar merit and
good fortune of Aquileia.}

The orator, who may be silent without danger, may praise without
difficulty, and without reluctance;\textsuperscript{78} and posterity will
confess, that the character of Theodosius\textsuperscript{79} might furnish the
subject of a sincere and ample panegyric. The wisdom of his laws,
and the success of his arms, rendered his administration
respectable in the eyes both of his subjects and of his enemies.
He loved and practised the virtues of domestic life, which seldom
hold their residence in the palaces of kings. Theodosius was
chaste and temperate; he enjoyed, without excess, the sensual and
social pleasures of the table; and the warmth of his amorous
passions was never diverted from their lawful objects. The proud
titles of Imperial greatness were adorned by the tender names of
a faithful husband, an indulgent father; his uncle was raised, by
his affectionate esteem, to the rank of a second parent:
Theodosius embraced, as his own, the children of his brother and
sister; and the expressions of his regard were extended to the
most distant and obscure branches of his numerous kindred. His
familiar friends were judiciously selected from among those
persons, who, in the equal intercourse of private life, had
appeared before his eyes without a mask; the consciousness of
personal and superior merit enabled him to despise the accidental
distinction of the purple; and he proved by his conduct, that he
had forgotten all the injuries, while he most gratefully
remembered all the favors and services, which he had received
before he ascended the throne of the Roman empire. The serious or
lively tone of his conversation was adapted to the age, the rank,
or the character of his subjects, whom he admitted into his
society; and the affability of his manners displayed the image of
his mind. Theodosius respected the simplicity of the good and
virtuous: every art, every talent, of a useful, or even of an
innocent nature, was rewarded by his judicious liberality; and,
except the heretics, whom he persecuted with implacable hatred,
the diffusive circle of his benevolence was circumscribed only by
the limits of the human race. The government of a mighty empire
may assuredly suffice to occupy the time, and the abilities, of a
mortal: yet the diligent prince, without aspiring to the
unsuitable reputation of profound learning, always reserved some
moments of his leisure for the instructive amusement of reading.
History, which enlarged his experience, was his favorite study.
The annals of Rome, in the long period of eleven hundred years,
presented him with a various and splendid picture of human life:
and it has been particularly observed, that whenever he perused
the cruel acts of Cinna, of Marius, or of Sylla, he warmly
expressed his generous detestation of those enemies of humanity
and freedom. His disinterested opinion of past events was
usefully applied as the rule of his own actions; and Theodosius
has deserved the singular commendation, that his virtues always
seemed to expand with his fortune: the season of his prosperity
was that of his moderation; and his clemency appeared the most
conspicuous after the danger and success of a civil war. The
Moorish guards of the tyrant had been massacred in the first heat
of the victory, and a small number of the most obnoxious
criminals suffered the punishment of the law. But the emperor
showed himself much more attentive to relieve the innocent than
to chastise the guilty. The oppressed subjects of the West, who
would have deemed themselves happy in the restoration of their
lands, were astonished to receive a sum of money equivalent to
their losses; and the liberality of the conqueror supported the
aged mother, and educated the orphan daughters, of Maximus.\textsuperscript{80} A
character thus accomplished might almost excuse the extravagant
supposition of the orator Pacatus; that, if the elder Brutus
could be permitted to revisit the earth, the stern republican
would abjure, at the feet of Theodosius, his hatred of kings; and
ingenuously confess, that such a monarch was the most faithful
guardian of the happiness and dignity of the Roman people.\textsuperscript{81}

\pagenote[78]{Quam promptum laudare principem, tam tutum siluisse
de principe, (Pacat. in Panegyr. Vet. xii. 2.) Latinus Pacatus
Drepanius, a native of Gaul, pronounced this oration at Rome,
(A.D. 388.) He was afterwards proconsul of Africa; and his friend
Ausonius praises him as a poet second only to Virgil. See
Tillemont, Hist. des Empereurs, tom. v. p. 303.}

\pagenote[79]{See the fair portrait of Theodosius, by the younger
Victor; the strokes are distinct, and the colors are mixed. The
praise of Pacatus is too vague; and Claudian always seems afraid
of exalting the father above the son.}

\pagenote[80]{Ambros. tom. ii. Epist. xl. p. 55. Pacatus, from
the want of skill or of courage, omits this glorious
circumstance.}

\pagenote[81]{Pacat. in Panegyr. Vet. xii. 20.}

Yet the piercing eye of the founder of the republic must have
discerned two essential imperfections, which might, perhaps, have
abated his recent love of despostism. The virtuous mind of
Theodosius was often relaxed by indolence,\textsuperscript{82} and it was
sometimes inflamed by passion.\textsuperscript{83} In the pursuit of an important
object, his active courage was capable of the most vigorous
exertions; but, as soon as the design was accomplished, or the
danger was surmounted, the hero sunk into inglorious repose; and,
forgetful that the time of a prince is the property of his
people, resigned himself to the enjoyment of the innocent, but
trifling, pleasures of a luxurious court. The natural disposition
of Theodosius was hasty and choleric; and, in a station where
none could resist, and few would dissuade, the fatal consequence
of his resentment, the humane monarch was justly alarmed by the
consciousness of his infirmity and of his power. It was the
constant study of his life to suppress, or regulate, the
intemperate sallies of passion and the success of his efforts
enhanced the merit of his clemency. But the painful virtue which
claims the merit of victory, is exposed to the danger of defeat;
and the reign of a wise and merciful prince was polluted by an
act of cruelty which would stain the annals of Nero or Domitian.
Within the space of three years, the inconsistent historian of
Theodosius must relate the generous pardon of the citizens of
Antioch, and the inhuman massacre of the people of Thessalonica.

\pagenote[82]{Zosimus, l. iv. p. 271, 272. His partial evidence
is marked by an air of candor and truth. He observes these
vicissitudes of sloth and activity, not as a vice, but as a
singularity in the character of Theodosius.}

\pagenote[83]{This choleric temper is acknowledged and excused by
Victor Sed habes (says Ambrose, in decent and many language, to
his sovereign) nature impetum, quem si quis lenire velit, cito
vertes ad misericordiam: si quis stimulet, in magis exsuscitas,
ut eum revocare vix possis, (tom. ii. Epist. li. p. 998.)
Theodosius (Claud. in iv. Hon. 266, \&c.) exhorts his son to
moderate his anger.}

The lively impatience of the inhabitants of Antioch was never
satisfied with their own situation, or with the character and
conduct of their successive sovereigns. The Arian subjects of
Theodosius deplored the loss of their churches; and as three
rival bishops disputed the throne of Antioch, the sentence which
decided their pretensions excited the murmurs of the two
unsuccessful congregations. The exigencies of the Gothic war, and
the inevitable expense that accompanied the conclusion of the
peace, had constrained the emperor to aggravate the weight of the
public impositions; and the provinces of Asia, as they had not
been involved in the distress were the less inclined to
contribute to the relief, of Europe. The auspicious period now
approached of the tenth year of his reign; a festival more
grateful to the soldiers, who received a liberal donative, than
to the subjects, whose voluntary offerings had been long since
converted into an extraordinary and oppressive burden. The edicts
of taxation interrupted the repose, and pleasures, of Antioch;
and the tribunal of the magistrate was besieged by a suppliant
crowd; who, in pathetic, but, at first, in respectful language,
solicited the redress of their grievances. They were gradually
incensed by the pride of their haughty rulers, who treated their
complaints as a criminal resistance; their satirical wit
degenerated into sharp and angry invectives; and, from the
subordinate powers of government, the invectives of the people
insensibly rose to attack the sacred character of the emperor
himself. Their fury, provoked by a feeble opposition, discharged
itself on the images of the Imperial family, which were erected,
as objects of public veneration, in the most conspicuous places
of the city. The statues of Theodosius, of his father, of his
wife Flaccilla, of his two sons, Arcadius and Honorius, were
insolently thrown down from their pedestals, broken in pieces, or
dragged with contempt through the streets; and the indignities
which were offered to the representations of Imperial majesty,
sufficiently declared the impious and treasonable wishes of the
populace. The tumult was almost immediately suppressed by the
arrival of a body of archers: and Antioch had leisure to reflect
on the nature and consequences of her crime.\textsuperscript{84} According to the
duty of his office, the governor of the province despatched a
faithful narrative of the whole transaction: while the trembling
citizens intrusted the confession of their crime, and the
assurances of their repentance, to the zeal of Flavian, their
bishop, and to the eloquence of the senator Hilarius, the friend,
and most probably the disciple, of Libanius; whose genius, on
this melancholy occasion, was not useless to his country.\textsuperscript{85} But
the two capitals, Antioch and Constantinople, were separated by
the distance of eight hundred miles; and, notwithstanding the
diligence of the Imperial posts, the guilty city was severely
punished by a long and dreadful interval of suspense. Every rumor
agitated the hopes and fears of the Antiochians, and they heard
with terror, that their sovereign, exasperated by the insult
which had been offered to his own statues, and more especially,
to those of his beloved wife, had resolved to level with the
ground the offending city; and to massacre, without distinction
of age or sex, the criminal inhabitants;\textsuperscript{86} many of whom were
actually driven, by their apprehensions, to seek a refuge in the
mountains of Syria, and the adjacent desert. At length,
twenty-four days after the sedition, the general Hellebicus and
Caesarius, master of the offices, declared the will of the
emperor, and the sentence of Antioch. That proud capital was
degraded from the rank of a city; and the metropolis of the East,
stripped of its lands, its privileges, and its revenues, was
subjected, under the humiliating denomination of a village, to
the jurisdiction of Laodicea.\textsuperscript{87} The baths, the Circus, and the
theatres were shut: and, that every source of plenty and pleasure
might at the same time be intercepted, the distribution of corn
was abolished, by the severe instructions of Theodosius. His
commissioners then proceeded to inquire into the guilt of
individuals; of those who had perpetrated, and of those who had
not prevented, the destruction of the sacred statues. The
tribunal of Hellebicus and Caesarius, encompassed with armed
soldiers, was erected in the midst of the Forum. The noblest, and
most wealthy, of the citizens of Antioch appeared before them in
chains; the examination was assisted by the use of torture, and
their sentence was pronounced or suspended, according to the
judgment of these extraordinary magistrates. The houses of the
criminals were exposed to sale, their wives and children were
suddenly reduced, from affluence and luxury, to the most abject
distress; and a bloody execution was expected to conclude the
horrors of the day,\textsuperscript{88} which the preacher of Antioch, the
eloquent Chrysostom, has represented as a lively image of the
last and universal judgment of the world. But the ministers of
Theodosius performed, with reluctance, the cruel task which had
been assigned them; they dropped a gentle tear over the
calamities of the people; and they listened with reverence to the
pressing solicitations of the monks and hermits, who descended in
swarms from the mountains.\textsuperscript{89} Hellebicus and Caesarius were
persuaded to suspend the execution of their sentence; and it was
agreed that the former should remain at Antioch, while the latter
returned, with all possible speed, to Constantinople; and
presumed once more to consult the will of his sovereign. The
resentment of Theodosius had already subsided; the deputies of
the people, both the bishop and the orator, had obtained a
favorable audience; and the reproaches of the emperor were the
complaints of injured friendship, rather than the stern menaces
of pride and power. A free and general pardon was granted to the
city and citizens of Antioch; the prison doors were thrown open;
the senators, who despaired of their lives, recovered the
possession of their houses and estates; and the capital of the
East was restored to the enjoyment of her ancient dignity and
splendor. Theodosius condescended to praise the senate of
Constantinople, who had generously interceded for their
distressed brethren: he rewarded the eloquence of Hilarius with
the government of Palestine; and dismissed the bishop of Antioch
with the warmest expressions of his respect and gratitude. A
thousand new statues arose to the clemency of Theodosius; the
applause of his subjects was ratified by the approbation of his
own heart; and the emperor confessed, that, if the exercise of
justice is the most important duty, the indulgence of mercy is
the most exquisite pleasure, of a sovereign.\textsuperscript{90}

\pagenote[84]{The Christians and Pagans agreed in believing that
the sedition of Antioch was excited by the daemons. A gigantic
woman (says Sozomen, l. vii. c. 23) paraded the streets with a
scourge in her hand. An old man, says Libanius, (Orat. xii. p.
396,) transformed himself into a youth, then a boy, \&c.}

\pagenote[85]{Zosimus, in his short and disingenuous account, (l.
iv. p. 258, 259,) is certainly mistaken in sending Libanius
himself to Constantinople. His own orations fix him at Antioch.}

\pagenote[86]{Libanius (Orat. i. p. 6, edit. Venet.) declares,
that under such a reign the fear of a massacre was groundless and
absurd, especially in the emperor’s absence, for his presence,
according to the eloquent slave, might have given a sanction to
the most bloody acts.}

\pagenote[87]{Laodicea, on the sea-coast, sixty-five miles from
Antioch, (see Noris Epoch. Syro-Maced. Dissert. iii. p. 230.) The
Antiochians were offended, that the dependent city of Seleucia
should presume to intercede for them.}

\pagenote[88]{As the days of the tumult depend on the movable
festival of Easter, they can only be determined by the previous
determination of the year. The year 387 has been preferred, after
a laborious inquiry, by Tillemont (Hist. des. Emp. tom. v. p.
741-744) and Montfaucon, (Chrysostom, tom. xiii. p. 105-110.)}

\pagenote[89]{Chrysostom opposes their courage, which was not
attended with much risk, to the cowardly flight of the Cynics.}

\pagenote[90]{The sedition of Antioch is represented in a lively,
and almost dramatic, manner by two orators, who had their
respective shares of interest and merit. See Libanius (Orat. xiv.
xv. p. 389-420, edit. Morel. Orat. i. p. 1-14, Venet. 1754) and
the twenty orations of St. John Chrysostom, de Statuis, (tom. ii.
p. 1-225, edit. Montfaucon.) I do not pretend to much personal
acquaintance with Chrysostom but Tillemont (Hist. des. Empereurs,
tom. v. p. 263-283) and Hermant (Vie de St. Chrysostome, tom. i.
p. 137-224) had read him with pious curiosity and diligence.}

The sedition of Thessalonica is ascribed to a more shameful
cause, and was productive of much more dreadful consequences.
That great city, the metropolis of all the Illyrian provinces,
had been protected from the dangers of the Gothic war by strong
fortifications and a numerous garrison. Botheric, the general of
those troops, and, as it should seem from his name, a Barbarian,
had among his slaves a beautiful boy, who excited the impure
desires of one of the charioteers of the Circus. The insolent and
brutal lover was thrown into prison by the order of Botheric; and
he sternly rejected the importunate clamors of the multitude,
who, on the day of the public games, lamented the absence of
their favorite; and considered the skill of a charioteer as an
object of more importance than his virtue. The resentment of the
people was imbittered by some previous disputes; and, as the
strength of the garrison had been drawn away for the service of
the Italian war, the feeble remnant, whose numbers were reduced
by desertion, could not save the unhappy general from their
licentious fury. Botheric, and several of his principal officers,
were inhumanly murdered; their mangled bodies were dragged about
the streets; and the emperor, who then resided at Milan, was
surprised by the intelligence of the audacious and wanton cruelty
of the people of Thessalonica. The sentence of a dispassionate
judge would have inflicted a severe punishment on the authors of
the crime; and the merit of Botheric might contribute to
exasperate the grief and indignation of his master.

The fiery and choleric temper of Theodosius was impatient of the
dilatory forms of a judicial inquiry; and he hastily resolved,
that the blood of his lieutenant should be expiated by the blood
of the guilty people. Yet his mind still fluctuated between the
counsels of clemency and of revenge; the zeal of the bishops had
almost extorted from the reluctant emperor the promise of a
general pardon; his passion was again inflamed by the flattering
suggestions of his minister Rufinus; and, after Theodosius had
despatched the messengers of death, he attempted, when it was too
late, to prevent the execution of his orders. The punishment of a
Roman city was blindly committed to the undistinguishing sword of
the Barbarians; and the hostile preparations were concerted with
the dark and perfidious artifice of an illegal conspiracy. The
people of Thessalonica were treacherously invited, in the name of
their sovereign, to the games of the Circus; and such was their
insatiate avidity for those amusements, that every consideration
of fear, or suspicion, was disregarded by the numerous
spectators. As soon as the assembly was complete, the soldiers,
who had secretly been posted round the Circus, received the
signal, not of the races, but of a general massacre. The
promiscuous carnage continued three hours, without discrimination
of strangers or natives, of age or sex, of innocence or guilt;
the most moderate accounts state the number of the slain at seven
thousand; and it is affirmed by some writers that more than
fifteen thousand victims were sacrificed to the names of
Botheric. A foreign merchant, who had probably no concern in his
murder, offered his own life, and all his wealth, to supply the
place of one of his two sons; but, while the father hesitated
with equal tenderness, while he was doubtful to choose, and
unwilling to condemn, the soldiers determined his suspense, by
plunging their daggers at the same moment into the breasts of the
defenceless youths. The apology of the assassins, that they were
obliged to produce the prescribed number of heads, serves only to
increase, by an appearance of order and design, the horrors of
the massacre, which was executed by the commands of Theodosius.
The guilt of the emperor is aggravated by his long and frequent
residence at Thessalonica. The situation of the unfortunate city,
the aspect of the streets and buildings, the dress and faces of
the inhabitants, were familiar, and even present, to his
imagination; and Theodosius possessed a quick and lively sense of
the existence of the people whom he destroyed.\textsuperscript{91}

\pagenote[91]{The original evidence of Ambrose, (tom. ii. Epist.
li. p. 998.) Augustin, (de Civitat. Dei, v. 26,) and Paulinus,
(in Vit. Ambros. c. 24,) is delivered in vague expressions of
horror and pity. It is illustrated by the subsequent and unequal
testimonies of Sozomen, (l. vii. c. 25,) Theodoret, (l. v. c.
17,) Theophanes, (Chronograph. p. 62,) Cedrenus, (p. 317,) and
Zonaras, (tom. ii. l. xiii. p. 34.) Zosimus alone, the partial
enemy of Theodosius, most unaccountably passes over in silence
the worst of his actions.}

The respectful attachment of the emperor for the orthodox clergy,
had disposed him to love and admire the character of Ambrose; who
united all the episcopal virtues in the most eminent degree. The
friends and ministers of Theodosius imitated the example of their
sovereign; and he observed, with more surprise than displeasure,
that all his secret counsels were immediately communicated to the
archbishop; who acted from the laudable persuasion, that every
measure of civil government may have some connection with the
glory of God, and the interest of the true religion. The monks
and populace of Callinicum,\textsuperscript{9111} an obscure town on the frontier
of Persia, excited by their own fanaticism, and by that of their
bishop, had tumultuously burnt a conventicle of the Valentinians,
and a synagogue of the Jews. The seditious prelate was condemned,
by the magistrate of the province, either to rebuild the
synagogue, or to repay the damage; and this moderate sentence was
confirmed by the emperor. But it was not confirmed by the
archbishop of Milan.\textsuperscript{92} He dictated an epistle of censure and
reproach, more suitable, perhaps, if the emperor had received the
mark of circumcision, and renounced the faith of his baptism.
Ambrose considers the toleration of the Jewish, as the
persecution of the Christian, religion; boldly declares that he
himself, and every true believer, would eagerly dispute with the
bishop of Callinicum the merit of the deed, and the crown of
martyrdom; and laments, in the most pathetic terms, that the
execution of the sentence would be fatal to the fame and
salvation of Theodosius. As this private admonition did not
produce an immediate effect, the archbishop, from his pulpit,\textsuperscript{93}
publicly addressed the emperor on his throne;\textsuperscript{94} nor would he
consent to offer the oblation of the altar, till he had obtained
from Theodosius a solemn and positive declaration, which secured
the impunity of the bishop and monks of Callinicum. The
recantation of Theodosius was sincere;\textsuperscript{95} and, during the term of
his residence at Milan, his affection for Ambrose was continually
increased by the habits of pious and familiar conversation.

\pagenote[9111]{Raeca, on the Euphrates—M.}

\pagenote[92]{See the whole transaction in Ambrose, (tom. ii.
Epist. xl. xli. p. 950-956,) and his biographer Paulinus, (c.
23.) Bayle and Barbeyrac (Morales des Peres, c. xvii. p. 325,
\&c.) have justly condemned the archbishop.}

\pagenote[93]{His sermon is a strange allegory of Jeremiah’s rod,
of an almond tree, of the woman who washed and anointed the feet
of Christ. But the peroration is direct and personal.}

\pagenote[94]{Hodie, Episcope, de me proposuisti. Ambrose
modestly confessed it; but he sternly reprimanded Timasius,
general of the horse and foot, who had presumed to say that the
monks of Callinicum deserved punishment.}

\pagenote[95]{Yet, five years afterwards, when Theodosius was
absent from his spiritual guide, he tolerated the Jews, and
condemned the destruction of their synagogues. Cod. Theodos. l.
xvi. tit. viii. leg. 9, with Godefroy’s Commentary, tom. vi. p.
225.}

When Ambrose was informed of the massacre of Thessalonica, his
mind was filled with horror and anguish. He retired into the
country to indulge his grief, and to avoid the presence of
Theodosius. But as the archbishop was satisfied that a timid
silence would render him the accomplice of his guilt, he
represented, in a private letter, the enormity of the crime;
which could only be effaced by the tears of penitence. The
episcopal vigor of Ambrose was tempered by prudence; and he
contented himself with signifying\textsuperscript{96} an indirect sort of
excommunication, by the assurance, that he had been warned in a
vision not to offer the oblation in the name, or in the presence,
of Theodosius; and by the advice, that he would confine himself
to the use of prayer, without presuming to approach the altar of
Christ, or to receive the holy eucharist with those hands that
were still polluted with the blood of an innocent people. The
emperor was deeply affected by his own reproaches, and by those
of his spiritual father; and after he had bewailed the
mischievous and irreparable consequences of his rash fury, he
proceeded, in the accustomed manner, to perform his devotions in
the great church of Milan. He was stopped in the porch by the
archbishop; who, in the tone and language of an ambassador of
Heaven, declared to his sovereign, that private contrition was
not sufficient to atone for a public fault, or to appease the
justice of the offended Deity. Theodosius humbly represented,
that if he had contracted the guilt of homicide, David, the man
after God’s own heart, had been guilty, not only of murder, but
of adultery. “You have imitated David in his crime, imitate then
his repentance,” was the reply of the undaunted Ambrose. The
rigorous conditions of peace and pardon were accepted; and the
public penance of the emperor Theodosius has been recorded as one
of the most honorable events in the annals of the church.
According to the mildest rules of ecclesiastical discipline,
which were established in the fourth century, the crime of
homicide was expiated by the penitence of twenty years:\textsuperscript{97} and as
it was impossible, in the period of human life, to purge the
accumulated guilt of the massacre of Thessalonica, the murderer
should have been excluded from the holy communion till the hour
of his death. But the archbishop, consulting the maxims of
religious policy, granted some indulgence to the rank of his
illustrious penitent, who humbled in the dust the pride of the
diadem; and the public edification might be admitted as a weighty
reason to abridge the duration of his punishment. It was
sufficient, that the emperor of the Romans, stripped of the
ensigns of royalty, should appear in a mournful and suppliant
posture; and that, in the midst of the church of Milan, he should
humbly solicit, with sighs and tears, the pardon of his sins.\textsuperscript{98}
In this spiritual cure, Ambrose employed the various methods of
mildness and severity. After a delay of about eight months,
Theodosius was restored to the communion of the faithful; and the
edict which interposes a salutary interval of thirty days between
the sentence and the execution, may be accepted as the worthy
fruits of his repentance.\textsuperscript{99} Posterity has applauded the virtuous
firmness of the archbishop; and the example of Theodosius may
prove the beneficial influence of those principles, which could
force a monarch, exalted above the apprehension of human
punishment, to respect the laws, and ministers, of an invisible
Judge. “The prince,” says Montesquieu, “who is actuated by the
hopes and fears of religion, may be compared to a lion, docile
only to the voice, and tractable to the hand, of his keeper.”\textsuperscript{100}
The motions of the royal animal will therefore depend on the
inclination, and interest, of the man who has acquired such
dangerous authority over him; and the priest, who holds in his
hands the conscience of a king, may inflame, or moderate, his
sanguinary passions. The cause of humanity, and that of
persecution, have been asserted, by the same Ambrose, with equal
energy, and with equal success.

\pagenote[96]{Ambros. tom. ii. Epist. li. p. 997-1001. His
epistle is a miserable rhapsody on a noble subject. Ambrose could
act better than he could write. His compositions are destitute of
taste, or genius; without the spirit of Tertullian, the copious
elegance of Lactantius the lively wit of Jerom, or the grave
energy of Augustin.}

\pagenote[97]{According to the discipline of St. Basil, (Canon
lvi.,) the voluntary homicide was four years a mourner; five a
hearer; seven in a prostrate state; and four in a standing
posture. I have the original (Beveridge, Pandect. tom. ii. p.
47-151) and a translation (Chardon, Hist. des Sacremens, tom. iv.
p. 219-277) of the Canonical Epistles of St. Basil.}

\pagenote[98]{The penance of Theodosius is authenticated by
Ambrose, (tom. vi. de Obit. Theodos. c. 34, p. 1207,) Augustin,
(de Civitat. Dei, v. 26,) and Paulinus, (in Vit. Ambros. c. 24.)
Socrates is ignorant; Sozomen (l. vii. c. 25) concise; and the
copious narrative of Theodoret (l. v. c. 18) must be used with
precaution.}

\pagenote[99]{Codex Theodos. l. ix. tit. xl. leg. 13. The date
and circumstances of this law are perplexed with difficulties;
but I feel myself inclined to favor the honest efforts of
Tillemont (Hist. des Emp. tom. v. p. 721) and Pagi, (Critica,
tom. i. p. 578.)}

\pagenote[100]{Un prince qui aime la religion, et qui la craint,
est un lion qui cede a la main qui le flatte, ou a la voix qui
l’appaise. Esprit des Loix, l. xxiv. c. 2.}

\section{Part \thesection.}

After the defeat and death of the tyrant of Gaul, the Roman world
was in the possession of Theodosius. He derived from the choice
of Gratian his honorable title to the provinces of the East: he
had acquired the West by the right of conquest; and the three
years which he spent in Italy were usefully employed to restore
the authority of the laws, and to correct the abuses which had
prevailed with impunity under the usurpation of Maximus, and the
minority of Valentinian. The name of Valentinian was regularly
inserted in the public acts: but the tender age, and doubtful
faith, of the son of Justina, appeared to require the prudent
care of an orthodox guardian; and his specious ambition might
have excluded the unfortunate youth, without a struggle, and
almost without a murmur, from the administration, and even from
the inheritance, of the empire. If Theodosius had consulted the
rigid maxims of interest and policy, his conduct would have been
justified by his friends; but the generosity of his behavior on
this memorable occasion has extorted the applause of his most
inveterate enemies. He seated Valentinian on the throne of Milan;
and, without stipulating any present or future advantages,
restored him to the absolute dominion of all the provinces, from
which he had been driven by the arms of Maximus. To the
restitution of his ample patrimony, Theodosius added the free and
generous gift of the countries beyond the Alps, which his
successful valor had recovered from the assassin of Gratian.\textsuperscript{101}
Satisfied with the glory which he had acquired, by revenging the
death of his benefactor, and delivering the West from the yoke of
tyranny, the emperor returned from Milan to Constantinople; and,
in the peaceful possession of the East, insensibly relapsed into
his former habits of luxury and indolence. Theodosius discharged
his obligation to the brother, he indulged his conjugal
tenderness to the sister, of Valentinian; and posterity, which
admires the pure and singular glory of his elevation, must
applaud his unrivalled generosity in the use of victory.

\pagenote[101]{It is the niggard praise of Zosimus himself, (l.
iv. p. 267.) Augustin says, with some happiness of expression,
Valentinianum.... misericordissima veneratione restituit.}

The empress Justina did not long survive her return to Italy;
and, though she beheld the triumph of Theodosius, she was not
allowed to influence the government of her son.\textsuperscript{102} The
pernicious attachment to the Arian sect, which Valentinian had
imbibed from her example and instructions, was soon erased by the
lessons of a more orthodox education. His growing zeal for the
faith of Nice, and his filial reverence for the character and
authority of Ambrose, disposed the Catholics to entertain the
most favorable opinion of the virtues of the young emperor of the
West.\textsuperscript{103} They applauded his chastity and temperance, his
contempt of pleasure, his application to business, and his tender
affection for his two sisters; which could not, however, seduce
his impartial equity to pronounce an unjust sentence against the
meanest of his subjects. But this amiable youth, before he had
accomplished the twentieth year of his age, was oppressed by
domestic treason; and the empire was again involved in the
horrors of a civil war. Arbogastes,\textsuperscript{104} a gallant soldier of the
nation of the Franks, held the second rank in the service of
Gratian. On the death of his master he joined the standard of
Theodosius; contributed, by his valor and military conduct, to
the destruction of the tyrant; and was appointed, after the
victory, master-general of the armies of Gaul. His real merit,
and apparent fidelity, had gained the confidence both of the
prince and people; his boundless liberality corrupted the
allegiance of the troops; and, whilst he was universally esteemed
as the pillar of the state, the bold and crafty Barbarian was
secretly determined either to rule, or to ruin, the empire of the
West. The important commands of the army were distributed among
the Franks; the creatures of Arbogastes were promoted to all the
honors and offices of the civil government; the progress of the
conspiracy removed every faithful servant from the presence of
Valentinian; and the emperor, without power and without
intelligence, insensibly sunk into the precarious and dependent
condition of a captive.\textsuperscript{105} The indignation which he expressed,
though it might arise only from the rash and impatient temper of
youth, may be candidly ascribed to the generous spirit of a
prince, who felt that he was not unworthy to reign. He secretly
invited the archbishop of Milan to undertake the office of a
mediator; as the pledge of his sincerity, and the guardian of his
safety. He contrived to apprise the emperor of the East of his
helpless situation, and he declared, that, unless Theodosius
could speedily march to his assistance, he must attempt to escape
from the palace, or rather prison, of Vienna in Gaul, where he
had imprudently fixed his residence in the midst of the hostile
faction. But the hopes of relief were distant, and doubtful: and,
as every day furnished some new provocation, the emperor, without
strength or counsel, too hastily resolved to risk an immediate
contest with his powerful general. He received Arbogastes on the
throne; and, as the count approached with some appearance of
respect, delivered to him a paper, which dismissed him from all
his employments. “My authority,” replied Arbogastes, with
insulting coolness, “does not depend on the smile or the frown of
a monarch;” and he contemptuously threw the paper on the ground.
The indignant monarch snatched at the sword of one of the guards,
which he struggled to draw from its scabbard; and it was not
without some degree of violence that he was prevented from using
the deadly weapon against his enemy, or against himself. A few
days after this extraordinary quarrel, in which he had exposed
his resentment and his weakness, the unfortunate Valentinian was
found strangled in his apartment; and some pains were employed to
disguise the manifest guilt of Arbogastes, and to persuade the
world, that the death of the young emperor had been the voluntary
effect of his own despair.\textsuperscript{106} His body was conducted with decent
pomp to the sepulchre of Milan; and the archbishop pronounced a
funeral oration to commemorate his virtues and his misfortunes.\textsuperscript{107}
On this occasion the humanity of Ambrose tempted him to make
a singular breach in his theological system; and to comfort the
weeping sisters of Valentinian, by the firm assurance, that their
pious brother, though he had not received the sacrament of
baptism, was introduced, without difficulty, into the mansions of
eternal bliss.\textsuperscript{108}

\pagenote[102]{Sozomen, l. vii. c. 14. His chronology is very
irregular.}

\pagenote[103]{See Ambrose, (tom. ii. de Obit. Valentinian. c.
15, \&c. p. 1178. c. 36, \&c. p. 1184.) When the young emperor gave
an entertainment, he fasted himself; he refused to see a handsome
actress, \&c. Since he ordered his wild beasts to to be killed, it
is ungenerous in Philostor (l. xi. c. 1) to reproach him with the
love of that amusement.}

\pagenote[104]{Zosimus (l. iv. p. 275) praises the enemy of
Theodosius. But he is detested by Socrates (l. v. c. 25) and
Orosius, (l. vii. c. 35.)}

\pagenote[105]{Gregory of Tours (l. ii. c. 9, p. 165, in the
second volume of the Historians of France) has preserved a
curious fragment of Sulpicius Alexander, an historian far more
valuable than himself.}

\pagenote[106]{Godefroy (Dissertat. ad. Philostorg. p. 429-434)
has diligently collected all the circumstances of the death of
Valentinian II. The variations, and the ignorance, of
contemporary writers, prove that it was secret.}

\pagenote[107]{De Obitu Valentinian. tom. ii. p. 1173-1196. He is
forced to speak a discreet and obscure language: yet he is much
bolder than any layman, or perhaps any other ecclesiastic, would
have dared to be.}

\pagenote[108]{See c. 51, p. 1188, c. 75, p. 1193. Dom Chardon,
(Hist. des Sacramens, tom. i. p. 86,) who owns that St. Ambrose
most strenuously maintains the indispensable necessity of
baptism, labors to reconcile the contradiction.}

The prudence of Arbogastes had prepared the success of his
ambitious designs: and the provincials, in whose breast every
sentiment of patriotism or loyalty was extinguished, expected,
with tame resignation, the unknown master, whom the choice of a
Frank might place on the Imperial throne. But some remains of
pride and prejudice still opposed the elevation of Arbogastes
himself; and the judicious Barbarian thought it more advisable to
reign under the name of some dependent Roman. He bestowed the
purple on the rhetorician Eugenius;\textsuperscript{109} whom he had already
raised from the place of his domestic secretary to the rank of
master of the offices. In the course, both of his private and
public service, the count had always approved the attachment and
abilities of Eugenius; his learning and eloquence, supported by
the gravity of his manners, recommended him to the esteem of the
people; and the reluctance with which he seemed to ascend the
throne, may inspire a favorable prejudice of his virtue and
moderation. The ambassadors of the new emperor were immediately
despatched to the court of Theodosius, to communicate, with
affected grief, the unfortunate accident of the death of
Valentinian; and, without mentioning the name of Arbogastes, to
request, that the monarch of the East would embrace, as his
lawful colleague, the respectable citizen, who had obtained the
unanimous suffrage of the armies and provinces of the West.\textsuperscript{110}
Theodosius was justly provoked, that the perfidy of a Barbarian,
should have destroyed, in a moment, the labors, and the fruit, of
his former victory; and he was excited by the tears of his
beloved wife,\textsuperscript{111} to revenge the fate of her unhappy brother, and
once more to assert by arms the violated majesty of the throne.
But as the second conquest of the West was a task of difficulty
and danger, he dismissed, with splendid presents, and an
ambiguous answer, the ambassadors of Eugenius; and almost two
years were consumed in the preparations of the civil war. Before
he formed any decisive resolution, the pious emperor was anxious
to discover the will of Heaven; and as the progress of
Christianity had silenced the oracles of Delphi and Dodona, he
consulted an Egyptian monk, who possessed, in the opinion of the
age, the gift of miracles, and the knowledge of futurity.
Eutropius, one of the favorite eunuchs of the palace of
Constantinople, embarked for Alexandria, from whence he sailed up
the Nile, as far as the city of Lycopolis, or of Wolves, in the
remote province of Thebais.\textsuperscript{112} In the neighborhood of that city,
and on the summit of a lofty mountain, the holy John\textsuperscript{113} had
constructed, with his own hands, an humble cell, in which he had
dwelt above fifty years, without opening his door, without seeing
the face of a woman, and without tasting any food that had been
prepared by fire, or any human art. Five days of the week he
spent in prayer and meditation; but on Saturdays and Sundays he
regularly opened a small window, and gave audience to the crowd
of suppliants who successively flowed from every part of the
Christian world. The eunuch of Theodosius approached the window
with respectful steps, proposed his questions concerning the
event of the civil war, and soon returned with a favorable
oracle, which animated the courage of the emperor by the
assurance of a bloody, but infallible victory.\textsuperscript{114} The
accomplishment of the prediction was forwarded by all the means
that human prudence could supply. The industry of the two
master-generals, Stilicho and Timasius, was directed to recruit
the numbers, and to revive the discipline of the Roman legions.
The formidable troops of Barbarians marched under the ensigns of
their national chieftains. The Iberian, the Arab, and the Goth,
who gazed on each other with mutual astonishment, were enlisted
in the service of the same prince;\textsuperscript{1141} and the renowned Alaric
acquired, in the school of Theodosius, the knowledge of the art
of war, which he afterwards so fatally exerted for the
destruction of Rome.\textsuperscript{115}

\pagenote[109]{Quem sibi Germanus famulam delegerat exul, is the
contemptuous expression of Claudian, (iv. Cons. Hon. 74.)
Eugenius professed Christianity; but his secret attachment to
Paganism (Sozomen, l. vii. c. 22, Philostorg. l. xi. c. 2) is
probable in a grammarian, and would secure the friendship of
Zosimus, (l. iv. p. 276, 277.)}

\pagenote[110]{Zosimus (l. iv. p. 278) mentions this embassy; but
he is diverted by another story from relating the event.}

\pagenote[111]{Zosim. l. iv. p. 277. He afterwards says (p. 280)
that Galla died in childbed; and intimates, that the affliction
of her husband was extreme but short.}

\pagenote[112]{Lycopolis is the modern Siut, or Osiot, a town of
Said, about the size of St. Denys, which drives a profitable
trade with the kingdom of Senaar, and has a very convenient
fountain, “cujus potu signa virgini tatis eripiuntur.” See
D’Anville, Description de l’Egypte, p. 181 Abulfeda, Descript.
Egypt. p. 14, and the curious Annotations, p. 25, 92, of his
editor Michaelis.}

\pagenote[113]{The Life of John of Lycopolis is described by his
two friends, Rufinus (l. ii. c. i. p. 449) and Palladius, (Hist.
Lausiac. c. 43, p. 738,) in Rosweyde’s great Collection of the
Vitae Patrum. Tillemont (Mem. Eccles. tom. x. p. 718, 720) has
settled the chronology.}

\pagenote[114]{Sozomen, l. vii. c. 22. Claudian (in Eutrop. l. i.
312) mentions the eunuch’s journey; but he most contemptuously
derides the Egyptian dreams, and the oracles of the Nile.}

\pagenote[1141]{Gibbon has embodied the picturesque verses of
Claudian:—

.... Nec tantis dissona linguis Turba, nec armorum cultu diversion
unquam}

\pagenote[115]{Zosimus, l. iv. p. 280. Socrates, l. vii. 10.
Alaric himself (de Bell. Getico, 524) dwells with more
complacency on his early exploits against the Romans.

.... Tot Augustos Hebro qui teste fugavi.

Yet his vanity could scarcely have proved this plurality of
flying emperors.}

The emperor of the West, or, to speak more properly, his general
Arbogastes, was instructed by the misconduct and misfortune of
Maximus, how dangerous it might prove to extend the line of
defence against a skilful antagonist, who was free to press, or
to suspend, to contract, or to multiply, his various methods of
attack.\textsuperscript{116} Arbogastes fixed his station on the confines of
Italy; the troops of Theodosius were permitted to occupy, without
resistance, the provinces of Pannonia, as far as the foot of the
Julian Alps; and even the passes of the mountains were
negligently, or perhaps artfully, abandoned to the bold invader.
He descended from the hills, and beheld, with some astonishment,
the formidable camp of the Gauls and Germans, that covered with
arms and tents the open country which extends to the walls of
Aquileia, and the banks of the Frigidus,\textsuperscript{117} or Cold River.\textsuperscript{118}
This narrow theatre of the war, circumscribed by the Alps and the
Adriatic, did not allow much room for the operations of military
skill; the spirit of Arbogastes would have disdained a pardon;
his guilt extinguished the hope of a negotiation; and Theodosius
was impatient to satisfy his glory and revenge, by the
chastisement of the assassins of Valentinian. Without weighing
the natural and artificial obstacles that opposed his efforts,
the emperor of the East immediately attacked the fortifications
of his rivals, assigned the post of honorable danger to the
Goths, and cherished a secret wish, that the bloody conflict
might diminish the pride and numbers of the conquerors. Ten
thousand of those auxiliaries, and Bacurius, general of the
Iberians, died bravely on the field of battle. But the victory
was not purchased by their blood; the Gauls maintained their
advantage; and the approach of night protected the disorderly
flight, or retreat, of the troops of Theodosius. The emperor
retired to the adjacent hills; where he passed a disconsolate
night, without sleep, without provisions, and without hopes;\textsuperscript{119}
except that strong assurance, which, under the most desperate
circumstances, the independent mind may derive from the contempt
of fortune and of life. The triumph of Eugenius was celebrated by
the insolent and dissolute joy of his camp; whilst the active and
vigilant Arbogastes secretly detached a considerable body of
troops to occupy the passes of the mountains, and to encompass
the rear of the Eastern army. The dawn of day discovered to the
eyes of Theodosius the extent and the extremity of his danger;
but his apprehensions were soon dispelled, by a friendly message
from the leaders of those troops who expressed their inclination
to desert the standard of the tyrant. The honorable and lucrative
rewards, which they stipulated as the price of their perfidy,
were granted without hesitation; and as ink and paper could not
easily be procured, the emperor subscribed, on his own tablets,
the ratification of the treaty. The spirit of his soldiers was
revived by this seasonable reenforcement; and they again marched,
with confidence, to surprise the camp of a tyrant, whose
principal officers appeared to distrust, either the justice or
the success of his arms. In the heat of the battle, a violent
tempest,\textsuperscript{120} such as is often felt among the Alps, suddenly arose
from the East. The army of Theodosius was sheltered by their
position from the impetuosity of the wind, which blew a cloud of
dust in the faces of the enemy, disordered their ranks, wrested
their weapons from their hands, and diverted, or repelled, their
ineffectual javelins. This accidental advantage was skilfully
improved, the violence of the storm was magnified by the
superstitious terrors of the Gauls; and they yielded without
shame to the invisible powers of heaven, who seemed to militate
on the side of the pious emperor. His victory was decisive; and
the deaths of his two rivals were distinguished only by the
difference of their characters. The rhetorician Eugenius, who had
almost acquired the dominion of the world, was reduced to implore
the mercy of the conqueror; and the unrelenting soldiers
separated his head from his body as he lay prostrate at the feet
of Theodosius. Arbogastes, after the loss of a battle, in which
he had discharged the duties of a soldier and a general, wandered
several days among the mountains. But when he was convinced that
his cause was desperate, and his escape impracticable, the
intrepid Barbarian imitated the example of the ancient Romans,
and turned his sword against his own breast. The fate of the
empire was determined in a narrow corner of Italy; and the
legitimate successor of the house of Valentinian embraced the
archbishop of Milan, and graciously received the submission of
the provinces of the West. Those provinces were involved in the
guilt of rebellion; while the inflexible courage of Ambrose alone
had resisted the claims of successful usurpation. With a manly
freedom, which might have been fatal to any other subject, the
archbishop rejected the gifts of Eugenius,\textsuperscript{1201} declined his
correspondence, and withdrew himself from Milan, to avoid the
odious presence of a tyrant, whose downfall he predicted in
discreet and ambiguous language. The merit of Ambrose was
applauded by the conqueror, who secured the attachment of the
people by his alliance with the church; and the clemency of
Theodosius is ascribed to the humane intercession of the
archbishop of Milan.\textsuperscript{121}

\pagenote[116]{Claudian (in iv. Cons. Honor. 77, \&c.) contrasts
the military plans of the two usurpers:—

.... Novitas audere priorem Suadebat; cautumque dabant exempla
sequentem. Hic nova moliri praeceps: hic quaerere tuta Providus. 
Hic fusis; colectis viribus ille. Hic vagus excurrens; hic
claustra reductus Dissimiles, sed morte pares......}

\pagenote[117]{The Frigidus, a small, though memorable, stream in
the country of Goretz, now called the Vipao, falls into the
Sontius, or Lisonzo, above Aquileia, some miles from the
Adriatic. See D’Anville’s ancient and modern maps, and the Italia
Antiqua of Cluverius, (tom. i. c. 188.)}

\pagenote[118]{Claudian’s wit is intolerable: the snow was dyed
red; the cold ver smoked; and the channel must have been choked
with carcasses the current had not been swelled with blood.
Confluxit populus: totam pater undique secum Moverat Aurorem;
mixtis hic Colchus Iberis, Hic mitra velatus Arabs, hic crine
decoro Armenius, hic picta Saces, fucataque Medus, Hic gemmata
tiger tentoria fixerat Indus.—De Laud. Stil. l. 145.—M.}

\pagenote[119]{Theodoret affirms, that St. John, and St. Philip,
appeared to the waking, or sleeping, emperor, on horseback, \&c.
This is the first instance of apostolic chivalry, which
afterwards became so popular in Spain, and in the Crusades.}

\pagenote[120]{Te propter, gelidis Aquilo de monte procellis

Obruit adversas acies; revolutaque tela Vertit in auctores, et
turbine reppulit hastas
O nimium dilecte Deo, cui fundit ab antris Aeolus armatas hyemes;
cui militat Aether, Et conjurati veniunt ad classica venti.

These famous lines of Claudian (in iii. Cons. Honor. 93, \&c. A.D.
396) are alleged by his contemporaries, Augustin and Orosius; who
suppress the Pagan deity of Aeolus, and add some circumstances
from the information of eye-witnesses. Within four months after
the victory, it was compared by Ambrose to the miraculous
victories of Moses and Joshua.}

\pagenote[1201]{Arbogastes and his emperor had openly espoused
the Pagan party, according to Ambrose and Augustin. See Le Beau,
v. 40. Beugnot (Histoire de la Destruction du Paganisme) is more
full, and perhaps somewhat fanciful, on this remarkable reaction
in favor of Paganism, but compare p 116.—M.}

\pagenote[121]{The events of this civil war are gathered from
Ambrose, (tom. ii. Epist. lxii. p. 1022,) Paulinus, (in Vit.
Ambros. c. 26-34,) Augustin, (de Civitat. Dei, v. 26,) Orosius,
(l. vii. c. 35,) Sozomen, (l. vii. c. 24,) Theodoret, (l. v. c.
24,) Zosimus, (l. iv. p. 281, 282,) Claudian, (in iii. Cons. Hon.
63-105, in iv. Cons. Hon. 70-117,) and the Chronicles published
by Scaliger.}

After the defeat of Eugenius, the merit, as well as the
authority, of Theodosius was cheerfully acknowledged by all the
inhabitants of the Roman world. The experience of his past
conduct encouraged the most pleasing expectations of his future
reign; and the age of the emperor, which did not exceed fifty
years, seemed to extend the prospect of the public felicity. His
death, only four months after his victory, was considered by the
people as an unforeseen and fatal event, which destroyed, in a
moment, the hopes of the rising generation. But the indulgence of
ease and luxury had secretly nourished the principles of disease.\textsuperscript{122}
The strength of Theodosius was unable to support the sudden
and violent transition from the palace to the camp; and the
increasing symptoms of a dropsy announced the speedy dissolution
of the emperor. The opinion, and perhaps the interest, of the
public had confirmed the division of the Eastern and Western
empires; and the two royal youths, Arcadius and Honorius, who had
already obtained, from the tenderness of their father, the title
of Augustus, were destined to fill the thrones of Constantinople
and of Rome. Those princes were not permitted to share the danger
and glory of the civil war;\textsuperscript{123} but as soon as Theodosius had
triumphed over his unworthy rivals, he called his younger son,
Honorius, to enjoy the fruits of the victory, and to receive the
sceptre of the West from the hands of his dying father. The
arrival of Honorius at Milan was welcomed by a splendid
exhibition of the games of the Circus; and the emperor, though he
was oppressed by the weight of his disorder, contributed by his
presence to the public joy. But the remains of his strength were
exhausted by the painful effort which he made to assist at the
spectacles of the morning. Honorius supplied, during the rest of
the day, the place of his father; and the great Theodosius
expired in the ensuing night. Notwithstanding the recent
animosities of a civil war, his death was universally lamented.
The Barbarians, whom he had vanquished and the churchmen, by whom
he had been subdued, celebrated, with loud and sincere applause,
the qualities of the deceased emperor, which appeared the most
valuable in their eyes. The Romans were terrified by the
impending dangers of a feeble and divided administration, and
every disgraceful moment of the unfortunate reigns of Arcadius
and Honorius revived the memory of their irreparable loss.

\pagenote[122]{This disease, ascribed by Socrates (l. v. c. 26)
to the fatigues of war, is represented by Philostorgius (l. xi.
c. 2) as the effect of sloth and intemperance; for which Photius
calls him an impudent liar, (Godefroy, Dissert. p. 438.)}

\pagenote[123]{Zosimus supposes, that the boy Honorius
accompanied his father, (l. iv. p. 280.) Yet the quanto
flagrabrant pectora voto is all that flattery would allow to a
contemporary poet; who clearly describes the emperor’s refusal,
and the journey of Honorius, after the victory (Claudian in iii.
Cons. 78-125.)}

In the faithful picture of the virtues of Theodosius, his
imperfections have not been dissembled; the act of cruelty, and
the habits of indolence, which tarnished the glory of one of the
greatest of the Roman princes. An historian, perpetually adverse
to the fame of Theodosius, has exaggerated his vices, and their
pernicious effects; he boldly asserts, that every rank of
subjects imitated the effeminate manners of their sovereign; and
that every species of corruption polluted the course of public
and private life; and that the feeble restraints of order and
decency were insufficient to resist the progress of that
degenerate spirit, which sacrifices, without a blush, the
consideration of duty and interest to the base indulgence of
sloth and appetite.\textsuperscript{124} The complaints of contemporary writers,
who deplore the increase of luxury, and depravation of manners,
are commonly expressive of their peculiar temper and situation.
There are few observers, who possess a clear and comprehensive
view of the revolutions of society; and who are capable of
discovering the nice and secret springs of action, which impel,
in the same uniform direction, the blind and capricious passions
of a multitude of individuals. If it can be affirmed, with any
degree of truth, that the luxury of the Romans was more shameless
and dissolute in the reign of Theodosius than in the age of
Constantine, perhaps, or of Augustus, the alteration cannot be
ascribed to any beneficial improvements, which had gradually
increased the stock of national riches. A long period of calamity
or decay must have checked the industry, and diminished the
wealth, of the people; and their profuse luxury must have been
the result of that indolent despair, which enjoys the present
hour, and declines the thoughts of futurity. The uncertain
condition of their property discouraged the subjects of
Theodosius from engaging in those useful and laborious
undertakings which require an immediate expense, and promise a
slow and distant advantage. The frequent examples of ruin and
desolation tempted them not to spare the remains of a patrimony,
which might, every hour, become the prey of the rapacious Goth.
And the mad prodigality which prevails in the confusion of a
shipwreck, or a siege, may serve to explain the progress of
luxury amidst the misfortunes and terrors of a sinking nation.

\pagenote[124]{Zosimus, l. iv. p. 244.}

The effeminate luxury, which infected the manners of courts and
cities, had instilled a secret and destructive poison into the
camps of the legions; and their degeneracy has been marked by the
pen of a military writer, who had accurately studied the genuine
and ancient principles of Roman discipline. It is the just and
important observation of Vegetius, that the infantry was
invariably covered with defensive armor, from the foundation of
the city, to the reign of the emperor Gratian. The relaxation of
discipline, and the disuse of exercise, rendered the soldiers
less able, and less willing, to support the fatigues of the
service; they complained of the weight of the armor, which they
seldom wore; and they successively obtained the permission of
laying aside both their cuirasses and their helmets. The heavy
weapons of their ancestors, the short sword, and the formidable
pilum, which had subdued the world, insensibly dropped from their
feeble hands. As the use of the shield is incompatible with that
of the bow, they reluctantly marched into the field; condemned to
suffer either the pain of wounds, or the ignominy of flight, and
always disposed to prefer the more shameful alternative. The
cavalry of the Goths, the Huns, and the Alani, had felt the
benefits, and adopted the use, of defensive armor; and, as they
excelled in the management of missile weapons, they easily
overwhelmed the naked and trembling legions, whose heads and
breasts were exposed, without defence, to the arrows of the
Barbarians. The loss of armies, the destruction of cities, and
the dishonor of the Roman name, ineffectually solicited the
successors of Gratian to restore the helmets and the cuirasses of
the infantry. The enervated soldiers abandoned their own and the
public defence; and their pusillanimous indolence may be
considered as the immediate cause of the downfall of the empire.\textsuperscript{125}

\pagenote[125]{Vegetius, de Re Militari, l. i. c. 10. The series
of calamities which he marks, compel us to believe, that the
Hero, to whom he dedicates his book, is the last and most
inglorious of the Valentinians.}

